\tikzset{declare function = {
            % Legendre polynomials from n = 0 to n = 10
            leg_P_0(\x) = 1;
            leg_P_1(\x) = \x;
            leg_P_2(\x) = (1/2)* (3*\x^2 - 1);
            leg_P_3(\x) = (1/2)* (5*\x^3 - 3*\x);
            leg_P_4(\x) = (1/8)* (35*\x^4 - 30*\x^2 + 3);
            leg_P_5(\x) = (1/8)* (63*\x^5 - 70*\x^3 + 15*\x);
            leg_P_6(\x) = (1/16)* (231*\x^6 - 315*\x^4 + 105*\x^2 - 5);
            leg_P_7(\x) = (1/16)* (429*\x^7 - 693*\x^5 + 315*\x^3 - 35*\x);
            leg_P_8(\x) = (1/128)* (6435*\x^8 - 12012*\x^6 + 6930*\x^4 - 1260*\x^2
            + 35);
            leg_P_9(\x) = (1/128)* (12155*\x^9 - 25740*\x^7 + 18018*\x^5 - 4620*\x^3
            + 315*\x);
            leg_P_10(\x) = (1/256)* (46189*\x^10 - 109395*\x^8 + 90090*\x^6 - 30030*\x^4
            + 3465*\x^2 - 63);
            %----------------------------------------------------------------------------%
            % Legendre functions from n = 0 to 5
            leg_Q_0(\x) = (1/2) * ln((1+\x)/(1-\x));
            leg_Q_1(\x) = \x * leg_Q_0(\x) - 1;
            leg_Q_2(\x) = (3*\x^2 - 1)/2 * leg_Q_0(\x) - (3*\x/2);
            leg_Q_3(\x) = (5*\x^3 - 3*\x)/2 * leg_Q_0(\x) - (5*\x^2/2) + 2/3;
            %----------------------------------------------------------------------------%
            % Approximation to bessel function for large x
            Bes_asymp(\n, \x) = (cos(x - \c*0.5*pi - 0.25*pi))*sqrt(2/(x*pi));
            %----------------------------------------------------------------------------%
            % Unit step function using built-in signum function
            Hea(\x) = 0.5 * (1 + sign(\x));
        }
}