\section{Tridiagonalization and QR-Factorization}

\begin{enumerate}
    \item Tridiagonalization algorithm written in \texttt{numpy}.
          \begin{align}
              \vec{v}_1           & = \begin{bNiceMatrix}[margin]
                                          0 \\ 0.7384 \\ 0.6743
                                      \end{bNiceMatrix} &
              \vec{A}_1 = \vec{B} & =
              \begin{bNiceArray}{rrr}[margin]
                  \CodeBefore
                  \cellcolor[opacity = 0.1]{y_h}{1-1,2-2,3-3}
                  \Body
                  0.98    & -0.4418 & 0      \\
                  -0.4418 & 0.8701  & 0.3718 \\
                  0       & 0.3718  & 0.4898
              \end{bNiceArray}
          \end{align}

    \item Tridiagonalization algorithm written in \texttt{numpy}.
          \begin{align}
              \vec{v}_1           & = \begin{bNiceMatrix}[margin]
                                          0 \\ 0.9239 \\ 0.3827
                                      \end{bNiceMatrix} &
              \vec{A}_1 = \vec{B} & =
              \begin{bNiceArray}{rrr}[margin]
                  \CodeBefore
                  \cellcolor[opacity = 0.1]{y_h}{1-1,2-2,3-3}
                  \Body
                  0       & -1.4142 & 0  \\
                  -1.4142 & 1       & 0  \\
                  0       & 0       & -1
              \end{bNiceArray}
          \end{align}

    \item Tridiagonalization algorithm written in \texttt{numpy}.
          \begin{align}
              \vec{v}_1           & = \begin{bNiceMatrix}[margin]
                                          0 \\ 0.8817 \\ 0.4719
                                      \end{bNiceMatrix} &
              \vec{A}_1 = \vec{B} & =
              \begin{bNiceArray}{rrr}[margin]
                  \CodeBefore
                  \cellcolor[opacity = 0.1]{y_h}{1-1,2-2,3-3}
                  \Body
                  7       & -3.6056 & 0      \\
                  -3.6056 & 13.4615 & 3.6923 \\
                  0       & 3.6923  & 3.5385
              \end{bNiceArray}
          \end{align}

    \item Tridiagonalization algorithm written in \texttt{numpy}.
          \begin{align}
              \vec{v}_1 & = \begin{bNiceMatrix}[margin]
                                0 \\ 0.9856 \\ 0.1196 \\ 0.1196
                            \end{bNiceMatrix} &
              \vec{A}_1 & =
              \begin{bNiceArray}{rrrr}[margin]
                  \CodeBefore
                  \cellcolor[opacity = 0.1]{y_h}{1-1,2-2,3-3,4-4}
                  \Body
                  5       & -4.2426 & 0   & 0   \\
                  -4.2426 & 6       & -1  & -1  \\
                  0       & -1      & 3.5 & 1.5 \\
                  0       & -1      & 1.5 & 3.5
              \end{bNiceArray} \\
              \vec{v}_2 & = \begin{bNiceMatrix}[margin]
                                0 \\ 0 \\ 0.9239 \\ 0.3827
                            \end{bNiceMatrix}   &
              \vec{B}   & =
              \begin{bNiceArray}{rrrr}[margin]
                  \CodeBefore
                  \cellcolor[opacity = 0.1]{y_h}{1-1,2-2,3-3,4-4}
                  \Body
                  5       & -4.2426 & 0      & 0 \\
                  -4.2426 & 6       & 1.4142 & 0 \\
                  0       & 1.4142  & 5      & 0 \\
                  0       & 0       & 0      & 2
              \end{bNiceArray}
          \end{align}

    \item Tridiagonalization algorithm written in \texttt{numpy}.
          \begin{align}
              \vec{v}_1 & = \begin{bNiceMatrix}[margin]
                                0 \\ 0.9406 \\ 0.0787 \\ 0.3303
                            \end{bNiceMatrix} &
              \vec{A}_1 & =
              \begin{bNiceArray}{rrrr}[margin]
                  \CodeBefore
                  \cellcolor[opacity = 0.1]{y_h}{1-1,2-2,3-3,4-4}
                  \Body
                  3       & -67.587  & -67.587 & 0        \\
                  -67.587 & 143.5324 & -45.223 & -3.4156  \\
                  -67.587 & -45.223  & 23.487  & 1.2062   \\
                  0       & -3.4156  & 1.2062  & -34.0194
              \end{bNiceArray} \\
              \vec{v}_2 & = \begin{bNiceMatrix}[margin]
                                0 \\ 0 \\ 0.9993 \\ 0.0377
                            \end{bNiceMatrix}   &
              \vec{B}   & =
              \begin{bNiceArray}{rrrr}[margin]
                  \CodeBefore
                  \cellcolor[opacity = 0.1]{y_h}{1-1,2-2,3-3,4-4}
                  \Body
                  3       & -67.587  & 0       & 0        \\
                  -67.587 & 143.5324 & 45.3518 & 0        \\
                  0       & 45.3518  & 23.342  & 3.1262   \\
                  0       & 0        & 3.1262  & -33.8744
              \end{bNiceArray}
          \end{align}

    \item QR factorization coded in \texttt{numpy}
          \begin{table}[H]
              \centering
              \SetTblrInner{rowsep=0.4em}
              \begin{tblr}{
                  colspec = {Q[r]|[dotted]Q[l,$$]|[dotted]Q[l,$$]|
                  [dotted]Q[l,$$]|[dotted]Q[l,$$]},
                  colsep = 1em}
                  $n$ & \lambda_1 & \lambda_2 & \lambda_3 & \text{Error} \\ \hline
                  1   & 1.2925    & 0.8402    & 0.2072    & 0.2942       \\
                  5   & 1.4393    & 0.7207    & 0.1800039 & 0.022478     \\
                  10  & 1.439999  & 0.7200007 & 0.18      & 0.000703     \\ \hline
              \end{tblr}
          \end{table}

    \item QR factorization coded in \texttt{numpy}
          \begin{table}[H]
              \centering
              \SetTblrInner{rowsep=0.4em}
              \begin{tblr}{
                  colspec = {Q[r]|[dotted]Q[l,$$]|[dotted]Q[l,$$]|
                  [dotted]Q[l,$$]|[dotted]Q[l,$$]},
                  colsep = 1em}
                  $n$ & \lambda_1 & \lambda_2 & \lambda_3 & \text{Error} \\ \hline
                  1   & 11.2903   & 10.6144   & 2.0952    & 5.0173       \\
                  5   & 15.9966   & 6.0034    & 2.000012  & 0.18492      \\
                  10  & 16        & 6.0000    & 2         & 0.001372     \\ \hline
              \end{tblr}
          \end{table}

    \item QR factorization coded in \texttt{numpy}
          \begin{table}[H]
              \centering
              \SetTblrInner{rowsep=0.4em}
              \begin{tblr}{
                  colspec = {Q[r]|[dotted]Q[l,$$]|[dotted]Q[l,$$]|
                  [dotted]Q[l,$$]|[dotted]Q[l,$$]},
                  colsep = 1em}
                  $n$ & \lambda_1    & \lambda_2
                      & \lambda_3    & \text{Error}      \\ \hline
                  1   & 14.200439177 & -6.30462
                      & 2.1042282    & 0.06679           \\
                  5   & 14.20048773  & -6.30524639
                      & 2.10475866   & 0.001725          \\
                  10  & 141.20048787 & -6.30524661
                      & 2.10475874   & \num{2.997766d-5} \\ \hline
              \end{tblr}
          \end{table}

    \item QR factorization coded in \texttt{numpy}
          \begin{table}[H]
              \centering
              \SetTblrInner{rowsep=0.4em}
              \begin{tblr}{
                  colspec = {Q[r]|[dotted]Q[l,$$]|[dotted]Q[l,$$]|
                  [dotted]Q[l,$$]|[dotted]Q[l,$$]},
                  colsep = 1em}
                  $n$ & \lambda_1 & \lambda_2 & \lambda_3    & \text{Error} \\ \hline
                  1   & 141.066   & 68.9666   & -30.0326     & 4.926        \\
                  5   & 141.39996 & 68.64025  & -30.0402101  & 0.2747       \\
                  10  & 141.401   & 68.63922  & -30.04022035 & 0.0074       \\ \hline
              \end{tblr}
          \end{table}

    \item Coded in \texttt{numpy}. The convergence gets faster when the spectral
          shifting theorem is used to bring the smallest eigenvalue to zero, or when
          the smallest eigenvalue is already zero.
\end{enumerate}