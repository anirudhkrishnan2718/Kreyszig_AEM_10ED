\section{Power Method for Eigenvalues}

\begin{enumerate}
    \item Finding the rayleigh criterion using the power method,
          \begin{table}[H]
              \centering
              \SetTblrInner{rowsep=0.4em}
              \begin{tblr}{
                  colspec = {Q[r]|[dotted]Q[r,$$]|[dotted]Q[r,$$]},
                  colsep = 1em}
                  n & q       & \delta  \\ \hline
                  1 & 10      & 3       \\
                  2 & 10.99   & 0.30275 \\
                  3 & 10.9999 & 0.0275  \\ \hline
              \end{tblr}
          \end{table}

    \item Finding the rayleigh criterion using the power method,
          \begin{table}[H]
              \centering
              \SetTblrInner{rowsep=0.4em}
              \begin{tblr}{
                  colspec = {Q[r]|[dotted]Q[r,$$]|[dotted]Q[r,$$]},
                  colsep = 1em}
                  n & q    & \delta \\ \hline
                  1 & 0    & 4      \\
                  2 & 6    & 4      \\
                  3 & 7.85 & 1.231  \\ \hline
              \end{tblr}
          \end{table}

    \item Finding the rayleigh criterion using the power method,
          \begin{table}[H]
              \centering
              \SetTblrInner{rowsep=0.4em}
              \begin{tblr}{
                  colspec = {Q[r]|[dotted]Q[r,$$]|[dotted]Q[r,$$]},
                  colsep = 1em}
                  n & q     & \delta \\ \hline
                  1 & 4     & 1.633  \\
                  2 & 4.786 & 0.6186 \\
                  3 & 4.917 & 0.3985 \\ \hline
              \end{tblr}
          \end{table}

    \item Finding the rayleigh criterion using the power method,
          \begin{table}[H]
              \centering
              \SetTblrInner{rowsep=0.4em}
              \begin{tblr}{
                  colspec = {Q[r]|[dotted]Q[r,$$]|[dotted]Q[r,$$]},
                  colsep = 1em}
                  n & q     & \delta \\ \hline
                  1 & 1.333 & 2.175  \\
                  2 & 6.574 & 1.742  \\
                  3 & 7.158 & 0.474  \\ \hline
              \end{tblr}
          \end{table}

    \item Finding the rayleigh criterion using the power method with scaling,
          \begin{table}[H]
              \centering
              \SetTblrInner{rowsep=0.4em}
              \begin{tblr}{
                  colspec = {Q[r]|[dotted]Q[r,$$]|[dotted]Q[r,$$]},
                  colsep = 1em}
                  n & q     & \delta \\ \hline
                  1 & 4     & 1.633  \\
                  2 & 4.786 & 0.6186 \\
                  3 & 4.917 & 0.3985 \\ \hline
              \end{tblr}
          \end{table}

    \item Finding the rayleigh criterion using the power method with scaling,
          \begin{table}[H]
              \centering
              \SetTblrInner{rowsep=0.4em}
              \begin{tblr}{
                  colspec = {Q[r]|[dotted]Q[r,$$]|[dotted]Q[r,$$]},
                  colsep = 1em}
                  n & q        & \delta  \\ \hline
                  1 & 12.33333 & 2.49443 \\
                  2 & 12.96211 & 0.61442 \\
                  3 & 12.99797 & 0.14230 \\ \hline
              \end{tblr}
          \end{table}

    \item Finding the rayleigh criterion using the power method with scaling,
          \begin{table}[H]
              \centering
              \SetTblrInner{rowsep=0.4em}
              \begin{tblr}{
                  colspec = {Q[r]|[dotted]Q[r,$$]|[dotted]Q[r,$$]},
                  colsep = 1em}
                  n & q       & \delta  \\ \hline
                  1 & 5.5     & 0.5     \\
                  2 & 5.57737 & 0.31147 \\
                  3 & 5.60179 & 0.18990 \\ \hline
              \end{tblr}
          \end{table}

    \item Finding the rayleigh criterion using the power method with scaling,
          \begin{table}[H]
              \centering
              \SetTblrInner{rowsep=0.4em}
              \begin{tblr}{
                  colspec = {Q[r]|[dotted]Q[r,$$]|[dotted]Q[r,$$]},
                  colsep = 1em}
                  n & q       & \delta \\ \hline
                  1 & 10.5    & 2.9580 \\
                  2 & 11.1302 & 1.3688 \\
                  3 & 11.1831 & 0.9637 \\ \hline
              \end{tblr}
          \end{table}

    \item Given the initial guess is already an eigenvalue,
          \begin{align}
              \vec{y}  & = \vec{Ax} = \lambda \vec{x}                   &
              m_0      & = \vec{x}^T \vec{x}                              \\
              m_1      & = \lambda\ \vec{x}^T \vec{x} = \lambda m_0     &
              m_2      & = \lambda^2\ \vec{x}^T \vec{x} = \lambda^2 m_0   \\
              \delta^2 & = \lambda^2 - (\lambda)^2 = 0                  &
              \delta   & = 0
          \end{align}

    \item Using the set of real unit eigenvectors
          $ \{\vec{z_1},\vec{z_2},\dots,\vec{z_n}\} $
          \begin{align}
              \vec{x}                  & = a_1\vec{z}_1 + \dots + a_n\vec{z}_n &
              \vec{x}^T \vec{x}        & = a_1^2 + \dots + a_n^2                 \\
              \vec{y}                  & = \lambda_1 a_1 \vec{z}_1 + \dots
              + \lambda_n a_n\vec{z}_n &
              \vec{x}^T \vec{y}        & = (\lambda_1 a_1)^2 + \dots
              + (\lambda_n a_n)^2
          \end{align}
          Subsitutting these values into the rayleigh quotient, where $ \lambda_1 $ is
          the dominant eigenvalue,
          \begin{align}
              q       & = \frac{m_1}{m_0} = \frac{\vec{x}^T \vec{y}}
              {\vec{x}^T \vec{x}}
                      &
              q^{(1)} & = \frac{\sum_{j} \lambda_j^2\ a_j^2}{\sum_{j} a_j^2} \\
              q^{(m)} & = \frac{\sum_{j} \lambda_j^{2m+1}\ a_j^2}
              {\sum_{j} \lambda_j^{2m} a_j^{2}}
          \end{align}
          After a large number of iterations,
          \begin{align}
              \text{numerator}   & = \lambda_1^{2m+1}\ \Bigg[ a_1^2
                  + \Bigg( \frac{\lambda_2}{\lambda_1} \Bigg)^{2m+1}\ a_2^2
                  + \dots
              + \Bigg( \frac{\lambda_n}{\lambda_1} \Bigg)^{2m+1}\ a_n^2 \Bigg] \\
              \text{denominator} & = \lambda_1^{2m}\ \Bigg[ a_1^2
                  + \Bigg( \frac{\lambda_2}{\lambda_1} \Bigg)^{2m}\ a_2^2
                  + \dots
              + \Bigg( \frac{\lambda_n}{\lambda_1} \Bigg)^{2m}\ a_n^2 \Bigg]   \\
          \end{align}
          Since the fractions are all less than 1, they approach zero with increasing
          $ m $,
          \begin{align}
              \lim_{m \to \infty} q^{(m)} = \frac{\lambda_1^{2m+1}\ a_1^2}
              {\lambda_1^{2m}\ a_1^2} = \lambda_1
          \end{align}

    \item Finding the rayleigh criterion using the power method,
          \begin{table}[H]
              \centering
              \SetTblrInner{rowsep=0.4em}
              \begin{tblr}{
                  colspec = {Q[r]|[dotted]Q[r,$$]|[dotted]Q[r,$$]},
                  colsep = 1em}
                  n  & q       & \delta \\ \hline
                  1  & 1       & 1.633  \\
                  5  & -2.0515 & 1.9603 \\
                  10 & -2.9798 & 0.3173 \\ \hline
              \end{tblr}
          \end{table}
          Using a CAS to look at the spectrum of the original matrix $ \vec{A} $
          \begin{align}
              \{\lambda_i\} & = \{0, 3, 5\}
          \end{align}

    \item Power method
          \begin{enumerate}
              \item Algorithm coded in \texttt{numpy}
                    \begin{table}[H]
                        \centering
                        \SetTblrInner{rowsep=0.4em}
                        \begin{tblr}{
                            colspec = {Q[r]|[dotted]Q[l]},
                            colsep = 1em}
                            $n$ & $ q $       \\ \hline
                            1   & 16          \\
                            5   & 32.3168     \\
                            10  & 32.0022     \\
                            15  & 32.00000608 \\
                            20  & 32.00000012 \\ \hline
                        \end{tblr}
                    \end{table}

              \item Using the shifting theorem for $ \vec{B} = \vec{A} + k\vec{I} $,
                    the best results for 10 iterations starting from the same initial
                    guess are for $ k^* = -10 $

              \item Refer problem $ 8 $. Coded in \texttt{numpy}

              \item The eigenvalues from a CAS are $ \{-1, 1\} $. Within machine number,
                    $ q \approxeq 0 $ for all steps and $ \delta = 1 $ for all steps.
                    \par Since the vector oscillates between the same two values for
                    this initial guess, the iterative process will not converge.

              \item A simple matrix resembling the identity matrix can yield
                    errors larger than this theoretical bound
                    \begin{align}
                        \vec{A}   & = \bmattt{1}{1}{0}{1} &
                        \vec{x}_0 & = \bmatcol{3}{1}        \\
                        \vec{x}_1 & = \bmatcol{4}{1}      &
                        \vec{x}_m & = \bmatcol{(m+3)}{1}    \\
                        \delta_1  & = 0.1                 &
                        q_1       & = 0.3
                    \end{align}
                    This matrix is nonsymmetric but has no dominant eigenvalue.
                    The actual error is larger than $ \delta $

              \item Using the \texttt{numpy} code, the convergence is faster when
                    the second eigenvalue is smaller than first.
          \end{enumerate}
\end{enumerate}