\section{Inclusion of Matrix Eigenvalues}

\begin{enumerate}
    \item Plotting the Gerschgorin disks, and using a CAS to plot the eigenvalues,
          \begin{figure}[H]
              \centering
              \begin{tikzpicture}
                  \begin{axis}[set layers, width = 8cm,
                          xmin = -5, xmax = 15, ymin = -10, ymax = 10,
                          grid = both,
                          xlabel = \normalsize $ \Re{z} $,
                          ylabel = \normalsize $ \Im{z} $,
                          axis equal,
                          Ani]
                      \draw [draw = y_h, thick] (5,0) circle (6);
                      \draw [draw = y_p, thick] (0,0) circle (4);
                      \draw [draw = y_t, thick] (7,0) circle (6);
                      \node[GraphNode, inner sep = 1pt] at (axis cs:-1, 0) {};
                      \node[GraphNode, inner sep = 1pt] at (axis cs:4, 0) {};
                      \node[GraphNode, inner sep = 1pt] at (axis cs:9, 0) {};
                  \end{axis}
              \end{tikzpicture}
          \end{figure}

    \item Plotting the Gerschgorin disks, and using a CAS to plot the eigenvalues,
          \begin{figure}[H]
              \centering
              \begin{tikzpicture}
                  \begin{axis}[set layers, width = 5.5cm, height = 5.5cm,
                          xmin = 4.98, xmax = 5.02, ymax = 0.02, ymin = -0.02,
                          grid = both,
                          xlabel = \normalsize $ \Re{z} $,
                          ylabel = \normalsize $ \Im{z} $,
                          axis equal,Ani]
                      \draw [draw = y_h, thick] (5,0) circle (0.02);
                      \node[GraphNode, inner sep = 1pt] at (axis cs:4.99994, 0) {};
                  \end{axis}
              \end{tikzpicture}
              \begin{tikzpicture}
                  \begin{axis}[set layers, width = 5.5cm, height = 5.5cm,
                          xmin = 7.98, xmax = 8.02, ymax = 0.02, ymin = -0.02,
                          grid = both,
                          xlabel = \normalsize $ \Re{z} $,
                          axis equal,Ani]
                      \draw [draw = y_p, thick] (8,0) circle (0.02);
                      \node[GraphNode, inner sep = 1pt] at (axis cs:8, 0) {};
                  \end{axis}
              \end{tikzpicture}
              \begin{tikzpicture}
                  \begin{axis}[set layers, width = 5.5cm, height = 5.5cm,
                          xmin = 8.98, xmax = 9.02, ymax = 0.02, ymin = -0.02,
                          grid = both,
                          xlabel = \normalsize $ \Re{z} $,
                          axis equal,Ani]
                      \draw [draw = y_t, thick] (9,0) circle (0.02);
                      \node[GraphNode, inner sep = 1pt] at (axis cs:9, 0) {};
                  \end{axis}
              \end{tikzpicture}
          \end{figure}

    \item Plotting the Gerschgorin disks, and using a CAS to plot the eigenvalues,
    \item Plotting the Gerschgorin disks, and using a CAS to plot the eigenvalues,
          \begin{figure}[H]
              \centering
              \begin{tikzpicture}
                  \begin{axis}[set layers, width = 8cm,
                          xmin = -1, xmax = 1, ymin = -1, ymax = 1,
                          grid = both,
                          xlabel = \normalsize $ \Re{z} $,
                          ylabel = \normalsize $ \Im{z} $,
                          axis equal,
                          Ani]
                      \draw [draw = y_h, thick] (0,0) circle (0.5);
                      \draw [draw = y_p, thick] (0,0) circle (0.7);
                      \draw [draw = y_t, thick] (0,0) circle (0.4);
                      \node[GraphNode, inner sep = 1pt] at (axis cs:0, 0) {};
                      \node[GraphNode, inner sep = 1pt] at (axis cs:0, 0.51) {};
                      \node[GraphNode, inner sep = 1pt] at (axis cs:0, -0.51) {};
                  \end{axis}
              \end{tikzpicture}
              \begin{tikzpicture}
                  \begin{axis}[set layers, width = 8cm,
                          xmin = -1, xmax = 17, ymin = -9, ymax = 9,
                          grid = both,
                          xlabel = \normalsize $ \Re{z} $,
                          ylabel = \normalsize $ \Im{z} $,
                          axis equal,
                          Ani]
                      \draw [draw = y_h, thick] (1,0) circle (1);
                      \draw [draw = y_p, thick] (4,0) circle (3);
                      \draw [draw = y_t, thick] (12,0) circle (4);
                      \node[GraphNode, inner sep = 1pt] at (axis cs:13.07, 0) {};
                      \node[GraphNode, inner sep = 1pt] at (axis cs:0.878, 0) {};
                      \node[GraphNode, inner sep = 1pt] at (axis cs:3.05, 0) {};
                  \end{axis}
              \end{tikzpicture}
          \end{figure}

    \item Plotting the Gerschgorin disks, and using a CAS to plot the eigenvalues,
          \begin{figure}[H]
              \centering
              \begin{tikzpicture}
                  \begin{axis}[set layers, width = 8cm,
                          xmin = -2, xmax = 10, ymin = -6, ymax = 6,
                          grid = both,
                          xlabel = \normalsize $ \Re{z} $,
                          ylabel = \normalsize $ \Im{z} $,
                          axis equal,
                          Ani]
                      \draw [draw = y_h, thick] (2,0) circle (2.414);
                      \draw [draw = y_p, thick] (3,0) circle (1);
                      \draw [draw = y_t, thick] (8,0) circle (1.414);
                      \node[GraphNode, inner sep = 1pt] at (axis cs:8.825, 0) {};
                      \node[GraphNode, inner sep = 1pt] at (axis cs:1.16, 0) {};
                      \node[GraphNode, inner sep = 1pt] at (axis cs:3.51, 0) {};
                  \end{axis}
              \end{tikzpicture}
          \end{figure}

    \item Plotting the Gerschgorin disks, and using a CAS to plot the eigenvalues,
          \begin{figure}[H]
              \centering
              \begin{tikzpicture}
                  \begin{axis}[set layers, width = 5.5cm, height = 5.5cm,
                          xmin = 9.7, xmax = 10.3, ymin = -0.3, ymax = 0.3,
                          grid = both,
                          xlabel = \normalsize $ \Re{z} $,
                          ylabel = \normalsize $ \Im{z} $,
                          axis equal,Ani]
                      \draw [draw = y_h, thick] (10,0) circle (0.3);
                      \node[GraphNode, inner sep = 1pt] at (axis cs:10, 0) {};
                  \end{axis}
              \end{tikzpicture}
              \begin{tikzpicture}
                  \begin{axis}[set layers, width = 5.5cm, height = 5.5cm,
                          xmin = 5.7, xmax = 6.3, ymin = -0.3, ymax = 0.3,
                          grid = both,
                          xlabel = \normalsize $ \Re{z} $,
                          axis equal,Ani]
                      \draw [draw = y_p, thick] (6,0) circle (0.1);
                      \node[GraphNode, inner sep = 1pt] at (axis cs:6, 0) {};
                  \end{axis}
              \end{tikzpicture}
              \begin{tikzpicture}
                  \begin{axis}[set layers, width = 5.5cm, height = 5.5cm,
                          xmin = 2.7, xmax = 3.3, ymin = -0.3, ymax = 0.3,
                          grid = both,
                          xlabel = \normalsize $ \Re{z} $,
                          axis equal,Ani]
                      \draw [draw = y_t, thick] (3,0) circle (0.2);
                      \node[GraphNode, inner sep = 1pt] at (axis cs:3, 0) {};
                  \end{axis}
              \end{tikzpicture}
          \end{figure}

    \item Using a similarity transformation,
          \begin{align}
              \vec{T}               & = \begin{bNiceMatrix}[margin]
                                            100 & 0 & 0 \\
                                            0   & 1 & 0 \\
                                            0   & 0 & 1
                                        \end{bNiceMatrix} &
              \vec{T}^{-1} \vec{AT} & = \begin{bNiceMatrix}[margin]
                                            5 & \num{d-4} & \num{d-4} \\
                                            1 & 8         & \num{d-2} \\
                                            1 & \num{d-2} & 9
                                        \end{bNiceMatrix} \\
              r_5                   & = \num{2d-4}
          \end{align}

    \item Using a similarity transformation,
          \begin{align}
              \vec{T}               & = \begin{bNiceMatrix}[margin]
                                            1 & 0 & 0 \\
                                            0 & 1 & 0 \\
                                            0 & 0 & k
                                        \end{bNiceMatrix} &
              \vec{A}               & = \begin{bNiceMatrix}[margin]
                                            10   & 0.1 & 0.2 \\
                                            0.1  & 6   & 0   \\
                                            -0.2 & 0   & 3
                                        \end{bNiceMatrix} \\
              \vec{T}^{-1} \vec{AT} & = \begin{bNiceMatrix}[margin]
                                            10     & 0.1 & -0.2k \\
                                            0.1    & 6   & 0     \\
                                            -0.2/k & 0   & 3
                                        \end{bNiceMatrix}
          \end{align}
          Using Gerschgorin's extension, the disk for the third row has to remain
          disjoint from the other disks.
          \begin{align}
              10 - 0.1 - 0.2k     & > 3 + \frac{0.2}{k} &
              0.2k^2 - 6.9k + 0.2 & < 0                                \\
              k_1                 & = 34.37             & k_2 & = 0.03
          \end{align}
          This means the largest possible integer is $ k^* = 34 $.

    \item The eigenvalues are located at the diagonal entries with an error
          which is a disk of radius $ (n-1)\cdot\num{d-5} $. \par
          Since the matrix has $ n $ rows, the other $ (n-1) $ entries can at most have
          size $ \num{d-5} $, which makes the maximum radius of the Gerschgorin disk
          proportional to $ (n-1) $

    \item From Problem $ 1 $, the eigenvalue $ \lambda = 4 $ lies on the disk
          $ \abs{z - 0} \leq \abs{-2} + \abs{2} $

    \item Looking at the sum of aboslute values of elements within a row,
          \begin{align}
              \abs{a_{j1}} + \dots + \abs{a_{jn}} & \geq
              \abs{\lambda - a_{jj}} + \abs{a_{jj}} \geq \abs{\lambda}
          \end{align}
          using the triangle inqeuality. Maximizing the left side over all columns,
          \begin{align}
              \max_j{\sum_{k=1}^{n} \abs{a_{jk}}} & \geq \abs{\lambda} \\
              \lVert \vec{A} \rVert_\infty        & \geq \rho(\vec{A})
          \end{align}
          Since this is true for all the eigenvalues of the matrix, the quantity on the
          right is the spectral radius.

    \item Using Schur's inequality to find an upper bound on the spectral radius,
          \begin{align}
              \sum_{k=1}^{n} \sum_{j=1}^{n} \abs{a_{jk}}^2 & = 181           &
              \abs{\lambda}                                & \leq \sqrt{181}
          \end{align}

    \item Using Schur's inequality to find an upper bound on the spectral radius,
          \begin{align}
              \sum_{k=1}^{n} \sum_{j=1}^{n} \abs{a_{jk}}^2 & = 122           &
              \abs{\lambda}                                & \leq \sqrt{122}
          \end{align}

    \item Using Schur's inequality to find an upper bound on the spectral radius,
          \begin{align}
              \sum_{k=1}^{n} \sum_{j=1}^{n} \abs{a_{jk}}^2 & = 145.1           &
              \abs{\lambda}                                & \leq \sqrt{145.1}
          \end{align}

    \item Using Schur's inequality to find an upper bound on the spectral radius,
          \begin{align}
              \sum_{k=1}^{n} \sum_{j=1}^{n} \abs{a_{jk}}^2 & = 0.52           &
              \abs{\lambda}                                & \leq \sqrt{0.52}
          \end{align}

    \item Using Schur's inequality to find an upper bound on the spectral radius,
          \begin{align}
              \sum_{k=1}^{n} \sum_{j=1}^{n} \abs{a_{jk}}^2 & = 83           &
              \abs{\lambda}                                & \leq \sqrt{83}
          \end{align}

    \item Verifying that the matrix is normal,
          \begin{align}
              \vec{A}              & = \begin{bNiceMatrix}[margin]
                                           2    & \i & 1 + \i \\
                                           -\i  & 3  & 0      \\
                                           1-\i & 0  & 8
                                       \end{bNiceMatrix} &
              \vec{A}^\dag         & = \begin{bNiceMatrix}[margin]
                                           2    & \i & 1 + \i \\
                                           -\i  & 3  & 0      \\
                                           1-\i & 0  & 8
                                       \end{bNiceMatrix}    \\
              \vec{A}^\dag         & = \vec{A}                     &
              \vec{A}^\dag \vec{A} & = \vec{A} \vec{A}^\dag
          \end{align}

    \item For Hermitian matrices,
          \begin{align}
              \vec{A}^\dag         & = \vec{A}                         &
              \vec{A}^\dag \vec{A} & = \vec{A}^2 = \vec{A}\vec{A}^\dag
          \end{align}
          For skew-Hermitian matrices,
          \begin{align}
              \vec{A}^\dag         & = -\vec{A}                         &
              \vec{A}^\dag \vec{A} & = -\vec{A}^2 = \vec{A}\vec{A}^\dag
          \end{align}
          For unitary matrices,
          \begin{align}
              \vec{A}^\dag         & = \vec{A}^{-1}                  &
              \vec{A}^\dag \vec{A} & = \vec{I} = \vec{A}\vec{A}^\dag
          \end{align}

    \item From Gerschgorin's theorem and given the matrix is diagonally dominant,
          \begin{align}
              \abs{\lambda - a_{jj}}       & \leq \sum_{k \neq j} \abs{a_{jk}} &
              \sum_{k \neq j} \abs{a_{jk}} & \leq \abs{a_{jj}}                   \\
              \abs{\lambda - a_{jj}}       & \leq \abs{a_{jj}}
          \end{align}
          The eigenvalue is located inside a disk that excludes the origin. This means
          that zero is never one of the eigenvalues.
          \begin{align}
              \lambda_i       & \neq 0                             &
              \forall \quad i & \in \{1,2,\dots,n\}                  \\
              \det(\vec{A})   & = \prod_{i=1}^{n} \lambda_i \neq 0
          \end{align}
          Since the determinant is nonzero, the matrix is singular.

    \item Split $ \vec{A} $ into a diagonal matrix and the remainder, for some real
          scalar $ t $,
          \begin{align}
              \vec{A}                      & = \vec{B} + \vec{C}                   &
              \vec{A}_t                    & = \vec{B} + t\vec{C}                    \\
              \abs{\lambda - a_{jj}}       & \leq t\  \sum_{k \neq j} \abs{a_{jk}}   \\
              t                            & = 0                                   &
              \implies \quad \{\lambda_i\} & = \{a_{ii}\}
          \end{align}
          Starting from the $ t = 1 $ case, where $ \vec{A}_1 = \vec{A} $, assume that
          one of the disks does not contain an eigenvalue. \par
          As $ t $ decreases to zero, the radii of these disks also decrease to zero
          continuously. At the other extreme, at $ t = 0 $, the disks reduce to a
          set of points that are the eigenvalues themselves. \par
          This contradicts the initial assumption that the disk did not initially contain
          an eigenvalue, which proves the theorem.
\end{enumerate}