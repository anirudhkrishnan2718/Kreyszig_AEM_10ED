\section{Linear Systems: LU-Factorization, Matrix Inversion}

\begin{enumerate}
    \item LU factorization coded in \texttt{numpy}
          \begin{align}
              \vec{A}  & = \bmattt{4}{5}{12}{14}                     &
              \vec{LU} & = \bmattt{1}{0}{3}{1}\ \bmattt{4}{5}{0}{-1}   \\
              \vec{Ly} & = \vec{b} = \bmatcol{14}{36}                &
              \vec{y}  & = \bmatcol{14}{-6}                            \\
              \vec{Ux} & = \vec{y}                                   &
              \vec{x}  & = \bmatcol{-4}{6}
          \end{align}

    \item LU factorization coded in \texttt{numpy}
          \begin{align}
              \vec{A}  & = \bmattt{2}{9}{3}{-5}                           &
              \vec{LU} & = \bmattt{1}{0}{1.5}{1}\ \bmattt{2}{9}{0}{-18.5}   \\
              \vec{Ly} & = \vec{b} = \bmatcol{82}{-62}                    &
              \vec{y}  & = \bmatcol{82}{-185}                               \\
              \vec{Ux} & = \vec{y}                                        &
              \vec{x}  & = \bmatcol{-4}{10}
          \end{align}

    \item LU factorization coded in \texttt{numpy}
          \begin{align}
              \vec{\tilde{A}}
                                                    & =
              \begin{bNiceArray}{rrr|r}[margin]
                  \CodeBefore
                  \columncolor[opacity=0.05]{black}{4}
                  \Body
                  5  & 4  & 1  & 6.8  \\
                  10 & 9  & 4  & 17.6 \\
                  10 & 13 & 15 & 38.4 \\
              \end{bNiceArray}  &
              \vec{LU}                              & =
              \begin{bNiceMatrix}[margin]
                  1 & 0 & 0 \\
                  2 & 1 & 0 \\
                  2 & 5 & 1
              \end{bNiceMatrix}\ \begin{bNiceMatrix}[margin]
                                     5 & 4 & 1 \\
                                     0 & 1 & 2 \\
                                     0 & 0 & 3
                                 \end{bNiceMatrix} \\
              \vec{Ly}                              & =
              \vec{b} \quad \implies \quad
              \vec{y} = \begin{bNiceMatrix}[margin]
                            6.8 \\ 4 \\ 4.8
                        \end{bNiceMatrix} &
              \vec{Ux}                              & =
              \vec{y} \quad \implies \quad
              \vec{x} = \begin{bNiceMatrix}[margin]
                            0.4 \\ 0.8 \\ 1.6
                        \end{bNiceMatrix}
          \end{align}

    \item LU factorization coded in \texttt{numpy}
          \begin{align}
              \vec{\tilde{A}}
                                                    & =
              \begin{bNiceArray}{rrr|r}[margin]
                  \CodeBefore
                  \columncolor[opacity=0.05]{black}{4}
                  \Body
                  2  & 0 & 2  & 0  \\
                  -2 & 2 & 1  & 0  \\
                  1  & 2 & -2 & 18 \\
              \end{bNiceArray}  &
              \vec{LU}                              & =
              \begin{bNiceMatrix}[margin]
                  1   & 0   & 0 \\
                  -1  & 1   & 0 \\
                  0.5 & 0.5 & 1
              \end{bNiceMatrix}\ \begin{bNiceMatrix}[margin]
                                     2 & 1 & 2    \\
                                     0 & 3 & 3    \\
                                     0 & 0 & -4.5
                                 \end{bNiceMatrix} \\
              \vec{Ly}                              & =
              \vec{b} \quad \implies \quad
              \vec{y} = \begin{bNiceMatrix}[margin]
                            0 \\ 0 \\ 18
                        \end{bNiceMatrix} &
              \vec{Ux}                              & =
              \vec{y} \quad \implies \quad
              \vec{x} = \begin{bNiceMatrix}[margin]
                            2 \\ 4 \\ -4
                        \end{bNiceMatrix}
          \end{align}

    \item LU factorization coded in \texttt{numpy}
          \begin{align}
              \vec{\tilde{A}}
                                                    & =
              \begin{bNiceArray}{rrr|r}[margin]
                  \CodeBefore
                  \columncolor[opacity=0.05]{black}{4}
                  \Body
                  3  & 9   & 6  & 4.6  \\
                  18 & 48  & 39 & 27.2 \\
                  9  & -27 & 42 & 9    \\
              \end{bNiceArray}  &
              \vec{LU}                              & =
              \begin{bNiceMatrix}[margin]
                  1 & 0 & 0 \\
                  6 & 1 & 0 \\
                  3 & 9 & 1
              \end{bNiceMatrix}\ \begin{bNiceMatrix}[margin]
                                     3 & 9  & 6  \\
                                     0 & -6 & 3  \\
                                     0 & 0  & -3
                                 \end{bNiceMatrix} \\
              \vec{Ly}                              & =
              \vec{b} \quad \implies \quad
              \vec{y} = \begin{bNiceMatrix}[margin]
                            4.6 \\ -0.4 \\ -1.2
                        \end{bNiceMatrix} &
              \vec{Ux}                              & =
              \vec{y} \quad \implies \quad
              \vec{x} = \begin{bNiceMatrix}[margin]
                            -\frac{1}{15} \\ \frac{4}{15} \\ \frac{2}{5}
                        \end{bNiceMatrix}
          \end{align}

    \item Crout's factorization
          \begin{enumerate}
              \item Perform Doolittle factorization to obtain
                    \begin{align}
                        \vec{A}   & = \vec{L}_d\ \vec{U}_d &
                        \vec{U}_d & = \vec{DU}               \\
                        \vec{L}   & = \vec{L}_d\ \vec{D}   &
                        \vec{A}   & = \vec{LU}
                    \end{align}
                    Here, $ D $ is a diagonal matrix of $ \vec{U}_d $

              \item Solving Problem $ 5 $ by Crout's factorization
                    \begin{align}
                        \vec{\tilde{A}}
                                                              & =
                        \begin{bNiceArray}{rrr|r}[margin]
                            \CodeBefore
                            \columncolor[opacity=0.05]{black}{4}
                            \Body
                            3  & 9   & 6  & 4.6  \\
                            18 & 48  & 39 & 27.2 \\
                            9  & -27 & 42 & 9    \\
                        \end{bNiceArray}  &
                        \vec{LU}                              & =
                        \begin{bNiceMatrix}[margin]
                            3  & 0   & 0  \\
                            18 & -6  & 0  \\
                            9  & -54 & -3
                        \end{bNiceMatrix}\ \begin{bNiceMatrix}[margin]
                                               1 & 3 & 2    \\
                                               0 & 1 & -0.5 \\
                                               0 & 0 & 1
                                           \end{bNiceMatrix} \\
                        \vec{Ly}                              & =
                        \vec{b} \quad \implies \quad
                        \vec{y} = \begin{bNiceMatrix}[margin]
                                      4.6 \\ -0.4 \\ -1.2
                                  \end{bNiceMatrix} &
                        \vec{Ux}                              & =
                        \vec{y} \quad \implies \quad
                        \vec{x} = \begin{bNiceMatrix}[margin]
                                      -\frac{1}{15} \\ \frac{4}{15} \\ \frac{2}{5}
                                  \end{bNiceMatrix}
                    \end{align}

              \item Factorizing Problem $ 5 $ by all three methods
                    \begin{align}
                        \vec{A}
                                             & =
                        \begin{bNiceMatrix}[margin]
                            1  & -4 & 2  \\
                            -4 & 25 & 4  \\
                            2  & 4  & 24
                        \end{bNiceMatrix}                   \\
                        \vec{L}_d\ \vec{U}_d & =
                        \begin{bNiceMatrix}[margin]
                            1  & 0   & 0 \\
                            -4 & 1   & 0 \\
                            2  & 4/3 & 1
                        \end{bNiceMatrix} \begin{bNiceMatrix}[margin]
                                              1 & -4 & 2  \\
                                              0 & 9  & 12 \\
                                              0 & 0  & 4
                                          \end{bNiceMatrix} \\
                        \vec{L}_c\ \vec{U}_c & =
                        \begin{bNiceMatrix}[margin]
                            1  & 0  & 0 \\
                            -4 & 9  & 0 \\
                            2  & 12 & 4
                        \end{bNiceMatrix} \begin{bNiceMatrix}[margin]
                                              1 & -4 & 2   \\
                                              0 & 1  & 4/3 \\
                                              0 & 0  & 1
                                          \end{bNiceMatrix} \\
                        \vec{L}_y\ \vec{U}_y & =
                        \begin{bNiceMatrix}[margin]
                            1  & 0 & 0 \\
                            -4 & 3 & 0 \\
                            2  & 4 & 2
                        \end{bNiceMatrix} \begin{bNiceMatrix}[margin]
                                              1 & -4 & 2 \\
                                              0 & 3  & 4 \\
                                              0 & 0  & 2
                                          \end{bNiceMatrix}
                    \end{align}

              \item Perform Doolittle factorization and then divide each row
                    of $ \vec{U}_d $ by the square root of the diagonal term. Then,
                    its transpose is $ \vec{L} $

              \item When the original matrix $ A $ is symmetric,
                    \begin{align}
                        \vec{A}              & = \vec{L}_d\ \vec{U}_d     &
                        \vec{A}^T            & = \vec{U}_d^T\ \vec{L}_d^T   \\
                        \vec{U}_d^T          & = \vec{L}_c                &
                        \vec{L}_d^T          & = \vec{U}_c                  \\
                        \vec{L}_c\ \vec{U}_c & = \vec{A}^T = \vec{A}
                    \end{align}
                    Thus, the Doolittle and Crout's factorizationa re inter-convertible
                    directly, when $ \vec{A} $ is symmetric.
          \end{enumerate}

    \item LU factorization coded in \texttt{numpy}
          \begin{align}
              \vec{\tilde{A}}
                                                    & =
              \begin{bNiceArray}{rrr|r}[margin]
                  \CodeBefore
                  \columncolor[opacity=0.05]{black}{4}
                  \Body
                  9  & 6  & 12 & 17.4 \\
                  6  & 13 & 11 & 23.6 \\
                  12 & 11 & 26 & 30.8 \\
              \end{bNiceArray}  &
              \vec{LU}                              & =
              \begin{bNiceMatrix}[margin]
                  3 & 0 & 0 \\
                  2 & 3 & 0 \\
                  4 & 1 & 3
              \end{bNiceMatrix}\ \begin{bNiceMatrix}[margin]
                                     3 & 2 & 4 \\
                                     0 & 3 & 1 \\
                                     0 & 0 & 3
                                 \end{bNiceMatrix} \\
              \vec{Ly}                              & =
              \vec{b} \quad \implies \quad
              \vec{y} = \begin{bNiceMatrix}[margin]
                            17.4 \\ 12 \\ 3.6
                        \end{bNiceMatrix} &
              \vec{Ux}                              & =
              \vec{y} \quad \implies \quad
              \vec{x} = \begin{bNiceMatrix}[margin]
                            0.6 \\ 1.2 \\ 0.4
                        \end{bNiceMatrix}
          \end{align}

    \item LU factorization coded in \texttt{numpy}
          \begin{align}
              \vec{\tilde{A}}
                                                    & =
              \begin{bNiceArray}{rrr|r}[margin]
                  \CodeBefore
                  \columncolor[opacity=0.05]{black}{4}
                  \Body
                  4 & 6  & 8   & 0    \\
                  6 & 34 & 52  & -160 \\
                  8 & 52 & 129 & -452 \\
              \end{bNiceArray}  &
              \vec{LU}                              & =
              \begin{bNiceMatrix}[margin]
                  2 & 0 & 0 \\
                  3 & 5 & 0 \\
                  4 & 8 & 7
              \end{bNiceMatrix}\ \begin{bNiceMatrix}[margin]
                                     2 & 3 & 4 \\
                                     0 & 5 & 8 \\
                                     0 & 0 & 7
                                 \end{bNiceMatrix} \\
              \vec{Ly}                              & =
              \vec{b} \quad \implies \quad
              \vec{y} = \begin{bNiceMatrix}[margin]
                            0 \\ -160 \\ -196
                        \end{bNiceMatrix} &
              \vec{Ux}                              & =
              \vec{y} \quad \implies \quad
              \vec{x} = \begin{bNiceMatrix}[margin]
                            8 \\ 0 \\ -4
                        \end{bNiceMatrix}
          \end{align}

    \item LU factorization coded in \texttt{numpy}
          \begin{align}
              \vec{\tilde{A}}
                                                    & =
              \begin{bNiceArray}{rrr|r}[margin]
                  \CodeBefore
                  \columncolor[opacity=0.05]{black}{4}
                  \Body
                  1 & 0 & 3  & 14 \\
                  0 & 2 & 1  & 2  \\
                  3 & 8 & 14 & 54 \\
              \end{bNiceArray}  &
              \vec{LU}                              & =
              \begin{bNiceMatrix}[margin]
                  1 & 0          & 0 \\
                  0 & \sqrt{2}   & 0 \\
                  3 & 1/\sqrt{2} & 1
              \end{bNiceMatrix}\ \begin{bNiceMatrix}[margin]
                                     1 & 0        & 3          \\
                                     0 & \sqrt{2} & 1/\sqrt{2} \\
                                     0 & 0        & 1
                                 \end{bNiceMatrix} \\
              \vec{Ly}                              & =
              \vec{b} \quad \implies \quad
              \vec{y} = \begin{bNiceMatrix}[margin]
                            14 \\ 2 \\ 4
                        \end{bNiceMatrix} &
              \vec{Ux}                              & =
              \vec{y} \quad \implies \quad
              \vec{x} = \begin{bNiceMatrix}[margin]
                            2 \\ -1 \\ 4
                        \end{bNiceMatrix}
          \end{align}

    \item LU factorization coded in \texttt{numpy}
          \begin{align}
              \vec{\tilde{A}}
                                                    & =
              \begin{bNiceArray}{rrr|r}[margin]
                  \CodeBefore
                  \columncolor[opacity=0.05]{black}{4}
                  \Body
                  4 & 0 & 2 & 1.5 \\
                  0 & 4 & 1 & 4   \\
                  2 & 1 & 2 & 2.5 \\
              \end{bNiceArray}  &
              \vec{LU}                              & =
              \begin{bNiceMatrix}[margin]
                  2 & 0   & 0          \\
                  0 & 2   & 0          \\
                  1 & 0.5 & \sqrt{3/4}
              \end{bNiceMatrix}\ \begin{bNiceMatrix}[margin]
                                     2 & 0 & 1          \\
                                     0 & 2 & 0.5        \\
                                     0 & 0 & \sqrt{3/4}
                                 \end{bNiceMatrix} \\
              \vec{Ly}                              & =
              \vec{b} \quad \implies \quad
              \vec{y} = \begin{bNiceMatrix}[margin]
                            1.5 \\ 4 \\ 3/4
                        \end{bNiceMatrix} &
              \vec{Ux}                              & =
              \vec{y} \quad \implies \quad
              \vec{x} = \begin{bNiceMatrix}[margin]
                            -1/8 \\ 3/4 \\ 1
                        \end{bNiceMatrix}
          \end{align}

    \item LU factorization coded in \texttt{numpy}
          \begin{align}
              \vec{\tilde{A}}
                                                    & =
              \begin{bNiceArray}{rrrr|r}[margin]
                  \CodeBefore
                  \columncolor[opacity=0.05]{black}{5}
                  \Body
                  1  & -1 & 3  & 2  & 15  \\
                  -1 & 5  & -5 & -2 & -35 \\
                  3  & -5 & 19 & 3  & 94  \\
                  2  & -2 & 3  & 21 & 1
              \end{bNiceArray}  &
              \vec{LU}                              & =
              \begin{bNiceMatrix}[margin]
                  1  & 0  & 0  & 0 \\
                  -1 & 2  & 0  & 0 \\
                  3  & -1 & 3  & 0 \\
                  2  & 0  & -1 & 4
              \end{bNiceMatrix}\begin{bNiceMatrix}[margin]
                                   1 & -1 & 3  & 2  \\
                                   0 & 2  & -1 & 0  \\
                                   0 & 0  & 3  & -1 \\
                                   0 & 0  & 0  & 4
                               \end{bNiceMatrix} \\
              \vec{Ly}                              & =
              \vec{b} \quad \implies \quad
              \vec{y} = \begin{bNiceMatrix}[margin]
                            15 \\ -20 \\ 39 \\ -16
                        \end{bNiceMatrix} &
              \vec{Ux}                              & =
              \vec{y} \quad \implies \quad
              \vec{x} = \begin{bNiceMatrix}[margin]
                            2 \\ -3 \\ 4 \\ -1
                        \end{bNiceMatrix}
          \end{align}

    \item LU factorization coded in \texttt{numpy}
          \begin{align}
              \vec{\tilde{A}}
                                                    & =
              \begin{bNiceArray}{rrrr|r}[margin]
                  \CodeBefore
                  \columncolor[opacity=0.05]{black}{5}
                  \Body
                  4 & 2 & 4 & 0 & 20  \\
                  2 & 2 & 3 & 2 & 36  \\
                  4 & 3 & 6 & 3 & 60  \\
                  0 & 2 & 3 & 9 & 122
              \end{bNiceArray}  &
              \vec{LU}                              & =
              \begin{bNiceMatrix}[margin]
                  2 & 0 & 0 & 0 \\
                  1 & 1 & 0 & 0 \\
                  2 & 1 & 1 & 0 \\
                  0 & 2 & 1 & 2
              \end{bNiceMatrix}\begin{bNiceMatrix}[margin]
                                   2 & 1 & 2 & 0 \\
                                   0 & 1 & 1 & 2 \\
                                   0 & 0 & 1 & 1 \\
                                   0 & 0 & 0 & 2
                               \end{bNiceMatrix} \\
              \vec{Ly}                              & =
              \vec{b} \quad \implies \quad
              \vec{y} = \begin{bNiceMatrix}[margin]
                            20 \\ 26 \\ 14 \\ 56
                        \end{bNiceMatrix} &
              \vec{Ux}                              & =
              \vec{y} \quad \implies \quad
              \vec{x} = \begin{bNiceMatrix}[margin]
                            6 \\ -2 \\ 0 \\ 14
                        \end{bNiceMatrix}
          \end{align}

    \item Given $ \vec{A} $ and $ \vec{B} $ are positive definite,
          \begin{align}
              \vec{x}^T (-\vec{A})\ \vec{x}         & < 0                           \\
              \vec{x}^T \vec{A}^T \vec{x}           & = (\vec{x}^T \vec{A x})^T > 0 \\
              \vec{x}^T (\vec{A} + \vec{B}) \vec{x} & = \vec{x}^T \vec{A} \vec{x}
              + \vec{x}^T \vec{B} \vec{x} > 0
          \end{align}
          No such claim can be made for $ \vec{A} - \vec{B} $ since the end result of
          subtracting one positive number from another need not be positive.

    \item Cholesky method coded in \texttt{numpy}
          \begin{enumerate}
              \item Using Cholesky method on Example $ 2 $,
                    \begin{align}
                        \vec{A}                               &
                        = \begin{bNiceArray}{rrr|r}[margin]
                              \CodeBefore
                              \columncolor[opacity=0.05]{black}{4}
                              \Body
                              4  & 2  & 14 & 14   \\
                              2  & 17 & -5 & -101 \\
                              14 & -5 & 83 & 155  \\
                          \end{bNiceArray}  &
                        \vec{LU}                              & =
                        \begin{bNiceMatrix}[margin]
                            2 & 0  & 0 \\
                            1 & 4  & 0 \\
                            7 & -3 & 5
                        \end{bNiceMatrix}\ \begin{bNiceMatrix}[margin]
                                               2 & 1 & 7  \\
                                               0 & 4 & -3 \\
                                               0 & 0 & 5
                                           \end{bNiceMatrix} \\
                        \vec{Ly}                              & =
                        \vec{b} \quad \implies \quad
                        \vec{y} = \begin{bNiceMatrix}[margin]
                                      14 \\ -108 \\ 25
                                  \end{bNiceMatrix} &
                        \vec{Ux}                              & =
                        \vec{y} \quad \implies \quad
                        \vec{x} = \begin{bNiceMatrix}[margin]
                                      3 \\ -6 \\ 1
                                  \end{bNiceMatrix}
                    \end{align}
                    The code was already used in solving Problems $ 7,8,9 $

              \item Applying Cholesky factorization to the spline matrices
                    \begin{align}
                        \vec{A}  & = \begin{bNiceMatrix}[margin]
                                         2 & 1 & 0 \\
                                         1 & 4 & 1 \\
                                         0 & 1 & 2
                                     \end{bNiceMatrix} \\
                        \vec{LU} & =
                        \begin{bNiceMatrix}[margin]
                            \sqrt{2}   & 0          & 0           \\
                            1/\sqrt{2} & \sqrt{7/2} & 0           \\
                            0          & \sqrt{2/7} & \sqrt{12/7}
                        \end{bNiceMatrix}\ \begin{bNiceMatrix}[margin]
                                               \sqrt{2} & 1/\sqrt{2} & 0           \\
                                               0        & \sqrt{7/2} & \sqrt{2/7}  \\
                                               0        & 0          & \sqrt{12/7}
                                           \end{bNiceMatrix}
                    \end{align}
                    For the $ 4 \times 4 $ matrix,
                    \begin{align}
                        \vec{A}  & = \begin{bNiceMatrix}[margin]
                                         2 & 1 & 0 & 0 \\
                                         1 & 4 & 1 & 0 \\
                                         0 & 1 & 4 & 1 \\
                                         0 & 0 & 1 & 2
                                     \end{bNiceMatrix} \\
                        \vec{LU} & =
                        \begin{bNiceMatrix}[margin]
                            \sqrt{2}   & 0          & 0           & 0            \\
                            1/\sqrt{2} & \sqrt{7/2} & 0           & 0            \\
                            0          & \sqrt{2/7} & \sqrt{26/7} & 0            \\
                            0          & 0          & \sqrt{7/26} & \sqrt{45/26}
                        \end{bNiceMatrix}
                        \ \begin{bNiceMatrix}[margin]
                              \sqrt{2} & 1/\sqrt{2} & 0           & 0            \\
                              0        & \sqrt{7/2} & \sqrt{2/7}  & 0            \\
                              0        & 0          & \sqrt{26/7} & \sqrt{7/26}  \\
                              0        & 0          & 0           & \sqrt{45/26} \\
                          \end{bNiceMatrix}
                    \end{align}
          \end{enumerate}

    \item Finding inverse using Gauss-Jordan elimination
          \begin{align}
               & \begin{bNiceArray}{rr|rr}[margin]
                     \CodeBefore
                     \columncolor[opacity=0.05]{y_h}{1-2}
                     \Body
                     4  & 5  & 1 & 0 \\
                     12 & 14 & 0 & 1
                 \end{bNiceArray} &
               & \begin{bNiceArray}{rr|rr}[margin]
                     4 & 5  & 1  & 0 \\
                     0 & -1 & -3 & 1
                 \end{bNiceArray}    \\
               & \begin{bNiceArray}{rr|rr}[margin]
                     4 & 0  & -14 & 5 \\
                     0 & -1 & -3  & 1
                 \end{bNiceArray}    &
               & \begin{bNiceArray}{rr|rr}[margin]
                     \CodeBefore
                     \columncolor[opacity=0.05]{y_p}{3-4}
                     \Body
                     1 & 0 & -7/2 & 5/4 \\
                     0 & 1 & 3    & -1  \\
                 \end{bNiceArray}
          \end{align}

    \item Finding inverse using Gauss-Jordan elimination
          \begin{align}
               & \begin{bNiceArray}{rrr|rrr}[margin]
                     \CodeBefore
                     \columncolor[opacity=0.05]{y_h}{1-3}
                     \Body
                     2  & 1 & 2  & 1 & 0 & 0 \\
                     -2 & 2 & 1  & 0 & 1 & 0 \\
                     1  & 2 & -2 & 0 & 0 & 1
                 \end{bNiceArray} &
               & \begin{bNiceArray}{rrr|rrr}[margin]
                     2 & 1   & 2  & 1    & 0 & 0 \\
                     0 & 3   & 3  & 1    & 1 & 0 \\
                     0 & 1.5 & -3 & -0.5 & 0 & 1
                 \end{bNiceArray}  \\
               & \begin{bNiceArray}{rrr|rrr}[margin]
                     2 & 1 & 2    & 1  & 0    & 0 \\
                     0 & 3 & 3    & 1  & 1    & 0 \\
                     0 & 0 & -4.5 & -1 & -0.5 & 1
                 \end{bNiceArray}  &
               & \begin{bNiceArray}{rrr|rrr}[margin]
                     2 & 0 & 1    & 2/3 & -1/3 & 0 \\
                     0 & 3 & 3    & 1   & 1    & 0 \\
                     0 & 0 & -4.5 & -1  & -0.5 & 1
                 \end{bNiceArray}  \\
               & \begin{bNiceArray}{rrr|rrr}[margin]
                     2 & 0 & 0    & 4/9 & -4/9 & 2/9 \\
                     0 & 3 & 0    & 1/3 & 2/3  & 2/3 \\
                     0 & 0 & -4.5 & -1  & -0.5 & 1
                 \end{bNiceArray}  &
               & \begin{bNiceArray}{rrr|rrr}[margin]
                     \CodeBefore
                     \columncolor[opacity=0.05]{y_p}{4-6}
                     \Body
                     1 & 0 & 0 & 2/9 & -2/9 & 1/9  \\
                     0 & 1 & 0 & 1/9 & 2/9  & 2/9  \\
                     0 & 0 & 1 & 2/9 & 1/9  & -2/9
                 \end{bNiceArray}
          \end{align}

    \item Finding inverse using Gauss-Jordan elimination
          \begin{align}
               & \begin{bNiceArray}{rrr|rrr}[margin]
                     \CodeBefore
                     \columncolor[opacity=0.05]{y_h}{1-3}
                     \Body
                     1  & -4 & 2  & 1 & 0 & 0 \\
                     -4 & 25 & 4  & 0 & 1 & 0 \\
                     2  & 4  & 24 & 0 & 0 & 1
                 \end{bNiceArray}                        &
               & \begin{bNiceArray}{rrr|rrr}[margin]
                     1 & -4 & 2  & 1  & 0 & 0 \\
                     0 & 9  & 12 & 4  & 1 & 0 \\
                     0 & 12 & 20 & -2 & 0 & 1
                 \end{bNiceArray}                       \\
               & \begin{bNiceArray}{rrr|rrr}[margin]
                     1 & -4 & 2  & 1             & 0            & 0 \\
                     0 & 9  & 12 & 4             & 1            & 0 \\
                     0 & 0  & 4  & -\frac{22}{3} & -\frac{4}{3} & 1
                 \end{bNiceArray}           &
               & \begin{bNiceArray}{rrr|rrr}[margin]
                     1 & 0 & \frac{22}{3} & \frac{25}{9}  & \frac{4}{9}  & 0 \\
                     0 & 9 & 12           & 4             & 1            & 0 \\
                     0 & 0 & 4            & -\frac{22}{3} & -\frac{4}{3} & 1
                 \end{bNiceArray}  \\
               & \begin{bNiceArray}{rrr|rrr}[margin]
                     1 & 0 & 0 & \frac{146}{9} & \frac{26}{9} & -\frac{11}{6} \\
                     0 & 9 & 0 & 26            & 17           & -3            \\
                     0 & 0 & 4 & -\frac{22}{3} & -\frac{4}{3} & 1
                 \end{bNiceArray} &
               & \begin{bNiceArray}{rrr|rrr}[margin]
                     \CodeBefore
                     \columncolor[opacity=0.05]{y_p}{4-6}
                     \Body
                     1 & 0 & 0 & \frac{146}{9} & \frac{26}{9} & -\frac{11}{6} \\
                     0 & 1 & 0 & \frac{26}{9}  & \frac{17}{9} & -\frac{1}{3}  \\
                     0 & 0 & 1 & -\frac{11}{6} & -\frac{1}{3} & \frac{1}{4}
                 \end{bNiceArray}
          \end{align}

    \item Finding inverse using Gauss-Jordan elimination
          \begin{align}
               & \begin{bNiceArray}{rrr|rrr}[margin]
                     \CodeBefore
                     \columncolor[opacity=0.05]{y_h}{1-3}
                     \Body
                     0.01 & 0    & 0.03 & 1 & 0 & 0 \\
                     0    & 0.16 & 0.08 & 0 & 1 & 0 \\
                     0.03 & 0.08 & 0.14 & 0 & 0 & 1
                 \end{bNiceArray}   &
               & \begin{bNiceArray}{rrr|rrr}[margin]
                     0.01 & 0    & 0.03 & 1  & 0 & 0 \\
                     0    & 0.16 & 0.08 & 0  & 1 & 0 \\
                     0    & 0.08 & 0.05 & -3 & 0 & 1
                 \end{bNiceArray}  \\
               & \begin{bNiceArray}{rrr|rrr}[margin]
                     0.01 & 0    & 0.03 & 1  & 0    & 0 \\
                     0    & 0.16 & 0.08 & 0  & 1    & 0 \\
                     0    & 0    & 0.01 & -3 & -0.5 & 1
                 \end{bNiceArray}  &
               & \begin{bNiceArray}{rrr|rrr}[margin]
                     0.01 & 0    & 0    & 10 & 1.5  & -3 \\
                     0    & 0.16 & 0.08 & 0  & 1    & 0  \\
                     0    & 0    & 0.01 & -3 & -0.5 & 1
                 \end{bNiceArray} \\
               & \begin{bNiceArray}{rrr|rrr}[margin]
                     0.01 & 0    & 0    & 10 & 1.5  & -3 \\
                     0    & 0.16 & 0    & 24 & 5    & -8 \\
                     0    & 0    & 0.01 & -3 & -0.5 & 1
                 \end{bNiceArray} &
               & \begin{bNiceArray}{rrr|rrr}[margin]
                     \CodeBefore
                     \columncolor[opacity=0.05]{y_p}{4-6}
                     \Body
                     1 & 0 & 0 & 1000 & 150   & -300 \\
                     0 & 1 & 0 & 150  & 31.25 & -50  \\
                     0 & 0 & 1 & -300 & -50   & 100
                 \end{bNiceArray}
          \end{align}

    \item Finding inverse using Gauss-Jordan elimination
          \begin{align}
               & \begin{bNiceArray}{rrrr|rrrr}[margin]
                     \CodeBefore
                     \columncolor[opacity=0.05]{y_h}{1-4}
                     \Body
                     4 & 2 & 4 & 0 & 1 & 0 & 0 & 0 \\
                     2 & 2 & 3 & 2 & 0 & 1 & 0 & 0 \\
                     4 & 3 & 6 & 3 & 0 & 0 & 1 & 0 \\
                     0 & 2 & 3 & 9 & 0 & 0 & 0 & 1 \\
                 \end{bNiceArray}                            &
               & \begin{bNiceArray}{rrrr|rrrr}[margin]
                     4 & 2 & 4 & 0 & 1    & 0 & 0 & 0 \\
                     0 & 1 & 1 & 2 & -0.5 & 1 & 0 & 0 \\
                     0 & 1 & 2 & 3 & -1   & 0 & 1 & 0 \\
                     0 & 2 & 3 & 9 & 0    & 0 & 0 & 1 \\
                 \end{bNiceArray}                          \\
               & \begin{bNiceArray}{rrrr|rrrr}[margin]
                     4 & 2 & 4 & 0 & 1    & 0  & 0 & 0 \\
                     0 & 1 & 1 & 2 & -0.5 & 1  & 0 & 0 \\
                     0 & 0 & 1 & 1 & -0.5 & -1 & 1 & 0 \\
                     0 & 0 & 1 & 5 & 1    & -2 & 0 & 1 \\
                 \end{bNiceArray}                            &
               & \begin{bNiceArray}{rrrr|rrrr}[margin]
                     4 & 2 & 4 & 0 & 1    & 0  & 0  & 0 \\
                     0 & 1 & 1 & 2 & -0.5 & 1  & 0  & 0 \\
                     0 & 0 & 1 & 1 & -0.5 & -1 & 1  & 0 \\
                     0 & 0 & 0 & 4 & 1.5  & -1 & -1 & 1 \\
                 \end{bNiceArray}                          \\
               & \begin{bNiceArray}{rrrr|rrrr}[margin]
                     4 & 0 & 2 & -4 & 2    & -2 & 0  & 0 \\
                     0 & 1 & 1 & 2  & -0.5 & 1  & 0  & 0 \\
                     0 & 0 & 1 & 1  & -0.5 & -1 & 1  & 0 \\
                     0 & 0 & 0 & 4  & 1.5  & -1 & -1 & 1 \\
                 \end{bNiceArray}                           &
               & \begin{bNiceArray}{rrrr|rrrr}[margin]
                     4 & 0 & 0 & -6 & 3    & 0  & -2 & 0 \\
                     0 & 1 & 0 & 1  & 0    & 2  & -1 & 0 \\
                     0 & 0 & 1 & 1  & -0.5 & -1 & 1  & 0 \\
                     0 & 0 & 0 & 4  & 1.5  & -1 & -1 & 1 \\
                 \end{bNiceArray}                          \\
               & \begin{bNiceArray}{rrrr|rrrr}[margin]
                     4             & 0            & 0             & 0            &
                     \frac{21}{4}  & -\frac{3}{2} & - \frac{7}{2} & \frac{3}{2}    \\
                     0             & 1            & 0             & 0            &
                     - \frac{3}{8} & \frac{9}{4}  & -\frac{3}{4}  & -\frac{1}{4}   \\
                     0             & 0            & 1             & 0            &
                     - \frac{7}{8} & -\frac{3}{4} & \frac{5}{4}   & -\frac{1}{4}   \\
                     0             & 0            & 0             & 4            &
                     1.5           & -1           & -1            & 1              \\
                 \end{bNiceArray} &
               & \begin{bNiceArray}{rrrr|rrrr}[margin]
                     \CodeBefore
                     \columncolor[opacity=0.05]{y_p}{5-8}
                     \Body
                     1             & 0            & 0            & 0            &
                     \frac{21}{16} & -\frac{3}{8} & \frac{-7}{8} & \frac{3}{8}    \\
                     0             & 1            & 0            & 0            &
                     - \frac{3}{8} & \frac{9}{4}  & -\frac{3}{4} & -\frac{1}{4}   \\
                     0             & 0            & 1            & 0            &
                     - \frac{7}{8} & -\frac{3}{4} & \frac{5}{4}  & -\frac{1}{4}   \\
                     0             & 0            & 0            & 1            &
                     \frac{3}{8}   & -\frac{1}{4} & -\frac{1}{4} & \frac{1}{4}    \\
                 \end{bNiceArray}
          \end{align}

    \item Finding the determinant
          \begin{align}
              \vec{A}       & = \begin{bNiceMatrix}[margin]
                                    1/3  & 1/4   & 2     \\
                                    -1/9 & 1     & 1/7   \\
                                    4/63 & -3/28 & 13/49
                                \end{bNiceMatrix}               \\
              \det{\vec{A}} & = (1/3) [13/49 + (1/7)(3/28)]    \nonumber \\
                            & + (1/9)[(1/4)(13/49)+ (2)(3/28)] \nonumber \\
                            & + (4/63)[(1/4)(1/7) - 2] = 0
          \end{align}
          Using increasingly many significant digits,
          \begin{align}
              \det{\vec{A_2}} & = 0.089 + 0.031 + 0.063 = 0.18                     \\
              \det{\vec{A_3}} & = 0.0932 + 0.0311 - 0.124 = 0.0003                 \\
              \det{\vec{A_4}} & = 0.09352 + 0.03116 - 0.12469436 = -\num{1.436d-5} \\
              \det{\vec{A_5}} & = 0.093539 + 0.031178  - 0.12471 = \num{7d-6}
          \end{align}
          The determinant gets closer to the real value with more significant digits.
\end{enumerate}