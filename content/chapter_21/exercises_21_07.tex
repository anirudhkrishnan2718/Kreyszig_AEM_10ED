\section{Method for Hyperbolic PDEs}

\begin{enumerate}
    \item Showing only the result for the last timestep,
          \begin{align}
              h       & = 0.2, \qquad k = 0.2, \qquad
              r = \frac{k^2}{h^2} = 1                         \\
              \vec{u} & = \begin{bNiceMatrix}[margin]
                              0 & -0.05 & -0.1 & -0.15 & -0.2 & 0
                          \end{bNiceMatrix}
          \end{align}

    \item Showing only the result for the last timestep,
          \begin{align}
              h       & = 0.2, \qquad k = 0.2, \qquad
              r = \frac{k^2}{h^2} = 1                            \\
              \vec{u} & = \begin{bNiceMatrix}[margin]
                              0 & 0.032 & 0.096 & 0.144 & -0.128 & 0
                          \end{bNiceMatrix}
          \end{align}

    \item Showing only the result for the last timestep,
          \begin{align}
              h       & = 0.2, \qquad k = 0.2, \qquad
              r = \frac{k^2}{h^2} = 1                          \\
              \vec{u} & = \begin{bNiceMatrix}[margin]
                              0 & 0.032 & 0.048 & 0.48 & 0.032 & 0
                          \end{bNiceMatrix}
          \end{align}

    \item Using D'Alembert's solution to the wave equation, with the initial position and
          displacement given by $ f(x) $ and $ g(x) $ respectively.
          \begin{align}
              u(x, t)   & = \frac{f(x-ct) + f(x+ct)}{2} + \frac{1}{2c}
              \ \int_{x-ct}^{x+ct} g(s)\ \dl s                         \\
              c         & = 1, \qquad \Delta t = k                     \\
              u(x_i, 1) & = \frac{u_{i-1,0} + u_{i+1,0}}{2}
              + \frac{1}{2}\ \int_{x-k}^{x+k} g(s)\ \dl s
          \end{align}
          If $ g(s) $ is a constant function, then the integral is $ 2k\ g(x_i) $, and
          this equation simplifies to the one in this section in the text.

    \item Showing only the result for the last timestep, using the symmetry of the
          problem to tabulate only the left half of the bar.
          \begin{align}
              h & = 0.1, \qquad k = 0.1, \qquad
              r = \frac{k^2}{h^2} = 1                           \\
              \bmatcol{u(t=0.1)}{u(t=0.2)}
                & = \begin{bNiceMatrix}[margin]
                        0        & 0.354492 & 0.766    & 1.271    &
                        1.678508 & 1.834017 & \dots                 \\
                        0        & 0.575017 & 0.934508 & 1.135492 &
                        1.296    & 1.357017 & \dots
                    \end{bNiceMatrix}
          \end{align}

    \item Applying the B.C. at the left edge,
          \begin{align}
              u_{01}        & = \frac{u_{-1,0} + u_{1,0}}{2}   &
              2h \cdot 0.2j & = u_{-1,j} - u_{1,j}               \\
              u_{0,j+1}     & = u_{-1,j} + u_{1,j} - u_{0,j-1} &
                            & = 0.4jh + 2u_{1,j} - u_{0,j-1}
          \end{align}
          At the end of 5 time steps,
          \begin{align}
              \vec{u} & = \begin{bNiceMatrix}[margin]
                              0.852 & 1.728 & 2.168 & 2.762 & 3.24 & 4
                          \end{bNiceMatrix}
          \end{align}

    \item Using the given values, at $ t = 0.4 $
          \begin{align}
              h       & = 0.2, \qquad k = 0.2, \qquad
              r = \frac{k^2}{h^2} = 1                                       \\
              \vec{u} & = \begin{bNiceMatrix}[margin]
                              0 & 0.190211 & 0.307768 & 0.307768 & 0.190211 & 0
                          \end{bNiceMatrix}
          \end{align}
          Using D'Alembert's solution to the wave equation,
          \begin{align}
              u(x,t)  & = \frac{f(x+ct) - f(x-ct)}{2} + \frac{1}{2} \int_{x-ct}^{x+ct}
              g(s) \ \dl s                                                             \\
              u(x, t) & = 0 + 0 + \frac{1}{2} \int{x-t}{x+t} \sin(\pi s)\ \dl s        \\
              u(x, t) & = \frac{1}{2\pi} \Bigg[ \cos(\pi s) \Bigg]_{x+t}^{x-t}         \\
              u(x, t) & = \frac{\sin(\pi x)\ \sin(\pi t)}{\pi}                         \\
              \vec{u}_{\text{Ex}}
                      & = \begin{bNiceMatrix}[margin]
                              0 & 0.1779 & 0.2879 & 0.2879 & 0.1779 & 0
                          \end{bNiceMatrix}
          \end{align}

    \item Comparing the accuracy upon using a finer grid, for the matching
          positions as in Problem $ 7 $,
          \begin{align}
              \vec{u}_{0.2} & = \begin{bNiceMatrix}[margin]
                                    0 & 0.190211 & 0.307768 & 0.307768 & 0.190211 & 0
                                \end{bNiceMatrix} \\
              \vec{u}_{0.1} & = \begin{bNiceMatrix}[margin]
                                    0 & 0.180902 & 0.292705 & 0.292705 & 0.180902 & 0
                                \end{bNiceMatrix} \\
              \vec{u}_{\text{Ex}}
                            & = \begin{bNiceMatrix}[margin]
                                    0 & 0.1779 & 0.2879 & 0.2879 & 0.1779 & 0
                                \end{bNiceMatrix}
          \end{align}
          The smaller value of $ h $ gives a more accurate result.

    \item Using the formula $ 8 $ in the text, the wave is
          \begin{align}
              \begin{bNiceMatrix}[margin]
                  0 & 0.354492 & 0.766 & 1.271 & 1.678508 & 1.834017 & \dots
              \end{bNiceMatrix}
          \end{align}

    \item From D'Alembert's solution, with initial velocity being zero,
          \begin{align}
              u(x, t) & = \frac{f(x-t) + f(x+t)}{2}   &
              r^*     & = 1, \qquad k = h               \\
              u_{i1}  & = \frac{f(ih-h) + f(ih+h)}{2} &
              u_{i1}  & = f(ih,h) = u(ih,k)
          \end{align}
          Thus, $ u_{i1} $ are the exact values $ u(ih,k) $. Next, for $ j=2 $,
          \begin{align}
              u_{i2} & = u_{i-1,1} + u_{i+1,1} - u_{i,0} &
                     & = u(ih-h,k) + u(ih+h,k) - f(ih)     \\
                     & = \frac{f(ih-2h) + f(ih+2h)}{2}   &
                     & = u(ih, 2h) = u(ih,2k)
          \end{align}
          Thus, $ u_{i2} $ are the exact values. Now, assuming the relation holds for
          $ u_{ij} $ and $ u_{i,j-1} $,
          \begin{align}
              u_{i,j+1} & = u_{i-1,j} + u_{i+1,j} - u_{i,j-1}              \\
                        & = \frac{f(ih-h-jh) + f(ih+h+jh)}{2} = u(ih,jh+h)
          \end{align}
          Thus, $ u_{i,j+1} $ are also the exact values. Since $ j=1 $ and $ j=2 $ are
          already shown to be exact, induction proves the relation for all higher
          values of $ j $.
\end{enumerate}