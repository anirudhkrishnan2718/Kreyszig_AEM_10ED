\section{Heat Equation: Modeling Very Long Bars, Solution using Fourier transforms}

\begin{enumerate}
    \item Plotting the error function using \texttt{gnuplot},
          \begin{enumerate}
              \item Refer part $ b $
              \item USing a sequence of plots to visualize time advancing,
                    \begin{figure}[H]
                        \centering
                        \begin{tikzpicture}[declare function = {
                                        mode(\n,\t,\x) = sin(\n *\x) * exp(-\n^2 * \t);
                                    }]
                            \begin{axis}[
                                    legend pos = outer north east, title = {$u(x, t)$},
                                    grid = both,Ani,xlabel = $ x $,
                                    colormap/jet, colorbar, domain = -3:3,
                                    cycle list = {[samples of colormap = 5]},
                                    samples = 300]
                                \foreach \k in {-3, -1, 0, 1, 3}
                                    {
                                        \edef\temp{%
                                            \noexpand \addplot+ [thick, point meta = \k]
                                            gnuplot
                                                {(100 / 2) * (erf((1-x)/(2*sqrt(2^\k))) -
                                                    erf((-1-x)/(2*sqrt(2^\k))))};
                                        }\temp
                                    }
                                \addplot[thick, samples = 200,black, dashed]
                                {ifthenelse(abs(x) < 1, 100, 0)};
                            \end{axis}
                        \end{tikzpicture}
                    \end{figure}
              \item Since $ u(x, t) $ is independent of $ y $ the surface plot
                    is simply the same plot from part $ b $ extended in the $ y $
                    direction. \par
                    It would resemble a sheet which flattens over time to the $ xy $
                    plane.
          \end{enumerate}

    \item The solution in integral form is,
          \begin{align}
              f(x) & = \color{y_h} \begin{dcases}
                                       1 & \quad \abs{x} < a      \\
                                       0 & \quad \text{otherwise}
                                   \end{dcases}    \\
              A(p) & = \frac{1}{\pi} \intRL f(v) \cos(pv)\ \dl v
              = \frac{2}{\pi} \int_{0}^{a} \cos(pv)\ \dl v       \\
                   & = \frac{2}{p\pi} \Bigg[ \sin(px) \Bigg]_0^a
              = \color{y_p} \frac{2\sin(pa)}{p\pi}               \\
              B(p) & = \frac{1}{\pi} \intRL f(v) \sin(pv)\ \dl v
              = \frac{1}{\pi} \int_{-a}^{a} \sin(pv)\ \dl v
              = \color{y_p} 0
          \end{align}
          Using the Fourier integrals, the general solution is,
          \begin{align}
              u(x, t) = \infint \frac{2\sin(pa)\cos(px)}{p\pi}\ \exp(-c^2p^2 t)\ \dl p
          \end{align}

    \item The solution in integral form is,
          \begin{align}
              f(x) & = \color{y_h} \frac{1}{1 + x^2}                       \\
              A(p) & = \frac{1}{\pi} \intRL f(v) \cos(pv)\ \dl v
              = \frac{2}{\pi} \int_{0}^{a} \frac{\cos(pv)}{1 + v^2}\ \dl v \\
                   & = \color{y_p} e^{-p}                                  \\
              B(p) & = \frac{1}{\pi} \intRL f(v) \sin(pv)\ \dl v
              = \frac{1}{\pi} \int_{-a}^{a} \sin(pv)\ \dl v
              = \color{y_p} 0
          \end{align}
          Using the Fourier integrals, the general solution is,
          \begin{align}
              u(x, t) = \infint\ e^{-p}\cos(px)\ \exp(-c^2p^2 t)\ \dl p
          \end{align}

    \item The solution in integral form is,
          \begin{align}
              f(x) & = \color{y_h} \exp(-\abs{x})                               \\
              A(p) & = \frac{1}{\pi} \intRL f(v) \cos(pv)\ \dl v
              = \frac{2}{\pi} \infint e^{-v}\ \cos(pv)\ \dl v                   \\
                   & = \frac{2}{\pi} \Bigg[\frac{e^{-v}}{1 + p^2}\ [-\cos(pv) +
              p\sin(pv)]\Bigg]_0^\infty
              = \color{y_p} \frac{2}{\pi} \cdot \frac{1}{1 + p^2}               \\
              B(p) & = \frac{1}{\pi} \intRL f(v) \sin(pv)\ \dl v
              = \color{y_p} 0
          \end{align}
          Using the Fourier integrals, the general solution is,
          \begin{align}
              u(x, t) = \infint\ \frac{2}{\pi (1 + p^2)}\cos(px)
              \ \exp(-c^2p^2 t)\ \dl p
          \end{align}

    \item The solution in integral form is,
          \begin{align}
              f(x) & = \color{y_h} \begin{dcases}
                                       \abs{x} & \quad \abs{x} < 1      \\
                                       0       & \quad \text{otherwise}
                                   \end{dcases}          \\
              A(p) & = \frac{1}{\pi} \intRL f(v) \cos(pv)\ \dl v
              = \frac{2}{\pi} \int_{0}^{1} v\ \cos(pv)\ \dl v                \\
                   & = \frac{2}{\pi} \Bigg[\frac{pv\sin(pv) + \cos(pv)}{p^2}
                  \Bigg]_0^1
              = \color{y_p} \frac{2}{\pi} \cdot \frac{p\sin(p) + \cos(p) - 1}
              {p^2}                                                          \\
              B(p) & = \frac{1}{\pi} \intRL f(v) \sin(pv)\ \dl v
              = \color{y_p} 0
          \end{align}
          Using the Fourier integrals, the general solution is,
          \begin{align}
              u(x, t) = \infint\ \frac{2(p\sin p + \cos p - 1)}{\pi p^2}
              \ \cos(px)\ \exp(-c^2p^2 t)\ \dl p
          \end{align}

    \item The solution in integral form is,
          \begin{align}
              f(x) & = \color{y_h} \begin{dcases}
                                       x & \quad \abs{x} < 1      \\
                                       0 & \quad \text{otherwise}
                                   \end{dcases}                \\
              B(p) & = \frac{1}{\pi} \intRL f(v) \sin(pv)\ \dl v
              = \frac{2}{\pi} \int_{0}^{1} v\ \sin(pv)\ \dl v                \\
                   & = \frac{2}{\pi} \Bigg[\frac{\sin(pv) - pv\cos(pv)}{p^2}
                  \Bigg]_0^1
              = \color{y_p} \frac{2}{\pi} \cdot \frac{\sin p- p\cos p}
              {p^2}                                                          \\
              A(p) & = \frac{1}{\pi} \intRL f(v) \cos(pv)\ \dl v
              = \color{y_p} 0
          \end{align}
          Using the Fourier integrals, the general solution is,
          \begin{align}
              u(x, t) = \infint\ \frac{2(\sin p - p\cos p)}{\pi p^2} \sin(px)
              \ \exp(-c^2p^2 t)\ \dl p
          \end{align}

    \item The solution in integral form is,
          \begin{align}
              f(x) & = \color{y_h} \frac{\sin(x)}{x}                                  \\
              A(p) & = \frac{1}{\pi} \intRL f(v) \cos(pv)\ \dl v
              = \frac{2}{\pi} \infint \frac{\sin v}{v}\ \cos(pv)\ \dl v               \\
                   & = \frac{-2}{\pi} \infint \frac{1 - \cos(pv) - 1}{v}\ \sin(v)
              \ \dl v                                                                 \\
                   & = \frac{2}{\pi} \infint \frac{\sin(v)}{v}\ \dl v - \frac{2}{\pi}
              \infint  \frac{1 - \cos(pv)}{v}\ \sin(v)\ \dl v                         \\
              w    & = \frac{p v}{\pi} \qquad\qquad \dl w = \frac{p}{\pi}\ \dl v      \\
              A(p) & = 1 - \frac{2}{\pi} \infint \frac{1 - \cos(\pi w)}{w}
              \ \sin\Bigg( \frac{\pi w}{p} \Bigg)\ \dl w                              \\
                   & = \color{y_p} \begin{dcases}
                                       0 & \quad p > 1     \\
                                       1 & \quad 0 < p < 1
                                   \end{dcases}                                \\
              B(p) & = \frac{1}{\pi} \intRL f(v) \sin(pv)\ \dl v
              = \color{y_p} 0
          \end{align}
          Using the Fourier integrals, the general solution is,
          \begin{align}
              u(x, t) = \int_{0}^{1}\ \cos(px)\ \exp(-c^2p^2 t)\ \dl p
          \end{align}

    \item Checking the solution of Problem $ 7 $ against the initial conditions,
          \begin{align}
              u(x, 0) & = \int_{0}^{1} \cos(px)\ \dl p                             \\
                      & = \Bigg[ \frac{\sin(px)}{x} \Bigg]_0^1 = \frac{\sin(x)}{x}
          \end{align}

    \item Graphing the integrand of the error function,
          \begin{figure}[H]
              \centering
              \begin{tikzpicture}[declare function = {
                              mode(\n,\t,\x) = sin(\n *\x) * exp(-\n^2 * \t);
                          }]
                  \begin{axis}[
                          legend pos = outer north east, title ={  $ e^{-x^2} $},
                          grid = both,Ani,xlabel = $ x $,domain = -3:3,]
                      \addplot[GraphSmooth, black]{e^(-x^2)};
                  \end{axis}
              \end{tikzpicture}
          \end{figure}
          Checking if the error function is odd, using the fact that the integrand
          is even,
          \begin{align}
              \erf(-x) & = \frac{2}{\sqrt{\pi}}\ \int_{0}^{-x} e^{-w^2}\ \dl w
              = -\frac{2}{\sqrt{\pi}}\ \int_{-x}^{0} e^{-w^2}\ \dl w           \\
                       & = -\frac{2}{\sqrt{\pi}}\ \int_{0}^{x} e^{-w^2}\ \dl w
              = -\erf(x)
          \end{align}
          Proving the relations,
          \begin{align}
              \int_{a}^{b} e^{-w^2}\ \dl w & = \int_{0}^{b} e^{-w^2}\ \dl w
              - \int_{0}^{a} e^{-w^2}\ \dl w                                          \\
                                           & = \frac{\sqrt{\pi}}{2} (\erf b - \erf a)
          \end{align}
          Using $ a = -b $ in the above result, and the odd nature of $ \erf(x) $,
          \begin{align}
              \int_{-b}^{b} e^{-w^2}\ \dl w & = \frac{\sqrt{\pi}}{2}
              [\erf(b) - \erf(-b)] = \sqrt{\pi} \erf(b)
          \end{align}

    \item From the Maclaurin series of the integrand,
          \begin{align}
              e^{-w^2} & = 1 - \frac{w^2}{1!} + \frac{w^4}{2!} - \frac{w^6}{3!}
              + \dots                                                                  \\
              \int_{0}^{x} e^{-w^2}
              \ \dl w  & = x - \frac{x^3}{3} + \frac{x^5}{10} - \frac{x^7}{42} + \dots \\
              \erf(x)  & = \frac{2}{\sqrt{\pi}} \Bigg[ x - \frac{x^3}{3}
                  + \frac{x^5}{10} - \frac{x^7}{42} + \dots \Bigg]
          \end{align}

          \begin{figure}[H]
              \centering
              \pgfplotstableread[col sep=comma]{./tables/table_12_07_10.csv}
              \anitablefour
              \begin{tikzpicture}
                  \begin{axis}[
                          legend pos = south east, title ={Polynomial approximation
                                  of error function},
                          grid = both,Ani,xlabel = $ x $,]
                      \addplot[GraphSmooth, y_h, domain = 0:3, samples = 601]
                      gnuplot{erf(x)};
                      \addplot[GraphSmooth, color = y_p]
                      table[x index=0,y index=1, col sep=comma, ]{\anitablefour};
                      \addlegendentry{Exact}
                      \addlegendentry{Polynomial $ n = 20 $}
                  \end{axis}
              \end{tikzpicture}
          \end{figure}

    \item Using \texttt{sympy} as a CAS to perform the integration yields much more
          accurate results.
          \begin{align}
              \erf x & = \frac{2}{\sqrt{\pi}} \int_{0}^{x} e^{-w^2}\ \dl w
          \end{align}

          \begin{figure}[H]
              \centering
              \pgfplotstableread[col sep=comma]{./tables/table_12_07_11.csv}
              \anitablefive
              \begin{tikzpicture}
                  \begin{axis}[
                          legend pos = south east, title ={Polynomial approximation
                                  of error function},
                          grid = both,Ani,xlabel = $ x $,]
                      \addplot[GraphSmooth, y_h, domain = 0:3, samples = 601]
                      gnuplot{erf(x)};
                      \addplot[GraphSmooth, color = y_p]
                      table[x index=0,y index=1, col sep=comma, ]{\anitablefive};
                      \addlegendentry{Exact}
                      \addlegendentry{Integration using CAS}
                  \end{axis}
              \end{tikzpicture}
          \end{figure}

    \item Using \texttt{sympy} as a CAS to perform the integration for large values of
          $ x $.

          \begin{figure}[H]
              \centering
              \pgfplotstableread[col sep=comma]{./tables/table_12_07_12.csv}
              \anitablesix
              \begin{tikzpicture}
                  \begin{axis}[
                          legend pos = south east,
                          grid = both,Ani,xlabel = $ \log_{10} x $,]
                      \addplot[mark options={mark size = 1pt}, only marks,
                          color = black] table[x index=0,y index=1, col sep=comma, ]
                          {\anitablesix};
                      \addlegendentry{Integration using CAS}
                  \end{axis}
              \end{tikzpicture}
          \end{figure}

    \item Using the result from Problem $ 12 $,
          \begin{align}
              u(x, t) & = \frac{1}{2c\sqrt{\pi t}} \infint \exp\Bigg[
              -\frac{(x - w)^2}{4c^2 t} \Bigg]\ \dl w                              \\
              z       & = \frac{w - x}{2c\sqrt{t}}                                 \\
              z^-     & = \frac{-x}{2c\sqrt{t}} \qquad\qquad
              z^+ = \infty                                                         \\
              u(x, t) & = \frac{1}{\sqrt{\pi}} \int_{z^-}^{z^+} e^{-z^2}\ \dl z
              = \frac{1}{2}\ \Big[ \erf(z^+) - \erf(z^-) \Big]                     \\
                      & = \frac{1}{2} \Bigg[ 1 - \erf \Bigg( \frac{-x}{2c\sqrt{t}}
                  \Bigg) \Bigg]
          \end{align}

    \item Using the limits of integration $ z^+ $ and $ z^- $ as in the text,
          \begin{align}
              u(x, t) & = \frac{U_0}{2}\ \Big[ \erf(z^+) - \erf(z^-) \Big] \\
              z^+     & = \frac{1-x}{2c\sqrt{t}} \qquad\qquad
              z^- = \frac{-1-x}{2c\sqrt{t}}
          \end{align}

    \item Expressing the given function in terms of the error function,
          \begin{align}
              \Phi(x)           & = \frac{1}{\sqrt{2\pi}} \int_{-\infty}^{x}
              e^{-s^2/2}\ \dl s &
              w                 & = \frac{s}{\sqrt{2}} \qquad
              \dl w = \frac{\dl s}{\sqrt{2}}                                           \\
              \Phi(x)           & = \frac{1}{\sqrt{\pi}} \int_{-\infty}^{x/\sqrt{2}}
              e^{-w^2}\ \dl w   &
                                & = \frac{\erf(x/\sqrt{2}) - \erf(-\infty)}{2}         \\
                                & = \frac{\erf(x/\sqrt{2}) + \erf(\infty)}{2}        &
                                & = \frac{\erf(x/\sqrt{2}) + 1}{2}
          \end{align}
\end{enumerate}