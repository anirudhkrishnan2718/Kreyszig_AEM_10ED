\section{Laplace's Equation in Cylindrical and Spherical Coordinates}

\begin{enumerate}
    \item Deriving the Laplacian in spherical coordinates from the Laplacian in
          cylindrical coordinates,
          \begin{align}
              x & = s \cos\theta                   &
              y & = s \sin\theta                     \\
              z & = r \cos\phi                     &
              s & = r \sin \phi = \sqrt{x^2 + y^2}
          \end{align}
          Consider the $ xy $ plane with fixed $ z $ and then the $ sz $ plane with
          fixed $ \theta $.
          \begin{align}
              \nabla^2 u      & = u_{rr} + \frac{1}{r}\ u_r + \frac{1}{r^2}
              \ u_{\theta\theta} + u_{zz}                                   \\
              u_{xx} + u_{yy} & = u_{ss} + \frac{1}{s}\ u_s + \frac{1}{s^2}
              \ u_{\theta\theta}                                            \\
              u_{ss} + u_{zz} & = u_{rr} + \frac{1}{r}\ u_r + \frac{1}{r^2}
              \ u_{\phi\phi}
          \end{align}
          In the $ xy $ plane the magnitute is $ \sqrt{x^2 + y^2} = s $ and direction is
          $ \theta $. However, in the $ sz $ plane, the magnitude is now
          $ \sqrt{s^2 + z^2} = r $ and direction is $ \phi $. \par
          Summing the above terms and eliminating $ s $ in favour of the spherical
          coordinates.
          \begin{align}
              \nabla^2 u & = u_{rr} + \frac{1}{r}\ u_r + \frac{1}{r^2}\ u_{\phi\phi}
              + \frac{1}{s}\ u_s + \frac{1}{s^2}\ u_{\theta\theta}                   \\
                         & = u_{rr} + \frac{1}{r}\ u_r + \frac{1}{r^2}\ u_{\phi\phi}
              + \frac{1}{r^2\sin^2\phi}\ u_{\theta\theta}
              + \color{y_p} \frac{1}{s}\ u_s                                         \\
              u_s        & = u_r\ r_s + u_\theta\ \theta_s + u_\phi\ \phi_s          \\
                         & = u_r\ \diffp*{\ \Big[\sqrt{s^2 + z^2}\Big]}{s}
              + u_\theta\ (0) + u_\phi\ \diffp*{\ \Big[\arcsin(s/r)\Big]}{s}         \\
                         & = u_r\ \frac{s}{r} + u_\phi\ \frac{1}{\sqrt{r^2 - s^2}}
              = u_r\ \frac{s}{r} + u_\phi\ \frac{\cos\phi}{r}
          \end{align}
          Substituting into the above expression,
          \begin{align}
              \frac{1}{s}\ u_s & = \frac{1}{r}\ u_r + \frac{\cot\phi}{r^2}\ u_\phi \\
              \nabla^2u        & = u_{rr} + \frac{2}{r}\ u_r + \frac{1}{r^2}
              \ \Bigg[ u_{\phi\phi} + \frac{1}{\sin^2\phi}\ u_{\theta\theta}
                  + \cot\phi\ u_\phi \Bigg]
          \end{align}

    \item Converting the Laplacian in cylindrical coordinates back to Carteisan
          coordinates,
          \begin{align}
              r        & = \sqrt{x^2 + y^2}                     &
              \theta   & = \arctan(y/x)                           \\
              x        & = r\cos\theta                          &
              y        & = r\sin\theta                            \\
              u_r      & = u_x\ x_r + u_y\ y_r                  &
                       & = u_x \cos\theta + u_y \sin\theta        \\
              u_{r}    & = u_x\ \frac{x}{r} + u_y\ \frac{y}{r}    \\
              u_\theta & = u_x\ x_\theta + u_y\ y_\theta        &
                       & = -u_x\ r\sin\theta + u_y\ r\cos\theta   \\
              u_\theta & = -u_x\ y + u_y\ x
          \end{align}
          To find the second derivative,
          \begin{align}
              u_{rr}           & = \frac{x}{r}\ \Bigg[u_{xx}\ \frac{x}{r}
                  + u_x\ \frac{y^2}{r^3} + u_{yx}\ \frac{y}{r}
              - u_y\ \frac{xy}{r^3}\Bigg]                                      \\
                               & +  \frac{y}{r}\ \Bigg[u_{xy}\ \frac{x}{r}
                  - u_x\ \frac{xy}{r^3} + u_{yy}\ \frac{y}{r}
              + u_y\ \frac{x^2}{r^3}\Bigg]                                     \\
              u_{\theta\theta} & = (-y)[-u_{xx}\ y + u_{yx}\ x + u_y]
              + (x) [-u_{xy}\ y - u_x + u_{yy}\ x]                             \\
                               & = y^2\ u_{xx} - x\ u_x + x^2\ u_{yy} - y\ u_y
              -2xy\ u_{xy}
          \end{align}
          Consolidating the three terms,
          \begin{align}
              u_{\theta\theta} + r^2\ u_{rr} + r\ u_r & = u_{xx} \Big[ y^2
              + x^2 \Big] + u_{yy} \Big[ x^2 + y^2 \Big]                         \\
                                                      & + u_x \Big[
                  -x + x \Big] + u_y \Big[ -y + y \Big] + u_{xy} \Big[
              -2xy + 2xy\Big]                                                    \\
                                                      & = r^2\ (u_{xx} + u_{yy}) \\
              \frac{1}{r^2}\ u_{\theta\theta} +
              \frac{1}{r}\ u_r + u_{rr} + u_{zz}      &
              = u_{xx} + u_{yy} + u_{zz}
          \end{align}

    \item Plotting the first few Legendre polynomials in $ \cos\phi $,
          \begin{figure}[H]
              \centering
              \begin{tikzpicture}
                  \begin{axis}[
                          legend pos = outer north east, title = {$P_n(\cos\phi)$},
                          grid = both, domain = 0:2*pi,
                          PiStyleX, xtick distance = 0.25 * pi,Ani,
                          colormap/viridis,
                          cycle list = {[samples of colormap = 4]},
                          samples = 100,]
                      \addplot+ [thick] {leg_P_0(cos(x))};
                      \addplot+ [thick] {leg_P_1(cos(x))};
                      \addplot+ [thick] {leg_P_2(cos(x))};
                      \addplot+ [thick] {leg_P_3(cos(x))};
                      \addlegendentry{$ P_0 $};
                      \addlegendentry{$ P_1 $};
                      \addlegendentry{$ P_2 $};
                      \addlegendentry{$ P_3 $};
                  \end{axis}
              \end{tikzpicture}
          \end{figure}

    \item The zero surfaces correspond to the Legendre polynomials $ P_n(\cos\phi)
              = 0 $.
          \begin{align}
              P_1(\cos\phi) & = \cos\phi = 0                                 &
              \implies \phi & = \frac{\pi}{2}                                  \\
              P_2(\cos\phi) & = 3\cos^2\phi - 1 = 0                          &
              \implies \phi & = \frac{\pi}{6}, \frac{5\pi}{6}                  \\
              P_3(\cos\phi) & = 5\cos^3\phi - 3\cos\phi = 0                  &
              \implies \phi & = \frac{\pi}{2}, \frac{\pi}{6}, \frac{5\pi}{6}
          \end{align}
          Geometrically these are the $ xy $ plane, the double cones with angle
          $ 30^\circ $, and their union.

    \item Using \texttt{sympy} to evaluate the terms upto $ A_{10} $ and plotting
          the partial sums at $ r = R $,
          \begin{figure}[H]
              \centering
              \begin{tikzpicture}
                  \begin{axis}[
                          legend pos = outer north east, title = {$u(R=1)$},
                          grid = both, domain = 0:pi,
                          PiStyleX, xtick distance = 0.25 * pi,Ani,
                          colormap/jet,
                          cycle list = {[samples of colormap = 4]},
                          samples = 100, ytick = {0,55,110}]
                      \addplot[thick, samples = 200, black, dashed]{
                          ifthenelse(x<pi/2,110,0)};
                      \addplot+ [thick] {55 * leg_P_0(cos(x)) +
                          82.5 * leg_P_1(cos(x)) - 48.125 * leg_P_3(cos(x))};
                      \addplot+ [thick] {55 * leg_P_0(cos(x)) +
                          82.5 * leg_P_1(cos(x)) - 48.125 * leg_P_3(cos(x))
                          + 37.813 * leg_P_5(cos(x))};
                      \addplot+ [thick] {55 * leg_P_0(cos(x)) +
                          82.5 * leg_P_1(cos(x)) - 48.125 * leg_P_3(cos(x))
                          + 37.813 * leg_P_5(cos(x)) - 32.227 * leg_P_7(cos(x))};
                      \addplot+ [thick] {55 * leg_P_0(cos(x)) +
                          82.5 * leg_P_1(cos(x)) - 48.125 * leg_P_3(cos(x))
                          + 37.813 * leg_P_5(cos(x)) - 32.227 * leg_P_7(cos(x))
                          + 28.574 * leg_P_9(cos(x))};
                      \addlegendentry{$ f(R=1) $};
                      \addlegendentry{$ S_3 $};
                      \addlegendentry{$ S_5 $};
                      \addlegendentry{$ S_7 $};
                      \addlegendentry{$ S_9 $};
                  \end{axis}
              \end{tikzpicture}
          \end{figure}
          The partial sums seem to get better at approximating the boundary condition.

    \item Refer to plot in Problem $ 5 $. The oscillations move closer to the
          jump discontinuity as the number of terms in the Fourier series increases.

    \item Verifying that the solutions $ u_n $ satisfy Laplace's equation.
          \begin{align}
              u(r, \phi)             & = A_n r^n P_n(\cos\phi)        \\
              u_{r}                  & = n\ A_n r^{n-1} P_n(\cos\phi) \\
              \diffp*{\ (r^2u_r)}{r} & = \color{y_h} n(n+1)
              \ A_n r^{n}P_n(\cos\phi)
          \end{align}
          Differentiating w.r.t. $ \phi $,
          \begin{align}
              u_\phi                       & = A_nr^n\ P_n'(\cos\phi)            \\
              \frac{1}{\sin\phi}\ \diffp*
              {\ (\sin\phi\ u_\phi)}{\phi} & = \color{y_h} A_nr^n\ \Big[\cot\phi
                  \ P_n'(\cos\phi) + P_n''(\cos\phi)\Big]
          \end{align}
          Using the Legendre ODE, and abbreviating $ P_n(\cos\phi) $ to $ P $, it solves
          this ODE by definition,
          \begin{align}
              \diff[2] P\phi + \cot\phi\ \diff P\phi + n(n+1)P & = 0                  \\
              \frac{1}{r^2}\ \Bigg[ \diffp*{\ (r^2\ u_r)}{r} + \frac{1}{\sin\phi}
              \ \diffp*{\ (\sin\phi\ u_\phi)}{\phi} \Bigg]     & = \frac{A_nr^n}{r^2}
              \ (0) = 0
          \end{align}

          Verifying that the solutions $ u_n^* $ satisfy Laplace's equation.
          \begin{align}
              u^*(r, \phi)           & = B_n r^{-n-1} P_n(\cos\phi)         \\
              u^*_{r}                & = -(n+1)\ B_n r^{-n-2} P_n(\cos\phi) \\
              \diffp*{\ (r^2u_r)}{r} & = \color{y_p} n(n+1)
              \ B_n r^{-n-1}P_n(\cos\phi)
          \end{align}
          Differentiating w.r.t. $ \phi $,
          \begin{align}
              u_\phi                       & = B_nr^{-n-1}\ P_n'(\cos\phi)            \\
              \frac{1}{\sin\phi}\ \diffp*
              {\ (\sin\phi\ u_\phi)}{\phi} & = \color{y_p} B_nr^{-n-1}\ \Big[\cot\phi
                  \ P_n'(\cos\phi) + P_n''(\cos\phi)\Big]
          \end{align}
          Using the Legendre ODE, and abbreviating $ P_n(\cos\phi) $ to $ P $, it solves
          this ODE by definition,
          \begin{align}
              \diff[2] P\phi + \cot\phi\ \diff P\phi + n(n+1)P & = 0                  \\
              \frac{1}{r^2}\ \Bigg[ \diffp*{\ (r^2\ u_r)}{r} + \frac{1}{\sin\phi}
              \ \diffp*{\ (\sin\phi\ u_\phi)}{\phi} \Bigg]     & = \frac{A_nr^{-n-1}}
              {r^2}\ (0) = 0
          \end{align}

    \item Since the potential depends only on $ r $,
          \begin{align}
              u(r)       & = \frac{c}{r}                                          &
              \nabla^2 u & = u_{rr} + \frac{2}{r}\ u_r                              \\
              \nabla^2 u & = \frac{2c}{r^3} - \frac{2}{r} \cdot \frac{c}{r^2} = 0
          \end{align}

    \item Since the potential depends only on $ r $,
          \begin{align}
              \nabla^2 u    & = u_{rr} + \frac{2}{r}\ u_r                  &
              \nabla^2 u    & = \frac{1}{r^2}\ \diffp*{\ (r^2 u_r)}{r} = 0   \\
              r^2 \diffp ur & = c_1                                        &
              \diffp ur     & = \frac{k_1}{r^2}                              \\
              u             & = \color{y_p}\frac{k_1}{r} + k_2
          \end{align}
          This solution is unique due to the uniqueness of general solutions of ODEs.

    \item Since the potential depends only on $ r $,
          \begin{align}
              \nabla^2 u           & = u_{rr} + \frac{1}{r}\ u_r = 0 &
              r^2\ u_{rr} + r\ u_r & = 0                               \\
              \lambda^2            & = 0                             &
              \lambda              & = \{0, 0\}                        \\
              u                    & = \color{y_p}k_1 + k_2\ln(r)
          \end{align}
          This solution is unique due to the uniqueness of general solutions of ODEs.

    \item Substituting into the Laplace equation in Cartesian coordinates,
          \begin{align}
              u          & = \frac{c_1}{\sqrt{x^2 + y^2 + z^2}} + c_2                &
              \nabla^2 u & = u_{xx} + u_{yy} + u_{zz}                                  \\
              u_{x}      & = \frac{-c_1x}{(x^2 + y^2 + z^2)^{3/2}}                   &
              u_{xx}     & = -c_1\ \frac{r^3 - 3x^2r}{r^6}                             \\
              \nabla^2 u & =\frac{-c_1}{r^6}\ \Big[ 3r^3 - 3r(x^2 + y^2 + z^2) \Big] &
              \nabla^2 u & = 0
          \end{align}
          Substituting into the Laplace equation in Spherical coordinates,
          \begin{align}
              u                   & = u'' + \frac{2u'}{r} &
              u'                  & = \frac{-c_1}{r^2}      \\
              u''                 & = \frac{2c_1}{r^3}    &
              u'' + \frac{2u'}{r} & = 0
          \end{align}

    \item From Problem $ 10 $, the potential depending only on $ r $ is,
          \begin{align}
              u    & = c_1 \ln(r) + c_2                               \\
              u(2) & = c_1\ln(2) + c_2 = 220                        &
              u(4) & = c_1\ln(4) + c_2 = 140                          \\
              c_1  & = \frac{-80}{\ln(2)}                           &
              c_2  & = 300                                            \\
              u    & = \color{y_p} -\frac{80}{\ln(2)}\ \ln(r) + 300
          \end{align}

    \item From Problem $ 9 $, the potential depending only on $ r $ is,
          \begin{align}
              u    & = \frac{c_1}{r} + c_2              \\
              u(2) & = \frac{c_1}{2} + c_2 = 220      &
              u(4) & = \frac{c_1}{4} + c_2 = 140        \\
              c_1  & = 320                            &
              c_2  & = 60                               \\
              u    & = \color{y_h} \frac{320}{r} + 60
          \end{align}
          Plotting the two potentials in the domain $ r \in [2, 4] $, the cylindrical
          potential is larger than the spherical potential
          \begin{figure}[H]
              \centering
              \begin{tikzpicture}
                  \begin{axis}[
                          legend pos = north east,
                          grid = both, Ani,
                          domain = 2:4,
                      ]
                      \addplot[GraphSmooth,y_h]{(320/x) + 60};
                      \addplot[GraphSmooth, y_p]{-ln(x) * 80/(ln(2)) + 300};
                      \addlegendentry{spherical};
                      \addlegendentry{cylindrical};
                  \end{axis}
              \end{tikzpicture}
          \end{figure}
          The equipotential lines are cylinders and spheres respectively. These would
          look like circular cross sections in the $ xy $ plane. The above plot is more
          useful in comparing the two potentials.

    \item Using the Laplacian for spherical coordinates when the potential is dependent
          only on $ r $,
          \begin{align}
              \nabla^2 u                & = u_{rr} + \frac{2}{r}\ u_r &
              \color{y_h}c^2 \Bigg[ u_{rr}
              + \frac{2}{r}\ u_r \Bigg] & = \color{y_h}u_t
          \end{align}
          The initial condition is $ u(r, 0) = f(r) $ and the boundary condition is
          $ u(R, t) = 0 $. \par
          Introducing the new variable $ v = ur $,
          \begin{align}
              v_t            & = r\ u_t                             &
              v_r            & = r\ u_r + u                           \\
              v_{rr}         & = r\ u_{rr} + 2u_r                   &
              \frac{v_t}{r}  & = c^2 \Bigg[ \frac{v_{rr}}{r} \Bigg]   \\
              \color{y_p}v_t & = \color{y_p}c^2\ v_{rr}
          \end{align}
          The constraint on $ v $ in order for the temperature to be bounded at $ r = 0 $
          is that $ v(0, t) = 0 $. \par
          The new I.C. $ v(r, 0) = rf(r) $ and the new B.C. is $ v(R, t) = 0 $.
          \begin{align}
              v(r, t)                  & = A(r) \cdot B(t)                 &
              A \cdot \dot{B}          & = c^2\ B \cdot A''                  \\
              \frac{1}{c^2B}\ \diff Bt & = \frac{1}{A}\ \diff[2] Ar = -k^2   \\
          \end{align}
          Solving the two ODEs in $ t, r $ separately,
          \begin{align}
              \diff Bt    & = -(ck)^2\ B                  &
              B           & = p_1 \exp(-\lambda^2 t)        \\
              \diff[2] Ar & = -k^2A                       &
              A           & = q_1 \cos(kr) + q_2 \sin(kr)
          \end{align}
          Applying the boundary conditions on $ A(r) $,
          \begin{align}
              A(0)              & = 0                           &
              \implies q_1      & = 0                             \\
              A(R)              & = 0                           &
              \implies \sin(kR) & = 0                             \\
              k                 & = \color{y_h} \frac{n\pi}{R}  &
              \lambda           & = \color{y_h} \frac{cn\pi}{R}
          \end{align}
          Consolidating the two functions into $ v(r, t) $, and combining the two
          constants $ p, q $
          \begin{align}
              v_n(r, t) & = \color{y_p} d_n\ \sin\Big( \frac{n\pi r}{R}\Big) \cdot
              \exp(-\lambda_n^2 t)                                                 \\
              v(r, 0)   & = \iser[n]{1} v_n(r, 0) = \iser[n]{1} d_n \sin\Big(
              \frac{n\pi r}{R} \Big) = rf(r)
          \end{align}
          Since this is simply the Fourier sine expansion of $ f(r) $,
          \begin{align}
              d_n & = \color{y_p} \frac{2}{R} \int_{0}^{R} rf(r)
              \ \sin \Big( \frac{n\pi r}{R} \Big)\ \dl r
          \end{align}

    \item The analog of Problem $12$ is hot and cold concentric cylinders. For
          Problem $ 13 $, it is concentric hot and cold spheres.

    \item Decomposing the boundary function into a linear superposition of
          Legendre polynomials in $ \cos\phi $,
          \begin{align}
              f(\phi) & = \cos\phi = P_1(\cos\phi)       &
              u(R=1)  & = \iser[n]{0} A_n\ P_n(\cos\phi)         \\
              A_n     & = \color{y_h} \begin{cases}
                                          1 & \quad n = 1            \\
                                          0 & \quad \text{otherwise}
                                      \end{cases}    &
              u_r     & = \color{y_p} r\ \cos\phi
          \end{align}

    \item Decomposing the boundary function into a linear superposition of
          Legendre polynomials in $ \cos\phi $,
          \begin{align}
              f(\phi) & = 1 = P_0(\cos\phi)              &
              u(R=1)  & = \iser[n]{0} A_n\ P_n(\cos\phi)         \\
              A_n     & = \color{y_h} \begin{cases}
                                          1 & \quad n = 0            \\
                                          0 & \quad \text{otherwise}
                                      \end{cases}    &
              u_r     & = \color{y_p} 1
          \end{align}

    \item Decomposing the boundary function into a linear superposition of
          Legendre polynomials in $ \cos\phi $,
          \begin{align}
              f(\phi) & = 1 - \cos^2\phi = \frac{-2P_2 + 2P_0}{3}                    &
              u(R=1)  & = \iser[n]{0} A_n\ P_n(\cos\phi)                               \\
              A_n     & = \color{y_h} \begin{cases}
                                          2/3  & \quad n = 0            \\
                                          -2/3 & \quad n = 2            \\
                                          0    & \quad \text{otherwise}
                                      \end{cases}                             &
              u_r     & = \color{y_p} \frac{2}{3} - \frac{(3\cos^2\phi - 1)}{3}\ r^2
          \end{align}

    \item Decomposing the boundary function into a linear superposition of
          Legendre polynomials in $ \cos\phi $,
          \begin{align}
              f(\phi) & = \cos(2\phi) = \frac{4P_2 - P_0}{3} &
              u(R=1)  & = \iser[n]{0} A_n\ P_n(\cos\phi)            \\
              A_n     & = \color{y_h} \begin{cases}
                                          -1/3 & \quad n = 0            \\
                                          4/3  & \quad n = 2            \\
                                          0    & \quad \text{otherwise}
                                      \end{cases}     &
              u_r     & = \color{y_p} -\frac{1}{3}
              + \frac{(6\cos^2\phi - 2)}{3}\ r^2
          \end{align}

    \item Decomposing the boundary function into a linear superposition of
          Legendre polynomials in $ \cos\phi $,
          \begin{align}
              f(\phi) & = \cos(2\phi) = 4P_3 - 2P_2 + P_1 - 2P_0                   \\
              u(R=1)  & = \iser[n]{0} A_n\ P_n(\cos\phi)                           \\
              A_n     & = \color{y_h} \begin{cases}
                                          -1 & \quad n = 0            \\
                                          2  & \quad n = 1            \\
                                          -2 & \quad n = 2            \\
                                          4  & \quad n = 3            \\
                                          0  & \quad \text{otherwise}
                                      \end{cases}                  \\
              u_r     & = \color{y_p} -1 + 2r\ P_1(\cos\phi) - 2r^2\ P_2(\cos\phi)
              + 4r^3\ P_3(\cos\phi)
          \end{align}

    \item From the boundary condition in Problem $ 17 $, the potential outside the
          sphere is,
          \begin{align}
              u(r, \phi) & = \iser[n]{0} \frac{B_n}{r^{n+1}}\ P_n(\cos\phi) &
              f(\phi)    & = P_0 = \iser[n]{0} B_n\ P_n(\cos\phi)             \\
              u(r, \phi) & = \color{y_h} \frac{1}{r}
          \end{align}
          This happens to be the same as the potential due to a point charge at the
          origin.

    \item  Using the same Fourier-Legendre coefficients, in Problem $ 16 $,
          \begin{align}
              u^*(r, \phi) & = \frac{1}{r^2}\ P_1(\cos\phi)
          \end{align}
          Using the same Fourier-Legendre coefficients, in Problem $ 19 $,
          \begin{align}
              u^*(r, \phi) & = -\frac{1}{3}\ \frac{P_0(\cos\phi)}{r} + \frac{4}{3}
              \ \frac{P_2(\cos\phi)}{r^3}
          \end{align}

    \item Plotting the $ xz $ plane cross sections of equipotential surfaces,
          \begin{align}
              u & = r\ \cos\phi & r & = \frac{k}{\cos\phi}
          \end{align}
          \begin{figure}[H]
              \centering
              \begin{tikzpicture}
                  \begin{polaraxis}[rotate = -90, x dir = reverse,
                          enlargelimits = false, ytick = {0, 0.5, 1},
                          width = 10cm, height = 10cm, xmin = 0, xmax = 180,
                          xticklabel style={color = blue, circle, anchor = east},
                          yticklabel style={anchor = north west},
                          grid = both, xtick = {0, 45, 90, 135, 180},
                          xticklabels={0, $ \pi/4 $, $ \pi/2 $, $ 3\pi/4 $, $ \pi $},
                          trig format plots = deg]
                      \addplot [thick, y_h, dashed,
                          domain = 0:180, restrict y to domain = 0:1]{0.25/cos(90-x)};
                      \addplot [thick, y_p, dashed,
                          domain = 0:180, restrict y to domain = 0:1]{0.5/cos(90-x)};
                      \addplot [thick, y_t, dashed,
                          domain = 0:180, restrict y to domain = 0:1]{0.75/cos(90-x)};
                  \end{polaraxis}
              \end{tikzpicture}
          \end{figure}
          These are planes perpendicular to the $x$ axis, which show up as lines parallel
          to the $ z $ axis in the plot.

    \item Transmission line,
          \begin{enumerate}
              \item Consider a small segment of cable at position $ x $ and width
                    $ \Delta x $,
                    \begin{align}
                        u_B - u_A         & = Rj + L\ \diffp jt &
                        Rj + L\ \diffp jt & = -\diffp ux
                    \end{align}
                    Here the PDE comes from the infinitesimal limit of the wire segment.
                    The negative sign comes from the fact that potential is decreasing
                    left to right.

              \item Using Kirchoff's current law, on a small segment of wire,
                    \begin{align}
                        j_B - j_A  & = \Delta j_C + \Delta j_G &
                        -\diffp jx & = C\ \diffp ut + Gu
                    \end{align}
                    Here, $ G $ is the reciprocal of the resistance from the wire to the
                    ground, which is different from $ R $, the resistance per unit length
                    of the wire itself.

              \item Eliminating current $ j $,
                    \begin{align}
                        -j_x              & = C\ u_t + Gu                        &
                        -j_{xt}           & = C\ u_{tt} + G\ u_t                   \\
                        j_{t}             & = \frac{-1}{L}\ u_x - \frac{R}{L}\ j &
                        j_{tx}            & = \frac{-1}{L}\ u_{xx} - \frac{R}{L}
                        \ j_x                                                      \\
                        \color{y_h}u_{xx} & = \color{y_h}LC\ u_{tt} + (RC + LG)
                        \ u_t + RG\ u
                    \end{align}
                    Eliminating potential $ u $,
                    \begin{align}
                        -u_x               & = Rj + L\ j_t                        &
                        -u_{xt}            & = R\ j_{t} + L\ j_{tt}                 \\
                        u_{t}              & = \frac{-1}{C}\ j_x - \frac{G}{C}\ u &
                        u_{tx}             & = \frac{-1}{C}\ j_{xx} - \frac{G}{C}
                        \ u_x                                                       \\
                        \color{y_p} j_{xx} & = \color{y_p} LC\ j_{tt} + (RC + LG)
                        \ j_t + RG\ j
                    \end{align}

              \item For the special case where $ G \to 0 $ and frequency
                    is negligible, leading to $ L \to 0 $,
                    \begin{align}
                        u_{xx} & = RC\ u_t &
                        j_{xx} & = RC\ j_t
                    \end{align}
                    Since this is the heat equation with B.C. $ u(0, t)
                        = u(L, t) = 0$. \par
                    Further, the I.C is $ u(x, 0) = U_0 $.
                    Decomposing the function $ u(x, t) $ into $ V(x) \cdot W(t)$,
                    \begin{align}
                        c^2     & = \frac{1}{RC}                                     \\
                        V(x)    & = \color{y_p}\sin\Big( \frac{n\pi x}{L} \Big)      \\
                        W(t)    & = \color{y_p}\exp\Bigg[ -\left( \frac{n\pi}{L}
                        \right)^2 \cdot \frac{t}{RC} \Bigg]                          \\
                        u(x, 0) & = \iser[n]{1} A_n \sin\Big( \frac{n\pi x}{L} \Big) \\
                        A_n     & = \frac{2}{L}\int_{0}^{L} f(x)\
                        \sin\Big( \frac{n\pi x}{L} \Big)\ \dl x
                    \end{align}
                    Applying the given boundary condition to the general solution above,
                    \begin{align}
                        A_n     & = \frac{2U_0}{L}\int_{0}^{L}\
                        \sin\Big( \frac{n\pi x}{L} \Big)\ \dl x           \\
                                & = \frac{2U_0}{n\pi}\ \Bigg[ \cos\Big(
                            \frac{n\pi x}{L} \Big)\Bigg]_L^0
                        = \color{y_p} \frac{2U_0}{n\pi}\ [1 - \cos(n\pi)] \\
                        u(x, t) & = \iser[n]{1} A_n\ V_n(x)\ W_n(t)
                    \end{align}

              \item For the special case where $ L \gg R $ and frequency
                    is negligible, leading to $ C \gg G $,
                    \begin{align}
                        u_{xx} & = LC\ u_{tt} &
                        j_{xx} & = LC\ j_{tt}
                    \end{align}
                    Since this is the wave equation with B.C. $ u(0, t)
                        = u(L, t) = 0$ and \par
                    Further, the I.C is $ u(x, 0) = U_0 \sin(\pi x/l) $ and
                    $ u_t(x, 0) = 0 $.
                    Decomposing the function $ u(x, t) $ into $ V(x) \cdot W(t)$,
                    \begin{align}
                        c^2     & = \frac{1}{LC}, \qquad \lambda_n = \frac{cn\pi}{l} &
                        V(x)    & = \color{y_p}\sin\Big( \frac{n\pi x}{L} \Big)        \\
                        W(t)    & = \color{y_p}\cos(\lambda_n t)                     &
                        u(x, 0) & = \iser[n]{1} A_n \sin\Big( \frac{n\pi x}{L} \Big)
                    \end{align}
                    Applying the given boundary condition to the general solution above,
                    \begin{align}
                        A_n     & = \color{y_p} \begin{cases}
                                                    U_0 & \quad n = 1            \\
                                                    0   & \quad \text{otherwise}
                                                \end{cases}                     &
                        u(x, t) & = U_0\ \cos(\lambda_1 t)\ \sin\Big( \frac{\pi x}{l}
                        \Big)
                    \end{align}
          \end{enumerate}

    \item Transforming $ u \to v $, to reflect on the unit sphere,
          \begin{align}
              W(r, \theta, \phi)         & = \frac{U(1/r, \theta, \phi)}{r} &
              r \cdot W(r, \theta, \phi) & =  U(\rho, \theta, \phi)           \\
              U_\theta                   & = r \cdot W_\theta               &
              U_{\theta\theta}           & = r \cdot W_{\theta\theta}         \\
              U_\phi                     & = r \cdot W_\phi                 &
              U_{\phi\phi}               & = r \cdot W_{\phi\phi}
          \end{align}
          Relating $ U_\rho $ to $ W_r $,
          \begin{align}
              U_\rho                                & = (W + rW_r) \cdot \diffp r\rho
              = \Bigg( \frac{-1}{\rho^2} \Bigg) \ [W + rW_r]                           \\
              U_{\rho\rho}                          & = \Bigg( \frac{1}{\rho^4} \Bigg)
              \ [rW_{rr} + 2W_r] + \Bigg( \frac{2}{\rho^3} \Bigg) [W + rW_r]           \\
              U_{\rho\rho} + \frac{2}{\rho}\ U_\rho & = r^5 \Bigg[ W_{rr}
              + \frac{2}{r}\ W_r \Bigg]                                                \\
              \frac{1}{\rho^2}\ U_{\phi\phi}
              + \frac{\cot\phi}{\rho^2}\ U_\phi     & = r^3\ W_{\phi\phi}
              + r^3\ \cot\phi\ W_\phi                                                  \\
              \frac{1}{\rho^2 \sin^2\phi}
              \ U_{\theta\theta}                    & = \frac{r^3}{\sin^2\phi}
              \ W_{\theta\theta}                                                       \\
              \nabla^2 U(\rho, \theta, \phi)        & = r^5\ [\nabla^2 W(r, \theta
              , \phi)]
          \end{align}
          Since $ U $ satisfies Laplace's equation, so does $ W $.
\end{enumerate}