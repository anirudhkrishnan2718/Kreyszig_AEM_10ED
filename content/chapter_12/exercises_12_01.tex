\section{Basic Concepts of PDEs}

\begin{enumerate}
    \item Writing a general second order ODE in two variables $ u(x, y) $,
          \begin{align}
              \mathcal{P}[u] & = A\ \diffp[2] ux + B\ \diffp[2] uy + C\ \diffp {u}{x,y}
              + D\ \diffp ux + E\ \diffp uy + F\ u = 0
          \end{align}
          Using the linearity of partial differentiation,
          \begin{align}
              \mathcal{L}[u_1]        & = 0                            &
              \mathcal{L}[u_2]        & = 0                              \\
              \implies \mathcal{L}[u] & = \mathcal{L}[c_1u_1 + c_2u_2] &
              c_1 \mathcal{L}[u_1]
              + c_2 \mathcal{L}[u_2]  & = 0
          \end{align}
          A similar proof follows for $ v(x, y, z) $ which whose differential operator
          will contain more terms, but still be linear. So, the proof remains the same.

    \item Verifying that the given function solves the one dimensional wave equation,
          \begin{align}
              u(x,t)       & = x^2 + t^2         &
              \diffp[2] ut & = c^2\ \diffp[2] ux   \\
              2            & = c^2\ (2)          &
              c^2          & = 1
          \end{align}
          \begin{figure}[H]
              \centering
              \begin{tikzpicture}
                  \begin{axis}[
                          title = {$ u = x^2 + t^2 $},
                          width = 8cm, height = 8cm,
                          grid = none,
                          ticks = none,
                          view={30}{30},
                          xlabel=$x$,ylabel=$t$,zlabel=$u$,
                          enlargelimits=0.1, colormap/viridis,
                          Ani]
                      \addplot3 [surf, opacity= 0.75,
                          domain = -2:2, faceted color = black] {x^2 + y^2};
                  \end{axis}
              \end{tikzpicture}
          \end{figure}

    \item Verifying that the given function solves the one dimensional wave equation,
          \begin{align}
              u(x,t)                 & = \cos(4t)\ \sin(2x)          &
              \diffp[2] ut           & = c^2\ \diffp[2] ux             \\
              \sin(2x) [-16\cos(4t)] & = c^2\ \cos(4t)\ [-4\sin(2x)] &
              c^2                    & = 4
          \end{align}
          \begin{figure}[H]
              \centering
              \begin{tikzpicture}
                  \begin{axis}[
                          title = {$ u = \cos(4t)\ \sin(2x) $},
                          width = 8cm, height = 8cm,
                          grid = none,
                          ticks = none,
                          view={30}{30},
                          xlabel=$x$,ylabel=$t$,zlabel=$u$,
                          enlargelimits=0.1, colormap/viridis,
                          Ani]
                      \addplot3 [surf, opacity= 0.75,
                          domain = -0.5*pi:0.5*pi, faceted color = black]
                      {cos(4*y) * sin(2*x)};
                  \end{axis}
              \end{tikzpicture}
          \end{figure}

    \item Verifying that the given function solves the one dimensional wave equation,
          \begin{align}
              u(x,t)                      & = \sin(kct)\ \cos(kx)            &
              \diffp[2] ut                & = c^2\ \diffp[2] ux                \\
              \cos(kx) [-k^2c^2\sin(kct)] & = c^2\ \sin(kct)\ [-k^2\cos(kx)] &
              c                           & = \text{free}
          \end{align}
          \begin{figure}[H]
              \centering
              \begin{tikzpicture}
                  \begin{axis}[
                          title = {$ u = \sin(t)\ \cos(x) $},
                          width = 8cm, height = 8cm,
                          grid = none,
                          ticks = none,
                          view={30}{30},
                          xlabel=$x$,ylabel=$t$,zlabel=$u$,
                          enlargelimits=0.1, colormap/viridis,
                          Ani]
                      \addplot3 [surf, opacity= 0.75,
                          domain = -pi:pi, faceted color = black]
                      {sin(y) * cos(x)};
                  \end{axis}
              \end{tikzpicture}
          \end{figure}

    \item Verifying that the given function solves the one dimensional wave equation,
          \begin{align}
              u(x,t)                           & = \sin(a t)\ \sin(bx) &
              \diffp[2] ut                     & = c^2\ \diffp[2] ux     \\
              \sin(bx) [-a^2 \sin(a t)]        &
              = c^2\ \sin(a t)\ [-b^2\sin(bx)] &
              c                                & = \frac{a^2}{b^2}
          \end{align}
          \begin{figure}[H]
              \centering
              \begin{tikzpicture}
                  \begin{axis}[
                          title = {$ u = \sin(t)\ \cos(x) $},
                          width = 8cm, height = 8cm,
                          grid = none,
                          ticks = none,
                          view={30}{30},
                          xlabel=$x$,ylabel=$t$,zlabel=$u$,
                          enlargelimits=0.1, colormap/viridis,
                          Ani]
                      \addplot3 [surf, opacity= 0.75,
                          domain = -pi:pi, faceted color = black]
                      {sin(y) * cos(x)};
                  \end{axis}
              \end{tikzpicture}
          \end{figure}

    \item Verifying that the given function solves the one dimensional heat equation,
          \begin{align}
              u(x,t)             & = e^{-t}\ \sin(x)         &
              \diffp ut          & = c^2\ \diffp[2] ux         \\
              [-e^{-t}]\ \sin(x) & = c^2\ e^{-t}\ [-\sin(x)] &
              c^2                & = 1
          \end{align}
          \begin{figure}[H]
              \centering
              \begin{tikzpicture}
                  \begin{axis}[
                          title = {$ u = e^{-t}\ \sin x $},
                          width = 8cm, height = 8cm,
                          grid = none,
                          ticks = none,
                          view={30}{30},
                          xlabel=$x$,ylabel=$t$,zlabel=$u$,
                          enlargelimits=0.1, colormap/jet,
                          Ani]
                      \addplot3 [surf, opacity= 0.75,
                          domain = -0.5*pi:2*pi, y domain = -1:1,
                          faceted color = black]
                      {e^(-y) * sin(x)};
                  \end{axis}
              \end{tikzpicture}
          \end{figure}

    \item Verifying that the given function solves the one dimensional heat equation,
          \begin{align}
              u(x,t)                        & = e^{-\omega^2c^2 t}\ \cos(\omega x) &
              \diffp ut                     & = c^2\ \diffp[2] ux                    \\
              [-\omega^2c^2\ e^{-\omega^2c^2 t}]
              \ \cos(\omega x)              & = c^2\ e^{-\omega^2c^2 t}
              \ [-\omega^2\ \cos(\omega x)] &
              c                             & = \text{free}
          \end{align}
          \begin{figure}[H]
              \centering
              \begin{tikzpicture}
                  \begin{axis}[
                          title = {$ u = e^{-\omega^2 c^2 t}\ \cos(\omega x) $},
                          width = 8cm, height = 8cm,
                          grid = none,
                          ticks = none,
                          view={30}{30},
                          xlabel=$x$,ylabel=$t$,zlabel=$u$,
                          enlargelimits=0.1, colormap/jet,
                          Ani]
                      \addplot3 [surf, opacity= 0.75,
                          domain = -0.5*pi:2*pi, y domain = -1:1,
                          faceted color = black]
                      {e^(-y) * cos(x)};
                  \end{axis}
              \end{tikzpicture}
          \end{figure}

    \item Verifying that the given function solves the one dimensional heat equation,
          \begin{align}
              u(x,t)           & = e^{-9 t}\ \sin(\omega x)                   &
              \diffp ut        & = c^2\ \diffp[2] ux                            \\
              [-9\ e^{-9 t}]
              \ \sin(\omega x) & = c^2\ e^{-9 t}\ [-\omega^2\ \sin(\omega x)] &
              c^2              & = \frac{9}{\omega^2}
          \end{align}
          \begin{figure}[H]
              \centering
              \begin{tikzpicture}
                  \begin{axis}[
                          title = {$ u = e^{-9t}\ \sin(\omega x) $},
                          width = 8cm, height = 8cm,
                          grid = none,
                          ticks = none,
                          view={30}{30},
                          xlabel=$x$,ylabel=$t$,zlabel=$u$,
                          enlargelimits=0.1, colormap/jet,
                          Ani]
                      \addplot3 [surf, opacity= 0.75,
                          domain = -0.5*pi:2*pi, y domain = -1/9:1/9,
                          faceted color = black]
                      {e^(-9*y) * sin(x)};
                  \end{axis}
              \end{tikzpicture}
          \end{figure}

    \item Verifying that the given function solves the one dimensional heat equation,
          \begin{align}
              u(x,t)      & = e^{-\pi^2 t}\ \cos(25x)                &
              \diffp ut   & = c^2\ \diffp[2] ux                        \\
              [-\pi^2\ e^{-\pi^2 t}]
              \ \cos(25x) & = c^2\ e^{-\pi^2 t}\ [-25^2\ \cos(25 x)] &
              c^2         & = \frac{\pi^2}{25^2}
          \end{align}
          \begin{figure}[H]
              \centering
              \begin{tikzpicture}
                  \begin{axis}[
                          title = {$ u = e^{-\pi^2 t}\ \cos(25 x) $},
                          width = 8cm, height = 8cm,
                          grid = none,
                          ticks = none,
                          view={30}{30},
                          xlabel=$x$,ylabel=$t$,zlabel=$u$,
                          enlargelimits=0.1, colormap/jet,
                          Ani]
                      \addplot3 [surf, opacity= 0.75,
                          domain = -0.08*pi:0.08*pi, y domain = -0.5/pi:0.5/pi,
                          faceted color = black]
                      {e^(-pi*pi*y) * cos(25*x)};
                  \end{axis}
              \end{tikzpicture}
          \end{figure}

    \item Verifying that the given function solves the two dimensional Laplace equation,
          \begin{align}
              u(x,t) & = e^{x}\ \cos(y)                 &
              0      & = \diffp[2] ux + \diffp[2] uy      \\
              0      & = [e^x] \cos(y) + e^x [-\cos(y)]
          \end{align}
          \begin{figure}[H]
              \centering
              \begin{tikzpicture}
                  \begin{axis}[
                          title = {$ u = e^{x}\ \cos(y) $},
                          width = 8cm, height = 8cm,
                          grid = none,
                          ticks = none,
                          view={30}{30},
                          xlabel=$x$,ylabel=$y$,zlabel=$u$,
                          enlargelimits=0.1, colormap/jet,
                          Ani]
                      \addplot3 [surf, opacity= 0.75,
                          domain = -1:1, y domain = -2*pi:2*pi,
                          faceted color = black]
                      {e^(x) * cos(y)};
                  \end{axis}
              \end{tikzpicture}
          \end{figure}

    \item Verifying that the given function solves the two dimensional Laplace equation,
          \begin{align}
              u(x,t)       & = \arctan(y/x)                 &
              0            & = \diffp[2] ux + \diffp[2] uy    \\
              \diffp[2] ux & = \frac{y(2x)}{(x^2 + y^2)^2}  &
              \diffp[2] uy & = \frac{-x(2y)}{(x^2 + y^2)^2}
          \end{align}
          \begin{figure}[H]
              \centering
              \begin{tikzpicture}
                  \begin{axis}[
                          title = {$ u = \arctan(y/x) $},
                          width = 8cm, height = 8cm,
                          grid = none,
                          ticks = none,
                          view={30}{30},
                          xlabel=$x$,ylabel=$y$,zlabel=$u$,
                          enlargelimits=0.1, colormap/jet,
                          Ani]
                      \addplot3 [surf, opacity= 0.75,
                          domain = -10:-0.1, y domain = -pi/2:pi/2,
                          faceted color = black] {atan(y/x)};
                      \addplot3 [surf, opacity= 0.75,
                          domain = 0.1:10, y domain = -pi/2:pi/2,
                          faceted color = black] {atan(y/x)};
                  \end{axis}
              \end{tikzpicture}
          \end{figure}

    \item Verifying that the given function solves the two dimensional Laplace equation,
          \begin{align}
              u(x,t)       & = \cos(y)\ \sinh(x)           &
              0            & = \diffp[2] ux + \diffp[2] uy   \\
              \diffp[2] ux & = \cos(y)\ [\sinh(x)]         &
              \diffp[2] uy & = [-\cos(y)]\ \sinh(x)
          \end{align}
          \begin{figure}[H]
              \centering
              \begin{tikzpicture}
                  \begin{axis}[
                          title = {$ u = \cos(y)\ \sinh(x) $},
                          width = 8cm, height = 8cm,
                          grid = none,
                          ticks = none,
                          view={30}{30},
                          xlabel=$x$,ylabel=$y$,zlabel=$u$,
                          enlargelimits=0.1, colormap/jet,
                          Ani]
                      \addplot3 [surf, opacity= 0.75,
                          domain = -2:2, y domain = -2*pi:2*pi,
                          faceted color = black] {cos(y) * sinh(x)};
                  \end{axis}
              \end{tikzpicture}
          \end{figure}

    \item Verifying that the given function solves the two dimensional Laplace equation,
          \begin{align}
              u(x,t)       & = \frac{x}{x^2 + y^2}                    &
              0            & = \diffp[2] ux + \diffp[2] uy              \\
              \diffp[2] ux & = \frac{2x\ (x^2 - 3y^2)}{(x^2 + y^2)^3} &
              \diffp[2] uy & = \frac{2x\ (3y^2 - x^2)}{(x^2 + y^2)^3}
          \end{align}
          \begin{figure}[H]
              \centering
              \begin{tikzpicture}
                  \begin{axis}[
                          title = {$ u = \frac{x}{x^2 + y^2} $},
                          width = 8cm, height = 8cm,
                          grid = none,
                          ticks = none,
                          view={30}{30},
                          xlabel=$x$,ylabel=$y$,zlabel=$u$,
                          enlargelimits=0.1, colormap/jet,
                          Ani]
                      \addplot3 [surf, opacity= 0.75,
                          domain = -2:2, y domain = -2:2, restrict z to domain = -20:20,
                          faceted color = black] {x / (x^2 + y^2)};
                  \end{axis}
              \end{tikzpicture}
          \end{figure}

    \item Verifying special forms of solutions,
          \begin{enumerate}
              \item For the wave equation,
                    \begin{align}
                        u(x, t)                     & = v(x + ct) + w(x - ct) &
                        \diffp[2] ut                & = c^2\ \diffp[2] ux       \\
                        \diffp[2] vt\ c^2           & = \diffp[2] vx          &
                        \diffp[2] wt\ c^2           & = \diffp[2] wx            \\
                        c^2\ \diffp[2] {(v + w)}{t} & = \diffp[2]{(v+w)}{x}
                    \end{align}
                    This uses the transformation of variables,
                    \begin{align}
                        \diffp {v}{(ct)} \cdot \diffp {(ct)}{t} = \diffp {v}{t}
                    \end{align}
                    This function does satisfy the given PDE.

              \item Verifying the given functions against Laplace's equation, \par
                    \textcolor{y_h}{Yes}, \textcolor{y_p}{No}, \textcolor{y_h}{Yes},

                    \begin{align}
                        u(x, y)      & = \frac{y}{x}                &
                        f            & = \color{y_h} \frac{2y}{x^3}   \\
                        \diffp[2] ux & = \color{y_p} \frac{2y}{x^3} &
                        \diffp[2] uy & = \color{y_p} 0
                    \end{align}
                    \begin{align}
                        u(x, y)      & = \sin(xy)                          &
                        f            & = \color{y_h} (x^2 + y^2)\ \sin(xy)   \\
                        \diffp[2] ux & = \color{y_p} -y^2\ \sin(xy)        &
                        \diffp[2] uy & = \color{y_p} -x^2\ \sin(xy)
                    \end{align}
                    \begin{align}
                        u(x, y)      & = e^{x^2 - y^2}                           &
                        f            & = \color{y_h} 4(x^2 + y^2)\ e^{x^2 - y^2}   \\
                        \diffp[2] ux & = \color{y_p} (2 + 4x^2)\ e^{x^2 - y^2}   &
                        \diffp[2] uy & = \color{y_p} (-2 + 4y^2)\ e^{x^2 - y^2}
                    \end{align}
                    \begin{align}
                        u(x, y)        & = \frac{1}{\sqrt{x^2 + y^2}}              &
                        f              & = \color{y_h} (x^2 + y^2)^{-3/2}            \\
                        \diffp ux      & = \frac{-x}{(x^2 + y^2)^{3/2}}            &
                        \diffp uy      & = \frac{-y}{(x^2 + y^2)^{3/2}}              \\
                        \diffp[2] ux   & = \frac{-r^3 + 3x^2 r}{r^6}               &
                        \diffp[2] uy   & = \frac{-r^3 + 3y^2 r}{r^6}                 \\
                        \diffp[2] ux
                        + \diffp[2] uy & = \frac{1}{r^3}                           &
                                       & = \color{y_p} \frac{1}{(x^2 + y^2)^{3/2}}
                    \end{align}

              \item $3d$ Laplace equation satisfied the given function,
                    \textcolor{y_h}{Yes}
                    \begin{align}
                        u              & = \frac{1}{r}                              &
                        r^2            & = x^2 + y^2 + z^2                            \\
                        \difcp ux      & = \frac{-x}{r^3}                           &
                        \difcp[2] ux   & = \frac{-r^3 + (x)\ (3xr)}{r^6}              \\
                        \difcp[2] ux + \difcp[2] uy
                        + \difcp[2] uz & = \frac{-3r^3 + 3r (x^2 + y^2 + z^2)}{r^6} &
                                       & = 0
                    \end{align}
                    $2d$ Laplace equation satisfied by the given function,
                    \textcolor{y_h}{Yes}
                    \begin{align}
                        u                & = \ln(x^2 + y^2)                  &
                        r^2              & = x^2 + y^2                         \\
                        \difcp ux        & = \frac{2}{r} \cdot \frac{x}{r}
                        = \frac{2x}{r^2} &
                        \difcp[2] ux     & = \frac{2r^2 - 4x^2}{r^4}           \\
                        \difcp[2] ux
                        + \difcp[2] uy   & = \frac{4r^2 - 4(x^2 + y^2)}{r^4} &
                                         & = 0
                    \end{align}
                    From part $ b $, the fourth Poisson equation is the $ f(x, y) $
                    needed on the right hand side. This also means that it does not
                    satisfy the $2d$ Poisson equation
              \item Verifying whether the given functions satisfy the PDEs, \par
                    \textcolor{y_h}{Yes}, \textcolor{y_h}{Yes}, \textcolor{y_h}{Yes}
                    \begin{align}
                        u               & = v(x) + w(y)   &
                        \difcp{u}{x, y} & = 0               \\
                        \difcp ux       & = \difcp vx + 0 &
                        \difcp{u}{x, y} & = 0 + 0
                    \end{align}
                    \begin{align}
                        u                         & = v(x) \cdot w(y)               &
                        u \cdot \difcp{u}{x, y}   & = \difcp ux \cdot \difcp uy       \\
                        \difcp {u}{x,y}           & = \difcp vx \cdot \difcp wy     &
                        \difcp ux \cdot \difcp uy & = [w \cdot \difcp vx]
                        \cdot [v\ \difcp wy]                                          \\
                                                  &                                 &
                                                  & = [v \cdot w]\ [\difcp vx \cdot
                        \difcp wy]
                    \end{align}
                    \begin{align}
                        u            & = v(x+2t) + w(x - 2t)               &
                        \difcp[2] ut & = 4\ \difcp[2] ux                     \\
                        \difcp[2] ut & = 4\ \difcp[2] vt + 4\ \difcp[2] wt &
                        \difcp[2] ux & = \difcp[2] vx + \difcp[2] wx
                    \end{align}
          \end{enumerate}

    \item Checking if the given function satisfies Laplace's equation,
          \begin{align}
              u(x, y)                     & = a \ln(x^2 + y^2) + b &
              \difcp[2] ux + \difcp[2] uy & = 0                      \\
              \difcp[2] ux                & = 2a\ \frac{y^2 - x^2}
              {(x^2 + y^2)^2}             &
              \difcp[2] uy                & = 2a\ \frac{x^2 - y^2}
              {(x^2 + y^2)^2}                                        \\
          \end{align}
          Using the boundary conditions,
          \begin{align}
              x^2 + y^2    & = 1                     &
              \implies 110 & = b                       \\
              x^2 + y^2    & = 100                   &
              \implies 0   & = a\ln(100) + b           \\
              b            & = 110                   &
              a            & = \frac{-110}{\ln(100)}
          \end{align}

    \item Solving using ODE methods,
          \begin{align}
              \difcp[2] uy & = 0                   &
              \difcp uy    & = f(x)                  \\
              u            & = y \cdot f(x) + g(x)
          \end{align}

    \item Solving using ODE methods,
          \begin{align}
              \difcp[2] ux & = -16\pi^2\ u                                       &
              u            & = f(y) \cdot \cos(4\pi x) + g(y) \cdot \sin(4\pi x)
          \end{align}

    \item Solving using ODE methods,
          \begin{align}
              \difcp[2] uy & = \frac{4u}{25}                              &
              u            & = f(x) \cdot e^{-2y/5} + g(x) \cdot e^{2y/5}
          \end{align}

    \item Solving using ODE methods,
          \begin{align}
              \diffp uy & = -y^2\ u               &
              \ln(u)    & = \frac{-y^3}{3} + f(x)   \\
              u         & = g(x) \cdot e^{-y^3/3}
          \end{align}

    \item Solving the homogeneous ODE in $ x $,
          \begin{align}
              2\ \diffp[2] ux + 9\ \diffp ux + 4u & = -3 \cos x - 29 \sin x            \\
              2 u'' + 9u' + 4u                    & = 0                                \\
              \lambda                             & = \frac{-9 \pm 7}{4}
              = \{-4, -1/2\}                                                           \\
              u_h                                 & = c_1(y) e^{-4x} + c_2(y) e^{-x/2}
          \end{align}
          Solving the non-homogeneous ODE in $x$,
          \begin{align}
              u_p & = K \cos(x) + M \sin(x)   \\
              -3  & = -2K + 4K + 9M         &
              -29 & = -2M + 4M - 9K           \\
              u_p & = 3\cos x - \sin x
          \end{align}
          Combining the two parts of the ODE solution,
          \begin{align}
              u(x, y) & = {\color{y_h} c_1(y) e^{-4x} + c_2(y) e^{-x/2}}
              + {\color{y_p} 3\cos x - \sin x}
          \end{align}

    \item Solving the homogeneous ODE in $ y $,
          \begin{align}
              \diffp[2] uy + 6\ \diffp uy + 13u & = 4e^{3y}             \\
              u'' + 6u' + 13u                   & = 0                   \\
              \lambda                           & = \frac{-6 \pm 4i}{2}
              = \{-3 \pm 2i\}                                           \\
              u_h                               & = e^{-3y}\
              [c_1(x) \cos(2y) + c_2(x) \sin(2y)]
          \end{align}
          Solving the non-homogeneous ODE in $y$,
          \begin{align}
              u_p & = K\ e^{3y}        \\
              4   & = 9K + 18K + 13K &
              K   & = 0.1
          \end{align}
          Combining the two parts of the ODE solution,
          \begin{align}
              u(x, y) & = {\color{y_h} e^{-3y}\ [c_1(x) \cos(2y) + c_2(x) \sin(2y)]}
              + {\color{y_p} 0.1\ e^{3y}}
          \end{align}

    \item Solving the homogeneous ODE in $ y $,
          \begin{align}
              \diffp {u}{y, x} & = \diffp ux   &
              \diffp uy        & = u + f(y)      \\
              \ln[u + f(y)]    & = y + c_2     &
              u                & = Ae^y + B(y)
          \end{align}

    \item Solving the homogeneous ODE in $ y $,
          \begin{align}
              x^2\ \diffp[2] ux + 2x\ \diffp ux - 2u & = 0            &
              x^2\ u'' + 2x\ 6u - 2u                 & = 0              \\
              m^2 + (2-1)m - 2                       & = 0            &
              m                                      &
              = \frac{-1 \pm 3}{2} = \{-2, 1\}                          \\
              u_h                                    & = f(y)\ x^{-2}
              + g(y)x
          \end{align}

    \item Using the given equation and cylindrical coordinates,
          \begin{align}
              y\ \difcp zx       & = x\ \difcp zy                         &
              x                  & = r\cos \theta \qquad y = r\sin \theta   \\
              \diffp {z}{\theta} & = \difcp zx \cdot \difcp {x}{\theta}
              + \difcp zy \cdot
              \difcp {y}{\theta} &
              \difcp {z}{\theta} & = \Big[-r\sin \theta \cdot \cot \theta
              + r\cos \theta\Big]\ \difcp zy                                \\
              \difcp{z}{\theta}  & = 0
          \end{align}
          Since $ z $ is independent of $ \theta $, the surface is symmetric about the
          polar axis which makes it a surface of revolution about the $ z $ axis.

    \item Solving the system of ODEs,
          \begin{align}
              \difcp[2] ux & = 0                        &
              \difcp[2] uy & = 0                          \\
              u            & = f(y) \cdot (c_1 x + c_2) &
              u            & = g(x) \cdot (b_1y + b_2)    \\
              u            & = (c_1x + c_2)(b_1y + b_2)
          \end{align}
\end{enumerate}