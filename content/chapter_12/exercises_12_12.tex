\section{Solution of PDEs by Laplace Transforms}

\begin{enumerate}
    \item Verifying the solution,
          \begin{align}
              w(x, t) & = \sin\Big(t - \frac{x}{c}\Big) \cdot u\Big(t
              - \frac{x}{c}\Big)                                      \\
              w(0, t) & = \begin{cases}
                              \sin(t) & \quad t \in [0, 2\pi]  \\
                              0       & \quad \text{otherwise}
                          \end{cases}
          \end{align}
          The boundary condition on the $ x = 0 $ is satisfied.
          \begin{align}
              \lim_{x \to \infty} w(x, t) & = f(t - x/c) \cdot 0 = 0
          \end{align}
          Since the step function is never turned on at infinite distance, the
          boundary condition on large $ x $ is also true.
          \begin{align}
              w(x, 0) & = \sin(-x/c) \cdot u(-x/c) = 0 \quad \forall \quad x>0,\ c>0
          \end{align}
          The initial deflection is satisfied.
          \begin{align}
              w(x, t)   & =
              \begin{dcases}
                  \sin \Big( t - \frac{x}{c} \Big) & \quad t \in \Bigg[ \frac{x}{c},
                  \frac{x}{c} + 2\pi \Bigg]                                          \\
                  0                                & \quad \text{otherwise}
              \end{dcases}  \\
              w_t(x, t) & =
              \begin{dcases}
                  \cos \Big( t - \frac{x}{c} \Big) & \quad t \in \Bigg[ \frac{x}{c},
                  \frac{x}{c} + 2\pi \Bigg]                                          \\
                  0                                & \quad \text{otherwise}
              \end{dcases} \\
              w_t(x, 0) & =
              \begin{dcases}
                  \cos \Big(\frac{x}{c} \Big) & \quad x \in [-2\pi c, 0] \\
                  0                           & \quad \text{otherwise}
              \end{dcases}
          \end{align}
          Clearly $ w_t(x, 0) $ is zero everywhere on the positive x axis, which
          satisfies the initial condition on the velocity.

          Substituting the new function into $ w(x, t) $,
          \begin{align}
              f(t)    & = \begin{cases}
                              \sin(t) & \quad t > 2\pi         \\
                              0       & \quad \text{otherwise} \\
                          \end{cases}   \\
              u(x, t) & = \begin{dcases}
                              \sin \Big( t - \frac{x}{c} \Big) &
                              \quad t > \frac{x}{c} + 2\pi       \\
                              0                                &
                              \quad \text{otherwise}
                          \end{dcases}
          \end{align}

    \item A triangular wave whose defined by,
          \begin{align}
              f(x) & = \begin{cases}
                           x   & \quad x \in [0, 1/2)   \\
                           1-x & \quad x \in [1/2, 1]   \\
                           0   & \quad \text{otherwise} \\
                       \end{cases}
          \end{align}
          Plotting this function moving to the right at speed $ c $,
          \begin{figure}[H]
              \centering
              \begin{tikzpicture}[declare function = {tri(\x)
                              = ifthenelse(and(\x>0, \x<1),
                              -abs(\x - 0.5) + 0.5,
                              0);}]
                  \begin{axis}[
                          legend pos = north east, axis equal,
                          grid = both, Ani,
                          domain = 0:6, height = 10 cm,
                          colormap/viridis,
                          cycle list = {[samples of colormap = 4]},
                      ]
                      \addplot+[thick, samples = 200]{tri(x)};
                      \addplot+[thick, samples = 200]{1 + tri(x - 1)};
                      \addplot+[thick, samples = 200]{2 + tri(x - 2)};
                      \addplot+[thick, samples = 200]{3 + tri(x - 3)};
                  \end{axis}
              \end{tikzpicture}
          \end{figure}

    \item The speed is related to $ c $ using,
          \begin{align}
              v & = c = \sqrt{\frac{T}{\rho}}
          \end{align}
          where the tension in the string is $ T $ and its mass per unit length is
          $ \rho $.

    \item Solving the PDE using Laplace transforms w.r.t. $ t $,
          \begin{align}
              \diffp wx + x\ \diffp wt        & = x                             &
              w(x, 0)                         & = 1 \qquad w(0, t) = 1            \\
              \diffp Wx + x\ \Big[sW - 1\Big] & = \frac{x}{s}                   &
              \diffp Wx                       & = x\ (1/s + 1 - sW)               \\
              \ln(1/s + 1 -sW)                & = \frac{-sx^2}{2} + \alpha^*(s) &
              \frac{1}{s} + 1 - sW            & = \alpha^*(s) \cdot e^{-sx^2/2}   \\
              W                               & =
              \frac{1}{s} + \frac{1}{s^2} - \alpha(s)\ e^{-sx^2/2}
          \end{align}
          Recovering the solution using the inverse Laplace transform,
          \begin{align}
              W(0, s)       & = \Lap\{w(0, t)\} = \Lap\{(1)\}
              = \frac{1}{s} &
              \alpha(s)     & = \frac{1}{s^2}                           \\
              W(x, s)       & = \color{y_p} \frac{1}{s} + \frac{1}{s^2}
              - \frac{e^{-sx^2/2}}{s^2}                                 \\
              w(x, t)       & = (1 + t) - \Big( t - \frac{x^2}{2} \Big)
              \cdot u \Bigg[ t - \frac{x^2}{2} \Bigg]
          \end{align}
          Expressing the solution in piecewise form,
          \begin{align}
              u(x, t) & = \color{y_h} \begin{dcases}
                                          1 + t     & \quad t < x^2/2 \\
                                          1 + x^2/2 & \quad t > x^2/2 \\
                                      \end{dcases}
          \end{align}

    \item Solving the PDE using Laplace transforms w.r.t. $ t $,
          \begin{align}
              x\ \diffp wx + \diffp wt        & = xt                            \\
              x\ \diffp Wx + \Big[sW\Big]     & = \frac{x}{s^2}               &
              \diffp Wx + \frac{s}{x} \cdot W & = \frac{1}{s^2}                 \\
              \text{I.F.}                     & = \exp\Big( \int (s/x)
              \ \dl x \Big)                   &
              \text{I.F.}                     & = x^s                           \\
              x^s\ W                          & = \int \frac{x^s}{s^2}\ \dl x &
              W                               & = \frac{x}{s^2(s+1)}
              + \frac{\alpha(s)}{x^s}
          \end{align}
          Applying the B.C. and I.C. to $ W(x, s) $,
          \begin{align}
              W(0, s)                & = \Lap\{w(0, t)\} = 0                  &
              \alpha(s)              & = 0                                      \\
              W(x, s)                & = \color{y_p} x\ \Bigg[\frac{-1}{s}
                  + \frac{1}{s^2}
              + \frac{1}{s+1} \Bigg] &
              w(x, t)                & = \color{y_h} x\ \Big[ -1 + t + e^{-t}
                  \Big]
          \end{align}

    \item Solving the PDE using Laplace transforms w.r.t. $ t $,
          \begin{align}
              \diffp wx + 2x\ \diffp wt        & = 2x                          &
              w(x, 0)                          & = 1 \qquad w(0, t) = 1          \\
              \diffp Wx + 2x\ \Big[sW - 1\Big] & = \frac{2x}{s}                &
              \diffp Wx                        & = 2x\ (1/s + 1 - sW)            \\
              \ln(1/s + 1 -sW)                 & = -sx^2 + \alpha^*(s)         &
              \frac{1}{s} + 1 - sW             & = \alpha^*(s) \cdot e^{-sx^2}   \\
              W                                & =
              \frac{1}{s} + \frac{1}{s^2} - \alpha(s)\ e^{-sx^2}
          \end{align}
          Recovering the solution using the inverse Laplace transform,
          \begin{align}
              W(0, s)       & = \Lap\{w(0, t)\} = \Lap\{(1)\}
              = \frac{1}{s} &
              \alpha(s)     & = \frac{1}{s^2}                           \\
              W(x, s)       & = \color{y_p} \frac{1}{s} + \frac{1}{s^2}
              - \frac{e^{-sx^2}}{s^2}                                   \\
              w(x, t)       & = (1 + t) - \Big( t - x^2 \Big)
              \cdot u \Big[ t - x^2 \Big]
          \end{align}
          Expressing the solution in piecewise form,
          \begin{align}
              u(x, t) & = \color{y_h} \begin{dcases}
                                          1 + t   & \quad t < x^2 \\
                                          1 + x^2 & \quad t > x^2 \\
                                      \end{dcases}
          \end{align}

    \item TBC

    \item Solving the PDE using Laplace transforms w.r.t. $ t $,
          \begin{align}
              \diffp[2] wx & = 100\ \diffp[2] wt + 100\ \diffp wt
              + 25w                                               \\
              w(x, 0)      & = 0 \qquad w(0, t) = \sin(t)
              \qquad w_t(x, 0) = 0                                \\
              \diff[2] Wx  & = (10s + 5)^2\ W                     \\
              W            & = A(s) \exp[(10s + 5)x] + B(s)
              \exp[-(10s + 5)x] + C(s)
          \end{align}
          Applying the initial conditions,
          \begin{align}
              \lim_{x \to \infty} w(x, t) & = 0                        &
              \lim_{x \to \infty} W(x, s) & = 0                          \\
              A(s) = C(s)                 & = 0                          \\
              W(0, s)                     & = \Lap\{w(0, t)\}
              = \Lap\{\sin(t)\}
              = \frac{1}{s}               &
              B(s)                        & = \frac{1}{1 + s^2}          \\
              W(x, s)                     & = \color{y_p}
              \frac{e^{-(10s+5)x}}{1+s^2}                                \\
              w(x, t)                     & = e^{-5x}\ \sin(t - 10x)\
              u \Big[ t - 10x \Big]
          \end{align}
          Expressing the solution in piecewise form,
          \begin{align}
              u(x, t) & = \color{y_h} \begin{dcases}
                                          0                      & \quad t < 10x \\
                                          e^{-5x}\ \sin(t - 10x) & \quad t > 10x \\
                                      \end{dcases}
          \end{align}

    \item Starting with the heat equation and taking the Laplace transform,
          \begin{align}
              w_t                   & = c^2\ w_{xx}           &
              w(x, 0)               & = 0                       \\
              sW                    & = c^2\ \diffp[2] Wx     &
              W                     & = A(s)\ e^{\sqrt{s}x/c}
              + B(s)\ e^{-\sqrt{s}x/c}                          \\
              \lim_{x \to \infty} w & = 0                     &
              \implies A(s)         & = 0                       \\
              W(0, s)               & = \Lap\{w(0, t)\}       &
              \Lap\{f(t)\}          & = F(s)
          \end{align}
          Using the initial conditions, and referring to the table of Laplace
          transforms,
          \begin{align}
              W(0, s) & = F(s) = B(s)                                                &
              W(x, s) & = \color{y_h} F(s)\ e^{-\sqrt{s}x/c}                           \\
              G(s)    & = e^{-\sqrt{s}x/c}                                           &
              g(t)    & = \frac{x}{2c\sqrt{\pi t^3}}\ \exp \Bigg[ -\frac{x^2}{4c^2t}
                  \Bigg]
          \end{align}

    \item Substituting the functional form of $ g(t) $ into the convolution integral,
          \begin{align}
              w(x, t) & = \frac{x}{2c\sqrt{\pi}} \int_0^t \frac{\exp[-x^2/(4c^2\tau)]}
              {\tau^{3/2}}\ f(t - \tau)\ \dl \tau
          \end{align}

    \item Since the limits of integration mean that $ t \geq \tau $, the step
          function $ u(t - \tau) $ is always on.
          \begin{align}
              w_0(x, t) & = \frac{x}{2c\sqrt{\pi}} \int_0^t
              \frac{\exp[-x^2/(4c^2\tau)]}{\tau^{3/2}}\ \dl \tau                  \\
              y         & = \frac{x}{2c\sqrt{\tau}} \qquad\qquad
              \dl y = \frac{-x}{4c\ \tau^{3/2}}\ \dl \tau                         \\
              w_0(x, t) & = \frac{-2}{\sqrt{\pi}}\ \int_{\infty}^{x/(2c\sqrt{t})}
              e^{-y^2}\ \dl y                                                     \\
                        & = \erf(\infty) - \erf\Big( \frac{x}{2c\sqrt{t}} \Big)   \\
                        & = 1 - \erf\Big( \frac{x}{2c\sqrt{t}} \Big)
          \end{align}

    \item Using the form of $ w_0(x, t) $ from Problem $ 11 $,
          \begin{align}
              W_0(x, s)       & = F(s) \cdot e^{-x\sqrt{s}/c}
              = \frac{e^{-x\sqrt{s}/c}}{s}                                          \\
              \Lap\{u[t-a]\}  & = \frac{e^{-as}}{s}                                 \\
              \diffp {w_0}{t} & = \frac{x}{2c\sqrt{\pi}}\
              \exp \Bigg[ - \frac{x^2}{4c^2\ t} \Bigg]\ t^{-3/2}                    \\
              w(x, t)         & = \int_{0}^{t} \diffp*{\ \Big[w_0(\tau)\Big]}{\tau}
              \ f(t - \tau)\ \dl \tau
          \end{align}
          The last step directly uses the result of Problem $ 10 $.
\end{enumerate}