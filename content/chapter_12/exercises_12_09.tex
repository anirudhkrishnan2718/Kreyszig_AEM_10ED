\section{Rectangular Membrane. Double Fourier Series}

\begin{enumerate}
    \item Frequency of the eigenfunction $ u_{mn} $ is given by
          \begin{align}
              f_{mn}       & = \frac{\lambda_{mn}}{2\pi}        &
              \lambda_{mn} & = \sqrt{\frac{T\pi^2}{\rho}} \cdot
              \sqrt{\frac{m^2}{a^2} + \frac{n^2}{b^2}}            \\
              T            & \to 2T                             &
              \implies f   & \to \sqrt{2} f                       \\
              \rho         & \to \rho/2                         &
              \implies f   & \to \sqrt{2} f                       \\
              a,b          & \to 2a, 2b                         &
              \implies f   & \to  f/2
          \end{align}

    \item The small angle assumption enables the following approximation,
          \begin{align}
              \sin(\theta) \approxeq \theta \approxeq \tan(\theta)
          \end{align}
          The assumption that the tension in the membrane is constant at all points
          in space and through time is unrealistic.

    \item Nodal lines are points on the membrane that do not move. Since one half of
          the nodal lines are the same as the other half with $ x, y $ interchanged,
          only the nodal lines for $ n \geq m $ are depicted here.
          \begin{align}
              F_{mn} & = \sin(m\pi x)\ \sin(n\pi y) &
              a      & = b = 1
          \end{align}
          \begin{figure}[H]
              \centering
              \begin{tikzpicture}
                  \begin{axis}[xmin=0,xmax=1,ymin=0,ymax=1,
                          title = {$F_{11}$}, width = 8cm,
                          xtick distance = 0.25, ytick distance = 0.25,
                          grid = both,Ani,
                          enlargelimits = false]
                      \draw[thick, y_p] (axis cs:0.005, 0.005) -- (axis cs:0.995, 0.005)
                      -- (axis cs:0.995, 0.995) -- (axis cs:0.005, 0.995) -- cycle;
                  \end{axis}
              \end{tikzpicture}
              \begin{tikzpicture}
                  \begin{axis}[xmin=0,xmax=1,ymin=0,ymax=1,
                          title = {$F_{12}$}, width = 8cm,
                          xtick distance = 0.25, ytick distance = 0.25,
                          grid = both,Ani,
                          enlargelimits = false]
                      \draw[thick, y_p] (axis cs:0, 0.5) -- (axis cs:1, 0.5);
                  \end{axis}
              \end{tikzpicture}
              \begin{tikzpicture}
                  \begin{axis}[xmin=0,xmax=1,ymin=0,ymax=1,
                          title = {$F_{13}$}, width = 8cm,
                          xtick distance = 0.25, ytick distance = 0.3333,
                          grid = both,Ani,
                          enlargelimits = false]
                      \draw[thick, y_p] (axis cs:0, 0.3333) -- (axis cs:1, 0.3333);
                      \draw[thick, y_p] (axis cs:0, 0.6666) -- (axis cs:1, 0.6666);
                  \end{axis}
              \end{tikzpicture}
              \begin{tikzpicture}
                  \begin{axis}[xmin=0,xmax=1,ymin=0,ymax=1,
                          title = {$F_{14}$}, width = 8cm,
                          xtick distance = 0.25, ytick distance = 0.25,
                          grid = both,Ani,
                          enlargelimits = false]
                      \draw[thick, y_p] (axis cs:0, 0.25) -- (axis cs:1, 0.25);
                      \draw[thick, y_p] (axis cs:0, 0.5) -- (axis cs:1, 0.5);
                      \draw[thick, y_p] (axis cs:0, 0.75) -- (axis cs:1, 0.75);
                  \end{axis}
              \end{tikzpicture}
              \begin{tikzpicture}
                  \begin{axis}[xmin=0,xmax=1,ymin=0,ymax=1,
                          title = {$F_{22}$}, width = 8cm,
                          xtick distance = 0.25, ytick distance = 0.25,
                          grid = both,Ani,
                          enlargelimits = false]
                      \draw[thick, y_h] (axis cs:0, 0.5) -- (axis cs:1, 0.5);
                      \draw[thick, y_h] (axis cs:0.5, 0) -- (axis cs:0.5, 1);
                  \end{axis}
              \end{tikzpicture}
              \begin{tikzpicture}
                  \begin{axis}[xmin=0,xmax=1,ymin=0,ymax=1,
                          title = {$F_{23}$}, width = 8cm,
                          xtick distance = 0.25, ytick distance = 0.3333,
                          grid = both,Ani,
                          enlargelimits = false]
                      \draw[thick, y_h] (axis cs:0, 0.3333) -- (axis cs:1, 0.3333);
                      \draw[thick, y_h] (axis cs:0, 0.6666) -- (axis cs:1, 0.6666);
                      \draw[thick, y_h] (axis cs:0.5, 0) -- (axis cs:0.5, 1);
                  \end{axis}
              \end{tikzpicture}
          \end{figure}
          \begin{figure}[H]
              \centering
              \begin{tikzpicture}
                  \begin{axis}[xmin=0,xmax=1,ymin=0,ymax=1,
                          title = {$F_{24}$}, width = 8cm,
                          xtick distance = 0.25, ytick distance = 0.25,
                          grid = both,Ani,
                          enlargelimits = false]
                      \draw[thick, y_h] (axis cs:0, 0.25) -- (axis cs:1, 0.25);
                      \draw[thick, y_h] (axis cs:0, 0.5) -- (axis cs:1, 0.5);
                      \draw[thick, y_h] (axis cs:0, 0.75) -- (axis cs:1, 0.75);
                      \draw[thick, y_h] (axis cs:0.5, 0) -- (axis cs:0.5, 1);
                  \end{axis}
              \end{tikzpicture}
              \begin{tikzpicture}
                  \begin{axis}[xmin=0,xmax=1,ymin=0,ymax=1,
                          title = {$F_{33}$}, width = 8cm,
                          xtick distance = 0.3333, ytick distance = 0.3333,
                          grid = both,Ani,
                          enlargelimits = false]
                      \draw[thick, y_t] (axis cs:0, 0.3333) -- (axis cs:1, 0.3333);
                      \draw[thick, y_t] (axis cs:0, 0.6666) -- (axis cs:1, 0.6666);
                      \draw[thick, y_t] (axis cs:0.3333, 0) -- (axis cs:0.3333, 1);
                      \draw[thick, y_t] (axis cs:0.6666, 0) -- (axis cs:0.6666, 1);
                  \end{axis}
              \end{tikzpicture}
              \begin{tikzpicture}
                  \begin{axis}[xmin=0,xmax=1,ymin=0,ymax=1,
                          title = {$F_{34}$}, width = 8cm,
                          xtick distance = 0.3333, ytick distance = 0.25,
                          grid = both,Ani,
                          enlargelimits = false]
                      \draw[thick, y_t] (axis cs:0, 0.25) -- (axis cs:1, 0.25);
                      \draw[thick, y_t] (axis cs:0, 0.5) -- (axis cs:1, 0.5);
                      \draw[thick, y_t] (axis cs:0, 0.75) -- (axis cs:1, 0.75);
                      \draw[thick, y_t] (axis cs:0.3333, 0) -- (axis cs:0.3333, 1);
                      \draw[thick, y_t] (axis cs:0.6666, 0) -- (axis cs:0.6666, 1);
                  \end{axis}
              \end{tikzpicture}
              \begin{tikzpicture}
                  \begin{axis}[xmin=0,xmax=1,ymin=0,ymax=1,
                          title = {$F_{44}$}, width = 8cm,
                          xtick distance = 0.25, ytick distance = 0.25,
                          grid = both,Ani,
                          enlargelimits = false]
                      \draw[thick, azure4] (axis cs:0, 0.25) -- (axis cs:1, 0.25);
                      \draw[thick, azure4] (axis cs:0, 0.5) -- (axis cs:1, 0.5);
                      \draw[thick, azure4] (axis cs:0, 0.75) -- (axis cs:1, 0.75);
                      \draw[thick, azure4] (axis cs:0.25, 0) -- (axis cs:0.25, 1);
                      \draw[thick, azure4] (axis cs:0.5, 0) -- (axis cs:0.5, 1);
                      \draw[thick, azure4] (axis cs:0.75, 0) -- (axis cs:0.75, 1);
                  \end{axis}
              \end{tikzpicture}
          \end{figure}

    \item Representing the given initial displacement as a double Fourier series,
          \begin{align}
              f(x, y) & = \iser[m]{1} \iser[n]{1} B_{mn}\ \sin(m\pi x)\ \sin(n \pi y) \\
              K_m(y)  & = 2 \int_{0}^{1} f(x, y)\ \sin(m\pi x)\ \dl x                 \\
                      & = \Bigg[ \frac{2}{m\pi}\ \cos(m\pi x) \Bigg]_1^0
              = \color{y_h} \frac{2[1 - \cos(m\pi)]}{m\pi}                            \\
              B_{mn}  & = 2 \int_{0}^{1} K_m(y)\ \sin(n\pi y)\ \dl y                  \\
                      & = \color{y_p} \frac{4}{mn\pi^2}
              \ [1 - \cos(m\pi)][1 - \cos(n\pi)]
          \end{align}

    \item Representing the given initial displacement as a double Fourier series,
          \begin{align}
              f(x, y) & = \iser[m]{1} \iser[n]{1} B_{mn}\ \sin(m\pi x)\ \sin(n \pi y)  \\
              K_m(y)  & = 2 \int_{0}^{1} (y)\ \sin(m\pi x)\ \dl x                      \\
                      & = \Bigg[ \frac{2y}{m\pi}\ \cos(m\pi x) \Bigg]_1^0
              = \color{y_h} \frac{2y}{m\pi}\ [1 - \cos(m\pi)]                          \\
              B_{mn}  & = 2 \int_{0}^{1} K_m(y)\ \sin(n\pi y)\ \dl y                   \\
                      & = \frac{4[1 - \cos(m\pi)]}{m\pi}
              \int_{0}^{1} (y)\ \sin(n\pi y)\ \dl y                                    \\
                      & = \frac{4[1 - \cos(m\pi)]}{m\pi}
              \ \Bigg[ \frac{\sin(n\pi y) - n\pi y \cos(n\pi y)}{n^2 \pi^2} \Bigg]_0^1 \\
                      & = \color{y_p} \frac{-4}{mn\pi^2}
              \ [1 - \cos(m\pi)][\cos(n\pi)]
          \end{align}

    \item Representing the given initial displacement as a double Fourier series,
          \begin{align}
              f(x, y) & = \iser[m]{1} \iser[n]{1} B_{mn}\ \sin(m\pi x)
              \ \sin(n \pi y)                                               \\
              K_m(y)  & = 2 \int_{0}^{1} (x)\ \sin(m\pi x)\ \dl x           \\
                      & = 2\Bigg[ \frac{\sin(m\pi x) - m\pi x \cos(m\pi x)}
                  {m^2 \pi^2} \Bigg]_0^1
              = \color{y_h} \frac{-2\cos(m\pi)}{m\pi}                       \\
              B_{mn}  & = 2 \int_{0}^{1} K_m(y)\ \sin(n\pi y)\ \dl y        \\
                      & = \frac{-4\cos(m\pi)}{m\pi}
              \int_{0}^{1} \sin(n\pi y)\ \dl y                              \\
                      & = \frac{-4\cos(m\pi)}{m\pi}
              \ \Bigg[ \frac{\cos(n\pi y)}{n\pi} \Bigg]_1^0                 \\
                      & = \color{y_p} \frac{-4}{mn\pi^2}
              \ [1 - \cos(n\pi)][\cos(m\pi)]
          \end{align}

    \item Representing the given initial displacement as a double Fourier series,
          \begin{align}
              f(x, y) & = \iser[m]{1} \iser[n]{1} B_{mn}\ \sin\Big( \frac{m\pi x }{a}
              \Big)\ \sin\Big( \frac{n\pi y }{b}\Big)                                 \\
              K_m(y)  & = \frac{2}{a} \int_{0}^{a} (xy)\ \sin\Big( \frac{m\pi x }{a}
              \Big)\ \dl x                                                            \\
                      & = \frac{2y}{m^2 \pi^2}\ \Bigg[ a\ \sin\Big( \frac{m\pi x }{a}
                  \Big) + (m\pi x) \cos\Big( \frac{m\pi x }{a} \Big) \Bigg]_0^a
              = \color{y_h} \frac{2ay\cos(m\pi)}{m\pi}                                \\
              B_{mn}  & = \frac{2}{b} \int_{0}^{b} K_m(y)\ \sin\Big(\frac{n\pi y }{b}
              \Big)\ \dl y                                                            \\
                      & = \frac{4a\cos(m\pi)}{bm\pi}
              \int_{0}^{b} (y)\ \sin\Big( \frac{n\pi y }{b}\Big)\ \dl y               \\
                      & = \frac{4a\cos(m\pi)}{m\pi\ (n^2\pi^2)}
              \ \Bigg[ b\ \sin\Big( \frac{n\pi y }{b}
              \Big) + (n\pi y) \cos\Big( \frac{n\pi y }{b} \Big) \Bigg]_0^b           \\
                      & = \color{y_p} \frac{4ab}{mn\pi^2}
              \ [\cos(n\pi)\ \cos(m\pi)]
          \end{align}

    \item Representing the given initial displacement as a double Fourier series,
          \begin{align}
              f(x, y) & = \iser[m]{1} \iser[n]{1} B_{mn}\ \sin\Big( \frac{m\pi x }{a}
              \Big)\ \sin\Big( \frac{n\pi y }{b}\Big)                                 \\
              K_m(y)  & = \frac{2}{a} \int_{0}^{a} (xy)(a-x)(b-y)\ \sin\Big(
              \frac{m\pi x }{a}\Big)\ \dl x                                           \\
                      & = \frac{2y(b-y)}{a m^3 \pi^3}\ \Bigg[ h(x)
              \ \sin\Big( \frac{m\pi x }{a} \Big)
              + [(m^2\pi^2 x)(x-a) - 2a^2]\ \cos\Big( \frac{m\pi x }{a} \Big)
              \Bigg]_0^a                                                              \\
                      & = \color{y_h} \frac{4ay(b-y)}{m^3\pi^3}
              \ [1 - \cos(m\pi)]                                                      \\
              B_{mn}  & = \frac{2}{b} \int_{0}^{b} K_m(y)\ \sin\Big(\frac{n\pi y }{b}
              \Big)\ \dl y                                                            \\
                      & = \frac{8a[1 - \cos(m\pi)]}{bm^3\pi^3}
              \int_{0}^{b} (y)(b-y)\ \sin\Big( \frac{n\pi y }{b}\Big)\ \dl y          \\
                      & = \frac{8a[1 - \cos(m\pi)]}{bm^3n^3\pi^6}
              \ \Bigg[ h(y)\ \sin\Big( \frac{n\pi y }{b} \Big)
              + [(n^2\pi^2 y)(y-b) - 2b^2]\ \cos\Big( \frac{n\pi y }{b} \Big)
              \Bigg]_0^b                                                              \\
                      & = \color{y_p} \frac{16ab}{m^3n^3\pi^6}
              \ [1 - \cos(n\pi)]\ [1 - \cos(m\pi)]
          \end{align}

    \item Using \texttt{gnuplot} to plot the series sum,
          \begin{enumerate}
              \item For problem $ 5 $,
                    \begin{align}
                        f(x, y) & = \iser[m]{1} \iser[n]{1} B_{mn}\ \sin(m\pi x)
                        \ \sin(n\pi y) = y                                       \\
                        B_{mn}  & = \frac{-4}{mn\pi^2}
                        \ [1 - \cos(m\pi)][\cos(n\pi)]
                    \end{align}
                    \begin{figure}[H]
                        \centering
                        \begin{tikzpicture}[declare function = {
                                        Fou(\m,\n,\x,\y) = sin(\m * pi * \x)
                                        * sin(\n * pi * y) * cos(\n * pi)
                                        * ((-8) / (\m * \n * pi^2));
                                    }]
                            \begin{axis}[
                                    title = {$ S_{5,5} $ partial sum},
                                    width = 14cm, height = 14cm,
                                    grid = none,
                                    ticks = none,
                                    view={30}{30},domain = 0:0.5, y domain = 0:0.5,
                                    xlabel=$x$,ylabel=$y$,zlabel=$u$,
                                    colormap/jet,
                                    Ani]
                                \addplot3 [surf, opacity= 0.75, samples = 50,
                                    faceted color = black] {Fou(1,1,x,y) + Fou(1,2,x,y)
                                    + Fou(1,3,x,y) + Fou(1,4,x,y) + Fou(1,5,x,y) +
                                    Fou(3,1,x,y) + Fou(3,2,x,y)
                                    + Fou(3,3,x,y) + Fou(3,4,x,y) + Fou(3,5,x,y) +
                                    Fou(5,1,x,y) + Fou(5,2,x,y)
                                    + Fou(5,3,x,y) + Fou(5,4,x,y) + Fou(5,5,x,y)};
                            \end{axis}
                        \end{tikzpicture}
                    \end{figure}

                    For problem $ 6 $,
                    \begin{align}
                        f(x, y) & = \iser[m]{1} \iser[n]{1} B_{mn}\ \sin(m\pi x)
                        \ \sin(n\pi y) = y                                       \\
                        B_{mn}  & = \frac{-4}{mn\pi^2}
                        \ [1 - \cos(n\pi)][\cos(m\pi)]
                    \end{align}
                    \begin{figure}[H]
                        \centering
                        \begin{tikzpicture}[declare function = {
                                        Fou(\m,\n,\x,\y) = sin(\m * pi * \x)
                                        * sin(\n * pi * y) * cos(\m * pi)
                                        * ((-8) / (\m * \n * pi^2));
                                    }]
                            \begin{axis}[
                                    title = {$ S_{5,5} $ partial sum},
                                    width = 14cm, height = 14cm,
                                    grid = none,
                                    ticks = none,
                                    view={30}{30},domain = 0:0.5, y domain = 0:0.5,
                                    xlabel=$x$,ylabel=$y$,zlabel=$u$,
                                    colormap/jet,
                                    Ani]
                                \addplot3 [surf, opacity= 0.75, samples = 50,
                                    faceted color = black] {Fou(1,1,x,y) + Fou(2,1,x,y)
                                    + Fou(3,1,x,y) + Fou(4,1,x,y) + Fou(5,1,x,y) +
                                    Fou(1,3,x,y) + Fou(2,3,x,y)
                                    + Fou(3,3,x,y) + Fou(4,3,x,y) + Fou(5,3,x,y) +
                                    Fou(1,5,x,y) + Fou(2,5,x,y)
                                    + Fou(3,5,x,y) + Fou(4,5,x,y) + Fou(5,5,x,y)};
                            \end{axis}
                        \end{tikzpicture}
                    \end{figure}

              \item For problem $ 4 $,
                    \begin{align}
                        f(x, y) & = \iser[m]{1} \iser[n]{1} B_{mn}\ \sin(m\pi x)
                        \ \sin(n\pi y) = y                                       \\
                        B_{mn}  & = \frac{-4}{mn\pi^2}
                        \ [1 - \cos(n\pi)][\cos(m\pi)]
                    \end{align}
                    \begin{figure}[H]
                        \centering
                        \begin{tikzpicture}[declare function = {
                                        Fou(\m,\n,\x,\y) = sin(\m * pi * \x)
                                        * sin(\n * pi * y)
                                        * ((16) / (\m * \n * pi^2));
                                    }]
                            \begin{axis}[
                                    title = {$ S_{9,9} $ partial sum},
                                    width = 14cm, height = 14cm,
                                    grid = none,
                                    ticks = none,
                                    view={30}{30},domain = 0:0.5, y domain = 0:0.5,
                                    xlabel=$x$,ylabel=$y$,zlabel=$u$,
                                    colormap/jet,
                                    Ani]
                                \addplot3 [surf, opacity= 0.75, samples = 50,
                                    faceted color = black] {Fou(1,1,x,y) + Fou(3,1,x,y)
                                    + Fou(5,1,x,y) + Fou(7,1,x,y) + Fou(9,1,x,y) +
                                    Fou(1,3,x,y) + Fou(3,3,x,y)
                                    + Fou(5,3,x,y) + Fou(7,3,x,y) + Fou(9,3,x,y) +
                                    Fou(1,5,x,y) + Fou(3,5,x,y)
                                    + Fou(5,5,x,y) + Fou(7,5,x,y) + Fou(9,5,x,y) +
                                    Fou(1,7,x,y) + Fou(3,7,x,y)
                                    + Fou(5,7,x,y) + Fou(7,7,x,y) + Fou(9,7,x,y) +
                                    Fou(1,9,x,y) + Fou(3,9,x,y)
                                    + Fou(5,9,x,y) + Fou(7,9,x,y) + Fou(9,9,x,y)};
                            \end{axis}
                        \end{tikzpicture}
                    \end{figure}

              \item TBC.
          \end{enumerate}

    \item The same number has to be decomposed into the sums of squares of integers
          in two different ways. This is achieved using the Brahmagupta-Fibonacci
          identity. Using \texttt{python},
          \begin{align}
              (a^2 + b^2)(c^2 + d^2)      & = \color{y_h}(ac - bd)^2 + (ad + bc)^2 \\
                                          & = \color{y_p}(ac + bd)^2 + (ad - bc)^2 \\
              65 = (1^2 + 2^2)(2^2 + 3^2) & = \color{y_h}(4)^2 + (7)^2             \\
                                          & = \color{y_p}(8)^2 + (1)^2
          \end{align}
          Brute force algorithm which fits integers $ a,b,c,d $ into the above formula
          such that the right hand side does not have zero terms. \par
          Further, impose the constraint $ (ad + bc) \neq (ac + bd) $. Nodal lines TBC.

    \item Given the side length $ a = b = \pi $ and $ c^2 = 1 $, and zero initial
          velocity,
          \begin{align}
              f(x, y)    & = \iser[m]{1} \iser[n]{1} B_{mn}\ \sin(mx)
              \ \sin(ny)                                                          \\
              K_m(y)     & = \frac{2}{\pi} \int_{0}^{\pi} (0.1\sin(2x)\sin(4y))
              \ \sin(mx)\ \dl x                                                   \\
                         & = \color{y_h} \begin{cases}
                                             0.1\sin(4y) & \quad m = 2            \\
                                             0           & \quad \text{otherwise} \\
                                         \end{cases}     \\
              B_{mn}     & = \frac{2}{\pi} \int_{0}^{\pi} K_m(y)\ \sin(ny)\ \dl y \\
                         & = \frac{2}{\pi} \int_{0}^{\pi} (0.1\sin 4y)\ \sin(ny)
              \ \dl y                                                             \\
                         & = \color{y_p} \begin{cases}
                                             0.1 & \quad m = 2, n = 4     \\
                                             0   & \quad \text{otherwise} \\
                                         \end{cases}             \\
              u(x, y, t) & = 0.1 \cos(\sqrt{20}t)\ \sin(2x)\ \sin(4y)
          \end{align}

    \item Using the fact that $ f(x) $ is already a term in the double Fourier
          series,
          \begin{align}
              u(x, y, t) & = 0.01 \cos(\sqrt{2}t)\ \sin(x)\ \sin(y)
          \end{align}

    \item Using the result from Problem $ 8 $,
          \begin{align}
              f(x, y)    & = \iser[m]{1} \iser[n]{1} B_{mn}\ \sin(mx)\ \sin(ny)  \\
              K_m(y)     & = \frac{2}{\pi} \int_{0}^{\pi} (xy)(\pi - x)(\pi - y)
              \ \sin(mx)\ \dl x                                                  \\
                         & = \frac{2y(\pi-y)}{m^3 \pi}\ \Bigg[ h(x)
              \ \sin(mx) + [(m^2x)(x-\pi) - 2]\ \cos(mx)\Bigg]_0^\pi             \\
                         & = \color{y_h} \frac{4y(\pi-y)}{m^3\pi}
              \ [1 - \cos(m\pi)]                                                 \\
              B_{mn}     & = \frac{2}{b} \int_{0}^{b} K_m(y)
              \ \sin\Big(\frac{n\pi y }{b} \Big)\ \dl y                          \\
                         & = \frac{8[1 - \cos(m\pi)]}{m^3\pi^2}
              \int_{0}^{b} (y)(b-y)\ \sin(ny)\ \dl y                             \\
                         & = \color{y_p} \frac{16}{m^3n^3\pi^2}
              \ [1 - \cos(n\pi)]\ [1 - \cos(m\pi)]                               \\
              u(x, y, t) & = 0.1B_{mn}\ \sin(mx)\ \sin(ny)
          \end{align}

    \item The nodal lines are located when,
          \begin{align}
              \sin\Big( \frac{m\pi x}{4} \Big)                & = 0 &
              \text{or}\quad \sin\Big( \frac{n\pi y}{2} \Big) & = 0   \\
          \end{align}
          The amplitude of each term varies as $ (mn)^{-3} $. Clearly, this decays very
          quickly in $ m,n $, which means that the first term dominates the series sum.

    \item Nodal lines are points on the membrane that do not move. Since one half of
          the nodal lines are the same as the other half with $ x, y $ interchanged,
          only the nodal lines for $ n \geq m $ are depicted here.
          \begin{align}
              F_{mn} & = \sin(m\pi x)\ \sin(n\pi y) &
              a      & = 4 \qquad\qquad b = 2
          \end{align}
          \begin{figure}[H]
              \centering
              \begin{tikzpicture}
                  \begin{axis}[xmin=0,xmax=4,ymin=0,ymax=2, axis equal,
                          title = {$F_{11}$}, width = 10.45cm, height = 6cm,
                          xtick distance = 1, ytick distance = 1,
                          grid = both,Ani,
                          enlargelimits = false]
                      \draw[thick, y_p] (axis cs:0.005, 0.005) -- (axis cs:3.995, 0.005)
                      -- (axis cs:3.995, 1.995) -- (axis cs:0.005, 1.995) -- cycle;
                  \end{axis}
              \end{tikzpicture}
              \vspace{1em}
              \begin{tikzpicture}
                  \begin{axis}[xmin=0,xmax=4,ymin=0,ymax=2, axis equal,
                          title = {$F_{12}$}, width = 10.45cm, height = 6cm,
                          xtick distance = 1, ytick distance = 1,
                          grid = both,Ani,
                          enlargelimits = false]
                      \draw[thick, y_p] (axis cs:0, 1) -- (axis cs:4, 1);
                  \end{axis}
              \end{tikzpicture}
              \vspace{1em}
              \begin{tikzpicture}
                  \begin{axis}[xmin=0,xmax=4,ymin=0,ymax=2, axis equal,
                          title = {$F_{13}$}, width = 10.45cm, height = 6cm,
                          xtick distance = 1, ytick distance = 2*0.3333,
                          grid = both,Ani,
                          enlargelimits = false]
                      \draw[thick, y_p] (axis cs:0, 2*0.3333) -- (axis cs:4, 2*0.3333);
                      \draw[thick, y_p] (axis cs:0, 2*0.6666) -- (axis cs:4, 2*0.6666);
                  \end{axis}
              \end{tikzpicture}
              \vspace{1em}
              \begin{tikzpicture}
                  \begin{axis}[xmin=0,xmax=4,ymin=0,ymax=2, axis equal,
                          title = {$F_{14}$}, width = 10.45cm, height = 6cm,
                          xtick distance = 1, ytick distance = 0.5,
                          grid = both,Ani,
                          enlargelimits = false]
                      \draw[thick, y_p] (axis cs:0, 0.5) -- (axis cs:4, 0.5);
                      \draw[thick, y_p] (axis cs:0, 1) -- (axis cs:4, 1);
                      \draw[thick, y_p] (axis cs:0, 1.5) -- (axis cs:4, 1.5);
                  \end{axis}
              \end{tikzpicture}
          \end{figure}

          \begin{figure}[H]
              \centering
              \begin{tikzpicture}
                  \begin{axis}[xmin=0,xmax=4,ymin=0,ymax=2, axis equal,
                          title = {$F_{22}$}, width = 10.45cm, height = 6cm,
                          xtick distance = 1, ytick distance = 0.5,
                          grid = both,Ani,
                          enlargelimits = false]
                      \draw[thick, y_h] (axis cs:0, 1) -- (axis cs:4, 1);
                      \draw[thick, y_h] (axis cs:2, 0) -- (axis cs:2, 2);
                  \end{axis}
              \end{tikzpicture}
              \vspace{1em}
              \begin{tikzpicture}
                  \begin{axis}[xmin=0,xmax=4,ymin=0,ymax=2, axis equal,
                          title = {$F_{23}$}, width = 10.45cm, height = 6cm,
                          xtick distance = 1, ytick distance = 2*0.3333,
                          grid = both,Ani,
                          enlargelimits = false]
                      \draw[thick, y_h] (axis cs:0, 2*0.3333) -- (axis cs:4, 2*0.3333);
                      \draw[thick, y_h] (axis cs:0, 2*0.6666) -- (axis cs:4, 2*0.6666);
                      \draw[thick, y_h] (axis cs:2, 0) -- (axis cs:2, 2);
                  \end{axis}
              \end{tikzpicture}
              \vspace{1em}
              \begin{tikzpicture}
                  \begin{axis}[xmin=0,xmax=4,ymin=0,ymax=2, axis equal,
                          title = {$F_{24}$}, width = 10.45cm, height = 6cm,
                          xtick distance = 1, ytick distance = 0.5,
                          grid = both,Ani,
                          enlargelimits = false]
                      \draw[thick, y_h] (axis cs:0, 0.5) -- (axis cs:4, 0.5);
                      \draw[thick, y_h] (axis cs:0, 1) -- (axis cs:4, 1);
                      \draw[thick, y_h] (axis cs:0, 1.5) -- (axis cs:4, 1.5);
                      \draw[thick, y_h] (axis cs:2, 0) -- (axis cs:2, 2);
                  \end{axis}
              \end{tikzpicture}
              \vspace{1em}
              \begin{tikzpicture}
                  \begin{axis}[xmin=0,xmax=4,ymin=0,ymax=2, axis equal,
                          title = {$F_{33}$}, width = 10.45cm, height = 6cm,
                          xtick distance = 2*0.3333, ytick distance = 1.3333,
                          grid = both,Ani,
                          enlargelimits = false]
                      \draw[thick, y_t] (axis cs:0, 2*0.3333) -- (axis cs:4, 2*0.3333);
                      \draw[thick, y_t] (axis cs:0, 2*0.6666) -- (axis cs:4, 2*0.6666);
                      \draw[thick, y_t] (axis cs:4*0.3333, 0) -- (axis cs:4*0.3333, 4);
                      \draw[thick, y_t] (axis cs:4*0.6666, 0) -- (axis cs:4*0.6666, 4);
                  \end{axis}
              \end{tikzpicture}
          \end{figure}
          \begin{figure}[H]
              \centering
              \begin{tikzpicture}
                  \begin{axis}[xmin=0,xmax=4,ymin=0,ymax=2, axis equal,
                          title = {$F_{34}$}, width = 10.45cm, height = 6cm,
                          xtick distance = 4*0.3333, ytick distance = 0.5,
                          grid = both,Ani,
                          enlargelimits = false]
                      \draw[thick, y_t] (axis cs:0, 0.5) -- (axis cs:4, 0.5);
                      \draw[thick, y_t] (axis cs:0, 1) -- (axis cs:4, 1);
                      \draw[thick, y_t] (axis cs:0, 1.5) -- (axis cs:4, 1.5);
                      \draw[thick, y_t] (axis cs:4*0.3333, 0) -- (axis cs:4*0.3333, 2);
                      \draw[thick, y_t] (axis cs:4*0.6666, 0) -- (axis cs:4*0.6666, 2);
                  \end{axis}
              \end{tikzpicture}
              \vspace{1em}
              \begin{tikzpicture}
                  \begin{axis}[xmin=0,xmax=4,ymin=0,ymax=2, axis equal,
                          title = {$F_{44}$}, width = 10.45cm, height = 6cm,
                          xtick distance = 1, ytick distance = 0.5,
                          grid = both,Ani,
                          enlargelimits = false]
                      \draw[thick, azure4] (axis cs:0, 0.5) -- (axis cs:4, 0.5);
                      \draw[thick, azure4] (axis cs:0, 1) -- (axis cs:4, 1);
                      \draw[thick, azure4] (axis cs:0, 1.5) -- (axis cs:4, 1.5);
                      \draw[thick, azure4] (axis cs:1, 0) -- (axis cs:1, 2);
                      \draw[thick, azure4] (axis cs:2, 0) -- (axis cs:2, 2);
                      \draw[thick, azure4] (axis cs:3, 0) -- (axis cs:3, 2);
                  \end{axis}
              \end{tikzpicture}
          \end{figure}

    \item Using integration by parts,
          \begin{align}
              I_1 & = \int_{0}^{4} (4x - x^2) \sin \Big( \frac{m\pi x}{4} \Big)\ \dl x \\
                  & = \Bigg[\frac{4(x^2 - 4x)}{m\pi}\ \cos \Big( \frac{m\pi x}{4}
                  \Big)\Bigg]_0^4 + \int_{0}^{4} \frac{4(4 - 2x)}{m\pi}\ \cos \Big(
              \frac{m\pi x}{4} \Big) \ \dl x                                           \\
                  & = \Bigg[ \frac{16(4 - 2x)}{m^2\pi^2}
                  \sin \Big( \frac{m\pi x}{4} \Big) \Bigg]_0^4 + \int_{0}^{4}
              \frac{32}{m^2\pi^2}\ \sin\Big( \frac{m\pi x}{4} \Big)\ \dl x             \\
                  & = \Bigg[ \frac{128}{m^3\pi^3}\ \cos\Big( \frac{m\pi x}{4}
                  \Big)\Bigg]_4^0 = \color{y_h} \frac{128}{m^3\pi^3}\ [1 - \cos(m\pi)]
          \end{align}
          For the integral in $ y $,
          \begin{align}
              I_2 & = \int_{0}^{2} (2y - y^2) \sin \Big( \frac{n\pi y}{2} \Big)\ \dl y \\
                  & = \Bigg[\frac{2(y^2 - 2y)}{n\pi}\ \cos \Big( \frac{n\pi y}{2}
                  \Big)\Bigg]_0^2 + \int_{0}^{2} \frac{2(2 - 2y)}{n\pi}\ \cos \Big(
              \frac{n\pi y}{2} \Big) \ \dl y                                           \\
                  & = \Bigg[ \frac{4(2 - 2y)}{n^2\pi^2}
                  \sin \Big( \frac{n\pi y}{2} \Big) \Bigg]_0^2 + \int_{0}^{2}
              \frac{8}{n^2\pi^2}\ \sin\Big( \frac{n\pi y}{2} \Big)\ \dl y              \\
                  & = \Bigg[ \frac{16}{n^3\pi^3}\ \cos\Big( \frac{n\pi y}{2}
                  \Big)\Bigg]_2^0 = \color{y_p} \frac{16}{n^3\pi^3}\ [1 - \cos(n\pi)]
          \end{align}

    \item Using the code form Problem $ 10 $, and taking a simple example,
          \begin{align}
              145      & = \color{y_h} 8^2 + 9^2               &
                       & = \Bigl(\frac{m_1}{2}\Bigr)^2 + n_1^2   \\
                       & = \color{y_p} 12^2 + 1^2              &
                       & = \Bigl(\frac{m_2}{2}\Bigr)^2 + n_2^2   \\
              m_1, n_1 & = 16, 3                               &
              m_2, n_2 & = 24, 1
          \end{align}

    \item Minimizing the frequency with respect to the ratio of the sides of the
          rectangle,
          \begin{align}
              \lambda_{11}                  & = c\pi
              \sqrt{\left(\frac{1}{a}\right)^2 +
              \left( \frac{1}{b} \right)^2} &
              b                             & = \frac{K}{a}                 \\
              \diff {\lambda_{11}}{a}       & = 0                         &
              \implies 0                    & = -2a^{-3} + \frac{2a}{K^2}   \\
              K^2                           & = a^4                       &
              a                             & = \sqrt{K} = b
          \end{align}
          Since both sides are equal, the minimum is achieved for a square membrane.

    \item Using the standard result, and the fact that the initial deflection is already
          a term in the double Fourier series,
          \begin{align}
              B_{mn}   & = \begin{dcases}
                               1 & \quad m=6, n=2         \\
                               0 & \quad \text{otherwise}
                           \end{dcases}                   \\
              u(x,y,t) & = \cos\Bigg( \sqrt{\Big( \frac{6}{a} \Big)^2 +
                  \Big( \frac{2}{b} \Big)^2}\ \pi t \Bigg)\ \sin \Big( \frac{6\pi x}{a}
              \Big)\ \sin\Big( \frac{2\pi y}{b} \Big)
          \end{align}

    \item Using Newton's second law,
          \begin{align}
              u_{tt}                                 & = \frac{T}{\rho}\ \nabla^2 u
              + \frac{F_{\text{ext}}}{\rho \Delta A} &
              P                                      & = \frac{F}{\Delta A}         \\
              u_{tt}                                 & = c^2\ \nabla^2 u
              + \frac{P}{\rho}
          \end{align}
\end{enumerate}