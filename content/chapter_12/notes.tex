\chapter{Partial Differential Equations}

\section{Basic Concepts of PDEs}

\begin{description}
    \item[Partial Differential equation] A differential equation where a function of two
        or more variables is differentiated with respect to just one of them. In
        real-world applications, second order PDEs are the most common.

    \item[Order] The order of the highest derivative in the PDE
    \item[Homogeneous PDE] A PDE is homogeneous if all of its terms involve some partial
        derivative of the function.
    \item[Linear PDE] A PDE that is of the first degree in the unknown function and all
        of its partial derivatives.

    \item[Solution] A solution to a PDE in some region $ R $ in space has all partial
        derivatives that appear in the PDE in some domain $ D $ containing $ R $, and
        satisfies the PDE everywhere in $ R $.

    \item[Particular solution] A particular solution to the ODE is obtained by using
        boundary conditions, (specified at two points in $ xy $ space) or using initial
        conditions (where time $ t $ is a variable)

    \item[Superposition] Similar to ODEs, linear homogeneous PDEs also follow the
        superposition rule for solutions.
        \begin{align}
            u & = c_1 u_1 + c_2 u_2
        \end{align}
        is also a solution for some constants $ c_1, c_2 $ provided $ u_1,u_2 $ are
        solutions.

    \item[Solution by direct integration] Very simple PDEs can often be solved by
        direct integration with respect to one variable at a time. This method fails for
        all but the simplest PDEs.
\end{description}

\section{Modeling: Vibrating String, Wave Equation}

\begin{description}
    \item[Physical system] A string stretched along the $ x $ axis with the two ends
        constrained to be at $ (0, 0) $ and $ (L, 0) $. \par
        The string performs small transverse vibrations as a result of being released at
        $ t = 0 $ with some perturbation.

    \item[Solution] Solutions are of the form $ u(x, t) $ which describe the y-coordinate
        of every point on the string  $ x \in [0, L] $ at all times $ t > 0 $

    \item[Assumptions] In order to simplify the system,
        \begin{itemize}
            \item The string is homogeneous and perfectly elastic.
            \item Gravitational force is negligible compared to the tension within the
                  string.
            \item Points on the string only move in the vertical (transverse) direction.
        \end{itemize}

    \item[PDE] Using Newton's second law, and the fact that the net horizontal force
        acting on a small piece of string must be zero,
        \begin{align}
            \diffp[2] ut & = c^2\ \diffp[2] ux & c^2 & = \frac{T}{\rho}
        \end{align}
        The constant $ c^2 $ is a positive constant dependent on the tension in the
        string $ T $ and the mass per unit length of the string $ \rho $.
\end{description}

\section{Solution by Separating Variables, Use of Fourier Series}

\begin{description}
    \item[Boudnary conditions] Since the string is fastened at both ends, the two B.C.
        are,
        \begin{align}
            u(0, t) & = 0 & u(L, t) & = 0 \qquad \forall \ t \geq 0
        \end{align}

    \item[Initial conditions] Since the string can have some initial deflection and
        initial velocity, the two I.C. are,
        \begin{align}
            u(x, 0)                      & = f(x)                           &
            \diffp {u}{t}\Bigg|_{(x, 0)} & = g(x) \qquad \forall x \in[0,L]
        \end{align}

    \item[Separation of variables] The first step in solving the PDE is to assume that
        the final result is the product of two functions each dependent on only one of
        the variables $ x, t $.
        \begin{align}
            u(x, t)      & = F(x) \cdot G(t)           \\
            \diffp[2] ut & = F(x) \cdot \diffp[2] Gt &
            \diffp[2] ux & = G(t) \cdot \diffp[2] Fx
        \end{align}
        Substituting into the wave equation leads to one ODE each in $ x $ and $ t $,
        which can be solved separately to obtain $ F(x) $ and $ G(t) $.
        \begin{align}
            \diff[2] Fx & = kF      &
            \diff[2] Gt & = c^2\ kG
        \end{align}

    \item[Satisfying boundary conditions] To make the solution satisfy the boundary
        conditions,
        \begin{align}
            u(0, t) = 0 & \implies F(0) = 0 &
            u(L, t) = 0 & \implies F(L) = 0
        \end{align}
        This places a constraint on $ k $ in the system of ODEs above.
        \begin{align}
            k           & = -p^2                                 &
            F(x)        & = A\ \cos(px) + B\ \sin(px)              \\
            A           & = 0                                    &
            B\ \sin(pL) & = 0                                      \\
            F_n(x)      & = \sin\left( \frac{n\pi}{L}\ x \right)
        \end{align}
        This is an infinite set of solutions for natural numbers $ n $.
        Using this set of values for $ k $, the ODE for $ G $ is solved to yield
        \begin{align}
            \lambda_n & = cp = \frac{cn\pi}{L}                              &
            G_n(t)    & = B_n\ \cos(\lambda_n t) + B_n^*\ \sin(\lambda_n t)
        \end{align}
        Here, the set of eigenvalues $ \{\lambda_n\} $ is called the spectrum and each
        corresponding solution is called an eigenfunction $ u_n(x, t) $.

    \item[Normal mode] The points on a string that do not move in addition to the
        endpoints are called nodes. These correspond to,
        \begin{align}
            \sin\left( \frac{n\pi}{L}\ x \right) & = 0                                &
            x                                    & = \Bigg\{\frac{L}{n}, \frac{2L}{n}
            , \dots , \frac{(n-1)L}{n}\Bigg\}
        \end{align}
        The motion $ u_n $ having $ (n-1) $ nodes is called a normal mode of the system.
        \par The fundamental mode corresponds to $ n = 1 $.

    \item[Fourier series solution] In order to satisfy the initial conditions, the
        infinite set of solutions $ \{u_n\} $ can be expanded in terms of Fourier
        integrals. \par
        Given an initial postiion $ f(x) $,
        \begin{align}
            u(x, 0)        & = \iser[n]{0}\ B_n\ \sin\left( \frac{n\pi}{L}\ x
            \right) = f(x) &
            x              & \in [0, L]                                       \\
            B_n            & = \frac{2}{L}\ \int_{0}^{L} f(x)\
            \sin\left( \frac{n\pi}{L}\ x
            \right)\ \dl x &
            n              & = \{1,2,3,\dots\}
        \end{align}
        Similarly, given an initial velocity $ g(x) $,
        \begin{align}
            \diffp ut\Bigg|_{t=0} & = \iser[n]{0}\ (B_n^* \lambda_n)
            \ \sin\left( \frac{n\pi}{L}\ x
            \right) = g(x)        &
            x                     & \in [0, L]                            \\
            B_n^*                 & = \frac{2}{cn\pi}\ \int_{0}^{L} g(x)\
            \sin\left( \frac{n\pi}{L}\ x
            \right)\ \dl x        &
            n                     & = \{1,2,3,\dots\}
        \end{align}

    \item[Closed form solution] For the special case of zero initial velocity, which
        makes $ B_n^* = 0 $. Further,
        \begin{align}
            u(x, t) & = \frac{f^*(x - ct) + f^*(x + ct)}{2}
        \end{align}
        Here, $ f^* $ is the odd periodic extension of $ f(x) $ with period $ 2L $.

    \item[Generalized solution] Suppose the initial deflection is such that $ f'(x) $ and
        $ f''(x) $ are only piecewise continuous, or that these one-sided derivatives are
        not zero. \par
        Then, there will be finitely many points at which the $ \difcp[2]{}{x} $ terms in
        the PDE do not exist. The wave equation is still a solution at all the remaining
        points in the domain.

    \item[Physical interpretation] The standing wave that is set up in the string can
        be considered a superposition of waves with the same waveform traveling in
        opposite directions at equal speed ($ c $ above).
\end{description}

\section{D'Alembert's Solution of the Wave Equation}

\begin{description}
    \item[D'Alembert's solution] By introducing two new variables,
        \begin{align}
            v                         & = x + ct                                &
            w                         & = x - ct                                  \\
            u                         & \to u(v, w)                               \\
            \difcp[2] ux              & = \difcp[2] uv + \difcp[2] uw
            + 2\ \difcp{u}{v, w}      &
            \difcp[2] ut              & = c^2\ \Big[\difcp[2] uv + \difcp[2] uw
            - 2\ \difcp{u}{v, w}\Big] &
        \end{align}
        Substituting into the wave equation yields,
        \begin{align}
            \difcp{u}{v, w} & = 0                             \\
            u(x, t)         & = \phi (x + ct) + \psi (x - ct)
        \end{align}
        Differentiating $ v, w $  with respect to $ (x + ct), (x - ct) $ respectively,
        using the chain rule,
        \begin{align}
            u(x, 0)                  & = \phi(x) + \psi(x)  = f(x)         &
                                     & \text{initial deflection}             \\
            \difcp ut\Bigg|_{(x, 0)} & = c\ \phi'(x) - c\ \psi'(x)  = g(x) &
                                     & \text{initial velocity}
        \end{align}
        Solving this system yields the earlier relation for $ u(x, t) $,
        \begin{align}
            u(x, t) & = \frac{f(x + ct) + f(x - ct)}{2} + \frac{1}{2c}
            \ \int_{x-ct}^{x + ct}\ g(s)\ \dl s
        \end{align}
        Thus, the initial deflection and velocity uniquely determine the solution of
        the wave equation.

    \item[Method of characteristics] A method of approaching quasilinear PDEs of the
        form,
        \begin{align}
            A\ \difcp[2] ux + 2B\ \difcp{u}{x, y} + C\ \difcp[2] uy
             & = F(u, x, y, \difcp ux, \difcp uy)
        \end{align}
        This PDE is linear in the highest order derivatives (here second order).
        \begin{table}[ht]
            \centering
            \SetTblrInner{rowsep=0.5em}
            \begin{tblr}{colspec={Q[l]|Q[l]|Q[l]}, colsep = 2em}
                \textbf{Type} & \textbf{Condition} & \textbf{Example} \\ \hline[dotted]
                Hyperbolic    & $AC - B^2 < 0$     & Wave equation    \\
                Parabolic     & $AC - B^2 = 0$     & Heat equation    \\
                Elliptic      & $AC - B^2 > 0$     & Laplace equation \\ \hline
            \end{tblr}
        \end{table}
        Here, the conversion $ y \to t $ is accomplished by using $ y = ct $
        for example. \par
        $ A, B, C $ may be functions of mixed type, which means they are of different
        types in different regions of the $ xt $ plane.

    \item[Characteristic equation] The characteristic equation of the above PDE is,
        \begin{align}
            y' & = \diff yx & A\ y'^2 - 2B\ y' + C & = 0
        \end{align}
        The solutions of this equation are expressed as level curves in $ (x, y) $,
        \begin{align}
            \Phi(x, y) & = c_1 & \Psi(x, y) & = c_2
        \end{align}
        Now, the new variables $ (v, w) $ can be arrived at from the existing $ (x, y) $
        using the transformations, \par
        \begin{table}[ht]
            \centering
            \SetTblrInner{rowsep=0.75em}
            \begin{tblr}{colspec={Q[l]|Q[l, $$]|Q[l, $$]|Q[l]}, colsep = 2em}
                \textbf{Type}          & \vec{v}                             &
                \vec{w}                & \text{\textbf{Normal Form}}           \\
                \hline[dotted]
                Hyperbolic             & \Phi                                &
                \Psi                   & $\difcp{u}{v, w} = F_1$               \\
                Parabolic              & x                                   &
                \Psi = \Phi            & $\difcp[2] uw = F_2$                  \\
                Elliptic               & \frac{\Phi + \Psi}{2}               &
                \frac{\Phi - \Psi}{2i} & $\difcp[2] uv + \difcp[2] uw = F_3$   \\ \hline
            \end{tblr}
        \end{table}
        For example, the general form of a hyperbolic PDE resembles D'Alembert's
        solution. The wave equation happens to lead to the very special case of
        $ F_1(v, w, u, \difcp uv, \difcp uw) = 0 $
\end{description}

\section{Modeling: Heat Flow from a Body in Space. Heat Equation}

\begin{description}
    \item[Assumptions] In order to simplify the mode,
        \begin{itemize}
            \item The specific heat $ \sigma $ and density $ \rho $ of the material are
                  constant. There are no heat sources or sinks within the body.
            \item Heat flow is proportional to the gradient of the temperature in the
                  direction of decreasing temperature. (empirical result)
                  \begin{align}
                      \vec{v} & = -K\ \nabla u
                  \end{align}
                  Here, $ u(x, y, z, t) $ is the temperature, and $\vec{v} $ is the
                  heat flow.
            \item The thermal conductivity $ K $ is constant.
        \end{itemize}

    \item[Physical system] Consider a region $ T $ in $ 3d $ space bounded by a surface
        $ R $ such that the divergence theorem applies. With $ \vec{n} $ being the outer
        normal vector,
        \begin{align}
            \vec{v} \dotp \vec{n}\ \Delta A
        \end{align}
        is the amount of heat leaving the region $ T $ through the small surface
        $ \dl A $. (positive sign by convention is heat lost). \par
        The total heat leaving the surface, is
        \begin{align}
            \iint_S (\vec{v} \dotp \vec{n})\ \dl A & =
            -K\ \iiint_T \nabla^2 u\ \dl x\ \dl y\ \dl z
        \end{align}
        Since there are no sources or sinks of heat inside $ T $, this must equal the
        rate of decrease of heat $ H $ in the region $ T $,
        \begin{align}
            -\diffp Ht & = - \iiint_T\ (\rho \sigma)\ \diffp ut\ \dl x\ \dl y\ \dl z
        \end{align}
        Equating these two sides of yields the heat equation.
        \begin{align}
            \diffp ut & = c^2\ \nabla^2 u       &
            c^2       & = \frac{K}{\rho \sigma}
        \end{align}
        This equation also models the diffusion of gases.
\end{description}

\section{Heat Equation: Solution by Fourier Series, Dirichlet Problem}

\begin{description}
    \item[One dimensional heat equation] For the special case of one dimension, with
        heat constrained to flow along the $ x $ axis,
        \begin{align}
            \diffp ut & = c^2\ \diffp[2] ux
        \end{align}

    \item[Separation into ODEs] Similar to the wave equation, the full solution
        $ u(x, t) $ can be represented as the product of functions of $ x $ and $ t $.
        \begin{align}
            u(x, t)   & = F(x) \cdot G(t)                   \\
            F_n(x)    & = a\ \cos(p_n x) + b\ \sin(p_n x) &
            p_n       & = \frac{n\pi}{L}                    \\
            G_n(t)    & = B_n\ \exp(-\lambda_n^2 t)       &
            \lambda_n & = \frac{cn \pi}{L}
        \end{align}

    \item[Special cases] Starting with the boundary conditions where both ends are kept
        at zero temperature, and the initial temperature throughout the bar is $ f(x) $,
        \begin{align}
            u(0, t) & = 0    & u(L, t) & = 0 \\
            u(x, 0) & = f(x)
        \end{align}
        This simplifies the general solution to give,
        \begin{align}
            A      & = 0                                                            &
            F_n(x) & = \sin\left( \frac{n\pi x}{L} \right)                            \\
            G_n(t) & = B_n \exp(-\lambda_n^2 t)                                     &
            B_n    & = \frac{2}{L}\ \int_{0}^{L}\ f(x)\ \sin\left( \frac{n\pi x}{L}
            \right)\ \dl x
        \end{align}
        Each of the terms in the Fourier series for $ f(x) $ corresponds to an eigenvalue
        $ \lambda_n $. Since the exponential decay of heat in time is of the order
        $ \lambda_n^2 \propto n^2 $, the higher order terms decay faster.

    \item[Insulated ends] For the special case where the two ends of the bar are
        insulated so that no heat can flow through them,
        \begin{align}
            \diffp ux \Big|_{(x = 0)} & = \diffp ux \Big|_{(x = L)} = 0     \\
            u(x, t)                   & = A_0 + \iser[n]{1} A_n\ \cos\left(
            \frac{n\pi x}{L} \right)\ \exp(-\lambda_n^2 t)
        \end{align}
        The new constant term corresponds to zero eigenvalue and is the special case of
        constant initial temperature over the entire bar.

    \item[Steady heat problems in 2d] Steady-state problems, where the temperature is
        independent of time, leads to the heat equation reducing to Laplace's equation,
        \begin{align}
            \diffp ut & = 0 & \implies \quad \nabla^2 u & = 0
        \end{align}
        The three kinds of problem based on the boudnary conditions on the boundary
        $ C $ of the region $ R $,
        \begin{itemize}
            \item Dirichlet B.C. - prescribing $ u $ on the boundary
            \item Neumann B.C. - prescribing $ \difcp un $ on the boundary
            \item Robin B.C. - a mix of both
        \end{itemize}
        Here, $ \difcp un $ is the normal derivative of the temperature $ u $. These
        problems are also solved by decomposing $ u(x, y) $ into a product of functions
        in $ x $ and $ y $. \par

    \item[Existence of solutions] This requires that $ f, f' $ are continuous and
        $ f'' $ is at least piecewise continuous on the boundary $ C $
\end{description}


\section{Heat Equation: Modeling Very Long Bars, Solution using Fourier transforms}

\begin{description}
    \item[Infinite bars] When the bar is of infinite length, boundary conditions no
        longer exist. The particular solution is determined solely by the initial
        condition.

    \item[Solution by Fourier integrals] Since the initial condition $ u(x, 0) = f(x) $
        is no longer a periodic function with period $ L $, the general solution becomes,
        \begin{align}
            \diffp ut   & = c^2\ \difcp[2] ux                                   \\
            u(x, t ; p) & = \Big[ A\cos(px) + B\sin(px) \Big]\ \exp(-c^2p^2\ t) \\
        \end{align}
        The Fourier series sum in the earlier section is replaced by a Fourier integral,
        with $ A, B $ also being functions of $ p $,
        \begin{align}
            u(x, t) & = \infint u(x, t; p)\ \dl p                               \\
            u(x, 0) & = \infint \Big[ A\cos(px) + B\sin(px) \Big]\ \dl p = f(x)
        \end{align}
        The spectrum of eigenvalues is now the entire line $ \mathcal{R}^+ $ instead of
        discrete values $ cn\pi/L $ as in the finite case. \par
        The above expression for $ f(x) $ is simply the Fourier sine and cosine integral
        for $ f(x) $.
        \begin{align}
            u(x, t) & = \frac{1}{\pi} \intRL f(v) \Bigg[ \infint
                \exp(-c^2p^2\ t)\cos(px - pv)\ \dl p \Bigg]\ \dl v
        \end{align}

    \item[General solution] Using a special finite integral,
        \begin{align}
            \infint e^{-b^2} \cos(2bs)\ \dl s & = \frac{\sqrt{\pi}}{2}\ e^{-b^2} \\
            u(x, t)                           & = \frac{1}{2c \sqrt{\pi t}}
            \intRL f(v)\ \exp\Bigg[-\frac{(x-v)^2}{4c^2 t}\Bigg]\ \dl v          \\
            u(x, t)                           & = \frac{1}{\sqrt{\pi}} \intRL
            f(x + 2cz\sqrt{t})\ e^{-z^2}\ \dl z
        \end{align}
        The above simplificcation uses the change of variables $ v \to z $. \par
        If $ f(x) $ is bounded for all values of $ x $ and integrable in every finite
        interval, then the above expression satisfies the heat equation.

    \item[Solution using Fourier transforms] Taking the Fourier transform results
        in an ODE in time $ (t) $ which is usually easier to solve.
\end{description}

\section{Modeling: Membrane, Two-Dimensional Wave Equation}

\begin{description}
    \item[Assumptions] The assumptions are all analogous to the one dimensional wave
        equation for a stretched string.
        \begin{itemize}
            \item The membrane is homogeneous. Its mass per unit area $ (\rho) $ is
                  constant.
            \item The tension $ (T) $ in the membrane is the same at all points and does
                  not change during the motion.
            \item The deflection in the membrane is small and only in the
                  transverse direction.
        \end{itemize}

    \item[PDE using Newton's law] Equating the net vertical force acting on a small
        area of the membrane to its acceleration yields,
        \begin{align}
            \diffp[2] ut & = c^2\ \Bigg( \diffp[2] ux + \diffp[2] uy \Bigg)
            = c^2\ \nabla^2 u                                               \\
            c^2          & = \frac{T}{\rho}
        \end{align}
\end{description}

\section{Rectangular Membrane. Double Fourier Series}

\begin{description}
    \item[Statement of the problem] The full problem is stated as,
        \begin{align}
            \diffp[2] ut & = c^2\ \nabla^2 u &  & \text{PDE}                  \\
            u            & = 0               &  & \text{boundary condition}   \\
            u(x, y, 0)   & = f(x, y)         &  & \text{initial displacement} \\
            u_t(x, y, 0) & = g(x, y)         &  & \text{initial velocity}
        \end{align}
        Here, the boundary condition is the simplest possible case, where the
        displacement is identically zero on the boundary for all time.

    \item[Separation of variables] Let the full solution be a product of time and space
        dependent functions $ G, F $,
        \begin{align}
            u(x, t)         & = F(x, y) \cdot G(t)                              \\
            F\ \diffp[2] Gt & = c^2G\ \Bigg[ \diffp[2] Fx + \diffp[2] Fy \Bigg] \\
            \ddot{G}        & = -\lambda^2 G                                    \\
            \lambda         & = c\nu
        \end{align}

    \item[Two dimensional Helmholtz equation] A further separation of the space dependent
        term in the solution is possible, using
        \begin{align}
            F(x, y)                     & = H(x) \cdot Q(y)         &
            \difcp[2] Fx + \difcp[2] Fy & = -\nu^2 F                  \\
            \diff[2] Hx                 & = -k^2H                   &
            \diff[2] Qy                 & = (k^2 - \nu^2)Q = -p^2 Q
        \end{align}

    \item[Rectangular membrane] Consider a particular solution for the simple case
        of a rectangular membrane of width $ a $ and height $ b $.
        \begin{align}
            H(x)        & = A\cos(kx) + B \sin(kx) &
            Q(y)        & = C\cos(py) + D \sin(py)   \\
            H(x)\Big|_0 & = H(x)\Big|_a = 0        &
            Q(y)\Big|_0 & = Q(y)\Big|_b = 0
        \end{align}
        Each of the functions $ H, Q $ are identical to the corresponding one dimensional
        string case, resulting in,
        \begin{align}
            F_{mn} & = H_m(x) \cdot Q_n(y) = \sin\Bigg( \frac{m\pi x}{a} \Bigg)
            \sin\Bigg( \frac{n\pi y}{b}\Bigg)
        \end{align}
        for positive integers $ m, n $.

    \item[Eigenvalues and eigenfunctions] Since the eigenvalues of the PDE are
        $ \lambda = c\nu $,
        \begin{align}
            \lambda_{mn} & = c\sqrt{k^2 + p^2} = c\pi
            \sqrt{\Bigg(\frac{m}{a}\Bigg)^2 + \Bigg(\frac{y}{b}\Bigg)^2}               \\
            G_{mn}(t)    & = B_{mn} \cos(\lambda_{mn} t) + B^*_{mn}\sin(\lambda_{mn}t) \\
            u_{mn}(x, t) & = G_{mn}(t) \cdot F_{mn}(x, y)
        \end{align}
        This is a general solution to the PDE apart from matching the initial
        displacement $ f(x, y) $ and initial velocity $ g(x, y) $. \par
        There are many possible solutions corresponding to the same eigenvalue
        $ \lambda $, which physically corresponds to the membrane exhibiting different
        modes.

    \item[Double Fourier series] Similar to the one dimensional fourier series for an
        elastic vibrating string,
        \begin{align}
            u(x, y, t) & = \iser[m]{0} \iser[n]{0} u_{mn}(x, y, t)               \\
            u(x, y, 0) & = \iser[m]{0} \iser[n]{0} B_{mn}\
            \sin\Bigg( \frac{m\pi x}{a} \Bigg) \sin\Bigg( \frac{n\pi y}{b}\Bigg) \\
            B_{mn}     & = \frac{4}{ab} \int_{0}^{b} \int_{0}^{a} f(x, y)
            \sin\Bigg( \frac{m\pi x}{a} \Bigg) \sin\Bigg( \frac{n\pi y}{b}\Bigg)
            \ \dl x\ \dl y
        \end{align}
        This requires that the initial displacement $ f(x, y) $ is continuous in $ R $.
        Further $ f_x, f_y, f_{xy} $ are also continuous in $ R $. \par
        The initial velocity $ g(x, y) $ can be obtained by a similar procedure acting
        on the $ B^*_{mn} $.
\end{description}

\section{Circular Membrane, Fourier-Bessel Series}

\begin{description}
    \item[Laplacian in Polar coordinates] Using the chain rule in partial
        differentiation,
        \begin{align}
            x          & = r \cos \theta               &
            y          & = r \sin \theta                 \\
            u_x        & = u_r r_x + u_\theta \theta_x &
            u_y        & = u_r r_y + u_\theta \theta_y   \\
            \nabla^2 u & = u_{xx} + u_{yy}
        \end{align}
        Performing the differentiation twice and substituting yields,
        \begin{align}
            \nabla^2 u & = \diffp[2] ur + \frac{1}{r}\ \diffp ur + \frac{1}{r^2}
            \ \diffp[2] {u}{\theta}
        \end{align}

    \item[Circular membrane] A circular flat membrane which is elastic and offers no
        resistance to bending can be described using the PDE,
        \begin{align}
            \diffp[2] ut & = c^2\ \nabla^2 u & c^2       & = \frac{T}{\rho} \\
            u(R, t)      & = 0               & \forall\  & t \geq 0         \\
            u(r, 0)      & = f(r)            & u_t(r, 0) & = g(r)           \\
        \end{align}
        The above PDE assumes radial symmetry in the deflection.

    \item[Separation of variables] Once again, the radial and time components of the
        solutions can be separated to yield two ODEs,
        \begin{align}
            u(r, t)                                    & = W(r) \cdot G(t)   \\
            \diff[2] Gt                                & = -\lambda^2 G    &
            \lambda                                    & = ck                \\
            \diff[2] Wr + \frac{1}{r}\ \diff Wr + k^2W & = 0
        \end{align}

    \item[Bessel ODE] The ODE for $ W(r) $ can be reformulated into Bessel's ODE with
        zero parameter using the substitution,
        \begin{align}
            s                                       & = kr &
            s^2\ \diff[2] Ws + s\ \diff Ws + s^2\ W & = 0
        \end{align}
        Only Bessel functions of the first kind are useful solutions to this equation,
        since the solution has to remain bounded at infinite $ r $.

    \item[Full solution to PDE] In order to satisfy the boundary conditions, the
        constraint on $ k $ is,
        \begin{align}
            W(r)\Big|_{R} & = 0                  & \implies J_0(kr) \Big|_{R} & = 0 \\
            k             & = \frac{\alpha_m}{R}
        \end{align}
        Here, $ \alpha_m $ is a positive zero of $ J_0 $. Since this is an infinite set,
        the solution is also an infinite series sum. \par
        Incorporating the initial deflection and velocity, the full solution is,
        \begin{align}
            u(r, t)   & = \iser{1} \Big[ A_m \cos(\lambda_m t) + B_m \sin(\lambda_m t)
            \Big]\ J_0 \Big( \frac{\alpha_m\ r}{R} \Big)                               \\
            \lambda_m & = \frac{c\ \alpha_m}{R}
        \end{align}

    \item[Nodal line] Similar to the rectangular membrane, nodal lines here are
        circles,
        \begin{align}
            r & = \frac{\alpha_k}{\alpha_m}\ R & k & < m
        \end{align}
        Thus, the $ m^{\text{th}} $ normal mode has $ (m-1) $ modal lines.

    \item[Initial conditions] The Fourier-Bessel series is used to find the
        coefficients of each term in the full solution using,
        \begin{align}
            u(r, 0) & = f(r) = \iser{0} A_m\  J_0 \Big( \frac{\alpha_m\ r}{R} \Big) \\
            A_m     & = \frac{2}{R^2\ J_1^2(\alpha_m)}\ \int_{0}^{R}
            r\ f(r)\ J_0 \Big( \frac{\alpha_m\ r}{R} \Big)\ \dl r
        \end{align}
        The abvoe forumla is usable if $ f(r) $ is differentiable in the interval
        $ r \in [0, R] $.
\end{description}

\section{Laplace's Equation in Cylindrical and Spherical Coordinates}

\begin{description}
    \item[Potential theory] In physics, a potential is a scalar function whose gradient
        happens to be a conservative vector field. Potential theory refers to solutions
        of Laplace's equation $ \nabla^2 u  = 0$

    \item[Gravitational potential] As an example, consider the potential $ u $ at a point
        $ P $, caused by a point mass located at $ P_0 $,
        \begin{align}
            u & = \frac{c}{r} = \frac{c}{\sqrt{(x - x_0)^2 + (y - y_0)^2 + (z - z_0)^2}}
        \end{align}
        Next, the potential due to a mass in region $ T $ with density
        $ \rho(x_0, y_0, z_0) $ at a point outisde $ T $ is given by
        \begin{align}
            u(x, y, z) & = k\iiint\ \frac{\rho(x_0, y_0, z_0)}{r}\ \dl x_0\ \dl y_0
            \ \dl z_0
        \end{align}

    \item[Cylindrical coordinates] Since this involves adding an independent $ z $
        coordinate, the Laplacian adds a corresponding $ u_{zz} $ term.
        \begin{align}
            \nabla^2 u & = u_{rr} + \frac{1}{r}\ u_r + \frac{1}{r^2}\ u_{\theta\theta}
            + u_{zz}
        \end{align}

    \item[Spherical coordinates] Using the chain rule for multiplication,
        \begin{align}
            x          & = r\cos\theta\sin\phi \qquad y = r\sin\theta\sin\phi
            \qquad z = r\cos\phi                                              \\
            \nabla^2 u & = \frac{1}{r^2}\ \Bigg[ \diffp*{\ (r^2\ u_r)}{r}
                + \frac{1}{\sin\phi}\ \diffp*{\ (\sin\phi\ u_\phi)}{\phi}
                + \frac{1}{\sin^2\phi}\ \diffp[2]u\theta\Bigg]
        \end{align}
        Here, the azimuthal coordinate $ \phi \in [0, \pi] $ and the polar coordinate
        $ \theta \in [0, 2\pi] $

    \item[Simple Dirichlet example] Consider the common physical system represented
        by the Dirichlet BVP
        \begin{align}
            \nabla^2 u                     & = 0 & u(R, \phi) & = f(\phi) \\
            \lim_{r \to \infty} u(r, \phi) & = 0
        \end{align}
        The BC, and therefore the solutions are independent of $ \theta $. The potential
        decaying to zero at long distances is a general feature of physically plausible
        potentials.

    \item[Separation of variables] Separating into functiond depending on $ r, \phi $,
        \begin{align}
            u(r, \phi)                 & = G(r) \cdot H(\phi)                    \\
            r^2\ \diff[2] Gr
            + 2r\ \diff Gr - n(n+1)\ G & = 0                                     \\
            w                          & = \cos \phi \qquad (1-w^2) = \sin^2\phi \\
            (1-w^2)\ \diff[2]Hw
            - 2w\ \diff Hw + n(n+1)\ H & = 0
        \end{align}
        The solutions by equating each side of the PDE to $ k = n(n+1) $, are,
        \begin{align}
            G_n(r) & = A_n r^n + \frac{B_n}{r^{n+1}}                          \\
            H_n(r) & = P_n(\cos\phi) \qquad \forall\ n \in \{0, 1, 2, \dots\}
        \end{align}

    \item[Fourier-Legendre series] For problems inside the sphere, where $ r=0 $
        needs to exist, only the $ A_n $ part of the solution matters,
        \begin{align}
            u(r, \phi) & = \iser{0} A_n\ r^n\ P_n(\cos\phi)                          \\
            A_n        & = \frac{2n+1}{2R^n}\ \int_{0}^{\pi} f(\phi)\ P_n(\cos \phi)
            \ \sin\phi\ \dl \phi
        \end{align}
        Conversely, for the exterior of the sphere, where the potential at long distances
        needs to decay to zero, only the $ B_n $ part of the solution matters.
        \begin{align}
            u(r, \phi) & = \iser{0} \frac{B_n}{r^{n+1}}\ P_n(\cos\phi) \\
            A_n        & = \frac{2n+1}{2}\ R^{n+1}
            \ \int_{0}^{\pi} f(\phi)\ P_n(\cos \phi) \ \sin\phi\ \dl \phi
        \end{align}
\end{description}

\section{Solution of PDEs by Laplace Transforms}

\begin{description}
    \item[Lapalcian applied to PDEs] The general procedure is as follows,
        \begin{itemize}
            \item Incorporate the I.C. when taking the Laplace transform w.r.t. time
                  $ (t) $.
            \item Since time is replaced by $ s $, the remaining equation is an ODE in
                  $ x $ with $ s $ being a parameter.
            \item Use the Laplace transform of the B.C. in order to determine the
                  constants of integration.
            \item Apply inverse Laplace transform to recover the solution in terms of
                  position and time.
        \end{itemize}
\end{description}