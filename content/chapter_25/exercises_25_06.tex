\section{Acceptance Sampling}

\begin{enumerate}
    \item The specific values are,
          \begin{align}
              \theta & = 0.01 & P(A\ ;\ \theta) & = 0.9825 \\
              \theta & = 0.02 & P(A\ ;\ \theta) & = 0.9384 \\
              \theta & = 0.1  & P(A\ ;\ \theta) & = 0.4060
          \end{align}
          \begin{figure}[H]
              \centering
              \pgfplotstableread[col sep=comma]{./tables/table_25_06_01.csv}
              \anitableone
              \begin{tikzpicture}
                  \begin{axis}[title = {OC Curve}, xlabel = $ \theta $,
                          ylabel = $ P(\theta) $, grid = both,Ani]
                      \addplot[GraphSmooth, y_h, mark=*, mark options =
                              {mark size = 1pt, color = black}]
                      table[x index = 0, y index = 1] {\anitableone};
                  \end{axis}
              \end{tikzpicture}
          \end{figure}

    \item Changing the sample size to $ n = 50 $,
          \begin{align}
              \theta & = 0.01 & P(A\ ;\ \theta) & = 0.9098 \\
              \theta & = 0.02 & P(A\ ;\ \theta) & = 0.7357 \\
              \theta & = 0.1  & P(A\ ;\ \theta) & = 0.0404
          \end{align}
          \begin{figure}[H]
              \centering
              \pgfplotstableread[col sep=comma]{./tables/table_25_06_02.csv}
              \anitabletwo
              \pgfplotstableread[col sep=comma]{./tables/table_25_06_01.csv}
              \anitableone
              \begin{tikzpicture}
                  \begin{axis}[title = {OC Curve}, xlabel = $ \theta $,
                          ylabel = $ P(\theta) $, grid = both,Ani, xmax = 0.5]
                      \addplot[GraphSmooth, y_p]
                      table[x index = 0, y index = 1] {\anitabletwo};
                      \addplot[GraphSmooth, y_h]
                      table[x index = 0, y index = 1] {\anitableone};
                      \addlegendentry{n = 50}
                      \addlegendentry{n = 20}
                  \end{axis}
              \end{tikzpicture}
          \end{figure}
          The effect is to make the OC curve decay to zero much faster. This is
          reasonable, since it is much more probable that a sample contains a defective
          item, with larger $ n $, keeping $ \theta $ constant.

    \item Changing the acceptance limit to $ c = 0 $,
          \begin{align}
              \theta & = 0.01 & P(A\ ;\ \theta) & = 0.8187 \\
              \theta & = 0.02 & P(A\ ;\ \theta) & = 0.6703 \\
              \theta & = 0.1  & P(A\ ;\ \theta) & = 0.1353
          \end{align}
          \begin{figure}[H]
              \centering
              \pgfplotstableread[col sep=comma]{./tables/table_25_06_03.csv}
              \anitablethree
              \pgfplotstableread[col sep=comma]{./tables/table_25_06_01.csv}
              \anitableone
              \begin{tikzpicture}
                  \begin{axis}[title = {OC Curve}, xlabel = $ \theta $,
                          ylabel = $ P(\theta) $, grid = both,Ani, xmax = 0.5]
                      \addplot[GraphSmooth, y_p]
                      table[x index = 0, y index = 1] {\anitablethree};
                      \addplot[GraphSmooth, y_h]
                      table[x index = 0, y index = 1] {\anitableone};
                      \addlegendentry{c = 0}
                      \addlegendentry{c = 1}
                  \end{axis}
              \end{tikzpicture}
          \end{figure}
          The effect is to make the OC curve decay to zero much faster. This is
          reasonable, since it is much more probable that a sample contains one defective
          item than zero, keeping $ \theta $ constant.

    \item From the definition of AQL and RQL, the given defective fractions are
          $ \theta_0 $ and $ \theta_1 $ respectively.
          \begin{align}
              \text{Risk}_p & = 1 - P(A\ ;\ \theta_0) = 0.0616 \\
              \text{Risk}_c & = P(A\ ;\ \theta_1) = 0.1991
          \end{align}

    \item Given $ n = 25 $ and $ c = 1 $,
          \begin{align}
              \text{Risk}_p & = 1 - P(A\ ;\ \texttt{AQL}) &
                            & = 0.055
          \end{align}
          \begin{figure}[H]
              \centering
              \pgfplotstableread[col sep=comma]{./tables/table_25_06_05.csv}
              \anitablefive
              \begin{tikzpicture}
                  \begin{axis}[title = {OC Curve}, xlabel = $ \theta $,
                          ylabel = $ P(\theta) $, grid = both,Ani]
                      \addplot[GraphSmooth, y_h, mark=*, mark options =
                              {mark size = 1pt, color = black}]
                      table[x index = 0, y index = 1] {\anitablefive};
                  \end{axis}
              \end{tikzpicture}
          \end{figure}

    \item Given $ n = 25 $ and $ c = 1 $,
          \begin{align}
              \text{Risk}_p & = 1 - P(A\ ;\ \texttt{AQL}) &
                            & = 0.055
          \end{align}
          \begin{figure}[H]
              \centering
              \pgfplotstableread[col sep=comma]{./tables/table_25_06_06.csv}
              \anitablesix
              \begin{tikzpicture}
                  \begin{axis}[title = {AOQ Curve}, xlabel = $ \theta $,
                          ylabel = $ \text{AOQ}(\theta) $, grid = both,Ani,
                          xmax = 0.5]
                      \addplot[only marks, mark=*, mark options =
                              {mark size = 1.5pt, color = y_h}]
                      table[x index = 0, y index = 2] {\anitablesix};
                      \addplot[GraphSmooth, black, thin, domain = 0:1] {x*e^(-25*x)
                          *(1 + 25*x)};
                      \node[GraphNode, fill = y_p] at (axis cs:0.0647, 0.0336){};
                  \end{axis}
              \end{tikzpicture}
          \end{figure}
          Since $ c $ is small, the analytical maximum can be found using differentiation
          \begin{align}
              f(x) & = (25x) \cdot e^{-25x} (1 + 25x)                                  \\
              x^*  & = 0.0647                         & f_{\text{max}} & = \text{AOQL}
              = 0.0336
          \end{align}

    \item From the definition of AQL and RQL, the given defective fractions are
          $ \theta_0 $ and $ \theta_1 $ respectively.
          \begin{align}
              \text{Risk}_p & = 1 - P(A\ ;\ \texttt{AQL}) = 1 - \frac{\binom{2}{0}
              \ \binom{18}{2}}{\binom{20}{2}} = 0.1947                             \\
              \text{Risk}_c & = P(A\ ;\ \texttt{RQL}) = \frac{\binom{12}{0}
                  \ \binom{8}{2}}{\binom{20}{2}} = 0.1474
          \end{align}

    \item Comparing $ c = 3 $ and $ c = 2 $, the probabilties go down as the number of
          objects being sampled goes up, for the same $ \theta $.
          \begin{table}[H]
              \centering
              \begin{tblr}{colspec = {l|Q[r,$$]|[dotted]Q[r,$$]},
                  colsep = 1.2em}
                  $ \theta $ & c=2    & c=3    \\ \hline
                  0.1        & 0.8053 & 0.7158 \\
                  0.2        & 0.6316 & 0.4912 \\
              \end{tblr}
          \end{table}
          \begin{figure}[H]
              \centering
              \pgfplotstableread[col sep=comma]{./tables/table_25_06_08_a.csv}
              \anitableeighta
              \pgfplotstableread[col sep=comma]{./tables/table_25_06_08_b.csv}
              \anitableeightb
              \begin{tikzpicture}
                  \begin{axis}[title = {OC Curve}, xlabel = $ \theta $,
                          ylabel = $ \text{P}(\theta) $, grid = both,Ani]
                      \addplot[GraphSmooth, y_h]
                      table[x index = 0, y index = 1] {\anitableeighta};
                      \addplot[GraphSmooth, y_p]
                      table[x index = 0, y index = 1] {\anitableeightb};
                      \addlegendentry{$ n=2 $};
                      \addlegendentry{$ n=3 $};
                  \end{axis}
              \end{tikzpicture}
          \end{figure}

    \item Using the binomial distribution with the defective probability being
          $ \theta $,
          \begin{align}
              P(A\ ;\ \theta) & = \sum_{x=0}^{1} \binom{n}{x}\ (1-\theta)^{n-x}
              \cdot \theta^x                                                    \\
                              & = (1-\theta)^n + n\theta\ (1-\theta)^{n-1}
          \end{align}
          \begin{figure}[H]
              \centering
              \begin{tikzpicture}
                  \begin{axis}[
                          title = {OC curve using binomial distribution},
                          grid = both, Ani, xlabel = {$ \theta $},
                          ylabel = {$ P(\theta) $}, colormap/jet, colorbar,
                          cycle list = {[samples of colormap = 4]}]
                      \foreach \k in {2,...,5}
                          {
                              \edef\temp{%
                                  \noexpand \addplot+[very thick, samples = 100,
                                      domain = 0:1, point meta = \k]
                                  {(1-x)^(\k) + \k*x*(1-x)^(\k-1)};
                              }\temp
                          }
                  \end{axis}
              \end{tikzpicture}
          \end{figure}

    \item Using the formula in Problem 9, with $ n=2 $
          \begin{align}
              P(A\ ;\ \theta) & = \sum_{x=0}^{0} \binom{n}{x}\ (1-\theta)^{n-x}
              \cdot \theta^x  &
                              & = (1-\theta)^2
          \end{align}
          \begin{figure}[H]
              \centering
              \pgfplotstableread[col sep=comma]{./tables/table_25_06_08_a.csv}
              \anitableeighta
              \pgfplotstableread[col sep=comma]{./tables/table_25_06_10.csv}
              \anitableten
              \begin{tikzpicture}
                  \begin{axis}[title = {OC Curve}, xlabel = $ \theta $,
                          ylabel = $ \text{P}(\theta) $, grid = both,Ani]
                      \addplot[GraphSmooth, y_h]
                      table[x index = 0, y index = 1] {\anitableeighta};
                      \addplot[GraphSmooth, y_s]
                      table[x index = 0, y index = 1] {\anitableten};
                      \addlegendentry{Hypergeometric};
                      \addlegendentry{Binomial};
                  \end{axis}
              \end{tikzpicture}
          \end{figure}
          The approximation is very close to the accurate values from the hypergeometric
          PDF.

    \item Given $ c = 1,\ n = 3$ and using the binomial distribution,
          \begin{align}
              P(A\ ;\ 0.5) & = \binom{3}{0} \cdot (1-0.5)^3 \cdot 0.5^0
              + \binom{3}{1} \cdot (1-0.5)^2 \cdot 0.5^1 = \frac{1}{2}
          \end{align}
          \begin{figure}[H]
              \centering
              \pgfplotstableread[col sep=comma]{./tables/table_25_06_11.csv}
              \anitableeleven
              \begin{tikzpicture}
                  \begin{axis}[title = {OC Curve}, xlabel = $ \theta $,
                          ylabel = $ \text{P}(\theta) $, grid = both,Ani]
                      \addplot[GraphSmooth, y_h]
                      table[x index = 0, y index = 1] {\anitableeleven};
                      \node[GraphNode, fill = y_p] at (axis cs:0.5, 0.5){};
                  \end{axis}
              \end{tikzpicture}
          \end{figure}

    \item Using the binomial distribution, with $ n = 100,\ p = 0.05 $
          \begin{align}
              \text{Risk}_p & = 1 - P(A\ ;\ \texttt{AQL}) = 1 -
              \sum_{x=0}^{c} \binom{n}{x} q^{n-x}\ p^x                       \\
              1 - 0.02      & = \sum_{x=0}^{c} \binom{100}{x} (0.95)^{100-x}
              \ (0.05)^x = 0.98
          \end{align}
          Using the normal approximation to the binomial distribution,
          \begin{align}
              Z & \sim \frac{X - np}{\sqrt{npq}}
          \end{align}
          Since the producer's risk has to be $ 2\% $ or larger, the smallest
          possible value of $ c $ is 9

    \item The consumer's risk formula for the data in Problem $ 12 $  with
          $ \texttt{RQL}  = 0.12 $ is,
          \begin{align}
              \text{Risk}_c & = P(A\ ;\ \texttt{RQL}) = \sum_{x=0}^{9} \binom{100}{x}
              (0.12)^x\ (0.88)^{100-x}                                                \\
                            & = 0.2256
          \end{align}

    \item Given $ c = 1,\ n = 20$ and using the Poisson distribution,
          \begin{figure}[H]
              \centering
              \pgfplotstableread[col sep=comma]{./tables/table_25_06_14.csv}
              \anitablefourteen
              \begin{tikzpicture}
                  \begin{axis}[title = {OC Curve}, xlabel = $ \theta $,
                          ylabel = $ \text{P}(\theta) $, grid = both,Ani,
                          xmax = 0.5, width = 8cm]
                      \addplot[only marks, mark=*, mark options =
                              {mark size = 1.5pt, color = y_h}]
                      table[x index = 0, y index = 1] {\anitablefourteen};
                      \addplot[GraphSmooth, black, thin, domain = 0:1] {e^(-20*x)
                          *(1 + 20*x)};
                  \end{axis}
              \end{tikzpicture}
              \begin{tikzpicture}
                  \begin{axis}[title = {AOQ Curve}, xlabel = $ \theta $,
                          ylabel = $ \text{AOQ}(\theta) $, grid = both,Ani,
                          xmax = 0.5, width = 8cm, ymax = 0.05]
                      \addplot[only marks, mark=*, mark options =
                              {mark size = 1.5pt, color = y_p}]
                      table[x index = 0, y index = 2] {\anitablefourteen};
                      \addplot[GraphSmooth, black, thin, domain = 0:1] {x*e^(-20*x)
                          *(1 + 20*x)};
                      \node[GraphNode, fill = black,
                      label={45:{$ \texttt{AOQL} = 0.042 $}}]
                      at (axis cs:0.081, 0.042){};
                  \end{axis}
              \end{tikzpicture}
          \end{figure}

    \item Given $ c = 0,\ n = 5$ and using the Poisson distribution,
          \begin{figure}[H]
              \centering
              \pgfplotstableread[col sep=comma]{./tables/table_25_06_15.csv}
              \anitablefifteen
              \begin{tikzpicture}
                  \begin{axis}[title = {OC Curve}, xlabel = $ \theta $,
                          ylabel = $ \text{P}(\theta) $, grid = both,Ani
                          , width = 8cm]
                      \addplot[only marks, mark=*, mark options =
                              {mark size = 1.5pt, color = y_h}]
                      table[x index = 0, y index = 1] {\anitablefifteen};
                      \addplot[GraphSmooth, black, thin, domain = 0:1] {e^(-5*x)};
                  \end{axis}
              \end{tikzpicture}
              \begin{tikzpicture}
                  \begin{axis}[title = {AOQ Curve}, xlabel = $ \theta $,
                          ylabel = $ \text{AOQ}(\theta) $, grid = both,Ani
                          , width = 8cm, ymax = 0.085]
                      \addplot[only marks, mark=*, mark options =
                              {mark size = 1.5pt, color = y_p}]
                      table[x index = 0, y index = 2] {\anitablefifteen};
                      \addplot[GraphSmooth, black, thin, domain = 0:1] {x*e^(-5*x)};
                      \node[GraphNode, fill = black!0,
                      label={45:{$ \texttt{AOQL} = 0.073 $}}]
                      at (axis cs:0.2, 0.073){};
                  \end{axis}
              \end{tikzpicture}
          \end{figure}
\end{enumerate}