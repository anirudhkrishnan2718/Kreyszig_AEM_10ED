\section{Heat Problems}

\begin{enumerate}
    \item Proceeding directly,
          \begin{align}
              T(x, 0) & = 20                 & T(x, d) & = 100 \\
              T(x, y) & = 20 + \frac{80y}{d}
          \end{align}
          Using the result from Example $ 1 $,
          \begin{align}
              F^*(w)  & = 20 + \frac{80w}{d} w         &
              w       & = -\i z                          \\
              F(z)    & = 20 + \frac{80}{d}(-\i x + y) &
              T(x, y) & = 20 + \frac{80y}{d}
          \end{align}
          The results match. The conformal mapping was a rotation by $ -\pi/2 $.

    \item Using the conformal mapping,
          \begin{align}
              w                 & = e^{\i \pi/4}\ z                      &
              u + \i\ v         & = \frac{(x-y) + \i\ (x + y)}{\sqrt{2}}   \\
              T^*(-\sqrt{8}, v) & = 40, \quad T^*(\sqrt{8}, v) = -20     &
              T^*(u, v)         & = au + b
          \end{align}
          Using the boundary conditions and then back-transforming to the $ z $ plane,
          \begin{align}
              T^*(u, v) & = 10 - \frac{15u}{\sqrt{2}} &
              T(x, y)   & = 10 + 7.5(y-x)
          \end{align}

    \item Plotting the isotherms and heat flow lines,
          \begin{figure}[H]
              \centering
              \begin{tikzpicture}[declare function = {
                              a = -167.05; r = 0.2679;}]
                  \begin{axis}[title = Example 2,
                          xlabel = $ x $, ylabel = $ y $,
                          width = 8cm,Ani, enlargelimits = true,
                          axis equal, view = {0}{90}, grid = both,
                          domain = -101:101, restrict y to domain = -101:101,
                          colormap/jet, colorbar horizontal
                      ]
                      \addplot3 [thick,contour gnuplot={
                                  levels = {500,400,300,200,100,60},
                                  labels = false}, samples = 100]
                      {500 - 95.54 * 0.5 * ln(x^2 + y^2)};
                      \foreach \k in {-1,-0.75,...,1}
                          {
                              \edef\temp{%
                                  \noexpand \addplot[GraphSmooth, domain = 1:100,
                                      color = black, dashed]
                                  ({x * cos(\k*pi)},{x * sin(\k*pi)});
                              }\temp
                          }
                  \end{axis}
              \end{tikzpicture}
              \begin{tikzpicture}[declare function = {
                              a = -167.05; r = 0.2679;}]
                  \begin{axis}[title = Example 3,
                          xlabel = $ x $, ylabel = $ y $,
                          width = 8cm,Ani, enlargelimits = true,
                          axis equal, view = {0}{90}, grid = both,
                          domain = 0:1.1, restrict y to domain = 0:1.1,
                          colormap/viridis, colorbar horizontal,
                          cycle list = {[samples of colormap = 6]}]
                      \foreach \k in {0,0.2,0.4,0.6,0.8,1}
                          {
                              \edef\temp{%
                                  \noexpand \addplot+[domain = 0:1, thick]
                                  ({x * cos(0.5*pi*\k)},{x * sin(0.5*pi*\k)});
                              }\temp
                          }
                      \foreach \k in {0.2,0.4,0.6,0.8}
                          {
                              \edef\temp{%
                                  \noexpand \addplot[GraphSmooth, domain = 0:0.5*pi,
                                      color = black, very thick, dotted]
                                  ({\k * cos(x)},{\k * sin(x)});
                              }\temp
                          }
                      \addplot[GraphSmooth, domain = 0:0.5*pi, very thick,
                          color = black] ({cos(x)},{sin(x)});
                  \end{axis}
              \end{tikzpicture}
              \begin{tikzpicture}[declare function={
                              alpha(\x,\y) = 0.5*((\x+1)^2 + \y^2)^(1/2)
                              - 0.5*((\x-1)^2 + \y^2)^(1/2);
                              beta(\x,\y) = 0.5*((\x+1)^2 + \y^2)^(1/2)
                              + 0.5*((\x-1)^2 + \y^2)^(1/2);
                          }]
                  \begin{axis}[title = Example 4,
                          xlabel = $ x $, ylabel = $ y $,
                          width = 8cm,Ani,
                          axis equal, view = {0}{90}, grid = both,
                          domain = -3:3, restrict y to domain = -3:3,
                          colormap/viridis, colorbar horizontal,
                          cycle list = {[samples of colormap = 6]}]
                      \addplot3 [thick,contour gnuplot={
                                  number = 11,
                                  labels = false}, samples = 100]
                      {10 + (20/pi) * asin(alpha(x, y))};
                      \addplot3 [very thick, dotted, contour/draw color = {black},
                          contour gnuplot={
                                  number = 5,
                                  labels = false}, samples = 100]
                      {10 + (20/pi) * sign(y) * ln(beta(x, y)
                          + sqrt((beta(x, y))^2 - 1))};
                      \addplot[GraphSmooth, domain = -1:1, very thick,
                          color = black] {0};
                  \end{axis}
              \end{tikzpicture}
          \end{figure}

    \item Using the conformal map,
          \begin{align}
              f(z)      & = -\i z^2 = u + \i\ v         &
              f(z)      & = (2xy) + \i\ (y^2 - x^2)       \\
              T^*(0, v) & = 200, \quad T^*(20, v) = 0   &
              T^*(u, v) & = 200 - 10u                     \\
              T(x, y)   & = \color{y_h} 200 - 20xy      &
              F(z)      & = \color{y_p} 200 + 10\i\ z^2
          \end{align}

    \item Using the conformal map,
          \begin{align}
              f(z)           & = -\i \Ln{z} = u + \i\ v                &
              f(z)           & = \Arg{z} - \i \ln \abs{z}                \\
              T^*(-\pi/4, v) & = -20, \quad T^*(\pi/4, v) = 20         &
              T^*(u, v)      & = \frac{80u}{\pi}                         \\
              T(x, y)        & = \color{y_h} \frac{80}{\pi}\ \Arg{z}   &
              F(z)           & = \color{y_p} \frac{-80\i}{\pi}\ \Ln{z}
          \end{align}

    \item Using the conformal map,
          \begin{align}
              f(z)      & = -\i \Ln{z} = u + \i\ v                 &
              f(z)      & = \Arg{z} - \i \ln \abs{z}                 \\
              T^*(0, v) & = 0, \quad T^*(\pi/3, v) = 50            &
              T^*(u, v) & = \frac{150u}{\pi}                         \\
              T(x, y)   & = \color{y_h} \frac{150}{\pi}\ \Arg{z}   &
              F(z)      & = \color{y_p} \frac{-150\i}{\pi}\ \Ln{z}
          \end{align}

    \item Using the conformal map,
          \begin{align}
              f(z)      & = -\i \Ln{z} = u + \i\ v                               &
              f(z)      & = \Arg{z} - \i \ln \abs{z}                               \\
              T^*(0, v) & = T_1, \quad T^*(\pi/2, v) = T_2                       &
              T^*(u, v) & = T_1 + \frac{2(T_2 - T_1)}{\pi}\ u                      \\
              T(x, y)   & = \color{y_h} T_1 + \frac{2(T_2 - T_1)}{\pi}\ \Arg{z}  &
              F(z)      & = \color{y_p} T_1 - \frac{2(T_2 - T_1)\i}{\pi}\ \Ln{z}
          \end{align}

    \item Using the conformal map,
          \begin{align}
              f(z)            & = -\i \Ln{(z - a)} = u + \i\ v                &
              f(z)            & = \Arg{(z - a)} - \i \ln \abs{z - a}            \\
              T^*(0, v)       & = T_1, \quad T^*(\pi, v) = T_2                &
              T^*(u, v)       & = T_1 + \frac{(T_2 - T_1)u}{\pi}                \\
              T(x, y)         & = \color{y_h} T_1 + \frac{(T_2 - T_1)}{\pi}
              \ \Arg{(z - a)} &
              F(z)            & = \color{y_p} T_1 - \frac{(T_2 - T_1)\i}{\pi}
              \ \Ln{(z - a)}
          \end{align}

    \item Consider the conformal map acting on the upper half $ z $ plane,
          \begin{align}
              f(z)         & = -\i \Big[\Ln(z-b) - \Ln(z-a)\Big] = u + \i\ v \\
              x > b        & \implies x + 0\i \to u = 0 - 0                  \\
              x < a        & \implies x + 0\i \to u = \pi - \pi              \\
              x \in (a, b) & \implies x + 0\i \to u = \pi - 0
          \end{align}
          The boundary conditions in the $ w $ plane are,
          \begin{align}
              T^*(0, v)   & = 0                                                   &
              T^*(\pi, v) & = T_1                                                   \\
              T^*(u, v)   & = \frac{T_1}{\pi}\ u                                  &
              T(x, y)     & = \frac{-\i T_1}{\pi}\ \Ln\Bigg[\frac{z-b}{z-a}\Bigg]
          \end{align}

    \item Consider the conformal map acting on the upper half $ z $ plane,
          \begin{align}
              T(x, y)      & = T_1 + \frac{(T_2 - T_1)}{\pi} \Arg{(z - b)}
              + \frac{(T_3 - T_2)}{\pi} \Arg{(z-a)}                        \\
              x > b        & \implies T = T_1 + 0 + 0                      \\
              x \in (a, b) & \implies T = T_1 + (T_2 - T_1) + 0            \\
              x < a        & \implies T = T_1 + (T_2 - T_1) + (T_3 - T_2)
          \end{align}
          The boundary conditions in the $ w $ plane are,
          \begin{align}
              \Phi(x, y) & = T_1 - \frac{\i}{\pi} \Bigg[ (T_2 - T_1)\Ln(z-b)
                  + (T_3 - T_2)\Ln(z-a) \Bigg]
          \end{align}
          This is a generalization of the result from Problem $ 9 $.

    \item Consider the conformal map acting on the upper half $ z $ plane,
          \begin{align}
              f(z)          & = -\i \Big[\Ln(z-1) - \Ln(z+1)\Big] = u + \i\ v \\
              x > 1         & \implies x + 0\i \to u = 0 - 0                  \\
              x < -1        & \implies x + 0\i \to u = \pi - \pi              \\
              x \in (-1, 1) & \implies x + 0\i \to u = \pi - 0
          \end{align}
          The boundary conditions in the $ w $ plane are,
          \begin{align}
              T^*(0, v)   & = 0                                                  &
              T^*(\pi, v) & = 100                                                  \\
              T^*(u, v)   & = \frac{100}{\pi}\ u                                 &
              T(x, y)     & = \frac{-100\i}{\pi}\ \Ln\Bigg[\frac{z-1}{z+1}\Bigg]
          \end{align}

    \item Consider the conformal map,
          \begin{align}
              w                & = \cosh z                             &
              u + \i v         & = \cosh x \cos y + \i\ \sinh x \sin y   \\
              x                & > 0, \quad  y = 0                     &
              \implies \quad v & = 0, \quad u > 1                        \\
              x                & > 0, \quad  y = \pi                   &
              \implies \quad v & = 0, \quad u < -1                       \\
              x                & = 0, \quad  y \in [0, \pi]            &
              \implies \quad v & = 0, \quad u \in [-1, 1]
          \end{align}
          This is exactly the boundary conditions in Problem $ 12 $,
          \begin{align}
              T(w) & = \frac{-100\i}{\pi}\ \Ln\Bigg[\frac{w-1}{w+1}\Bigg]               &
              T(z) & = \frac{-100\i}{\pi}\ \Ln\Bigg[\frac{\cosh(z)-1}{\cosh(z)+1}\Bigg]
          \end{align}

    \item Once again, this problem transforms to Problem $ 11 $, using
          \begin{align}
              w                  & = z^2                              \\
              x > 1, \quad y = 0 & \implies u > 1, \quad v = 0        \\
              x < 1, \quad y = 0 & \implies u \in [0, 1], \quad v = 0 \\
              y > 1, \quad x = 0 & \implies u < -1, \quad v = 0       \\
              y < 1, \quad x = 0 & \implies u < [-1, 0], \quad v = 0
          \end{align}
          Using the result from Problem $ 11 $,
          \begin{align}
              T(w) & = \frac{-100\i}{\pi}\ \Ln\Bigg[\frac{w-1}{w+1}\Bigg]     &
              T(z) & = \frac{-100\i}{\pi}\ \Ln\Bigg[\frac{z^2-1}{z^2+1}\Bigg]
          \end{align}

    \item From Example $ 3 $ in the text,
          \begin{align}
              \Bigg[ \diffp Tr \Bigg]_{r = R_1} & = 0                           &
              \Bigg[ \diffp Tr \Bigg]_{r = R_2} & = 0                             \\
              T(R_1<r<R_2, 0)                   & = 0                           &
              T(R_1<r<R_2, \pi/2)               & = 200                           \\
              T(r, \theta)                      & = \frac{400}{\pi}\ \theta     &
              F(z)                              & = \frac{-400 \i}{\pi}\ \Ln{z}
          \end{align}

    \item From Example $ 3 $ in the text,
          \begin{align}
              T(r, \theta) & = -20 + \frac{80}{\pi/4}\ \theta   &
              F(z)         & = -20 - \frac{320 \i}{\pi}\ \Ln{z}
          \end{align}

    \item From Example $ 3 $ in the text,
          \begin{align}
              T(r, \theta) & = 20 + \frac{480}{\pi/3}\ \theta   &
              F(z)         & = 20 - \frac{1440 \i}{\pi}\ \Ln{z}
          \end{align}

    \item Since in Example $ 4 $ from the text, the $ y $ axis is at the mean potential,
          this problem can be extrapolated to $ T_2 = 200, T_1 = 0 $
          \begin{align}
              T(w) & = \frac{0 + 200}{2} + \frac{100}{\pi/2}\ \Re(\arcsin z) \\
              F(z) & = 100 + \frac{200}{\pi}\ \arcsin(z)
          \end{align}

    \item Using Example $ 3 $ from the text,
          \begin{align}
              T(r, \theta) & = 100 - \frac{130}{\pi/2}\ \theta &
              T(r, \theta) & = 100 + \frac{260\i}{\pi}\ \Ln{z}
          \end{align}

    \item Three plates at $ x < -1, x \in (-1, 1) $ and $ x > 1 $, each at
          potential $ \SI{0}{\V}, \SI{100}{\V} $ and $ \SI{0}{\V} $ respectively.

    \item The temperature on the minor arc and major arc between $ Z_1 $ and $ Z_2 $ is
          $ \SI{10}{\celsius} $ and $ \SI{200}{\celsius} $ respectively. \par
          Find the temperature everywhere inside the unit disk.

\end{enumerate}