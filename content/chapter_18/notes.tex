\chapter{Complex Analysis and Potential Theory}

\section{Electrostatic Fields}

\begin{description}
    \item[Laplace equation] The electrostatic potential in a region free of charges is
        given by Laplace's equation
        \begin{align}
            \nabla^2 \Phi & = 0
        \end{align}
        The surfaces $ \Phi = c $ are called equipotential surfaces.

    \item[Complex potential] Consider a harmonic function $ \Phi(x, y) $ whose
        harmonic conjugate is $ \Psi(x, y) $.
        \begin{align}
            F(z) & = \Phi + \i\ \Psi
        \end{align}
        is now an analytic function of $ z $, and is called the complex potential
        corresponding to the real potential $ \Phi $.

    \item[Conformality] Since equipotential lines have to intersect electromagnetic
        lines of force at right angles, the complex potential is a prepackaged conformal
        map of $ z $ that yields both $ \Phi $ and $ \Psi $.

    \item[Superposition] Potantials due to more than one source at a given point can
        be superposed to yield the net potential. For complex potentials, this happens to
        be simple scalar addition of the real and imaginary parts separately.
\end{description}

\section{Use of Conformal Mapping}

\begin{description}
    \item[Dirichlet problem] A type of boundary value problem where the value of the
        potential is specified everywhere on the boundary surface.

    \item[Conformal mapping of harmonic functions] Let a function $ \Phi^* $ be
        harmonic in a domain $ D^* $ in the $ w $ plane.
        Suppose $ w = f(z) $ is analytic in a domain $ D $ in the $ z $ plane, and maps
        $ D $ conformally onto $ D^* $. Then,
        \begin{align}
            \Phi(x, y) & = \Phi^*\Big[u(x, y),\ v(x, y) \Big]
        \end{align}
        is also harmonic in $ D $. If $ D^* $ is simply connected, then the harmonic
        conjugate $ \Psi^* $ exists.

    \item[Strategy] Use a conformal mapping to convert the complicated BVP into a
        standard BVP for which the result is known. Then, reverse transform
        $ w \to z $ to get the compmlex potential in $ z $ space.

\end{description}

\section{Heat Problems}

\begin{description}
    \item[Heat equation] A steady state heat flow, which means that temperature is
        independent of time reduces to,
        \begin{align}
            \nabla^2 T & = 0
        \end{align}
        which is the Laplace equation in temperature (a scalar like electrostatic
        potential)

    \item[Isotherms] The contours of constant temperature. The orthogonal family of
        curves is called the lines of heat flow.

    \item[Mixed BVP] Sometimes, the potential is specified on some portion of the
        boundary and its normal derivative is specified over the rest. This becomes a
        mixed boundary value problem.
\end{description}

\section{Fluid Flow}

\begin{description}
    \item[Complex potential] A conplex function of the form,
        \begin{align}
            F(z) & = \Phi(x, y) + \i\ \Psi(x, y)
        \end{align}
        where $ \Phi $ is called the velocity potential and $ \Psi $ is the stream
        function. This function is analytic and therefore $ \Phi, \Psi $ are
        harmonic functions.

    \item[Streamlines] Curves of $ \Phi(x, y) = c $ are still called equipotential lines,
        and the trajectories of fluid flow are $ \Psi(x, y) = c $. \par
        The boundaries across which fluid cannot flow must be streamlines.

    \item[Velocity] Anoalogous to the two-dimensional vector for the velocity of a fluid,
        \begin{align}
            V & = V_1 + \i\ V_2 = \overline{F'{z}}
        \end{align}
        is the complex velocity.

    \item[Stagnation points] Points along a fluid flow vector field, where the
        velocity is zero. This requires the flow to be steady (independent of time).

    \item[Assumptions] If a simply connected domain $ D $ has an incompressible
        and irrotational fluid flow at steady state, then the flow has a complex
        potential $ F(z) $ which is an analytic function.

    \item[Circulation] The line integral of the tangential component of the velocity
        along a curve.
        \begin{align}
            C & = \int_{C} V_t\ \dl t
        \end{align}

    \item[Rotation] In two dimensions, it is the curl of the velocity vector.
        \begin{align}
            \omega(x, y) & = \frac{1}{2}\ \Bigg[\diffp{V_2}{x} - \diffp{V_1}{y}\Bigg]
        \end{align}
        A special case where $ \nabla \times \vec{V} = 0 $ is called an irrotational
        vector field. \par
        Without the $ 1/2 $ factor, this expression is called the vorticity.

    \item[Incompressible] This refers to the fact that there are no sources or sinks
        of fluid anywhere in the domain. In vector calculus, this is denoted by
        $ \nabla \dotp \vec{V} = 0 $. \par
        The fluid inflow per second into a fixed volume must equal the fluid outflow, at
        all points in the domain.

    \item[Complex velocity potential]  If the above conditions are met, then the line
        integral of the tangential velocity over any closed loop vanishes by Green's
        theorem. \par
        The path independence of this integral establishes the complex velocity
        potential (which is a scalar).
        \begin{align}
            \Phi(x, y) & = \int_{z_0}^{z} (V_1\ \dl x + V_2\ \dl y)
        \end{align}
\end{description}

\section{Poisson's Integral Formula for Potentials}

\begin{description}
    \item[Poisson's Integral formula] Applying Cauchy's integral formula on a closed
        line integral on a circle of radius $ R > 1 $,
        \begin{align}
            \Phi(r, \theta) & = \frac{1}{2\pi}\ \int_{0}^{2\pi} \Phi(R, \alpha)
            \ \frac{R^2 - r^2}{R^2 + r^2 - 2Rr\cos(\theta - \alpha)}\ \dl \theta
        \end{align}
        This is the potential at all points inside the disk $ \abs{z} \leq R $ in terms
        of the potential on the disk's boundary $ \Phi(R, \alpha) $. \par
        If the boundary is only piecewise continuous, then the formula is valid at all
        points on the open disk $ \abs{z} < R $, and all points on the boundary except
        these discontinuities.

    \item[Series solution] The potential can be expanded into an infinite series using
        the geometric series,
        \begin{align}
            \Phi(r, \theta) & = a_0 + \iser[n]{1} \Big( \frac{r}{R} \Big)^n\
            \Big[ a_n \cos(n\theta) + b_n \sin(n\theta) \Big]                  \\
            a_0             & = \frac{1}{2\pi} \int_{0}^{2\pi} \Phi(R, \alpha)
            \ \dl \alpha                                                       \\
            a_n             & = \frac{1}{\pi} \int_{0}^{2\pi} \Phi(R, \alpha)
            \ \cos(n\alpha)\ \dl \alpha                                        \\
            b_n             & = \frac{1}{\pi} \int_{0}^{2\pi} \Phi(R, \alpha)
            \ \sin(n\alpha)\ \dl \alpha
        \end{align}
        The derivation uses the Fourier series expansion of the integrand in the Poisson
        integral formula. \par
        At the boundary, $ r = R $ and this series reduces to the fourier series of
        $ \Phi(R, \alpha) $. This method is valid whenever $ \Phi $ has a Fourier series
        expansion.
\end{description}

\section{General Properties of Harmonic Functions}

\begin{description}
    \item[Mean Value Property] Let $ F(z) $ be analytic in a simply connected domain
        $ D $. Then, the value of $ F(z) $ at any point $ z_0 $ in $ D $ is equal to its
        mean value on any circle in $ D $ with center $ z_0 $

    \item[Mean value of harmonic functions] Relacing $ F(z) $ above with a harmonic
        function $ \Phi(x,y) $, the relation holds. \par
        Additionally, this value is equal to the mean value of $ \Phi $ over any disk
        in $ D $ centered on $ z_0 $.

    \item[Maximum modulus theorem] Let $ F(z) $ be analytic and non-constant in a domain
        containing a bounded region $ R $ and its boundary. \par
        The absolute value of $ F(z) $ cannot have its maximum at an interior point of
        $ R $, only on its boundary. \par
        If $ F(z) \neq 0 \quad \forall \quad z \in R $, this relation also holds for the
        minimum value of $ \abs{F(z)} $.

    \item[Properties of Harmonic functions] Using $ \Phi(x,y), R, C $ as defined
        above,
        \begin{itemize}
            \item If $ \Phi(x, y) $ is not constant, it has neither a maximum nor a
                  minimum in $ R $. If they exist, they have to be on the boundary $ C $.
            \item If $ \Phi(x,y) $ is constant on $ C $, then it is constant in
                  the entire region $ R $ bounded by $ C $.
            \item If $ h(x, y) $ is harmonic in $ R $ and on $ C $, and if
                  $ h = \Phi $ on $ C $, then this also holds everywhere in $ R $.
        \end{itemize}

    \item[Uniqueness theorem for Dirichlet BVP] If a BVP for the Laplace equation in
        two variables has a solution, it must be unique. \par
        This follows from the above properties of harmonic functions, which are uniquely
        determined in $ R $, by their value on the boundary $ C $.
\end{description}