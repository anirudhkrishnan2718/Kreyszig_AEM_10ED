\section{Some Applications of Eigenvalue Problems}

\begin{enumerate}
    \item Finding eigenvalues,
          \begin{align}
              \vec{A}                        & = \bmattt{3}{1.5}{1.5}{3}      &
              \det(\vec{A} - \lambda\vec{I}) & = 0                              \\
              0                              & = (\lambda - 3)^2 - (1.5)^2    &
              0                              & = \lambda^2 - 6 \lambda + 6.75   \\
              \{\lambda_i\}                  & = \{ 1.5, 4.5 \}
          \end{align}
          Finding the eigenvectors,
          \begin{align}
              \lambda_1                             & = \color{y_h} 1.5 &
              \bmattt{1.5}{1.5}{1.5}{1.5} \vec{x}   & = \vec{0}         &
              \vec{u}_1                             & = \color{y_h}
              \bmatcol{1}{-1},\ \SI{-45}{\degree}                         \\
              \lambda_2                             & = \color{y_p} 4.5 &
              \bmattt{-1.5}{1.5}{1.5}{-1.5} \vec{x} & = \vec{0}         &
              \vec{u}_2                             & = \color{y_p}
              \bmatcol{1}{1},\ \SI{45}{\degree}
          \end{align}

    \item Finding eigenvalues,
          \begin{align}
              \vec{A}                        & = \bmattt{2}{0.4}{0.4}{2}      &
              \det(\vec{A} - \lambda\vec{I}) & = 0                              \\
              0                              & = (\lambda - 2)^2 - (0.4)^2    &
              0                              & = \lambda^2 - 4 \lambda + 3.84   \\
              \{\lambda_i\}                  & = \{ 1.6, 2.4 \}
          \end{align}
          Finding the eigenvectors,
          \begin{align}
              \lambda_1                             & = \color{y_h} 1.6 &
              \bmattt{0.4}{0.4}{0.4}{0.4} \vec{x}   & = \vec{0}         &
              \vec{u}_1                             & = \color{y_h}
              \bmatcol{1}{-1},\ \SI{-45}{\degree}                         \\
              \lambda_2                             & = \color{y_p} 2.4 &
              \bmattt{-0.4}{0.4}{0.4}{-0.4} \vec{x} & = \vec{0}         &
              \vec{u}_2                             & = \color{y_p}
              \bmatcol{1}{1},\ \SI{45}{\degree}
          \end{align}

    \item Finding eigenvalues,
          \begin{align}
              \vec{A}                        & = \bmattt{7}{\sqrt{6}}{\sqrt{6}}{2} &
              \det(\vec{A} - \lambda\vec{I}) & = 0                                   \\
              0                              & = (\lambda - 2)(\lambda - 7) - 6    &
              0                              & = \lambda^2 - 9 \lambda + 8           \\
              \{\lambda_i\}                  & = \{ 1, 8 \}
          \end{align}
          Finding the eigenvectors,
          \begin{align}
              \lambda_1                                   & = \color{y_h} 1 &
              \bmattt{6}{\sqrt{6}}{\sqrt{6}}{1} \vec{x}   & = \vec{0}       &
              \vec{u}_1                                   & = \color{y_h}
              \bmatcol{1}{-\sqrt{6}},\ \SI{-67.8}{\degree}                    \\
              \lambda_2                                   & = \color{y_p} 8 &
              \bmattt{-1}{\sqrt{6}}{\sqrt{6}}{-6} \vec{x} & = \vec{0}       &
              \vec{u}_2                                   & = \color{y_p}
              \bmatcol{\sqrt{6}}{1},\ \SI{22.2}{\degree}
          \end{align}

    \item Finding eigenvalues,
          \begin{align}
              \vec{A}                        & = \bmattt{5}{2}{2}{13}            &
              \det(\vec{A} - \lambda\vec{I}) & = 0                                 \\
              0                              & = (\lambda - 5)(\lambda - 13) - 4 &
              0                              & = \lambda^2 - 18 \lambda + 61       \\
              \{\lambda_i\}                  & = \{ 9 \pm 2\sqrt{5} \}
          \end{align}
          Finding the eigenvectors,
          \begin{align}
              \lambda_1                     & = \color{y_h} 9 - 2\sqrt{5} &
              \bmattt{-4 + 2\sqrt{5}}
              {2}{2}{4 + 2\sqrt{5}} \vec{x} & = \vec{0}                   &
              \vec{u}_1                     & = \color{y_h}
              \bmatcol{1}{2 - \sqrt{5}},\ \SI{-76.72}{\degree}              \\
              \lambda_2                     & = \color{y_p} 9 + 2\sqrt{5} &
              \bmattt{-4 - 2\sqrt{5}}
              {2}{2}{4 - 2\sqrt{5}} \vec{x} & = \vec{0}                   &
              \vec{u}_2                     & = \color{y_p}
              \bmatcol{1}{2 + \sqrt{5}},\ \SI{13.28}{\degree}
          \end{align}

    \item Finding eigenvalues,
          \begin{align}
              \vec{A}                        & = \bmattt{1}{0.5}{0.5}{1}      &
              \det(\vec{A} - \lambda\vec{I}) & = 0                              \\
              0                              & = (\lambda - 1)^2 - 0.25       &
              0                              & = \lambda^2 - 2 \lambda + 0.75   \\
              \{\lambda_i\}                  & = \{ 0.5, 1.5 \}
          \end{align}
          Finding the eigenvectors,
          \begin{align}
              \lambda_1                             & = \color{y_h} 0.5 &
              \bmattt{0.5}{0.5}{0.5}{0.5} \vec{x}   & = \vec{0}         &
              \vec{u}_1                             & = \color{y_h}
              \bmatcol{1}{-1},\ \SI{-45}{\degree}                         \\
              \lambda_2                             & = \color{y_p} 1.5 &
              \bmattt{-0.5}{0.5}{0.5}{-0.5} \vec{x} & = \vec{0}         &
              \vec{u}_2                             & = \color{y_p}
              \bmatcol{1}{1},\ \SI{45}{\degree}
          \end{align}

    \item Finding eigenvalues,
          \begin{align}
              \vec{A}                        & = \bmattt{1.25}{0.75}{0.75}{1.25} &
              \det(\vec{A} - \lambda\vec{I}) & = 0                                 \\
              0                              & = (\lambda - 1.25)^2 - (0.75)^2   &
              0                              & = \lambda^2 - 2.5 \lambda + 1       \\
              \{\lambda_i\}                  & = \{ 0.5, 2 \}
          \end{align}
          Finding the eigenvectors,
          \begin{align}
              \lambda_1                                 & = \color{y_h} 0.5 &
              \bmattt{0.75}{0.75}{0.75}{0.75} \vec{x}   & = \vec{0}         &
              \vec{u}_1                                 & = \color{y_h}
              \bmatcol{1}{-1},\ \SI{-45}{\degree}                             \\
              \lambda_2                                 & = \color{y_p} 2   &
              \bmattt{-0.75}{0.75}{0.75}{-0.75} \vec{x} & = \vec{0}         &
              \vec{u}_2                                 & = \color{y_p}
              \bmatcol{1}{1},\ \SI{45}{\degree}
          \end{align}

    \item Finding eigenvalues,
          \begin{align}
              \vec{A}                        & = \bmattt{0.2}{0.5}{0.8}{0.5}          &
              \det(\vec{A} - \lambda\vec{I}) & = 0                                      \\
              0                              & = (\lambda - 0.2)(\lambda - 0.5) - 0.4 &
              0                              & = \lambda^2 - 0.7 \lambda - 0.3          \\
              \{\lambda_i\}                  & = \{ -0.3, 1 \}
          \end{align}
          Finding the eigenvectors,
          \begin{align}
              \lambda_1                             & = \color{y_h} -0.3 &
              \bmattt{0.5}{0.5}{0.8}{0.8} \vec{x}   & = \vec{0}          &
              \vec{u}_1                             & = \color{y_h}
              \bmatcol{1}{-1}                                              \\
              \lambda_2                             & = \color{y_p} 1    &
              \bmattt{-0.8}{0.5}{0.8}{-0.5} \vec{x} & = \vec{0}          &
              \vec{u}_2                             & = \color{y_p}
              \bmatcol{5}{8}
          \end{align}
          Since a Markov process has steady state when $ \lambda  = 1$, the steady state
          is $\vec{u}_2$.

    \item Finding the eigenvalues,
          \begin{align}
              \vec{A}                         & =
              \begin{bNiceMatrix}[r, margin]
                  0.4 & 0.3 & 0.3 \\
                  0.3 & 0.6 & 0.1 \\
                  0.3 & 0.1 & 0.6
              \end{bNiceMatrix}  &
              \det(\vec{A} - \lambda \vec{I}) & = 0                        \\
              0                               & = \lambda^3 - 1.6\lambda^2
              + 0.65 \lambda - 0.05           &
              \{\lambda_i\}                   & = \{0.1, 0.5, 1\}
          \end{align}
          Finding the steady state using the unity eigenvalue,
          \begin{align}
              \lambda_1                      & = \color{y_h} 1 &
              \begin{bNiceMatrix}[r, margin]
                  -0.6 & 0.3  & 0.3  \\
                  0.3  & -0.4 & 0.1  \\
                  0.3  & 0.1  & -0.4
              \end{bNiceMatrix} \vec{x} & = 0             &
              \vec{u}_1                      & =
              \color{y_h} \begin{bNiceMatrix}[r, margin]
                              1 \\ 1 \\ 1
                          \end{bNiceMatrix}
          \end{align}

    \item Finding the eigenvalues,
          \begin{align}
              \vec{A}                         & =
              \begin{bNiceMatrix}[r, margin]
                  0.6 & 0.1 & 0.2 \\
                  0.4 & 0.1 & 0.4 \\
                  0   & 0.8 & 0.4
              \end{bNiceMatrix}  &
              \det(\vec{A} - \lambda \vec{I}) & = 0                        \\
              0                               & = \lambda^3 - 1.1\lambda^2
              - 0.02 \lambda + 0.12           &
              \{\lambda_i\}                   & = \{-0.3, 0.4, 1\}
          \end{align}
          Finding the steady state using the unity eigenvalue,
          \begin{align}
              \lambda_1                      & = \color{y_h} 1 &
              \begin{bNiceMatrix}[r, margin]
                  -0.4 & 0.1  & 0.2  \\
                  0.4  & -0.9 & 0.4  \\
                  0    & 0.8  & -0.6
              \end{bNiceMatrix} \vec{x} & = 0             &
              \vec{u}_1                      & =
              \color{y_h} \begin{bNiceMatrix}[r, margin]
                              11 \\ 12 \\ 16
                          \end{bNiceMatrix}
          \end{align}

    \item Finding the eigenvalues,
          \begin{align}
              \vec{A}                         & =
              \begin{bNiceMatrix}[r, margin]
                  0   & 9   & 5 \\
                  0.4 & 0   & 0 \\
                  0   & 0.4 & 0
              \end{bNiceMatrix}  &
              \det(\vec{A} - \lambda \vec{I}) & = 0                               \\
              0                               & = \lambda^3 - 3.6 \lambda - 0.8 &
              \{\lambda_i\}                   & = \{2, -1 \pm 0.2\sqrt{15}\}
          \end{align}
          Finding the growth rate using the positive eigenvalue,
          \begin{align}
              \lambda_1                      & = \color{y_h} 2 &
              \begin{bNiceMatrix}[r, margin]
                  -2  & 9   & 5  \\
                  0.4 & -2  & 0  \\
                  0   & 0.4 & -2
              \end{bNiceMatrix} \vec{x} & = 0             &
              \vec{u}_1                      & =
              \color{y_h} \frac{1}{31} \cdot \begin{bNiceMatrix}[r, margin]
                                                 25 \\ 5 \\ 1
                                             \end{bNiceMatrix}
          \end{align}

    \item Finding the eigenvalues,
          \begin{align}
              \vec{A}                         & =
              \begin{bNiceMatrix}[r, margin]
                  0   & 3.45 & 0.6 \\
                  0.9 & 0    & 0   \\
                  0   & 0.45 & 0
              \end{bNiceMatrix}  &
              \det(\vec{A} - \lambda \vec{I}) & = 0                                   \\
              0                               & = \lambda^3 - 3.105 \lambda - 0.243 &
              \{\lambda_i\}                   & = \{1.8, -0.9 \pm 0.15\sqrt{30}\}
          \end{align}
          Finding the growth rate using the positive eigenvalue,
          \begin{align}
              \lambda_1                      & = \color{y_h} 1.8 &
              \begin{bNiceMatrix}[r, margin]
                  0   & 3.45 & 0.6 \\
                  0.9 & 0    & 0   \\
                  0   & 0.45 & 0
              \end{bNiceMatrix} \vec{x} & = 0               &
              \vec{u}_1                      & =
              \color{y_h} \begin{bNiceMatrix}[r, margin]
                              25 \\ 5 \\ 1
                          \end{bNiceMatrix}
          \end{align}

    \item Finding the eigenvalues,
          \begin{align}
              \vec{A}                         & =
              \begin{bNiceMatrix}[r, margin]
                  0   & 3   & 2   & 2 \\
                  0.5 & 0   & 0   & 0 \\
                  0   & 0.5 & 0   & 0 \\
                  0   & 0   & 0.1 & 0
              \end{bNiceMatrix}  &
              \det(\vec{A} - \lambda \vec{I}) & = 0                               \\
              0                               & = \lambda^3 - 1.5\lambda^2
              -0.5\lambda - 0.05              &
              \{\lambda_i\}                   & = \{1.8, -0.9 \pm 0.15\sqrt{30}\}
          \end{align}
          Finding the growth rate using the positive eigenvalue,
          \begin{align}
              \lambda_1                            & = \color{y_h} 1.375 &
              \begin{bNiceMatrix}[r, margin]
                  -1.375 & 3      & 2      & 2      \\
                  0.5    & -1.375 & 0      & 0      \\
                  0      & 0.5    & -1.375 & 0      \\
                  0      & 0      & 0.1    & -1.375
              \end{bNiceMatrix} \vec{x} & = 0                 &
              \vec{u}_1                            & =
              \color{y_h} \begin{bNiceMatrix}[r, margin]
                              103.94 \\ 37.80 \\ 13.75 \\ 1
                          \end{bNiceMatrix}
          \end{align}

    \item The total purchase by industry $ j $ is,
          \begin{align}
              \sum_{k=1}^{n} a_{jk}\ p_j \equiv \vec{A}_j \dotp \vec{p}
          \end{align}
          The total revenue of industry $ j $ is simply $ p_j $, which gives the equation
          \begin{align}
              \vec{Ap} & = \lambda\ \vec{p}
          \end{align}
          For revenue to be equal to expenditure, $ \lambda = 1 $. \par
          Finding the eigenvalues,
          \begin{align}
              \vec{A}                         & =
              \begin{bNiceMatrix}[r, margin]
                  0.1 & 0.5 & 0   \\
                  0.8 & 0   & 0.4 \\
                  0.1 & 0.5 & 0.6
              \end{bNiceMatrix}  &
              \det(\vec{A} - \lambda \vec{I}) & = 0                         \\
              0                               & = \lambda^3 - 0.7 \lambda^2
              - 0.54 \lambda + 0.24           &
              \{\lambda_i\}                   & = \{-0.66, 0.36, 1\}
          \end{align}
          Finding the equilibrium price using the unity eigenvalue,
          \begin{align}
              \lambda_1                      & = \color{y_h} 1 &
              \begin{bNiceMatrix}[r, margin]
                  -0.9 & 0.5 & 0    \\
                  0.8  & -1  & 0.4  \\
                  0.1  & 0.5 & -0.4
              \end{bNiceMatrix} \vec{x} & = 0             &
              \vec{u}_1                      & =
              \color{y_h} \begin{bNiceMatrix}[r, margin]
                              10 \\ 18 \\ 25
                          \end{bNiceMatrix}
          \end{align}

    \item Since a column is the set of all fractions of a certain quantity, it
          must sum to 1 by definition. (Similar to a Markov process). \par
          Consider adding to the first row of $ \vec{A} - \vec{I} $,
          every other row. This leaves the determinant unchanged.
          \begin{align}
              \vec{A} - \vec{I}       &
              = \begin{bNiceMatrix}[r, margin]
                    v_1 - 1       & v_3           & v_5        \\
                    v_2           & v_4 - 1       & v_6        \\
                    1 - v_1 - v_2 & 1 - v_3 - v_4 & -v_5 - v_6
                \end{bNiceMatrix} \\
              \det(\vec{A} - \vec{I}) &
              = \begin{bNiceMatrix}[r, margin]
                    0             & 0             & 0          \\
                    v_2           & v_4 - 1       & v_6        \\
                    1 - v_1 - v_2 & 1 - v_3 - v_4 & -v_5 - v_6
                \end{bNiceMatrix} = 0
          \end{align}
          This means that $ \det(\vec{A} - \lambda\vec{I}) = 0 $ will always have the
          solution $ \lambda = 1 $. (Illustrated for $ n = 3 $, but this result holds
          for general $ n $).

    \item Given the equation,
          \begin{align}
              \vec{x} - \vec{Ax}                        & = y                   &
              (\vec{I} - \vec{A})\vec{x}                & = \vec{y}               \\
              \vec{Bx} = \begin{bNiceMatrix}[r, margin]
                             0.9  & -0.4 & -0.2 \\
                             -0.5 & 1    & -0.9 \\
                             -0.1 & -0.4 & 0.6
                         \end{bNiceMatrix} \vec{x} & = \vec{y}             &
              \vec{x}                                   & = \vec{B}^{-1}\vec{y}   \\
              \vec{y}                                   & =
              \begin{bNiceMatrix}[r, margin]
                  0.1 \\ 0.3 \\ 0.1
              \end{bNiceMatrix}            &
              \vec{x}                                   & =
              \begin{bNiceMatrix}[r, margin]
                  \frac{11}{15} \\ \frac{71}{90} \\ \frac{11}{9}
              \end{bNiceMatrix}
          \end{align}

    \item Sum of the main diagonal entries is,
          \begin{align}
              \det(\vec{A} - \lambda \vec{I}) & = (-1)^n (\lambda - \lambda_1)
              (\lambda - \lambda_2) \dots (\lambda - \lambda_n)                \\
                                              & = (-1)^n \Big[\lambda^n -
                  \tr(\vec{A})\ \lambda^{n-1} + \dots\Big]
          \end{align}
          Comparing coefficients of $ \lambda^{n-1} $ proves the relation
          \begin{align}
              \tr(\vec{A}) & = \sum \lambda_i
          \end{align}

    \item Starting with $ \vec{B} = \vec{A} - k \vec{I} $, which has eigenvalues
          $ \{\mu_i\} $ and eigenvectors $ \{\vec{b}_i\} $
          \begin{align}
              \det(\vec{B} - \mu \vec{I})              & = 0 &
              \det\Big[\vec{A} - (k + \mu)\vec{I}\Big] & = 0   \\
          \end{align}
          So, if the eigenvales of $ \vec{A} $ form the set $ \{\lambda_i\} $, then the
          corresponding set of eigenvalues of $ \vec{B} $ is $ \{\lambda_i - k\} $.
          \par Solving for the eigenvector $ \mu_1 $,
          \begin{align}
              \vec{Bx}            & = \mu_1 \vec{x}     &
              \vec{Ax}            & = \lambda_1 \vec{x}   \\
              \vec{Ax} - k\vec{x} & = \mu_1 \vec{x}
          \end{align}
          These are the same equation, which means they have the same solution. Thus,
          spectral shifts do not change the eigenvector.

    \item Scalar multiples of $ \vec{A} $,
          \begin{align}
              \det(\vec{A} - \lambda\vec{I})            & = 0          &
              \det(k\vec{A} - \mu \vec{I})              & = 0            \\
              \det\Big[k(\vec{A} - \lambda\vec{I})\Big] & = k^n\ 0 = 0 &
              \mu                                       & = k\lambda
          \end{align}
          The eigenvectors are found using,
          \begin{align}
              (\vec{A} - \lambda_1 \vec{I})\ \vec{x}   & = 0 &
              (k\vec{A} - k\lambda_1 \vec{I})\ \vec{x} & = 0
          \end{align}
          Since the second equation is the same as the first, the solutions are the same.
          \par Looking at positive integer powers of $ \vec{A} $, by induction
          \begin{align}
              \text{If} \quad \vec{A}^m \vec{x}       & = \lambda^m \vec{x}           \\
              \text{then} \quad \vec{A}^{m+1} \vec{x} & = \vec{A}\ (\vec{A}^m\vec{x})
              = \vec{A}\ \lambda^m \vec{x}
              = \lambda^{m+1}\vec{x}
          \end{align}
          Since the above relation is true for $ m = 1 $, induction makes it true for
          all positive integers.
          \par The eigenvectors are found using the equation,
          \begin{align}
              \vec{Ax}_1          & = \lambda_1\ \vec{x}_1            \\
              \vec{A}^m \vec{x}_1 & = \lambda_1\ \vec{A}^m\ \vec{x}_1
              = \lambda_1^m \vec{x}_1
              \vec{}
          \end{align}
          Thus the eigenvectors remain unchanged.

    \item Looking at the sum of matrices,
          \begin{align}
              \vec{Ax}             & = \lambda\ \vec{x}         &
              \vec{Bx}             & = \mu\ \vec{x}               \\
              (\vec{A + B})\vec{x} & = (\lambda + \mu)\ \vec{x}
          \end{align}
          In combination with the results of Prob. $ 17, 18 $, the result is proved
          trivially.

    \item Some terms of a Leslie matrix are zero by defintion, since an organism
          can only go from the current age range to the next higher age range. \par
          Now, with $ b,c,d,e > 0 $,
          \begin{align}
              \vec{L}                        & =
              \begin{bNiceMatrix}[r, margin]
                  a & b & c \\
                  d & 0 & 0 \\
                  0 & e & 0
              \end{bNiceMatrix}                              \\
              \det(\vec{A} - \lambda\vec{I}) & = (a-\lambda) (\lambda^2)
              - d(-b\lambda - ce) = 0                                    \\
              \lambda^3 - a\lambda^2
              - bd\lambda - cde              & = 0                       \\
              cde                            & > 0 \implies
              \lambda_1 \lambda_2 \lambda_3  > 0
          \end{align}
          This means none of the roots are zero, and either no roots are negative or
          two roots are negative. \par
          Thus, at least one root is guaranteed positive. \par
          For the case of the other two roots being complex, their product is always
          positive.

\end{enumerate}