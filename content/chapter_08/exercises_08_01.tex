\section{The Matrix Eigenvalue Problem, Determining Eigenvalues and Eigenvectors}

\begin{enumerate}
    \item Finding the eigenvalues,
          \begin{align}
              \vec{A}                        & = \bmattt{3}{0}{0}{-0.6} &
              \det(\vec{A} - \lambda\vec{I}) & = 0                        \\
              (\lambda - 3)(\lambda + 0.6)   & = 0                      &
              \{\lambda_i\}                  & = \{-0.6, 3\}
          \end{align}
          Finding the eigenvectors,
          \begin{align}
              \lambda_1 & = \color{y_h} -0.6           &
              \vec{u}_1 & = \color{y_h} \bmatcol{0}{1}   \\
              \lambda_2 & = \color{y_p} 3              &
              \vec{u}_2 & = \color{y_p} \bmatcol{1}{0}
          \end{align}

    \item Finding the eigenvalues,
          \begin{align}
              \vec{A}                        & = \bmattt{0}{0}{0}{0} &
              \det(\vec{A} - \lambda\vec{I}) & = 0                     \\
              (\lambda + 0)(\lambda + 0)     & = 0                   &
              \{\lambda_i\}                  & = \{0, 0\}
          \end{align}
          Finding the eigenvectors,
          \begin{align}
              \lambda_1 & = \color{y_h} 0              &
              \vec{u}_1 & = \color{y_h} \bmatcol{1}{0}   \\
              \lambda_2 & = \color{y_p} 0              &
              \vec{u}_2 & = \color{y_p} \bmatcol{0}{1}
          \end{align}

    \item Finding the eigenvalues,
          \begin{align}
              \vec{A}                        & = \bmattt{5}{-2}{9}{-6} &
              \det(\vec{A} - \lambda\vec{I}) & = 0                       \\
              (\lambda + 6)(\lambda-5) + 18  & = 0                     &
              \lambda^2 + \lambda - 12       & = 0                       \\
              \{\lambda_i\}                  & = \{-4, 3\}
          \end{align}
          Finding the eigenvectors,
          \begin{align}
              \lambda_1                     & = \color{y_h} -4 &
              \bmattt{9}{-2}{9}{-2} \vec{x} & = \vec{0}        &
              \vec{u}_1                     & = \color{y_h}
              \bmatcol{2}{9}                                     \\
              \lambda_2                     & = \color{y_p} 3  &
              \bmattt{2}{-2}{9}{-9} \vec{x} & = \vec{0}        &
              \vec{u}_1                     & = \color{y_p}
              \bmatcol{1}{1}
          \end{align}

    \item Finding the eigenvalues,
          \begin{align}
              \vec{A}                        & = \bmattt{1}{2}{2}{4} &
              \det(\vec{A} - \lambda\vec{I}) & = 0                     \\
              (\lambda - 1)(\lambda - 4) - 4 & = 0                   &
              \lambda^2 - 5\lambda           & = 0                     \\
              \{\lambda_i\}                  & = \{0, 5\}
          \end{align}
          Finding the eigenvectors,
          \begin{align}
              \lambda_1                     & = \color{y_h} 0 &
              \bmattt{1}{2}{2}{4} \vec{x}   & = \vec{0}       &
              \vec{u}_1                     & = \color{y_h}
              \bmatcol{2}{-1}                                   \\
              \lambda_2                     & = \color{y_p} 5 &
              \bmattt{-4}{2}{2}{-1} \vec{x} & = \vec{0}       &
              \vec{u}_1                     & = \color{y_p}
              \bmatcol{1}{2}
          \end{align}

    \item Finding the eigenvalues,
          \begin{align}
              \vec{A}                        & = \bmattt{0}{3}{-3}{0} &
              \det(\vec{A} - \lambda\vec{I}) & = 0                      \\
              \lambda^2 + 9                  & = 0                    &
              \{\lambda_i\}                  & = \{-3i, 3i\}
          \end{align}
          Finding the eigenvectors,
          \begin{align}
              \lambda_1                      & = \color{y_h} -3i &
              \bmattt{3i}{3}{-3}{3i} \vec{x} & = \vec{0}         &
              \vec{u}_1                      & = \color{y_h}
              \bmatcol{1}{-i}                                      \\
              \lambda_2                      & = \color{y_p} 3i  &
              \bmattt{-3i}{3}{-3}{-3i}       & = \vec{0}         &
              \vec{u}_1                      & = \color{y_p}
              \bmatcol{1}{i}
          \end{align}

    \item Finding the eigenvalues,
          \begin{align}
              \vec{A}                        & = \bmattt{1}{2}{0}{3} &
              \det(\vec{A} - \lambda\vec{I}) & = 0                     \\
              (\lambda - 1)(\lambda - 3) + 0 & = 0                   &
              \{\lambda_i\}                  & = \{1, 3\}
          \end{align}
          Finding the eigenvectors,
          \begin{align}
              \lambda_1                    & = \color{y_h} 1 &
              \bmattt{0}{2}{0}{2} \vec{x}  & = \vec{0}       &
              \vec{u}_1                    & = \color{y_h}
              \bmatcol{1}{0}                                   \\
              \lambda_2                    & = \color{y_p} 3 &
              \bmattt{-2}{2}{0}{0} \vec{x} & = \vec{0}       &
              \vec{u}_1                    & = \color{y_p}
              \bmatcol{1}{1}
          \end{align}

    \item Finding the eigenvalues,
          \begin{align}
              \vec{A}                        & = \bmattt{0}{1}{0}{0} &
              \det(\vec{A} - \lambda\vec{I}) & = 0                     \\
              \lambda^2                      & = 0                   &
              \{\lambda_i\}                  & = \{0, 0\}
          \end{align}
          Finding the eigenvectors,
          \begin{align}
              \lambda_1                   & = \color{y_h} 0 &
              \bmattt{0}{1}{0}{0} \vec{x} & = \vec{0}       &
              \vec{u}_1                   & = \color{y_h}
              \bmatcol{1}{0}
          \end{align}

    \item Finding the eigenvalues,
          \begin{align}
              \vec{A}                              & = \bmattt{a}{b}{-b}{a} &
              \det(\vec{A} - \lambda\vec{I})       & = 0                      \\
              (\lambda - a)^2 + b^2                & = 0                    &
              \lambda^2 - 2a \lambda + (a^2 + b^2) & = 0                      \\
              \{\lambda_i\}                        & = \{a \pm bi\}
          \end{align}
          Finding the eigenvectors,
          \begin{align}
              \lambda_1                        & = \color{y_h} a - bi &
              \bmattt{bi}{b}{-b}{bi} \vec{x}   & = \vec{0}            &
              \vec{u}_1                        & = \color{y_h}
              \bmatcol{1}{-i}                                           \\
              \lambda_2                        & = \color{y_p} a + bi &
              \bmattt{-bi}{b}{-b}{-bi} \vec{x} & = \vec{0}            &
              \vec{u}_1                        & = \color{y_p}
              \bmatcol{1}{i}
          \end{align}

    \item Finding the eigenvalues,
          \begin{align}
              \vec{A}                        & = \bmattt{0.8}{-0.6}{0.6}{0.8} &
              \det(\vec{A} - \lambda\vec{I}) & = 0                              \\
              (\lambda - 0.8)^2 + (0.6)^2    & = 0                            &
              \lambda^2 - 1.6 \lambda + 1    & = 0                              \\
              \{\lambda_i\}                  & = \{0.8 \pm 0.6i\}
          \end{align}
          Finding the eigenvectors,
          \begin{align}
              \lambda_1                                & = \color{y_h} 0.8 - 0.6i &
              \bmattt{0.6i}{-0.6}{0.6}{0.6i} \vec{x}   & = \vec{0}                &
              \vec{u}_1                                & = \color{y_h}
              \bmatcol{1}{i}                                                        \\
              \lambda_2                                & = \color{y_p} 0.8 + 0.6i &
              \bmattt{-0.6i}{-0.6}{0.6}{-0.6i} \vec{x} & = \vec{0}                &
              \vec{u}_1                                & = \color{y_p}
              \bmatcol{1}{-i}
          \end{align}

    \item Finding the eigenvalues,
          \begin{align}
              \vec{A}                                   & = \bmattt{\cos \theta}
              {-\sin \theta} {\sin \theta}{\cos \theta} &
              \det(\vec{A} - \lambda\vec{I})            & = 0                      \\
              (\lambda - \cos \theta)^2
              + (\sin \theta)^2                         & = 0                    &
              \lambda^2 - 2\lambda \cos \theta + 1      & = 0                      \\
              \{\lambda_i\}                             & =
              \{\cos \theta \pm i\ \sin \theta\}
          \end{align}
          Finding the eigenvectors,
          \begin{align}
              \lambda_1                              & = \color{y_h}
              \cos \theta - i\ \sin \theta           &
              \bmattt{i\ \sin \theta}{-\sin \theta}
              {\sin \theta}{i\ \sin \theta} \vec{x}  & = \vec{0}     &
              \vec{u}_1                              & = \color{y_h}
              \bmatcol{1}{i}                                           \\
              \lambda_2                              & = \color{y_p}
              \cos \theta + i\ \sin \theta           &
              \bmattt{-i\ \sin \theta}{-\sin \theta}
              {\sin \theta}{-i\ \sin \theta} \vec{x} & = \vec{0}     &
              \vec{u}_1                              & = \color{y_p}
              \bmatcol{1}{-i}
          \end{align}

    \item Finding the eigenvalues,
          \begin{align}
              \vec{A}                                    & =
              \begin{bNiceMatrix}[r, margin]
                  6  & 2 & -2 \\
                  2  & 5 & 0  \\
                  -2 & 0 & 7
              \end{bNiceMatrix}             &
              \det(\vec{A} - \lambda \vec{I})            & = 0             \\
              \lambda^3 - 18\lambda^2 + 99 \lambda - 162 & = 0           &
              \{\lambda_i\}                              & = \{3, 6, 9\}
          \end{align}
          Finding the eigenvectors,
          \begin{align}
              \lambda_1                      & = \color{y_h} 3 &
              \begin{bNiceMatrix}[r, margin]
                  3  & 2 & -2 \\
                  2  & 2 & 0  \\
                  -2 & 0 & 4
              \end{bNiceMatrix} \vec{x} & = 0             &
              \vec{u}_1                      & =
              \color{y_h} \begin{bNiceMatrix}[r, margin]
                              2 \\ -2 \\ 1
                          \end{bNiceMatrix}          \\
              \lambda_2                      & = \color{y_p} 6 &
              \begin{bNiceMatrix}[r, margin]
                  0  & 2  & -2 \\
                  2  & -1 & 0  \\
                  -2 & 0  & 1
              \end{bNiceMatrix} \vec{x} & = 0             &
              \vec{u}_2                      & =
              \color{y_p} \begin{bNiceMatrix}[r, margin]
                              1 \\ 2 \\ 2
                          \end{bNiceMatrix}          \\
              \lambda_3                      & = \color{y_t} 6 &
              \begin{bNiceMatrix}[r, margin]
                  -3 & 2  & -2 \\
                  2  & -4 & 0  \\
                  -2 & 0  & -2
              \end{bNiceMatrix} \vec{x} & = 0             &
              \vec{u}_3                      & =
              \color{y_t} \begin{bNiceMatrix}[r, margin]
                              -2 \\ -1 \\ 2
                          \end{bNiceMatrix}
          \end{align}

    \item Finding the eigenvalues,
          \begin{align}
              \vec{A}                                  & =
              \begin{bNiceMatrix}[r, margin]
                  3 & 5 & 3 \\
                  0 & 4 & 6 \\
                  0 & 0 & 1
              \end{bNiceMatrix}           &
              \det(\vec{A} - \lambda \vec{I})          & = 0             \\
              \lambda^3 - 8\lambda^2 + 19 \lambda - 12 & = 0           &
              \{\lambda_i\}                            & = \{1, 3, 4\}
          \end{align}
          Finding the eigenvectors,
          \begin{align}
              \lambda_1                      & = \color{y_h} 1 &
              \begin{bNiceMatrix}[r, margin]
                  2 & 5 & 3 \\
                  0 & 3 & 6 \\
                  0 & 0 & 0
              \end{bNiceMatrix} \vec{x} & = 0             &
              \vec{u}_1                      & =
              \color{y_h} \begin{bNiceMatrix}[r, margin]
                              7 \\ -4 \\ 2
                          \end{bNiceMatrix}          \\
              \lambda_2                      & = \color{y_p} 3 &
              \begin{bNiceMatrix}[r, margin]
                  0 & 5 & 3  \\
                  0 & 1 & 6  \\
                  0 & 0 & -2
              \end{bNiceMatrix} \vec{x} & = 0             &
              \vec{u}_2                      & =
              \color{y_p} \begin{bNiceMatrix}[r, margin]
                              1 \\ 0 \\ 0
                          \end{bNiceMatrix}          \\
              \lambda_3                      & = \color{y_t} 4 &
              \begin{bNiceMatrix}[r, margin]
                  -1 & 5 & 3  \\
                  0  & 0 & 6  \\
                  0  & 0 & -3
              \end{bNiceMatrix} \vec{x} & = 0             &
              \vec{u}_3                      & =
              \color{y_t} \begin{bNiceMatrix}[r, margin]
                              5 \\ 1 \\ 0
                          \end{bNiceMatrix}
          \end{align}

    \item Finding the eigenvalues,
          \begin{align}
              \vec{A}                                     & =
              \begin{bNiceMatrix}[r, margin]
                  13 & 5 & 2  \\
                  2  & 7 & -8 \\
                  5  & 4 & 7
              \end{bNiceMatrix}              &
              \det(\vec{A} - \lambda \vec{I})             & = 0             \\
              \lambda^3 - 27\lambda^2 + 243 \lambda - 729 & = 0           &
              \{\lambda_i\}                               & = \{9, 9, 9\}
          \end{align}
          Finding the eigenvectors,
          \begin{align}
              \lambda_1                      & = \color{y_h} 1 &
              \begin{bNiceMatrix}[r, margin]
                  4 & 5  & 2  \\
                  2 & -2 & -8 \\
                  5 & 4  & -2
              \end{bNiceMatrix} \vec{x} & = 0             &
              \vec{u}_1                      & =
              \color{y_h} \begin{bNiceMatrix}[r, margin]
                              2 \\ -2 \\ 1
                          \end{bNiceMatrix}
          \end{align}

    \item Finding the eigenvalues,
          \begin{align}
              \vec{A}                                     & =
              \begin{bNiceMatrix}[r, margin]
                  2 & 0   & -1 \\
                  0 & 0.5 & 0  \\
                  1 & 0   & 4
              \end{bNiceMatrix}              &
              \det(\vec{A} - \lambda \vec{I})             & = 0               \\
              \lambda^3 - 6.5\lambda^2 + 12 \lambda - 4.5 & = 0             &
              \{\lambda_i\}                               & = \{0.5, 3, 3\}
          \end{align}
          Finding the eigenvectors,
          \begin{align}
              \lambda_1                      & = \color{y_h} 0.5 &
              \begin{bNiceMatrix}[r, margin]
                  1.5 & 0 & -1  \\
                  0   & 0 & 0   \\
                  1   & 0 & 3.5
              \end{bNiceMatrix} \vec{x} & = 0               &
              \vec{u}_1                      & =
              \color{y_h} \begin{bNiceMatrix}[r, margin]
                              0 \\ 1 \\ 0
                          \end{bNiceMatrix}            \\
              \lambda_2                      & = \color{y_p} 3   &
              \begin{bNiceMatrix}[r, margin]
                  -1 & 0    & -1  \\
                  0  & -2.5 & 0   \\
                  1  & 0    & 0.5
              \end{bNiceMatrix} \vec{x} & = 0               &
              \vec{u}_2                      & =
              \color{y_p} \begin{bNiceMatrix}[r, margin]
                              -1 \\ 0 \\ 1
                          \end{bNiceMatrix}
          \end{align}

    \item Finding the eigenvalues,
          \begin{align}
              \vec{A}                         & =
              \begin{bNiceMatrix}[r, margin]
                  -1 & 0  & 12 & 0  \\
                  0  & -1 & 0  & 12 \\
                  0  & 0  & -1 & -4 \\
                  0  & 0  & -4 & -1
              \end{bNiceMatrix}  &
              \det(\vec{A} - \lambda \vec{I}) & = 0                                    \\
              0                               & = \lambda^4 + 4\lambda^3 - 10\lambda^2
              - 28 \lambda - 15               &
              \{\lambda_i\}                   & = \{-5, -1, -1, 3\}
          \end{align}
          Finding the eigenvectors,
          \begin{align}
              \lambda_1                      & = \color{y_h} -5 &
              \begin{bNiceMatrix}[r, margin]
                  4 & 0 & 12 & 0  \\
                  0 & 4 & 0  & 12 \\
                  0 & 0 & 4  & -4 \\
                  0 & 0 & -4 & 4
              \end{bNiceMatrix} \vec{x} & = 0              &
              \vec{u}_1                      & =
              \color{y_h} \begin{bNiceMatrix}[r, margin]
                              -3 \\ -3 \\ 1 \\ 1
                          \end{bNiceMatrix}              \\
              \lambda_2                      & = \color{y_p} -1 &
              \begin{bNiceMatrix}[r, margin]
                  0 & 0 & 12 & 0  \\
                  0 & 0 & 0  & 12 \\
                  0 & 0 & 0  & -4 \\
                  0 & 0 & -4 & 0
              \end{bNiceMatrix} \vec{x} & = 0              &
              \vec{u}_2                      & =
              {\color{y_p} \begin{bNiceMatrix}[r, margin]
                               1 \\ 0 \\ 0 \\ 0
                           \end{bNiceMatrix}},\
              \vec{u}_3 = {\color{y_p} \begin{bNiceMatrix}[r, margin]
                                           0 \\ 1 \\ 0 \\ 0
                                       \end{bNiceMatrix}} \\
              \lambda_3                      & = \color{y_t} 3  &
              \begin{bNiceMatrix}[r, margin]
                  -4 & 0  & 12 & 0  \\
                  0  & -4 & 0  & 12 \\
                  0  & 0  & -4 & -4 \\
                  0  & 0  & -4 & -4
              \end{bNiceMatrix} \vec{x} & = 0              &
              \vec{u}_4                      & =
              \color{y_t} \begin{bNiceMatrix}[r, margin]
                              -3 \\ 3 \\ -1 \\ 1
                          \end{bNiceMatrix}
          \end{align}

    \item Finding the eigenvalues,
          \begin{align}
              \vec{A}                         & =
              \begin{bNiceMatrix}[r, margin]
                  -3 & 0 & 4  & 2  \\
                  0  & 1 & -2 & 4  \\
                  2  & 4 & -1 & -2 \\
                  0  & 2 & -2 & 3
              \end{bNiceMatrix}  &
              \det(\vec{A} - \lambda \vec{I}) & = 0                       \\
              0                               & = \lambda^4 - 22\lambda^2
              + 24 \lambda + 45               &
              \{\lambda_i\}                   & = \{-5, -1, 3, 3\}
          \end{align}
          Finding the eigenvectors,
          \begin{align}
              \lambda_1                      & = \color{y_h} -5 &
              \begin{bNiceMatrix}[r, margin]
                  2 & 0 & 4  & 2  \\
                  0 & 6 & -2 & 4  \\
                  2 & 4 & 4  & -2 \\
                  0 & 2 & -2 & 8
              \end{bNiceMatrix} \vec{x} & = 0              &
              \vec{u}_1                      & =
              \color{y_h} \begin{bNiceMatrix}[r, margin]
                              -11 \\ 1 \\ 5 \\ 1
                          \end{bNiceMatrix}           \\
              \lambda_2                      & = \color{y_p} -1 &
              \begin{bNiceMatrix}[r, margin]
                  -2 & 0 & 4  & 2  \\
                  0  & 2 & -2 & 4  \\
                  2  & 4 & 0  & -2 \\
                  0  & 2 & -2 & 4
              \end{bNiceMatrix} \vec{x} & = 0              &
              \vec{u}_2                      & = \color{y_p}
              \begin{bNiceMatrix}[r, margin]
                  3 \\ -1 \\ 1 \\ 1
              \end{bNiceMatrix}                       \\
              \lambda_3                      & = \color{y_t} 3  &
              \begin{bNiceMatrix}[r, margin]
                  -6 & 0  & 4  & 2  \\
                  0  & -2 & -2 & 4  \\
                  2  & 4  & -4 & -2 \\
                  0  & 2  & -2 & 0
              \end{bNiceMatrix} \vec{x} & = 0              &
              \vec{u}_4                      & =
              \color{y_t} \begin{bNiceMatrix}[r, margin]
                              1 \\ 1 \\ 1 \\ 1
                          \end{bNiceMatrix}
          \end{align}

    \item The matrix corresponding to the linear transform is,
          \begin{align}
              \vec{A}                        & = \bmattt{0}{-1}{1}{0} &
              \det(\vec{A} - \lambda\vec{I}) & = 0                      \\
              \lambda^2 + 1                  & = 0                    &
              \{\lambda_i\}                  & = \{0 \pm i\}
          \end{align}
          Finding the eigenvectors,
          Finding the eigenvectors,
          \begin{align}
              \lambda_1                      & = \color{y_h} 0 - i &
              \bmattt{i}{-1}{1}{i} \vec{x}   & = \vec{0}           &
              \vec{u}_1                      & = \color{y_h}
              \bmatcol{1}{i}                                         \\
              \lambda_2                      & = \color{y_p} 0 + i &
              \bmattt{-i}{-1}{1}{-i} \vec{x} & = \vec{0}           &
              \vec{u}_1                      & = \color{y_p}
              \bmatcol{i}{1}
          \end{align}
          Since the eigenvectors are complex, there exists no vector in $ \mathcal{R}^2 $
          whose direction is preserved under this transform.

    \item The matrix corresponding to the linear transform is,
          \begin{align}
              \vec{A}                        & = \bmattt{-1}{0}{0}{1} &
              \det(\vec{A} - \lambda\vec{I}) & = 0                      \\
              \lambda^2 - 1                  & = 0                    &
              \{\lambda_i\}                  & = \{-1, 1\}
          \end{align}
          Finding the eigenvectors,
          Finding the eigenvectors,
          \begin{align}
              \lambda_1                    & = \color{y_h} -1 &
              \bmattt{0}{0}{0}{2} \vec{x}  & = \vec{0}        &
              \vec{u}_1                    & = \color{y_h}
              \bmatcol{1}{0}                                    \\
              \lambda_2                    & = \color{y_p} 1  &
              \bmattt{-2}{0}{0}{0} \vec{x} & = \vec{0}        &
              \vec{u}_1                    & = \color{y_p}
              \bmatcol{0}{1}
          \end{align}
          Any vectors coinciding with the $ x_1 $ axis reflect onto themselves. \par
          Any vectors coinciding with the $ x_2 $ axis reflect onto their additive
          inverse. (Since this is Cartesian $ \mathcal{R}^2 $)

    \item The matrix corresponding to the linear transform is,
          \begin{align}
              \vec{A}                        & = \bmattt{0}{0}{0}{1} &
              \det(\vec{A} - \lambda\vec{I}) & = 0                     \\
              \lambda^2 - \lambda            & = 0                   &
              \{\lambda_i\}                  & = \{0, 1\}
          \end{align}
          Finding the eigenvectors,
          Finding the eigenvectors,
          \begin{align}
              \lambda_1                    & = \color{y_h} 0 &
              \bmattt{0}{0}{0}{1} \vec{x}  & = \vec{0}       &
              \vec{u}_1                    & = \color{y_h}
              \bmatcol{1}{0}                                   \\
              \lambda_2                    & = \color{y_p} 1 &
              \bmattt{-1}{0}{0}{0} \vec{x} & = \vec{0}       &
              \vec{u}_1                    & = \color{y_p}
              \bmatcol{0}{1}
          \end{align}
          Any vectors coinciding with the $ x_1 $ axis project to the zero vector. \par
          Any vectors coinciding with the $ x_2 $ axis project onto themselves

    \item The projection onto the plane is obtained by subtracting the projection onto
          the normal vector from the vector itself.
          \begin{align}
              \vec{v}                        & =
              \vec{a} - (\vec{a} \dotp \vec{n})\ \vec{n}
                                             &
              \vec{n}                        & =
              \begin{bNiceMatrix}[r, margin]
                  \frac{1}{\sqrt{2}}  \\
                  \frac{-1}{\sqrt{2}} \\
                  0
              \end{bNiceMatrix} \\
              \vec{v}                        & =
              \begin{bNiceMatrix}[r, margin]
                  v_1 \\ v_2 \\ v_3
              \end{bNiceMatrix} -
              \begin{bNiceMatrix}[r, margin]
                  0.5v_1 - 0.5v_2 \\
                  0.5v_2 - 0.5v_1 \\
                  0
              \end{bNiceMatrix} &
                                             & =
              \begin{bNiceMatrix}[r, margin]
                  0.5v_1 + 0.5v_2 \\
                  0.5v_1 + 0.5v_2 \\
                  v_3
              \end{bNiceMatrix}
          \end{align}

          The matrix corresponding to the linear transform is,
          \begin{align}
              \vec{A}                          & =
              \begin{bNiceMatrix}[r, margin]
                  0.5 & 0.5 & 0 \\
                  0.5 & 0.5 & 0 \\
                  0   & 0   & 1
              \end{bNiceMatrix}   &
              \det(\vec{A} - \lambda \vec{I})  & = 0             \\
              \lambda^3 - 2\lambda^2 + \lambda & = 0           &
              \{\lambda_i\}                    & = \{0, 1, 1\}
          \end{align}
          Finding the eigenvectors,
          \begin{align}
              \lambda_1                      & = \color{y_h} 0 &
              \begin{bNiceMatrix}[r, margin]
                  0.5 & 0.5 & 0 \\
                  0.5 & 0.5 & 0 \\
                  0   & 0   & 1
              \end{bNiceMatrix} \vec{x} & = 0             &
              \vec{u}_1                      & =
              \color{y_h} \begin{bNiceMatrix}[r, margin]
                              1 \\ -1 \\ 0
                          \end{bNiceMatrix}          \\
              \lambda_2                      & = \color{y_p} 1 &
              \begin{bNiceMatrix}[r, margin]
                  -0.5 & 0.5  & 0 \\
                  0.5  & -0.5 & 0 \\
                  0    & 0    & 0
              \end{bNiceMatrix} \vec{x} & = 0             &
              \vec{u}_2                      & =
              {\color{y_p} \begin{bNiceMatrix}[r, margin]
                               0 \\ 0 \\ 1
                           \end{bNiceMatrix}},\
              \vec{u}_3 = {\color{y_t} \begin{bNiceMatrix}[r, margin]
                                           1 \\ 1 \\ 0
                                       \end{bNiceMatrix}}
          \end{align}
          Any vectors in the plane get mapped onto themselves. \par
          Any vectors along the normal to the plane $ \vec{n} $ get mapped to the zero
          vector.

    \item Defect is the difference between algebraic and geometric multiplicity.
          \begin{align}
              \Delta_\lambda & = M_\lambda - m_\lambda
          \end{align}
          From this problem set, 7, 13, 14, 16 have nonzero defect.

    \item From this problem set 1, 3, 4, 11, 12 have multiple eigenvalues, all of
          which are distinct.

    \item Eigenvalues of a real matrix are roots of a polynomial with real coefficients.
          Since these roots are either real or complex conjugate pairs, the statemenet
          is true.

    \item Starting with the characteristic polynomial,
          \begin{align}
              \det(\vec{A} - \lambda \vec{I}) & = P(\lambda)                &
              P(\lambda = 0)                  & = \prod_{i=1}^{n} \lambda_i   \\
              \prod_{i=1}^{n} \lambda_i       & = \det(\vec{A})
          \end{align}
          Thus, the product of eigenvalues is equal to the determinant. If at least one
          eigenvalue is zero, $\vec{A}$ is non-invertible. \par
          Conversely, if $ \vec{A} $ is non-invertible, then it is singular and at least
          one eigenvalue is zero.
          \begin{align}
              {\color{y_h} \vec{A}}\vec{v}       & = {\color{y_h}
              \lambda} \vec{v}                   &
              \vec{A}^{-1}\ \vec{Av}             & = \vec{A}^{-1}
              \ \lambda \vec{v} = \vec{v}                         \\
              {\color{y_p} \vec{A}^{-1}} \vec{v} & = {\color{y_p}
              \frac{1}{\lambda}}\ \vec{v}
          \end{align}
          If $ \vec{A} $ has eigenvector $ \lambda $, then $ \vec{A}^{-1} $ has
          corresponding eigenvector $ \lambda^{-1} $.

    \item $ \vec{A}^T $ has the same characteristic polynomial as $ \vec{A} $. All
          further steps are identical, leading to identical eigenvalues. Examples TBC.
\end{enumerate}