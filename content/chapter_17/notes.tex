\chapter{Conformal Mapping}

\section{Geometry of Analytic Functions: Conformal Mapping}

\begin{description}
    \item[Complex function] A mapping from a complex variable to another.
        \begin{align}
            w & \equiv f(z) = u(x, y) + \i\ v(x, y) &
            z & = x + \i\ y
        \end{align}
        The domain of definition $ D $ of this function maps onto the range $ R $, both
        of which are complex planes.

    \item[Mapping terminology] A map is
        \begin{itemize}
            \item surjective, if every element of $ R $ is the image of at least
                  one element in $ D $.
            \item injective, if different elements of $ D $ have different images
                  in $ R $.
            \item bijective, if it is both of the above.
        \end{itemize}

        The result of a mapping function acting on a point $ z_0 $ is called its image
        $ f(z_0) $.
        \begin{align}
            w_0 & = f(z_0)
        \end{align}

    \item[Conformal map] A mapping that preserves angles between oriented curves in
        direction as well as sense. \par
        The mapping by an analytic function is conformal, except at its critical points
        (where the derivative $ f' $ is zero).
        \begin{figure}[H]
            \centering
            \begin{tikzpicture}
                \begin{axis}[set layers,
                        width = 8cm, height = 8cm, title = Domain,
                        xmin = 0, xmax = 1.2, ymin = 0, ymax = 2, grid = both,
                        xlabel = \normalsize $ \Re{z} $, ylabel = \normalsize $ \Im{z} $,
                        axis equal,
                        ytick = {0.577, 1.732,1},
                        xtick = {0,1},
                        Ani]
                    \draw [draw = black!0, fill = black, fill opacity = 0.08]
                    (0,0.577) -- (1, 0.577) -- (1, 1.732) -- (0, 1.732)
                    -- cycle;
                    \draw [very thick, y_h] (0, 0.577) -- (0, 1.732);
                    \draw [very thick, y_p] (1, 0.577) -- (1, 1.732);
                    \draw [very thick, y_t] (0, 0.577) -- (1, 0.577);
                    \draw [very thick, azure4] (0, 1.732) -- (1, 1.732);
                \end{axis}
            \end{tikzpicture}
            \begin{tikzpicture}
                \begin{axis}[set layers,
                        width = 8cm, height = 8cm, title = {Image under $ e^z $},
                        xmin = 0, xmax = 3, ymin = 0, ymax = 3, grid = both,
                        xlabel = \normalsize $ \Re{w} $, ylabel = \normalsize $ \Im{w} $,
                        axis equal, xtick = {1,2}, ytick = {1,2},
                        ytick = {1,2.71},yticklabels={\normalsize 1, \normalsize $ e $},
                        xtick = {1,2.71},xticklabels={\normalsize 1, \normalsize $ e $},
                        Ani]
                    \draw [draw = black!0, fill = black, fill opacity = 0.08]
                    (30:2.71)--(30:1) arc (30:60:1) -- (60:2.71) arc(60:30:2.71)
                    -- cycle;
                    \draw [draw = black, dotted, thick] (0:1) arc (0:90:1);
                    \draw [draw = black, dotted, thick] (0:2.71) arc (0:90:2.71);
                    \addplot[GraphSmooth, dotted, thick, black, domain= 0:4,
                        variable=\t]
                    ({\t*cos(pi/6)}, {\t*sin(pi/6)});
                    \addplot[GraphSmooth, dotted, thick, black, domain= 0:4,
                        variable=\t]
                    ({\t * cos(pi/3)}, {\t * sin(pi/3)});
                    \addplot[GraphSmooth, very thick, y_h, domain= pi/6:pi/3, variable=\t]
                    ({cos(\t)}, {sin(\t)});
                    \addplot[GraphSmooth, very thick, y_p, domain= pi/6:pi/3, variable=\t]
                    ({2.71 * cos(\t)}, {2.71 * sin(\t)});
                    \addplot[GraphSmooth, very thick, y_t, domain= 1:2.71, variable=\t]
                    ({\t*cos(pi/6)}, {\t*sin(pi/6)})
                    node[midway, below, sloped]{$ \phi = \pi/6 $};
                    \addplot[GraphSmooth, very thick, azure4, domain= 1:2.71, variable=\t]
                    ({\t * cos(pi/3)}, {\t * sin(pi/3)})
                    node[midway, above, sloped]{$ \phi = \pi/3 $};
                \end{axis}
            \end{tikzpicture}
        \end{figure}

    \item[Magnification ratio] The ratio of the lenghts of a line segment in the domain
        $ D $ to the length of the corresponding image in $ R $.
        \begin{align}
            M & = \lim_{z \to z_0} \abs{\frac{f(z) - f(z_0)}{z - z_0}}
            = \abs{f'(z_0)}
        \end{align}

    \item[Jacobian] The condition $ f'(z)  \neq 0$ is equivalent to a nonzero
        Jacobian at $ z = z_0 $.
        \begin{align}
            \abs{f'(z)}^2 & = \begin{vNiceMatrix}[margin]
                                  \difcp ux & \difcp uy \\
                                  \difcp vx & \difcp vy \\
                              \end{vNiceMatrix} = \frac{\partial(u, v)}{\partial(x, y)}
        \end{align}
        This requires the Cauchy-Riemann equations and
        \begin{align}
            f'(z)         & = \diffp ux + \i\ \diffp vx                             &
            \abs{f'(z)}^2 & = (\difcp ux)^2 + (\difcp uy)^2                           \\
            \abs{f'(z)}^2 & = \difcp ux \cdot \difcp vy - \difcp uy \cdot \difcp vx
        \end{align}

\end{description}

\section{Linear Fractional Transformations (Mobius Transformations)}

\begin{description}
    \item[Utility] Mobius transformations help map one kind of region into another in
        order ease the solution of BVPs.
        \begin{align}
            w & = \frac{az + b}{cz + d} & ad & \neq bc
        \end{align}
        Where $ a,b,c,d $ are complex numbers.

    \item[Special Mobius transformations] Of particular interest in BVPs are
        \begin{align}
            w & = z + b                  &  & \text{Translation}                  \\
            w & = az, \qquad \abs{a} = 1 &  & \text{Rotation}                     \\
            w & = az + b                 &  & \text{Linear transformation}        \\
            w & = \frac{1}{z}            &  & \text{Inversion in the unit circle}
        \end{align}
        All Mobius transformations are a composite of these four special maps.

    \item[Circles and straight lines] Every linear fractional transformation maps the
        totality of circles and straight lines in $ \mathcal{C} $ onto the totality of
        circles and straight lines in $ \mathcal{C^*} $. \par
        Here, the plane $ C^* $ is the plane containing the image of the map.

    \item[Extended complex plane] The Mobius transformation is injective except for
        points of the form $ z = -d/c $. The special point $ \infty $ is now defined as
        the extra point in $ \mathcal{C^*} $ that is the image of this special $ z $.
        (assuming $ c \neq 0 $)

    \item[Inverse mapping] The inverse of the standard Mobius transformation is,
        \begin{align}
            z & = \frac{dw - b}{-cw + a}
        \end{align}
        Now, for $ c \neq 0 $, the special point $ w = a/c $ maps onto $ \infty $ in the
        original $ \mathcal{C} $ plane. Taken together with the above result,
        Mobius transformations map the extended complex plane onto iself.

    \item[Fixed points] Points that are mapped onto themselves.
        \begin{align}
            f(z) = w = z
        \end{align}
        The special identity map $ f(z) = z $ has the entire complex plane as its
        fixed point. \par
        A mobius transformation (that is not the identity map) has at most two fixed
        points. If the mobius map is known to have three or more fixed points, it has to
        be the identity map.
\end{description}

\section{Special Linear Fractional Transformations}

\begin{description}
    \item[Three points to map] Three given distinct points $ z_1,z_2,z_3 $
        can always be mapped onto three prescribed distinct points $ w_1,w_2,w_3 $,
        by one and only one L.F.T.
        \begin{align}
            \Bigg( \frac{w - w_1}{w - w_3} \Bigg) \Bigg( \frac{w_2 - w_3}
            {w_2 - w_1} \Bigg) & = \Bigg( \frac{z - z_1}{z - z_3} \Bigg)
            \Bigg( \frac{z_2 - z_3}{z_2 - z_1} \Bigg)
        \end{align}
        Any fraction containing $ \infty $ is evaluated to $ 1 $.

    \item[Standard domains] Start with choosing the points $ z_1,z_2,z_3 $ on the
        boudnary of the domain and then the points $ w_1, w_2, w_3 $ on the boundary of
        the image $ D^* $. \par
        To map a unit disk centered at $ z_0 $ onto another unit disk centered at the
        origin,
        \begin{align}
            w & = \frac{(z - z_0)}{cz - 1} & c & = \overline{z_0} \qquad \abs{z_0} < 1
        \end{align}
        To map the upper half plane onto the unit circle,
        \begin{align}
            w & = \frac{z - \i}{-\i z + 1}
        \end{align}

    \item[Cayley transform] A special transform that maps the real line onto the
        unit circle and the upper half plane onto the interior of the unit circle.
        \begin{align}
            w & = \frac{z - \i}{z + \i}
        \end{align}
\end{description}

\section{Conformal Mapping by Other Functions}

\begin{description}
    \item[Sine function] The mapping uses the fundamental period
        $ x \in [-\pi/2, \pi/2] $.
        \begin{align}
            w         & = \sin z                              &
            u + \i\ v & = \sin x \cosh y + \i\ \cos x \sinh y
        \end{align}
        The mapping is not conformal at the critical points $ \pm \pi/2 $. \par
        In the figure, vertical lines in the
        \textcolor{y_h}{upper} and \textcolor{y_p}{lower} half plane, map onto
        hyperbolas. \par
        Whereas, the horizontal lines in the \textcolor{y_t}{right} and
        \textcolor{azure4}{left} half planes, map onto ellipses.
        \begin{figure}[H]
            \centering
            \begin{tikzpicture}
                \begin{axis}[
                        title = {mapping a grid under $ \sin z $},
                        grid = both, Ani,
                        width = 8cm, height = 8cm, enlargelimits = false,
                        axis equal,
                    ]
                    \foreach \k in {-0.4,-0.3,-0.2,-0.1,0,0.1,0.2,0.3,0.4}
                        {
                            \edef\temp{%
                                \noexpand \addplot[GraphSmooth, semithick,
                                    y_h, domain = 0:1.5]
                                ({sin(\k*pi)*cosh(x)}, {cos(\k*pi)*sinh(x)});
                                \noexpand \addplot[GraphSmooth, semithick,
                                    y_p, domain = -1.5:0]
                                ({sin(\k*pi)*cosh(x)}, {cos(\k*pi)*sinh(x)});
                                \noexpand \addplot[GraphSmooth, semithick,
                                    y_t, domain = 0:1.5]
                                ({sin(x)*cosh(\k*pi)}, {cos(x)*sinh(\k*pi)});
                                \noexpand \addplot[GraphSmooth, semithick,
                                    azure4, domain = -1.5:0]
                                ({sin(x)*cosh(\k*pi)}, {cos(x)*sinh(\k*pi)});
                            }\temp
                        }
                \end{axis}
            \end{tikzpicture}
        \end{figure}
        The other functions are derived from this map using,
        \begin{align}
            \cos z   & = \sin(z + \pi/2) & \sinh(z) & = -\i\ \sin(\i z) \\
            \cosh(z) & = \cos(\i z)
        \end{align}

    \item[Tangent function] The function $ f(z) = \tan(z) $ is similarly, a
        composition of L.F.T.s and an exponentiation.
        \begin{align}
            \tan(z) & = -\i\ \frac{e^{2\i z} - 1}{e^{2\i z} + 1} &
            w_1     & = e^{2\i z}                                  \\
            w_2     & = \frac{w_1 - 1}{w_1 + 1}                  &
            w       & = \i\ w_2
        \end{align}
        This maps the infinite vertical strip $ x \in (-\pi/4, \pi/4) $ onto the
        unit circle.

\end{description}

\section{Riemann Surfaces}

\begin{description}
    \item[Utility] Consider maps such as $ w = z^2 $, which cover the $ w $ plane
        twice over when traversing the entire $ z $ plane. \par
        Multiple passes of the $ w $ plane can be differentiated in order to make the
        many-to-one function a one-to-one function.

    \item[Branch point] A point in the $ w $ plane that is the image of just one
        point in the $ z $ plane.

    \item
\end{description}