\section{Riemann Surfaces}

\begin{enumerate}
    \item The function is double valued, with branch point $ z = 0 $,
          \begin{align}
              \abs{w} & = \frac{1}{2} & \Arg{w}             & = \frac{\Arg{z}}{2} \\
              \theta  & \in [0, 4\pi] & \implies \quad \phi & \in [0, 2\pi]
          \end{align}
          This means that $ w $ moves once around the circle $ \abs{w} = 0.5 $.

    \item Using the given substitution,
          \begin{align}
              z - 1  & = r_1 \exp(\i \alpha)                        &
              z - 2  & = r_2 \exp(\i \beta)                           \\
              f(z)   & = \sqrt{r_1r_2}\ \exp[0.5\i(\alpha + \beta)] &
              \alpha & \to \alpha + 2\pi                              \\
              \beta  & \to \beta + 2\pi                             &
              \implies \quad \frac{\alpha + \beta}{2}
                     & \to \frac{\alpha + \beta}{2} + 2\pi
          \end{align}
          Branch points are at $ z = 1, z = 2 $ since $ w = 0 $ is the image of only
          one point $ z = 0 $.

    \item Four sheets (since the function is 4-valued originally) with a branch
          point at $ z = -1 $.

    \item Finding the branch points, using the fact that $ z = 0 $ only maps to one
          image.
          \begin{align}
              f(z) & = \sqrt{\i z - 2 + \i}                     &
              z^*  & = \frac{2 - \i}{\i} = \color{y_h} -1 - 2\i   \\
              s    & = \color{y_p} 2
          \end{align}

    \item Finding the branch points, using the fact that $ w^{1/3} $ has three
          images except for $ w = 0 $,
          \begin{align}
              f(z) & = z^2 + (4z + \i)^{1/3}     &
              z^*  & = \color{y_h} -\frac{\i}{4}   \\
              s    & = \color{y_p} 3
          \end{align}

    \item Finding the branch points, using the fact that $ \ln(w) $ has infinitely many
          images except for $ w = 1 $,
          \begin{align}
              f(z) & = \ln{(6z - 2\i)}               &
              z^*  & = \color{y_h} \frac{2 + 2\i}{6}   \\
              s    & = \color{y_p} \infty
          \end{align}

    \item Finding the branch points, using the fact that $ w^{1/n} $ has $ n $
          images except for $ w = 0 $,
          \begin{align}
              f(z) & = (z - z_0)^{1/n} &
              z^*  & = \color{y_h} z_0   \\
              s    & = \color{y_p} n
          \end{align}

    \item Finding the branch points, using the fact that $ \sqrt{w} $ has $ 2 $
          images except for $ w = 0 $,
          \begin{align}
              f(z) & = e^{\sqrt{z}}  &
              z^*  & = \color{y_h} 0   \\
              s    & = \color{y_p} 2
          \end{align}
          Using the fact that $ e^w $ is never zero, the second function has no
          branch points.

    \item Finding the branch points, using the fact that $ \sqrt{w} $ has $ 2 $
          images except for $ w = 0 $,
          \begin{align}
              f(z) & = \sqrt{z^3 + z}        &
              z^*  & = \color{y_h} 0, \pm \i   \\
              s    & = \color{y_p} 2
          \end{align}

    \item Finding the branch points, using the fact that $ \sqrt{w} $ has $ 2 $
          images except for $ w = 0 $,
          \begin{align}
              f(z) & = \sqrt{(4 - z^2)(1 - z^2)} &
              z^*  & = \color{y_h} \pm 1, \pm 2    \\
              s    & = \color{y_p} 2
          \end{align}
\end{enumerate}