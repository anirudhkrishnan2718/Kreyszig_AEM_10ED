\section{Qualitative Methods for Nonlinear Systems}

\begin{enumerate}
    \item Consider the closed trajectory intersecting with the two axes. This is an
          ellipse with major axis $ 2a $ and minor axis $ 2b $.
          \begin{table}[H]
              \centering
              \SetTblrInner{rowsep=0.5em}
              \begin{tblr}{colspec={Q[r]|Q[l]|Q[l]|Q[l]}, colsep = 2em}
                  \textbf{Intersection}    & \textbf{Position} $ y $ &
                  \textbf{Velocity} $ y' $ & \textbf{Trajectory}       \\ \hline[dotted]
                  $ +x $ axis              & $ a $                   &
                  $0$                      & Right edge                \\
                  $ -x $ axis              & $ -a $                  &
                  $0$                      & Left edge                 \\
                  $ +y $ axis              & $ 0 $                   &
                  $b$                      & Mean position \newline
                  traveling left                                       \\
                  $ -y $ axis              & $ 0 $                   &
                  $-b$                     & Mean position \newline
                  traveling right                                      \\ \hline
              \end{tblr}
          \end{table}
          For the open trajectory, the penduum never reverses angular direction.
          Maxima and minima in the open trajectory represent the bottom and top of
          the vertical plane circular trajectory. \par
          When the trajectory intersects the $ y_2 $ axis, the pendulum is at the mean
          position moving at maximum angular speed (because its potential energy is
          lowest at this point in the cyclic path).

    \item A limit cycle is the limiting set of a different trajectory that may
          not have had its initial conditions close to the limit cycle. The limit
          cycle is the steady state behaviour after all transient effects have worn
          off.\par
          A closed trajectory surrounding a center is constrained to start and always
          stay on this trajectory.

    \item System of ODEs for Van der Pol oscillator is,
          \begin{align}
              y_1' & = y_2 & y_2' & = \mu(1 - y_1^{2})y_2 - y_1
          \end{align}
          After simulating until steady state, the attractor is plotted here for
          different values of $ \mu $,
          \begin{figure}[H]
              \centering
              \pgfplotstableread[col sep=comma]{./tables/van_der_pol_many.csv}\anitable
              \begin{subfigure}[b]{0.48\textwidth}
                  \begin{tikzpicture}
                      \begin{axis}[
                              % xmin = -1, xmax = 1, ymin = -1, ymax = 1,
                              % restrict y to domain = -1:1,
                              xlabel = $ y_1 $,
                              ylabel = $ y_2 $,
                              axis equal,
                              width = 8cm,
                              legend pos = north west,
                              legend style={nodes={scale=0.75}},
                              grid = both,
                              Ani]
                          \addplot[GraphSmooth, color = red5] table[x index=0,y index=1,
                                  col sep=comma, ]{\anitable};
                          \addlegendentry{$\mu = 0.2$}
                          \addplot[GraphSmooth, color = yellow8] table[x index=2,y index=3,
                                  col sep=comma, ]{\anitable};
                          \addlegendentry{$\mu = 0.4$}
                          \addplot[GraphSmooth, color = green8] table[x index=4,y index=5,
                                  col sep=comma, ]{\anitable};
                          \addlegendentry{$\mu = 0.6$}
                          \addplot[GraphSmooth, color = blue3] table[x index=6,y index=7,
                                  col sep=comma, ]{\anitable};
                          \addlegendentry{$\mu = 0.8$}
                      \end{axis}
                  \end{tikzpicture}
              \end{subfigure}
              \hfill
              \begin{subfigure}[b]{0.48\textwidth}
                  \begin{tikzpicture}
                      \begin{axis}[
                              % xmin = -1, xmax = 1, ymin = -1, ymax = 1,
                              % restrict y to domain = -1:1,
                              xlabel = $ y_1 $,
                              ylabel = $ y_2 $,
                              axis equal,
                              width = 8cm,
                              legend pos = north west,
                              legend style={nodes={scale=0.75}},
                              grid = both,
                              Ani]
                          \addplot[GraphSmooth, color = red5] table[x index=8,y index=9,
                                  col sep=comma, ]{\anitable};
                          \addlegendentry{$\mu = 1.0$}
                          \addplot[GraphSmooth, color = green8] table[x index=10,y index=11,
                                  col sep=comma, ]{\anitable};
                          \addlegendentry{$\mu = 1.5$}
                          \addplot[GraphSmooth, color = blue3] table[x index=12,y index=13,
                                  col sep=comma, ]{\anitable};
                          \addlegendentry{$\mu = 2.0$}
                      \end{axis}
                  \end{tikzpicture}
              \end{subfigure}
          \end{figure}
          All the simulations have initial condition $ (1, 0) $. The spiraling of the
          individual trajectories into the limit cycle is not plotted here, for the sake
          of clarity. \par
          The limit cycle starts very closely resembling the circle for
          $ \mu \approxeq 0 $, then deforms into the shape in the graph smoothly as
          $ \mu $ increases.

    \item Finding critical points,
          \begin{align}
              y_1'         & = 4y_1 - y_1^{2}     & y_2' & = y_2 \\
              y_1(4 - y_1) & = 0                  & y_2  & = 0   \\
              P_1          & = \color{y_h}(0, 0)  &
              P_2          & = \color{y_p} (4, 0)
          \end{align}
          Linearizing the system for $ P_1 $
          \begin{align}
              \vec{y'}       & = \bmattt{4}{0}{0}{1} \vec{y}    &
              0              & = \lambda^{2} - 5\lambda + 4       \\
              p < 0 \qquad q & > 0 \qquad \Delta >0             &
                             & \color{y_h} \text{Improper node}
          \end{align}
          Linearizing the system for $ P_2 $
          \begin{align}
              z_1            & = y_1 - 4                       &
              z_2            & = y_2                             \\
              z_1'           & = 4(z_1 + 4) - (z_1 + 4)^{2}    &
              z_2'           & = z_2                             \\
              \vec{y'}       & = \bmattt{-4}{0}{0}{1} \vec{y}  &
              0              & = \lambda^{2} + 3\lambda - 4      \\
              p < 0 \qquad q & < 0 \qquad \Delta >0            &
                             & \color{y_p} \text{Saddle point}
          \end{align}

    \item Finding critical points,
          \begin{align}
              y_1'            & = y_2                & y_2' & = -y_1 + 0.5y_1^{2} \\
              y_1(0.5y_1 - 1) & = 0                  & y_2  & = 0                 \\
              P_1             & = \color{y_h}(0, 0)  &
              P_2             & = \color{y_p} (2, 0)
          \end{align}
          Linearizing the system for $ P_1 $
          \begin{align}
              \vec{y'}       & = \bmattt{0}{1}{-1}{0} \vec{y} &
              0              & = \lambda^{2} + 1                \\
              p = 0 \qquad q & > 0 \qquad \Delta < 0          &
                             & \color{y_h} \text{Center}
          \end{align}
          Linearizing the system for $ P_2 $
          \begin{align}
              z_1            & = y_1 - 2                       &
              z_2            & = y_2                             \\
              z_1'           & = z_2                           &
              z_2'           & = 0.5(z_1 + 2)z_1                 \\
              \vec{y'}       & = \bmattt{0}{1}{1}{0} \vec{y}   &
              0              & = \lambda^{2} - 1                 \\
              p = 0 \qquad q & < 0 \qquad \Delta > 0           &
                             & \color{y_p} \text{Saddle point}
          \end{align}

    \item Finding critical points,
          \begin{align}
              y_1'         & = y_2                 & y_2' & = -y_1(1 + y_1) \\
              y_1(y_1 + 1) & = 0                   & y_2  & = 0             \\
              P_1          & = \color{y_h}(0, 0)   &
              P_2          & = \color{y_p} (-1, 0)
          \end{align}
          Linearizing the system for $ P_1 $
          \begin{align}
              \vec{y'}       & = \bmattt{0}{1}{-1}{0} \vec{y} &
              0              & = \lambda^{2} + 1                \\
              p = 0 \qquad q & > 0 \qquad \Delta < 0          &
                             & \color{y_h} \text{Center}
          \end{align}
          Linearizing the system for $ P_2 $
          \begin{align}
              z_1            & = y_1 + 1                       &
              z_2            & = y_2                             \\
              z_1'           & = z_2                           &
              z_2'           & = -(z_1 - 1)z_1                   \\
              \vec{y'}       & = \bmattt{0}{1}{1}{0} \vec{y}   &
              0              & = \lambda^{2} - 1                 \\
              p = 0 \qquad q & < 0 \qquad \Delta > 0           &
                             & \color{y_p} \text{Saddle point}
          \end{align}

    \item Finding critical points,
          \begin{align}
              y_1'                & = -y_1 + y_2(1 - y_2) &
              y_2'                & = -y_1 - y_2            \\
              -y_1 + y_2(1 - y_2) & = 0                   &
              y_1 + y_2           & = 0                     \\
              P_1                 & = \color{y_h}(0, 0)   &
              P_2                 & = \color{y_p} (-2, 2)
          \end{align}
          Linearizing the system for $ P_1 $
          \begin{align}
              \vec{y'}       & = \bmattt{-1}{1}{-1}{-1} \vec{y}     &
              0              & = \lambda^{2} + 2\lambda + 2           \\
              p < 0 \qquad q & > 0 \qquad \Delta < 0                &
                             & \color{y_h} \text{Attractive Spiral}
          \end{align}
          Linearizing the system for $ P_2 $
          \begin{align}
              z_1            & = y_1 + 2                         &
              z_2            & = y_2 - 2                           \\
              z_1'           & = 2 - z_1 - (z_2 + 2)(z_2 + 1)    &
              z_2'           & = 2 - z_1 - z_2 - 2                 \\
              \vec{y'}       & = \bmattt{-1}{-3}{-1}{-1} \vec{y} &
              0              & = \lambda^{2} + 2\lambda - 2        \\
              p < 0 \qquad q & < 0 \qquad \Delta > 0             &
                             & \color{y_p} \text{Saddle point}
          \end{align}

    \item Finding critical points,
          \begin{align}
              y_1'         & = y_2(1 - y_2)          &
              y_2'         & = y_1(1 - y_1)            \\
              y_2(1 - y_2) & = 0                     &
              y_1(1 - y_1) & = 0                       \\
              P_1          & = \color{y_h}(0, 0)     &
              P_2          & = \color{y_p} (0, 1)      \\
              P_3          & = \color{brown6} (1, 0) &
              P_4          & = \color{blue2} (1, 1)
          \end{align}
          Linearizing the system for $ P_1 $
          \begin{align}
              \vec{y'}       & = \bmattt{0}{1}{1}{0} \vec{y}   &
              0              & = \lambda^{2} - 1                 \\
              p = 0 \qquad q & < 0 \qquad \Delta > 0           &
                             & \color{y_h} \text{Saddle Point}
          \end{align}
          Linearizing the system for $ P_2 $
          \begin{align}
              z_1            & = y_1                          &
              z_2            & = y_2 - 1                        \\
              z_1'           & =  -z_2(z_2 + 1)               &
              z_2'           & = z_1(1 - z_1)                   \\
              \vec{y'}       & = \bmattt{0}{-1}{1}{0} \vec{y} &
              0              & = \lambda^{2} + 1                \\
              p = 0 \qquad q & > 0 \qquad \Delta < 0          &
                             & \color{y_p} \text{Center}
          \end{align}
          Linearizing the system for $ P_3 $
          \begin{align}
              z_1            & = y_1 - 1                      &
              z_2            & = y_2                            \\
              z_1'           & =  z_2(1 - z_2)                &
              z_2'           & = -z_1(1 + z_1)                  \\
              \vec{y'}       & = \bmattt{0}{1}{-1}{0} \vec{y} &
              0              & = \lambda^{2} + 1                \\
              p = 0 \qquad q & > 0 \qquad \Delta < 0          &
                             & \color{brown6} \text{Center}
          \end{align}
          Linearizing the system for $ P_4 $
          \begin{align}
              z_1            & = y_1 - 1                         &
              z_2            & = y_2 - 1                           \\
              z_1'           & = -z_2(1 + z_2)                   &
              z_2'           & = -z_1(1 + z_1)                     \\
              \vec{y'}       & = \bmattt{0}{-1}{-1}{0} \vec{y}   &
              0              & = \lambda^{2} - 1                   \\
              p = 0 \qquad q & > 0 \qquad \Delta > 0             &
                             & \color{blue2} \text{Saddle point}
          \end{align}

    \item Finding critical points,
          \begin{align}
              y'' - 9y + y^{3}       & = 0                       \\
              y_1'                   & = y_2                   &
              y_2'                   & = y_1(9 - y_1^{2})        \\
              y_2                    & = 0                     &
              y_1 (y_1 - 3)(y_1 + 3) & = 0                       \\
              P_1                    & = \color{y_h}(0, 0)     &
              P_2                    & = \color{y_p} (-3, 0)     \\
              P_3                    & = \color{brown6} (3, 0)
          \end{align}
          Linearizing the system for $ P_1 $
          \begin{align}
              \vec{y'}       & = \bmattt{0}{1}{9}{0} \vec{y}   &
              0              & = \lambda^{2} - 2                 \\
              p = 0 \qquad q & < 0 \qquad \Delta > 0           &
                             & \color{y_h} \text{Saddle point}
          \end{align}
          Linearizing the system for $ P_2 $
          \begin{align}
              z_1            & = y_1 + 3                       &
              z_2            & = y_2                             \\
              z_1'           & = z_2                           &
              z_2'           & = (z_1 - 3)(z_1)(6 - z_1)         \\
              \vec{y'}       & = \bmattt{0}{1}{-18}{0} \vec{y} &
              0              & = \lambda^{2} + 18                \\
              p = 0 \qquad q & > 0 \qquad \Delta < 0           &
                             & \color{y_p} \text{Center}
          \end{align}
          Linearizing the system for $ P_3 $
          \begin{align}
              z_1            & = y_1 - 3                       &
              z_2            & = y_2                             \\
              z_1'           & = z_2                           &
              z_2'           & = -(z_1 + 3)(z_1)(6 + z_1)        \\
              \vec{y'}       & = \bmattt{0}{1}{-18}{0} \vec{y} &
              0              & = \lambda^{2} + 18                \\
              p = 0 \qquad q & > 0 \qquad \Delta < 0           &
                             & \color{brown6} \text{Center}
          \end{align}

    \item Finding critical points,
          \begin{align}
              y'' + y - y^{3}        & = 0                       \\
              y_1'                   & = y_2                   &
              y_2'                   & = y_1(y_1^{2} - 1)        \\
              y_2                    & = 0                     &
              y_1 (y_1 + 1)(y_1 - 1) & = 0                       \\
              P_1                    & = \color{y_h}(0, 0)     &
              P_2                    & = \color{y_p} (-1, 0)     \\
              P_3                    & = \color{brown6} (1, 0)
          \end{align}
          Linearizing the system for $ P_1 $
          \begin{align}
              \vec{y'}       & = \bmattt{0}{1}{-1}{0} \vec{y} &
              0              & = \lambda^{2} + 1                \\
              p = 0 \qquad q & > 0 \qquad \Delta < 0          &
                             & \color{y_h} \text{Center}
          \end{align}
          Linearizing the system for $ P_2 $
          \begin{align}
              z_1            & = y_1 + 1                       &
              z_2            & = y_2                             \\
              z_1'           & = z_2                           &
              z_2'           & = (z_1 - 1)(z_1 - 2)(z_1)         \\
              \vec{y'}       & = \bmattt{0}{1}{2}{0} \vec{y}   &
              0              & = \lambda^{2} - 2                 \\
              p = 0 \qquad q & < 0 \qquad \Delta > 0           &
                             & \color{y_p} \text{Saddle point}
          \end{align}
          Linearizing the system for $ P_3 $
          \begin{align}
              z_1            & = y_1 - 1                          &
              z_2            & = y_2                                \\
              z_1'           & = z_2                              &
              z_2'           & = (z_1 + 1)(z_1)(2 + z_1)            \\
              \vec{y'}       & = \bmattt{0}{1}{2}{0} \vec{y}      &
              0              & = \lambda^{2} - 2                    \\
              p = 0 \qquad q & < 0 \qquad \Delta > 0              &
                             & \color{brown6} \text{Saddle point}
          \end{align}

    \item Finding critical points,
          \begin{align}
              y'' + \cos y & = 0                                \\
              y_1'         & = y_2                            &
              y_2'         & = -\cos(y_1)                       \\
              y_2          & = 0                              &
              \cos(y_1)    & = 0                                \\
              P_1          & = \color{y_h}(2n\pi - \pi/2, 0)  &
              P_2          & = \color{y_p} (2n\pi + \pi/2, 0)   \\
          \end{align}
          Linearizing the system for $ P_1 $
          \begin{align}
              z_1            & = y_1 - 2n\pi + \pi/2          &
              z_2            & = y_2                            \\
              z_1'           & = z_2                          &
              z_2'           & = -\cos(z_1 + 2n\pi - \pi/2)     \\
                             &                                &
                             & = -\sin(z_1 + 2n\pi)             \\
                             &                                &
                             & = -z_1 + \frac{1}{6}z_1^{3}      \\
              \vec{y'}       & = \bmattt{0}{1}{-1}{0} \vec{y} &
              0              & = \lambda^{2} + 1                \\
              p = 0 \qquad q & > 0 \qquad \Delta < 0          &
                             & \color{y_h} \text{Center}
          \end{align}
          Linearizing the system for $ P_2 $
          \begin{align}
              z_1            & = y_1 - 2n\pi - \pi/2           &
              z_2            & = y_2                             \\
              z_1'           & = z_2                           &
              z_2'           & = -\cos(z_1 + 2n\pi + \pi/2)      \\
                             &                                 &
                             & = \sin(z_1 + 2n\pi)               \\
                             &                                 &
                             & = z_1 - \frac{1}{6}z_1^{3}        \\
              \vec{y'}       & = \bmattt{0}{1}{1}{0} \vec{y}   &
              0              & = \lambda^{2} - 1                 \\
              p = 0 \qquad q & < 0 \qquad \Delta > 0           &
                             & \color{y_p} \text{Saddle point}
          \end{align}

    \item Finding critical points,
          \begin{align}
              y'' + 9y + y^{2} & = 0                     \\
              y_1'             & = y_2                 &
              y_2'             & = -y_1(y_1 + 9)         \\
              y_2              & = 0                   &
              y_1(y_1 + 9)     & = 0                     \\
              P_1              & = \color{y_h}(0, 0)   &
              P_2              & = \color{y_p} (-9, 0)   \\
          \end{align}
          Linearizing the system for $ P_1 $
          \begin{align}
              \vec{y'}       & = \bmattt{0}{1}{-9}{0} \vec{y} &
              0              & = \lambda^{2} + 9                \\
              p = 0 \qquad q & > 0 \qquad \Delta < 0          &
                             & \color{y_h} \text{Center}
          \end{align}
          Linearizing the system for $ P_2 $
          \begin{align}
              z_1            & = y_1 + 9                       &
              z_2            & = y_2                             \\
              z_1'           & = z_2                           &
              z_2'           & = (9 - z_1)(z_1)                  \\
              \vec{y'}       & = \bmattt{0}{1}{9}{0} \vec{y}   &
              0              & = \lambda^{2} - 9                 \\
              p = 0 \qquad q & < 0 \qquad \Delta > 0           &
                             & \color{y_p} \text{Saddle point}
          \end{align}

    \item Finding critical points,
          \begin{align}
              y'' + \sin y & = 0                              \\
              y_1'         & = y_2                          &
              y_2'         & = -\sin(y_1)                     \\
              y_2          & = 0                            &
              \cos(y_1)    & = 0                              \\
              P_1          & = \color{y_h}(2n\pi, 0)        &
              P_2          & = \color{y_p} (2n\pi + \pi, 0)   \\
          \end{align}
          Linearizing the system for $ P_1 $
          \begin{align}
              z_1            & = y_1 - 2n\pi                  &
              z_2            & = y_2                            \\
              z_1'           & = z_2                          &
              z_2'           & = -\sin(z_1 + 2n\pi)             \\
                             &                                &
                             & = -z_1 + \frac{1}{6}z_1^{3}      \\
              \vec{y'}       & = \bmattt{0}{1}{-1}{0} \vec{y} &
              0              & = \lambda^{2} + 1                \\
              p = 0 \qquad q & > 0 \qquad \Delta < 0          &
                             & \color{y_h} \text{Center}
          \end{align}
          Linearizing the system for $ P_2 $
          \begin{align}
              z_1            & = y_1 - 2n\pi - \pi             &
              z_2            & = y_2                             \\
              z_1'           & = z_2                           &
              z_2'           & = -\sin(z_1 + 2n\pi + \pi)        \\
                             &                                 &
                             & = z_1 - \frac{1}{6}z_1^{3}        \\
              \vec{y'}       & = \bmattt{0}{1}{1}{0} \vec{y}   &
              0              & = \lambda^{2} - 1                 \\
              p = 0 \qquad q & < 0 \qquad \Delta > 0           &
                             & \color{y_p} \text{Saddle point}
          \end{align}

    \item
          \begin{enumerate}
              \item Van der Pol equation,
                    \begin{align}
                        y_1'             & = y_2                             &
                        y_2'             & = \mu y_2 - y_1  -\mu y_1^{2} y_2   \\
                        \vec{A}          & = \bmattt{0}{1}{-1}{\mu}          &
                        0                & = \lambda^{2} - \mu\lambda + 1      \\
                        p = \mu \qquad q & = 1                               &
                        \Delta           & = \mu^{2} - 4
                    \end{align}
                    For $ \mu = 0, \mu > 0, \mu< 0 $, the critical point at $ (0, 0) $
                    is a center, spiral source and spiral sink respectively.
              \item Rayleigh equation,
                    \begin{align}
                        z'' - \mu \left( 1 - \frac{z'^{2}}{3} \right) z' + z & = 0 \\
                        z''' - \mu\left( 1 - \frac{z'^2}{3} \right)z''
                        + \mu z' \left( \frac{2z'}{3} \right)z'' + z'        & = 0 \\
                        z''' - ( 1 - z'^2)\mu z''
                        + z'                                                 & = 0 \\
                    \end{align}
                    Using the substitution $ z' \rightarrow y $, the above equation
                    reduces to the Van der Pol ODE.
              \item Duffing equation,
                    \begin{align}
                        y'' + \omega_0^{2}y + \beta y^{3}  & = 0                                \\
                        y_1'                               & = y_2                            &
                        y_2'                               & = -y_1(\omega_0^{2}
                        + \beta y_1^{2})                                                        \\
                        y_2                                & = 0                              &
                        -y_1(\omega_0^{2} + \beta y_1^{2}) & = 0                                \\
                        P_1                                & = \color{y_h}(0, 0)              &
                        P_2                                & = \color{y_p}
                        (-\alpha, 0)                                                            \\
                        P_2                                & = \color{brown6}
                        (\alpha, 0)                        &
                        \alpha                             & = \frac{\omega_0}{\sqrt{-\beta}}
                    \end{align}
                    $ P_2 $ and $ P_3 $ require $ \beta < 0 $ in order to exist. \par
                    Linearizing the system for $ P_1 $
                    \begin{align}
                        \vec{y'}       & = \bmattt{0}{1}{-\omega_0^{2}}{0} \vec{y} &
                        0              & = \lambda^{2} + \omega_0^{2}                \\
                        p = 0 \qquad q & > 0 \qquad \Delta < 0                     &
                                       & \color{y_h} \text{Center}
                    \end{align}
                    Linearizing the system for $ P_2 $
                    \begin{align}
                        z_1            & = y_1 + \alpha                             &
                        z_2            & = y_2                                        \\
                        z_1'           & = z_2                                      &
                        z_2'           & = (\alpha - z_1)(\beta z_1)(z_1 - 2\alpha)   \\
                        \vec{y'}       & = \bmattt{0}{1}{\omega_0^{2}}{0} \vec{y}   &
                        0              & = \lambda^{2} - \omega_0^{2}                 \\
                        p = 0 \qquad q & < 0 \qquad \Delta > 0                      &
                                       & \color{y_p} \text{Saddle point}
                    \end{align}
                    Linearizing the system for $ P_2 $
                    \begin{align}
                        z_1            & = y_1 - \alpha                              &
                        z_2            & = y_2                                         \\
                        z_1'           & = z_2                                       &
                        z_2'           & = -(\alpha + z_1)(\beta z_1)(z_1 + 2\alpha)   \\
                        \vec{y'}       & = \bmattt{0}{1}{\omega_0^{2}}{0} \vec{y}    &
                        0              & = \lambda^{2} - \omega_0^{2}                  \\
                        p = 0 \qquad q & < 0 \qquad \Delta > 0                       &
                                       & \color{brown6} \text{Saddle point}
                    \end{align}
          \end{enumerate}

    \item Reframing ODE as a phase plane parametric function,
          \begin{align}
              y'' - 4y + y^{3}     & = 0                             \\
              \diff{y_2}{y_1}\ y_2 & = 4y_1 - y_1^{3}                \\
              \diff{y_2}{y_1}      & = \frac{y_1 (4 - y_1^{2})}{y_2} \\
              0.5 y_2^{2}          & = 2y_1^{2} - 0.25y_1^{4} + C
          \end{align}
          Drawing contour plots for this equation in $ y_1,\ y_2 $.
          \begin{figure}[H]
              \centering
              \begin{tikzpicture}
                  \begin{axis}[
                          set layers,
                          enlargelimits = true,
                          colorbar,
                          colormap/jet,
                          width = 12cm,
                          height = 12cm,
                          Ani, grid = both,
                          axis equal,
                          view     = {0}{90}, % for a view 'from above'
                      ]
                      \addplot3 [
                          domain = -10:10,
                          thick,
                          contour gnuplot={
                                  levels={-3,-1,...,11},
                                  labels=false,
                              },
                          samples=200
                      ] {0.5*y^(2) - 2*x^(2) + 0.25*x^(4)};
                  \end{axis}
              \end{tikzpicture}
          \end{figure}
\end{enumerate}