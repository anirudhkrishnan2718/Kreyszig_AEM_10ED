\section{Criteria for Critical Points, Stability}
\begin{enumerate}
    \item Finding the type and stability of critical point,
          \begin{align}
              y_1'                       & = y_1                 &
              y_2'                       & = 2y_2                  \\
              \vec{A}                    & = \bmattt{1}{0}{0}{2} &
              \lambda^{2} - 3\lambda + 2 & = 0
          \end{align}
          Using $ p = 3,\ q = 2,\ \Delta = 1 $ gives an unstable  improper node at
          $ (0, 0) $.
          \begin{align}
              \lambda_1     & = 1                       &
              \vec{v^{(1)}} & = \bmatcol{1}{0}            \\
              \lambda_2     & = 2                       &
              \vec{v^{(2)}} & = \bmatcol{0}{1}            \\
              \vec{y}       & = c_1 \bmatcol{1}{0}e^{t}
              + c_2 \bmatcol{0}{1}e^{2t}
          \end{align}
          \begin{figure}[H]
              \centering
              \begin{tikzpicture}
                  \begin{axis}[
                          declare function = {
                                  u(\x) = e^(\x);
                                  v(\x) = e^(2*\x);
                              },
                          xmin = -1, xmax = 1, ymin = -1, ymax = 1,
                          % restrict y to domain = -1:1,
                          title = {Improper Node with
                                  ${\color{y_p} \vec{x^{(1)}}}$ and
                                  ${\color{y_h} \vec{x^{(2)}}}$},
                          xlabel = $ y_1 $,
                          ylabel = $ y_2 $,
                          axis equal,
                          width = 8cm,
                          legend pos = north west,
                          grid = both,
                          domain = -4:0,
                          Ani]
                      \foreach \c in {1/2, 1} {%
                              \edef\temp{%
                                  \noexpand \addplot[ samples = 100, color=blue3,
                                      arrow inside={end=stealth,opt={scale=2}}{0.5}]
                                  ({\c*u(x)}, {-v(x)});
                                  \noexpand \addplot[ samples = 100, color=blue3,
                                      arrow inside={end=stealth,opt={scale=2}}{0.5}]
                                  ({\c*u(x)}, {v(x)});
                                  \noexpand \addplot[ samples = 100, color=blue3,
                                      arrow inside={end=stealth,opt={scale=2}}{0.5}]
                                  ({-\c*u(x)}, {-v(x)});
                                  \noexpand \addplot[ samples = 100, color=blue3,
                                      arrow inside={end=stealth,opt={scale=2}}{0.5}]
                                  ({-\c*u(x)}, {v(x)});
                              }\temp
                          }
                      \addplot[ samples = 100, color=y_p,
                          arrow inside={end=stealth,opt={scale=2}}{0.65}]
                      ({u(x)}, {0});
                      \addplot[ samples = 100, color=y_h,
                          arrow inside={end=stealth,opt={scale=2}}{0.65}]
                      ({0}, {-v(x)});
                      \addplot[ samples = 100, color=y_p,
                          arrow inside={end=stealth,opt={scale=2}}{0.65}]
                      ({-u(x)}, {0});
                      \addplot[ samples = 100, color=y_h,
                          arrow inside={end=stealth,opt={scale=2}}{0.65}]
                      ({0}, {v(x)});
                      \node[GraphNode, fill = white, draw = black] at (axis cs:0,0) {};
                  \end{axis}
              \end{tikzpicture}
          \end{figure}

    \item Finding the type and stability of critical point,
          \begin{align}
              y_1'                        & = -4y_1                 &
              y_2'                        & = -3y_2                   \\
              \vec{A}                     & = \bmattt{-4}{0}{0}{-3} &
              \lambda^{2} + 7\lambda + 12 & = 0
          \end{align}
          Using $ p = -7,\ q = 12,\ \Delta = 1 $ gives an improper node at
          $ (0, 0) $.
          \begin{align}
              \lambda_1     & = -4                        &
              \vec{v^{(1)}} & = \bmatcol{1}{0}              \\
              \lambda_2     & = -3                        &
              \vec{v^{(2)}} & = \bmatcol{0}{1}              \\
              \vec{y}       & = c_1 \bmatcol{1}{0}e^{-4t}
              +  c_2 \bmatcol{0}{1} e^{-3t}
          \end{align}
          \begin{figure}[H]
              \centering
              \begin{tikzpicture}
                  \begin{axis}[
                          declare function = {
                                  u(\x) = e^(-4*\x);
                                  v(\x) = e^(-3*\x);
                              },
                          xmin = -1, xmax = 1, ymin = -1, ymax = 1,
                          % restrict y to domain = -1:1,
                          title = {Improper Node with
                                  ${\color{y_p} \vec{x^{(1)}}}$ and
                                  ${\color{y_h} \vec{x^{(2)}}}$},
                          xlabel = $ y_1 $,
                          ylabel = $ y_2 $,
                          axis equal,
                          width = 8cm,
                          legend pos = north west,
                          grid = both,
                          domain = 0:1.5,
                          Ani]
                      \foreach \c in {1/2, 3} {%
                              \edef\temp{%
                                  \noexpand \addplot[ samples = 100, color=blue3,
                                      arrow inside={end=stealth,opt={scale=2}}{0.75}]
                                  ({\c*u(x)}, {-v(x)});
                                  \noexpand \addplot[ samples = 100, color=blue3,
                                      arrow inside={end=stealth,opt={scale=2}}{0.75}]
                                  ({\c*u(x)}, {v(x)});
                                  \noexpand \addplot[ samples = 100, color=blue3,
                                      arrow inside={end=stealth,opt={scale=2}}{0.75}]
                                  ({-\c*u(x)}, {-v(x)});
                                  \noexpand \addplot[ samples = 100, color=blue3,
                                      arrow inside={end=stealth,opt={scale=2}}{0.75}]
                                  ({-\c*u(x)}, {v(x)});
                              }\temp
                          }
                      \addplot[ samples = 100, color=y_p,
                          arrow inside={end=stealth,opt={scale=2}}{0.5}]
                      ({u(x)}, {0});
                      \addplot[ samples = 100, color=y_h,
                          arrow inside={end=stealth,opt={scale=2}}{0.5}]
                      ({0}, {-v(x)});
                      \addplot[ samples = 100, color=y_p,
                          arrow inside={end=stealth,opt={scale=2}}{0.5}]
                      ({-u(x)}, {0});
                      \addplot[ samples = 100, color=y_h,
                          arrow inside={end=stealth,opt={scale=2}}{0.5}]
                      ({0}, {v(x)});
                      \node[GraphNode, fill = white, draw = black] at (axis cs:0,0) {};
                  \end{axis}
              \end{tikzpicture}
          \end{figure}

    \item Finding the type and stability of critical point,
          \begin{align}
              y_1'            & = y_2                  &
              y_2'            & = -9y_1                  \\
              \vec{A}         & = \bmattt{0}{1}{-9}{0} &
              \lambda^{2} + 9 & = 0
          \end{align}
          Using $ p = 0,\ q = 9,\ \Delta = -36 $ gives a center at
          $ (0, 0) $.
          \begin{align}
              \lambda_1     & = -3\ \i                            &
              \vec{v^{(1)}} & = \bmatcol{\i}{3}                     \\
              \lambda_2     & = 3\ \i                             &
              \vec{v^{(2)}} & = \bmatcol{-\i}{3}                    \\
              \vec{y}       & = c_1 \bmatcol{\sin(3t)}{3\cos(3t)}
              + c_2 \bmatcol{\sin(3t)}{3\cos(3t)}
          \end{align}
          \begin{figure}[H]
              \centering
              \begin{tikzpicture}
                  \begin{axis}[
                          % xmin = -1, xmax = 1, ymin = -1, ymax = 1,
                          % restrict y to domain = -1:1,
                          declare function = {
                                  u(\x) = sin(3*\x);
                                  v(\x) = 3*cos(3*\x);
                              },
                          title = {Center},
                          xlabel = $ y_1 $,
                          ylabel = $ y_2 $,
                          axis equal,
                          width = 8cm,
                          legend pos = north west,
                          grid = both,
                          domain = 0:2*pi/3,
                          Ani]
                      \foreach \c in {0.5, 2, 4} {%
                              \edef\temp{%
                                  \noexpand \addplot[ samples = 200, color=blue3,
                                      arrow inside={end=stealth,opt={scale=2}}{0.25}]
                                  ({(\c + 1)*u(x)}, {(\c + 1)*v(x)});
                              }\temp
                          }
                      \node[GraphNode, fill = white, draw = black] at (axis cs:0,0) {};
                  \end{axis}
              \end{tikzpicture}
          \end{figure}

    \item Finding the type and stability of critical point,
          \begin{align}
              y_1'            & = 2y_1 + y_2           &
              y_2'            & = 5y_1 - 2y_2            \\
              \vec{A}         & = \bmattt{2}{1}{5}{-2} &
              \lambda^{2} - 9 & = 0
          \end{align}
          Using $ p = 0,\ q = -9,\ \Delta = 36 $ gives a saddle point at
          $ (0, 0) $.
          \begin{align}
              \lambda_1     & = -3                         &
              \vec{v^{(1)}} & = \bmatcol{-1}{5}              \\
              \lambda_2     & = 3                          &
              \vec{v^{(2)}} & = \bmatcol{1}{1}               \\
              \vec{y}       & = c_1 e^{-3t}\bmatcol{-1}{5}
              + c_2 e^{3t} \bmatcol{1}{1}
          \end{align}
          \begin{figure}[H]
              \centering
              \begin{tikzpicture}
                  \begin{axis}[
                          declare function = {
                                  u(\x) = e^(-3*\x);
                                  v(\x) = e^(3*\x);
                              },
                          xmin = -10, xmax = 10, ymin = -10, ymax = 10,
                          % restrict y to domain = -1:1,
                          title = {Saddle point with
                                  ${\color{y_p} \vec{x^{(1)}}}$ and
                                  ${\color{y_h} \vec{x^{(2)}}}$},
                          xlabel = $ y_1 $,
                          ylabel = $ y_2 $,
                          axis equal,
                          width = 8cm,
                          legend pos = north west,
                          grid = both,
                          domain = -0.5:1,
                          Ani]
                      \foreach \c in {1} {%
                              \edef\temp{%
                                  \noexpand \addplot[ samples = 100, color=blue3,
                                      arrow inside={end=stealth,opt={scale=2}}{0.5}]
                                  ({-\c*u(x) + v(x)}, {5*\c*u(x) + v(x)});
                                  \noexpand \addplot[ samples = 100, color=blue3,
                                      arrow inside={end=stealth,opt={scale=2}}{0.5}]
                                  ({\c*u(x) + v(x)}, {-5*\c*u(x) + v(x)});
                                  \noexpand \addplot[ samples = 100, color=blue3,
                                      arrow inside={end=stealth,opt={scale=2}}{0.5}]
                                  ({-\c*u(x) - v(x)}, {5*\c*u(x) - v(x)});
                                  \noexpand \addplot[ samples = 100, color=blue3,
                                      arrow inside={end=stealth,opt={scale=2}}{0.5}]
                                  ({\c*u(x) - v(x)}, {-5*\c*u(x) - v(x)});
                              }\temp
                          }
                      \addplot[ samples = 100, color=y_p,
                          arrow inside={end=stealth,opt={scale=2}}{0.75}]
                      ({-u(x)}, {5*u(x)});
                      \addplot[ samples = 100, color=y_h,
                          arrow inside={end=stealth,opt={scale=2}}{0.25}]
                      ({v(x)}, {v(x)});
                      \addplot[ samples = 100, color=y_p,
                          arrow inside={end=stealth,opt={scale=2}}{0.75}]
                      ({u(x)}, {-5*u(x)});
                      \addplot[ samples = 100, color=y_h,
                          arrow inside={end=stealth,opt={scale=2}}{0.25}]
                      ({-v(x)}, {-v(x)});
                      \node[GraphNode, fill = white, draw = black] at (axis cs:0,0) {};
                  \end{axis}
              \end{tikzpicture}
          \end{figure}

    \item Finding the type and stability of critical point,
          \begin{align}
              y_1'                       & = -2y_1 + 2y_2           &
              y_2'                       & = -2y_1 - 2y_2             \\
              \vec{A}                    & = \bmattt{-2}{2}{-2}{-2} &
              \lambda^{2} + 4\lambda + 8 & = 0
          \end{align}
          Using $ p = 0,\ q = 9,\ \Delta = -36 $ gives a center at
          $ (0, 0) $.
          \begin{align}
              \lambda_1     & = -2 - 2i                                 &
              \vec{v^{(1)}} & = \bmatcol{i}{1}                            \\
              \lambda_2     & = -2 + 2i                                 &
              \vec{v^{(2)}} & = \bmatcol{-i}{1}                           \\
              \vec{y}       & = c_1 e^{-2t}\bmatcol{\sin(2t)}{\cos(2t)}
              + c_2 e^{-2t} \bmatcol{\sin(2t)}{\cos(2t)}
          \end{align}
          \begin{figure}[H]
              \centering
              \begin{tikzpicture}
                  \begin{axis}[
                          xmin = -1, xmax = 1, ymin = -1, ymax = 1,
                          % restrict y to domain = -1:1,
                          declare function = {
                                  u(\x) = e^(-2*\x)*sin(2*\x);
                                  v(\x) = e^(-2*\x)*3*cos(2*\x);
                              },
                          title = {Spiral},
                          xlabel = $ y_1 $,
                          ylabel = $ y_2 $,
                          axis equal,
                          width = 8cm,
                          legend pos = north west,
                          grid = both,
                          domain = 0:pi,
                          Ani]
                      \foreach \c in {2} {%
                              \edef\temp{%
                                  \noexpand \addplot[ samples = 200, color=blue3,
                                      arrow inside={end=stealth,opt={scale=2}}{0.8}]
                                  ({(\c + 1)*u(x)}, {(\c + 1)*v(x)});
                                  \noexpand \addplot[ samples = 200, color=blue3,
                                      arrow inside={end=stealth,opt={scale=2}}{0.8}]
                                  ({(-\c + 1)*u(x)}, {(-\c + 1)*v(x)});
                                  \noexpand \addplot[ samples = 200, color=blue3,
                                      arrow inside={end=stealth,opt={scale=2}}{0.8}]
                                  ({(\c - 1)*u(x)}, {(\c - 1)*v(x)});
                                  \noexpand \addplot[ samples = 200, color=blue3,
                                      arrow inside={end=stealth,opt={scale=2}}{0.8}]
                                  ({(-\c - 1)*u(x)}, {(-\c - 1)*v(x)});
                              }\temp
                          }
                      \node[GraphNode, fill = white, draw = black] at (axis cs:0,0) {};
                  \end{axis}
              \end{tikzpicture}
          \end{figure}

    \item Finding the type and stability of critical point,
          \begin{align}
              y_1'                         & = -6y_1 - y_2             &
              y_2'                         & = -9y_1 - 6y_2              \\
              \vec{A}                      & = \bmattt{-6}{-1}{-9}{-6} &
              \lambda^{2} + 12\lambda + 27 & = 0
          \end{align}
          Using $ p = -12,\ q = 27,\ \Delta = 36 $ gives a center at
          $ (0, 0) $.
          \begin{align}
              \lambda_1     & = -9                        &
              \vec{v^{(1)}} & = \bmatcol{1}{3}              \\
              \lambda_2     & = -3                        &
              \vec{v^{(2)}} & = \bmatcol{-1}{3}             \\
              \vec{y}       & = c_1 e^{-9t}\bmatcol{1}{3}
              + c_2 e^{-3t} \bmatcol{-1}{3}
          \end{align}
          \begin{figure}[H]
              \centering
              \begin{tikzpicture}
                  \begin{axis}[
                          declare function = {
                                  u(\x) = e^(-9*\x);
                                  v(\x) = e^(-3*\x);
                              },
                          xmin = -2, xmax = 2, ymin = -2, ymax = 2,
                          % restrict y to domain = -1:1,
                          title = {Proper Node with
                                  ${\color{y_p} \vec{x^{(1)}}}$ and
                                  ${\color{y_h} \vec{x^{(2)}}}$},
                          xlabel = $ y_1 $,
                          ylabel = $ y_2 $,
                          axis equal,
                          width = 8cm,
                          legend pos = north west,
                          grid = both,
                          domain = 0:1.2,
                          Ani]
                      \foreach \c in {4} {%
                              \edef\temp{%
                                  \noexpand \addplot[ samples = 100, color=blue3,
                                      arrow inside={end=stealth,opt={scale=2}}{0.9}]
                                  ({\c*u(x) - v(x)}, {3*\c*u(x) + 3*v(x)});
                                  \noexpand \addplot[ samples = 100, color=blue3,
                                      arrow inside={end=stealth,opt={scale=2}}{0.9}]
                                  ({-\c*u(x) - v(x)}, {-3*\c*u(x) + 3*v(x)});
                                  \noexpand \addplot[ samples = 100, color=blue3,
                                      arrow inside={end=stealth,opt={scale=2}}{0.9}]
                                  ({\c*u(x) + v(x)}, {3*\c*u(x) - 3*v(x)});
                                  \noexpand \addplot[ samples = 100, color=blue3,
                                      arrow inside={end=stealth,opt={scale=2}}{0.9}]
                                  ({-\c*u(x) + v(x)}, {-3*\c*u(x) - 3*v(x)});
                              }\temp
                          }
                      \addplot[ samples = 100, color=y_p,
                          arrow inside={end=stealth,opt={scale=2}}{0.75}]
                      ({u(x)}, {3*u(x)});
                      \addplot[ samples = 100, color=y_h,
                          arrow inside={end=stealth,opt={scale=2}}{0.75}]
                      ({-v(x)}, {3*v(x)});
                      \addplot[ samples = 100, color=y_p,
                          arrow inside={end=stealth,opt={scale=2}}{0.75}]
                      ({-u(x)}, {-3*u(x)});
                      \addplot[ samples = 100, color=y_h,
                          arrow inside={end=stealth,opt={scale=2}}{0.75}]
                      ({v(x)}, {-3*v(x)});
                      \node[GraphNode, fill = white, draw = black] at (axis cs:0,0) {};
                  \end{axis}
              \end{tikzpicture}
          \end{figure}

    \item Finding the type and stability of critical point,
          \begin{align}
              y_1'                       & = y_1 + 2y_2          &
              y_2'                       & = 2y_1 + y_2            \\
              \vec{A}                    & = \bmattt{1}{2}{2}{1} &
              \lambda^{2} - 2\lambda - 3 & = 0
          \end{align}
          Using $ p = 2,\ q = -3,\ \Delta = 16 $ gives a saddle point at
          $ (0, 0) $.
          \begin{align}
              \lambda_1     & = -1                        &
              \vec{v^{(1)}} & = \bmatcol{-1}{1}             \\
              \lambda_2     & = 3                         &
              \vec{v^{(2)}} & = \bmatcol{1}{1}              \\
              \vec{y}       & = c_1 e^{-t}\bmatcol{-1}{1}
              + c_2 e^{3t} \bmatcol{1}{1}
          \end{align}
          \begin{figure}[H]
              \centering
              \begin{tikzpicture}
                  \begin{axis}[
                          declare function = {
                                  u(\x) = e^(-\x);
                                  v(\x) = e^(3*\x);
                              },
                          xmin = -8, xmax = 8, ymin = -8, ymax = 8,
                          % restrict y to domain = -1:1,
                          title = {Saddle point with
                                  ${\color{y_p} \vec{x^{(1)}}}$ and
                                  ${\color{y_h} \vec{x^{(2)}}}$},
                          xlabel = $ y_1 $,
                          ylabel = $ y_2 $,
                          axis equal,
                          width = 8cm,
                          legend pos = north west,
                          grid = both,
                          domain = -1.75:1,
                          Ani]
                      \foreach \c in {2} {%
                              \edef\temp{%
                                  \noexpand \addplot[ samples = 100, color=blue3,
                                      arrow inside={end=stealth,opt={scale=2}}{0.4}]
                                  ({-\c*u(x) + v(x)}, {\c*u(x) + v(x)});
                                  \noexpand \addplot[ samples = 100, color=blue3,
                                      arrow inside={end=stealth,opt={scale=2}}{0.4}]
                                  ({\c*u(x) + v(x)}, {-\c*u(x) + v(x)});
                                  \noexpand \addplot[ samples = 100, color=blue3,
                                      arrow inside={end=stealth,opt={scale=2}}{0.4}]
                                  ({-\c*u(x) - v(x)}, {\c*u(x) - v(x)});
                                  \noexpand \addplot[ samples = 100, color=blue3,
                                      arrow inside={end=stealth,opt={scale=2}}{0.4}]
                                  ({\c*u(x) - v(x)}, {-\c*u(x) - v(x)});
                              }\temp
                          }
                      \addplot[ samples = 100, color=y_p,
                          arrow inside={end=stealth,opt={scale=2}}{0.75}]
                      ({-u(x)}, {u(x)});
                      \addplot[ samples = 100, color=y_h,
                          arrow inside={end=stealth,opt={scale=2}}{0.15}]
                      ({v(x)}, {v(x)});
                      \addplot[ samples = 100, color=y_p,
                          arrow inside={end=stealth,opt={scale=2}}{0.75}]
                      ({u(x)}, {-u(x)});
                      \addplot[ samples = 100, color=y_h,
                          arrow inside={end=stealth,opt={scale=2}}{0.15}]
                      ({-v(x)}, {-v(x)});
                      \node[GraphNode, fill = white, draw = black] at (axis cs:0,0) {};
                  \end{axis}
              \end{tikzpicture}
          \end{figure}

    \item Finding the type and stability of critical point,
          \begin{align}
              y_1'                        & = -y_1 + 4y_2           &
              y_2'                        & = 3y_1 - 2y_2             \\
              \vec{A}                     & = \bmattt{-1}{4}{3}{-2} &
              \lambda^{2} + 3\lambda - 10 & = 0
          \end{align}
          Using $ p = -3,\ q = -10,\ \Delta =49 $ gives a saddle point at
          $ (0, 0) $.
          \begin{align}
              \lambda_1     & = -5                         &
              \vec{v^{(1)}} & = \bmatcol{-1}{1}              \\
              \lambda_2     & = 2                          &
              \vec{v^{(2)}} & = \bmatcol{4}{3}               \\
              \vec{y}       & = c_1 e^{-5t}\bmatcol{-1}{1}
              + c_2 e^{2t} \bmatcol{4}{3}
          \end{align}
          \begin{figure}[H]
              \centering
              \begin{tikzpicture}
                  \begin{axis}[
                          declare function = {
                                  u(\x) = e^(-5*\x);
                                  v(\x) = e^(2*\x);
                              },
                          xmin = -10, xmax = 10, ymin = -10, ymax = 10,
                          % restrict y to domain = -1:1,
                          title = {Saddle point with
                                  ${\color{y_p} \vec{x^{(1)}}}$ and
                                  ${\color{y_h} \vec{x^{(2)}}}$},
                          xlabel = $ y_1 $,
                          ylabel = $ y_2 $,
                          axis equal,
                          width = 8cm,
                          legend pos = north west,
                          grid = both,
                          domain = -0.5:0.5,
                          Ani]
                      \foreach \c in {2} {%
                              \edef\temp{%
                                  \noexpand \addplot[ samples = 100, color=blue3,
                                      arrow inside={end=stealth,opt={scale=2}}{0.75}]
                                  ({-\c*u(x) + 4*v(x)}, {\c*u(x) + 3*v(x)});
                                  \noexpand \addplot[ samples = 100, color=blue3,
                                      arrow inside={end=stealth,opt={scale=2}}{0.75}]
                                  ({\c*u(x) + 4*v(x)}, {-\c*u(x) + 3*v(x)});
                                  \noexpand \addplot[ samples = 100, color=blue3,
                                      arrow inside={end=stealth,opt={scale=2}}{0.75}]
                                  ({-\c*u(x) - 4*v(x)}, {\c*u(x) - 3*v(x)});
                                  \noexpand \addplot[ samples = 100, color=blue3,
                                      arrow inside={end=stealth,opt={scale=2}}{0.75}]
                                  ({\c*u(x) - 4*v(x)}, {-\c*u(x) - 3*v(x)});
                              }\temp
                          }
                      \addplot[ samples = 100, color=y_p,
                          arrow inside={end=stealth,opt={scale=2}}{0.75}]
                      ({-u(x)}, {u(x)});
                      \addplot[ samples = 100, color=y_h, domain = -1.5:1,
                          arrow inside={end=stealth,opt={scale=2}}{0.2}]
                      ({4*v(x)}, {3*v(x)});
                      \addplot[ samples = 100, color=y_p,
                          arrow inside={end=stealth,opt={scale=2}}{0.75}]
                      ({u(x)}, {-u(x)});
                      \addplot[ samples = 100, color=y_h, domain = -1.5:1,
                          arrow inside={end=stealth,opt={scale=2}}{0.2}]
                      ({-4*v(x)}, {-3*v(x)});
                      \node[GraphNode, fill = white, draw = black] at (axis cs:0,0) {};
                  \end{axis}
              \end{tikzpicture}
          \end{figure}

    \item Finding the type and stability of critical point,
          \begin{align}
              y_1'                        & = 4y_1 + y_2          &
              y_2'                        & = 4y_1 + 4y_2           \\
              \vec{A}                     & = \bmattt{4}{1}{4}{4} &
              \lambda^{2} - 8\lambda + 12 & = 0
          \end{align}
          Using $ p = 8,\ q = 12,\ \Delta = 16 $ gives an improper node at
          $ (0, 0) $.
          \begin{align}
              \lambda_1     & = 2                         &
              \vec{v^{(1)}} & = \bmatcol{-1}{2}             \\
              \lambda_2     & = 6                         &
              \vec{v^{(2)}} & = \bmatcol{1}{2}              \\
              \vec{y}       & = c_1 e^{2t}\bmatcol{-1}{2}
              + c_2 e^{6t} \bmatcol{1}{2}
          \end{align}
          \begin{figure}[H]
              \centering
              \begin{tikzpicture}
                  \begin{axis}[
                          declare function = {
                                  u(\x) = e^(2*\x);
                                  v(\x) = e^(6*\x);
                              },
                          xmin = -2, xmax = 2, ymin = -2, ymax = 2,
                          % restrict y to domain = -1:1,
                          title = {Improper Node with
                                  ${\color{y_p} \vec{x^{(1)}}}$ and
                                  ${\color{y_h} \vec{x^{(2)}}}$},
                          xlabel = $ y_1 $,
                          ylabel = $ y_2 $,
                          axis equal,
                          width = 8cm,
                          legend pos = north west,
                          grid = both,
                          domain = -2:0.1,
                          Ani]
                      \foreach \c in {1, 0.5} {%
                              \edef\temp{%
                                  \noexpand \addplot[ samples = 100, color=blue3,
                                      arrow inside={end=stealth,opt={scale=2}}{0.25}]
                                  ({-\c*u(x) + v(x)}, {2*\c*u(x) + 2*v(x)});
                                  \noexpand \addplot[ samples = 100, color=blue3,
                                      arrow inside={end=stealth,opt={scale=2}}{0.25}]
                                  ({\c*u(x) + v(x)}, {-2*\c*u(x) + 2*v(x)});
                                  \noexpand \addplot[ samples = 100, color=blue3,
                                      arrow inside={end=stealth,opt={scale=2}}{0.25}]
                                  ({-\c*u(x) - v(x)}, {2*\c*u(x) - 2*v(x)});
                                  \noexpand \addplot[ samples = 100, color=blue3,
                                      arrow inside={end=stealth,opt={scale=2}}{0.25}]
                                  ({\c*u(x) - v(x)}, {-2*\c*u(x) - 2*v(x)});
                              }\temp
                          }
                      \addplot[ samples = 100, color=y_p,
                          arrow inside={end=stealth,opt={scale=2}}{0.5}]
                      ({-u(x)}, {2*u(x)});
                      \addplot[ samples = 100, color=y_h,
                          arrow inside={end=stealth,opt={scale=2}}{0.4}]
                      ({v(x)}, {2*v(x)});
                      \addplot[ samples = 100, color=y_p,
                          arrow inside={end=stealth,opt={scale=2}}{0.5}]
                      ({u(x)}, {-2*u(x)});
                      \addplot[ samples = 100, color=y_h,
                          arrow inside={end=stealth,opt={scale=2}}{0.4}]
                      ({-v(x)}, {-2*v(x)});
                      \node[GraphNode, fill = white, draw = black] at (axis cs:0,0) {};
                  \end{axis}
              \end{tikzpicture}
          \end{figure}

    \item Finding the type and stability of critical point,
          \begin{align}
              y_1'                        & = y_2                   &
              y_2'                        & = -5y_1 - 2y_2            \\
              \vec{A}                     & = \bmattt{0}{1}{-5}{-2} &
              \lambda^{2} - 8\lambda + 12 & = 0
          \end{align}
          Using $ p = 8,\ q = 12,\ \Delta = 16 $ gives a spiral at
          $ (0, 0) $.
          \begin{align}
              \lambda_1 & = -1 - 2i                                \qquad\qquad
              \vec{v^{(1)}}  = \bmatcol{-0.2+0.4i}{1}                           \\
              \lambda_2 & = -1 + 2i                                \qquad\qquad
              \vec{v^{(2)}}  = \bmatcol{-0.2-0.4i}{1}                           \\
              \vec{y}   & = c_1 e^{(-1-2i)t}\bmatcol{-0.2+0.4i}{1}
              + c_2 e^{(-1+2i)t} \bmatcol{-0.2-0.4i}{1}                         \\
              \vec{y}   & = e^{-t}c_1 \bmatcol{-0.2\cos(2t)
              + 0.4\sin(2t)}{\cos(2t)}  \nonumber                               \\
                        & + e^{-t}c_2 \bmatcol{-0.2\cos(2t)
                  + 0.4\sin(2t)}{\cos(2t)}
          \end{align}
          \begin{figure}[H]
              \centering
              \begin{tikzpicture}
                  \begin{axis}[
                          xmin = -1, xmax = 1, ymin = -1, ymax = 1,
                          % restrict y to domain = -1:1,
                          declare function = {
                                  u(\x) = e^(-\x)*(-0.2*cos(2*\x) + 0.4*sin(2*\x));
                                  v(\x) = e^(-\x)*cos(2*\x);
                              },
                          title = {Spiral},
                          xlabel = $ y_1 $,
                          ylabel = $ y_2 $,
                          axis equal,
                          width = 8cm,
                          legend pos = north west,
                          grid = both,
                          domain = 0:2*pi,
                          Ani]
                      \foreach \c in {2} {%
                              \edef\temp{%
                                  \noexpand \addplot[ samples = 200, color=blue3,
                                      arrow inside={end=stealth,opt={scale=2}}{0.6}]
                                  ({(\c + 1)*u(x)}, {(\c + 1)*v(x)});
                                  \noexpand \addplot[ samples = 200, color=blue3,
                                      arrow inside={end=stealth,opt={scale=2}}{0.6}]
                                  ({(-\c + 1)*u(x)}, {(-\c + 1)*v(x)});
                                  \noexpand \addplot[ samples = 200, color=blue3,
                                      arrow inside={end=stealth,opt={scale=2}}{0.6}]
                                  ({(\c - 1)*u(x)}, {(\c - 1)*v(x)});
                                  \noexpand \addplot[ samples = 200, color=blue3,
                                      arrow inside={end=stealth,opt={scale=2}}{0.6}]
                                  ({(-\c - 1)*u(x)}, {(-\c - 1)*v(x)});
                              }\temp
                          }
                      \node[GraphNode, fill = white, draw = black] at (axis cs:0,0) {};
                  \end{axis}
              \end{tikzpicture}
          \end{figure}

    \item Damped oscillations,
          \begin{align}
              y'' + 2y' + 2y             & = 0                     &
              y_1                        & = y, \qquad y_2 = y'      \\
              y_2'                       & = -2y_1 -2y_2           &
              y_1'                       & = y_2                     \\
              \vec{A}                    & = \bmattt{0}{1}{-2}{-2} &
              \lambda^{2} + 2\lambda + 2 & = 0
          \end{align}
          Using $ p = -2,\ q = 2,\ \Delta = -4 $ gives a spiral at
          $ (0, 0) $.
          \begin{align}
              \lambda_1 & = -1 - \i                                \qquad\qquad
              \vec{v^{(1)}}  = \bmatcol{-1+\i}{2}                               \\
              \lambda_2 & = -1 + \i                                \qquad\qquad
              \vec{v^{(2)}}  = \bmatcol{-1-\i}{2}                               \\
              \vec{y}   & = c_1 e^{(-1-\i)t}\bmatcol{-1+\i}{2}
              + c_2 e^{(-1+\i)t} \bmatcol{-1-\i}{2}                             \\
              \vec{y}   & = e^{-t}c_1 \bmatcol{\cos(t)+ \sin(t)}{2\cos(t)}
              + e^{-t}c_2 \bmatcol{\cos(t) + \sin(t)}{2\cos(t)}
          \end{align}

    \item Harmonic oscillations,
          \begin{align}
              y'' + \frac{1}{9}y        & = 0                      &
              y_1                       & = y, \qquad y_2 = y'       \\
              y_2'                      & = -\frac{1}{9}y_1        &
              y_1'                      & = y_2                      \\
              \vec{A}                   & = \bmattt{0}{1}{-1/9}{0} &
              \lambda^{2} + \frac{1}{9} & = 0
          \end{align}
          Using $ p = 0,\ q = 1/9,\ \Delta = -4/9 $ gives a center at
          $ (0, 0) $.
          \begin{align}
              \lambda_1 & = \frac{-\i}{3}                                \qquad\qquad
              \vec{v^{(1)}}  = \bmatcol{3i}{1}                                        \\
              \lambda_2 & = \frac{\i}{3}                                \qquad\qquad
              \vec{v^{(2)}}  = \bmatcol{-3i}{1}                                       \\
              \vec{y}   & = c_1 e^{-t\ \i/3}\bmatcol{3\ \i}{1}
              + c_2 e^{t\ \i/3} \bmatcol{-3\ \i}{1}                                   \\
              \vec{y}   & = c_1 \bmatcol{3\sin(t/3)}{\cos(t/3)}
              + c_2 \bmatcol{3\sin(t/3)}{\cos(t/3)}
          \end{align}
          \begin{figure}[H]
              \centering
              \begin{tikzpicture}
                  \begin{axis}[
                          % xmin = -1, xmax = 1, ymin = -1, ymax = 1,
                          % restrict y to domain = -1:1,
                          declare function = {
                                  u(\x) = 3*sin(\x/3);
                                  v(\x) = cos(\x/3);
                              },
                          title = {Center},
                          xlabel = $ y_1 $,
                          ylabel = $ y_2 $,
                          axis equal,
                          width = 8cm,
                          legend pos = north west,
                          grid = both,
                          domain = 0:6*pi,
                          Ani]
                      \foreach \c in {0.5, 2, 4} {%
                              \edef\temp{%
                                  \noexpand \addplot[ samples = 200, color=blue3,
                                      arrow inside={end=stealth,opt={scale=2}}{0.5}]
                                  ({(\c + 1)*u(x)}, {(\c + 1)*v(x)});
                              }\temp
                          }
                      \node[GraphNode, fill = white, draw = black] at (axis cs:0,0) {};
                  \end{axis}
              \end{tikzpicture}
          \end{figure}

    \item Refer to notes for sections 4.3 and 4.4

    \item From example 1, the general solution after making the change
          $ -t \to \tau $ is, the nature of the critical point remains
          the same, except that the sink becomes a source. \par
          This follows from the indices of the exponential becoming both
          positive.
          \begin{align}
              y & = c_1 \bmatcol{1}{1}e^{2\tau} + c_2\bmatcol{-1}{1}e^{4\tau}
          \end{align}
          \begin{figure}[H]
              \centering
              \begin{tikzpicture}
                  \begin{axis}[
                          declare function = {
                                  u(\x) = e^(-2*\x);
                                  v(\x) = e^(-4*\x);
                              },
                          xmin = -1, xmax = 1, ymin = -1, ymax = 1,
                          % restrict y to domain = -1:1,
                          title = {Improper Sink node with
                                  ${\color{y_p} \vec{x^{(1)}}}$ and
                                  ${\color{y_h} \vec{x^{(2)}}}$},
                          xlabel = $ y_1 $,
                          ylabel = $ y_2 $,
                          axis equal,
                          width = 8cm,
                          legend pos = north west,
                          grid = both,
                          domain = 0:3,
                          Ani]
                      \foreach \c in {-0.5, -1, 0.5, 1} {%
                              \edef\temp{%
                                  \noexpand \addplot[ samples = 100, color=blue3,
                                      arrow inside={end=stealth,opt={scale=2}}{0.65}]
                                  ({\c*u(x) + v(x)}, {\c*u(x) - v(x)});
                                  \noexpand \addplot[ samples = 100, color=blue3,
                                      arrow inside={end=stealth,opt={scale=2}}{0.65}]
                                  ({\c*u(x) - v(x)}, {\c*u(x) + v(x)});
                              }\temp
                          }
                      \addplot[ samples = 100, color=y_p,
                          arrow inside={end=stealth,opt={scale=2}}{0.5}]
                      ({u(x)}, {u(x)});
                      \addplot[ samples = 100, color=y_h,
                          arrow inside={end=stealth,opt={scale=2}}{0.5}]
                      ({v(x)}, {-v(x)});
                      \addplot[ samples = 100, color=y_p,
                          arrow inside={end=stealth,opt={scale=2}}{0.5}]
                      ({-u(x)}, {-u(x)});
                      \addplot[ samples = 100, color=y_h,
                          arrow inside={end=stealth,opt={scale=2}}{0.5}]
                      ({-v(x)}, {v(x)});
                      \node[GraphNode, fill = white, draw = black] at (axis cs:0,0)
                      {};
                  \end{axis}
              \end{tikzpicture}
              \begin{tikzpicture}
                  \begin{axis}[
                          declare function = {
                                  u(\x) = e^(2*\x);
                                  v(\x) = e^(4*\x);
                              },
                          xmin = -1, xmax = 1, ymin = -1, ymax = 1,
                          % restrict y to domain = -1:1,
                          title = {Improper Source node with
                                  ${\color{y_p} \vec{x^{(1)}}}$ and
                                  ${\color{y_h} \vec{x^{(2)}}}$},
                          xlabel = $ y_1 $,
                          ylabel = $ y_2 $,
                          axis equal,
                          width = 8cm,
                          legend pos = north west,
                          grid = both,
                          domain = -3:0,
                          Ani]
                      \foreach \c in {-0.5, -1, 0.5, 1} {%
                              \edef\temp{%
                                  \noexpand \addplot[ samples = 100, color=blue3,
                                      arrow inside={end=stealth,opt={scale=2}}{0.35}]
                                  ({\c*u(x) + v(x)}, {\c*u(x) - v(x)});
                                  \noexpand \addplot[ samples = 100, color=blue3,
                                      arrow inside={end=stealth,opt={scale=2}}{0.35}]
                                  ({\c*u(x) - v(x)}, {\c*u(x) + v(x)});
                              }\temp
                          }
                      \addplot[ samples = 100, color=y_p,
                          arrow inside={end=stealth,opt={scale=2}}{0.5}]
                      ({u(x)}, {u(x)});
                      \addplot[ samples = 100, color=y_h,
                          arrow inside={end=stealth,opt={scale=2}}{0.5}]
                      ({v(x)}, {-v(x)});
                      \addplot[ samples = 100, color=y_p,
                          arrow inside={end=stealth,opt={scale=2}}{0.5}]
                      ({-u(x)}, {-u(x)});
                      \addplot[ samples = 100, color=y_h,
                          arrow inside={end=stealth,opt={scale=2}}{0.5}]
                      ({-v(x)}, {v(x)});
                      \node[GraphNode, fill = white, draw = black] at (axis cs:0,0)
                      {};
                  \end{axis}
              \end{tikzpicture}
          \end{figure}

    \item The new system is, for some small perturbation $ \mu > 0 $,
          \begin{align}
              \vec{y}       & = \bmattt{\mu}{1}{-4}{\mu}                           &
              0             & = \lambda^{2} - (2\mu)\lambda + (4 + \mu^{2})          \\
              p             & = 2\mu \qquad q = (4 + \mu^{2})                      &
              \Delta        & = -16                                                  \\
              \lambda_1     & = \mu - 2\ \i                                        &
              \vec{v^{(1)}} & = \bmatcol{\i}{2}                                      \\
              \lambda_2     & = \mu + 2\ \i                                        &
              \vec{v^{(2)}} & = \bmatcol{-\i}{2}                                     \\
              \vec{y}       & = e^{\mu t} (c_1 + c_2)\bmatcol{\sin(2t)}{2\cos(2t)}
          \end{align}

          This spiral is symmetric in $ y_1 $ and $ y_2 $ and for the special
          case $ \mu = 0 $, reduces to a center. In this problem, $ \mu = 0.1 $.

    \item From the previous problem, $ \mu \neq 0 $, causes the center to become
          a spiral, which is a source $ (\mu > 0) $ or sink $ (\mu < 0) $.

    \item Types of perturbation $ a_{jk} \to a_{jk} + b $ required to
          produce different kinds of critical points.
          \begin{enumerate}
              \item To get a saddle point,
                    \begin{align}
                        \vec{A}               & = \bmattt{b}{1+b}{-4+b}{b}           \\
                        0                     & = \lambda^{2} + 2b\lambda + (3b + 4) \\
                        q = (3b+4)            & < 0 \qquad \qquad
                        b \in \left( -\infty, \frac{-4}{3} \right)                   \\
                        \Delta = b^2 - 3b - 4 & > 0 \qquad\qquad
                        b \in (-\infty,-1) \cup (4, \infty)                          \\
                        \text{Saddle point}\  & \implies b \in
                        \left( -\infty, \frac{-4}{3} \right)
                    \end{align}
              \item To get a stable and attractive node.
                    \begin{align}
                        \vec{A}                             & = \bmattt{b}{1+b}
                        {-4+b}{b}                                                 \\
                        0                                   &
                        = \lambda^{2} + 2b\lambda + (3b + 4)                      \\
                        q = (3b+4)                          & > 0 \qquad \qquad
                        b \in \left( \frac{-4}{3}, \infty \right)                 \\
                        p = -2b                             & < 0 \qquad\qquad
                        b \in (0, \infty)                                         \\
                        \Delta = b^2 - 3b - 4               & \geq 0 \qquad\qquad
                        b \in (-\infty,-1] \cup [4, \infty)                       \\
                        \text{Stable and attractive node}\  & \implies b \in
                        [4, \infty)
                    \end{align}
              \item To get a stable and attractive spiral.
                    \begin{align}
                        \vec{A}                               &
                        = \bmattt{b}{1+b}{-4+b}{b}                                \\
                        0                                     &
                        = \lambda^{2} + 2b\lambda + (3b + 4)                      \\
                        q = (3b+4)                            & > 0 \qquad \qquad
                        b \in \left( \frac{-4}{3}, \infty \right)                 \\
                        \Delta = b^2 - 3b - 4                 & < 0 \qquad\qquad
                        b \in (-1, 4)                                             \\
                        p = -2b                               & < 0 \qquad\qquad
                        b \in (0, \infty)                                         \\
                        \text{Stable and attractive spiral}\  & \implies b \in
                        (0, 4)
                    \end{align}
              \item To get a unstable spiral,
                    \begin{align}
                        \vec{A}                  & = \bmattt{b}{1+b}{-4+b}{b} \\
                        0                        & = \lambda^{2} + 2b\lambda
                        + (3b + 4)                                            \\
                        q = (3b+4)               & > 0 \qquad \qquad
                        b \in \left( \frac{-4}{3}, \infty \right)             \\
                        \Delta = b^2 - 3b - 4    & < 0 \qquad\qquad
                        b \in (-1, 4)                                         \\
                        p = -2b                  & > 0 \qquad\qquad
                        b \in (-\infty, 0)                                    \\
                        \text{Unstable spiral}\  & \implies b \in
                        (-1, 0)
                    \end{align}
              \item To get a unstable node,
                    \begin{align}
                        \vec{A}                  & = \bmattt{b}{1+b}{-4+b}{b} \\
                        0                        & = \lambda^{2} + 2b\lambda
                        + (3b + 4)                                            \\
                        q = (3b+4)               & > 0 \qquad \qquad
                        b \in \left( \frac{-4}{3}, \infty \right)             \\
                        \Delta = b^2 - 3b - 4    & \geq 0 \qquad\qquad
                        b \in (-\infty,-1] \cup [4, \infty)                   \\
                        p = -2b                  & > 0 \qquad\qquad
                        b \in (-\infty, 0)                                    \\
                        \text{Unstable spiral}\  & \implies b \in
                        \Bigg( \frac{-4}{3}, -1 \Bigg]
                    \end{align}
          \end{enumerate}
    \item Plotting some phase portraits. Cant show the effect of continuous change
          in $ b $ on paper.
          \begin{figure}[H]
              \centering
              \begin{tikzpicture}
                  \begin{axis}[
                          declare function = {
                                  u(\x) = e^(-2-sqrt(6)*\x);
                                  v(\x) = e^(-2+sqrt(6)*\x);
                              },
                          xmin = -3, xmax = 3, ymin = -3, ymax = 3,
                          title style = {align = center},
                          title = {Saddle point with
                                  $ b = -2 $ \\ ${\color{y_p} \vec{x^{(1)}}}$
                                  and ${\color{y_h} \vec{x^{(2)}}}$},
                          xlabel = $ y_1 $,
                          ylabel = $ y_2 $,
                          axis equal,
                          width = 8cm,
                          legend pos = north west,
                          grid = both,
                          domain = -0.75:0.75,
                          Ani]
                      \foreach \c in {1, 2} {%
                              \edef\temp{%
                                  \noexpand \addplot[ samples = 100, color=blue3,
                                      arrow inside={end=stealth,opt={scale=2}}{0.65}]
                                  ({-\c*sqrt(6)*u(x) + sqrt(6)*v(x)},
                                  {\c*6*u(x) + 6*v(x)});
                                  \noexpand \addplot[ samples = 100, color=blue3,
                                      arrow inside={end=stealth,opt={scale=2}}{0.65}]
                                  ({\c*sqrt(6)*u(x) + sqrt(6)*v(x)},
                                  {-\c*6*u(x) + 6*v(x)});
                                  \noexpand \addplot[ samples = 100, color=blue3,
                                      arrow inside={end=stealth,opt={scale=2}}{0.65}]
                                  ({-\c*sqrt(6)*u(x) - sqrt(6)*v(x)},
                                  {\c*6*u(x) - 6*v(x)});
                                  \noexpand \addplot[ samples = 100, color=blue3,
                                      arrow inside={end=stealth,opt={scale=2}}{0.65}]
                                  ({\c*sqrt(6)*u(x) - sqrt(6)*v(x)},
                                  {-\c*6*u(x) - 6*v(x)});
                              }\temp
                          }
                      \addplot[ samples = 100, color=y_p,
                          arrow inside={end=stealth,opt={scale=2}}{0.75}]
                      ({-sqrt(6)*u(x)}, {6*u(x)});
                      \addplot[ samples = 100, color=y_h,
                          arrow inside={end=stealth,opt={scale=2}}{0.2}]
                      ({sqrt(6)*v(x)}, {6*v(x)});
                      \addplot[ samples = 100, color=y_p,
                          arrow inside={end=stealth,opt={scale=2}}{0.75}]
                      ({sqrt(6)*u(x)}, {-6*u(x)});
                      \addplot[ samples = 100, color=y_h,
                          arrow inside={end=stealth,opt={scale=2}}{0.2}]
                      ({-sqrt(6)*v(x)}, {-6*v(x)});
                      \node[GraphNode, fill = white, draw = black] at (axis cs:0,0)
                      {};
                  \end{axis}
              \end{tikzpicture}
              \begin{tikzpicture}
                  \begin{axis}[
                          xmin = -1, xmax = 1, ymin = -1, ymax = 1,
                          % restrict y to domain = -1:1,
                          title style = {align = center},
                          title = {Degenerate Node with $ b = -1 $ \\
                                  ${\color{y_p} \vec{x^{(1)}}}$ and
                                  ${\color{y_h} \vec{x^{(2)}}}$},
                          xlabel = $ y_1 $,
                          ylabel = $ y_2 $,
                          axis equal,
                          width = 8cm,
                          legend pos = north west,
                          grid = both,
                          domain = -1:3,
                          Ani]
                      \foreach \c in {1/2} {%
                              \edef\temp{%
                                  \noexpand \addplot[thin, samples = 100, color=blue3,
                                      arrow inside={end=stealth,opt={scale=2}}{0.75}]
                                  ({(-0.2)*e^(-x)}, {(\c + x)*e^(-x)});
                                  \noexpand \addplot[thin, samples = 100, color=blue3,
                                      arrow inside={end=stealth,opt={scale=2}}{0.75}]
                                  ({(-0.2)*e^(-x)}, {(-\c + x)*e^(-x)});
                                  \noexpand \addplot[thin, samples = 100, color=blue3,
                                      arrow inside={end=stealth,opt={scale=2}}{0.75}]
                                  ({(0.2)*e^(-x)}, {(\c - x)*e^(-x)});
                                  \noexpand \addplot[thin, samples = 100, color=blue3,
                                      arrow inside={end=stealth,opt={scale=2}}{0.75}]
                                  ({(0.2)*e^(-x)}, {(-\c - x)*e^(-x)});
                              }\temp
                          }
                      \addplot[thick, samples = 100, color=y_p,
                          arrow inside={end=stealth,opt={scale=2}}{0.8}]
                      ({0}, {e^(-x)});
                      \addplot[thick, samples = 100, color=y_p,
                          arrow inside={end=stealth,opt={scale=2}}{0.8}]
                      ({0}, {-e^(-x)});
                      \addplot[thick, samples = 100, color=y_h,
                          arrow inside={end=stealth,opt={scale=2}}{0.8}]
                      ({-0.2*e^(-x)}, {(x)*e^(-x)});
                      \addplot[thick, samples = 100, color=y_h,
                          arrow inside={end=stealth,opt={scale=2}}{0.8}]
                      ({0.2*e^(-x)}, {(-x)*e^(-x)});
                      \node[GraphNode, fill = white, draw = black] at (axis cs:0,0)
                      {};
                  \end{axis}
              \end{tikzpicture}
          \end{figure}
          \begin{figure}[H]
              \centering
              \begin{tikzpicture}
                  \begin{axis}[
                          xmin = -1, xmax = 1, ymin = -1, ymax = 1,
                          % restrict y to domain = -1:1,
                          declare function = {
                                  u(\x) = e^(-0.5*\x)*sin(1.5*\x);
                                  v(\x) = e^(-0.5*\x)*3*cos(1.5*\x);
                              },
                          title = {Stable Spiral with $ b = -0.5 $},
                          xlabel = $ y_1 $,
                          ylabel = $ y_2 $,
                          axis equal,
                          width = 8cm,
                          legend pos = north west,
                          grid = both,
                          domain = 0:2.5*pi,
                          Ani]
                      \foreach \c in {1/4} {%
                              \edef\temp{%
                                  \noexpand \addplot[ samples = 200, color=y_h,
                                      arrow inside={end=stealth,opt={scale=2}}{0.75}]
                                  ({(\c + 1)*u(x)}, {(\c + 1)*v(x)});
                                  \noexpand \addplot[ samples = 200, color=y_h,
                                      arrow inside={end=stealth,opt={scale=2}}{0.75}]
                                  ({(-\c + 1)*u(x)}, {(-\c + 1)*v(x)});
                                  \noexpand \addplot[ samples = 200, color=y_p,
                                      arrow inside={end=stealth,opt={scale=2}}{0.75}]
                                  ({(\c - 1)*u(x)}, {(\c - 1)*v(x)});
                                  \noexpand \addplot[ samples = 200, color=y_p,
                                      arrow inside={end=stealth,opt={scale=2}}{0.75}]
                                  ({(-\c - 1)*u(x)}, {(-\c - 1)*v(x)});
                              }\temp
                          }
                      \node[GraphNode, fill = white, draw = black] at (axis cs:0,0)
                      {};
                  \end{axis}
              \end{tikzpicture}
              \begin{tikzpicture}
                  \begin{axis}[
                          %   xmin = -1, xmax = 1, ymin = -1, ymax = 1,
                          % restrict y to domain = -1:1,
                          declare function = {
                                  u(\x) = e^(\x)*sqrt(6)*sin(sqrt(6)*\x);
                                  v(\x) = e^(\x)*3*cos(sqrt(6)*\x);
                              },
                          title = {Unstable Spiral with $ b = 1 $},
                          xlabel = $ y_1 $,
                          ylabel = $ y_2 $,
                          axis equal,
                          width = 8cm,
                          legend pos = north west,
                          grid = both,
                          domain = -3:1,
                          Ani]
                      \foreach \c in {1/4} {%
                              \edef\temp{%
                                  \noexpand \addplot[ samples = 200, color=y_h,
                                      arrow inside={end=stealth,opt={scale=2}}{0.75}]
                                  ({(\c + 1)*u(x)}, {(\c + 1)*v(x)});
                                  \noexpand \addplot[ samples = 200, color=y_h,
                                      arrow inside={end=stealth,opt={scale=2}}{0.75}]
                                  ({(-\c + 1)*u(x)}, {(-\c + 1)*v(x)});
                                  \noexpand \addplot[ samples = 200, color=y_p,
                                      arrow inside={end=stealth,opt={scale=2}}{0.75}]
                                  ({(\c - 1)*u(x)}, {(\c - 1)*v(x)});
                                  \noexpand \addplot[ samples = 200, color=y_p,
                                      arrow inside={end=stealth,opt={scale=2}}{0.75}]
                                  ({(-\c - 1)*u(x)}, {(-\c - 1)*v(x)});
                              }\temp
                          }
                      \node[GraphNode, fill = white, draw = black] at (axis cs:0,0)
                      {};
                  \end{axis}
              \end{tikzpicture}
          \end{figure}
          \begin{figure}[H]
              \centering
              \begin{tikzpicture}
                  \begin{axis}[
                          xmin = -1, xmax = 1, ymin = -1, ymax = 1,
                          % restrict y to domain = -1:1,
                          title style = {align = center},
                          title = {Degenerate Node with $ b = 4 $ \\
                                  ${\color{y_p} \vec{x^{(1)}}}$ and
                                  ${\color{y_h} \vec{x^{(2)}}}$},
                          xlabel = $ y_1 $,
                          ylabel = $ y_2 $,
                          axis equal,
                          width = 8cm,
                          legend pos = north west,
                          grid = both,
                          domain = -0.75:0.5,
                          Ani]
                      \foreach \c in {1/2} {%
                              \edef\temp{%
                                  \noexpand \addplot[thin, samples = 100, color=blue3,
                                      arrow inside={end=stealth,opt={scale=2}}{0.1}]
                                  ({(\c + x)*e^(4*x)}, {0.2*e^(4*x)});
                                  \noexpand \addplot[thin, samples = 100, color=blue3,
                                      arrow inside={end=stealth,opt={scale=2}}{0.1}]
                                  ({(-\c + x)*e^(4*x)}, {0.2*e^(4*x)});
                                  \noexpand \addplot[thin, samples = 100, color=blue3,
                                      arrow inside={end=stealth,opt={scale=2}}{0.1}]
                                  ({(\c - x)*e^(4*x)}, {-0.2*e^(4*x)});
                                  \noexpand \addplot[thin, samples = 100, color=blue3,
                                      arrow inside={end=stealth,opt={scale=2}}{0.1}]
                                  ({(-\c - x)*e^(4*x)}, {-0.2*e^(4*x)});
                              }\temp
                          }
                      \addplot[thick, samples = 100, color=y_p,
                          arrow inside={end=stealth,opt={scale=2}}{0.1}]
                      ({e^(4*x)}, {0});
                      \addplot[thick, samples = 100, color=y_p,
                          arrow inside={end=stealth,opt={scale=2}}{0.1}]
                      ({-e^(4*x)}, {0});
                      \addplot[thick, samples = 100, color=y_h,
                          arrow inside={end=stealth,opt={scale=2}}{0.2}]
                      ({x*e^(4*x)}, {(0.2)*e^(4*x)});
                      \addplot[thick, samples = 100, color=y_h,
                          arrow inside={end=stealth,opt={scale=2}}{0.2}]
                      ({-x*e^(4*x)}, {(-0.2)*e^(4*x)});
                      \node[GraphNode, fill = white, draw = black] at (axis cs:0,0) {};
                  \end{axis}
              \end{tikzpicture}
          \end{figure}

    \item TBC.
    \item Plotting the points on the $ p-q $ plane,
          \begin{figure}[H]
              \centering
              \begin{tikzpicture}
                  \begin{axis}[
                          xmax = 8, xmin = -8, ymax = 12, ymin = -4,
                          % restrict y to domain = -1:1,
                          title = {Stability Chart},
                          axis lines = center,
                          xlabel = {\large $ p $},
                          ylabel = {\large $ q $},
                          width = 12cm, height = 12cm,
                          legend pos = north west,
                          % grid = both,
                          domain = -10:10,
                          Ani]
                      \addplot[GraphSmooth, thick, color = black] {0.25*x^(2)};
                      \node[GraphNode,fill = y_h, label={-45:{E10}}] at (axis cs:2, 1) {};
                      \node[GraphNode,fill = y_h, label={0:{E11}}] at (axis cs:0, -1) {};
                      \node[GraphNode,fill = y_h, label={0:{E12}}] at (axis cs:0, 4) {};
                      \node[GraphNode,fill = y_h, label={90:{E13}}] at (axis cs:-2, 2) {};
                      \node[GraphNode,fill = y_h, label={180:{E14}}] at (axis cs:6, 9) {};
                      \node[GraphNode, fill = y_p, label={0:{P1}}] at (axis cs:3, 2) {};
                      \node[GraphNode, fill = y_p, label={0:{P3}}] at (axis cs:0, 9) {};
                      \node[GraphNode, fill = y_p, label={0:{P5}}] at (axis cs:-4, 8) {};
                  \end{axis}
              \end{tikzpicture}
          \end{figure}
\end{enumerate}