\chapter{Systems of ODEs, Phase Plane, Qualitative Methods}
\section{Systems of ODEs as Models in Engineering Applications}
\begin{description}
    \item[Conversion to system of ODEs] Any ODE of order $ n $ can be converted to a system
        of $ n $ ODEs of first order. Linear algebra provides easy methods of solving this
        system of linear equations using eigenvectors and eigenvalues.
        \begin{align}
            y^{(n)}                          & = F\left( t,\ y,\ y',\ y'',\ \dots,\ y^{(n-1)} \right)  \\
            \text{Set} \qquad y_1            & = y, \qquad y_2 = y', \quad \dots \quad y_n = y^{(n-1)} \\
            \text{System becomes}\qquad y_1' & = y_2 \nonumber                                         \\
            y_2'                             & = y_3 \nonumber                                         \\
            \vdots \nonumber                                                                           \\
            y_{n-1}'                         & = y_n \nonumber                                         \\
            y_n'                             & = F(t,\ y_1,\ y_2,\ \dots,\ y_n)
        \end{align}
    \item[Eigenvalues] When a system of linear equations is expressed in vector form,
        \begin{align}
            \renewcommand*{\arraystretch}{1.5}
            y_1'                 & = a_{11}y_1 + a_{12}y_2 \nonumber         \\
            y_2'                 & = a_{21}y_1 + a_{22}y_2                   \\
            \bmatcol{y_1'}{y_2'} & = \bmattt{a_{11}}{a_{12}}{a_{21}}{a_{22}} \\
            \vec{y'}             & = \vec{A}\vec{y}
        \end{align}
        For a matrix equation to have a non-trivial solution,
        \begin{align}
            \vec{A}\vec{x}                     & = \lambda \vec{x}                         \\
            (\vec{A} - \lambda\vec{I}) \vec{x} & = 0                                       \\
            \det(\vec{A} - \lambda\vec{I})     & = \begin{vNiceMatrix}[r, margin]
                                                       (a_{11} - \lambda) & a_{12}             \\
                                                       a_{21}             & (a_{22} - \lambda)
                                                   \end{vNiceMatrix}
        \end{align}
        This quadratic equation has solutions $ \lambda_1, \lambda_2 $ called eigenvalues.
        The corresponding eigenvectors $ \vec{v^{(1)}}, \vec{v^{(2)}} $ are found as
        solutions of the respective systems.
        \begin{align}
            (\vec{A} - \lambda_1\vec{I}) \vec{v^{(1)}} & = 0 \\
            (\vec{A} - \lambda_2\vec{I}) \vec{v^{(2)}} & = 0
        \end{align}
\end{description}

\section{Basic Theory of Systems of ODEs, Wronskian}
\begin{description}
    \item[General system of ODEs] Using a set of functions $ \{f_i\} $,
        \begin{align}
            y_1'     & = f_1(t,\ y_1,\ \dots,\ y_n) \\
            y_2'     & = f_2(t,\ y_1,\ \dots,\ y_n) \\
            \vdots                                  \\
            y_n'     & = f_n(t,\ y_1,\ \dots,\ y_n) \\
            \text{converting to vector notation,}   \\
            \vec{y'} & = \vec{f}(t, \vec{y})
        \end{align}
        Introducing the set of solutions as a vector, and another vector for the I.C.
        \begin{align}
            \vec{y}      & = \vec{h}(t) = \begin{bNiceMatrix}[r, margin]
                                              h_1(t) \\ h_2(t) \\ \vdots \\ h_n(t)
                                          \end{bNiceMatrix} &
            \vec{y}(t_0) & = \vec{K} = \begin{bNiceMatrix}[r, margin]
                                           K(0) \\ K(1) \\ \vdots \\ K_n
                                       \end{bNiceMatrix}
        \end{align}
    \item[Existence and Uniqueness] Given the preconditions,
        \begin{align}
            \{f_1,\ f_2,\ \dots,\ f_n\} \qquad       &
            \text{are continuous}                      \\
            \left\{ \diffp{f_i}{y_k} \right\} \qquad &
            \text{are continuous for all}\ j,k
        \end{align}
        Continuity is guaranteed in some point $ (t_0,\ K_0,\ K_1,\ \dots,\ K_n) $ in
        this interval of continuity in $ (t,\ y_0,\ y_1,\ \dots,\ y_n) $ space. \par
        Then, a solution to the system of ODEs exists in some interval
        $ t \in (t_0 - \alpha, t_0 + \alpha) $ and is guaranteed to be unique.
    \item[Linear System] A subset of the above general system of ODEs obeying,
        \begin{align}
            \begin{bNiceMatrix}[r, margin]
                y_1' \\ \vdots \\ y_n'
            \end{bNiceMatrix} & = \begin{bNiceMatrix}[r, margin]
                                      a_{11} & \dots  & a_{1n} \\
                                      \vdots & \ddots & \vdots \\
                                      a_{n1} & \dots  & a_{nn}
                                  \end{bNiceMatrix} \begin{bNiceMatrix}[r, margin]
                                                        y_1 \\ \vdots \\ y_n
                                                    \end{bNiceMatrix} + \begin{bNiceMatrix}[r, margin]
                                                                            g_1 \\ \vdots \\ g_n
                                                                        \end{bNiceMatrix} \\
            \vec{y'}                       & = \vec{Ay} + \vec{g}
        \end{align}
        The above system is homogeneous if $ \vec{g} = 0 $, with $ a_{jk} =
            \difsp{f_j}{y_k} $. \par
        If the elements of $ \vec{g} $ and $ \vec{a} $ are continuous functions of
        $ t $ in some interval $ t \in (\alpha, \beta) $ which contains the point
        $ t = t_0 $, then a solution exists and is guaranteed to be unique.
    \item[Superposition Principle] If $ \vec{y^{(1)}} $ and $ \vec{y^{(2)}} $ are
        solutions of the h-linear system of ODEs on some interval, then
        \begin{align}
            \vec{y^{(3)}} & = c_1 \vec{y^{(1)}} + c_2 \vec{y^{(2)}}
        \end{align}
        is also a solution. (using the linearity of matrix multiplication and of
        differentiation)
    \item[Basis] A set of L.I. solutions $ \{ \vec{y^{(i)}} \} $ form a basis in the
        interval $ \mathcal{J} $ if the elements of $ \vec{a} $ are continuous in
        $ \mathcal{J} $. \par

        A linear combination of this basis is a general solution that contains all
        possible solutions.
    \item[Wronskian] The old Wronskian of a set of solutions $ \{z_i\} $ becomes
        a matrix whose columns are members of the basis set $ \vec{y^{(i)}} $. \par

        The rows are successive derivatives (transformed into subscript variables).
        \begin{align}
            \begin{bNiceMatrix}[r, margin]
                z_1         & z_1         & \dots  & z_n         \\
                z_1'        & z_2'        & \dots  & z_n'        \\
                \vdots      & \vdots      & \ddots & \vdots      \\
                z_1^{(n-1)} & z_2^{(n-1)} & \dots  & z_n^{(n-1)}
            \end{bNiceMatrix} & \rightarrow
            \begin{bNiceMatrix}[r, margin]
                y_1^{(1)} & y_1^{(2)} & \dots  & y_1^{(n)} \\
                y_2^{(1)} & y_2^{(2)} & \dots  & y_2^{(n)} \\
                \vdots    & \vdots    & \ddots & \vdots    \\
                y_n^{(1)} & y_n^{(2)} & \dots  & y_n^{(n)}
            \end{bNiceMatrix}                                                \\
            W = \det(\vec{Y})                                   &
            = \det\begin{bNiceMatrix}[r, margin] \vec{y^{(1)}} &
                   \dots                     & \vec{y^{(n)}}\end{bNiceMatrix} \\
        \end{align}
        The Wronskian is either identically zero everywhere (if L.D.) or nowhere
        (if L.I.) in the interval $ \mathcal{J} $ under consideration.

    \item[Fundamental matrix] The above matrix $ \vec{Y} $ is called a fundamental
        matrix if the solutions $ \{\vec{y^{(i)}}\} $ form a basis.
        \begin{align}
            \vec{y}                        & = c_1\vec{y^{(1)}} + \dots + c_2\vec{y^{(n)}}                            \\
            \vec{y}                        & = \vec{Yc}                                                               \\
            \begin{bNiceMatrix}[r, margin]
                y_1 \\ \vdots \\ y_n
            \end{bNiceMatrix} & = \begin{bNiceMatrix}[r, margin] \vec{y^{(1)}} &
                                             & \vec{y^{(n)}}\end{bNiceMatrix}
            \begin{bNiceMatrix}[r, margin]
                c_1 \\ \vdots \\ c_n
            \end{bNiceMatrix}
        \end{align}
\end{description}

\section{Constant-Coefficient Systems, Phase Plane Method}
\begin{description}
    \item[Constant coefficient system] If the matrix $ \vec{a} $ has terms
        independent of $ t $, then the initial guess
        \begin{align}
            \vec{y'} & = \vec{Ay}        & \vec{y} & = \vec{x} e^{\lambda t} \\
            \vec{Ax} & = \lambda \vec{x}
        \end{align}
        leads to an eigenvalue problem. Here, $ \lambda, \vec{x} $ are the pairs
        of eigenvalues and eigenvectors respectively.
    \item[L.I. eigenvectors condition] For the above matrix $ \vec{A} $ to have
        a set of L.I. eigenvectors (as is common in real world applications), conditions
        are one of
        \begin{itemize}
            \item $\vec{A}$ is symmetric. $ a_{jk} = a_{kj} $ or $\vec{A} = \vec{A^T} $.
            \item $\vec{A}$ is skew symmetric. $ a_{jk} = -a_{kj} $
                  or $\vec{A} = -\vec{A^T} $.
            \item $ \vec{A} $ has $ n $ distinct eigenvalues.
        \end{itemize}
    \item[General solution] If $ \vec{A} $ has a set of L.I. eigenvectors as above,
        then the corresponding solutions form a basis. \par
        A general solution is of the form,
        \begin{align}
            \vec{y} & = c_1 \vec{x^{(1)}}e^{\lambda_1 t} + \quad \cdots \quad
            + c_n \vec{x^{(n)}}e^{\lambda_n t}
        \end{align}
        From this point onwards, the limited case of two member families of
        ODEs is taken up.
\end{description}


\begin{description}
    \item[Phase plane] The use of a parameter to plot the two solutions $ y_1 $ vs.
        $y_2$ instead of the usual pair of $ y $ vs. $ t $ curves.
    \item[Critical Points] Points in the phase plane where the tangent direction is
        not defined (for example at $ (0, 0) $ in the expression below).
        \begin{align}
            \diff{y_2}{y_1} & = \frac{y_2' \dl t}{y_1' \dl t} = \frac{y_2'}{y_1'}   \\
                            & = \frac{a_{21}y_1 + a_{22}y_2}{a_{11}y_1 + a_{12}y_2}
        \end{align}
    \item[Improper Node] A critical point $ P_0 $ at which all but two trajectories
        have the same limiting direction of the tangent $ L_1 $. The two exceptional
        trajectories also have a different (and equal) limit $ L_2 $.

    \item[Proper Node] A critical point at which every trajectory has a distinct
        limiting direction, and for any given direction $ \vec{d} $, there is some
        trajectory whose limiting direction at $ P_0 $ is equal to $ \vec{d} $.
    \item[Saddle Point] Two incoming and two outgoing trajectories intersect $ P_0 $,
        while all other trajectories bypass $ P_0 $.
    \item[Center] A critical point, enclosed by many closed trajectories, none of
        which ever pass through $ P_0 $.
    \item[Spiral] A critical point which all trajectories approach as
        $ t \rightarrow \infty $ which otherwise resembles a center.
    \item[Degenerate Node] In the (almost never physical) case when no basis of
        eigenvectors can be found. The usual condition is that $ \vec{A} $ is neither
        symmetric nor skew-symmetric and happens to have degenerate eigenvalues.
        \begin{figure}[H]
            \centering
            \begin{subfigure}[b]{0.49\textwidth}
                \begin{tikzpicture}
                    \begin{axis}[
                            declare function = {
                                    u(\x) = e^(-2*\x);
                                    v(\x) = e^(-4*\x);
                                },
                            xmin = -1, xmax = 1, ymin = -1, ymax = 1,
                            % restrict y to domain = -1:1,
                            title = {Improper Node with
                                    ${\color{y_p} \vec{x^{(1)}}}$ and
                                    ${\color{y_h} \vec{x^{(2)}}}$},
                            xlabel = $ y_1 $,
                            ylabel = $ y_2 $, ylabel shift = {-1em},
                            axis equal,
                            width = 8cm,
                            legend pos = north west,
                            grid = both,
                            domain = 0:3,
                            Ani]
                        \foreach \c in {-0.5, -1, 0.5, 1} {%
                                \edef\temp{%
                                    \noexpand \addplot[ samples = 100, color=blue3,
                                        arrow inside={end=stealth,opt={scale=2}}{0.65}]
                                    ({\c*u(x) + v(x)}, {\c*u(x) - v(x)});
                                    \noexpand \addplot[ samples = 100, color=blue3,
                                        arrow inside={end=stealth,opt={scale=2}}{0.65}]
                                    ({\c*u(x) - v(x)}, {\c*u(x) + v(x)});
                                }\temp
                            }
                        \addplot[ samples = 100, color=y_p,
                            arrow inside={end=stealth,opt={scale=2}}{0.5}]
                        ({u(x)}, {u(x)});
                        \addplot[ samples = 100, color=y_h,
                            arrow inside={end=stealth,opt={scale=2}}{0.5}]
                        ({v(x)}, {-v(x)});
                        \addplot[ samples = 100, color=y_p,
                            arrow inside={end=stealth,opt={scale=2}}{0.5}]
                        ({-u(x)}, {-u(x)});
                        \addplot[ samples = 100, color=y_h,
                            arrow inside={end=stealth,opt={scale=2}}{0.5}]
                        ({-v(x)}, {v(x)});
                        \node[GraphNode, fill = white, draw = black] at (axis cs:0,0) {};
                    \end{axis}
                \end{tikzpicture}
            \end{subfigure}
            \hfill
            \begin{subfigure}[b]{0.49\textwidth}
                \begin{tikzpicture}
                    \begin{axis}[
                            declare function = {
                                    u(\x) = e^(\x);
                                    v(\x) = e^(\x);
                                },
                            xmin = -1, xmax = 1, ymin = -1, ymax = 1,
                            % restrict y to domain = -1:1,
                            title = {Proper Node with
                                    ${\color{y_p} \vec{x^{(1)}}}$ and
                                    ${\color{y_h} \vec{x^{(2)}}}$},
                            xlabel = $ y_1 $,
                            ylabel = $ y_2 $, ylabel shift = {-1em},
                            axis equal,
                            width = 8cm,
                            legend pos = north west,
                            grid = both,
                            domain = -4:0,
                            Ani]
                        \foreach \c in {-sqrt(3), -1/sqrt(3), sqrt(3), 1/sqrt(3)} {%
                                \edef\temp{%
                                    \noexpand \addplot[ samples = 100, color=blue3,
                                        arrow inside={end=stealth,opt={scale=2}}{0.35}]
                                    ({\c*u(x)}, {-v(x)});
                                    \noexpand \addplot[ samples = 100, color=blue3,
                                        arrow inside={end=stealth,opt={scale=2}}{0.35}]
                                    ({\c*u(x)}, {v(x)});
                                }\temp
                            }
                        \addplot[ samples = 100, color=y_p,
                            arrow inside={end=stealth,opt={scale=2}}{0.65}]
                        ({u(x)}, {0});
                        \addplot[ samples = 100, color=y_h,
                            arrow inside={end=stealth,opt={scale=2}}{0.65}]
                        ({0}, {-v(x)});
                        \addplot[ samples = 100, color=y_p,
                            arrow inside={end=stealth,opt={scale=2}}{0.65}]
                        ({-u(x)}, {0});
                        \addplot[ samples = 100, color=y_h,
                            arrow inside={end=stealth,opt={scale=2}}{0.65}]
                        ({0}, {v(x)});
                        \node[GraphNode, fill = white, draw = black] at (axis cs:0,0) {};
                    \end{axis}
                \end{tikzpicture}
            \end{subfigure}
        \end{figure}

        \begin{figure}[H]
            \centering
            \begin{subfigure}[b]{0.48\textwidth}
                \begin{tikzpicture}
                    \begin{axis}[
                            declare function = {
                                    u(\x) = e^(\x);
                                    v(\x) = e^(-\x);
                                },
                            xmin = -2, xmax = 2, ymin = -2, ymax = 2,
                            % restrict y to domain = -1:1,
                            title = {Saddle point with
                                    ${\color{y_p} \vec{x^{(1)}}}$ and
                                    ${\color{y_h} \vec{x^{(2)}}}$},
                            xlabel = $ y_1 $,
                            ylabel = $ y_2 $,
                            axis equal,
                            width = 8cm,
                            legend pos = north west,
                            grid = both,
                            domain = -3:3,
                            Ani]
                        \foreach \c in {1/8, 1/2, 1} {%
                                \edef\temp{%
                                    \noexpand \addplot[ samples = 100, color=blue3,
                                        arrow inside={end=stealth,opt={scale=2}}{0.5, 0.7, 0.8, 0.9}]
                                    ({\c*u(x)}, {-v(x)});
                                    \noexpand \addplot[ samples = 100, color=blue3,
                                        arrow inside={end=stealth,opt={scale=2}}{0.5, 0.7, 0.8, 0.9}]
                                    ({\c*u(x)}, {v(x)});
                                    \noexpand \addplot[ samples = 100, color=blue3,
                                        arrow inside={end=stealth,opt={scale=2}}{0.5, 0.7, 0.8, 0.9}]
                                    ({-\c*u(x)}, {-v(x)});
                                    \noexpand \addplot[ samples = 100, color=blue3,
                                        arrow inside={end=stealth,opt={scale=2}}{0.5, 0.7, 0.8, 0.9}]
                                    ({-\c*u(x)}, {v(x)});
                                }\temp
                            }
                        \addplot[ samples = 100, color=y_p,
                            arrow inside={end=stealth,opt={scale=2}}{0.05}]
                        ({u(x)}, {0});
                        \addplot[ samples = 100, color=y_h,
                            arrow inside={end=stealth,opt={scale=2}}{0.95}]
                        ({0}, {-v(x)});
                        \addplot[ samples = 100, color=y_p,
                            arrow inside={end=stealth,opt={scale=2}}{0.05}]
                        ({-u(x)}, {0});
                        \addplot[ samples = 100, color=y_h,
                            arrow inside={end=stealth,opt={scale=2}}{0.95}]
                        ({0}, {v(x)});
                        \node[GraphNode, fill = white, draw = black] at (axis cs:0,0) {};
                    \end{axis}
                \end{tikzpicture}
            \end{subfigure}
            \hfill
            \begin{subfigure}[b]{0.48\textwidth}
                \begin{tikzpicture}
                    \begin{axis}[
                            % xmin = -1, xmax = 1, ymin = -1, ymax = 1,
                            % restrict y to domain = -1:1,
                            title = {Center},
                            xlabel = $ y_1 $,
                            ylabel = $ y_2 $,
                            axis equal,
                            width = 8cm,
                            legend pos = north west,
                            grid = both,
                            domain = 0:2*pi,
                            Ani]
                        \foreach \c in {0.4,0.8,...,2.0} {%
                                \edef\temp{%
                                    \noexpand \addplot[ samples = 100, color=blue3,
                                        arrow inside={end=stealth,opt={scale=2}}{0, 0.5}]
                                    ({\c*sin(x)}, {\c*cos(x)});
                                }\temp
                            }
                        \node[GraphNode, fill = white, draw = black] at (axis cs:0,0) {};
                    \end{axis}
                \end{tikzpicture}
            \end{subfigure}
        \end{figure}

        \begin{figure}[H]
            \centering
            \begin{subfigure}[b]{0.49\textwidth}
                \begin{tikzpicture}
                    \begin{axis}[
                            xmin = -1, xmax = 1, ymin = -1, ymax = 1,
                            % restrict y to domain = -1:1,
                            title = {Spiral},
                            xlabel = $ y_1 $,
                            ylabel = $ y_2 $, ylabel shift = {-1em},
                            axis equal,
                            width = 8cm,
                            legend pos = north west,
                            grid = both,
                            domain = 0:2*pi,
                            Ani]
                        \foreach \c in {1,2,4} {%
                                \edef\temp{%
                                    \noexpand \addplot[ samples = 100, color=blue3,
                                        arrow inside={end=stealth,opt={scale=2}}{0.8}]
                                    ({\c*e^(-x)*sin(x)}, {\c*e^(-x)*cos(x)});
                                    \noexpand \addplot[ samples = 100, color=blue3,
                                        arrow inside={end=stealth,opt={scale=2}}{0.8}]
                                    ({-\c*e^(-x)*sin(x)}, {-\c*e^(-x)*cos(x)});
                                }\temp
                            }
                        \node[GraphNode, fill = white, draw = black] at (axis cs:0,0) {};
                    \end{axis}
                \end{tikzpicture}
            \end{subfigure}
            \hfill
            \begin{subfigure}[b]{0.49\textwidth}
                \begin{tikzpicture}
                    \begin{axis}[
                            xmin = -10, xmax = 10, ymin = -10, ymax = 10,
                            % restrict y to domain = -1:1,
                            title = {Degenerate Node with
                                    ${\color{y_p} \vec{x^{(1)}}}$ and
                                    ${\color{y_h} \vec{x^{(2)}}}$},
                            xlabel = $ y_1 $,
                            ylabel = $ y_2 $, ylabel shift = {-1em},
                            axis equal,
                            width = 8cm,
                            legend pos = north west,
                            grid = both,
                            domain = -1:1,
                            Ani]
                        \foreach \c in {1/2,1/4,2} {%
                                \edef\temp{%
                                    \noexpand \addplot[thin, samples = 100, color=blue3,
                                        arrow inside={end=stealth,opt={scale=2}}{0.15}]
                                    ({(\c + x)*e^(3*x)}, {(-\c - x + 1)*e^(3*x)});
                                    \noexpand \addplot[thin, samples = 100, color=blue3,
                                        arrow inside={end=stealth,opt={scale=2}}{0.15}]
                                    ({(\c - x)*e^(3*x)}, {(-\c + x - 1)*e^(3*x)});
                                    \noexpand \addplot[thin, samples = 100, color=blue3,
                                        arrow inside={end=stealth,opt={scale=2}}{0.15}]
                                    ({(-\c + x)*e^(3*x)}, {(\c - x + 1)*e^(3*x)});
                                    \noexpand \addplot[thin, samples = 100, color=blue3,
                                        arrow inside={end=stealth,opt={scale=2}}{0.15}]
                                    ({(-\c - x)*e^(3*x)}, {(\c + x - 1)*e^(3*x)});
                                }\temp
                            }
                        \addplot[thick, samples = 100, color=y_p,
                            arrow inside={end=stealth,opt={scale=2}}{0.3}]
                        ({e^(3*x)}, {-e^(3*x)});
                        \addplot[thick, samples = 100, color=y_p,
                        arrow inside={end=stealth,opt={scale=2}}{0.3}]
                        (-{e^(3*x)}, {e^(3*x)});
                        \addplot[thick, samples = 100, color=y_h,
                            arrow inside={end=stealth,opt={scale=2}}{0.3}]
                        ({x*e^(3*x)}, {(-x+1)*e^(3*x)});
                        \addplot[thick, samples = 100, color=y_h,
                            arrow inside={end=stealth,opt={scale=2}}{0.3}]
                        ({-x*e^(3*x)}, {-1*(-x+1)*e^(3*x)});
                        \node[GraphNode, fill = white, draw = black] at (axis cs:0,0) {};
                    \end{axis}
                \end{tikzpicture}
            \end{subfigure}
        \end{figure}
\end{description}

\section{Criteria for Critical Points, Stability}
\begin{description}
    \item[Characteristic equation] For the two member system of linear ODEs with
        constant coefficients,
        \begin{align}
            \vec{A}                        & = \bmattt{a_{11}}{a_{12}}{a_{21}}{a_{22}} \\
            \det(\vec{A} - \lambda\vec{I}) & =
            \lambda^{2} - (a_{11} + a_{22})\lambda + \det(\vec{A}) = 0                 \\
            \lambda^{2} - p\lambda + q     & = 0
        \end{align}
    \item[Critical point categories] Rearranging the characteristic equation into its
    factors $ \lambda_1,\ \lambda_2 $,
        \begin{align}
            p                           & = \lambda_1 + \lambda_2       &
            q                           & = \lambda_1 \lambda_2           \\
            \text{Discriminant}\ \Delta & = p^{2} - 4q                  &
            \Delta                      & = (\lambda_1 - \lambda_2)^{2}
        \end{align}
        \begin{table}[ht]
            \centering
            \SetTblrInner{rowsep=0.5em}
            \begin{tblr}{colspec={Q[r]|Q[l]}, colsep = 2em}
                \textbf{Critical Point} &
                \textbf{Eigenvalues} $ \lambda_1, \lambda_2 $            \\ \hline[dotted]
                Node                    & Real, same sign                \\
                Saddle point            & Real, opposite signs           \\
                Center                  & Purely imaginary               \\
                Spiral point            & Complex with nonzero real part \\ \hline
            \end{tblr}
        \end{table}
    \item[Stability] From physics, stability is a measure of the effect in the future
        of a small change in the system at present time.
    \item[Stable critical point] If for every disk $ D_\epsilon $ centered on $ P_0 $,
        there is a disk $ D_{\delta} $ with $ \delta, \epsilon >0 $, such that every trajectory
        which has $ P(t = t_1) = P_1 \in D_{\delta}$ has all its points corresponding to
        $ t \geq t_1 $ in $ D_{\epsilon} $.
    \item[Attractive critical point] Every trajectory in $ D_\delta $ for a stable
        critical point, approaches $ P_0 $ asymptotically as $ t \rightarrow \infty $.

        \begin{table}[ht]
            \centering
            \SetTblrInner{rowsep=0.5em}
            \begin{tblr}{colspec={Q[r]|Q[l]}, colsep = 2em}
                \textbf{Stability}    &
                \textbf{Condition on} $ \vec{p},\ \vec{q} $        \\ \hline[dotted]
                Stable and attractive & $ p<0 $ and $ q > 0 $      \\
                Stable                & $ p \leq 0 $ and $ q > 0 $ \\
                Unstable              & $ p > 0 $ or $\ \ q < 0 $  \\ \hline
            \end{tblr}
        \end{table}
\end{description}

\section{Qualitative Methods for Nonlinear Systems}
\begin{description}
    \item[Assumptions] The system of ODEs is autonomous, and the functions $ f_1, f_2 $
        are independent of $ t $. \par
        Also, the system of ODEs has finitely many critical points. This
        means that each critical point is isolated. \par
        For analysis, each critical point is treated as the origin when being analyzed,
        using the coordinate transformation,
        \begin{align}
            P_0         & = (a, b)  & (a, b)      & \rightarrow (0, 0) \\
            \tilde{y_i} & = y_1 - a & \tilde{y_2} & = y_2 - b
        \end{align}
    \item[Linearization] The system is linearlised around its critical point
        $ (0, 0) $ using,
        \begin{align}
            \vec{y} & = \vec{f}(\vec{y}) \equiv \vec{Ay} + \vec{h}(\vec{y}) \\
            y_1'    & = a_{11}y_1 + a_{12}y_2 + h_1(y_1, y_2)               \\
            y_2'    & = a_{21}y_1 + a_{22}y_2 + h_2(y_1, y_2)
        \end{align}
        Notice $ \vec{A} $ here is independent of $ y $ since $ P_0  = (0, 0)$ is
        a critical point giving. \par
        If $ f_1, f_2 $ are continuous and have continuous partial derivatives in a
        region around $ P_0 $, and if $ \det(\vec{A}) \neq 0 $, then the kind and
        stability of the critical points of the original system are the same as
        those of the linearized system. \par
        Exceptions occur if the linearized system has equal roots or purely imaginary
        roots.
        The general method is as follows,
        \begin{enumerate}
            \item Set up the mathematical model.
            \item Identify the critical points as conditions for $ \vec{y} = 0 $
            \item Drop all nonlinear terms in order to linearize the system
            \item Treat each critical point in turn, transforming coordinates as
                  needed.
        \end{enumerate}
    \item[Transformation to first order ODE] A second order autonomous ODE can
        be transformed using,
        \begin{align}
            F(y, y', y'') & = 0                    & y & = y_1, \qquad y' = y_2 \\
            y''           & = \diff{y_2}{y_1}\ y_2
        \end{align}
    \item[Limit Cycle] A closed trajectory in phase space into which other trajectories
        spiral asymptotically. Similar to an attractive node, but instead of a single point
        in phase space, this is a closed stable trajectory. \par

        In the real world, this requires systems with variable damping (which can also be
        negative), which push trajectories starting outside and inside the limit cycle
        inward and outward towards it
        respectively.
\end{description}

\section{Nonhomogeneous Linear Systems of ODEs}

\begin{description}
    \item[General solution] Assuming $ \vec{g} \not \equiv \vec{0}$ and the entries of
        $ \vec{A} $ are continuous on some interval $ \mathcal{J} $ of the $t-$axis,\par
        a general solution of the nh-system is given by,
        \begin{align}
            \vec{y'} & = \vec{Ay} + \vec{g}            \\
            \vec{y}  & = \vec{y^{(h)}} + \vec{y^{(p)}}
        \end{align}
    \item[Undetermined coefficients] The method is analogous to the single ODE
        procedure, for the same set of candidate functions. The only change is in the
        modification rule.
    \item[Variation of parameters] Start with a basis of solutions $ \vec{Z} $ of the
        h-ODE system.
        \begin{align}
            \vec{Z'}      & = A\vec{Z}                      \\
            \vec{y^{(p)}} & = \vec{Z}(t)\vec{u}(t)          \\
            \vec{y'}      & = \vec{Ay} + \vec{g}  \nonumber \\
            \vec{Zu'}     & = \vec{g}                       \\
            \vec{u'}      & = \vec{Z^{-1}g}
        \end{align}
        $ \vec{Z^{-1}} $ is guaranteed to exist since $ \vec{Z} $ is a basis and
        its Wronskian is nonzero.
\end{description}