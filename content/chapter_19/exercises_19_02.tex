\section{Solution of Equations by Iteration}

\begin{enumerate}
    \item In example $ 1 $, the slope of $ g(x) $ is positive near the intersection of
          $ y = g(x) $ and $ y = x $, which makes the iterative process monotonic.

    \item Solving the iterative relation,
          \begin{align}
              x & = 1 - x^3
          \end{align}
          \begin{figure}[H]
              \centering
              \pgfplotstableread[col sep=comma]{./tables/table_18_02_02.csv}
              \anitableeleven
              \begin{tikzpicture}
                  \begin{axis}[
                          title = {$ x_0 = 0.9 $},width = 8cm,
                          grid = both,Ani, domain = -0.5:1.5]
                      \addplot[thin, black] table[x = 0, y = 1]{\anitableeleven};
                      \addplot[GraphSmooth, y_h] {x};
                      \addplot[GraphSmooth, y_p] {1 - x^3};
                  \end{axis}
              \end{tikzpicture}
          \end{figure}
          This leads to a limit cycle since $ g'(0) = 0 $. Initial guesses with
          large enough absolute value diverge to infinity.

    \item Using the fixed point method,
          \begin{align}
              g(x) & = x & g(x) & = \frac{\cos x}{2}
          \end{align}
          \begin{figure}[H]
              \centering
              \pgfplotstableread[col sep=comma]{./tables/table_18_02_03.csv}
              \anitableeleven
              \begin{tikzpicture}
                  \begin{axis}[
                          title = {$ x_0 = 1 $},width = 8cm,
                          grid = both,Ani, domain = 0.2:1.1]
                      \addplot[thin, black] table[x = 0, y = 1]{\anitableeleven};
                      \addplot[GraphSmooth, y_h] {x};
                      \addplot[GraphSmooth, y_p] {0.5*cos(x)};
                  \end{axis}
              \end{tikzpicture}
          \end{figure}
          The result is $ x_{10} = 0.450184 $ exact to $ 6S $.

    \item Using the fixed point method,
          \begin{align}
              g(x) & = x & g(x) & = \csc x
          \end{align}
          \begin{figure}[H]
              \centering
              \pgfplotstableread[col sep=comma]{./tables/table_18_02_04.csv}
              \anitableeleven
              \begin{tikzpicture}
                  \begin{axis}[
                          title = {$ x_0 = 1 $},width = 8cm,
                          grid = both,Ani, domain = 0.9:1.2]
                      \addplot[thin, black] table[x = 0, y = 1]{\anitableeleven};
                      \addplot[GraphSmooth, y_h] {x};
                      \addplot[GraphSmooth, y_p] {1/sin(x)};
                  \end{axis}
              \end{tikzpicture}
          \end{figure}
          The result is $ x_{10} = 1.11385 $ exact to $ 6S $.

    \item Showing the function with its roots and then the iterative process,
          \begin{align}
              f(x) & = x^3 - 5x^2 + 1.01x + 1.88           &
              g(x) & = x = \frac{5x^2 - 1.01x - 1.88}{x^2}
          \end{align}
          \begin{figure}[H]
              \centering
              \begin{tikzpicture}
                  \begin{axis}[
                          title = {$ f(x) = 0 $},width = 8cm,
                          grid = both,Ani, domain = -2:6]
                      \addplot[GraphSmooth, y_s] {x^3 - 5*x^2 + 1.01*x + 1.88};
                      \node[GraphNode, inner sep = 1pt] at (axis cs:-0.5,0){};
                      \node[GraphNode, inner sep = 1pt] at (axis cs:0.8,0){};
                      \node[GraphNode, inner sep = 1pt] at (axis cs:4.7,0){};
                  \end{axis}
              \end{tikzpicture}
          \end{figure}
          \begin{figure}[H]
              \centering
              \pgfplotstableread[col sep=comma]{./tables/table_18_02_05_a.csv}
              \anitabletwelvea
              \pgfplotstableread[col sep=comma]{./tables/table_18_02_05_b.csv}
              \anitabletwelveb
              \begin{tikzpicture}
                  \begin{axis}[
                          title = {$ x_0 = 5 $},width = 8cm,
                          grid = both,Ani, domain = 4.68:5.02]
                      \addplot[thin, black] table[x = 0, y = 1]{\anitabletwelvea};
                      \addplot[GraphSmooth, y_h] {x};
                      \addplot[GraphSmooth, y_p] {(5*x^2 - 1.01*x - 1.88)/(x^2)};
                  \end{axis}
              \end{tikzpicture}
              \begin{tikzpicture}
                  \begin{axis}[
                          title = {$ x_0 = 4 $},width = 8cm,
                          grid = both,Ani, domain = 3.98:4.72]
                      \addplot[thin, black] table[x = 0, y = 1]{\anitabletwelveb};
                      \addplot[GraphSmooth, y_h] {x};
                      \addplot[GraphSmooth, y_p] {(5*x^2 - 1.01*x - 1.88)/(x^2)};
                  \end{axis}
              \end{tikzpicture}
          \end{figure}
          \begin{figure}[H]
              \centering
              \pgfplotstableread[col sep=comma]{./tables/table_18_02_05_c.csv}
              \anitabletwelvec
              \pgfplotstableread[col sep=comma]{./tables/table_18_02_05_d.csv}
              \anitabletwelved
              \begin{tikzpicture}
                  \begin{axis}[
                          title = {$ x_0 = 1 $},width = 8cm,
                          grid = both,Ani, domain = 0.95:4.75]
                      \addplot[thin, black] table[x = 0, y = 1]{\anitabletwelvec};
                      \addplot[GraphSmooth, y_h] {x};
                      \addplot[GraphSmooth, y_p] {(5*x^2 - 1.01*x - 1.88)/(x^2)};
                  \end{axis}
              \end{tikzpicture}
              \begin{tikzpicture}
                  \begin{axis}[
                          title = {$ x_0 = -1 $},width = 8cm,
                          grid = both,Ani, domain = -1.02:4.72,
                          restrict y to domain = -5:5]
                      \addplot[thin, black] table[x = 0, y = 1]{\anitabletwelved};
                      \addplot[GraphSmooth, y_h] {x};
                      \addplot[GraphSmooth, y_p, samples = 500]
                      {(5*x^2 - 1.01*x - 1.88)/(x^2)};
                  \end{axis}
              \end{tikzpicture}
          \end{figure}
          All of the guesses converge to the same root $ x_3^* = 4.7 $. This is because
          of the nature of $ g'(x) $

    \item Using the $ x^2 $ term to write the recursive relation,
          \begin{align}
              g(x)  & = \frac{x^3 + 1.01x + 1.88}{5x} &
              g'(x) & = \frac{50x^3 - 47}{125x^2}
          \end{align}
          \begin{figure}[H]
              \centering
              \begin{tikzpicture}
                  \begin{axis}[
                          title = {Convergence to $ x^*_2 = 1 $},width = 8cm,
                          grid = both,Ani, domain = 0.5:1.5, axis equal]
                      \addplot[GraphSmooth, y_h] {x};
                      \addplot[GraphSmooth, y_p] {(x^3 + 1.01*x + 1.88)/(5*x)};
                  \end{axis}
              \end{tikzpicture}
          \end{figure}
          Since $ \abs{g'(x)} \leq K < 1 $ for $ x $ near this solution, the convergence
          criterion is satisfied.

    \item Using the fixed point method,
          \begin{align}
              g(x)  & = \arcsin(e^{-x})                    &
              g'(x) & = \frac{-e^{-x}}{\sqrt{1 - e^{-2x}}}
          \end{align}
          \begin{figure}[H]
              \centering
              \pgfplotstableread[col sep=comma]{./tables/table_18_02_07_a.csv}
              \anitabletwelvea
              \pgfplotstableread[col sep=comma]{./tables/table_18_02_07_b.csv}
              \anitabletwelveb
              \begin{tikzpicture}
                  \begin{axis}[
                          title = {$ x_0 = 0.5 $},width = 8cm,
                          grid = both,Ani, domain = 0.49:0.65]
                      \addplot[thin, black] table[x = 0, y = 1]{\anitabletwelvea};
                      \addplot[GraphSmooth, y_h] {x};
                      \addplot[GraphSmooth, y_p] {asin(e^(-x))};
                  \end{axis}
              \end{tikzpicture}
              \begin{tikzpicture}
                  \begin{axis}[
                          title = {$ x_0 = 0.7 $},width = 8cm,
                          grid = both,Ani, domain = 0.5:0.71]
                      \addplot[thin, black] table[x = 0, y = 1]{\anitabletwelveb};
                      \addplot[GraphSmooth, y_h] {x};
                      \addplot[GraphSmooth, y_p] {asin(e^(-x))};
                  \end{axis}
              \end{tikzpicture}
          \end{figure}
          The result is $ x_{30} = 0.588533 $ exact to $ 6S $.

    \item Using the fixed point method,
          \begin{align}
              g(x)  & = \frac{x+0.12}{x^3}   &
              g'(x) & = -\frac{50x+3}{50x^2}
          \end{align}
          \begin{figure}[H]
              \centering
              \pgfplotstableread[col sep=comma]{./tables/table_18_02_08.csv}
              \anitablethirteen
              \begin{tikzpicture}
                  \begin{axis}[
                          title = {$ x_0 = 1 $},width = 8cm,
                          grid = both,Ani, domain = 0.99:1.05]
                      \addplot[thin, black] table[x = 0, y = 1]{\anitablethirteen};
                      \addplot[GraphSmooth, y_h] {x};
                      \addplot[GraphSmooth, y_p] {(x+0.12)^(1/4)};
                  \end{axis}
              \end{tikzpicture}
          \end{figure}
          The result is $ x_{30} = 1.03717 $ exact to $ 6S $.

    \item Using the fixed point method,
          \begin{align}
              g(x)  & = x^4 - 0.12 &
              g'(x) & = 4x^3
          \end{align}
          \begin{figure}[H]
              \centering
              \pgfplotstableread[col sep=comma]{./tables/table_18_02_09.csv}
              \anitablethirteen
              \begin{tikzpicture}
                  \begin{axis}[
                          title = {$ x_0 = -0.2 $},width = 8cm,
                          grid = both,Ani, domain = -0.22:-0.1]
                      \addplot[thin, black] table[x = 0, y = 1]{\anitablethirteen};
                      \addplot[GraphSmooth, y_h] {x};
                      \addplot[GraphSmooth, y_p] {x^4 - 0.12};
                  \end{axis}
              \end{tikzpicture}
          \end{figure}
          The result is $ x_{20} = -0.119794 $ exact to $ 6S $.


    \item Using the fixed point method,
          \begin{align}
              g(x)  & = \frac{1}{\cosh x} &
              g'(x) & = \arctan(\sinh x)
          \end{align}
          \begin{figure}[H]
              \centering
              \pgfplotstableread[col sep=comma]{./tables/table_18_02_10_a.csv}
              \anitabletwelvea
              \pgfplotstableread[col sep=comma]{./tables/table_18_02_10_b.csv}
              \anitabletwelveb
              \begin{tikzpicture}
                  \begin{axis}[
                          title = {$ x_0 = 0.6 $},width = 8cm,
                          grid = both,Ani, domain = 0.58:0.85]
                      \addplot[thin, black] table[x = 0, y = 1]{\anitabletwelvea};
                      \addplot[GraphSmooth, y_h] {x};
                      \addplot[GraphSmooth, y_p] {1/cosh(x)};
                  \end{axis}
              \end{tikzpicture}
              \begin{tikzpicture}
                  \begin{axis}[
                          title = {$ x_0 = 0.85 $},width = 8cm,
                          grid = both,Ani, domain = 0.7:0.87]
                      \addplot[thin, black] table[x = 0, y = 1]{\anitabletwelveb};
                      \addplot[GraphSmooth, y_h] {x};
                      \addplot[GraphSmooth, y_p] {1/cosh(x)};
                  \end{axis}
              \end{tikzpicture}
          \end{figure}
          The result is $ x_{10} = 0.765010 $ exact to $ 6S $.

    \item Using the fixed point method,
          \begin{align}
              g(x) & = \frac{1 + x^4/64 - x^6/2304}{x/4}
          \end{align}
          \begin{figure}[H]
              \centering
              \pgfplotstableread[col sep=comma]{./tables/table_18_02_11.csv}
              \anitabletwelvea
              \begin{tikzpicture}
                  \begin{axis}[
                          title = {Maclaurin series of $ J_0 $},width = 8cm,
                          grid = both,Ani, domain = 0:3]
                      \addplot[GraphSmooth, y_s] {1 - (x^2/4) + (x^4/64) - (x^6/2304)};
                      \node[GraphNode, inner sep = 1pt] at (axis cs:2.39,0){};
                  \end{axis}
              \end{tikzpicture}
              \begin{tikzpicture}
                  \begin{axis}[
                          title = {$ x_0 = 2.3 $},width = 8cm,
                          grid = both,Ani, domain = 2.28:2.45]
                      \addplot[thin, black] table[x = 0, y = 1]{\anitabletwelvea};
                      \addplot[GraphSmooth, y_h] {x};
                      \addplot[GraphSmooth, y_p] {(1 + (x^4/64) - (x^6/2304))/(0.25*x)};
                  \end{axis}
              \end{tikzpicture}
          \end{figure}
          The result is $ x_{10} = 2.39165 $ exact to $ 6S $.

    \item Using the fixed point method,
          \begin{align}
              f(x)   & = x^3 + 2x^2 - 3x - 4                   &
              g_1(x) & = (-3x^2 + 3x + 4)^{1/3}                  \\
              g_2(x) & = (-0.5x^3 + 1.5x + 2)^{1/2}            &
              g_3(x) & = \frac{x^3 + 2x^2 - 4}{3}                \\
              g_4(x) & = \frac{-2x^2 + 3x + 4}{x^2}            &
              g_5(x) & = \frac{-x^3 + 3x + 4}{2x}                \\
              g_6(x) & = \frac{2x^3 + 2x^2 + 4}{3x^2 + 4x - 3}
          \end{align}
          \begin{figure}[H]
              \centering
              \pgfplotstableread[col sep=comma]{./tables/table_18_02_12_a.csv}
              \anitabletwelvea
              \pgfplotstableread[col sep=comma]{./tables/table_18_02_12_b.csv}
              \anitabletwelveb
              \begin{tikzpicture}
                  \begin{axis}[
                          title = {$ x_1^* = 1.56155, \qquad n = 15 $}
                          ,width = 8cm,
                          grid = both,Ani, domain = 1.45:1.6]
                      \addplot[thin, black] table[x = 0, y = 1]{\anitabletwelvea};
                      \addplot[GraphSmooth, y_h] {x};
                      \addplot[GraphSmooth, y_p] {(-2*x^2 + 3*x + 4)^(1/3)};
                  \end{axis}
              \end{tikzpicture}
              \begin{tikzpicture}
                  \begin{axis}[
                          title = {$ x_2^* = 1.56155, \qquad n = 35 $}
                          ,width = 8cm,
                          grid = both,Ani, domain = 1.48:1.62]
                      \addplot[thin, black] table[x = 0, y = 1]{\anitabletwelveb};
                      \addplot[GraphSmooth, y_h] {x};
                      \addplot[GraphSmooth, y_p] {(-0.5*x^3 + 1.5*x + 2)^(1/2)};
                  \end{axis}
              \end{tikzpicture}
          \end{figure}
          \begin{figure}[H]
              \centering
              \pgfplotstableread[col sep=comma]{./tables/table_18_02_12_c.csv}
              \anitabletwelvec
              \pgfplotstableread[col sep=comma]{./tables/table_18_02_12_d.csv}
              \anitabletwelved
              \begin{tikzpicture}
                  \begin{axis}[
                          title = {$ x_3^* = -1, \qquad n = 16 $}
                          ,width = 8cm,
                          grid = both,Ani, domain = -1.2:1.8]
                      \addplot[thin, black] table[x = 0, y = 1]{\anitabletwelvec};
                      \addplot[GraphSmooth, y_h] {x};
                      \addplot[GraphSmooth, y_p] {(x^3 + 2*x^2 - 4)/3};
                  \end{axis}
              \end{tikzpicture}
              \begin{tikzpicture}
                  \begin{axis}[
                          title = {$ x_4^* = -2.56155, \qquad n = 12 $}
                          ,width = 8cm,
                          grid = both,Ani, domain = -3:6,
                          restrict y to domain = -10:10]
                      \addplot[thin, black] table[x = 0, y = 1]{\anitabletwelved};
                      \addplot[GraphSmooth, y_h] {x};
                      \addplot[GraphSmooth, y_p] {(-2*x^2 + 3*x + 4)/(x^2)};
                  \end{axis}
              \end{tikzpicture}
          \end{figure}
          \begin{figure}[H]
              \centering
              \pgfplotstableread[col sep=comma]{./tables/table_18_02_12_e.csv}
              \anitabletwelvee
              \pgfplotstableread[col sep=comma]{./tables/table_18_02_12_f.csv}
              \anitabletwelvef
              \begin{tikzpicture}
                  \begin{axis}[
                          title = {$ x_5^* = -1 \quad$
                                  almost stuck in limit cycle}
                          ,width = 8cm,
                          grid = both,Ani, domain = -1.05:-0.95,
                          %   restrict y to domain = -1.1:-0.9
                      ]
                      \addplot[thin, black] table[x = 0, y = 1]{\anitabletwelvee};
                      \addplot[GraphSmooth, y_h] {x};
                      \addplot[GraphSmooth, y_p] {(-x^3 + 3*x + 4)/(2*x)};
                  \end{axis}
              \end{tikzpicture}
              \begin{tikzpicture}
                  \begin{axis}[
                          title = {$ x_6^* = 1.56155, \qquad n = 4 $}
                          ,width = 8cm,
                          grid = both,Ani, domain = 1.49:1.58]
                      \addplot[thin, black] table[x = 0, y = 1]{\anitabletwelvef};
                      \addplot[GraphSmooth, y_h] {x};
                      \addplot[GraphSmooth, y_p] {(2*x^3 + 2*x^2 + 4)/
                          (3*x^2 + 4*x - 3)};
                  \end{axis}
              \end{tikzpicture}
          \end{figure}

    \item Range of a continuous function $ g $ lies within its domain. This means that
          the range $ R $ is a proper subset of the domain $ D $.
          \begin{align}
              f(x) & = g(x) - x            \\
              g(a) & < a        & g(b) > b
          \end{align}
          for some $ a, b $ being the ends of the domain $ D $
          \begin{align}
              f(a) & < 0 & f(b) & > 0
          \end{align}
          By the intermediate value theorem of calculus, $ f(x) $ being continuous
          means there exists at least one value $ c \in [a,b] $ for which $ f(x) = 0 $.
          \begin{figure}[H]
              \centering
              \begin{tikzpicture}
                  \begin{axis}[
                          title = {$ R \in[0,0.879] $ and $ D \in [-1,1] $},
                          width = 8cm,
                          legend pos = outer north east,
                          grid = both,Ani, domain = -1:1]
                      \addplot[GraphSmooth, y_h] {x};
                      \addplot[GraphSmooth, y_p] {-(x-1)*(x+1)*(2*x+0.5)*(2*x-0.5)};
                      \addlegendentry{$ x $}
                      \addlegendentry{$ P_4(x) $}
                  \end{axis}
              \end{tikzpicture}
          \end{figure}
          This is an example of a polynomial with roots $ \pm 1, \pm 0.5 $, which
          intersects the line $ y=x $ multiple times within its domain.

    \item Using the Newton Raphson method,
          \begin{align}
              f(x)    & = x^3 - 7                        &
              x_{n+1} & = x_n - \frac{x_n^3 - 7}{3x_n^2}   \\
              x_0     & = 2                              &
              x^*     & = 1.91293,\qquad (n=3)
          \end{align}
          \begin{figure}[H]
              \centering
              \begin{tikzpicture}
                  \begin{axis}[ width = 8cm,grid = both,Ani,
                          domain = -3:3]
                      \addplot[GraphSmooth, y_p] {x^3 - 7};
                      \node[GraphNode, inner sep = 1pt] at (axis cs:1.913,0){};
                  \end{axis}
              \end{tikzpicture}
          \end{figure}

    \item Using the Newton Raphson method,
          \begin{align}
              f(x)    & = 2x - \cos x                                  &
              x_{n+1} & = x_n - \frac{2x_n - \cos(x_n)}{2 + \sin(x_n)}   \\
              x_0     & = 1                                            &
              x^*     & = 0.450184,\qquad (n=3)
          \end{align}
          \begin{figure}[H]
              \centering
              \begin{tikzpicture}
                  \begin{axis}[ width = 8cm,grid = both,Ani,
                          domain = -pi:pi]
                      \addplot[GraphSmooth, y_p] {2*x - cos(x)};
                      \node[GraphNode, inner sep = 1pt] at (axis cs:0.45,0){};
                  \end{axis}
              \end{tikzpicture}
          \end{figure}
          Convergence is much faster compared to the fixed point iteration method in
          problem $ 3 $.

    \item Using the Newton Raphson method,
          \begin{align}
              f(x)    & = 2x - \cos x                                  &
              x_{n+1} & = x_n - \frac{2x_n - \cos(x_n)}{2 + \sin(x_n)}   \\
              x_0     & = 1                                            &
              x^*     & = 0.450184,\qquad (n=3)
          \end{align}
          \begin{figure}[H]
              \centering
              \begin{tikzpicture}
                  \begin{axis}[ width = 8cm,grid = both,Ani,
                          domain = -pi:pi]
                      \addplot[GraphSmooth, y_p] {2*x - cos(x)};
                      \node[GraphNode, inner sep = 1pt] at (axis cs:0.45,0){};
                  \end{axis}
              \end{tikzpicture}
          \end{figure}

          Other choices of initial guess also converge to the same root. This convergence
          is slightly slower for far away $ x_0 $.

    \item Solving problem $ 5 $, using Newton Raphson method,
          \begin{align}
              f(x)    & = x^3 - 5x^2 + 1.01x + 1.88                    \\
              x_{n+1} & = x_n - \frac{x_n^3 - 5x_n^2 + 1.01x_n + 1.88}
              {3x_n^2 - 10x_n + 1.01}
          \end{align}
          \begin{figure}[H]
              \centering
              \begin{tikzpicture}
                  \begin{axis}[
                          title = {$ f(x) = 0 $},width = 8cm,
                          grid = both,Ani, domain = -3.5:6]
                      \addplot[GraphSmooth, y_s] {x^3 - 5*x^2 + 1.01*x + 1.88};
                      \node[GraphNode, inner sep = 1pt] at (axis cs:-0.5,0){};
                      \node[GraphNode, inner sep = 1pt] at (axis cs:0.8,0){};
                      \node[GraphNode, inner sep = 1pt] at (axis cs:4.7,0){};
                  \end{axis}
              \end{tikzpicture}
          \end{figure}
          The starting points $ x_0 = 5,4,1,-3 $ converge to $ 4.7, 4.7, 0.8, -0.5 $
          respectively. \par

    \item Using Newton Raphson method,
          \begin{align}
              x_{n+1} & = x_n - \frac{63x_n^5  - 70x_n^3 + 15x_n}
              {315x_n^4 - 210x_n^2 + 15}                            \\
              x_0     & = 1                                       &
              x^*     & = 0.906180,\qquad (n=4)
          \end{align}
          \begin{figure}[H]
              \centering
              \begin{tikzpicture}
                  \begin{axis}[
                          title = {$ P_5(x) = 0 $},width = 8cm,
                          grid = both,Ani, domain = -1:1]
                      \addplot[GraphSmooth, y_s] {(63*x^5 - 70*x^3 + 15*x)/8};
                      \node[GraphNode, inner sep = 1pt] at (axis cs:0.9062,0){};
                  \end{axis}
              \end{tikzpicture}
          \end{figure}
          Factorizing the polynomial, to yield a bi-quadratic equation
          \begin{align}
              P_5(x) & = \frac{x}{8} \cdot (63x^4 - 70x^2 + 15)              &
              x^2    & = 0, \frac{35 \pm 2\sqrt{70}}{63}                       \\
              x      & = \sqrt{\frac{35 + 2\sqrt{70}}{63}} = \sqrt{0.821162} &
              x^*    & = 0.906180
          \end{align}
          This matches the N.R. method to $ 6S $.

    \item Using Newton Raphson method,
          \begin{align}
              x_{n+1} & = x_n - \frac{-7x^4 + 8x^2 - 1}
              {-28x^3 + 16x}                              \\
              x_0     & = 0.3                           &
              x^*     & = 0.377964,\qquad (n=3)
          \end{align}
          \begin{figure}[H]
              \centering
              \begin{tikzpicture}
                  \begin{axis}[
                          title = {$ P_4^2(x) = 0 $},width = 8cm,
                          grid = both,Ani, domain = -1:1]
                      \addplot[GraphSmooth, y_s] {7.5(-7*x^4 + 8*x^2 - 1)};
                      \node[GraphNode, inner sep = 1pt] at (axis cs:0.378,0){};
                  \end{axis}
              \end{tikzpicture}
          \end{figure}
          Factorizing the polynomial, to yield a bi-quadratic equation
          \begin{align}
              P_4^2(x) & = \frac{15}{2} \cdot (-7x^4 + 8x^2 - 1) &
              x^2      & = \frac{1}{7}, 1                          \\
              x        & = \sqrt{1/7} = \sqrt{0.142857}          &
              x^*      & = 0.377964
          \end{align}
          This matches the N.R. method to $ 6S $.

    \item Using Newton Raphson method,
          \begin{align}
              x_{n+1} & = x_n - \frac{x + \ln x - 2}
              {1 + 1/x}                                \\
              x_0     & = 2                          &
              x^*     & = 1.55715,\qquad (n=3)
          \end{align}
          \begin{figure}[H]
              \centering
              \begin{tikzpicture}
                  \begin{axis}[width = 8cm,
                          grid = both,Ani, domain = 0.1:5]
                      \addplot[GraphSmooth, y_s] {x + ln(x) - 2};
                      \node[GraphNode, inner sep = 1pt] at (axis cs:1.5571,0){};
                  \end{axis}
              \end{tikzpicture}
          \end{figure}

    \item Using Newton Raphson method,
          \begin{align}
              x_{n+1} & = x_n - \frac{x^3 - 5x + 3}
              {3x^2 - 5}                              \\
              x_0     & = 2                         &
              x^*     & = 1.83424,\qquad (n=3)        \\
              x_0     & = 0                         &
              x^*     & = 0.656619,\qquad (n=3)       \\
              x_0     & = -2                        &
              x^*     & = -2.49086,\qquad (n=4)       \\
          \end{align}
          \begin{figure}[H]
              \centering
              \begin{tikzpicture}
                  \begin{axis}[width = 8cm,
                          grid = both,Ani, domain = -3:3]
                      \addplot[GraphSmooth, y_s] {x^3 - 5*x + 3};
                      \node[GraphNode, inner sep = 1pt] at (axis cs:-2.5,0){};
                      \node[GraphNode, inner sep = 1pt] at (axis cs:0.66,0){};
                      \node[GraphNode, inner sep = 1pt] at (axis cs:1.834,0){};
                  \end{axis}
              \end{tikzpicture}
          \end{figure}

    \item Using the transformation $ e^{-0.01x} = w $
          \begin{align}
              f_1     & - 100(1 - w^{20})                                      &
              f_2     & = 40w                                                    \\
              w_{n+1} & = w_n - \frac{-100 + 100w^{20} + 40w}{2000w^{19} + 40}   \\
              w_0     & = 2                                                    &
              w^*     & = 0.9756                                                 \\
              x^*     & = -100\ln(w^*) = 2.47                                  &
              T^*     & = 39.02
          \end{align}
          \begin{figure}[H]
              \centering
              \begin{tikzpicture}
                  \begin{axis}[width = 8cm, xlabel = $ w $, ylabel = $ g(w) $,
                          grid = both,Ani, domain = 0.8:1.1]
                      \addplot[GraphSmooth, y_s] {100*(x^(20) - 1) + 40*x};
                      \node[GraphNode, inner sep = 1pt] at (axis cs:0.9756,0){};
                  \end{axis}
              \end{tikzpicture}
          \end{figure}

    \item Using Newton Raphson method,
          \begin{align}
              x_{n+1} & = x_n - \frac{\cos x \cdot \cosh x - 1}
              {-\sin x \cosh x + \cos x \sinh x}                  \\
              x_0     & = 4.5                                   &
              x^*     & = 4.73004,\quad (n=4)
          \end{align}
          \begin{figure}[H]
              \centering
              \begin{tikzpicture}
                  \begin{axis}[width = 8cm,
                          grid = both,Ani, domain = -0.5:5.5]
                      \addplot[GraphSmooth, y_s] {cos(x) * cosh(x) - 1};
                      \node[GraphNode, inner sep = 1pt] at (axis cs:4.73,0){};
                  \end{axis}
              \end{tikzpicture}
          \end{figure}

          \begin{enumerate}
              \item Finding the point of intersection with the $ x $ axis,
                    \begin{align}
                        \frac{y - f(b)}{x - b} & = \frac{f(b) - f(a)}{b - a} \\
                        y = 0                  & \implies c - b =
                        \frac{-f(b) \cdot (b-a)}{f(b) - f(a)}                \\
                        c                      & = \frac{a\ f(b) - b\ f(a)}
                        {f(b) - f(a)}
                    \end{align}

              \item Solving using this iterative process, using
                    $ a_0 = 1, b_0 = 2 $
                    \begin{align}
                        f_1(x) & = x^4 - 2           & x^* & = 1.18921, \qquad (n=25) \\
                        f_2(x) & = \cos x - \sqrt{x} & x^* & = 0.641714, \qquad (n=8) \\
                        f_3(x) & = x + \ln x - 2     & x^* & = 1.57715, \qquad (n=6)
                    \end{align}
          \end{enumerate}

    \item Bisection method,
          \begin{enumerate}
              \item Algorithm in \texttt{sympy}
              \item Solving using this iterative process, using
                    \begin{align}
                        f_1(x) & = x - \cos x              &
                        a = 0,\quad b = 2                    \\
                        x^*    & = 0.739085, \qquad (n=21)
                    \end{align}
                    The N.R. method converges to the same $ 6S $ result in
                    $ 5 $ iterations starting with $ x_0 = 0.5 $, which is much faster.

              \item Using the bisection method
                    \begin{align}
                        f_1(x) & = e^{-x} - \ln x           &
                        x^*    & = 1.309800, \quad (n = 21)   \\
                        f_2(x) & = e^{x} + x^4 + x - 2      &
                        x^*    & = 0.429494, \quad (n = 21)
                    \end{align}
          \end{enumerate}

    \item Using the secant method,
          \begin{align}
              f(x) & = e^{-x} - \tan x    &
              x_0  & = 1, \quad x_1 = 0.7   \\
              x^*  & = 0.531391           &
              n    & = 5
          \end{align}

    \item Using the secant method,
          \begin{align}
              f(x) & = x^3 - 5x + 3     &
              x_0  & = 1, \quad x_1 = 2   \\
              x^*  & = 1.83424          &
              n    & = 5
          \end{align}

    \item Using the secant method,
          \begin{align}
              f(x) & = x - \cos x         &
              x_0  & = 0.5, \quad x_1 = 1   \\
              x^*  & = 0.739085           &
              n    & = 5
          \end{align}

    \item Using the secant method,
          \begin{align}
              f(x) & = \sin x - \cot x    &
              x_0  & = 1, \quad x_1 = 0.5   \\
              x^*  & = 0.904557           &
              n    & = 6
          \end{align}

    \item Refer notes. TBC.
          Methods involving more complex computations per step and more stringent
          assumptions on the function converge to the same result faster than crude
          methods.
\end{enumerate}