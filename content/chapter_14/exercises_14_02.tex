\section{Cauchy's Integral Theorem}

\begin{enumerate}
    \item Verifying, using path deformation to make a unit circle,
          \begin{align}
              z(t)       & = e^{\i t}                                         &
              t          & \in [0, 2\pi]                                        \\
              f[z(t)]    & = e^{2\i t}                                        &
              \dot{z}(t) & = \i\ e^{\i t}                                       \\
              I          & = \int_{0}^{2\pi} e^{2\i t}\ (\i\ e^{\i t})\ \dl t &
                         & = \Bigg[ \frac{e^{3\i t}}{3} \Bigg]_0^{2\pi}         \\
              I          & = 0
          \end{align}
          \begin{figure}[H]
              \centering
              \begin{tikzpicture}
                  \begin{axis}[legend pos = outer north east,
                          title = {Path is deformed to unit circle},
                          height = 8cm, width = 8cm, axis equal,
                          %   xmin = -1, xmax = 6,
                          grid = both,Ani,
                          colormap/jet,
                      ]
                      \addplot[GraphSmooth, mesh, point meta = \t, domain = 0:2*pi,
                          variable = \t]
                      ({cos(\t)}, {-sin(\t)});
                      \node[GraphNode, label =
                              {[fill = white]below left:\scriptsize $ t = 0 $}]
                      at (axis cs:1, 0){};
                      \node[GraphNode, label =
                              {[fill = white]above left:\scriptsize $ t = 2\pi $}]
                      at (axis cs:1, 0){};
                      \draw[dashed, black] (1,-1) -- (1,1) -- (-1,1) -- (-1,-1) -- cycle;
                  \end{axis}
              \end{tikzpicture}
          \end{figure}

    \item Checking the points where the functions are not analytic,
          \begin{enumerate}
              \item $ f(z) = 1/z $ is not analytic at $ z = 0 $. So any domain that does
                    not contain the origin is usable.
              \item The function is,
                    \begin{align}
                        g(z) & = \frac{\exp(1/z^2)}{z^2 + 16}
                    \end{align}
                    This function is not analytic at $ z = 0, 4\i, -4\i $, so any domain
                    excluding these points is usable.
          \end{enumerate}

    \item \textcolor{y_h}{Yes}. From example $ 6 $, deformation of path does not change
          the value of the integral, since the square and the unit circle enclose points
          where the function is analytic. \par
          The only point where $ f(z) = z^{-2} $ is not analytic is at the origin.

    \item \textcolor{y_p}{No}, since this violates the theorem of path deformation.
          The values of both integrals being different makes it so that the function is
          not analytic everywhere in between the two paths.

    \item For the given function,
          \begin{align}
              f(z) & = \frac{\cos(z^2)}{z^4 + 1} &
              S    & = \{\pm 1, \pm \i\}
          \end{align}
          The points in set $ S $ are points where $ f(z) $ is not analytic. So, the
          domain of analyticity is $ 5 $ connected.

    \item Integrating over the first path,
          \begin{align}
              z(t)       & = (1 + \i)\ t                               &
              t          & \in [0, 1]                                    \\
              \dot{z}(t) & = (1 + \i)                                  &
              f[z(t)]    & = \exp[(1+\i)t]                               \\
              I_1        & = \int_{0}^{1} (1 + \i)\ e^{(1+\i)t}\ \dl t &
                         & = \color{y_h} e^{1+\i} - e^{0}
          \end{align}
          Integrating over the second path,
          \begin{align}
              I_2 & = \int_{0}^{1} (1)\ e^{t}\ \dl t         &
                  & = \color{y_p} e^{1} - e^{0}                \\
              I_3 & = \int_{0}^{1} (\i)\ e^{1 + \i t}\ \dl t &
                  & = \color{y_p} e^{1+\i} - e^{1}             \\
          \end{align}
          Since $ I_2 + I_3 = I_1 $, path independence is verified.

    \item The distance between $ z_1 = 0 + 2\i $ and $ z_2 = 2 + 0\i $ is equal to
          $ 2\sqrt{2} $. This number is larger than 2 but less than 3. \par
          So \textcolor{y_h}{Yes} and \textcolor{y_p}{No}.

    \item Cauchy's integral theorem,
          \begin{enumerate}
              \item Refer notes for facts. Examples TBC.
              \item Resolving into partial fractions,
                    \begin{align}
                        f(z)    & = \frac{8z + 12\ \i}{4z^2 + 1}     &
                                & = \frac{a + b\ i}{2z + \i}
                        + \frac{c + d\ \i}{2z - \i}                    \\
                        4 + 0\i & = (a + c) + (b + d)\ \i            &
                        12\ \i  & = (b - d) + \i\ (c - a)              \\
                        f(z)    & = \color{y_h} \frac{-2}{z + 0.5\i}
                        + \frac{4}{z - 0.5\i}
                    \end{align}
                    Since the unit circle encloses both the points, $ (0, -0.5) $ and
                    $ (0, 0.5) $, the result is,
                    \begin{align}
                        I & = I_1 + I_2 = -2 (2\pi\ \i) + 4(2\pi\ \i) = 4\pi\ \i
                    \end{align}
                    Resolving into partial fractions,
                    \begin{align}
                        g(z)    & = \frac{z + 1}{z^2 + 2z}        &
                                & = \frac{a + b\ i}{z + 2}
                        + \frac{c + d\ \i}{z}                       \\
                        1 + 0\i & = (a + c) + (b + d)\ \i         &
                        1 + 0\i & = (2c) + \i\ (2d)                 \\
                        g(z)    & = \color{y_p} \frac{0.5}{z + 2}
                        + \frac{0.5}{z}
                    \end{align}
                    Since the unit circle encloses both the points, $ (0, 0) $ but not
                    $ (-2, 0) $, the result is,
                    \begin{align}
                        I & = 0 + I_2 = 0 + 0.5(2\pi\ \i) = \pi\ \i
                    \end{align}
              \item From the earlier team project, the deformation of paths led
                    to no change in the integral only when the function was analytic over
                    the region of path deformation, in agreement with the theorem. \par
                    For the new path, an analytic function yields,
                    \begin{align}
                        z(t)       & = t + \i\ at(1-t)                               &
                        \dot{z}(t) & = 1 + \i\ a(1 - 2t)                               \\
                        f(z)       & = z                                               \\
                        I_1        & = \int_{0}^{1} t - a^2(t - 3t^2 + 2t^3)\ \dl t  &
                                   & = \Bigg[ \frac{t^2}{2} - \frac{a^2(6t^2 - 12t^3
                        + 6t^4)}{12}  \Bigg]_0^1                                       \\
                                   & = \color{y_h} \frac{1}{2}                         \\
                        I_2        & = \int_{0}^{1} a(2t - 3t^2)\ \dl t              &
                                   & = \Big[ a (t^2 - t^3) \Big]_0^1
                        = \color{y_h} 0
                    \end{align}
                    Whereas a non-analytic function yields,
                    \begin{align}
                        z(t)       & = t + \i\ at(1-t)                           &
                        \dot{z}(t) & = 1 + \i\ a(1 - 2t)                           \\
                        f(z)       & = \Re z = t                                   \\
                        I_1        & = \int_{0}^{1} t + a\i\ (t - 2t^2)\ \dl t   &
                                   & = \Bigg[ \frac{t^2}{2} + a\i\ \frac{(3t^2
                        - 4t^3)}{6}  \Bigg]_0^1                                    \\
                                   & = \color{y_p} \frac{1}{2} - \frac{a}{6}\ \i
                    \end{align}
                    Notice the dependence on $ a $ for the non-analytic function.
          \end{enumerate}

    \item Cauchy's theorem \textcolor{y_h}{is applicable}, since the function is
          entire.
          \begin{align}
              I & = \oint_{C} \exp(-z^2)\ \dl z = 0
          \end{align}

    \item Cauchy's theorem \textcolor{y_h}{is applicable}, since the function only has
          poles outside the unit circle.
          \begin{align}
              \tan(z/4)        & = 0                               &
              \implies \quad z & = 4(n\pi + \pi/2)                   \\
              S                & = \{2\pi + 4n\pi \}               &
                               & = \{-2\pi, 2\pi, 6\pi,\dots\}       \\
              I                & = \oint_{C} \exp(-z^2)\ \dl z = 0
          \end{align}

    \item Cauchy's theorem \textcolor{y_p}{is not applicable}.
          \begin{align}
              2z - 1           & = 0                                    &
              \implies \quad z & = 0.5 + 0\i                              \\
              I                & = \oint_{C} \frac{0.5}{z - 0.5}\ \dl z
                               & = \i\ \pi
          \end{align}

    \item Cauchy's theorem \textcolor{y_p}{is not applicable}. Using the standard result
          for the integral of $ z^n $ around the unit circle,
          \begin{align}
              \bar{z}^3        & = 0                                      &
              \implies \quad z & = 0 + 0\i                                  \\
              I                & = \oint_{C} \frac{\abs{z}^6}{z^3}\ \dl z
                               & = 0
          \end{align}

    \item Cauchy's theorem \textcolor{y_h}{is applicable}, since $ (1.1)^{1/4} $ has a
          modulus greater than one.
          \begin{align}
              \bar{z}^4        & = 1.1                                   &
              \implies \quad z & = (1.1)^{1/4}\ e^{\i k\pi/2}              \\
              I                & = \oint_{C} \frac{1}{z^4 - 1.1} \ \dl z
                               & = 0
          \end{align}

    \item Cauchy's theorem \textcolor{y_p}{is not applicable}. Using the standard result
          for the integral of $ z^n $ around the unit circle,
          \begin{align}
              \bar{z}          & = 0                                     &
              \implies \quad z & = 0 + 0\ \i                               \\
              I                & = \oint_{C} \frac{z}{\abs{z}^2} \ \dl z
                               & = 0
          \end{align}

    \item Cauchy's theorem \textcolor{y_p}{is not applicable}, since the function is not
          analytic anywhere.
          \begin{align}
              z(t) & = \cos t + \i\ \sin t                                     &
              t    & \in [0, 2\pi]                                               \\
              I    & = \oint_{C} f(z) \ \dl z                                  &
                   & = \int_{0}^{2\pi} (\sin t)\ (-\sin t + \i\ \cos t)\ \dl t   \\
                   & = \int_{0}^{2\pi} \frac{[\cos(2t) - 1]}{2}\ \dl t         &
                   & = -\pi
          \end{align}

    \item Cauchy's theorem \textcolor{y_p}{is not applicable}. Using the standard result
          for the integral of $ z^n $ around the unit circle,
          \begin{align}
              \pi z - 1        & = 0                                             &
              \implies \quad z & = \frac{1}{\pi} + 0\ \i                           \\
              I                & = \oint_{C} \frac{(1/\pi)}{z - (1/\pi)} \ \dl z
                               & = 2\i
          \end{align}

    \item Cauchy's theorem \textcolor{y_p}{is not applicable}. Using the standard result
          for the integral of $ z^n $ around the unit circle,
          \begin{align}
              \abs{z}^2        & = 0                             &
              \implies \quad z & = 0 + 0\ \i                       \\
              z(t)             & = \cos t + \i\ \sin t           &
              t                & \in [0, 2\pi]                     \\
              f[z(t)]          & = 1                             &
              I                & = \oint_{C} \i\ e^{\i t}\ \dl t
              = 0
          \end{align}

    \item Cauchy's theorem \textcolor{y_p}{is not applicable}. Using the standard result
          for the integral of $ z^n $ around the unit circle,
          \begin{align}
              4z - 3           & = 0                                       &
              \implies \quad z & = \frac{3}{4} + 0\ \i                       \\
              I                & = \oint_{C} \frac{0.25}{z - 0.75} \ \dl z
                               & = \frac{\pi}{2}\ \i
          \end{align}

    \item Cauchy's theorem \textcolor{y_h}{is applicable}.
          \begin{align}
              \cot(z)          & = 0                             &
              \implies \quad z & = n\pi                            \\
              I                & = \oint_{C} z^3 \cot(z) \ \dl z
                               & = 0
          \end{align}

    \item Cauchy's theorem \textcolor{y_h}{is applicable}.
          \begin{align}
              \Ln(1-z)         & = 0                          &
              \implies \quad z & = 1 + 0\ \i                    \\
              I                & = \oint_{C} \Ln(1-z) \ \dl z
                               & = 0
          \end{align}
          \begin{figure}[H]
              \centering
              \begin{tikzpicture}
                  \begin{axis}[legend pos = outer north east,
                          height = 8cm, width = 8cm, axis equal,
                          xmin = -1.2, xmax = 1.2,
                          ymin = -1.2, ymax = 1.2,
                          grid = both,Ani,
                          colormap/jet,
                      ]
                      \node[GraphNode, fill = y_p] at (axis cs:1, 0){};
                      \draw[y_h]
                      (0,1) -- (1,1) -- (0,-1) -- (-1,-1) -- cycle;
                  \end{axis}
              \end{tikzpicture}
          \end{figure}

    \item Cauchy's theorem \textcolor{y_p}{is not applicable}. Using the standard result
          for the integral of $ z^n $ around the unit circle,
          \begin{align}
              z - 3\i          & = 0                                   &
              \implies \quad z & = 0 + 3\ \i                             \\
              z(t)             & = \pi\ e^{\i t}                       &
              t                & \in [0, 2\pi]                           \\
              I                & = \oint_{C} \frac{1}{z - 3\i} \ \dl z
                               & = 2\pi\ \i
          \end{align}

    \item Cauchy's theorem \textcolor{y_p}{is not applicable}. Using the standard result
          for the integral of $ z^n $ around the unit circle,
          \begin{align}
              f[z(t)]    & = \cos t                                             &
              z(t)       & = e^{\i t} \qquad t \in [0, \pi]                       \\
              \dot{z}(t) & = \i e^{\i t}                                          \\
              I_1        & = \int_{0}^{\pi} (\i \cos t - \sin t)\ \cos t\ \dl t
                         & = \color{y_p} \frac{\pi}{2}\ \i
          \end{align}
          For the straight line path,
          \begin{align}
              f[z(t)] & = t                                           &
              t       & \in [-1, 1]                                     \\
              z(t)    & = t + 0\ \i                                     \\
              I_2     & = \int_{-1}^{1} (1)(t)\ \dl t = \color{y_p} 0 &
              I       & = I_1 + I_2 = 0
          \end{align}

    \item Cauchy's theorem \textcolor{y_p}{is not applicable} \par
          Resolving using partial fractions,
          \begin{align}
              g(z) & = \frac{2z - 1}{z(z-1)}       &
                   & = \frac{a}{z} + \frac{b}{z-1}   \\
              a    & = 1                           &
              b    & = 1                             \\
          \end{align}
          Since the path encloses both points $ (0, 0) $ and $ (1, 0) $, the standard
          result for $ z^n $ gives,
          \begin{align}
              I & = \oint_{C} \Bigg[\frac{1}{z} + \frac{1}{z-1}\Bigg]\ \dl z
              = 4\pi\ \i
          \end{align}

    \item Cauchy's theorem \textcolor{y_p}{is not applicable}. \par
          Let the path be split into two parts, the clockwise circle centered on
          $ (-1, 0) $ called $ C_1 $ and the counterclockwise circle centered on
          $ (1, 0) $ called $ C_2 $. \par
          \begin{align}
              g(z) & = \frac{1}{z^2 - 1}                                         &
                   & = \frac{0.5}{z-1} - \frac{0.5}{z+1}                           \\
              I_1  & = \oint_C \frac{0.5}{z-1}                                   &
                   & = \oint_{C_1} \frac{0.5}{z-1} + \oint_{C_2} \frac{0.5}{z-1}   \\
                   & = \color{y_h} 0 + \pi\ \i                                     \\
              I_2  & = \oint_C \frac{0.5}{z+1}                                   &
                   & = \oint_{C_1} \frac{0.5}{z+1} + \oint_{C_2} \frac{0.5}{z+1}   \\
                   & = \color{y_p} -\pi\ \i + 0                                    \\
              I    & = I_1 - I_2 = 2\pi\ \i
          \end{align}
          Since the path is the union of two simple closed paths, each of the paths
          can be integrated over separately.

    \item Cauchy's theorem \textcolor{y_p}{is not applicable}. \par
          Since the function is not analytic only at the origin, path deformation
          theorem ensures that the closed integral over any path encircling the origin
          remains the same.
          \begin{align}
              I_1              & : \abs{z} = 1           &
                               & \text{clockwise}          \\
              I_2              & : \abs{z} = 2           &
                               & \text{counterclockwise}   \\
              I_2              & = -I_1                  &
              \implies \quad I & = 0
          \end{align}

    \item Cauchy's theorem \textcolor{y_p}{is not applicable}. \par
          \begin{align}
              \sinh(z/2) & = 0                       &
              z          & = 2n\pi\ \i                 \\
              I          & = \oint_C f(z)\ \dl z = 0
          \end{align}

          \begin{figure}[H]
              \centering
              \begin{tikzpicture}
                  \begin{axis}[legend pos = outer north east,
                          height = 8cm, width = 8cm, axis equal,
                          xmin = -1.2, xmax = 1.2,
                          ymin = -0.5, ymax = 6.5,
                          PiStyleY, ytick distance = 0.5*pi,
                          grid = both,Ani,
                          colormap/jet,
                      ]
                      \node[GraphNode, fill = black] at (axis cs:0, 6.28){};
                      \node[GraphNode, fill = black] at (axis cs:0, 0){};
                      \addplot[GraphSmooth, mesh, point meta = \t, domain = 0:2*pi,
                          variable = \t]
                      ({cos(\t)}, {pi/2 - sin(\t)});
                  \end{axis}
              \end{tikzpicture}
          \end{figure}

    \item Cauchy's theorem \textcolor{y_p}{is not applicable}. \par
          Since the function is not analytic only at the origin, path deformation
          theorem ensures that the closed integral over any path encircling the origin
          remains the same.
          \begin{align}
              I_1              & : \abs{z} = 3           &
                               & \text{clockwise}          \\
              I_2              & : \abs{z} = 1           &
                               & \text{counterclockwise}   \\
              I_2              & = -I_1                  &
              \implies \quad I & = 0
          \end{align}

    \item Cauchy's theorem \textcolor{y_p}{is not applicable}. \par
          The function is analytic at all enclosed by the path of integration.
          \begin{align}
              z^4 - 16 & = 0                       &
              z        & = \{\pm 2, \pm 2\i\}        \\
              I        & = \oint_C f(z)\ \dl z = 0
          \end{align}

          \begin{figure}[H]
              \centering
              \begin{tikzpicture}
                  \begin{axis}[legend pos = outer north east,
                          height = 8cm, width = 8cm, axis equal,
                          xmin = -2.2, xmax = 2.2,
                          ymin = -2.2, ymax = 2.2,
                          grid = both,Ani,
                          colormap/jet,
                      ]
                      \node[GraphNode, fill = y_p] at (axis cs:2, 0){};
                      \node[GraphNode, fill = y_p] at (axis cs:-2, 0){};
                      \node[GraphNode, fill = y_p] at (axis cs:0, 2){};
                      \node[GraphNode, fill = y_p] at (axis cs:0, -2){};
                      \draw[y_h]
                      (1,0) -- (0,1) -- (-1,0) -- (0,-1) -- cycle;
                  \end{axis}
              \end{tikzpicture}
          \end{figure}

    \item Cauchy's theorem \textcolor{y_p}{is not applicable} since the contour
          includes $ z = 0 $. Using path deformation to convert the path into the unit
          circle,
          \begin{align}
              f(z)       & = \frac{\sin z}{z}                                    &
                         & = \frac{\sin(e^{-\i t})}{e^{-\i t}}                     \\
              z(t)       & = e^{-\i t}                                           &
              t          & \in [0, 2\pi]                                           \\
              \dot{z}(t) & = -\i\ e^{-\i t}                                      &
              I          & = \int_{0}^{2\pi} \sin(e^{-\i t}) (-\i)\ \dl t          \\
              I_1        & = \int_{0}^{\pi} \sin\Big[\cos t - \i\ \sin t\Big]
              \ \dl t    &
              I_2        & = \int_{\pi}^{2\pi} \sin\Big[\cos t - \i\ \sin t\Big]
              \ \dl t                                                              \\
              u          & = t-\pi                                               &
              \dl u      & = \dl t                                                 \\
              I_2        & = \int_{0}^{\pi} \sin\Big[ \cos(u+\pi) - \i
                  \sin(u + \pi)\Big]
              \dl u      &
              I_2        & = -I_1                                                  \\
              I_1 + I_2  & = 0
          \end{align}

    \item Resolving into partial fractions,
          \begin{align}
              f(z) & = \frac{2z^3 + z^2 + 4}{z^2(z^2 + 4)} = \frac{1}{z+2\i}
              + \frac{1}{z - 2\i} + \frac{1}{z^2}
          \end{align}

          \begin{figure}[H]
              \centering
              \begin{tikzpicture}
                  \begin{axis}[legend pos = outer north east,
                          height = 8cm, width = 8cm, axis equal,
                          %   xmin = -1.2, xmax = 1.2,
                          %   ymin = -0.5, ymax = 6.5,
                          grid = both,Ani,
                          colormap/jet,
                      ]
                      \node[GraphNode, fill = black] at (axis cs:0, 2){};
                      \node[GraphNode, fill = black] at (axis cs:0, -2){};
                      \node[GraphNode, fill = black] at (axis cs:0, 0){};
                      \addplot[GraphSmooth, mesh, point meta = \t, domain = 0:2*pi,
                          variable = \t]
                      ({2 + 4*cos(\t)}, {-4*sin(\t)});
                  \end{axis}
              \end{tikzpicture}
          \end{figure}
          Using the standard result for $ z^n $, and the fact that all 3 poles are
          contained within the contour,
          \begin{align}
              I & = -2\pi\ \i - 2\pi\ \i - 0 = -4\pi\ \i
          \end{align}
\end{enumerate}