\section{Derivatives of Analytic Functions}

\begin{enumerate}
    \item Integrating,
          \begin{align}
              I            & = \oint_C \frac{\sin z }{z^4}\ \dl z &
              f^{(3)}(z_0) & = \Big[-\cos z\Big]_{(0)} = -1         \\
              I            & = (-1)\ \frac{2\pi\i}{3!}            &
                           & = \frac{-\pi}{3}\ \i
          \end{align}

    \item Integrating,
          \begin{align}
              I            & = \oint_C \frac{z^6}{(2z - 1)^6}\ \dl z                 &
                           & = \oint_C \frac{(z/2)^6}{(z-1/2)^6}\ \dl z                \\
              f^{(5)}(z_0) & = \Bigg[ \frac{6!\ z}{64} \Bigg]_{(0.5)} = \frac{45}{8}   \\
              I            & = \frac{45}{8}\ \frac{2\pi\i}{5!}                       &
                           & = \frac{3\pi}{32}\ \i
          \end{align}

    \item Integrating, with non-negative integer $ n $,
          \begin{align}
              I              & = \oint_C \frac{e^z}{z^n}\ \dl z &
              f^{(n-1)}(z_0) & = \Big[ e^z \Big]_{(0)} = 1        \\
              I              & = \frac{2\pi\i}{(n-1)!}
          \end{align}

    \item Integrating,
          \begin{align}
              I            & = \oint_C \frac{e^z \cos z}{(z - 0.25\pi)^3}\ \dl z   \\
              f^{(2)}(z_0) & = \Big[ -2e^z \sin z \Big]_{(0.25\pi)}              &
                           & = -\sqrt{2} e^{\pi/4}                                 \\
              I            & = -\sqrt{2}e^{\pi/4}\ \frac{2\pi\i}{2!}             &
                           & = -\sqrt{2}\pi e^{\pi/4}\ \i
          \end{align}

    \item Integrating,
          \begin{align}
              I            & = \oint_C \frac{\cos 2z}{(z - 0.5)^4}\ \dl z   \\
              f^{(3)}(z_0) & = \Big[ 8\sinh 2z \Big]_{(0.5)}              &
                           & = 8\sinh(1)                                    \\
              I            & = 8\sinh(1)\ \frac{2\pi\i}{3!}               &
                           & = \frac{8\pi}{3} \sinh(1)\ \i
          \end{align}

    \item Integrating,
          \begin{align}
              I       & = \oint_C \frac{1}{(z - 0.5\i)^2(z - 2\i)^2}\ \dl z &
              f(z)    & = \frac{1}{(z-2\i)^2}                                 \\
              f'(z_0) & = \Bigg[ \frac{-2}{(z - 2\i)^3} \Bigg]_{(0.5\i)}    &
                      & = \frac{16}{27}\ \i                                   \\
              I       & = \frac{16\ \i}{27}\ \frac{2\pi\i}{1!}              &
                      & = \frac{-32\pi}{27}
          \end{align}

    \item Integrating,
          \begin{align}
              I             & = \oint_C \frac{cos z}{z^{2n+1}}\ \dl z &
              f(z)          & = \cos z                                  \\
              f^{(2n)}(z_0) & = \Bigg[ (-1)^n \cos z \Bigg]_{(0)}     &
                            & = (-1)^n                                  \\
              I             & = (-1)^n\ \frac{2\pi\i}{(2n)!}
          \end{align}

    \item Integrating,
          \begin{align}
              I        & = \oint_C \frac{z^3 + \sin z}{(z - \i)^3}\ \dl z &
              f(z)     & = z^3 + \sin z                                     \\
              f''(z_0) & = \Bigg[ 6z - \sin z \Bigg]_{(\i)}               &
                       & = [6 - \sinh(1)]\ \i                               \\
              I        & = \pi(\sinh 1 - 6)
          \end{align}

    \item Integrating, with a minus sign to accommodate the clockwise orientation,
          \begin{align}
              I       & = -\oint_C \frac{\tan(\pi z)}{z^2}\ \dl z &
              f(z)    & = \tan(\pi z)                               \\
              f'(z_0) & = \Bigg[ \pi\sec^2(z) \Bigg]_{(0)}        &
                      & = \pi                                       \\
              I       & = -2\pi^2 \i
          \end{align}

          \begin{figure}[H]
              \centering
              \begin{tikzpicture}
                  \begin{axis}[legend pos = outer north east,
                          height = 8cm, width = 8cm, axis equal,
                          title = {Square},
                          xmin = -3, xmax = 3,
                          ymin = -3, ymax = 3,
                          grid = both,Ani,
                          colormap/jet,
                      ]
                      \filldraw[draw=y_h,fill=y_h,
                          fill opacity = 0.08]
                      (-2,0) -- (0,2) -- (2,0) -- (0,-2) -- cycle;
                      \node[GraphNode, fill = y_p] at (axis cs:0, 1){};
                  \end{axis}
              \end{tikzpicture}
              \begin{tikzpicture}
                  \begin{axis}[legend pos = outer north east,
                          height = 8cm, width = 8cm, axis equal,
                          title = {Ellipse},
                          xmin = -5, xmax = 5,
                          ymin = -5, ymax = 5,
                          grid = both,Ani,
                          colormap/jet,
                      ]
                      \filldraw[draw=y_h,fill=y_h,
                          fill opacity = 0.08]
                      (0,0) circle [x radius=1, y radius=4];
                      \node[GraphNode, fill = y_p] at (axis cs:0, 0){};
                  \end{axis}
              \end{tikzpicture}
          \end{figure}

    \item Integrating, with the outer circle being ccw, and the inner circle being cw,
          \begin{align}
              I       & = \oint_C \frac{4z^3 - 6}{ z(z - 1 - \i)^2}\ \dl z &
              f(z)    & = 4z^2 - \frac{6}{z}                                 \\
              f'(z_0) & = \Bigg[ 8z + \frac{6}{z^2} \Bigg]_{(1 + \i)}      &
                      & = 8 + 5\i                                            \\
              I       & = (8 + 5\i)\ \frac{2\pi\i}{1!}                     &
                      & = 2\pi\ (-5 + 8\i)
          \end{align}

    \item Integrating,
          \begin{align}
              I       & = -\oint_C \frac{(1+z)\sin z}{(2z - 1)^2}\ \dl z        &
              f(z)    & = \frac{(1+z)\sin z}{4}                                   \\
              f'(z_0) & = \Bigg[ \frac{(1+z)\cos z + \sin z}{4} \Bigg]_{(0.5)}  &
                      & = \frac{1.5 \cos(1/2) + \sin(1/2)}{4}                     \\
              I       & = \Big[1.5\cos(1/2) + \sin(1/2)\Big]\ \frac{\pi}{2}\ \i
          \end{align}

          \begin{figure}[H]
              \centering
              \begin{tikzpicture}
                  \begin{axis}[legend pos = outer north east,
                          height = 8cm, width = 8cm, axis equal,
                          title = {Concentric circles},
                          xmin = -3, xmax = 3,
                          ymin = -3, ymax = 3,
                          grid = both,Ani,
                          colormap/jet,
                      ]
                      \filldraw[draw=y_h,fill=y_h,
                          fill opacity = 0.08,even odd rule]
                      (axis cs:0,0) circle (2) (axis cs:0,0) circle (1);
                      \node[GraphNode, fill = y_p] at (axis cs:1, 1){};
                      \node[GraphNode, fill = y_p] at (axis cs:0, 0){};
                  \end{axis}
              \end{tikzpicture}
              \begin{tikzpicture}
                  \begin{axis}[legend pos = outer north east,
                          height = 8cm, width = 8cm, axis equal,
                          title = {$ \abs{z - \i} = 2 $},
                          xmin = -3, xmax = 3,
                          ymin = -2, ymax = 4,
                          grid = both,Ani,
                          colormap/jet,
                      ]
                      \filldraw[draw=y_h,fill=y_h, fill opacity = 0.08]
                      (axis cs:0,1) circle (2);
                      \node[GraphNode, fill = y_p] at (axis cs:0.5, 0){};
                  \end{axis}
              \end{tikzpicture}
          \end{figure}

    \item Integrating, with a minus sign to accommodate the clockwise orientation,
          \begin{align}
              I       & = -\oint_C \frac{e^{z^2}}{z(z - 2\i)^2}\ \dl z           &
              f(z)    & = \frac{\exp(z^2)}{z}                                      \\
              f'(z_0) & = \Bigg[ \frac{(2z^2 - 1) \exp(z^2)}{z^2} \Bigg]_{(2\i)} &
                      & = \frac{9e^{-4}}{4}                                        \\
              I       & = -\frac{9e^{-4}}{4}\ \frac{2\pi\i}{1!}                  &
                      & = -\frac{9e^{-4}}{2}\ \pi\ \i
          \end{align}

    \item Integrating,
          \begin{align}
              I       & = -\oint_C \frac{\Ln z}{(z - 2)^2}\ \dl z &
              f(z)    & = \Ln z                                     \\
              f'(z_0) & = \Bigg[ \frac{1}{z} \Bigg]_{(2)}         &
                      & = 0.5                                       \\
              I       & = \pi\i
          \end{align}

          \begin{figure}[H]
              \centering
              \begin{tikzpicture}
                  \begin{axis}[legend pos = outer north east,
                          height = 8cm, width = 8cm, axis equal,
                          title = {$ \abs{z - 3\i} = 2 $},
                          xmin = -3, xmax = 3,
                          ymin = -1, ymax = 5,
                          grid = both,Ani,
                          colormap/jet,
                      ]
                      \filldraw[draw=y_h,fill=y_h, fill opacity = 0.08]
                      (axis cs:0,3) circle (2);
                      \node[GraphNode, fill = y_p] at (axis cs:0, 2){};
                      \node[GraphNode, fill = y_p] at (axis cs:0, 0){};
                  \end{axis}
              \end{tikzpicture}
              \begin{tikzpicture}
                  \begin{axis}[legend pos = outer north east,
                          height = 8cm, width = 8cm, axis equal,
                          title = {$ \abs{z - 3} = 2 $},
                          xmin = 1, xmax = 5,
                          ymin = -2, ymax = 2,
                          grid = both,Ani,
                          colormap/jet,
                      ]
                      \filldraw[draw=y_h,fill=y_h, fill opacity = 0.08]
                      (axis cs:3,0) circle (2);
                      \node[GraphNode, fill = y_p] at (axis cs:2, 0){};
                  \end{axis}
              \end{tikzpicture}
          \end{figure}

    \item Integrating, with a minus sign to accommodate the clockwise orientation,
          \begin{align}
              I                   & = \oint_C \frac{\Ln{(z+3)}}{(z-2) (z + 1)^2}
              \ \dl z             &
              f(z)                & = \frac{\Ln{(z+3)}}{(z-2)}                   \\
              f'(z_0)             & = \Bigg[ \frac{1}{(z-2)(z+3)} -
                  \frac{\Ln{(z+3)}}{(z-2)^2}
              \Bigg]_{(-1)}       &
                                  & = -\frac{1}{6} - \frac{\ln(2)}{9}            \\
              I                   & = \Bigg[\frac{-3 - \ln(4)}{18}\Bigg]
              \ \frac{2\pi\i}{1!} &
                                  & = \frac{-3-\ln 4}{9}\ \pi\i
          \end{align}

    \item Integrating results in zero, since the integrand is analytic in the entire
          region between the curves (shaded). Path deformation and the fact that the
          orientation of the curves are opposite means that the integrals are equal and
          opposite.

          \begin{figure}[H]
              \centering
              \begin{tikzpicture}
                  \begin{axis}[legend pos = outer north east,
                          height = 8cm, width = 8cm, axis equal,
                          xmin = -2.2, xmax = 2.2,
                          ymin = -2.2, ymax = 2.2,
                          grid = both,Ani,
                          colormap/jet,
                      ]
                      \filldraw[draw=y_h,fill=y_h,
                          fill opacity = 0.08]
                      (-1.5,0) -- (0,1.5) -- (1.5,0) -- (0,-1.5) -- cycle;
                      \node[GraphNode, fill = y_p] at (axis cs:-1, 0){};
                      \node[GraphNode, fill = y_p] at (axis cs:2, 0){};
                  \end{axis}
              \end{tikzpicture}
              \begin{tikzpicture}
                  \begin{axis}[legend pos = outer north east,
                          height = 8cm, width = 8cm, axis equal,
                          xmin = -6.2, xmax = 6.2,
                          ymin = -6.2, ymax = 6.2,
                          grid = both,Ani,
                          colormap/jet,
                      ]
                      \filldraw[draw=y_h,fill=y_h,
                          fill opacity = 0.08,even odd rule]
                      (axis cs:0,0) circle (6) (axis cs:3,0) circle (2);
                      \node[GraphNode, fill = y_p] at (axis cs:4, 0){};
                  \end{axis}
              \end{tikzpicture}
          \end{figure}

    \item Integrating, with a minus sign to accommodate the clockwise orientation,
          \begin{align}
              I       & = -\oint_C \frac{e^{4z}}{z(z - 2\i)^2}\ \dl z        &
              f(z)    & = \frac{\exp(4z)}{z}                                   \\
              f'(z_0) & = \Bigg[ \frac{(4z - 1)\exp(4z)}{z^2} \Bigg]_{(2\i)} &
                      & = \frac{(1 - 8\i)e^{8\i}}{4}                           \\
              I       & = \frac{(1 - 8\i)e^{8\i}}{4}\ \frac{2\pi\i}{1!}      &
                      & = \frac{(\i + 8)\pi}{2}\ e^{8\i}
          \end{align}

    \item Integrating results in zero, since the integrand is analytic in the entire
          region between the curves (shaded). Path deformation and the fact that the
          orientation of the curves are opposite means that the integrals are equal and
          opposite.

          \begin{figure}[H]
              \centering
              \begin{tikzpicture}
                  \begin{axis}[legend pos = outer north east,
                          height = 8cm, width = 8cm, axis equal,
                          title = {Enclosed circles},
                          xmin = -4, xmax = 4,
                          ymin = -3, ymax = 5,
                          grid = both,Ani,
                          colormap/jet,
                      ]
                      \filldraw[draw=y_h,fill=y_h,
                          fill opacity = 0.08,even odd rule]
                      (axis cs:0,1) circle (3) (axis cs:0,0) circle (1);
                      \node[GraphNode, fill = y_p] at (axis cs:0, 0){};
                      \node[GraphNode, fill = y_p] at (axis cs:0, 2){};
                  \end{axis}
              \end{tikzpicture}
              \begin{tikzpicture}
                  \begin{axis}[legend pos = outer north east,
                          height = 8cm, width = 8cm, axis equal,
                          title = {Enclosed circles},
                          xmin = -6, xmax = 6,
                          ymin = -6, ymax = 6,
                          grid = both,Ani,
                          colormap/jet,
                      ]
                      \filldraw[draw=y_h,fill=y_h,
                          fill opacity = 0.08,even odd rule]
                      (axis cs:0,0) circle (5) (axis cs:3,0) circle (1.5);
                      \node[GraphNode, fill = y_p] at (axis cs:4, 0){};
                  \end{axis}
              \end{tikzpicture}
          \end{figure}

    \item Integrating, with integer $ n > 0 $,
          \begin{align}
              I              & = -\oint_C \frac{\sinh z}{z^n}\ \dl z &
              f(z)           & = \sinh z                               \\
              f^{(n-1)}(z_0) & = \begin{cases}
                                     1 & n\ \text{is even} \\
                                     0 & n\ \text{is odd}  \\
                                 \end{cases}                 \\
              I              & = \frac{2\pi\i}{(n-1)!}\ f^{(n-1)}(0)
          \end{align}
          For the case where, $ n \leq 0 $, the answer is zero, since the integrand
          becomes analytic everywhere.

    \item Integrating,
          \begin{align}
              I       & = -\oint_C \frac{e^{3z}}{(4z - \pi\i)^3}\ \dl z      &
              f(z)    & = \frac{e^{3z}}{64}                                    \\
              f'(z_0) & = \Bigg[ \frac{9e^{3z}}{64} \Bigg]_{(\pi/4\ \i)}     &
                      & = \frac{9}{64\sqrt{2}}\ (-1 + \i)                      \\
              I       & = \frac{9}{64\sqrt{2}}\ (-1 + \i)\ \frac{2\pi\i}{2!} &
              I       & = \frac{9\pi}{64\sqrt{2}}\ (-1-\i)
          \end{align}

          \begin{figure}[H]
              \centering
              \begin{tikzpicture}
                  \begin{axis}[legend pos = outer north east,
                          height = 8cm, width = 8cm, axis equal,
                          title = {Unit circle},
                          xmin = -1.2, xmax = 1.2,
                          ymin = -1.2, ymax = 1.2,
                          grid = both,Ani,
                          colormap/jet,
                      ]
                      \filldraw[draw=y_h,fill=y_h, fill opacity = 0.08]
                      (axis cs:0,0) circle (1);
                      \node[GraphNode, fill = y_p] at (axis cs:0, 0){};
                  \end{axis}
              \end{tikzpicture}
              \begin{tikzpicture}
                  \begin{axis}[legend pos = outer north east,
                          height = 8cm, width = 8cm, axis equal,
                          title = {Unit circle},
                          xmin = -1.2, xmax = 1.2,
                          ymin = -1.2, ymax = 1.2,
                          grid = both,Ani,
                          colormap/jet,
                      ]
                      \filldraw[draw=y_h,fill=y_h, fill opacity = 0.08]
                      (axis cs:0,0) circle (1);
                      \node[GraphNode, fill = y_p] at (axis cs:0, 0.7854){};
                  \end{axis}
              \end{tikzpicture}
          \end{figure}

    \item Growth
          \begin{enumerate}
              \item Suppose $ f(z) $ is bounded,
                    \begin{align}
                        \abs{f(z)}        & \leq M          &
                        \forall \qquad  z & \in \mathcal{C}   \\
                    \end{align}
                    By Lioville's theorem, this function is constant. This contradicts
                    the assumption that the function is not constant. R.A.A.

              \item $ f(z) $ is a polynomial of positive degree. It is a non
                    constant entire function. From part $ a $, it is unbounded.

              \item $ f(z) = e^x $. This is a real function of a real variable. It
                    is a non-constant entire function and thus obeys the relation in part
                    $ a $. \par
                    Fix an $ M_0 $ such that $ R_0 = \ln(M_0) $. Now,
                    \begin{align}
                        \abs{f(z)}             & > M_0 &
                        \forall \quad \Re{(z)} & > R_0
                    \end{align}
                    A complex number can have a modulus larger than $ R_0 $ while still
                    having a real part less than $R_0$. So, the relation in part $ b $ is
                    violated.

              \item If $ f(z) $ is never zero, then it has either a lower or upper bound
                    equal to zero, by definition. This violates result $ a $, and makes
                    the function constant, which is a contradiction.
          \end{enumerate}
\end{enumerate}