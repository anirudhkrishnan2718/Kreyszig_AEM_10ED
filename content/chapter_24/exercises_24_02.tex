\section{Experiments, Outcomes, Events}

\begin{enumerate}
    \item Each screw can either be left or right-handed.
          \begin{align}
              S & = \{LLL,\ RLL,LRL,LLR,\ LRR,RLR,RRL,\ RRR\}
          \end{align}

    \item Each coin can either land heads or tails.
          \begin{align}
              S & = \{HH,\ HT, TH,\ TT\}
          \end{align}

    \item Each die can land on the numbers one to six.
          \begin{align}
              S & = \begin{bNiceMatrix}[r, margin]
                        (1,1) & (1,2) & (1,3) & (1,4) & (1,5) & (1,6) \\
                        (2,1) & (2,2) & (2,3) & (2,4) & (2,5) & (2,6) \\
                        (3,1) & (3,2) & (3,3) & (3,4) & (3,5) & (3,6) \\
                        (4,1) & (4,2) & (4,3) & (4,4) & (4,5) & (4,6) \\
                        (5,1) & (5,2) & (5,3) & (5,4) & (5,5) & (5,6) \\
                        (6,1) & (6,2) & (6,3) & (6,4) & (6,5) & (6,6) \\
                    \end{bNiceMatrix}
          \end{align}

    \item Inifnitely many outcomes.
          \begin{align}
              S & = \{6, 16,26,36,46,56,116,126,136,146,156,\dots,556,1116,\dots\}
          \end{align}

    \item Inifnitely many outcomes.
          \begin{align}
              S & = \{TH,\ TTH,\ TTTH,\ TTTTH,\ TTTTTH,\dots\}
          \end{align}

    \item For three nonnegative real numbers $ T_j $,
          \begin{align}
              S & = \{T_1, T_2, T_3\}
          \end{align}

    \item For some nonnegative real number as the temperature in Kelvin and
          pressure in bar,
          \begin{align}
              S & = \{T, P\}
          \end{align}

    \item Using combinatorics,
          \begin{align}
              S & = \{AB,AC,AD,AE,\quad BC,BD,BE,\quad CD,CE,\quad DE\}
          \end{align}

    \item Representing the normal and defective gaskets with $ N $ and $ D $,
          \begin{align}
              S & = \{D, ND, NND, NNND, NNNND, \dots, NNNNNNNNND\}
          \end{align}
          At best, the tenth gasket drawn will be the defective one, which means this
          has 10 outcomes.

    \item Looking at the outcomes making belonging to both events,
          \begin{align}
              A \cap B & = \{5,5,5\}
          \end{align}
          The events are not mutually exclusive.

    \item Looking at the outcomes making belonging to both events,
          \begin{align}
              A & = \{3,6,9,12\} & B & = \{5,10\}
          \end{align}
          The events are mutually exclusive.

    \item The subsets are,
          \begin{align}
              S & = \{a,b,c\}                                                       \\
              E & = \{a\}, \{b\}, \{c\}, \{a,b\}, \{b,c\}, \{a,c\}, \{a,b,c\}, \phi
          \end{align}

    \item The events are,
          \begin{align}
              A             & = \{(1,1),\ (2,2),\ (3,3),\ (4,4),\ (5,5),\ (6,6)\} \\
              B             & = \{(1,1),\ (1,2),\ (1,3),\ (2,1),\ (2,2),\ (3,1)\} \\
              A \cup B      & = \{(1,1),\ (1,2),\ (1,3),\ (2,1),\ (2,2),\ (3,1)
              ,\ (3,3),\ (4,4),\ (5,5),\ (6,6)\}                                  \\
              A \cap B      & = \{(1,1),\ (2,2)\}                                 \\
              A^\complement & = \begin{bNiceMatrix}[r, margin]
                                          & (1,2) & (1,3) & (1,4) & (1,5) & (1,6) \\
                                    (2,1) &       & (2,3) & (2,4) & (2,5) & (2,6) \\
                                    (3,1) & (3,2) &       & (3,4) & (3,5) & (3,6) \\
                                    (4,1) & (4,2) & (4,3) &       & (4,5) & (4,6) \\
                                    (5,1) & (5,2) & (5,3) & (5,4) &       & (5,6) \\
                                    (6,1) & (6,2) & (6,3) & (6,4) & (6,5) &       \\
                                \end{bNiceMatrix}     \\
              B^\complement & = \begin{bNiceMatrix}[r, margin]
                                          &       &       & (1,4) & (1,5) & (1,6) \\
                                          &       & (2,3) & (2,4) & (2,5) & (2,6) \\
                                          & (3,2) & (3,3) & (3,4) & (3,5) & (3,6) \\
                                    (4,1) & (4,2) & (4,3) & (4,4) & (4,5) & (4,6) \\
                                    (5,1) & (5,2) & (5,3) & (5,4) & (5,5) & (5,6) \\
                                    (6,1) & (6,2) & (6,3) & (6,4) & (6,5) & (6,6) \\
                                \end{bNiceMatrix}
          \end{align}

    \item The outcomes within each event are,
          \begin{align}
              A & = \{LR, RL, RR\} & B & = \{RL, LR, LL\} \\
              C & = \{RR\}         & D & = \{LL\}         \\
          \end{align}
          $ A $ and $ B $ are not mutually exclusive.
          $ C $ and $ D $ are.

    \item The events in the Venn diagram are,
          \begin{itemize}
              \item[1:] see paris, don't have a good time and don't run out of money.
              \item[2:] see paris, have a good time and don't run out of money.
              \item[3:] don't see paris, have a good time and don't run out of money.
              \item[4:] don't see paris, have a good time and run out of money.
              \item[5:] don't see paris, don't have a good time and run out of money.
              \item[6:] see paris, don't have a good time and run out of money.
              \item[7:] see paris, have a good time and run out of money.
          \end{itemize}

    \item Using the definition of the complement,
          \begin{align}
              A^\complement               & = S-A           &
              (A^\complement)^\complement & = S - (S-A) = A   \\
              A \cup A^\complement        & = A \cup (S-A)  &
                                          & = S               \\
              A \cap A^\complement        & = A \cap (S-A)  &
                                          & = \phi            \\
              S^\complement               & = S - S = \phi  &
              \phi^\complement            & = S - \phi = S    \\
          \end{align}

    \item $A$ is a subset of $ B $. Then, there are no elements of $ A $ that are not in
          $ B $. Thus, $ A \cup B = B $ \par
          $ A \cup B = B $. Then, there are no elements of $ A $ that are not in
          $ B $. Thus, $ A \subseteq B $

    \item $A$ is a subset of $ B $. Then, there are no elements of $ A $ that are not in
          $ B $. Thus, $ A \cap B = A $ \par
          $ A \cap B = A $. Then, there are no elements of $ A $ that are not in
          $ B $. Thus, $ A \subseteq B $

    \item Checking De-Morgan's laws,
          \begin{figure}[H]
              \centering
              \begin{tikzpicture}
                  \begin{axis}[axis equal, height = 4cm,
                          title = {$ A^\complement $},axis lines = none, Ani,
                          xmin = -1.6,xmax = 1.6, ymin = -1.6, ymax = 1.6]
                      \filldraw[y_h!10] (-2.5,-1.6) rectangle (2.5,1.6);
                      \fill[white](-0.5,0) circle(1);
                      \draw[y_h, thick] (-0.5,0) circle (1);
                      \draw[black, thick] (-2.5,-1.6) rectangle (2.5,1.6);
                  \end{axis}
              \end{tikzpicture}
              \hspace{6em}
              \begin{tikzpicture}
                  \begin{axis}[axis equal, height = 4cm,
                          title = {$ B^\complement $},axis lines = none, Ani,
                          xmin = -1.6,xmax = 1.6, ymin = -1.6, ymax = 1.6]
                      \filldraw[y_p!10] (-2.5,-1.6) rectangle (2.5,1.6);
                      \fill[white](0.5,0) circle(1);
                      \draw[y_p, thick] (0.5,0) circle (1);
                      \draw[black, thick] (-2.5,-1.6) rectangle (2.5,1.6);
                  \end{axis}
              \end{tikzpicture}
          \end{figure}
          \begin{figure}[H]
              \centering
              \begin{tikzpicture}
                  \begin{axis}[axis equal, height = 4cm,
                          title = {$ (A \cup B)^\complement $},axis lines = none, Ani,
                          xmin = -1.6,xmax = 1.6, ymin = -1.6, ymax = 1.6]
                      \filldraw[y_h!50!y_p, fill opacity = 0.15]
                      (-2.5,-1.6) rectangle (2.5,1.6);
                      \fill[white](-0.5,0) circle(1);
                      \fill[white](0.5,0) circle(1);
                      \draw[y_h, thick] (-0.5,0) circle (1);
                      \draw[y_p, thick] (0.5,0) circle (1);
                      \draw[black, thick] (-2.5,-1.6) rectangle (2.5,1.6);
                  \end{axis}
              \end{tikzpicture}
              \hspace{6em}
              \begin{tikzpicture}
                  \begin{axis}[axis equal, height = 4cm,
                          title = {$ (A \cap B)^\complement $},axis lines = none, Ani,
                          xmin = -1.6,xmax = 1.6, ymin = -1.6, ymax = 1.6]
                      \filldraw[y_h!50!y_p, fill opacity = 0.15]
                      (-2.5,-1.6) rectangle (2.5,1.6);
                      \begin{scope}
                          \clip (-0.5,0) circle(1);
                          \clip (0.5,0) circle(1);
                          \fill[white!15](-0.5,0) circle(1);
                      \end{scope}
                      \draw[y_h, thick] (-0.5,0) circle (1);
                      \draw[y_p, thick] (0.5,0) circle (1);
                      \draw[black, thick] (-2.5,-1.6) rectangle (2.5,1.6);
                  \end{axis}
              \end{tikzpicture}
          \end{figure}

    \item Checking the relations using Venn diagrams,
          \begin{figure}[H]
              \centering
              \begin{tikzpicture}
                  \begin{axis}[axis equal, height = 6cm, width = 6cm,
                          title = {$ \textcolor{y_h}{\textbf{A}} \cup
                                      \textcolor{y_p}{\textbf{B}} $},
                          axis lines = none, Ani,
                          xmin = -3.5,xmax = 3.5, ymin = -3.5, ymax = 3.5]
                      \fill[gray!15](0,1) circle (2);
                      \fill[gray!15](0.866,-0.5) circle (2);
                      \draw[y_h, thick] (0,1) circle (2);
                      \draw[y_p, thick] (0.866,-0.5) circle (2);
                      \draw[y_t, thick] (-0.866,-0.5) circle (2);
                      \draw[black, thick] (-3.45,-3.45) rectangle (3.45,3.45);
                  \end{axis}
              \end{tikzpicture}
              \hspace{1em}
              \begin{tikzpicture}
                  \begin{axis}[axis equal, height = 6cm, width = 6cm,
                          title = {$ \textcolor{y_h}{\textbf{A}} \cup
                                      \textcolor{y_t}{\textbf{C}} $},
                          axis lines = none, Ani,
                          xmin = -3.5,xmax = 3.5, ymin = -3.5, ymax = 3.5]
                      \fill[gray!15](0,1) circle (2);
                      \fill[gray!15](-0.866,-0.5) circle (2);
                      \draw[y_h, thick] (0,1) circle (2);
                      \draw[y_p, thick] (0.866,-0.5) circle (2);
                      \draw[y_t, thick] (-0.866,-0.5) circle (2);
                      \draw[black, thick] (-3.45,-3.45) rectangle (3.45,3.45);
                  \end{axis}
              \end{tikzpicture}
              \hspace{1em}
              \begin{tikzpicture}
                  \begin{axis}[axis equal, height = 6cm, width = 6cm,
                          title = {$ \textcolor{y_h}{\textbf{A}} \cup
                                      (\textcolor{y_p}{\textbf{B}} \cap
                                      \textcolor{y_t}{\textbf{C}}) $},
                          axis lines = none, Ani,
                          xmin = -3.5,xmax = 3.5, ymin = -3.5, ymax = 3.5]
                      \begin{scope}
                          \clip (-0.866,-0.5) circle(2);
                          \clip (0.866,-0.5) circle(2);
                          \fill[gray!15](-0.866,-0.5) circle(2);
                      \end{scope}
                      \fill[gray!15](0,1) circle (2);
                      \draw[y_h, thick] (0,1) circle (2);
                      \draw[y_p, thick] (0.866,-0.5) circle (2);
                      \draw[y_t, thick] (-0.866,-0.5) circle (2);
                      \draw[black, thick] (-3.45,-3.45) rectangle (3.45,3.45);
                  \end{axis}
              \end{tikzpicture}
          \end{figure}
          \begin{figure}[H]
              \centering
              \begin{tikzpicture}
                  \begin{axis}[axis equal, height = 6cm, width = 6cm,
                          title = {$ \textcolor{y_h}{\textbf{A}} \cap
                                      \textcolor{y_p}{\textbf{B}} $},
                          axis lines = none, Ani,
                          xmin = -3.5,xmax = 3.5, ymin = -3.5, ymax = 3.5]
                      \begin{scope}
                          \clip (0,1) circle(2);
                          \clip (0.866,-0.5) circle(2);
                          \fill[gray!15](0.866,-0.5) circle(2);
                      \end{scope}
                      \draw[y_h, thick] (0,1) circle (2);
                      \draw[y_p, thick] (0.866,-0.5) circle (2);
                      \draw[y_t, thick] (-0.866,-0.5) circle (2);
                      \draw[black, thick] (-3.45,-3.45) rectangle (3.45,3.45);
                  \end{axis}
              \end{tikzpicture}
              \hspace{1em}
              \begin{tikzpicture}
                  \begin{axis}[axis equal, height = 6cm, width = 6cm,
                          title = {$ \textcolor{y_h}{\textbf{A}} \cap
                                      \textcolor{y_t}{\textbf{C}} $},
                          axis lines = none, Ani,
                          xmin = -3.5,xmax = 3.5, ymin = -3.5, ymax = 3.5]
                      \begin{scope}
                          \clip (0,1) circle(2);
                          \clip (-0.866,-0.5) circle(2);
                          \fill[gray!15](-0.866,-0.5) circle(2);
                      \end{scope}
                      \draw[y_h, thick] (0,1) circle (2);
                      \draw[y_p, thick] (0.866,-0.5) circle (2);
                      \draw[y_t, thick] (-0.866,-0.5) circle (2);
                      \draw[black, thick] (-3.45,-3.45) rectangle (3.45,3.45);
                  \end{axis}
              \end{tikzpicture}
              \hspace{1em}
              \begin{tikzpicture}
                  \begin{axis}[axis equal, height = 6cm, width = 6cm,
                          title = {{$ \textcolor{y_h}{\textbf{A}} \cap
                                              (\textcolor{y_p}{\textbf{B}} \cup
                                              \textcolor{y_t}{\textbf{C}}) $}},
                          axis lines = none, Ani,
                          xmin = -3.5,xmax = 3.5, ymin = -3.5, ymax = 3.5]
                      \begin{scope}
                          \clip (0,1) circle(2);
                          \clip (-0.866,-0.5) circle(2);
                          \fill[gray!15](-0.866,-0.5) circle(2);
                      \end{scope}
                      \begin{scope}
                          \clip (0,1) circle(2);
                          \clip (0.866,-0.5) circle(2);
                          \fill[gray!15](0.866,-0.5) circle(2);
                      \end{scope}
                      \draw[y_h, thick] (0,1) circle (2);
                      \draw[y_p, thick] (0.866,-0.5) circle (2);
                      \draw[y_t, thick] (-0.866,-0.5) circle (2);
                      \draw[black, thick] (-3.45,-3.45) rectangle (3.45,3.45);
                  \end{axis}
              \end{tikzpicture}
          \end{figure}
\end{enumerate}