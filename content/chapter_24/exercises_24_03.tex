\section{Probability}

\begin{enumerate}
    \item Three fair dice, each of which rolls a number from 1 to 6. The unwanted
          outcomes are,
          \begin{align}
              A                & = \{(6,6,6), (6,6,5), (5,6,6), (6,5,6)\} \\
              P(A^\complement) & = 1 - P(A) = 1 - \frac{4}{6^3}
              = \frac{53}{54}
          \end{align}

    \item Two fair dice, each of which rolls a number from 1 to 6. The required
          outcomes are,
          \begin{align}
              A                & = \{(2,2), (3,1), (1,3), (2,3), (3,2), (4,1), (1,4)
              , (1,5), (2,4), (3,3), (4,2), (5,1)\}                                  \\
              P(A^\complement) & = \frac{12}{6^2} = \frac{1}{3}
          \end{align}

    \item Drawing with replacement, $ P(D) = 0.1 $
          \begin{align}
              P(A) & = (0.9)^3 = \color{y_h}0.729
          \end{align}
          Drawing without replacement results in a slightly smaller probability,
          \begin{align}
              P(B) & = \frac{90}{100} \cdot \frac{89}{99} \cdot \frac{88}{98}
              = \color{y_p}0.7265
          \end{align}

    \item \begin{enumerate}
              \item Drawing with replacement, $ P(D) = 0.1 $ the direct method is,
                    \begin{align}
                        P(E_1) & = 3 \cdot 0.1 \cdot 0.9^2 + 3 \cdot (0.1)^2 \cdot 0.9
                        + 0.1^3
                        = \color{y_h}0.271
                    \end{align}
                    whereas the complement method gives,
                    \begin{align}
                        P(E_2) & = 1 - (0.9)^3 = \color{y_p} 0.271
                    \end{align}
              \item Drawing without replacement, $ P(D) = 0.1 $ the direct method is,
                    \begin{align}
                        P(E_1) & = \frac{10}{100} \cdot \frac{90}{99} \cdot
                        \frac{89}{98}
                        + \frac{90}{100} \cdot \frac{10}{99} \cdot \frac{89}{98}
                        + \frac{90}{100} \cdot \frac{89}{99} \cdot \frac{10}{98} \\
                               & + \frac{10}{100} \cdot \frac{9}{99} \cdot
                        \frac{90}{98}
                        + \frac{10}{100} \cdot \frac{90}{99} \cdot \frac{9}{98}
                        + \frac{90}{100} \cdot \frac{10}{99} \cdot \frac{9}{98}  \\
                               & + \frac{10}{100} \cdot \frac{9}{99}
                        \cdot \frac{8}{98} = \color{y_h} \frac{67}{245}
                    \end{align}
                    whereas the complement method gives,
                    \begin{align}
                        P(E_2) & = 1 - \frac{90}{100} \cdot \frac{89}{99}
                        \cdot \frac{88}{98} = \color{y_p} \frac{67}{245}
                    \end{align}
          \end{enumerate}

    \item $ P(L) = 1/3 $ and $ P(R) = 2/3 $
          \begin{align}
              P(E) & = 1 - \frac{1}{3} \cdot \frac{1}{3} = \frac{8}{9}
          \end{align}

    \item $ P(L) = 1/3 $ and $ P(R) = 2/3 $ when sampling without replacement,
          \begin{align}
              P(E) & = 1 - \frac{10}{30} \cdot \frac{9}{29} = \frac{26}{29}
          \end{align}
          The probability of the complement goes down, so the probability of the event
          should increase.

    \item When sampling from a large population and when each kind of item
          that can be drawn is present in much larger numbers than the number of draws.

    \item $ P(F) = 0.1 $ and $ P(S) = 0.9 $
          \begin{align}
              P(E) = 0.9^4 = 0.6561
          \end{align}

    \item 2 out of 500 sheets contain spots. In order to draw 5 clean sheets,
          \begin{align}
              P(E) & = \frac{498}{500} \cdot \frac{497}{499} \cdot \frac{496}{498} \cdot
              \frac{495}{497} \cdot \frac{494}{496} = \SI{98.008}{\percent}
          \end{align}
          The slight increase above $ \SI{98}{\percent} $ comes from the
          non-replacement.

    \item $ P(F) = P(M) = 0.5 $, when drawing with replacement,
          \begin{align}
              E & = \{FF, MFF, FMF, MMFF, FMMF, MMFF\}                        \\
                & =  \frac{1}{4} + \frac{1}{4} + \frac{3}{16} = \frac{11}{16}
          \end{align}

    \item $ N_U = N_O = 50 $ and $ N_D = 100 $
          \begin{enumerate}
              \item Two rods of desired length, drawing without replacement
                    \begin{align}
                        P(A) & = \frac{100}{200} \cdot \frac{99}{199} = \frac{99}{398}
                    \end{align}
              \item Exactly one of desired length, drawing without replacement
                    \begin{align}
                        P(B) & = 2 \cdot \frac{100}{200} \cdot \frac{100}{199}
                        = \frac{100}{199}
                    \end{align}
              \item Zero of desired length, drawing without replacement
                    \begin{align}
                        P(C) & = \frac{100}{200} \cdot \frac{99}{199} = \frac{99}{398}
                    \end{align}
          \end{enumerate}

    \item Let the probability of failure per switch be $ p $,
          \begin{align}
              P(A) & = 0.99 = (1-p)^4 & p & = \SI{0.251}{\percent}
          \end{align}

    \item Let the probability of failure per switch be $ p $,
          \begin{align}
              P(E) & = (1-p)^4 & = 1 - (1-0.04)^3 = \SI{11.53}{\percent}
          \end{align}

    \item For $k$ defective out of two drawn,
          \begin{align}
              P(E_0)                & = (0.98)^2 = 0.9604                &
              P(E_1)                & = 2 \cdot 0.02 \cdot 0.98 = 0.0392   \\
              P(E_2)                & = (0.02)^2 = 0.0004                &
              \sum_{j=0}^{2} P(E_j) & = 0.9604 + 0.0392 + 0.0004 = 1
          \end{align}
          Since these events together cover the entire sample space of drawing two
          plugs, their probabilities sum to unity.

    \item Looking at the probabilites using the complement method,
          \begin{enumerate}
              \item Firing one shot,
                    \begin{align}
                        P(A) & = 1 - \frac{1}{2} = 0.5
                    \end{align}
              \item Firing two shot,
                    \begin{align}
                        P(A) & = 1 - \frac{3^2}{4^2} = 0.4375
                    \end{align}
              \item Firing four shot,
                    \begin{align}
                        P(A) & = 1 - \frac{7^4}{8^4} = 0.4138
                    \end{align}
          \end{enumerate}

    \item Looking at the probabilites for the events,
          \begin{align}
              P(A) & = \frac{1}{2} & P(B) & = \frac{1}{2} \\
              P(C) & = \frac{1}{2}
          \end{align}
          Looking at the four relations,
          \begin{align}
              P(A \cup B) & = \frac{1}{4} & P(B \cup C)        & = \frac{1}{4} \\
              P(A \cup C) & = \frac{1}{4} & P(A \cup B \cup C) & = 0
          \end{align}
          the first three hold, but the fourth does not.

    \item $ B $ is a subset of $ A $,
          \begin{align}
              A    & = B + (A-B) & P(A) & = P(B) + P(A-B) \\
              P(A) & \geq P(B)
          \end{align}
          where the equality only holds if the two sets are identical.

    \item Let $ A \cap B = D $,
          \begin{align}
              P(D \cap C) & = P(D|C) \cdot P(C) = P(D) \cdot P(C|D)  \\
                          & = P(C| A \cap B) \cdot P(A \cap B)       \\
                          & = P(C| A \cap B) \cdot P(B|A) \cdot P(A)
          \end{align}

    \item Divisibility of a number by $ 2,3,5 $ is the same as the corresponding digit
          on the coin being 1.
          \begin{align}
              S = \{7, 15, 10, 6\}
          \end{align}
          Clearly, no number in this set is divisible by all three of the divisors, even
          though there exist 2 out of 4 numbers that are divisible by two of them.
\end{enumerate}

