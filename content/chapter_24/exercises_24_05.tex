\section{Random Variables. Probability Distributions}

\begin{enumerate}
    \item Using the normalization condition,
          \begin{align}
              1 & = \sum_{j=1}^{5} p_j = k(1 + 4 + 9 + 16 + 25) &
              k & = \frac{1}{55}
          \end{align}
          \begin{figure}[H]
              \centering
              \begin{tikzpicture}
                  \begin{axis}[width = 8cm,title = {Discrete PDF}, Ani,
                          grid = both]
                      \addplot+[ycomb, mark options={black, mark size = 1.5pt},
                          y_h, samples at = {1,2,3,4,5}] {x^2/55};
                  \end{axis}
              \end{tikzpicture}
              \begin{tikzpicture}
                  \begin{axis}[width = 8cm,title = {Discrete CDF}, Ani,
                          grid = both, xmin = 0.5, xmax = 5.5]
                      \addplot+[const plot, mark options={black, mark size = 1.5pt},
                          y_p] coordinates {(-1,0) (1,1/55) (2,5/55) (3,14/55) (4,30/55)
                              (5,55/55) (7,1)};
                  \end{axis}
              \end{tikzpicture}
          \end{figure}

    \item Using the normalization condition,
          \begin{align}
              1    & = \int_{0}^{5} kv^2\ \dl v                              &
              k    & = \frac{3}{125}                                           \\
              F(x) & = \int_{0}^{x} f(v)\ \dl v                              &
                   & = \Bigg[ \frac{v^3}{125} \Bigg]_{0}^x = \frac{x^3}{125}
          \end{align}
          \begin{figure}[H]
              \centering
              \begin{tikzpicture}
                  \begin{axis}[width = 8cm,title = {Continuous PDF}, Ani,
                          grid = both]
                      \addplot[GraphSmooth, y_h, domain = 0:5] {3*x^2/125};
                  \end{axis}
              \end{tikzpicture}
              \begin{tikzpicture}
                  \begin{axis}[width = 8cm,title = {Continuous CDF}, Ani,
                          grid = both, xmin = -0.5, xmax = 5.5]
                      \addplot[GraphSmooth, y_p, domain = 0:5] {x^3/125};
                      \addplot[GraphSmooth, y_p, domain = 5:6] {1};
                      \addplot[GraphSmooth, y_p, domain = -1:0] {0};
                  \end{axis}
              \end{tikzpicture}
          \end{figure}

    \item Using the normalization condition,
          \begin{align}
              1    & = \int_{-2}^{2} k\ \dl v                           &
              k    & = \frac{1}{4}                                        \\
              F(x) & = \int_{-2}^{x} f(v)\ \dl v                        &
              F(x) & = \Bigg[ \frac{v}{4} \Bigg]_{-2}^x = \frac{x+2}{4}   \\
              P(0 \leq X \leq 2)
                   & = F(2) - F(0) = \frac{1}{2}
          \end{align}
          \begin{figure}[H]
              \centering
              \begin{tikzpicture}
                  \begin{axis}[width = 8cm,title = {Continuous PDF}, Ani,
                          grid = both]
                      \addplot[GraphSmooth, y_h, domain = -2:2] {1/4};
                  \end{axis}
              \end{tikzpicture}
              \begin{tikzpicture}
                  \begin{axis}[width = 8cm,title = {Continuous CDF}, Ani,
                          grid = both, xmin = -2.5, xmax = 2.5]
                      \addplot[GraphSmooth, y_p, domain = 2:3] {1};
                      \addplot[GraphSmooth, y_p, domain = -2:2] {(x+2)/4};
                      \addplot[GraphSmooth, y_p, domain = -3:-2] {0};
                  \end{axis}
              \end{tikzpicture}
          \end{figure}

    \item Finding the unknown constants,
          \begin{align}
              P(-c < X < c) & = F(c) - F(-c) = \frac{c+2-(-c+2)}{4} &
              0.95          & = \frac{c}{2}                           \\
              c             & = 1.9
          \end{align}
          For the second part of the problem,
          \begin{align}
              P(0 < X < d) & = F(d) - F(0) = \frac{d+2-(2)}{4} &
              0.95         & = \frac{d}{4}                       \\
              d            & = 3.8
          \end{align}
          Such a value $ d $ does not exist, since the sample space $ S $ is only the
          interval $ [-2,2] $.

    \item The plots are,
          \begin{figure}[H]
              \centering
              \begin{tikzpicture}
                  \begin{axis}[width = 8cm,title = {Discrete PDF}, Ani,
                          grid = both]
                      \addplot+[ycomb, mark options={black, mark size = 1.5pt},
                          y_h] coordinates {(-2,1/8) (-1,3/8) (1,3/8) (2,1/8)};
                  \end{axis}
              \end{tikzpicture}
              \begin{tikzpicture}
                  \begin{axis}[width = 8cm,title = {Discrete CDF}, Ani,
                          grid = both, xmin = -2.5, xmax = 2.5, ymin = -0.1, ymax = 1.1]
                      \addplot+[const plot, mark options={black, mark size = 1.5pt},
                          y_p] coordinates {(-3,0) (-2,1/8) (-1,4/8) (1,7/8)
                              (2,1) (3,1)};
                  \end{axis}
              \end{tikzpicture}
          \end{figure}
          No since the probabilities have already summed to unity.

    \item The probabilities are,
          \begin{align}
              P(X=0) & = \frac{4}{10} \cdot \frac{3}{9} = \frac{2}{15} &
              P(X=1) & = \frac{6}{10} \cdot \frac{4}{9}
              + \frac{4}{10} \cdot \frac{6}{9} = \frac{8}{15}            \\
              P(X=2) & = \frac{6}{10} \cdot \frac{5}{9} = \frac{1}{3}
          \end{align}
          Now, the other probabilities are simply sums of the above discrete PDF.
          \begin{align}
              P(1<X<2)        & = 0                                          &
              P(X \leq 1)     & = \frac{2}{15} + \frac{8}{15} = \frac{2}{3}    \\
              P(X \geq 1)     & = \frac{8}{15} + \frac{1}{3} = \frac{13}{15} &
              P(X > 1)        & = \frac{1}{3}                                  \\
              P(0.5 < X < 10) & = \frac{8}{15} + \frac{1}{3} = \frac{13}{15}
          \end{align}

    \item Using the normalization condition,
          \begin{align}
              1 & = k\ (1^3 + 2^3 + 3^3 + 4^3) & k & = \frac{1}{100} \\
          \end{align}
          \begin{figure}[H]
              \centering
              \begin{tikzpicture}
                  \begin{axis}[width = 8cm,title = {Discrete PDF}, Ani,
                          grid = both]
                      \addplot+[ycomb, mark options={black, mark size = 1.5pt},
                          y_h, samples at = {0,1,...,4}] {0.01*x^3};
                  \end{axis}
              \end{tikzpicture}
              \begin{tikzpicture}
                  \begin{axis}[width = 8cm,title = {Discrete CDF}, Ani,
                          grid = both, xmin = -0.5, xmax = 4.5]
                      \addplot+[const plot, mark options={black, mark size = 1.5pt},
                          y_p] coordinates {(-2,0) (0,0) (1,1/100) (2,9/100) (3,36/100)
                              (4,100/100) (6,1)};
                  \end{axis}
              \end{tikzpicture}
          \end{figure}

    \item Using the normalization condition,
          \begin{align}
              F(x)   & = \begin{dcases}
                             1 - e^{-3x} & \quad x > 0    \\
                             0           & \quad x \leq 0 \\
                         \end{dcases} &
              f(x)   & = F'(x)
              =  \begin{dcases}
                     3e^{-3x} & \quad x > 0    \\
                     0        & \quad x \leq 0 \\
                 \end{dcases}               \\
              F(x^*) & = 0.9                           &
              x^*    & = \frac{\ln(0.1)}{-3} = 0.767
          \end{align}
          \begin{figure}[H]
              \centering
              \begin{tikzpicture}
                  \begin{axis}[width = 8cm,title = {Continuous PDF}, Ani,
                          grid = both]
                      \addplot[GraphSmooth, y_h, domain = -1:0] {0};
                      \addplot[GraphSmooth, y_h, domain = 0:4] {3*e^(-3*x)};
                      \node[GraphNode, fill = black] at (axis cs:0,0){};
                      \node[GraphNode, draw = black, fill = white] at (axis cs:0,3){};
                  \end{axis}
              \end{tikzpicture}
              \begin{tikzpicture}
                  \begin{axis}[width = 8cm,title = {Continuous CDF}, Ani,
                          grid = both, xmin = -0.5, xmax = 3.5]
                      \addplot[GraphSmooth, y_p, domain = 0:4] {1-e^(-3*x)};
                      \addplot[GraphSmooth, y_p, domain = -1:0] {0};
                  \end{axis}
              \end{tikzpicture}
          \end{figure}

    \item Using the normalization condition
          \begin{align}
              \int_{0.9}^{1.1} f(v)\ \dl v & = 1                             &
              1                            & = \Bigg[\frac{kv^2}{2}
              \Bigg]_{0.9}^{1.1}                                               \\
              k                            & = \frac{2}{(1.1^2 - 0.9^2)} = 5   \\
              P(0.95<X<1.05)               & = \frac{5(1.05^2 - 0.95^2)}{2}
              = \frac{1}{2}
          \end{align}

    \item Using the normalization condition
          \begin{align}
              \int_{119.9}^{120.1} f(v)\ \dl v & = 1
              = \Bigg[ kv \Bigg]_{119.9}^{120.1}                            \\
              k                                & = 5                        \\
              P(X<119.91)                      & = 5(119.91 - 119.9) = 0.05 \\
              P(X>120.09)                      & = 5(120.1 - 120.09) = 0.05 \\
              N                                & = 500 * (0.05 + 0.05) = 50
          \end{align}

    \item Let $ q $ be the probability of a single bulb not failing,
          \begin{align}
              f(x)       & = \begin{dcases}
                                 6\ [0.25 - (x-1.5)^2] & \quad x \in [1,2]      \\
                                 0                     & \quad \text{otherwise}
                             \end{dcases} \\
              P(X > 1.5) & = \int_{1.5}^{2} f(v)\ \dl v
              = \Bigg[1.5v - 2(x-1.5)^3\Bigg]_{1.5}^2                       \\
                         & = 0.5 = q
          \end{align}
          Since each bulb operates independently of the others,
          \begin{align}
              P(E) & = q^3 = \frac{1}{8}
          \end{align}

    \item The PDF is,
          \begin{align}
              f(x)           & = F'(x) = \begin{dcases}
                                             0    & \quad x < 2       \\
                                             0.4x & \quad x \in [2,3) \\
                                             0    & \quad x \geq 3
                                         \end{dcases}          \\
              P(2.5 < X < 5) & = F(5) - F(2.5) = 1 - \frac{2.25}{5} = 0.55
          \end{align}
          \begin{figure}[H]
              \centering
              \begin{tikzpicture}
                  \begin{axis}[width = 8cm,title = {Continuous PDF}, Ani,
                          grid = both]
                      \addplot[GraphSmooth, y_h, domain = 2:3] {0.4*x};
                      \addplot[GraphSmooth, y_h, domain = 1:2] {0};
                      \addplot[GraphSmooth, y_h, domain = 3:4] {0};
                  \end{axis}
              \end{tikzpicture}
              \begin{tikzpicture}
                  \begin{axis}[width = 8cm,title = {Continuous CDF}, Ani,
                          grid = both, xmin = 0.5, xmax = 4.5]
                      \addplot[GraphSmooth, y_p, domain = 2:3] {(x^2-4)/5};
                      \addplot[GraphSmooth, y_p, domain = 0:2] {0};
                      \addplot[GraphSmooth, y_p, domain = 3:5] {1};
                  \end{axis}
              \end{tikzpicture}
          \end{figure}

    \item The PDF is,
          \begin{align}
              f(x) & = \begin{dcases}
                           1-\abs{x} & \quad \abs{x} < 1      \\
                           0         & \quad \text{otherwise}
                       \end{dcases}                   \\
              F(x) & =  \int_{-1}^{x} f(v)\ \dl v = \Bigg[v + \frac{v^2}{2}
              \Bigg]_{-1}^x = \frac{1}{2} + x + \frac{x^2}{2} \qquad x<0    \\
              F(x) & =  \int_{-1}^{x} f(v)\ \dl v = \frac{1}{2} +
              \Bigg[v - \frac{v^2}{2} \Bigg]_{0}^x
              = \frac{1}{2} + x - \frac{x^2}{2} \qquad x>0
          \end{align}
          For a total of 1000 cans,
          \begin{align}
              P(X>0)      & P(Y > 100) = F(1) - F(0) = 0.5        &
              n_1         & = 500                                   \\
              P(X < -0.5) & = P(Y < 99.5) = F(-0.5) = \frac{1}{8}   \\
              P(X < -1)   & = P(Y < 99) = F(-1) = 0
          \end{align}
          \begin{figure}[H]
              \centering
              \begin{tikzpicture}
                  \begin{axis}[width = 8cm,title = {Continuous PDF}, Ani,
                          grid = both, xmin = -1.5, xmax = 1.5]
                      \addplot[GraphSmooth, y_h, domain = -3:-1] {0};
                      \addplot[GraphSmooth, y_h, domain = -1:1] {1 - abs(x)};
                      \addplot[GraphSmooth, y_h, domain = 1:3] {0};
                  \end{axis}
              \end{tikzpicture}
              \begin{tikzpicture}
                  \begin{axis}[width = 8cm,title = {Continuous CDF}, Ani,
                          grid = both, xmin = -1.5, xmax = 1.5]
                      \addplot[GraphSmooth, y_p, domain = -3:-1] {0};
                      \addplot[GraphSmooth, y_p, domain = -1:0] {0.5 + x + x^2/2};
                      \addplot[GraphSmooth, y_p, domain = 0:1] {0.5 + x - x^2/2};
                      \addplot[GraphSmooth, y_p, domain = 1:3] {1};
                  \end{axis}
              \end{tikzpicture}
          \end{figure}

    \item Let $ X $ be the number of die rolls including the first six,
          \begin{align}
              P(X=1) & = \frac{1}{6}                                        &
              P(X=2) & = \frac{5}{6} \cdot \frac{1}{6}
              = \frac{5}{6^2}                                                 \\
              P(X=3) & = \frac{5^2}{6^2} \cdot \frac{1}{6}                  &
              P(X=4) & = \frac{5^3}{6^3} \cdot \frac{1}{6}                    \\
              P(X=k) & = \frac{1}{6} \cdot \left( \frac{5}{6} \right)^{k-1}
          \end{align}
          The sum of all probabilities is,
          \begin{align}
              \sum_{k=1}^{\infty} P(X=k) & = \frac{1}{6}\ \sum_{k=1}^{\infty}
              (5/6)^{k-1}                                                          \\
                                         & = \frac{1}{6} \cdot \frac{1}{(1 - 5/6)}
              = 1
          \end{align}

    \item The complements are,
          \begin{align}
              X & > b                                 & X \geq b \\
              X & < c                                 & X \leq c \\
              X & < b \quad \text{and} \quad X > c    &
              X & \leq b \quad \text{and} \quad X > c
          \end{align}
\end{enumerate}