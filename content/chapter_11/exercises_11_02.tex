\section{Arbitrary Period, Even and Odd Functions, Half-Range Expansions}

\begin{enumerate}
    \item Checking the functions,
          \begin{align}
              e^{-x}                & \neq e^{x}              &
                                    & \text{Neither}            \\
              e^{-\abs{-x}}         & = e^{-\abs{x}}          &
                                    & \color{y_h} \text{Even}   \\
              (-x)^3\ \cos(-nx)     & = -x^3\ \cos(nx)        &
                                    & \color{y_p} \text{Odd}    \\
              (-x)^2\ \tan(-\pi x)  & = -x^2\ \tan(\pi x)     &
                                    & \color{y_p} \text{Odd}    \\
              \sinh(-x) - \cosh(-x) & = -\sinh(x) - \cosh(x)  &
                                    & \text{Neither}
          \end{align}

    \item Checking the functions,
          \begin{align}
              \sin^2(-x)            & \sin^2(x)               &
                                    & \color{y_h} \text{Even}   \\
              \sin\Big((-x)^2\Big)  & = \sin(x^2)             &
                                    & \color{y_h} \text{Even}   \\
              \ln(-x)               & = \text{not defined}    &
                                    & \text{Neither}            \\
              \frac{-x}{(-x)^2 + 1} & = -\frac{x}{x^2 + 1}    &
                                    & \color{y_p} \text{Odd}    \\
              (-x)\ \cot(-x)        & = x\ \cot(x)            &
                                    & \color{y_h} \text{Even}
          \end{align}

    \item For even functions $ f, g $
          \begin{align}
              f(-x) + g(-x)     & = f(x) + g(x)           &
                                & \color{y_h} \text{Even}   \\
              f(-x) \cdot g(-x) & = f(x) \cdot g(x)       &
                                & \color{y_h} \text{Even}
          \end{align}

    \item For odd functions $ f, g $
          \begin{align}
              f(-x) + g(-x)     & = -f(x) - g(x) = - [f(x) + g(x)] &
                                & \color{y_p} \text{Odd}             \\
              f(-x) \cdot g(-x) & = f(x) \cdot g(x)                &
                                & \color{y_h} \text{Even}
          \end{align}

    \item For an odd function $ f $
          \begin{align}
              \abs{f(-x)} & = \abs{-f(x)} = \abs{x} &
                          & \color{y_h} \text{Even}
          \end{align}

    \item For odd function $ f $ and even function $ g $,
          \begin{align}
              f(-x) \cdot g(-x) & = -f(x) \cdot g(x)     &
                                & \color{y_p} \text{Odd}
          \end{align}

    \item Functions need to be both even and odd,
          \begin{align}
              f(-x) & = f(x)  &
              f(-x) & = -f(x)   \\
              f(x)  & = 0
          \end{align}

    \item The function is even with period $ p = 2L = 2 $.
          \begin{align}
              a_0 & = \frac{1}{2} \int_{-1}^{1} f(x)\ \dl x     \\
                  & = \frac{1}{2} \Bigg[\int_{-1}^{0} -x\ \dl x
              + \int_{0}^{1} x\ \dl x\Bigg]                     \\
                  & = \Bigg[ \frac{-x^2}{4} \Bigg]_{-1}^{0}
              + \Bigg[ \frac{x^2}{4} \Bigg]_{0}^{1} = \color{y_h} \frac{1}{2}
          \end{align}
          Calculating the Fourier cosine coefficients,
          \begin{align}
              a_n & = \int_{-1}^{1} f(x)\ \cos(nx)\ \dl x              \\
                  & = \int_{-1}^{0} -x \cos(n\pi x)\ \dl x
              + \int_{0}^{1} x \cos(n\pi x)\ \dl x                     \\
                  & = -\Bigg[ \frac{x\sin(n\pi x)}{n\pi}
                  + \frac{\cos(n\pi x)}{n^2\pi^2} \Bigg]_{-1}^{0}
              + \Bigg[ \frac{x\sin(n\pi x)}{n\pi}
              + \frac{\cos(n\pi x)}{n^2\pi^2} \Bigg]_{0}^{1}           \\
                  & = \color{y_p} \frac{2}{n^2\pi^2}\ [\cos(n\pi) - 1]
          \end{align}

    \item The function is odd with period $ p = 2L = 4 $.
          \begin{align}
              a_0 & = \color{y_h} 0 &
              a_n & = \color{y_p} 0
          \end{align}
          Calculating the Fourier sine coefficients,
          \begin{align}
              a_n & = \frac{1}{2} \int_{-2}^{2} f(x)\ \sin\left( \frac{n\pi x }{2}
              \right)\ \dl x                                                           \\
                  & = \frac{1}{2} \Bigg[\int_{-2}^{0} (-1) \sin\left( \frac{n\pi x}{2}
                  \right)\ \dl x
              + \int_{0}^{2} (1) \sin\left( \frac{n\pi x }{2} \right)\ \dl x \Bigg]    \\
                  & = \Bigg[ \frac{1}{n\pi}\ \cos\left( \frac{n\pi x}{2}
                  \right) \Bigg]_{-2}^{0}
              - \Bigg[ \frac{1}{n\pi}\ \cos\left( \frac{n\pi x}{2}
              \right) \Bigg]_{0}^{2}                                                   \\
                  & = \color{brown6} \frac{2}{n\pi}\ [1 - \cos(n\pi)]
          \end{align}

    \item The function is odd with period $ p = 2L = 8 $.
          \begin{align}
              a_0 & = \color{y_h} 0 &
              a_n & = \color{y_p} 0
          \end{align}
          Calculating the Fourier sine coefficients,
          \begin{align}
              a_n & = \frac{1}{4} \int_{-4}^{4} f(x)\ \sin\left( \frac{n\pi x}{4}
              \right)\ \dl x                                                           \\
                  & = \frac{1}{4} \Bigg[
                  \int_{-4}^{0} (-x-4) \sin\left( \frac{n\pi x}{4} \right)\ \dl x
              + \int_{0}^{4} (-x+4) \sin\left( \frac{n\pi x }{4} \right)\ \dl x \Bigg] \\
                  & = \frac{1}{4}\ \Bigg[ \frac{4(x+4)}{n\pi}\
                  \cos\left(\frac{n\pi x}{4}\right) - \frac{16}{n^2 \pi^2}
              \sin\left( \frac{n\pi x}{4} \right) \Bigg]_{-4}^{0}                      \\
                  & + \frac{1}{4}\ \Bigg[ \frac{4(x-4)}{n\pi}\
                  \cos\left(\frac{n\pi x}{4}\right) - \frac{16}{n^2 \pi^2}
              \sin\left( \frac{n\pi x}{4} \right) \Bigg]_{0}^{4}                       \\
                  & = \color{brown6} \frac{8}{n\pi}
          \end{align}

    \item The function is even with period $ p = 2L = 2 $.
          \begin{align}
              b_n & = \color{brown6} 0
          \end{align}
          Calculating the constant term,
          \begin{align}
              a_0 & = \frac{1}{2} \int_{-1}^{1} f(x)\ \dl x              \\
                  & = \frac{1}{2} \Bigg[ \int_{-1}^{1} x^2\ \dl x \Bigg]
              = \frac{1}{2} \Bigg[\frac{x^3}{3}\Bigg]_{-1}^{1}
              = \color{y_h} \frac{1}{3}
          \end{align}
          Calculating the Fourier cosine coefficients,
          \begin{align}
              a_n & = \int_{-1}^{1} x^2\ \cos (n\pi x)\ \dl x                    \\
                  & = \Bigg[\frac{x^2}{n\pi} \sin(n\pi x) + \frac{2x}{n^2 \pi^2}
              \cos(n\pi x) - \frac{2}{n^3 \pi^3} \sin(n\pi x)\Bigg]_{-1}^{1}     \\
                  & = \color{y_p} \frac{4}{n^2 \pi^2} \cos(n\pi)
          \end{align}

    \item The function is even with period $ p = 2L = 4 $.
          \begin{align}
              b_n & = \color{brown6} 0
          \end{align}
          Calculating the constant term,
          \begin{align}
              a_0 & = \frac{1}{4} \int_{-1}^{1} f(x)\ \dl x                     \\
                  & = \frac{1}{4} \Bigg[ \int_{-2}^{2} \left( 1 - \frac{x^2}{4}
                  \right)\ \dl x \Bigg]
              = \frac{1}{4} \Bigg[x - \frac{x^3}{12}\Bigg]_{-2}^{2}
              = \color{y_h} \frac{2}{3}
          \end{align}
          Calculating the Fourier cosine coefficients,
          \begin{align}
              a_n & = \frac{1}{2}\ \int_{-2}^{2} \left( 1 - \frac{x^2}{4} \right)
              \ \cos \left( \frac{n\pi x}{2} \right)\ \dl x                            \\
                  & = \frac{1}{2}\ \Bigg[ \frac{2}{n\pi} \sin\left( \frac{n\pi x}{2}
              \right) \Bigg]_{-2}^{2}                                                  \\
                  & - \frac{1}{8} \Bigg[ \frac{2x^2}{n\pi} \sin\left( \frac{n\pi x}{2}
                  \right) - \frac{16}{n^3\pi^3} \sin\left( \frac{n\pi x}{2} \right)
                  + \frac{8x}{n^2 \pi^2} \cos\left( \frac{n\pi x}{2} \right)
              \Bigg]_{-2}^{2}                                                          \\
                  & = \color{y_p} \frac{-4}{n^2 \pi^2} \cos(n\pi)
          \end{align}

    \item The function is even with period $ p = 2L = 1 $.
          \begin{align}
              a_0 & = \int_{-1/2}^{1/2} f(x)\ \dl x
              = \Bigg[\int_{0}^{1/2} x\ \dl x \Bigg]
              = \Bigg[ \frac{x^2}{2} \Bigg]_{0}^{1/2}
              = \color{y_h} \frac{1}{8}
          \end{align}
          Calculating the Fourier cosine coefficients,
          \begin{align}
              a_n & = 2\ \int_{-1/2}^{1/2} f(x)\ \cos(nx)\ \dl x
              = 2\ \int_{0}^{1/2} (x) \cos(2n\pi x)\ \dl x              \\
                  & = 2\ \Bigg[ \frac{x}{2n \pi} \sin(2n\pi x)
              + \frac{1}{4n^2 \pi^2} \cos(2n\pi x) \Bigg]_{0}^{1/2}     \\
                  & = \color{y_p} \frac{1}{2n^2\pi^2}\ [\cos(n\pi) - 1]
          \end{align}
          Calculating the Fourier sine coefficients,
          \begin{align}
              b_n & = 2\ \int_{-1/2}^{1/2} f(x)\ \sin(nx)\ \dl x
              = 2\ \int_{0}^{1/2} (x) \sin(2n\pi x)\ \dl x          \\
                  & = 2\ \Bigg[ -\frac{x}{2n \pi} \cos(2n\pi x)
              + \frac{1}{4n^2 \pi^2} \sin(2n\pi x) \Bigg]_{0}^{1/2} \\
                  & = \color{brown6} \frac{-1}{2n\pi}\ \cos(n\pi)
          \end{align}

    \item The function is even with period $ p = 2L = 1 $.
          \begin{align}
              a_0 & = \int_{-1/2}^{1/2} f(x)\ \dl x
              = \Bigg[\int_{-1/2}^{1/2} \cos(\pi x)\ \dl x \Bigg]
              = \Bigg[ \frac{\sin(\pi x)}{\pi} \Bigg]_{-1/2}^{1/2}
              = \color{y_h} \frac{2}{\pi}
          \end{align}
          Calculating the Fourier cosine coefficients,
          \begin{align}
              a_n & = 2\ \int_{-1/2}^{1/2} f(x)\ \cos(nx)\ \dl x
              = 2\ \int_{1/2}^{1/2} \cos(\pi x)\ \cos(2n\pi x)\ \dl x         \\
                  & = \int_{-1/2}^{1/2} \cos[(2n+1)\pi x]
              + \cos[(2n-1)\pi  x] \dl x                                      \\
                  & = \Bigg[ \frac{\sin[(2n+1)\pi x]}{(2n+1)\pi}
              + \frac{\sin[(2n-1)\pi x]}{(2n-1)\pi}  \Bigg]_{-1/2}^{1/2}      \\
                  & = \color{y_p} \frac{4 \cdot (-1)^n }{\pi(1 + 2n)(1 - 2n)}
          \end{align}

    \item The function is odd with period $ p = 2L = 2\pi $.
          \begin{align}
              a_0 & = \color{y_h} 0 &
              a_n & = \color{y_p} 0
          \end{align}
          Calculating the Fourier sine coefficients,
          \begin{align}
              a_n & = \frac{2}{\pi} \int_{0}^{\pi} f(x) \sin(nx)\ \dl x               \\
                  & = \frac{2}{\pi} \Bigg[ \int_{0}^{\pi/2} x \sin(nx)\ \dl x
              - \int_{\pi/2}^{\pi} (x-\pi) \sin(nx)\ \dl x \Bigg]                     \\
                  & = \frac{2}{\pi} \Bigg[ \frac{x\cos(nx)}{n} - \frac{\sin(nx)}{n^2}
                  \Bigg]_{\pi/2}^{0} + \frac{2}{\pi} \Bigg[
              \frac{(x-\pi)\cos(nx)}{n} - \frac{\sin(nx)}{n^2} \Bigg]_{\pi/2}^{\pi}   \\
                  & = \color{brown6} \frac{4}{\pi n^2}\ \sin(n\pi/2)
          \end{align}

    \item The function is odd with period $ p = 2L = 2 $.
          \begin{align}
              a_0 & = \color{y_h} 0 &
              a_n & = \color{y_p} 0
          \end{align}
          Calculating the Fourier sine coefficients,
          \begin{align}
              a_n & = 2 \int_{0}^{1} f(x) \sin(n\pi x)\ \dl x
              = 2 \int_{0}^{1} x^2 \sin(n\pi x)\ \dl x                                \\
                  & = 2\ \Bigg[ -\frac{x^2 \cos(n\pi x)}{n\pi}
                  + \frac{2x\sin(n\pi x)}{n^2 \pi^2}
              + \frac{2\cos(n\pi x)}{n^3 \pi^3}\Bigg]_0^1                             \\
                  & = \color{brown6} \frac{-2}{n\pi} \cos(n\pi) + \frac{4}{n^3 \pi^3}
              [\cos(n\pi) - 1]
          \end{align}

    \item The function is even with period $ p = 2L = 2 $.
          \begin{align}
              b_n & = \color{brown6} 0
          \end{align}
          Finding the constant term,
          \begin{align}
              a_0 & = \int_{0}^{1} (-x + 1) \dl x
              = \Bigg[ -\frac{(x-1)^2}{2} \Bigg]_0^1 = \color{y_h} \frac{1}{2}
          \end{align}
          Calculating the Fourier cosine coefficients,
          \begin{align}
              a_n & = 2\ \int_{0}^{1} f(x)\ \cos(nx)\ \dl x
              = 2\ \int_{0}^{1} (-x + 1)\ \cos(n\pi x)\ \dl x          \\
                  & = 2\ \Bigg[ \frac{(1 - x) \sin(n\pi x)}{n\pi}
              - \frac{\cos(n\pi x)}{n^2 \pi^2}  \Bigg]_{0}^{1}         \\
                  & = \color{y_p} \frac{2}{n^2 \pi^2} [1 - \cos(n\pi)]
          \end{align}

    \item Half-wave rectifier acting on $ v(x) = V_0\cos(100 \pi x) $
          \begin{figure}[H]
              \centering
              \begin{tikzpicture}[declare function = {
                              v(\x) = cos(100 * pi * \x);
                          }]
                  \begin{axis}[
                          grid = both, Ani, title = {Half-wave Rectifier},
                          domain = -0.02:0.02,
                          legend pos = south east]
                      \addplot[GraphSmooth, black, dashed] {v(x)};
                      \addplot[GraphSmooth, y_p] {0.5 * (v(x) + abs(v(x)))};
                  \end{axis}
              \end{tikzpicture}
          \end{figure}
          The function is even with period $ p = 2L = 0.02 $.
          \begin{align}
              b_n & = \color{brown6} 0
          \end{align}
          Finding the constant term,
          \begin{align}
              a_0 & = \frac{1}{L} \int_{0}^{L} f\left( \frac{n\pi x}{L} \right) \dl x
              = 100 \int_{0}^{0.005} V_0\ \cos\left( 100\pi x \right) \dl x           \\
                  & = \Bigg[ \frac{V_0\ \sin(100\pi x)}{\pi} \Bigg]_{0}^{0.005}
              = \color{y_h} \frac{V_0}{\pi}
          \end{align}
          Calculating the Fourier cosine coefficients,
          \begin{align}
              a_n & = 200\ \int_{0}^{L} f(x)\ \cos\left( \frac{n\pi x}{L} \right)
              \ \dl x                                                             \\
                  & = 200\ \int_{0}^{0.005} V_0\ \cos(100\pi x)\ \cos(100n\pi x)
              \ \dl x                                                             \\
                  & = 100V_0\ \int_{0}^{0.005} \Bigg[ \cos[(n+1)100\pi x]
              + \cos[(n-1)100\pi x] \Bigg] \dl x                                  \\
                  & = 100V_0\ \Bigg[ \frac{\sin[(n+1)100\pi\ x]}{(n+1)100\pi}
              + \frac{\sin[(n-1)100\pi\ x]}{(n-1)100\pi}  \Bigg]_{0}^{1/200}      \\
                  & = \frac{V_0}{\pi} \Bigg[ \frac{\cos(n\pi/2)}{(n+1)}
                  - \frac{\cos(n\pi/2)}{(n-1)} \Bigg]
              = \color{y_p} \frac{-2V_0}{\pi (n^2 - 1)}\ \cos\left(
              \frac{n\pi}{2} \right)
          \end{align}

    \item Fourier series expansions of powers of $ \cos^3 x$,
          \begin{align}
              a_0 & = \frac{1}{2\pi}\ \int_{-\pi}^{\pi} \cos^3 x\ \dl x         \\
                  & = \Bigg[ \sin x - \frac{\sin^3 x}{3}\Bigg]_{-\pi}^{\pi}
              = \color{y_h} 0                                                   \\
              a_n & = \frac{1}{\pi} \int_{-\pi}^{\pi} \cos^3 x\ \cos(nx)\ \dl x \\
                  & = \frac{1}{8\pi} \int_{-\pi}^{\pi}
              \Bigg[3\cos[(n+1)x] + 3\cos[(n-1)x]                               \\
                  & + \cos[(n+3)x] + \cos[(n-3)x]\Bigg]\ \dl x                  \\
                  & = \frac{1}{8\pi} \Bigg[ \frac{3\sin[(n+1)x]}{(n+1)}
              + \frac{3\sin[(n-1)x]}{(n-1)} + \frac{\sin[(n+3)x]}{(n+3)}        \\
                  & + \frac{\sin[(n-3)x]}{(n-3)} \Bigg]_{-\pi}^{\pi}
              = \color{y_p} 0 \quad \forall \quad n \notin \{1, 3\}             \\
              a_1 & = {\color{y_p} \frac{3}{4}} \qquad
              a_3 = {\color{y_p} \frac{1}{4}}
          \end{align}
          A similar Fourier series expansion can be given for $ \sin^3(x) $. \par
          The expansion for $ \cos^4(x) $ is
          \begin{align}
              \cos^4(x) & = \frac{[1 + \cos(2x)]^2}{4}
              = \frac{1 + \cos^2(2x) + 2\cos(2x)}{4}           \\
                        & = \frac{3 + 4\cos(2x) + \cos(4x)}{8}
          \end{align}
          This did not require explicit computation of the Fourier coefficients since
          it is a power of $ \cos^2(x) $.

    \item Using the Fourier series from Problem $ 11 $,
          \begin{align}
              x^2             & = \frac{1}{3} + \iser[n]{1}
              \frac{4\cos(n\pi)}{n^2 \pi^2} \ \cos(n\pi x)             \\
              1               & = \frac{1}{3} + \frac{4}{\pi^2} \Bigg[
              \frac{1}{1} + \frac{1}{4} + \frac{1}{9} + \dots \Bigg]   \\
              \frac{\pi^2}{6} & = \iser[n]{1}\ \frac{1}{n^2}
          \end{align}

    \item Plotting the first few partial sums for
          \begin{enumerate}
              \item Problem 8
                    \begin{figure}[H]
                        \centering
                        \begin{tikzpicture}[declare function = {
                                        a_0 = 1/2;
                                        fou_c(\n, \x) = ((-4)/ (pi^2 * \n^2))
                                        * cos(\n * pi * \x);
                                    }]
                            \begin{axis}[
                                    grid = both, Ani,
                                    title = {$ y = \abs{x} $},
                                    domain = -1:1,
                                    legend pos = south east]
                                \addplot[GraphSmooth, black, thin, forget plot,
                                ]{abs(x)};
                                \addplot[GraphSmooth, y_h]
                                {a_0 + fou_c(1, x) + fou_c(3, x) + fou_c(5, x)};
                                \addplot[GraphSmooth, y_p]
                                {a_0 + fou_c(1, x) + fou_c(3, x) + fou_c(5, x)
                                    + fou_c(7, x) + fou_c(9, x) + fou_c(11, x)
                                    + fou_c(13, x) + fou_c(15, x)};
                                \addlegendentry{$ n = 5 $}
                                \addlegendentry{$ n = 15 $}
                            \end{axis}
                        \end{tikzpicture}
                    \end{figure}

              \item Problem 9
                    \begin{figure}[H]
                        \centering
                        \begin{tikzpicture}[declare function = {
                                        fou_s(\n, \x) = sin(\n * pi * 0.5 * \x)
                                        * 4 / (\n * pi);
                                    }]
                            \begin{axis}[
                                    grid = both, Ani, title = {$ f(x) $ piecewise},
                                    domain = -2:2,
                                    legend pos = south east]
                                \addplot[GraphSmooth, black, thin, forget plot,
                                    domain = -2:0] {-1};
                                \addplot[GraphSmooth, black, thin, forget plot,
                                    domain = 0:2] {1};
                                \addplot[GraphSmooth, y_h] {fou_s(1, x)
                                    + fou_s(3, x) + fou_s(5, x)};
                                \addplot[GraphSmooth, y_p] {fou_s(1, x)
                                    + fou_s(3, x) + fou_s(5, x) + fou_s(7, x)
                                    + fou_s(9, x) + fou_s(11, x) + fou_s(13, x)
                                    + fou_s(15, x) + fou_s(17, x) + fou_s(19, x)};
                                \addlegendentry{$ n = 5 $}
                                \addlegendentry{$ n = 19 $}
                            \end{axis}
                        \end{tikzpicture}
                    \end{figure}

              \item Problem 10
                    \begin{figure}[H]
                        \centering
                        \begin{tikzpicture}[declare function = {
                                        fou_s(\n, \x) = sin(\n * pi * 0.25 * \x)
                                        * 8 / (\n * pi);
                                    }]
                            \begin{axis}[
                                    grid = both, Ani, title = {$ f(x) $ piecewise},
                                    domain = -4:4, legend pos = south east]
                                \addplot[GraphSmooth, black, thin, forget plot,
                                    domain = -4:0] {-x - 4};
                                \addplot[GraphSmooth, black, thin, forget plot,
                                    domain = 0:4] {-x + 4};
                                \addplot[GraphSmooth, y_h] {fou_s(1, x)
                                    + fou_s(2, x) + fou_s(3, x) + fou_s(4, x)
                                    + fou_s(5, x)};
                                \addplot[GraphSmooth, y_p] {fou_s(1, x)
                                    + fou_s(2, x) + fou_s(3, x) + fou_s(4, x)
                                    + fou_s(5, x) + fou_s(6, x) + fou_s(7, x)
                                    + fou_s(8, x) + fou_s(9, x) + fou_s(10, x)
                                    + fou_s(11, x) + fou_s(12, x) + fou_s(13, x)
                                    + fou_s(14, x) + fou_s(15, x) + fou_s(16, x)
                                    + fou_s(17, x) + fou_s(18, x) + fou_s(19, x)};
                                \addlegendentry{$ n = 5 $}
                                \addlegendentry{$ n = 19 $}
                            \end{axis}
                        \end{tikzpicture}
                    \end{figure}

              \item Problem 11
                    \begin{figure}[H]
                        \centering
                        \begin{tikzpicture}[declare function = {
                                        a_0 = 1 / 3;
                                        fou_c(\n, \x) = (4/ (\n^2 * pi^2))*cos(\n * pi)
                                        *cos(\n * pi * \x);
                                    }]
                            \begin{axis}[
                                    grid = both, Ani, title = {$ f(x) = x^2$},
                                    domain = -1:1,
                                    legend pos = south east]
                                \addplot[GraphSmooth, black, dashed, forget plot] {x^2};
                                \addplot[GraphSmooth, y_h] {a_0 + fou_c(1, x)
                                    + fou_c(2, x) + fou_c(3, x) + fou_c(4, x)
                                    + fou_c(5, x)};
                                \addplot[GraphSmooth, brown6] {a_0 + fou_c(1, x)
                                    + fou_c(2, x) + fou_c(3, x) + fou_c(4, x)
                                    + fou_c(5, x) + fou_c(6, x) + fou_c(7, x)
                                    + fou_c(8, x) + fou_c(9, x) + fou_c(10, x)
                                    + fou_c(11, x) + fou_c(12, x) + fou_c(13, x)
                                    + fou_c(14, x) + fou_c(15, x)};
                                \addlegendentry{$ n = 5 $}
                                \addlegendentry{$ n = 15 $}
                            \end{axis}
                        \end{tikzpicture}
                    \end{figure}
          \end{enumerate}

    \item Using the linearity of Fourier transforms,
          \begin{align}
              f(x) & = \abs{x}                                       \qquad
              g(x)                = 1 - \abs{x} = 1 - f(x)                  \\
              F(x) & = \frac{1}{2} + \iser[n]{1} \frac{2}{n^2 \pi^2}
              \ [\cos(n\pi) - 1]                                            \\
              G(x) & = \frac{1}{2} - \iser[n]{1} \frac{2}{n^2 \pi^2}
              \ [\cos(n\pi) - 1]
          \end{align}
          The inverse mapping $ g \rightarrow f $ is also as simple.

    \item The odd expansion of the given function is, with $  p = 2L = 8 $
          \begin{align}
              f(x) & = \begin{dcases}
                           -1 & x \in [-4, 0] \\
                           1  & x \in [0, 4]
                       \end{dcases}    \\
              a_0  & = \color{y_h} 0       &
              a_n  & = \color{y_p} 0
          \end{align}
          Calculating the Fourier sine coefficients,
          \begin{align}
              a_n & = \frac{1}{4} \int_{-4}^{4} f(x)\ \sin\left( \frac{n\pi x}{4}
              \right)\ \dl x
              = \frac{1}{2} \int_{0}^{4} (1) \sin\left( \frac{n\pi x }{4}
              \right)\ \dl x                                                      \\
                  & = \Bigg[ \frac{-2}{n\pi}\ \cos\left( \frac{n\pi x}{4}
                  \right) \Bigg]_{0}^{4}
              = \color{brown6} \frac{2}{n\pi}\ [1 - \cos(n\pi)]
          \end{align}
          The even expansion of the given function is,
          \begin{align}
              a_0 & = 1 & a_n = b_n & = 0
          \end{align}

    \item The odd expansion of the given function is, with $  p = 2L = 8 $
          \begin{align}
              f(x) & = \begin{dcases}
                           -1 & x \in [-4, -2] \\
                           0  & x \in [-2, 2]  \\
                           1  & x \in [2, 4]
                       \end{dcases}    \\
              a_0  & = \color{y_h} 0        &
              a_n  & = \color{y_p} 0
          \end{align}
          Calculating the Fourier sine coefficients,
          \begin{align}
              b_n & = \frac{1}{4} \int_{-4}^{4} f(x)\ \sin\left( \frac{n\pi x}{4}
              \right)\ \dl x
              = \frac{1}{2} \int_{2}^{4} (1) \sin\left( \frac{n\pi x }{4}
              \right)\ \dl x                                                      \\
                  & = \Bigg[ \frac{-2}{n\pi}\ \cos\left( \frac{n\pi x}{4}
                  \right) \Bigg]_{2}^{4}
              = \color{brown6} \frac{2}{n\pi}\ [\cos(n\pi/2) - \cos(n\pi)]
          \end{align}
          The even expansion of the given function is, with $  p = 2L = 8 $
          \begin{align}
              f(x) & = \begin{dcases}
                           1 & x \in [-4, -2] \\
                           0 & x \in [-2, 2]  \\
                           1 & x \in [2, 4]
                       \end{dcases} \\
              b_n  & = \color{brown6} 0
          \end{align}
          Calculating the Fourier cosine coefficients,
          \begin{align}
              a_0 & = \frac{1}{8}\ \int_{-4}^{4} f(x)\ \dl x
              = \frac{1}{8}\ \Bigg[\int_{-4}^{-2} (1)\ \dl x
              + \int_{2}^{4} (1)\ \dl x \Bigg]                                    \\
                  & = \color{y_h} \frac{1}{2}                                     \\
              a_n & = \frac{1}{4} \int_{-4}^{4} f(x)\ \cos\left( \frac{n\pi x}{4}
              \right)\ \dl x
              = \frac{1}{2} \int_{2}^{4} (1) \cos\left( \frac{n\pi x }{4}
              \right)\ \dl x                                                      \\
                  & = \Bigg[ \frac{2}{n\pi}\ \sin\left( \frac{n\pi x}{4}
                  \right) \Bigg]_{2}^{4}
              = \color{y_p} \frac{-2}{n\pi}\ \sin(n\pi/2)
          \end{align}

    \item The odd expansion of the given function is, with $  p = 2L = 2\pi $
          \begin{align}
              f(x) & = \begin{dcases}
                           -x - \pi & x \in [-\pi, 0] \\
                           -x + \pi & x \in [0, \pi]
                       \end{dcases}    \\
              a_0  & = \color{y_h} 0               &
              a_n  & = \color{y_p} 0
          \end{align}
          Calculating the Fourier sine coefficients,
          \begin{align}
              b_n & = \frac{1}{\pi} \int_{-\pi}^{\pi} f(x)\ \sin(nx)\ \dl x
              = \frac{2}{\pi} \int_{0}^{\pi} (\pi - x) \sin(nx)\ \dl x      \\
                  & = \frac{2}{\pi}\Bigg[ \frac{(x-\pi)\cos(nx)}{n}
                  - \frac{\sin(nx)}{n^2}\Bigg]_{0}^{\pi}
              = \color{brown6} \frac{-2}{n}
          \end{align}
          The even expansion of the given function is, with $  p = 2L = 2\pi $
          \begin{align}
              f(x) & = \pi - \abs{x}    \\
              b_n  & = \color{brown6} 0
          \end{align}
          Calculating the Fourier cosine coefficients,
          \begin{align}
              a_0 & = \frac{1}{2\pi}\ \int_{-\pi}^{\pi} f(x)\ \dl x
              = \frac{1}{\pi}\ \int_{0}^{\pi} (\pi - x)\ \dl x                   \\
                  & = \frac{1}{\pi} \Bigg[ \frac{-(x-\pi)^2}{2} \Bigg]_{0}^{\pi}
              = \color{y_h} \frac{\pi}{2}                                        \\
              a_n & = \frac{1}{\pi} \int_{-\pi}^{\pi} f(x)\ \cos(nx)\ \dl x
              = \frac{2}{\pi} \int_{0}^{\pi} (\pi - x) \cos(nx)\ \dl x           \\
                  & = \frac{2}{\pi}\Bigg[ \frac{(\pi - x)\sin(nx)}{n}
                  - \frac{\cos(nx)}{n^2}\Bigg]_{0}^{\pi}
              = \color{y_p} \frac{2}{\pi n^2}\ [1 - \cos(n\pi)]
          \end{align}

    \item The odd expansion of the given function is, with $  p = 2L = 2\pi $
          \begin{align}
              f(x) & = \begin{dcases}
                           -\pi/2 & x \in [-\pi, -\pi/2]  \\
                           x      & x \in [-\pi/2, \pi/2] \\
                           \pi/2  & x \in [\pi/2, \pi]
                       \end{dcases}    \\
              a_0  & = \color{y_h} 0                   &
              a_n  & = \color{y_p} 0
          \end{align}
          Calculating the Fourier sine coefficients,
          \begin{align}
              b_n & = \frac{1}{\pi} \int_{-\pi}^{\pi} f(x)\ \sin(nx)\ \dl x \\
                  & = \frac{2}{\pi} \int_{0}^{\pi/2} (x) \sin(nx)\ \dl x
              + \int_{\pi/2}^{\pi} (\pi/2) \sin(nx)\ \dl x                  \\
                  & = \frac{2}{\pi}\Bigg[ -\frac{(x)\cos(nx)}{n}
                  + \frac{\sin(nx)}{n^2}\Bigg]_{0}^{\pi/2} +
              \Bigg[ -\frac{\cos(nx)}{n} \Bigg]_{\pi/2}^{\pi}               \\
                  & = \color{brown6} \frac{2\sin(n\pi/2)}{n\pi^2}
              - \frac{\cos(n\pi)}{n}
          \end{align}
          The even expansion of the given function is, with $  p = 2L = 2\pi $
          \begin{align}
              f(x) & = \begin{dcases}
                           \pi/2   & x \in [-\pi, -\pi/2]  \\
                           \abs{x} & x \in [-\pi/2, \pi/2] \\
                           \pi/2   & x \in [\pi/2, \pi]
                       \end{dcases} \\
              b_n  & = \color{brown6} 0
          \end{align}
          Calculating the Fourier cosine coefficients,
          \begin{align}
              a_0 & = \frac{1}{2\pi}\ \int_{-\pi}^{\pi} f(x)\ \dl x         \\
                  & = \frac{1}{\pi}\ \Bigg[ \int_{0}^{\pi/2} (x)\ \dl x
              + \int_{\pi/2}^{\pi} (\pi/2)\ \dl x \Bigg]                    \\
                  & =  \Bigg[ \frac{x^2}{2\pi} \Bigg]_{0}^{\pi/2}
              + \Bigg[ \frac{x}{2} \Bigg]_{\pi/2}^{\pi}
              = \color{y_h} \frac{3\pi}{8}                                  \\
              a_n & = \frac{1}{\pi} \int_{-\pi}^{\pi} f(x)\ \cos(nx)\ \dl x \\
                  & = \frac{2}{\pi} \int_{0}^{\pi/2} (x) \cos(nx)\ \dl x
              + \int_{\pi/2}^{\pi} (1) \cos(nx)\ \dl x                      \\
                  & = \frac{2}{\pi}\Bigg[ \frac{x \sin(nx)}{n}
                  + \frac{\cos(nx)}{n^2}\Bigg]_{0}^{\pi/2} +
              \Bigg[ \frac{\sin(nx)}{n} \Bigg]_{\pi/2}^{\pi}                \\
                  & = \color{y_p} \frac{2}{\pi n^2}\ [\cos(n\pi/2) - 1]
          \end{align}

    \item The odd expansion of the given function is, with $  p = 2L = 2\pi $
          \begin{align}
              f(x) & = \begin{dcases}
                           -\pi - x & x \in [-\pi, -\pi/2] \\
                           -\pi/2   & x \in [-\pi/2, 0]    \\
                           \pi/2    & x \in [0, \pi/2]     \\
                           \pi - x  & x \in [\pi/2, \pi]
                       \end{dcases}                             \\
              a_0  & = {\color{y_h} 0} \qquad\qquad
              a_n = {\color{y_p} 0}                                                \\
              b_n  & = \frac{2}{\pi} \Bigg[ \int_{0}^{\pi/2} (\pi/2)\sin(nx) \dl x
              + \int_{\pi/2}^{\pi} (\pi - x) \sin(nx) \dl x \Bigg]                 \\
                   & =  \Bigg[ \frac{-\cos(nx)}{n} \Bigg]_{0}^{\pi/2}
              + \frac{2}{\pi}\ \Bigg[ \frac{(x - \pi)\cos(nx)}{n}
              - \frac{\sin(nx)}{n^2} \Bigg]_{\pi/2}^{\pi}                          \\
                   & = \color{brown6} \frac{1}{n} + \frac{2\sin(n\pi/2)}{\pi n^2}
          \end{align}
          The even expansion of the given function is, with $  p = 2L = 2\pi $
          \begin{align}
              f(x) & = \begin{dcases}
                           x + \pi & x \in [-\pi, -\pi/2]  \\
                           \pi/2   & x \in [-\pi/2, \pi/2] \\
                           \pi - x & x \in [\pi/2, \pi]
                       \end{dcases} \\
              b_n  & = \color{brown6} 0
          \end{align}
          Calculating the Fourier cosine coefficients,
          \begin{align}
              a_0 & = \frac{1}{2\pi}\ \int_{-\pi}^{\pi} f(x)\ \dl x               \\
                  & = \frac{1}{\pi}\ \Bigg[ \int_{0}^{\pi/2} (\pi/2)\ \dl x
              + \int_{\pi/2}^{\pi} (\pi - x)\ \dl x \Bigg]                        \\
                  & =  \Bigg[ \frac{-(\pi-x)^2}{2\pi} \Bigg]_{\pi/2}^{\pi}
              + \Bigg[ \frac{x}{2} \Bigg]_{0}^{\pi/2}
              = \color{y_h} \frac{3\pi}{8}                                        \\
              a_n & = \frac{1}{\pi} \int_{-\pi}^{\pi} f(x)\ \cos(nx)\ \dl x       \\
                  & = \frac{2}{\pi} \int_{0}^{\pi/2} (\pi/2) \cos(nx)\ \dl x
              + \frac{2}{\pi}\ \int_{\pi/2}^{\pi} (\pi-x) \cos(nx)\ \dl x         \\
                  & = \frac{2}{\pi}\Bigg[ \frac{(\pi - x) \sin(nx)}{n}
                  - \frac{\cos(nx)}{n^2}\Bigg]_{\pi/2}^{\pi} +
              \Bigg[ \frac{\sin(nx)}{n} \Bigg]_{0}^{\pi/2}                        \\
                  & = \color{y_p} \frac{-2}{\pi n^2}\ [\cos(n\pi) - \cos(n\pi/2)]
          \end{align}

    \item The odd expansion of the given function is, with $  p = 2L $
          \begin{align}
              f(x) & = x                                                 \\
              a_0  & = {\color{y_h} 0} \qquad\qquad
              a_n = {\color{y_p} 0}                                      \\
              b_n  & = \frac{2}{L} \int_{0}^{L} (x)\sin\left(
              \frac{n\pi x}{L} \right) \dl x                             \\
                   & =  \frac{2}{L}\ \Bigg[ -\frac{xL}{n\pi}\ \cos\left(
                  \frac{n\pi x}{L} \right)
                  + \frac{L^2}{n^2\pi^2}\ \sin\left( \frac{n\pi x}{L} \right)
              \Bigg]_{0}^{L}                                             \\
                   & = \color{brown6} \frac{-2L}{n\pi}\ \cos(n\pi)
          \end{align}
          The even expansion of the given function is, with $  p = 2L $
          \begin{align}
              f(x) & = \abs{x}          &
              b_n  & = \color{brown6} 0
          \end{align}
          Calculating the Fourier cosine coefficients,
          \begin{align}
              a_0 & = \frac{1}{2L}\ \int_{-L}^{L} f(x)\ \dl x
              = \frac{1}{L}\ \int_{0}^{L} (x)\ \dl x                              \\
                  & =  \Bigg[ \frac{x^2}{2L} \Bigg]_{0}^{L}
              = \color{y_h} \frac{L}{2}                                           \\
              a_n & = \frac{1}{L} \int_{-L}^{L} f(x)\ \cos\left( \frac{n\pi x}{L}
              \right)\ \dl x                                                      \\
                  & = \frac{2}{L} \int_{0}^{L} (x) \cos\left( \frac{n\pi x}{L}
              \right)\ \dl x                                                      \\
                  & = \frac{2}{L}\Bigg[ \frac{xL }{n\pi}\ \sin\left(
                  \frac{n\pi x}{L} \right)
                  + \frac{L^2}{n^2\pi^2}\ \cos\left( \frac{n\pi x}{L}
              \right)\Bigg]_{0}^{L}                                               \\
                  & = \color{y_p} \frac{2L}{\pi^2 n^2}\ [\cos(n\pi) - 1]
          \end{align}

    \item The odd expansion of the given function is, with $  p = 2L = 2\pi $
          \begin{align}
              f(x) & = \sin(x)                                   \\
              a_0  & = {\color{y_h} 0} \qquad\qquad
              a_n = {\color{y_p} 0}                              \\
              b_n  & = \color{brown6} \begin{dcases}
                                          1 & \quad n = 1            \\
                                          0 & \quad \text{otherwise} \\
                                      \end{dcases}
          \end{align}
          The even expansion of the given function is, with $  p = 2L = 2\pi $
          \begin{align}
              f(x) & = \abs{\sin(x)}    &
              b_n  & = \color{brown6} 0
          \end{align}
          Calculating the Fourier cosine coefficients,
          \begin{align}
              a_0 & = \frac{1}{2\pi}\ \int_{-\pi}^{\pi} f(x)\ \dl x
              = \frac{1}{\pi}\ \int_{0}^{\pi} (\sin x)\ \dl x               \\
                  & =  \frac{1}{\pi}\ \Bigg[ -\cos x \Bigg]_{0}^{\pi}
              = \color{y_h} \frac{2}{\pi}                                   \\
              a_n & = \frac{1}{\pi} \int_{-\pi}^{\pi} f(x)\ \cos(nx)\ \dl x \\
                  & = \frac{2}{\pi} \int_{0}^{\pi} (\sin x) \cos(nx)\ \dl x \\
                  & = \frac{-1}{\pi}\Bigg[ \frac{\cos[(1+n)x]}{1+n}
              + \frac{\cos[(1-n)x]}{1-n} \Bigg]_{0}^{\pi}                   \\
                  & = \color{y_p}
              \begin{dcases}
                  \frac{-2}{\pi (n^2 - 1)}\ \Big[1 - \cos[(n+1)\pi]\Big] &
                  \quad n \geq 2                                           \\
                  0                                                      &
                  \quad n = 1
              \end{dcases}
          \end{align}

    \item The odd expansion of the given function is, with $  p = 2L $
          \begin{align}
              f(x) & = -g(x + \pi)                                                     \\
              a_0  & = {\color{y_h} 0} \qquad\qquad
              a_n = {\color{y_p} 0}                                                    \\
              f(x) & = -\iser[n]{1}\ \Bigg[\frac{1}{n} + \frac{2\sin(n\pi/2)}{\pi n^2}
              \Bigg] \ \sin(nx + n\pi)                                                 \\
                   & = \iser[n]{1}\ \Bigg[ {\color{brown6} \frac{-\cos(n\pi)}{n}
                          + \frac{2\sin(n\pi/2)}{\pi n^2} }\Bigg]\ \sin(nx)
          \end{align}
          The even expansion of the given function is, with $  p = 2L $
          \begin{align}
              f(x) & = g(x + \pi)       &
              b_n  & = \color{brown6} 0
          \end{align}
          Calculating the Fourier cosine coefficients,
          \begin{align}
              a_0 & = \color{y_h} \frac{3\pi}{8}                           \\
              a_n & = \iser[n]{1}\ \Bigg[ \frac{2}{\pi n^2}\ [\cos(n\pi/2)
              - \cos(n\pi)]\Bigg]\ \cos(nx + n\pi)                         \\
                  & = \iser[n]{1}\ \Bigg[ {\color{y_p} \frac{2}{\pi n^2}
              \ [\cos(n\pi/2) - 1]}\Bigg]\ \cos(nx)                        \\
          \end{align}


\end{enumerate}