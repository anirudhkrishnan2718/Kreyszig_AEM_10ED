\chapter{Fourier Analysis}

\section{Fourier Series}

\begin{description}
    \item[Periodic function] A function defined on the real line, except possibly at a
        finite number of points such that,
        \begin{align}
            f(x + p)  & = f(x) & \forall\ x \in \mathcal{R} \\
            f(x + np) & =f(x)  & \forall\ n \in \mathcal{I}
        \end{align}
        for some positive real $ p $, which is called the period. \par
        The smallest positive period is called the fundamental period.

    \item[Trigonometric system] A family of periodic functions all having period $ 2\pi $
        which are the simplest basis used to represent all periodic functions.
        \begin{align}
            T & = \{1,\ \cos x,\ \sin x,\ \cos(2x),\ \sin(2x), \dots,\ \cos(nx),
            \ \sin(nx), \dots\}
        \end{align}

    \item[Fourier Series] A periodic function si represented as a linear combination of
        the above basis with each member assigned a coefficient (called the Fourier
        coefficient).
        \begin{align}
            f(x) & = a_0 + \iser[n]{0} a_n \cos(nx) + b_n \sin(nx)          \\
            a_0  & = \frac{1}{2\pi}\ \int_{-\pi}^{\pi} f(x)\ \dl x          \\
            a_n  & = \frac{1}{\pi}\ \int_{-\pi}^{\pi} f(x)\ \cos(nx)\ \dl x \\
            b_n  & = \frac{1}{\pi}\ \int_{-\pi}^{\pi} f(x)\ \sin(nx)\ \dl x
        \end{align}

    \item[Orthogonality of a trigonometric system] If integral over one period (taken to
        be $ -\pi $ to $ \pi $ by convention) of the product of any two members of the
        basis is zero, then the basis is orthogonal.
        \begin{align}
            \int_{-\pi}^{\pi}\ \cos(nx)\ \cos(mx)\ \dl x & = 0             &
            n                                            & \neq m            \\
            \int_{-\pi}^{\pi}\ \sin(nx)\ \sin(mx)\ \dl x & = 0             &
            n                                            & \neq m            \\
            \int_{-\pi}^{\pi}\ \cos(nx)\ \sin(mx)\ \dl x & = 0             &
            n,m                                          & \in \mathcal{I}
        \end{align}
        The Euler formulas for the Fourier coefficients are derived from the application
        of the orthogonality condition to the Fouier series definition.

    \item[Convergence of Fourier series] Let $ f(x) $ be periodic with period $ 2\pi $
        and be piecewise continuous in $ [-\pi, \pi] $. Also, let it have both
        left-handed and right-handed derivatives defined everywhere in this interval.
        \par Then, its Fourier series converges and is equal to $ f(x) $ at all points
        except at the finitely many points of discontinuity of $ f(x) $. \par
        At such points, the series converges to the average of the left and right-handed
        limits of $ f(x) $.
\end{description}

\section{Arbitrary Period, Even and Odd Functions, Half-Range Expansions}

\begin{description}
    \item[Functions with arbitrary period] Instead of the standard period $ p = 2\pi $,
        generalizing to $ p = 2L $ simply involves a change of variable
        \begin{align}
            2\pi \rightarrow 2L & \implies x \rightarrow \frac{\pi x}{L}    \\
            f(x)                & \rightarrow a_0 + \iser[n]{0} \Bigg[
                a_n\ \cos\left( \frac{n\pi x}{L} \right) +
            b_n\ \sin\left( \frac{n\pi x}{L} \right) \Bigg]                 \\
            a_0                 & = \frac{1}{2L}\ \int_{-L}^{L} f(x)\ \dl x \\
            a_n                 & = \frac{1}{L}\ \int_{-L}^{L}
            f(x)\ \cos\left( \frac{n\pi x}{L} \right) \dl x                 \\
            b_n                 & = \frac{1}{L}\ \int_{-L}^{L}
            f(x)\ \sin\left( \frac{n\pi x}{L} \right) \dl x
        \end{align}

    \item[Even and Odd functions] Even functions and odd functions can be
        represented just by a Fourier cosine and sine series respectively.
        \begin{align}
            f(-x) = f(x)  & \implies b_n = 0 \qquad \text{even}      \\
            f(-x) = -f(x) & \implies a_0 = a_n = 0 \qquad \text{odd}
        \end{align}

    \item[Lineraity] The Fourier series is linear under addition and scalar
        multiplication.
        \begin{align}
            f(x) + g(x) & \rightarrow F(x) + G(x) \\
            c\ f(x)     & \rightarrow c\ F(x)
        \end{align}
        Where the uppercase is the fourier series expansion of the lowercase function.

    \item[Half-range expansions] Extending the function specified in the domain
        $ [0, L] $ into the domain $ [-L, L] $ as either an even or odd function in order
        to simplify the computation of Fourier coefficients.
\end{description}

\section{Forced Oscillations}

\begin{description}
    \item[Standard form ODE] A second order linear ODE is very common in physical systems
        undergoing forced damped oscillations.
        \begin{align}
            my'' + cy' + ky = r(t)
        \end{align}
        Here, the output $ y(t) $ is the solution to the ODE, corresponding to the input
        $ r(t) $. The constant coefficients $ m, c, k $ characterize the system.

    \item[Solution to ODE] Since the input can be represented as a Fourier series,
        the output can also be decomposed into a sum of outputs corresponding to each
        input term. \par
        The amplitude of each of the output terms happens to be a function of the input
        frequency and typically, one or two frequencies dominate the output in most real
        world systems.
\end{description}

\section{Approximation by Trigonometric Polynomials}

\begin{description}
    \item[Trigonometric polynomial] An approximation of a function using a series of
        trigonometric functions. The fourier series expansion happens to be an example.
        \begin{align}
            f(x) & = A_0 + \iser[n]{1} \Big[ A_n \cos(nx) + B_n \sin(nx) \Big]
        \end{align}
        Here, $ N $ is called the order.

    \item[Square Error] The error in this approximation is
        \begin{align}
            E & = \int_{-\pi}^{\pi} (f - F)^2\ \dl x
        \end{align}
        This error is a measure of the agreement between the approximation $ f $ and
        the actual function $ F $ over the entire period of the trigonometric function.

    \item[Minimum square error] The squared error of a trigonometric polynomial
        $ F $ with fixed order $ N $ is smallest in the domainm $[-\pi, \pi] $ if the
        coefficients are the Fourier coefficients $ A_n = a_n, B_n = b_n $
        \begin{align}
            E^* & = \int_{-\pi}^{\pi} f^2\ \dl x  - \pi\ \Bigg[ 2a_0^2
                + \sum_{n=1}^{N} (a_n^2 + b_n^2) \Bigg]
        \end{align}
        The above squared error can only decrease with increasing $ N $.

    \item[Bessel's Inequality] Given a function $ f $ and fourier coefficients $ a_0,
            \{a_n\}, \{b_n\} $, the condition on minimized squared error gives,
        \begin{align}
            2a_0^2 + \iser[n]{1} (a_n^2 + b_n^2) & \leq \frac{1}{\pi}\ \int_{-\pi}^{\pi}
            \ f(x)^2\ \dl x
        \end{align}

    \item[Parseval's identity] The integral of the square of a function over its
        standardized period $ [-\pi, \pi] $ is equal to the sum of the squares of its
        Fourier coefficients. (rough analog of the Pythagoras theorem)
        \begin{align}
            2a_0^2 + \iser[n]{1} (a_n^2 + b_n^2) & = \frac{1}{\pi}\ \int_{-\pi}^{\pi}
            \ f(x)^2\ \dl x
        \end{align}
\end{description}

\section{Sturm-Liouville Problems, Orthogonal Functions}

\begin{description}
    \item[Sturm-Liouville problem] A second order ODE, along with boundary conditions
        of the form,
        \begin{align}
            \diff{}{x}\Bigg[p(x)\ \diff yx\Bigg] +
            \Bigg[ q(x) + \lambda r(x) \Bigg]y & = 0                         \\
            k_1 y + k_2 y'                     & = 0 \qquad \text{at}\ x = a \\
            l_1 y + l_2 y'                     & = 0 \qquad \text{at}\ x = b
        \end{align}
        for some interval $ x \in [a,b] $. $ \lambda $ is a paramter and the two
        $ k, l $ are real constants. \par
        If $ p,q,r,p' $ are real valued and continuous in the interval $ [a,b] $ and
        $ r $ is the same sign throughout, then all eigenvalues $ \lambda $ of the
        Sturm-Liouville equation are real.

    \item[Orthogonal functions] Using a weight function $ r(x) > 0 $, two functions
        are orthogonal in the interval $ x \in [a, b] $ if,
        \begin{align}
            (y_m, y_n) & = \int_{a}^{b} r(x)\ y_m(x)\ y_n(x)\ \dl x = 0 &
                       & \forall\quad m \neq n
        \end{align}

    \item[Norm of function] The square integral of the function $ f(x) $ with respect
        to the weight function $ r(x) $.
        \begin{align}
            \lVert y_n \rVert = \sqrt{(y_n, y_n)} &
            = \sqrt{\int_{a}^{b} r(x)\ y_n^2(x)\ \dl x}
        \end{align}

    \item[Orthonormal functions] A set of functions that are orthogonal in some interval
        $ x \in [a, b] $ and additionally all have unit norm. Using the Kronecker Delta
        functions,
        \begin{align}
            (y_m, y_n)  & = \int_{a}^{b} r(x)\ y_m(x)\ y_n(x)\ \dl x = \delta_{mn} \\
            \delta_{mn} & = \begin{dcases}
                                0 & \quad \text{if}\ m \neq n \\
                                1 & \quad \text{if}\ m = n    \\
                            \end{dcases}
        \end{align}
        The weight function is not mentioned when it is the identically equal to 1.

    \item[Eigenfunctions of Sturm-Lioville problems] Let $ y_m(x) $ and $ y_n(x) $ be
        eigenfunctions that correspond to different eigenvalues $ \lambda_m $ and
        $ \lambda_n $ of the Sturm-Liouville problem. \par
        Then, $ y_m, y_n $ are orthogonal on the interval $ [a, b] $ with respect to
        their weight function $ r(x) $ \par
        If $ p(a) = p(b) $, then the boundary conditions also become periodic,
        \begin{align}
            y(a) & = y(b) & y'(a) & = y'(b)
        \end{align}

    \item[Orthogonal system] Many real world systems can be cast into Sturm-Liouville
        form leading to a set of orthonormal basis functions. Examples include the Bessel
        functions, Legendre polynomials and Fourier series expansions.


\end{description}