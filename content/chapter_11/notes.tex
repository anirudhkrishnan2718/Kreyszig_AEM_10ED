\chapter{Fourier Analysis}

\section{Fourier Series}

\begin{description}
    \item[Periodic function] A function defined on the real line, except possibly at a
        finite number of points such that,
        \begin{align}
            f(x + p)  & = f(x) & \forall\ x \in \mathcal{R} \\
            f(x + np) & =f(x)  & \forall\ n \in \mathcal{I}
        \end{align}
        for some positive real $ p $, which is called the period. \par
        The smallest positive period is called the fundamental period.

    \item[Trigonometric system] A family of periodic functions all having period $ 2\pi $
        which are the simplest basis used to represent all periodic functions.
        \begin{align}
            T & = \{1,\ \cos x,\ \sin x,\ \cos(2x),\ \sin(2x), \dots,\ \cos(nx),
            \ \sin(nx), \dots\}
        \end{align}

    \item[Fourier Series] A periodic function si represented as a linear combination of
        the above basis with each member assigned a coefficient (called the Fourier
        coefficient).
        \begin{align}
            f(x) & = a_0 + \iser[n]{0} a_n \cos(nx) + b_n \sin(nx)          \\
            a_0  & = \frac{1}{2\pi}\ \int_{-\pi}^{\pi} f(x)\ \dl x          \\
            a_n  & = \frac{1}{\pi}\ \int_{-\pi}^{\pi} f(x)\ \cos(nx)\ \dl x \\
            b_n  & = \frac{1}{\pi}\ \int_{-\pi}^{\pi} f(x)\ \sin(nx)\ \dl x
        \end{align}

    \item[Orthogonality of a trigonometric system] If integral over one period (taken to
        be $ -\pi $ to $ \pi $ by convention) of the product of any two members of the
        basis is zero, then the basis is orthogonal.
        \begin{align}
            \int_{-\pi}^{\pi}\ \cos(nx)\ \cos(mx)\ \dl x & = 0             &
            n                                            & \neq m            \\
            \int_{-\pi}^{\pi}\ \sin(nx)\ \sin(mx)\ \dl x & = 0             &
            n                                            & \neq m            \\
            \int_{-\pi}^{\pi}\ \cos(nx)\ \sin(mx)\ \dl x & = 0             &
            n,m                                          & \in \mathcal{I}
        \end{align}
        The Euler formulas for the Fourier coefficients are derived from the application
        of the orthogonality condition to the Fouier series definition.

    \item[Convergence of Fourier series] Let $ f(x) $ be periodic with period $ 2\pi $
        and be piecewise continuous in $ [-\pi, \pi] $. Also, let it have both
        left-handed and right-handed derivatives defined everywhere in this interval.
        \par Then, its Fourier series converges and is equal to $ f(x) $ at all points
        except at the finitely many points of discontinuity of $ f(x) $. \par
        At such points, the series converges to the average of the left and right-handed
        limits of $ f(x) $.
\end{description}

\section{Arbitrary Period, Even and Odd Functions, Half-Range Expansions}

\begin{description}
    \item[Functions with arbitrary period] Instead of the standard period $ p = 2\pi $,
        generalizing to $ p = 2L $ simply involves a change of variable
        \begin{align}
            2\pi \to 2L & \implies x \to \frac{\pi x}{L}            \\
            f(x)        & \to a_0 + \iser[n]{0} \Bigg[
                a_n\ \cos\left( \frac{n\pi x}{L} \right) +
            b_n\ \sin\left( \frac{n\pi x}{L} \right) \Bigg]         \\
            a_0         & = \frac{1}{2L}\ \int_{-L}^{L} f(x)\ \dl x \\
            a_n         & = \frac{1}{L}\ \int_{-L}^{L}
            f(x)\ \cos\left( \frac{n\pi x}{L} \right) \dl x         \\
            b_n         & = \frac{1}{L}\ \int_{-L}^{L}
            f(x)\ \sin\left( \frac{n\pi x}{L} \right) \dl x
        \end{align}

    \item[Even and Odd functions] Even functions and odd functions can be
        represented just by a Fourier cosine and sine series respectively.
        \begin{align}
            f(-x) = f(x)  & \implies b_n = 0 \qquad \text{even}      \\
            f(-x) = -f(x) & \implies a_0 = a_n = 0 \qquad \text{odd}
        \end{align}

    \item[Lineraity] The Fourier series is linear under addition and scalar
        multiplication.
        \begin{align}
            f(x) + g(x) & \to F(x) + G(x) \\
            c\ f(x)     & \to c\ F(x)
        \end{align}
        Where the uppercase is the fourier series expansion of the lowercase function.

    \item[Half-range expansions] Extending the function specified in the domain
        $ [0, L] $ into the domain $ [-L, L] $ as either an even or odd function in order
        to simplify the computation of Fourier coefficients.
\end{description}

\section{Forced Oscillations}

\begin{description}
    \item[Standard form ODE] A second order linear ODE is very common in physical systems
        undergoing forced damped oscillations.
        \begin{align}
            my'' + cy' + ky = r(t)
        \end{align}
        Here, the output $ y(t) $ is the solution to the ODE, corresponding to the input
        $ r(t) $. The constant coefficients $ m, c, k $ characterize the system.

    \item[Solution to ODE] Since the input can be represented as a Fourier series,
        the output can also be decomposed into a sum of outputs corresponding to each
        input term. \par
        The amplitude of each of the output terms happens to be a function of the input
        frequency and typically, one or two frequencies dominate the output in most real
        world systems.
\end{description}

\section{Approximation by Trigonometric Polynomials}

\begin{description}
    \item[Trigonometric polynomial] An approximation of a function using a series of
        trigonometric functions. The fourier series expansion happens to be an example.
        \begin{align}
            f(x) & = A_0 + \iser[n]{1} \Big[ A_n \cos(nx) + B_n \sin(nx) \Big]
        \end{align}
        Here, $ N $ is called the order.

    \item[Square Error] The error in this approximation is
        \begin{align}
            E & = \int_{-\pi}^{\pi} (f - F)^2\ \dl x
        \end{align}
        This error is a measure of the agreement between the approximation $ f $ and
        the actual function $ F $ over the entire period of the trigonometric function.

    \item[Minimum square error] The squared error of a trigonometric polynomial
        $ F $ with fixed order $ N $ is smallest in the domainm $[-\pi, \pi] $ if the
        coefficients are the Fourier coefficients $ A_n = a_n, B_n = b_n $
        \begin{align}
            E^* & = \int_{-\pi}^{\pi} f^2\ \dl x  - \pi\ \Bigg[ 2a_0^2
                + \sum_{n=1}^{N} (a_n^2 + b_n^2) \Bigg]
        \end{align}
        The above squared error can only decrease with increasing $ N $.

    \item[Bessel's Inequality] Given a function $ f $ and fourier coefficients $ a_0,
            \{a_n\}, \{b_n\} $, the condition on minimized squared error gives,
        \begin{align}
            2a_0^2 + \iser[n]{1} (a_n^2 + b_n^2) & \leq \frac{1}{\pi}\ \int_{-\pi}^{\pi}
            \ f(x)^2\ \dl x
        \end{align}

    \item[Parseval's identity] The integral of the square of a function over its
        standardized period $ [-\pi, \pi] $ is equal to the sum of the squares of its
        Fourier coefficients. (rough analog of the Pythagoras theorem)
        \begin{align}
            2a_0^2 + \iser[n]{1} (a_n^2 + b_n^2) & = \frac{1}{\pi}\ \int_{-\pi}^{\pi}
            \ f(x)^2\ \dl x
        \end{align}
\end{description}

\section{Sturm-Liouville Problems, Orthogonal Functions}

\begin{description}
    \item[Sturm-Liouville problem] A second order ODE, along with boundary conditions
        of the form,
        \begin{align}
            \diff{}{x}\Bigg[p(x)\ \diff yx\Bigg] +
            \Bigg[ q(x) + \lambda r(x) \Bigg]y & = 0                         \\
            k_1 y + k_2 y'                     & = 0 \qquad \text{at}\ x = a \\
            l_1 y + l_2 y'                     & = 0 \qquad \text{at}\ x = b
        \end{align}
        for some interval $ x \in [a,b] $. $ \lambda $ is a paramter and the two
        $ k, l $ are real constants. \par
        If $ p,q,r,p' $ are real valued and continuous in the interval $ [a,b] $ and
        $ r $ is the same sign throughout, then all eigenvalues $ \lambda $ of the
        Sturm-Liouville equation are real.

    \item[Orthogonal functions] Using a weight function $ r(x) > 0 $, two functions
        are orthogonal in the interval $ x \in [a, b] $ if,
        \begin{align}
            (y_m, y_n) & = \int_{a}^{b} r(x)\ y_m(x)\ y_n(x)\ \dl x = 0 &
                       & \forall\quad m \neq n
        \end{align}

    \item[Norm of function] The square integral of the function $ f(x) $ with respect
        to the weight function $ r(x) $.
        \begin{align}
            \lVert y_n \rVert = \sqrt{(y_n, y_n)} &
            = \sqrt{\int_{a}^{b} r(x)\ y_n^2(x)\ \dl x}
        \end{align}

    \item[Orthonormal functions] A set of functions that are orthogonal in some interval
        $ x \in [a, b] $ and additionally all have unit norm. Using the Kronecker Delta
        functions,
        \begin{align}
            (y_m, y_n)  & = \int_{a}^{b} r(x)\ y_m(x)\ y_n(x)\ \dl x = \delta_{mn} \\
            \delta_{mn} & = \begin{dcases}
                                0 & \quad \text{if}\ m \neq n \\
                                1 & \quad \text{if}\ m = n    \\
                            \end{dcases}
        \end{align}
        The weight function is not mentioned when it is the identically equal to 1.

    \item[Eigenfunctions of Sturm-Lioville problems] Let $ y_m(x) $ and $ y_n(x) $ be
        eigenfunctions that correspond to different eigenvalues $ \lambda_m $ and
        $ \lambda_n $ of the Sturm-Liouville problem. \par
        Then, $ y_m, y_n $ are orthogonal on the interval $ [a, b] $ with respect to
        their weight function $ r(x) $ \par
        If $ p(a) = p(b) $, then the boundary conditions also become periodic,
        \begin{align}
            y(a) & = y(b) & y'(a) & = y'(b)
        \end{align}

    \item[Orthogonal system] Many real world systems can be cast into Sturm-Liouville
        form leading to a set of orthonormal basis functions. Examples include the Bessel
        functions, Legendre polynomials and Fourier series expansions.
\end{description}

\section{Orthogonal Series, Generalized Fourier Series}

\begin{description}
    \item[Orthogonal series] A convergent (infinite) series that can be used to
        represent a function,
        \begin{align}
            y(x) & = \iser{0} a_m\ y_m(x) = a_0 y_0(x) + a_1 y_1(x) + \dots
        \end{align}
        This is also called an eigenfunction expansion if the functions $ \{y_m\} $
        are the eigenfunctions of a Sturm-Liouville problem.

    \item[Fourier coefficients] Similar to the Fourier series expansion, this family
        of functions is also orthogonal and thus,
        \begin{align}
            a_m & = \frac{(f, y_m)}{\lVert y_m \rVert^2}
            = \frac{1}{\lVert y_m \rVert^2}\ \int_{a}^{b} r(x)\ f(x)\ y_m(x)\ \dl x
        \end{align}
        Common families of polynomials used are Legendre polynomials, Bessel functions
        etc.

    \item[Orthogonality of Bessel functions] The set of Bessel functions of the first
        kind $ J_n (k_{n, i}x) $ are orthogonal on the interval $ x \in [0, R] $ with
        weight function $ r(x) = x $.
        \begin{align}
            J_n(kR) & = 0                        &
            k_{n,m} & =  \frac{\alpha_{n, m}}{R}   \\
            \int_{0}^{R} x\ J_n(k_{n,m}x)\ J_n(k_{n, j}x)
            \ \dl x & = 0                        &
            j       & \neq m
        \end{align}
        This assumes $ n $ is some fixed nonnegative integer. The Fourier-Bessel
        coefficients are found using
        \begin{align}
            \lVert J_n(k_{n,m}x) \rVert^2 & = \frac{R^2}{2}\ J^2_{n+1}(k_{n,m}R)   \\
            a_m                           & = \frac{2}{R^2J^2_{n+1}(\alpha_{n,m})}
            \int_{0}^{R} x\ f(x)\ J_n(k_{n,m}x)\ \dl x
        \end{align}

    \item[Mean squared convergence] A sequence of functions $ \{f_k\} $ is called
        convergent with the limit $ f $ if,
        \begin{align}
            \lim_{k \to \infty} \lVert f_k - f \rVert & = 0              &
            s_k(x)                                    & = \sum_{m=0}^{k}
            a_m\ y_m(x)                                                    \\
            \lim_{k \to \infty} \int_{a}^{b} r(x)\
            \Big[ s_k(x) - f(x) \Big]^2\ \dl x        & = 0
        \end{align}

    \item[Completeness] An orthonormal set $ \{y_i\} $ on the domain $ x \in [a, b] $
        is complete (or total) for a set of functions $ S $ defined on the same domain if,
        some linear combination can be found such that for any given $ \epsilon $,
        \begin{align}
            \lVert f - (a_0y_0 + a_1y_1 + \dots + a_ky_k) \rVert < \epsilon
        \end{align}
        for some large enough $ k $.

    \item[Inequalities] Using the above definition of convergence, the generalized form
        of Bessel's inequality is,
        \begin{align}
            \lVert f \rVert^2    & = \int_{a}^{b} r(x)\ f^2(x)\ \dl x &
            \sum_{m=0}^{k} a^2_m & \leq \lVert f \rVert^2
        \end{align}
        Using the fact that the left side of the above inequality is monotonically
        increasing and if the set $ \{y_m\} $ is complete for a set of functions $ S $
        in the same domain, gives the generalized Parseval's identity.
        \begin{align}
            \iser{0} a^2_m & = \lVert f \rVert^2
        \end{align}
        Additionally, a function in $ S $ that is orthogonal to every member of
        $ \{y_m\} $ then it has to have zero norm. If continuous, it is identically
        zero in the domain.

\end{description}

\section{Fourier Integral}

\begin{description}
    \item[Amplitude spectrum] The sequence of Fourier coefficients, each of which
        represent the maximum value of that term in the Fourier series expansion of a
        function.

    \item[Absolutely integrable] Consider a function $ f_L(x) $ that is periodic with
        period $ 2L $. It is absolutely integrable if the limit
        \begin{align}
            \lim_{a \to -\infty} \int_{a}^{0} \abs{f(x)}\ \dl x +
            \lim_{b \to \infty} \int_{0}^{b} \abs{f(x)}\ \dl x
        \end{align}
        exists and is finite.

    \item[Fourier integral] In the limit of the above function having period infinity,
        \begin{align}
            f(x) & = \lim_{L \to \infty } f_L(x)                              \\
            f(x) & = \infint \Big[A(w)\ \cos(wx) + B(w)\ \sin(wx)\Big]\ \dl x
        \end{align}
        Where the coefficients themselves are the Fourier coefficients of $ f(x) $
        integrated over the entire real line.
        \begin{align}
            A(w) & = \frac{1}{\pi}\ \intRL f(u)\ \cos(wu)\ \dl u \\
            B(w) & = \frac{1}{\pi}\ \intRL f(u)\ \sin(wu)\ \dl u
        \end{align}

    \item[Existence of Fourier integral] The Fourier integral formulation exists if
        $ f(x) $ in every finite interval is
        \begin{itemize}
            \item absolutely integrable.
            \item is piecewise continuous.
            \item has both left and right handed derivatives at every point.
        \end{itemize}
        Similar to the Fourier expansion, the value of the Fourier integral at a point
        of discontinuity of $ f(x) $ is the average of the left and right-handed limits
        of the function at that point.

    \item[Dirichlet discontinuous factor] The infinite integral that often shows up
        when calculating Fourier integrals given by,
        \begin{align}
            \infint \frac{\cos(wx)\ \sin(w)}{w}\ \dl w =
            \begin{dcases}
                \pi/2 & \qquad x \in [0, 1) \\
                \pi/4 & \qquad x = 1        \\
                0     & \qquad x > 1        \\
            \end{dcases}
        \end{align}

    \item[Sine integral] A special case of the above Dirichlet discontinuous factor
        for $ x = 0 $
        \begin{align}
            \text{Si}(u)                     & = \int_{0}^{u}
            \frac{\sin(w)}{w}\ \dl w                                       \\
            \lim_{u \to \infty} \text{Si}(u) & = \infint \frac{\sin(w)}{w}
            \ \dl w = \frac{\pi}{2}
        \end{align}

    \item[Gibbs phenomenon] The persistent large oscillations in the Fourier series
        approximations of $ f(x) $ at points of discontinuity.

    \item[Odd and even fourier integrals] Similar to Fourier series expansions,
        odd functions make $ B(w) = 0 $ and vice versa for even functions.

    \item[Laplace integrals] In the process of deriving the Fourier sine and cosine
        integrals of $ e^{-kx} $, useful results are found,
        \begin{align}
            f(x)                                         & = e^{-kx} \qquad
            \forall \qquad x > 0, \quad k > 0                                        \\
            \infint \frac{\cos(wx)}{k^2 + w^2}\ \dl w    & = \frac{\pi}{2k}\ e^{-kx} \\
            \infint \frac{w\ \sin(wx)}{k^2 + w^2}\ \dl w & = \frac{\pi}{2}\ e^{-kx}
        \end{align}
        This is a means of arriving at these integral working backwards.
\end{description}

\section{Fourier Cosine and Sine Transforms}

\begin{description}
    \item[Integral Transforms] An integral used to transform a function of one variable
        into another function of a different variable using a bridge function of both
        variables.
        \begin{align}
            F(s) = \int_{x_1}^{x_2} f(x)\ B(x, s)\ \dl x
        \end{align}
        The two most important integral transforms are the Laplace and Fourier transforms.

    \item[Fourier Cosine transform] For an even function $ f(x) $, the Fourier cosine
        integral can be used to introduce,
        \begin{align}
            \widehat{f}_c(w) & = \sqrt{\frac{2}{\pi}}\ \infint f(x)\ \cos(wx)\ \dl x          \\
            f(x)             & = \sqrt{\frac{2}{\pi}}\ \infint \widehat{f}_c\ \cos(wx)\ \dl w
        \end{align}
        Here, the bridge function $ \cos(wx) $ is used to transform a function in the
        $ x $ space into another function in the $ w $ space (and vice versa).

    \item[Fourier Sine transform] For an odd function $ f(x) $, the Fourier sine
        integral can be used to introduce,
        \begin{align}
            \widehat{f}_s(w) & = \sqrt{\frac{2}{\pi}}\ \infint f(x)\ \sin(wx)\ \dl x          \\
            f(x)             & = \sqrt{\frac{2}{\pi}}\ \infint \widehat{f}_s\ \sin(wx)\ \dl w
        \end{align}
        Here, the bridge function $ \sin(wx) $ is used to transform a function in the
        $ x $ space into another function in the $ w $ space (and vice versa). \par
        The shorthand notation for the forward transforms is
        \begin{align}
            \Fou_c(f)                  & = \widehat{f}_c &
            \Fou_s(f)                  & = \widehat{f}_s   \\
            \Fou_c^{-1}(\widehat{f}_c) & = f             &
            \Fou_s^{-1}(\widehat{f}_s) & = f
        \end{align}

    \item[Existence conditions] If $ f(x) $ is absolutely integrable on
        $ \mathcal{R}^+$, and is piecewise continuous on every finite interval, then its
        Fourier cosine and sine transforms exist.

    \item[Linearity] The Fourier cosine and sine transforms are linear operations.
        \begin{align}
            \Fou_c(af + bg) & = a\ \Fou_c(f) + b\ \Fou_c(g) \\
            \Fou_s(af + bg) & = a\ \Fou_s(f) + b\ \Fou_s(g)
        \end{align}
        for some scalars $ a, b $

    \item[Transforms of derivatives] Let $ f(x) $ be continuous and absolutely integrable
        on $ \mathcal{R} $, and let its derivative be piecewise continuous on every
        finite interval. \par
        Further, let $ f(x) \to 0 $ as $ x \to \infty $.
        \begin{align}
            \Fou_c \{f'(x)\} & = w\ \Fou_s \{f(x)\} - \sqrt{\frac{2}{\pi}} f(0) \\
            \Fou_s \{f'(x)\} & = -w\ \Fou_c \{f(x)\}
        \end{align}
        The condition on the infinite limit of $ f(x) $ allows the usage of integration
        by parts to derive the above relations. \par
        Using two I.B.P. steps, the relations for second derivatives are,
        \begin{align}
            \Fou_c \{f''(x)\} & = -w^2\ \Fou_c \{f(x)\} - \sqrt{\frac{2}{\pi}}\ f'(0) \\
            \Fou_s \{f''(x)\} & = -w^2\ \Fou_s \{f(x)\} + \sqrt{\frac{2}{\pi}}\ wf(0)
        \end{align}
\end{description}

\section{Fourier Transform, Discrete and Fast Fourier Transforms}

\begin{description}
    \item[Complex Fourier integral] Using the Euler formula on the Fourier sine and
        cosine integrals defined earlier,
        \begin{align}
            e^{ix} & = \cos(x) + i\ \sin(x)                                             \\
            f(x)   & = \frac{1}{2\pi} \intRL \intRL \Bigg[ f(u)\ e^{iw\ (x - u)} \Bigg]
            \ \dl u\ \dl w
        \end{align}

    \item[Fourier transform] Recognising the inner part of the above integral is
        a function of $ w $,
        \begin{align}
            \widehat{f}(w)   & = \sqrt{\frac{1}{2\pi}}\ \intRL f(x)\ e^{-iwx}
            \ \dl x          &
            \widehat{f}(w)   & = \Fou \{f\}                                   \\
            f(x)             & = \sqrt{\frac{1}{2\pi}}\ \intRL \widehat{f}(w)
            \ e^{iwx}\ \dl w &
            f(x)             & = \Fou^{-1}\{\widehat{f}(w)\}
        \end{align}
        The complex Fourier integral is thus mapping $ f(x) $ onto itself after
        performing both the Fourier transform and its inverse.

    \item[Existence of Fourier transform] Similar to the cosine and sine transforms,
        if $ f(x) $ is absolutely integrable on the real line and piecewise continuous on
        every finite interval, then its Fourier transform $ \widehat{f}(w) $ exists.

    \item[Spectrum] Physically, the Fourier transform can be considered a superposition
        of sinusoidal oscillations of all possible frequencies, which is then called
        the frequency spectrum. \par

    \item[Spectral density] The intensity of $ f(x) $ in the frequency interval
        $ (w, w + \Delta w) $. \par
        The total energy of the physical system, if $ w $ is the vibrational frequency
        is
        \begin{align}
            E & = \intRL \abs{\widehat{f}(w)}^2 \dl w
        \end{align}

    \item[Linearity of Fourier transform] Similar to the Fourier sine and cosine
        transforms, the complex Fourier transform is also linear.
        \begin{align}
            \Fou\{af + bg\} & = a\ \Fou\{f\} + b\ \Fou\{g\}
        \end{align}
        This is derived from the linearity of integration.

    \item[Fourier transform of derivative] Using integration by parts,
        \begin{align}
            \Fou\{f'(x)\}  & = iw\ \Fou\{f(x)\}   \\
            \Fou\{f''(x)\} & = -w^2\ \Fou\{f(x)\}
        \end{align}
        This requires $ f(x) \to 0 $ as $ \abs{x} \to \infty $ and for
        $ f'(x) $ to be absolutely integrable on the real line.

    \item[Fourier transform of Convolutions] Similar to Laplace transforms, the
        convolution of $ f(x) $ and $ g(x) $ has the property that its Fourier transform
        is the product of the individual transforms.
        \begin{align}
            \Fou\{f * g\} & = \sqrt{2\pi}\ \Fou\{f\} \cdot \Fou\{g\} \\
            (f * g)(x)    & = \intRL f(p)\ g(x-p)\ \dl p
            = \intRL f(x-p)\ g(p)\ \dl p
        \end{align}
        This relation requires $ f, g $ to be piecewise continuous, bounded and
        absolutely integrable on the real line. \par
        When applied to the inverse Fourier transform,
        \begin{align}
            (f * g)(x) & = \intRL \widehat{f}(w)\ \widehat{g}(w)\ e^{iwx}\ \dl w
        \end{align}

    \item[Discrete Fourier transform] Most real world experimental data is only sampled
        at finitely many (equally spaced) points in time. The function $ f(x) $ is now
        represented as a time series. \par

        Let $ f(x) $ be a periodic function of period $ 2\pi $ being sampled $ N $ times
        within one period.
        \begin{align}
            x_k & = \frac{2\pi}{N}\ k & k \in \{0, 1, \dots, N-1\}
        \end{align}
        Defining a complex trigonometric polynomial $ q(x) $ that interpolates $ f(x) $
        at the sampling points,
        \begin{align}
            q(x) & = \sum_{n=0}^{N-1} c_n\ e^{i nx} & f(x_k) & = q(x_k)
        \end{align}
        Using the orthogonality principle of complex numbers,
        \begin{align}
            c_n & = \frac{1}{N}\ \sum_{n=0}^{N-1} f(x_k)\ e^{-in x_k} &
            n   & =  \{0,1,2,\dots, N-1\}
        \end{align}
        The DFT of a signal sampled at $ N $ points becomes another vector with $ N $
        components representing discrete frequencies.
        \begin{align}
            f           & = \begin{bNiceMatrix}[margin]
                                f(x_0) \\ f(x_1) \\ \vdots \\ f(x_{N-1})
                            \end{bNiceMatrix}   &
            \widehat{f} & = \begin{bNiceMatrix}[margin]
                                \widehat{f}_0 \\ \widehat{f}_1\\ \vdots \\
                                \widehat{f}_{N-1}
                            \end{bNiceMatrix}
        \end{align}
        Here, the frequency spectrum of the signal is given by,
        \begin{align}
            \widehat{f}_n & = Nc_n = \sum_{n=0}^{N-1} f(x_k)\ e^{-in x_k}
        \end{align}

    \item[Fourier matrix] The above relation can be expressed in matrix notation as
        \begin{align}
            \vec{\widehat{f}}    & = \vec{F}_N\ \vec{f}                   &
            \vec{F}_N = [e_{nk}] & = e^{-in x_k} = w^{nk}                   \\
            w                    & = \exp\left( \frac{-2\pi i}{N} \right) &
            n,k                  & = \{0, 1, 2, \dots, (N-1)\}
        \end{align}
        $ w $ is a complex number with unit modulus. \par
        The inverse fourier transform requires the complex conjugate of $ \vec{F}_N $,
        \begin{align}
            \vec{F}_N^\dag \ \vec{F}_N = \vec{F}_N\ \vec{F}_N^\dag & = N \vec{I}   &
            \vec{F}_N^{-1}                                         & = \frac{1}{N}
            \ \vec{F}_N^\dag
        \end{align}

    \item[Fast Fourier Transform] Since the DFT procedure is of order $ N^2 $, a
        computationally tractable method of order $ N \log_2(N) $ is used instead. \par
        This involves starting with $ 2^p $ samples and splitting this problem into
        two halves recursively. Consider a vector $ \vec{f} $ sampled at $ N = 2M $
        points, \par

        This vector is split into odd and even components,
        \begin{align}
            \vec{f}_\text{ev}             & = \begin{bNiceMatrix}[margin]
                                                  f_0 \\ f_2 \\ \vdots \\ f_{N-2}
                                              \end{bNiceMatrix}   &
            \vec{f}_\text{od}             & = \begin{bNiceMatrix}[margin]
                                                  f_1 \\ f_3 \\ \vdots \\ f_{N-1}
                                              \end{bNiceMatrix}    \\
            \vec{\widehat{f}}_{\text{ev}} & = \vec{F}_M \ \vec{f}_{\text{ev}} &
            \vec{\widehat{f}}_{\text{od}} & = \vec{F}_M \ \vec{f}_{\text{od}}
        \end{align}
        Here, $ \vec{F}_M $ is the DFT matrix of half the size acting on the two halves
        of the input signal individually. \par
        Finally, the full DFT is obtained by superpositions of the above partial DFTs,
        \begin{align}
            \vec{\widehat{f}}_n     & = \vec{\widehat{f}}_{\text{ev}, n} + (w_N)^n
            \ \ \vec{\widehat{f}}_{\text{od}, n}                                   \\
            \vec{\widehat{f}}_{n+M} & = \vec{\widehat{f}}_{\text{ev}, n} - (w_N)^n
            \ \ \vec{\widehat{f}}_{\text{od}, n}
        \end{align}
        These are the first and second half of the complete DFT respectively, since
        $ n \in \{0,1,2,\dots,M - 1\} $
\end{description}