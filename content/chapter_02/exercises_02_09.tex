\section{Modeling: Electric Circuits}

\begin{enumerate}
    \item Solving the h-ODE,
          \begin{align}
              0                         & = RI' + \frac{1}{C}\ I                    \\
              \int\  \frac{1}{I}\ \dl I & = \frac{-1}{RC}\int \dl t                 \\
              I_{h}                     & = {\color{y_h} p_{1}\exp
              \left( \frac{-t}{RC} \right)}                                         \\
              I_{p}                     & = {\color{y_p} 0}                         \\
              I                         & ={\color{y_p} I_{p}} + {\color{y_h}I_{h}}
          \end{align}
          After the transient current exponentially decays to zero, the steady state
          current is zero, because the capacitor has charged up fully. Note that this
          result is independent of the constant emf $ E_{0} $.

    \item RL circuit with sinusoidally varying $ E $,
          \begin{align}
              \omega E_{0}\cos(\omega t) & = RI' + \frac{1}{C}\ I                 \\
              S                          & = \omega L - \frac{1}{\omega C}        \\
              I_{p}                      & = \frac{E_{0}}{R^{2} + S^{2}}
              \ \Big[-S\cos(\omega t) + R\sin(\omega t)\Big]                      \\
              I_{p}                      & = {\color{y_p} \frac{E_{0}
                  \cdot (\omega C)^{2}}{(\omega RC)^{2} + 1}
              \left[ R\sin(\omega t) + \frac{1}{\omega C} \cos(\omega t) \right]} \\
              I_h                        & = {\color{y_h} c_{1}
              \exp\left( \frac{-t}{RC} \right)}                                   \\
              I                          & ={\color{y_p} I_{p}}
              + {\color{y_h}I_{h}}
          \end{align}

    \item For an RL circuit with a constant emf,
          \begin{align}
              E                                & = LI' + RI                  \\
              \int\  \frac{1}{(RI - E)}\ \dl I & = \frac{-1}{L}\int \dl t    \\
              \frac{1}{R}\ln(RI - E)           & = \frac{-t}{L} + c_{1}      \\
              I                                & = {\color{y_p} \frac{E}{R}}
              + {\color{y_h} c_{1}\exp\left( \frac{-Rt}{L} \right)}          \\
              I                                & ={\color{y_p} I_{p}}
              + {\color{y_h}I_{h}}
          \end{align}
          Graphing the current for $ L = 0.25,\ R = 10,\ E = 48 $,
          \begin{figure}[H]
              \centering
              \begin{tikzpicture}[
                      declare function = {
                              I_p = 48/10;
                              I_h = e^(-40*x);
                          }
                  ]
                  \begin{axis}[
                          xlabel = {Time ($ t $)},
                          ylabel = {Current ($ I $)},
                          legend pos = north east,
                          grid = both,
                          xtick distance = 0.05,
                          xtick = {0,0.05,...,0.2},
                          Ani,
                      ]
                      \addplot[GraphSmooth, y_p, domain = 0:0.2]
                      {I_p + I_h};
                      \addlegendentry{$ I $};
                  \end{axis}
              \end{tikzpicture}
          \end{figure}

    \item For an RL circuit with a constant emf, solving the h-ODE
          \begin{align}
              E_{0}\sin(\omega t)        & = LI' + RI                     \\
              E_{0}\omega \cos(\omega t) & = LI'' + RI'                 &
              a                          & = R/L \qquad b = 0             \\
              \lambda_{1}, \lambda_{2}   & = \frac{-R/L \pm R/L}{2}     &
                                         & = \frac{-R}{L}, 0              \\
              I_{h}                      & = {\color{y_h} c_{1} + c_{2}
              \exp\left( \frac{-Rt}{L} \right) }
          \end{align}
          Solving the nh-ODE,
          \begin{align}
              S     & = \omega L - \frac{1}{\omega C}                      \\
              I_{p} & = \frac{E_{0}}{R^{2} + S^{2}}\ \Big[-S\cos(\omega t)
              + R\sin(\omega t)\Big]                                       \\
              I_{p} & = {\color{y_p} \frac{E_{0}}{R^{2} + (\omega L)^{2}}\
              \Big[R\sin(\omega t) - \omega L \cos(\omega t)\Big]}         \\
              I     & ={\color{y_p} I_{p}} + {\color{y_h}I_{h}}
          \end{align}
          Graphing the current for $ L = 0.25,\ R = 1,\ E = 10,\ \omega = 4$,
          \begin{figure}[H]
              \centering
              \begin{tikzpicture}[
                      declare function = {
                              I_p = sin(x - (pi/4));
                              I_h = e^(-0.1*x);
                          }
                  ]
                  \begin{axis}[
                          xlabel = {Time ($ t $)},
                          ylabel = {Current ($ I $)},
                          legend pos = north east,
                          grid = both,
                          domain = 0:50,
                          Ani,
                      ]
                      \addplot[GraphSmooth, y_p]{I_p + I_h};
                      \addplot[GraphSmooth, y_h]{I_h};
                      \addlegendentry{$ I $};
                      \addlegendentry{$ I_{h} $};
                  \end{axis}
              \end{tikzpicture}
          \end{figure}

    \item For an LC circuit with a negligible resistance and sinusoidal emf, solving
          the h-ODE, \par
          given $ L = 0.5,\ C = 0.005,\ \omega = 1,\ E_{0} = 1$,
          \begin{align}
              \cos t                   & = LI'' + \frac{1}{C}\ I        &
              a                        & = 0 \qquad b = \frac{-4}{LC}     \\
              \lambda_{1}, \lambda_{2} & = \frac{0 \pm i\sqrt{4/LC}}{2} &
                                       & = 0 \pm \frac{i}{\sqrt{LC}}      \\
              I_{h}                    & = {\color{y_h} c_{1}\cos(20 t)
              + c_{2}\sin(20 t)}
          \end{align}
          Solving the nh-ODE,
          \begin{align}
              S     & = \omega L - \frac{1}{\omega C} = \frac{-399}{2}              \\
              I_{p} & = \frac{E_{0}}{R^{2} + S^{2}}\ \Big[-S\cos(t) + R\sin(t)\Big] \\
              I_{p} & = {\color{y_p} \frac{-E_{0}}{S}\ [\cos(t)]}                   \\
              I     & ={\color{y_p} I_{p}} + {\color{y_h}I_{h}}
          \end{align}
          Applying the IC $ Q(0) = 0,\ I(0) = 0 $,
          \begin{align}
              I(0) = 0 & = c_{1} - \frac{E_{0}}{S}                 \\
              Q(0) = 0 & \implies I'(0) = 0 = 20c_{2}              \\
              I        & = {\color{y_h} \frac{E_{0}}{S} \cos(20t)}
              - {\color{y_p} \frac{E_{0}}{S}\ \cos(t)}
          \end{align}
          \begin{figure}[H]
              \centering
              \begin{tikzpicture}[
                      declare function = {
                              fac = (-2/399);
                              I_h = fac * cos(20*x);
                              I_p = -fac*cos(x);
                          }
                  ]
                  \begin{axis}[
                          PiStyleX,
                          xlabel = {Time ($ t $)},
                          ylabel = {Current ($ I $)},
                          legend pos = north east,
                          grid = both,
                          domain = 0:3*pi,
                          Ani,
                      ]
                      \addplot[GraphSmooth, y_p]{I_p + I_h};
                      \addplot[GraphSmooth, y_h]{I_h};
                      \addlegendentry{$ I $};
                      \addlegendentry{$ I_{h} $};
                  \end{axis}
              \end{tikzpicture}
          \end{figure}

    \item For an LC circuit with a negligible resistance and sinusoidal emf, solving
          the h-ODE, \par
          given $ L = 0.5,\ C = 0.005,\ \omega = 1,\ E_{t} = 2t^{2}$,
          \begin{align}
              4t                       & = LI'' + \frac{1}{C}\ I        &
              a                        & = 0 \qquad b = \frac{-4}{LC}     \\
              \lambda_{1}, \lambda_{2} & = \frac{0 \pm i\sqrt{4/LC}}{2} &
                                       & = 0 \pm \frac{i}{\sqrt{LC}}      \\
              I_{h}                    & = {\color{y_h} c_{1}\cos(20 t)
              + c_{2}\sin(20 t)}
          \end{align}
          Solving the nh-ODE,
          \begin{align}
              I_{p} & = K_{0} + K_{1}t                          \\
              K_{0} & = 0, \qquad K_{1} = 4C                    \\
              I_{p} & = {\color{y_p} 4Ct}                       \\
              I     & ={\color{y_p} I_{p}} + {\color{y_h}I_{h}}
          \end{align}
          Applying the IC $ Q(0) = 0,\ I(0) = 0 $,
          \begin{align}
              I(0) = 0 & = c_{1}                                                   \\
              Q(0) = 0 & \implies I'(0) = 0 = 20c_{2} + 4C                         \\
              I        & = {\color{y_h} \frac{-C}{5}\sin(20t)} + {\color{y_p} 4Ct}
          \end{align}
          \begin{figure}[H]
              \centering
              \begin{tikzpicture}[
                      declare function = {
                              I_h = -0.001*sin(20*x);
                              I_p = 0.02*x;
                          }
                  ]
                  \begin{axis}[
                          PiStyleX,
                          xlabel = {Time ($ t $)},
                          ylabel = {Current ($ I $)},
                          legend pos = north west,
                          grid = both,
                          domain = 0:1*pi,
                          Ani,
                      ]
                      \addplot[GraphSmooth, y_p]{I_p + I_h};
                      \addplot[GraphSmooth, y_h]{I_h};
                      \addlegendentry{$ I $};
                      \addlegendentry{$ I_{h} $};
                  \end{axis}
              \end{tikzpicture}
          \end{figure}

    \item Maximize $ I_{0} $, given $ R, L $ are constant,
          \begin{align}
              I_{0}                & = E_{0} \left[ R^{2} + (\omega L)^{2}
                  + \frac{1}{(\omega C)^{2}}
              - \frac{2L}{C} \right]^{-1/2}                                            \\
              \diff {I_{0}}{C}     & = \frac{-E_{0}}{2} \cdot \left( \frac{-2}
              {\omega^{2}C^{3}}
              + \frac{2L}{C^{2}} \right) \cdot [R^{2} + S^{2}]^{-3/2}                  \\
              \diff {I_{0}}{C} = 0 & \implies \frac{1}{\omega^{2}} = LC \implies S = 0
          \end{align}
          At zero complex impedance, the resistance of the circuit is minimized thus
          maximizing the current amplitude.

    \item Finding the steady state current,
          \begin{align}
              0.5I'' + 4I' + \frac{1}{0.1}\ I & = \diff* {500 \sin(2t)}{t}
              = 1000\cos(2t)                                                    \\
              S                               & = \omega L - \frac{1}{\omega C}
              = -4                                                              \\
              I_{p}                           & = \frac{E_{0}}{R^{2} + S^{2}}
              \ \Big[-S\cos(\omega t) + R\sin(\omega t)\Big]                    \\
              I_{p}                           & = {\color{y_p} \frac{125}{2}
              \ [\cos(2t) + \sin(2t)]}                                          \\
          \end{align}

    \item Finding the steady state current,
          \begin{align}
              0.1I'' + 4I' + \frac{1}{0.05}\ I & = \diff* {110}{t} = 0   \\
              I_{p}                            & = Kt + M                \\
              0                                & = 4K + 20M            &
              \cdots\cdots                     & [t^{0}]                 \\
              0                                & = 20K                 &
              \cdots\cdots                     & [t]                     \\
              I_{p}                            & = {\color{y_p} 0}       \\
          \end{align}

    \item Finding the steady state current,
          \begin{align}
              I'' + 2I' + 20I & = \diff* {157\sin(3t)}{t} = 471\cos(3t)         \\
              S               & = \omega L - \frac{1}{\omega C} = -\frac{11}{3} \\
              I_{p}           & = \frac{E_{0}}{R^{2} + S^{2}}
              \ \Big[-S\cos(\omega t) + R\sin(\omega t)\Big]                    \\
              I_{p}           & = {\color{y_p} 33\cos(2t) + 18\sin(2t)}         \\
          \end{align}

    \item Finding the steady state current,
          \begin{align}
              0.4I'' + 12I' + 80I & = \diff* {220\sin(10t)}{t} = 2200\cos(10t) \\
              S                   & = \omega L - \frac{1}{\omega C} = -4       \\
              I_{p}               & = \frac{E_{0}}{R^{2} + S^{2}}
              \ \Big[-S\cos(\omega t) + R\sin(\omega t)\Big]                   \\
              I_{p}               & = {\color{y_p} 5.5 \left[ \cos(10t)
              + 3\sin(10t) \right]}                                            \\
          \end{align}

    \item Finding the steady state current,
          \begin{align}
              0.1I'' + 0.2I' + \frac{1}{2}\ I & = \diff* {220\sin(314t)}{t}
              = 69080\cos(314t)                                                  \\
              S                               & = \omega L - \frac{1}{\omega C}
              = 31.4                                                             \\
              I_{p}                           & = \frac{E_{0}}{R^{2} + S^{2}}
              \ \Big[-S\cos(\omega t) + R\sin(\omega t)\Big]                     \\
              I_{p}                           & = {\color{y_p} -7.00534\cos(10t)
              + 0.04462 \sin(10t) }                                              \\
          \end{align}
    \item Finding the steady state current,
          \begin{align}
              1.2I'' + 12I' + \frac{3000}{20}\ I & = \diff* {12000\sin(25t)}{t}
              = 300000\cos(25t)                                                    \\
              S                                  & = \omega L - \frac{1}{\omega C}
              = 24                                                                 \\
              I_{p}                              & = \frac{E_{0}}{R^{2} + S^{2}}
              \ \Big[-S\cos(\omega t) + R\sin(\omega t)\Big]                       \\
              I_{p}                              & = {\color{y_p} -400\cos(25t)
              + 200\sin(25t) }                                                     \\
          \end{align}

    \item If $ R > 0 $, then prove $ I_{h} $ decays to zero with time, given
          $ R, L, C > 0 $,
          \begin{align}
              E_{0}\omega\cos(\omega t)     & = LI'' + RI' + \frac{1}{C}\ I        \\
              a                             & = \frac{R}{L}                      &
              b                             & = \frac{1}{LC}                       \\
              \alpha                        & = \frac{R}{2L}                     &
              \beta                         & = \frac{1}{2L}\sqrt{R^{2}
              - \frac{4L}{C}}                                                      \\
              I_{h}                         & = e^{-\alpha t}[c_{1}\cos(\beta t)
              + c_{2}\sin(\beta t)]         &
              \beta                         & < 0                                  \\
              I_{h}                         & = c_{1}e^{-(\alpha + \beta)t}
              + c_{2}e^{-(\alpha - \beta)t} &
              \beta                         & > 0
          \end{align}
          Since $ 0 < \beta < \alpha $ is guaranteed for the case $ \beta > 0 $, the two
          terms both have a negative exponent. In both cases, exponential decay is
          unavoidable for $ R > 0 $.

    \item From the above problem,
          \begin{align}
              \beta     & = \frac{1}{2L} \sqrt{R^{2} - \frac{4L}{C}}               \\
              \beta = 0 & \implies R_{\text{crit}} = 2\sqrt{\frac{L}{C}}           \\
              R         & \begin{dcases}
                              = R_{\text{crit}} & \text{critically damped} \\
                              < R_{\text{crit}} & \text{underdamped}       \\
                              > R_{\text{crit}} & \text{overdamped}        \\
                          \end{dcases}
          \end{align}

    \item Solving the h-ODE,
          \begin{align}
              0                        & = 0.2I'' + 8I' + \frac{1000}{12.5}\ I    &
              a                        & = 40 \qquad b = 400                        \\
              \lambda_{1}, \lambda_{2} & = \frac{-40 \pm \sqrt{0}}{2}             &
                                       & = -20, -20                                 \\
              I_{h}                    & = {\color{y_h} (c_{1} + c_{2}t)e^{-20t}}
          \end{align}
          Solving the nh-ODE,
          \begin{align}
              0.2I'' + 8I' + \frac{1000}{12.5}\ I & = \diff* {100\sin(10t)}{t}
              = 1000\cos(10t)                                                       \\
              S                                   & = \omega L - \frac{1}{\omega C}
              = -6                                                                  \\
              I_{p}                               & = \frac{E_{0}}{R^{2} + S^{2}}
              \ \Big[-S\cos(\omega t) + R\sin(\omega t)\Big]                        \\
              I_{p}                               & = {\color{y_p}  6\cos(25t)
              + 12\sin(25t) }                                                       \\
          \end{align}
          Applying the IC $ Q(0) = 0,\ I(0) = 0 $,
          \begin{align}
              I(0) = 0 & = c_{1} + 6                                \\
              Q(0) = 0 & \implies I'(0) = 0 = 300 + c_{2} - 20c_{1} \\
              I        & = {\color{y_h} [-6 + 14.7t]e^{-20t}}
              + {\color{y_p} 6[\cos(25t) + 2\sin(25t)]}
          \end{align}
          \begin{figure}[H]
              \centering
              \begin{tikzpicture}[
                      declare function = {
                              I_h = (-6 + 14.7*x)*e^(-20*x);
                              I_p = 6*(cos(25*x) + 2*sin(25*x));
                          }
                  ]
                  \begin{axis}[
                          PiStyleX, xtick distance = (1/25)*pi,
                          xlabel = {Time ($ t $)},
                          ylabel = {Current ($ I $)},
                          legend pos = north east,
                          grid = both,
                          domain = 0:0.2*pi,
                          Ani,
                      ]
                      \addplot[GraphSmooth, y_p]{I_p + I_h};
                      \addplot[GraphSmooth, y_h]{I_h};
                      \addlegendentry{$ I $};
                      \addlegendentry{$ I_{h} $};
                  \end{axis}
              \end{tikzpicture}
          \end{figure}

    \item Solving the h-ODE,
          \begin{align}
              0                        & = I'' + 6I' + 25I             &
              a                        & = 6 \qquad b = 25               \\
              \lambda_{1}, \lambda_{2} & = \frac{-6 \pm i\sqrt{64}}{2} &
                                       & = -3 \pm 4i                     \\
              I_{h}                    & = {\color{y_h} [c_{1}\cos(4t)
                  + c_{2}\sin(4t)]e^{-3t}}
          \end{align}
          Solving the nh-ODE,
          \begin{align}
              I'' + 6I' + 25I & = -600\sin t + 2400\cos t                              \\
              I_{p}           & = K\cos t + M\sin t                                    \\
              2400            & = -K + 6M + 25K              & \cdots\cdots & [\cos t] \\
              -600            & = -M - 6K + 25M              & \cdots\cdots & [\sin t] \\
              I_{p}           & = {\color{y_p}  100\cos(t) }                           \\
          \end{align}
          Applying the IC $ Q(0) = 0,\ I(0) = 0 $,
          \begin{align}
              I'(0) + 6I(0) + 25Q(0) & = E(0) = 600                               \\
              I(0) = 0               & = 100 + c_{1}                              \\
              Q(0) = 0               & \implies I'(0) = 600 = 4c_{2} - 3c_{1}     \\
              I                      & = {\color{y_h} [-100\cos(4t) + 75\sin(4t)]
              e^{-3t}} + {\color{y_p} 100 \cos(t)}
          \end{align}
          \begin{figure}[H]
              \centering
              \begin{tikzpicture}[
                      declare function = {
                              I_h = (-100*cos(4*x) + 75*sin(4*x))*e^(-3*x);
                              I_p = 100*cos(x);
                          }
                  ]
                  \begin{axis}[
                          PiStyleX, xtick distance = pi,
                          xlabel = {Time ($ t $)},
                          ylabel = {Current ($ I $)},
                          legend pos = north east,
                          grid = both,
                          domain = 0:3*pi,
                          Ani,
                      ]
                      \addplot[GraphSmooth, y_p]{I_p + I_h};
                      \addplot[GraphSmooth, y_h]{I_h};
                      \addlegendentry{$ I $};
                      \addlegendentry{$ I_{h} $};
                  \end{axis}
              \end{tikzpicture}
          \end{figure}

    \item Solving the h-ODE,
          \begin{align}
              0                        & = I'' + 18I' + 80I           &
              a                        & = 18 \qquad b = 80             \\
              \lambda_{1}, \lambda_{2} & = \frac{-18 \pm \sqrt{4}}{2} &
                                       & = -8, -10                      \\
              I_{h}                    & = {\color{y_h} c_{1}e^{-10t}
              + c_{2} e^{-8t}}
          \end{align}
          Solving the nh-ODE,
          \begin{align}
              I'' + 18I' + 80I & = -8200\sin(10t)                             \\
              I_{p}            & = K\cos(10t) + M\sin(10t)                    \\
              0                & = -100K + 180M + 80K                       &
              \cdots\cdots     & [\cos t]                                     \\
              -8200            & = -100M - 180K + 80M                       &
              \cdots\cdots     & [\sin t]                                     \\
              I_{p}            & = {\color{y_p}  45\cos(10t) + 5\sin(10t) }   \\
          \end{align}
          Applying the IC $ Q(0) = 0,\ I(0) = 0 $,
          \begin{align}
              E(0) = 820 & = I'(0) + 18I(0) + 80Q(0)                     \\
              I(0) = 0   & = 45 + c_{1} + c_{2}                          \\
              Q(0) = 0   & \implies I'(0) = 820 = -8c_{2} - 10c_{1} + 50 \\
              I          & = {\color{y_h} 160e^{-10t} - 205e^{-8t}}
              + {\color{y_p} 45 \cos(10t) + 5\sin(10t)}
          \end{align}
          \begin{figure}[H]
              \centering
              \begin{tikzpicture}[
                      declare function = {
                              I_h =160*e^(-10*x) - 205*e^(-8*x);
                              I_p = 45*cos(10*x) + 5*sin(10*x);
                          }
                  ]
                  \begin{axis}[
                          PiStyleX, xtick distance = 0.1*pi,
                          xlabel = {Time ($ t $)},
                          ylabel = {Current ($ I $)},
                          legend pos = north east,
                          grid = both,
                          domain = 0:0.6*pi,
                          Ani,
                      ]
                      \addplot[GraphSmooth, y_p]{I_p + I_h};
                      \addplot[GraphSmooth, y_h]{I_h};
                      \addlegendentry{$ I $};
                      \addlegendentry{$ I_{h} $};
                  \end{axis}
              \end{tikzpicture}
          \end{figure}

    \item  To understand the underlying similarities in the modeling of spring-mass
          systems and electrical circuits,
          \begin{table}[ht]
              \centering
              \SetTblrInner{rowsep=0.5em}
              \begin{tblr}{colspec={Q[r]|[dotted]Q[r, $$]|Q[l, $$]|[dotted]Q[l]},
                  colsep = 1em}
                  \SetCell[c=2]{c}\textbf{Spring Mass} &                           &
                  \SetCell[c=2]{c} \text{\textbf{Electrical
                  Circuit}}                            &                             \\
                  \hline[dotted]
                  Mass                                 & m                         &
                  L                                    & Inductance                  \\
                  Damping constant                     & c                         &
                  R                                    & Resistance                  \\
                  Spring modulus                       & k                         &
                  1/C                                  & reciprocal of Capacitance   \\
                  Driving Force                        & F_{0}\cos(\omega t)       &
                  E_{0}\omega \cos(\omega t)           & derivative of EMF           \\
                  Displacement                         & y(t)                      &
                  I(t)                                 & Current                     \\
                  \hline
              \end{tblr}
          \end{table}

          \begin{align}
              5I'' + 10I' + 60I & = 220 \cos(10t) \\
              I'' + 2I' + 12I   & = 44 \cos(10t)
          \end{align}
          From the above table, $ L = \SI{1}{\henry},
              \  R = \SI{2}{\ohm},\ C = 1/12 \unit{\farad}$, \par
          the derivative of the emf becomes, $ E(t)' = 44 \cos(10t) \unit{\volt} $,
          which means, $ E(t) =  4.4\sin(10t) \unit{\volt}$

    \item Deriving the solution to the electrical system ODE using complex numbers,
          \begin{align}
              L\tilde{I}'' +  R \tilde{I}' + \frac{1}{C}\ \tilde{I}
                                                 & = E_{0}\omega \exp(i\omega t)  \\
              I_{p}                              & = K\exp(i\omega t)             \\
              I_{p}'                             & = i\omega K\exp(i\omega t)     \\
              I_{p}''                            & = -\omega^{2}K\exp(i\omega t)  \\
              K\exp(i\omega t) [-\omega^{2}L + iR\omega + 1/C]
                                                 & = E_{0}i\omega \exp(i\omega t) \\
              K                                  & = \frac{E_{0}i}{-S + iR}
              = \frac{E_{0}}{R + iS}                                              \\
              |I_{p}| = \sqrt{K \bar{K}}         & = \frac{E_{0}}{\sqrt{S^{2}
              + R^{2}}}                                                           \\
              \text{given}\ R > 0, \qquad \arg Z & = \frac{S}{R} = \tan(-\theta)
          \end{align}
          These results agree with the text.
\end{enumerate}