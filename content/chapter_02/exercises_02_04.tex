\section{Modeling of Free Oscillations of a Mass-Spring System}

\colorlet{xcol}{blue!70!black}
\colorlet{vcol}{green!60!black}
\colorlet{acol}{red!50!blue!80!black!80}
\tikzstyle{force}=[->,BrickRed,very thick,line cap=round]
\tikzstyle{myarr}=[-{Latex[length=3,width=2]},thin]

\begin{enumerate}
    \item Initial velocity $ v_{0} $ and position $ y_{0} $, gives
          \begin{align}
              y     & = A \cos \omega_{0}t + B \sin \omega_{0}t \\
              y(0)  & = A = y_{0}                               \\
              y'(0) & = B\omega_{0} = v_{0}
          \end{align}

          Graphing solutions for $ \omega_{0} = \pi, y_{0} = 1$, with the different time series
          intersecting at $ t = n $
          \begin{figure}[H]
              \centering
              \begin{tikzpicture}
                  \begin{axis}[
                          %   ymode = log,
                          %   axis equal,
                          %   restrict y to domain = -5:20,
                          xlabel = Time ($ t $),
                          ylabel = Position ($ y $),
                          legend pos = south east,
                          grid = both,
                          width = 12cm,
                          height = 8cm,
                          Ani,
                      ]
                      \foreach [evaluate=\i as \n using (\i + 4)*100/(8)] \i in {-4, 0,...,4} {%
                              \edef\temp{%
                                  \noexpand
                                  \addplot[
                                      samples = 200,
                                      domain=0:4,
                                      color=blue!\n!red,
                                  ]
                                  {cos(pi*x) + (\i/pi)*sin(pi*x)};
                                  \noexpand \addlegendentry{$ v_{0} = \i$};
                              }\temp
                          }
                  \end{axis}
              \end{tikzpicture}
          \end{figure}

    \item Given $ m = \SI{20}{\newton} $ and $ y_{0} = \SI{2}{\cm} $,
          \begin{align}
              mg         & = ky_{0}                                &  & \text{Hooke's law} \\
              k          & = \frac{20}{0.02} = \SI{1000}{\N\per\m}                         \\
              \omega_{0} & = \sqrt{\frac{k}{m}} = \sqrt{500}                               \\
              f_{0}      & = \frac{\omega_{0}}{2\pi}               &  & = \SI{3.56}{\Hz}   \\
              T          & = \frac{1}{f_{0}}                       &  & = \SI{0.281}{\s}
          \end{align}

    \item Doubling the mass increases inertia, which decreases natural frequency
          \begin{align}
              f_{1} = 2\pi \sqrt{\frac{k}{2m}} = \frac{1}{\sqrt{2}}f_{0}
          \end{align}
          Doubling the spring modulus increases natural frequency
          \begin{align}
              f_{2} = 2\pi \sqrt{\frac{2k}{m}} = \sqrt{2}f_{0}
          \end{align}

    \item No, because the frequency $ f_{0} $ does not depend on the initial velocity $ v_{0} $.
          The initial velocity only increases the maximum amplitude of the oscillations, not the time
          period of the motion.

    \item Springs in parallel, with $ m = 5 $, $ k_{1} = 20 $, $ k_{2} = 45$,
          \begin{align}
              mg                    & = k_{1}y & \omega_{1} & = \sqrt{\frac{k_{1}}{m}} = \SI{2}{\N\per\m}    \\
              mg                    & = k_{2}y & \omega_{2} & = \sqrt{\frac{k_{2}}{m}} = \SI{3}{\N\per\m}    \\
              mg = (k_{1} + k_{2})y & = k_{3}y & \omega_{2} & = \sqrt{\frac{k_{3}}{m}} = \SI{3.61}{\N\per\m}
          \end{align}
          Springs in parallel have an effective spring constant given by,
          \begin{align}
              k_{\text{par}} & = \sum_{i=1}^{n}k_{i}
          \end{align}

    \item Springs in series, with $ k_{1} = 8 $, $ k_{2} = 12$. Consider a point in between the two
          springs. At static equilibrium,
          \begin{align}
              k_{1}y_{1} = k_{2}y_{2}
          \end{align}
          When replacing the springs by an effective spring with constant $ k_{3} $, and looking
          at a point between the mass and the bottom-most spring at static equilibrium,
          \begin{align}
              k_{1}y_{1}      & = k_{2}y_{2} = k_{3}y_{3} = mg      \\
              y_{3}           & = y_{1} + y_{2}                     \\
              \frac{1}{k_{3}} & = \frac{1}{k_{1}} + \frac{1}{k_{2}}
          \end{align}
          The above relation follows from the fact that the total extension of the system of
          springs in series is equal to the sum of the extension in each individual spring.
          Springs in parallel have an effective spring constant given by,
          \begin{align}
              k_{3} & = \frac{k_{1}k_{2}}{(k_{1} + k_{2})} \\
                    & = \SI{4.8}{\N\per\m}
          \end{align}

    \item For small angle $ \theta $, the relation $ \sin \theta \approxeq \theta$,
          \begin{figure}[H]
              \centering
              \begin{tikzpicture}
                  \def\ang{30}
                  \def\F{1.4}  % force magnitude
                  \coordinate (O) at (0,0);
                  \coordinate (FT) at (90+\ang:{\F*cos(\ang)});
                  \coordinate (FG) at (-90:\F);
                  \coordinate (FGx) at (-90+\ang:{0.7*\F});
                  \coordinate (MA) at (180+\ang:{\F*sin(\ang)});
                  \draw[dashed,black] (O) -- (FGx);
                  \draw[dashed,black] (MA) -- (FG);
                  \draw[force] (O) -- (FT) node[midway,above right=-2] {$T$};
                  \draw[force] (O) -- (FG) node[right=0] {$mg$};
                  \draw[force,acol] (O) -- (MA) node[left=0] {$ma$};
                  \draw pic[thin,"$\theta$",black,draw=black,angle radius=14,angle eccentricity=1.45] {angle=FG--O--FGx};
              \end{tikzpicture}
          \end{figure}
          For small displacement $ x $,
          \begin{align}
              mg\cos \theta                     & = T                  &       & \text{along string}                 \\
              mg\sin \theta                     & = ma                 &       & \text{along tangent}                \\
              mg\sin \theta \approxeq mg \theta & = ma                 & a     & = g\theta = \frac{gx}{l}            \\
              \omega_{0}                        & = \sqrt{\frac{g}{l}} & f_{0} & = \frac{1}{2\pi} \sqrt{\frac{g}{l}}
          \end{align}

    \item Let the equilibrium depth of cylinder bottom be $ y_{0} $, with density of water $ \rho $. Now the entire
          mass of water is acted upon by the gravitational force on the water above the equal line.
          \begin{align}
              (\rho V)g = mg & = \rho\ (\pi R^{2} y_{0})g
          \end{align}
          When a small downward displacement y happens, the restoring force is,
          \begin{align}
              ma     & = -mg + (\rho\pi g R^{2})\ (y_{0} + y)                                                                  \\
                     & = \rho\ (\pi R^{2} g y)                                                                                 \\
              \omega & =\sqrt{\frac{\rho\pi R^{2}g}{m}}       & T & = \frac{2\pi}{\omega} = \sqrt{\frac{4\pi m}{\rho R^{2} g}} \\
              m      & =\frac{\rho R^{2}g}{4\pi}\ T^{2}       &   & = \frac{9.8 \cdot 997 \cdot 0.09}{\pi}                     \\
                     & = \SI{279.9}{\kg}
          \end{align}

    \item $ 1 L $ of water vibrating in a U-shaped tube, displaced by $ y $ upwards,
          \begin{align}
              ma & = \rho (\pi R^{2})(2y)g
                 & \omega^{2}                                                        & = \frac{2\rho g \pi R^{2}}{m}                       \\
              f  & = \frac{\omega}{2\pi} = \sqrt{\frac{\rho g R^{2}}{2\pi (\rho V)}}
                 &                                                                   & = \sqrt{\frac{9.8 \cdot (0.01)^{2}}{2 \pi (0.001)}} \\
                 & = \SI{0.395}{\Hz}
          \end{align}

    \item
          \begin{enumerate}
              \item A pendulum's simple harmonic oscillation is governed by the relation,
                    \begin{align}
                        ma         & = mg \sin \theta \approxeq mg \theta                                                       \\
                                   & = mg \frac{x}{l}                                                                           \\
                        \omega_{2} & = \frac{g}{l}                        & T & = \frac{2\pi}{\omega} = 2\pi \sqrt{\frac{l}{g}} \\
                        T          & = \SI{2}{\s}
                    \end{align}
                    In 1 minute, the pendulum has completed 30 ticks.

              \item Given $ y_{0} = 0.01,\ m = 8,\ v_{0} = 0.1 $,
                    \begin{align}
                        mg & = ky_{0}                                                  &        & \text{static equilibrium}                  \\
                        ma & = -mg + k(y + y_{0}) = ky                                                                                       \\
                        a  & = \frac{k}{m}y = \frac{gy}{y_{0}}                         & \omega & = \sqrt{\frac{g}{y_{0}}} = \SI{31.305}{Hz} \\
                        y  & = y_{0}\cos \omega t + \frac{v_{0}}{\omega} \sin \omega t                                                       \\
                    \end{align}

              \item Torsional spring, with $ \theta' (0) = v_{0} = $ and $ \theta(0) = \theta_{0} $,
                    \begin{align}
                        \theta '' & = \frac{-K}{I_{0}}\ \theta                                        & \omega^{2} & = \frac{K}{I_{0}} \\
                        \theta    & = \theta_{0} \cos(\omega t) + \frac{v_{0}}{\omega} \sin(\omega t)                                  \\
                                  & = [0.5325] \cos(3.7 t) + [0.0943] \sin(3.7 t)
                    \end{align}
          \end{enumerate}

    \item Consider the standard solution for a damped oscillation,
          \begin{align}
              y(t)  & = c_{1}\exp\left[ -(\alpha - \beta)t \right] + c_{2}\exp\left[ -(\alpha + \beta)t \right] \\
              y'(0) & = -c_{1}(\alpha - \beta) - c_{2}(\alpha + \beta) = v_{0}                                  \\
              y(0)  & = c_{1} + c_{2} = y_{0}                                                                   \\
              v_{0} & = -c_{1}(\alpha - \beta) - (y_{0} - c_{1})(\alpha + \beta)                                \\
              c_{1} & =\frac{v_{0} + y_{0}(\alpha + \beta)}{2\beta}                                             \\
              c_{2} & = -\frac{v_{0} + y_{0}(\alpha - \beta)}{2\beta}
          \end{align}

    \item Checking the zero crossings of overdamped motion, $ \alpha > \beta > 0 $,
          \begin{align}
              y(t)                                    & = c_{1}\exp\left[ -(\alpha - \beta)t \right] + c_{2}\exp\left[ -(\alpha + \beta)t \right] \\
              y(t^{*})                                & = 0 \implies \frac{-c_{1}}{c_{2}} = \exp(-2\beta t^{*})                                   \\
              \ln \left( \frac{-c_{1}}{c_{2}} \right) & = -2\beta t^{*}                                                                           \\
              t^{*}                                   & =\frac{-1}{2\beta} \ln \left( \frac{-c_{1}}{c_{2}} \right)                                \\
              t^{*} > 0                               & \implies \frac{-c_{1}}{c_{2}} \in (0, 1)
          \end{align}
          $ t^{*} > 0 $, which means that the body crosses mean position at some real time, has at most one solution
          given by $ c_{1}c_{2} < 0 $ and $ |c_{2}| > |c_{1}| $.

    \item For critically damped motion,
          \begin{align}
              y(t)  & = (c_{1} + c_{2}t)e^{-\alpha t} & y'(t) & = (c_{2} - \alpha c_{1} - \alpha c_{2}t)e^{-\alpha t} \\
              y_{0} & = c_{1}                         & v_{0} & = c_{2} - \alpha c_{1}                                \\
              c_{2} & =v_{0} + \alpha y_{0}
          \end{align}
          Plotting graphs for $ \alpha = 1,\ y_{0} = 1 $,
          \begin{align}
              y & = [1 + (1 + v_{0})t]e^{-t} = 0 \\
              t & = \frac{-1}{1 + v_{0}}
          \end{align}
          \begin{figure}[H]
              \centering
              \begin{tikzpicture}
                  \begin{axis}[
                          %   ymode = log,
                          %   axis equal,
                          %   restrict y to domain = -5:20,
                          xlabel = Time ($ t $),
                          ylabel = Position ($ y $),
                          legend pos = north east,
                          grid = both,
                          width = 12cm,
                          height = 8cm,
                          Ani,
                      ]
                      \foreach [evaluate=\i as \n using (\i - 1)*100/(4)] \i in {1,...,5} {%
                              \edef\temp{%
                                  \noexpand
                                  \addplot[
                                      samples = 200,
                                      domain=0:6,
                                      color=blue!\n!red,
                                  ]
                                  {(1 + (-1/\i)*x)*e^(-x)};
                                  \noexpand \addlegendentry{$ t^{*} = \i$};
                              }\temp
                          }
                      \addplot[
                          samples = 200,
                          domain=0:6,
                          color= black,
                      ]
                      {(1 + (1 + 0)*x)*e^(-x)};
                      \addlegendentry{$ t^{*} = \infty$};
                  \end{axis}
              \end{tikzpicture}
          \end{figure}

    \item To avoid oscillations,
          \begin{align}
              c^{2} - 4mk & \geq 0                                      \\
              c           & \geq 2 \sqrt{mk} = 2 \sqrt{2000 \cdot 4500} \\
              c           & \geq \SI{6000}{\kg\per\s}
          \end{align}

    \item Applying binomial approximation,
          \begin{align}
              \omega^{*}           & = \sqrt{\frac{k}{m} - \frac{c^{2}}{4m^{2}}} = \omega\sqrt{1 - \frac{c^{2}}{4mk}} \\
              (1 - x)^{n}          & =  \left[ 1 - nx + \frac{n(n-1)}{2}x^{2} \right]                                 \\
              (1 - x)^{1/2}        & =  \left[ 1 - \frac{x}{2} - \frac{x^{2}}{8}  \right]                             \\
              x                    & = \frac{c^{2}}{4mk} = \frac{100}{4 \cdot 10 \cdot 90} = 1/36                     \\
              \omega^{*}           & = 0.986 \cdot \omega = 0.986 \cdot \sqrt{\frac{90}{10}} = \SI{2.95804398}{\Hz}   \\
              \omega_{\text{true}} & = \SI{2.95803989}{\Hz}
          \end{align}
          The binomial approximation using 2 terms provides an answer
          with an error of $ 0.00013\% $.

    \item For the maxima of underdamped motion, with $ \alpha > 0 $ and $ \omega > 0 $,
          \begin{align}
              y(t)      & = [A \cos(\omega t) + B\sin(\omega t)]e^{-\alpha t}                                    \\
              y'(t)     & = e^{-\alpha t}[(B\omega - A\alpha)\cos(\omega t) - (A\omega + B\alpha)\sin(\omega t)] \\
              y'(t) = 0 & \implies \tan(\omega t) = \frac{B\omega - A\alpha}{A\omega + B\alpha}
          \end{align}
          The times at which extrema are reached are equally spaced, as seen by the expression for
          $ y'(t) = 0 $ involving only constants. The period of oscillations is,
          \begin{align}
              \Delta t & = \frac{2\pi}{\omega } & \omega & = \sqrt{\frac{k}{m} - \frac{c^{2}}{4m^{2}}}
          \end{align}

    \item To find the maxima of the function,
          \begin{align}
              y(t)       & = e^{-t}\sin t                                \\
              y'(t)      & = e^{-t}[\cos t - \sin t]                     \\
              y''(t)     & = e^{-t}[-2\cos t]                            \\
              y'(t) = 0  & \implies \tan t = 1                           \\
              y''(t) < 0 & \implies \cos t > 0                           \\
                         & \implies t \in (2n\pi - \pi/2, 2n\pi + \pi/2)
          \end{align}
          Maxima occurs at $ t_{\text{max}} = 2n\pi + \arctan(1) $. The minima occur at
          $t_{\text{min}} = (2n-1)\pi + \arctan(1) $ correspondingly.

          \begin{figure}[H]
              \centering
              \begin{tikzpicture}
                  \begin{axis}[
                          PiStyleX,
                          xlabel = Time $ (t) $,
                          ylabel = Position $ (y) $,
                          legend pos = north west,
                          grid = both,
                          width = 12cm,
                          height = 8cm,
                          Ani,
                          domain = 0:6,
                          %   restrict y to domain = -10:10
                      ]
                      \addplot[GraphSmooth, color = red!0!blue, dotted]
                      {-e^(-x)} node[pos = 0.5,below]
                      {$ -e^{-t}$};
                      \addplot[GraphSmooth, color = red!100!blue, dotted]
                      {e^(-x)} node[pos = 0.5,above]
                      {$ e^{-t}$};
                      \addplot[GraphSmooth, color = red!50!blue]
                      {e^(-x)*sin(x)};
                      \node[GraphNode, label={90:$ t = \pi/4 $}] at (axis cs:pi/4, 0.322) {};
                      \node[GraphNode, label={-90:$ t = 5\pi/4 $}] at (axis cs:5*pi/4, -0.014) {};
                  \end{axis}
              \end{tikzpicture}
          \end{figure}

    \item For underdamped oscillations at successive maxima,
          \begin{align}
              y_{1}                                 & = Ce^{-\alpha t_{1}} \cos(\omega t_{1} + \delta)                                              \\
              y_{2}                                 & = Ce^{-\alpha t_{2}} \cos(\omega t_{2} + \delta)                                              \\
              \frac{y_{2}}{y_{1}}                   & = \exp[-\alpha(t_{2} - t_{1})]                   & t_{2} - t_{1} & = \frac{2\pi}{\omega}      \\
              \ln\left( \frac{y_{2}}{y_{1}} \right) & = \frac{2\pi \alpha}{\omega}                                                                  \\
              y'' + 2y' + 5y                        & = 0                                                                                           \\
              \lambda_{1}, \lambda_{2}              & = -1 \pm 2i                                      & y             & = Ce^{-t}\cos(2t + \delta) \\
              \Delta                                & = \frac{2\pi \cdot (1)}{2}                       &               & = \pi
          \end{align}

    \item Time between consecutive maxima is $ \SI{3}{\s} $, mass $ m = \SI{0.5}{\kg} $,
          \begin{align}
              \frac{y(30)}{y(0)} & = \exp\left[ -\alpha(t_{2} - t_{1}) \right] &        & = \exp(-30\alpha) = 0.5  \\
              \alpha             & = \frac{\ln 2}{30}                                                              \\
              T                  & = \frac{2\pi}{\omega} = \SI{3}{\s}          & \omega & = \frac{2\pi}{3}         \\
              \alpha             & = \frac{c}{2m}                              & c      & = \SI{0.0231}{\kg\per\s}
          \end{align}

    \item
          \begin{enumerate}
              \item Solving the general ODE with $ y(0) = 1,\ y'(0) = 0 $,
                    \begin{align}
                        y'' + cy' + y            & = 0                                                                                                                                       \\
                        \lambda_{1}, \lambda_{2} & = \frac{-c}{2} \pm \frac{\sqrt{c^{2} - 4}}{2} = -\alpha \pm \beta                                                                         \\
                        y                        & = \begin{dcases}
                                                         c_{1}e^{-(\alpha - \beta)t} + c_{2}e^{-(\alpha + \beta) t}  & c > 2 \\
                                                         (c_{1} + c_{2}t)e^{-\alpha t}                               & c = 2 \\
                                                         e^{-\alpha t}[c_{1}\cos(-i\beta t) + c_{2} \sin(-i\beta t)] & c < 2
                                                     \end{dcases}
                    \end{align}
                    Applying the IC to each case individually, overdamped motion gives,
                    \begin{align}
                        c_{1} + c_{2} & = 1                             & -c_{1}(\alpha - \beta) - c_{2}(\alpha + \beta) & = 0                             \\
                        c_{1}         & = \frac{\alpha + \beta}{2\beta} & c_{2}                                          & = \frac{\beta - \alpha}{2\beta}
                    \end{align}
                    critically damped motion gives,
                    \begin{align}
                        c_{1} & = 1      & -c_{1}\alpha + c_{2} & = 0 \\
                        c_{2} & = \alpha
                    \end{align}
                    underdamped motion gives,
                    \begin{align}
                        c_{1} & = 1                     & -\alpha c_{1} - i\beta c_{2} & = 0 \\
                        c_{2} & = \frac{i\alpha}{\beta}
                    \end{align}

                    For the alternative IC, $ y(0) = 1,\ y'(0) = -2 $, \newline
                    Applying the IC to each case individually, overdamped motion gives,
                    \begin{align}
                        c_{1} + c_{2} & = 1                                 & -c_{1}(\alpha - \beta) - c_{2}(\alpha + \beta) & = -2                                \\
                        c_{1}         & = \frac{\alpha + \beta - 2}{2\beta} & c_{2}                                          & = \frac{\beta - \alpha + 2}{2\beta}
                    \end{align}
                    critically damped motion gives,
                    \begin{align}
                        c_{1} & = 1          & -c_{1}\alpha + c_{2} & = -2 \\
                        c_{2} & = \alpha - 2
                    \end{align}
                    underdamped motion gives,
                    \begin{align}
                        c_{1} & = 1                           & -\alpha c_{1} - i\beta c_{2} & = -2 \\
                        c_{2} & = \frac{i(\alpha - 2)}{\beta}
                    \end{align}

                    \begin{figure}[H]
                        \centering
                        \begin{tikzpicture}
                            \begin{axis}[
                                    title = {Critically damped},
                                    ylabel = Position ($ y $),
                                    legend pos = north east,
                                    grid = both,
                                    width = 12cm,
                                    height = 8cm,
                                    Ani,
                                ]
                                \foreach [evaluate=\i as \n using (\i + 2)*100/(4)] \i in {-2,0,2} {%
                                        \edef\temp{%
                                            \noexpand
                                            \addplot[
                                                samples = 200,
                                                domain=0:6,
                                                color=blue!\n!red,
                                            ]
                                            {(1 + (1 + \i)*x)*e^(-x)};
                                            \noexpand \addlegendentry{$ v_{0} = \i$};
                                        }\temp
                                    }
                            \end{axis}
                        \end{tikzpicture}
                    \end{figure}

                    \begin{figure}[H]
                        \centering
                        \begin{tikzpicture}
                            \begin{axis}[
                                    title = {Overdamped},
                                    ylabel = Position ($ y $),
                                    legend pos = north east,
                                    grid = both,
                                    width = 12cm,
                                    height = 8cm,
                                    Ani,
                                ]
                                \foreach [evaluate=\i as \n using (\i + 4)*100/(8)] \i in {-4,0,4} {%
                                        \edef\temp{%
                                            \noexpand
                                            \addplot[
                                                samples = 200,
                                                domain=0:8,
                                                color=blue!\n!red,
                                            ]
                                            {((sqrt(2) + 1 + \i)/(2))*e^((1 - sqrt(2))*x) +
                                                ((-sqrt(2) + 1 - \i)/(2))*e^((-1 - sqrt(2))*x)};
                                            \noexpand \addlegendentry{$ v_{0} = \i$};
                                        }\temp
                                    }
                            \end{axis}
                        \end{tikzpicture}
                    \end{figure}

                    \begin{figure}[H]
                        \centering
                        \begin{tikzpicture}
                            \begin{axis}[
                                    title = Underdamped,
                                    ylabel = Position ($ y $),
                                    legend pos = north east,
                                    grid = both,
                                    width = 12cm,
                                    height = 8cm,
                                    Ani,
                                ]
                                \foreach [evaluate=\i as \n using (\i + 2)*100/(4)] \i in {-2,0,2} {%
                                        \edef\temp{%
                                            \noexpand
                                            \addplot[
                                                samples = 200,
                                                domain=0:20,
                                                color=blue!\n!red,
                                            ]
                                            {e^(-x*0.2)*cos(x*0.9797) +
                                                e^(-x*0.2)*sin(x*0.9797)*((0.2 + \i)/ 0.9797)};
                                            \noexpand \addlegendentry{$ v_{0} = \i$};
                                        }\temp
                                    }
                            \end{axis}
                        \end{tikzpicture}
                    \end{figure}

              \item To show the transition of damping levels,
                    \begin{figure}[H]
                        \centering
                        \begin{tikzpicture}
                            \begin{axis}[
                                    ylabel = Position ($ y $),
                                    legend pos = north east,
                                    grid = both,
                                    width = 12cm,
                                    height = 8cm,
                                    Ani,
                                    domain = 0:40,
                                ]
                                \addplot[
                                    samples = 200,
                                    color=blue,
                                ]
                                {e^(-x*0.05)*cos(x*0.9987) +
                                    e^(-x*0.05)*sin(x*0.9987)*((0.1)/ 0.9987)};
                                \addlegendentry{$ c = 0.1$};
                                \addplot[
                                    samples = 200,
                                    color=blue!50!red,
                                ]
                                {(1 + x)*e^(-x)};
                                \addlegendentry{$ c = 2$};
                                \addplot[
                                    samples = 200,
                                    color=red,
                                ]
                                {(1.00689687752485)*e^(-0.0827625302982197*x) +
                                    (-0.00689687752485164)*e^(-12.0827625302982*x)};
                                \addlegendentry{$ c = \sqrt{148}$};
                            \end{axis}
                        \end{tikzpicture}
                    \end{figure}

              \item Amplitude decaying to 0.1\% of its initial value at $ t^{*} $, \newline
                    For critically damped motion,
                    \begin{align}
                        [y_{0} + ((c/2) y_{0} + v_{0})t]e^{- ct/2} = 0.001 \cdot y_{0}
                    \end{align}
                    This cannot yield an explicit relation, but provides an implicit relation between
                    $ c $ and $ t^{*} $. Similar implicit relations can be established for underdamped and
                    overdamped motion.

              \item Solving analytically, TBC
                    \begin{align}
                        0      & = y'' + cy' + y                                                                                                                                       \\
                        \alpha & = c/2 \qquad \beta = \frac{\sqrt{c^{2} - 4}}{2}                                                                                                       \\
                        y      & = \begin{dcases}
                                       \left( \frac{\beta + \alpha}{2\beta} \right)e^{-(\alpha - \beta)t} + \left( \frac{\beta - \alpha}{2\beta} \right)e^{-(\alpha + \beta) t} & c > 2 \\
                                       (1 + \alpha t)e^{-\alpha t}                                                                                                              & c = 2 \\
                                       e^{-\alpha t}\left[\cos(-i\beta t) + \left( \frac{i\alpha}{\beta} \right) \sin(-i\beta t)\right]                                         & c < 2
                                   \end{dcases}
                    \end{align}

              \item The difference is that in B, the system has a zero crossing and in A it does not.
          \end{enumerate}
\end{enumerate}