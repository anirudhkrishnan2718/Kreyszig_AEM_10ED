\section{Existence and Uniqueness of Solutions, Wronskian}

\begin{enumerate}
    \item \begin{align}
              W(y_{1}, y_{2})  & = y_{1}y_{2}' - y_{1}'y_{2}                       &
              W                & = -(y_{1}'y_{2} - y_{1}y_{2}')                      \\
                               & = \frac{y_{1}y_{2}' - y_{1}'y_{2}}{y_{1}^{2}}
              \ y_{1}^{2}      &
                               & = -\frac{(y_{1}'y_{2} - y_{1}y_{2}')}{y_{2}^{2}}
              \ y_{2}^{2}                                                            \\
                               & = \left( \frac{y_{2}}{y_{1}} \right)'\ y_{1}^{2}  &
                               & = -\left( \frac{y_{1}}{y_{2}} \right)'\ y_{2}^{2}   \\
              \text{if}\ y_{1} & \neq 0                                            &
              \text{if}\ y_{2} & \neq 0
          \end{align}

    \item the functions are L.I. by the quotient method,
          \begin{align}
              y_{1} & = e^{4x}                                   &
              y_{2} & = e^{-1.5x}                                  \\
              Q(x)  & = \frac{y_{1}}{y_{2}} = e^{5.5x}           &
              Q     & \neq 0\ \text{for some}\ x \in \mathcal{R}
          \end{align}
          the functions are L.I. using the Wronskian,
          \begin{align}
              W(y_{1}, y_{2}) & = -1.5e^{2.5x} - 4e^{2.5x}                 \\
                              & = -5.5e^{2.5x}                             \\
              W               & \neq 0\ \text{for some}\ x \in \mathcal{R}
          \end{align}

    \item the functions are L.I. by the quotient method,
          \begin{align}
              y_{1} & = e^{-0.4x}                                &
              y_{2} & = e^{-2.6x}                                  \\
              Q(x)  & = \frac{y_{1}}{y_{2}} = e^{2.2x}           &
              Q     & \neq 0\ \text{for some}\ x \in \mathcal{R}
          \end{align}
          the functions are L.I. using the Wronskian,
          \begin{align}
              W(y_{1}, y_{2}) & = -2.6e^{-3x} + 0.4e^{-3x}                 \\
                              & = -2.2e^{-3x}                              \\
              W               & \neq 0\ \text{for some}\ x \in \mathcal{R}
          \end{align}

    \item the functions are L.I. by the quotient method,
          \begin{align}
              y_{1} & = x                                        &
              y_{2} & = 1/x                                        \\
              Q(x)  & = \frac{y_{1}}{y_{2}} = x^{2}              &
              Q     & \neq 0\ \text{for some}\ x \in \mathcal{R}
          \end{align}
          the functions are L.I. using the Wronskian,
          \begin{align}
              W(y_{1}, y_{2}) & = \frac{-1}{x} - \frac{1}{x}               \\
                              & =\frac{-2}{x}                              \\
              W               & \neq 0\ \text{for some}\ x \in \mathcal{R}
          \end{align}

    \item the functions are L.I. by the quotient method,
          \begin{align}
              y_{1} & = x^{3}                                    &
              y_{2} & = x^{2}                                      \\
              Q(x)  & = \frac{y_{1}}{y_{2}} = x                  &
              Q     & \neq 0\ \text{for some}\ x \in \mathcal{R}
          \end{align}
          the functions are L.I. using the Wronskian,
          \begin{align}
              W(y_{1}, y_{2}) & = 2x^{4} - 3x^{4}                          \\
                              & = -x^{4}                                   \\
              W               & \neq 0\ \text{for some}\ x \in \mathcal{R}
          \end{align}

    \item the functions are L.I. by the quotient method,
          \begin{align}
              y_{1} & = e^{-x}\cos(\omega x)                     &
              y_{2} & = e^{-x}\sin(\omega x)                       \\
              Q(x)  & = \frac{y_{1}}{y_{2}} = \cot(\omega x)     &
              Q     & \neq 0\ \text{for some}\ x \in \mathcal{R}
          \end{align}
          the functions are L.I. using the Wronskian,
          \begin{align}
              W(y_{1}, y_{2}) & = e^{-2x}\cos(\omega x)[\omega \cos(\omega x)
              - \sin(\omega x)]                                                \\
                              & - e^{-2x}\sin(\omega x)[-\omega \sin(\omega x)
              - \cos(\omega x)]                                                \\
                              & =e^{-2x}[\omega]                               \\
              W               & \neq 0\ \text{for some}\ x \in \mathcal{R}
          \end{align}

    \item the functions are L.I. by the quotient method,
          \begin{align}
              y_{1} & = \cosh ax                                 &
              y_{2} & = \sinh ax                                   \\
              Q(x)  & = \frac{y_{2}}{y_{1}} = \tanh ax           &
              Q     & \neq 0\ \text{for some}\ x \in \mathcal{R}
          \end{align}
          the functions are L.I. using the Wronskian,
          \begin{align}
              W(y_{1}, y_{2}) & = a\cosh^{2} ax - a\sinh^{2} ax            \\
                              & = a                                        \\
              W               & \neq 0\ \text{for some}\ x \in \mathcal{R}
          \end{align}

    \item the functions are L.I. by the quotient method,
          \begin{align}
              y_{1} & = x^{k}\cos(\ln x)                         &
              y_{2} & = x^{k}\sin(\ln x)                           \\
              Q(x)  & = \frac{y_{1}}{y_{2}} = \cot(\ln x)        &
              Q     & \neq 0\ \text{for some}\ x \in \mathcal{R}
          \end{align}
          the functions are L.I. using the Wronskian,
          \begin{align}
              W(y_{1}, y_{2}) & = x^{2k-2}\cos(\ln x)[ k\sin(\ln x) + \cos(\ln x)] \\
                              & - x^{2k-2}\sin(\ln x)[-\sin(\ln x) + k\cos(\ln x)] \\
                              & = x^{2k-2}                                         \\
              W               & \neq 0\ \text{for some}\ x \in \mathcal{R}
          \end{align}

    \item the functions are L.I. using the Wronskian,
          \begin{align}
              y_{1}           & = \cos(5x)                                 &
              y_{2}           & = \sin(5x)                                   \\
              W(y_{1}, y_{2}) & = 5\cos^{2}(5x) + 5\sin^{2}(5x)              \\
                              & = 5                                          \\
              W               & \neq 0\ \text{for some}\ x \in \mathcal{R}
          \end{align}
          to construct the ODE given these solutions,
          \begin{align}
              \alpha                   & = 0        & \beta & = 5               \\
              \lambda_{1}, \lambda_{2} & = 0 \pm 5i & a     & = 0, \quad b = 25 \\
              y'' + 25y                & = 0
          \end{align}
          solving the IVP $ y(0) = 3,\ y'(0) = -5 $,
          \begin{align}
              c_{1} & = 3                    & 5c_{2} & = -5 \\
              y     & = 3\cos(5x) - \sin(5x)
          \end{align}

    \item the functions are L.I. using the Wronskian,
          \begin{align}
              y_{1}           & = x^{m_{1}}                                &
              y_{2}           & = x^{m_{2}}                                  \\
              W(y_{1}, y_{2}) & = x^{m_{1} + m_{2} - 1}[m_{2} - m_{1}]       \\
              W               & \neq 0\ \text{for some}\ x \in \mathcal{R}
          \end{align}
          to construct the Euler-Cauchy ODE given these solutions,
          \begin{align}
              \lambda_{1}, \lambda_{2} & = m_{1}, m_{2}       \\
              (a-1)                    & = -(m_{1} + m_{2}) &
              b                        & = m_{1}m_{2}         \\
              x^{2}y'' + (1 - m_{1} - m_{2})xy' + m_{1}m_{2}y
                                       & = 0
          \end{align}
          solving the IVP $ y(1) = -2,\ y'(1) = 2m_{1} - 4m_{2} $,
          \begin{align}
              c_{1} + c_{2}           & = -2                      &
              m_{1}c_{1} + m_{2}c_{2} & = 2m_{1} - 4m_{2}           \\
              y                       & = 2x^{m_{1}} -4 x^{m_{2}}
          \end{align}

    \item the functions are L.I. using the Wronskian,
          \begin{align}
              y_{1}           & = e^{-2.5x}\cos(0.3x)                               &
              y_{2}           & = e^{-2.5x}\sin(0.3x)                                 \\
              W(y_{1}, y_{2}) & = e^{-5x}\cos(0.3x)[0.3\cos(0.3x) - 2.5\sin(0.3x)]    \\
                              & - e^{-5x}\sin(0.3x)[-0.3\sin(0.3x) - 2.5\cos(0.3x)]   \\
                              & = e^{-5x}[0.3]                                        \\
              W               & \neq 0\ \text{for some}\ x \in \mathcal{R}
          \end{align}
          to construct the ODE given these solutions,
          \begin{align}
              \lambda_{1}, \lambda_{2} & = -2.5 \pm 0.3i              \\
              a                        & = 5             & b & = 6.34 \\
              y'' + 5y' + 6.34y        & = 0
          \end{align}
          solving the IVP $ y(0) = 3,\ y'(0) = -7.5 $,
          \begin{align}
              c_{1} & = 3                    & -2.5c_{1} + 0.3c_{2} & = -7.5 \\
              c_{2} & = 0                                                    \\
              y     & = 3e^{-2.5x}\cos(0.3x)
          \end{align}

    \item the functions are L.I. using the Wronskian,
          \begin{align}
              y_{1}           & = x^{2}                                    &
              y_{2}           & = x^{2}\ln x                                 \\
              W(y_{1}, y_{2}) & = x^{3}[1 + 2\ln x - 2\ln x]                 \\
                              & = x^{3}                                      \\
              W               & \neq 0\ \text{for some}\ x \in \mathcal{R}
          \end{align}
          to construct the Euler-Cauchy ODE given these solutions,
          \begin{align}
              \lambda_{1}, \lambda_{2} & = 2            \\
              (a-1)                    & = -4 & b & = 4 \\
              x^{2}y'' - 3xy' + 4y     & = 0
          \end{align}
          solving the IVP $ y(1) = 4,\ y'(1) = 6 $,
          \begin{align}
              c_{1} & = 4                      & 2c_{1} + c_{2} & = 6 \\
              c_{2} & = -2                                            \\
              y     & = 4x^{2}  - 2 x^{2}\ln x
          \end{align}

    \item the functions are L.I. using the Wronskian,
          \begin{align}
              y_{1}           & = 1                                        &
              y_{2}           & = e^{-2x}                                    \\
              W(y_{1}, y_{2}) & = -2e^{-2x}                                  \\
              W               & \neq 0\ \text{for some}\ x \in \mathcal{R}
          \end{align}
          to construct the ODE given these solutions,
          \begin{align}
              \lambda_{1}, \lambda_{2} & = 0, -2           \\
              a                        & = 2     & b & = 0 \\
              y'' + 2y'                & = 0
          \end{align}
          solving the IVP $ y(0) = 1,\ y'(0) = -1 $,
          \begin{align}
              c_{1} + c_{2} & = 1                 & -2c_{2} & = -1  \\
              c_{2}         & = 0.5               & c_{1}   & = 0.5 \\
              y             & = 0.5 (1 + e^{-2x})
          \end{align}

    \item the functions are L.I. using the Wronskian,
          \begin{align}
              y_{1}           & = e^{-kx}\cos(\pi x)                       &
              y_{2}           & = e^{-kx}\sin(\pi x)                         \\
              W(y_{1}, y_{2}) & = e^{-2kx}\cos(\pi x)[\pi \cos(\pi x)
              - k\sin(\pi x)]                                                \\
                              & - e^{-2kx}\sin(\pi x) [-\pi \sin(\pi x)
              - k\cos(\pi x)]                                                \\
                              & = \pi e^{-2kx}                               \\
              W               & \neq 0\ \text{for some}\ x \in \mathcal{R}
          \end{align}
          to construct the ODE given these solutions,
          \begin{align}
              \alpha                          & = -k                            &
              \beta                           & = \pi                             \\
              \lambda_{1}, \lambda_{2}        & = -k \pm i\pi                   &
              a                               & = 2k, \quad b = k^{2} + \pi^{2}   \\
              y'' + 2ky' + (k^{2} + \pi^{2})y & = 0
          \end{align}
          solving the IVP $ y(0) = 1,\ y'(0) = -k - \pi $,
          \begin{align}
              c_{1}               & = 1                                        &
              -kc_{1} + \pi c_{2} & = -k-\pi                                     \\
              c_{2}               & = -1                                         \\
              y                   & =e^{-kx}\left[ \cos(kx) - \sin(kx) \right]
          \end{align}

    \item the functions are L.I. using the Wronskian,
          \begin{align}
              y_{1}           & = \cosh(1.8 x)                             &
              y_{2}           & = \sinh(1.8 x)                               \\
              W(y_{1}, y_{2}) & = 1.8\cosh^{2}x - 1.8\sinh^{2}x              \\
                              & = 1.8                                        \\
              W               & \neq 0\ \text{for some}\ x \in \mathcal{R}
          \end{align}
          to construct the ODE given these solutions,
          \begin{align}
              \lambda_{1}, \lambda_{2} & = \pm 1.8 & a & = 0, \quad b = 3.24 \\
              y'' - 3.24y              & = 0
          \end{align}
          solving the IVP $ y(0) = 14.20,\ y'(0) = 16.38 $,
          \begin{align}
              c_{1} & = 14.20                       & 1.8c_{2} & = 16.38 \\
              c_{2} & = 9.1                                              \\
              y     & = 14.20 \cosh x + 9.1 \sinh x
          \end{align}

    \item
          \begin{enumerate}
              \item Solving the ODE using both kinds of basis functions,
                    \begin{align}
                        y''  - y & = 0                               \\
                        y        & = c_{1}e^{x} + c_{2}e^{-x}      &
                                 & \text{exponential functions}      \\
                        y        & = c_{3}\cosh(x) + c_{4}\sinh(x) &
                                 & \text{hyperbolic functions}       \\
                        c_{3}    & = 0.5(c_{1} + c_{2})            &
                        c_{4}    & = 0.5(c_{1} - c_{2})
                    \end{align}

              \item Claim : Solutions of a basis set can be $ 0 $ at the same point
                    (say $ X $).
                    \begin{align}
                        y_{1}(X)                      & = y_{2}(X) = 0              \\
                        c_{1}y_{1}(X) + c_{2}y_{2}(X) & = 0 \qquad
                        \text{without needing}\ c_{1}, c_{2} = 0                    \\
                        y_{2}(X)                      & = \frac{-c_{1}}{c_{2}}y_{1} \\
                        W(X)                          & =  \Bigg[ y_{1}y_{2}'
                        - y_{2}'y_{1} \Bigg]_{x = X}                                \\
                                                      & = \frac{-c_{2}}
                        {c_{1}}y_{1}(X)y_{1}'(X)
                        + \frac{c_{2}}{c_{1}}
                        y_{1}'(X)y_{1}(X)                                           \\
                                                      & = 0
                    \end{align}
                    The fact that $ W(X) = 0 $ means that the two functions are linearly
                    dependent in $ \mathcal{I} $. But this contradicts the assumption
                    that they are members of a basis set. This contradiction falsifies
                    the original claim.

              \item Refer previous part. Instead of $ y_{1}(X) = y_{2}(X) = 0 $, the new
                    claim is that $ y_{1}'(X) = y_{2}'(X) = 0 $.
                    Procedure is the same from the Wronskian step onward.

              \item These formulas exist if one of the solutions is not the trivial
                    solution, which is true for most physical systems.

              \item Sketching,
                    \begin{figure}[H]
                        \centering
                        \begin{tikzpicture}
                            \begin{axis}[
                                    %   axis equal,
                                    % restrict y to domain = -100:100,
                                    legend pos = north west,
                                    grid = both,
                                    width = 12cm,
                                    height = 8cm,
                                    Ani,
                                ]
                                \addplot[GraphSmooth, y_h,domain=-1:0]{0};
                                \addplot[forget plot,GraphSmooth, y_h,domain=0:1]
                                {x^(3)};
                                \addlegendentry{$y_{1}$};
                                \addplot[GraphSmooth, y_p, domain=-1:0, dashed]
                                {x^(3)};
                                \addplot[forget plot,GraphSmooth, y_p, domain=0:1,
                                    dashed]{0};
                                \addlegendentry{$y_{2}$};
                            \end{axis}
                        \end{tikzpicture}
                    \end{figure}
                    Wronskian is zero in $ (-1,0) $ and $ (0, 1) $ separately because
                    one of the two functions is zero.
                    It is also zero at $ x = 0 $. \par
                    However, over the entire interval $ (-1, 1) $, they are L.I, because,
                    \begin{align}
                        c_{1}y_{1} + c_{2}y_{2} & = 0                               \\
                        x                       & < 0    & \implies  c_{2} & \neq 0 \\
                        x                       & \geq 0 & \implies  c_{1} & \neq 0
                    \end{align}
                    Looking at the functions and their derivatives, both are continuous
                    in $ (-1, 1) $, which means that these functions can be solutions
                    to an ODE. \par
                    Consider an Euler-Cauchy equation with $ m_{1}, m_{2} = 0, 3 $
                    \begin{align}
                        (a - 1)             & = -3 & b              & = 0 \\
                        m^{2}  - 3m         & = 0  & x^{2}y'' -2xy' & = 0 \\
                        y'' - \frac{2}{x}y' & = 0
                    \end{align}
                    If the interval under consideration is $ (-1, 1) $, then
                    $ p(x) = 2/x $ is discontinuous at $ x = 0 $, which means that
                    Theorem 2 does not apply to this ODE.

              \item Eliminating $ q(x) $,
                    \begin{align}
                        y_{1}'' + p(x)y_{1}' + q(x)y_{1}   & = 0                      \\
                        y_{2}'' + p(x)y_{2}' + q(x)y_{2}   & = 0                      \\
                        y_{2}'' + py_{2}' - y_{2}\left( \frac{y_{1}''
                        + py_{1}'}{y_{1}} \right)          & = 0                      \\
                        y_{1}y_{2}'' - y_{1}''y_{2} + p (y_{1}y_{2}'
                        - y_{1}'y_{2})                     & = 0                      \\
                        \diff Wx                           & = \diff* {(y_{1}y_{2}'
                        - y_{1}'y_{2})}{x}                                            \\
                        \diff Wx = y_{1}y_{2}'' - y_{1}''y_{2}
                        + y_{1}'y_{2}' - y_{1}'y_{2}'      & = y_{1}y_{2}''
                        - y_{1}''y_{2}                                                \\
                        W' + pW                            & = 0                      \\
                        \int_{x_{0}}^{x}\frac{1}{W}\ \dl W & = -\int_{x_{0}}^{t} p(t)
                        \ \dl t                                                       \\
                        \frac{W(x)}{W(x_{0})}              & = \exp
                        \left( -\int_{x_{0}}^{x} p(t)\ \dl t \right)
                    \end{align}
                    Applying Abel's formula to problem 6, TBC
          \end{enumerate}
\end{enumerate}