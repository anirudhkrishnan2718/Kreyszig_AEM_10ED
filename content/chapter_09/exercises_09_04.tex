\section{Vector and Scalar Functions and Their Fields, Vector Calculus: Derivatives}
\begin{enumerate}
    \item Isotherms are hyperbolas.
          \begin{figure}[H]
              \centering
              \begin{tikzpicture}
                  \begin{axis}[
                          colorbar,
                          colormap/jet,
                          Ani,
                          grid = both,
                          axis equal,
                          title = {$ c = x^2 - y^2 $},
                          view     = {0}{90},
                      ]
                      \addplot3 [
                          domain = -4:4,
                          thick,
                          contour gnuplot={
                                  number = 9,
                                  %   levels={0, 10, -10},
                                  labels=false,
                              },
                          samples=100
                      ] {x^2 - y^2};
                  \end{axis}
              \end{tikzpicture}
          \end{figure}

    \item Isotherms are hyperbolas.
          \begin{figure}[H]
              \centering
              \begin{tikzpicture}
                  \begin{axis}[
                          set layers,
                          colorbar,
                          colormap/jet,
                          Ani,
                          grid = both,
                          axis equal,
                          title = {$ c = xy $},
                          % enlargelimits = true,
                          view     = {0}{90}, % for a view 'from above'
                      ]
                      \addplot3 [
                          domain = -4:4,
                          thick,
                          contour gnuplot={
                                  % number = 10,
                                  levels={-4, -2, -1, 1, 2, 4},
                                  labels=false,
                              },
                          samples=100
                      ] {(x*y)};
                  \end{axis}
              \end{tikzpicture}
          \end{figure}

    \item Isotherms are straight lines.
          \begin{figure}[H]
              \centering
              \begin{tikzpicture}
                  \begin{axis}[
                          set layers,
                          colorbar,
                          colormap/jet,
                          Ani,
                          grid = both,
                          axis equal,
                          title = {$ c = 3x - 4y $},
                          % enlargelimits = true,
                          view     = {0}{90}, % for a view 'from above'
                      ]
                      \addplot3 [
                          domain = -4:4,
                          thick,
                          contour gnuplot={
                                  % number = 10,
                                  levels={-4, -2, 0, 2, 4},
                                  labels=false,
                              },
                          samples=100
                      ] {(3*x - 4*y)};
                  \end{axis}
              \end{tikzpicture}
          \end{figure}

    \item Isotherms are straight lines.
          \begin{figure}[H]
              \centering
              \begin{tikzpicture}
                  \begin{axis}[
                          PiStyleX,
                          PiStyleY,
                          xtick distance = pi,
                          ytick distance = pi,
                          set layers,
                          colorbar,
                          colormap/jet,
                          Ani,
                          grid = both,
                          axis equal,
                          title = {$ c = \arctan(y/x) $},
                          % enlargelimits = true,
                          view     = {0}{90}, % for a view 'from above'
                      ]
                      \addplot3 [
                          domain = -2*pi:2*pi,
                          thick,
                          contour gnuplot={
                                  % number = 10,
                                  levels={-pi/3, -pi/4, -pi/6,0,pi/6, pi/4, pi/3},
                                  labels=false,
                              },
                          samples=100
                      ] {atan(y/x)};
                  \end{axis}
              \end{tikzpicture}
          \end{figure}

    \item Isotherms are circles centered on the $ y $-axis.
          \begin{figure}[H]
              \centering
              \begin{tikzpicture}
                  \begin{axis}[
                          set layers,
                          colorbar,
                          colormap/jet,
                          Ani,
                          grid = both,
                          axis equal,
                          title = {$ c = \dfrac{y}{x^2 + y^2} $},
                          % enlargelimits = true,
                          view     = {0}{90}, % for a view 'from above'
                      ]
                      \addplot3 [
                          domain = -1:1,
                          thick,
                          contour gnuplot={
                                  % number = 10,
                                  levels={-4, -2, 0, 2, 4},
                                  labels=false,
                              },
                          samples=100
                      ] {y/(x^2 + y^2)};
                  \end{axis}
              \end{tikzpicture}
          \end{figure}

    \item Isotherms are circles centered on the $ x $-axis.
          \begin{figure}[H]
              \centering
              \begin{tikzpicture}
                  \begin{axis}[
                          set layers,
                          colorbar,
                          colormap/jet,
                          Ani,
                          grid = both,
                          axis equal,
                          title = {$ c = \dfrac{x}{x^2 + y^2} $},
                          % enlargelimits = true,
                          view     = {0}{90}, % for a view 'from above'
                      ]
                      \addplot3 [
                          domain = -1:1,
                          thick,
                          contour gnuplot={
                                  % number = 10,
                                  levels={-2, -1, -0.5, 0.5, 1, 2},
                                  labels=false,
                              },
                          samples=100
                      ] {x/(x^2 + y^2)};
                  \end{axis}
              \end{tikzpicture}
          \end{figure}

    \item Isotherms are ellipses.
          \begin{figure}[H]
              \centering
              \begin{tikzpicture}
                  \begin{axis}[
                          set layers,
                          colorbar,
                          colormap/jet,
                          Ani,
                          grid = both,
                          axis equal,
                          title = {$ c = 9x^2 + 4y^2 $},
                          % enlargelimits = true,
                          view     = {0}{90}, % for a view 'from above'
                      ]
                      \addplot3 [
                          domain = -4:4,
                          thick,
                          contour gnuplot={
                                  % number = 10,
                                  levels={0.5, 1, 2, 4},
                                  labels=false,
                              },
                          samples=200
                      ] {9*x^2 + 4*y^2};
                  \end{axis}
              \end{tikzpicture}
          \end{figure}

    \item Plotting the isotherms using \texttt{gnuplot}
          \begin{enumerate}
              \item The isotherms are hyperbolas.
                    \begin{figure}[H]
                        \centering
                        \begin{tikzpicture}
                            \begin{axis}[
                                    set layers,
                                    colorbar,
                                    colormap/jet,
                                    Ani,
                                    grid = both,
                                    axis equal,
                                    title = {$ c = x^2 - 4x - y^2 $},
                                    % enlargelimits = true,
                                    view     = {0}{90}, % for a view 'from above'
                                ]
                                \addplot3 [
                                    domain = -6:8,
                                    thick,
                                    contour gnuplot={
                                            % number = 10,
                                            levels={-20, -8,-4,0, 12},
                                            labels=false,
                                        },
                                    samples=200
                                ] {(x^2 - 4*x - y^2)};
                            \end{axis}
                        \end{tikzpicture}
                    \end{figure}

              \item The isotherms are hyperbola analogs with three principal axes.
                    \begin{figure}[H]
                        \centering
                        \begin{tikzpicture}
                            \begin{axis}[
                                    set layers,
                                    colorbar,
                                    colormap/jet,
                                    Ani,
                                    grid = both,
                                    axis equal,
                                    title = {$ c = x^2y - y^3/3 $},
                                    % enlargelimits = true,
                                    view     = {0}{90}, % for a view 'from above'
                                ]
                                \addplot3 [
                                    domain = -4:4,
                                    thick,
                                    contour gnuplot={
                                            % number = 10,
                                            levels={-4, -2, 0, 2, 4},
                                            labels=false,
                                        },
                                    samples=100
                                ] {(x^2*y - (y^3)/3)};
                            \end{axis}
                        \end{tikzpicture}
                    \end{figure}

              \item The isotherms are periodic in x.
                    \begin{figure}[H]
                        \centering
                        \begin{tikzpicture}
                            \begin{axis}[
                                    PiStyleX, xtick distance = pi,
                                    set layers,
                                    colorbar,
                                    colormap/jet,
                                    Ani,
                                    grid = both,
                                    axis equal,
                                    title = {$ c = \cos x \sinh y $},
                                    % enlargelimits = true,
                                    view     = {0}{90}, % for a view 'from above'
                                ]
                                \addplot3 [
                                    domain = -2*pi:2*pi,
                                    thick,
                                    contour gnuplot={
                                            % number = 10,
                                            levels={-4, -1, 0, 1, 4},
                                            labels=false,
                                        },
                                    samples=100
                                ] {cos(x) * sinh(y)};
                            \end{axis}
                        \end{tikzpicture}
                    \end{figure}

              \item The isotherms are periodic in x.
                    \begin{figure}[H]
                        \centering
                        \begin{tikzpicture}
                            \begin{axis}[
                                    PiStyleX, xtick distance = pi,
                                    set layers,
                                    colorbar,
                                    colormap/jet,
                                    Ani,
                                    grid = both,
                                    axis equal,
                                    title = {$ c = \sin x \sinh y $},
                                    % enlargelimits = true,
                                    view     = {0}{90}, % for a view 'from above'
                                ]
                                \addplot3 [
                                    domain = -2*pi:2*pi,
                                    thick,
                                    contour gnuplot={
                                            % number = 10,
                                            levels={-4, -1, 0, 1, 4},
                                            labels=false,
                                        },
                                    samples=100
                                ] {sin(x) * sinh(y)};
                            \end{axis}
                        \end{tikzpicture}
                    \end{figure}

              \item The isotherms are periodic in y.
                    \begin{figure}[H]
                        \centering
                        \begin{tikzpicture}
                            \begin{axis}[
                                    PiStyleY, ytick distance = pi,
                                    set layers,
                                    colorbar,
                                    colormap/jet,
                                    Ani,
                                    grid = both,
                                    axis equal,
                                    title = {$ c = e^x\ \sin(y) $},
                                    % enlargelimits = true,
                                    view     = {0}{90}, % for a view 'from above'
                                ]
                                \addplot3 [
                                    domain = -2*pi:2*pi,
                                    thick,
                                    contour gnuplot={
                                            % number = 10,
                                            levels={-4, -1, 0, 1, 4},
                                            labels=false,
                                        },
                                    samples=100
                                ] {e^x * sin(y)};
                            \end{axis}
                        \end{tikzpicture}
                    \end{figure}

              \item The isotherms are periodic in y.
                    \begin{figure}[H]
                        \centering
                        \begin{tikzpicture}
                            \begin{axis}[
                                    PiStyleY, ytick distance = pi,
                                    set layers,
                                    colorbar,
                                    colormap/jet,
                                    Ani,
                                    grid = both,
                                    axis equal,
                                    title = {$ c = e^{2x}\ \cos(y) $},
                                    % enlargelimits = true,
                                    view     = {0}{90}, % for a view 'from above'
                                ]
                                \addplot3 [
                                    domain = -2*pi:2*pi,
                                    thick,
                                    contour gnuplot={
                                            % number = 10,
                                            levels={-4, -1, 0, 1, 4},
                                            labels=false,
                                        },
                                    samples=100
                                ] {e^(2*x) * cos(y)};
                            \end{axis}
                        \end{tikzpicture}
                    \end{figure}

              \item The isotherms are hyperbola analogs with six principal axes.
                    \begin{figure}[H]
                        \centering
                        \begin{tikzpicture}
                            \begin{axis}[
                                    set layers,
                                    colorbar,
                                    colormap/jet,
                                    Ani,
                                    grid = both,
                                    axis equal,
                                    title = {$ c = x^4 - 6x^2y^2 + y^4 $},
                                    % enlargelimits = true,
                                    view     = {0}{90}, % for a view 'from above'
                                ]
                                \addplot3 [
                                    domain = -3:3,
                                    thick,
                                    contour gnuplot={
                                            % number = 10,
                                            levels={-8,-2,0,2,8},
                                            labels=false,
                                        },
                                    samples=100
                                ] {x^4 - 6*x^2*y^2 + y^4};
                            \end{axis}
                        \end{tikzpicture}
                    \end{figure}

              \item The isotherms are hyperbola analogs with six principal axes.
                    \begin{figure}[H]
                        \centering
                        \begin{tikzpicture}
                            \begin{axis}[
                                    set layers,
                                    colorbar,
                                    colormap/jet,
                                    Ani,
                                    grid = both,
                                    axis equal,
                                    title = {$ c = x^2 - 2x - y^2 $},
                                    % enlargelimits = true,
                                    view     = {0}{90}, % for a view 'from above'
                                ]
                                \addplot3 [
                                    domain = -4:7,
                                    thick,
                                    contour gnuplot={
                                            % number = 10,
                                            levels={-7, -3, -1, 1, 5},
                                            labels=false,
                                        },
                                    samples=100
                                ] {x^2 - 2*x - y^2};
                            \end{axis}
                        \end{tikzpicture}
                    \end{figure}
          \end{enumerate}

    \item The level surfaces are a family of parallel planes with the common
          normal vector $ \vec{n} = (4, -3, 2) $

    \item For constant $ z $, the cross sections in the $ xy $ plane are circles
          \begin{align}
              x^2 + y^2 = \frac{c - z_0^2}{9}
          \end{align}
          For constant $ x $, the cross sections in the $ yz $ plane are ellipses
          \begin{align}
              y^2 + \frac{z^2}{3^2} & = c - x_0^2
          \end{align}
          Thus the level surfaces are ellipsoids with circular symmetry about the $ z $
          axis.

    \item Level surfaces are cylinders with an elliptical cross section in the $ xy $
          plane. The cylindrical axis is the $ z $ axis (since it does not appear in the
          function)

    \item The equation can be rewritten as
          \begin{align}
              x^2 + y^2 & = (z - c)^2
          \end{align}
          This is an infinite cone centered at $ z = c $, since at constant $ z $, the
          cross-sections in the $ xy $ plane are
          \begin{align}
              x^2 + y^2 & = (z_0 - c)^2  \geq 0
          \end{align}

    \item For constant $ x $, the cross sections in the $ yz $ plane are
          \begin{align}
              y^2 & = z - (x_0^2 + c)
          \end{align}
          This is a parabola facing upward with vertex at $ z = x_0^2 + c $. In 3d,
          the shape is a paraboloid with vertex at $ z = c $

    \item At constant $ z $, the cross section in the $ xy $ plane are
          \begin{align}
              x - y^2 & = c
          \end{align}
          This is a parabola facing the positive $x$ direction with vertex at $ x = c $

    \item Plotting the \texttt{quiver} plot, to visualize the $ 2d $ vector field.
          \begin{figure}[H]
              \centering
              \begin{tikzpicture}
                  \def\U{1}
                  \def\V{1}
                  \def\LEN{sqrt(\U * \U + \V * \V)}
                  \begin{axis}[
                          legend pos = outer north east,
                          title = {$\vec{v} = \vec{i} + \vec{j}$},
                          width = 8cm,
                          height = 8cm,
                          Ani,
                          axis equal,
                          view     = {0}{90}, % for a view 'from above'
                          domain = -4:4,
                          restrict y to domain = -4:4,
                      ]
                      \addplot3 [
                          forget plot,
                          color = gray!50,
                          point meta = {\LEN},
                          quiver={u={(\U) / \LEN},
                                  v={(\V) / \LEN},
                                  scale arrows = 0.5,},
                          -stealth,
                          samples=12,
                      ] (x, y, 0);
                      \addplot[GraphSmooth, y_h] {x + 1};
                      \addplot[GraphSmooth, y_p] {x - 1};
                  \end{axis}
              \end{tikzpicture}
          \end{figure}

    \item Plotting the \texttt{quiver} plot, to visualize the $ 2d $ vector field.
          \begin{figure}[H]
              \centering
              \begin{tikzpicture}
                  \def\U{-y}
                  \def\V{x}
                  \def\LEN{sqrt(\U * \U + \V * \V)}
                  \begin{axis}[
                          legend pos = outer north east,
                          title = {$\vec{v} = -y\ \vec{i} + x\ \vec{j}$},
                          width = 8cm,
                          height = 8cm,
                          Ani,
                          axis equal,
                          view     = {0}{90}, % for a view 'from above'
                          domain = -4:4,
                          restrict y to domain = -4:4,
                      ]
                      \addplot3 [
                          forget plot,
                          color = gray!50,
                          point meta = {\LEN},
                          quiver={u={(\U) / \LEN},
                                  v={(\V) / \LEN},
                                  scale arrows = 0.5,},
                          -stealth,
                          samples=12,
                      ] (x, y, 0);
                      \addplot[GraphSmooth, y_h, variable = \t, domain = -pi:pi]
                      ({2*cos(t)}, {2*sin(t)});
                      \addplot[GraphSmooth, y_p, variable = \t, domain = -pi:pi]
                      ({3*cos(t)}, {3*sin(t)});
                  \end{axis}
              \end{tikzpicture}
          \end{figure}

    \item Plotting the \texttt{quiver} plot, to visualize the $ 2d $ vector field.
          \begin{figure}[H]
              \centering
              \begin{tikzpicture}
                  \def\U{0}
                  \def\V{x}
                  \def\LEN{sqrt(\U * \U + \V * \V)}
                  \begin{axis}[
                          legend pos = outer north east,
                          title = {$\vec{v} = x\ \vec{j}$},
                          width = 8cm,
                          height = 8cm,
                          Ani,
                          axis equal,
                          view     = {0}{90}, % for a view 'from above'
                          domain = -4:4,
                          restrict y to domain = -4:4,
                      ]
                      \addplot3 [
                          forget plot,
                          color = gray!50,
                          point meta = {\LEN},
                          quiver={u={(\U) / \LEN},
                                  v={(\V) / \LEN},
                                  scale arrows = 0.5,},
                          -stealth,
                          samples=12,
                      ] (x, y, 0);
                      \addplot [GraphSmooth, mark=none, y_h]
                      coordinates {(2, -4) (2, 4)};
                      \addplot [GraphSmooth, mark=none, y_p]
                      coordinates {(-2, -4) (-2, 4)};
                  \end{axis}
              \end{tikzpicture}
          \end{figure}

    \item Plotting the \texttt{quiver} plot, to visualize the $ 2d $ vector field.
          \begin{figure}[H]
              \centering
              \begin{tikzpicture}
                  \def\U{x}
                  \def\V{y}
                  \def\LEN{sqrt(\U * \U + \V * \V)}
                  \begin{axis}[
                          legend pos = outer north east,
                          title = {$\vec{v} = x\ \vec{i} + y\ \vec{j}$},
                          width = 8cm,
                          height = 8cm,
                          Ani,
                          axis equal,
                          view     = {0}{90}, % for a view 'from above'
                          domain = -4:4,
                          restrict y to domain = -4:4,
                      ]
                      \addplot3 [
                          forget plot,
                          color = gray!50,
                          point meta = {\LEN},
                          quiver={u={(\U) / \LEN},
                                  v={(\V) / \LEN},
                                  scale arrows = 0.5,},
                          -stealth,
                          samples=12,
                      ] (x, y, 0);
                      \addplot[GraphSmooth, y_h] {x};
                      \addplot[GraphSmooth, y_p] {-x};
                  \end{axis}
              \end{tikzpicture}
          \end{figure}

    \item Plotting the \texttt{quiver} plot, to visualize the $ 2d $ vector field.
          \begin{figure}[H]
              \centering
              \begin{tikzpicture}
                  \def\U{x}
                  \def\V{-y}
                  \def\LEN{sqrt(\U * \U + \V * \V)}
                  \begin{axis}[
                          legend pos = outer north east,
                          title = {$\vec{v} = x\ \vec{i} - y\ \vec{j}$},
                          width = 8cm,
                          height = 8cm,
                          Ani,
                          axis equal,
                          view     = {0}{90}, % for a view 'from above'
                          domain = -4:4,
                          restrict y to domain = -4:4,
                      ]
                      \addplot3 [
                          forget plot,
                          color = gray!50,
                          point meta = {\LEN},
                          quiver={u={(\U) / \LEN},
                                  v={(\V) / \LEN},
                                  scale arrows = 0.3,},
                          -stealth,
                          samples=12,
                      ] (x, y, 0);
                      \addplot[GraphSmooth, y_h] {2/x};
                      \addplot[GraphSmooth, y_p] {-2/x};
                  \end{axis}
              \end{tikzpicture}
          \end{figure}

    \item Plotting the \texttt{quiver} plot, to visualize the $ 2d $ vector field.
          \begin{figure}[H]
              \centering
              \begin{tikzpicture}
                  \def\U{y}
                  \def\V{-x}
                  \def\LEN{sqrt(\U * \U + \V * \V)}
                  \begin{axis}[
                          legend pos = outer north east,
                          title = {$\vec{v} = y\ \vec{i} - x\ \vec{j}$},
                          width = 8cm,
                          height = 8cm,
                          Ani,
                          axis equal,
                          view     = {0}{90}, % for a view 'from above'
                          domain = -4:4,
                          restrict y to domain = -4:4,
                      ]
                      \addplot3 [
                          forget plot,
                          color = gray!50,
                          point meta = {\LEN},
                          quiver={u={(\U) / \LEN},
                                  v={(\V) / \LEN},
                                  scale arrows = 0.5,},
                          -stealth,
                          samples=12,
                      ] (x, y, 0);
                      \addplot[GraphSmooth, y_h, variable = \t, domain = -pi:pi]
                      ({2*cos(t)}, {2*sin(t)});
                      \addplot[GraphSmooth, y_p, variable = \t, domain = -pi:pi]
                      ({3*cos(t)}, {3*sin(t)});
                  \end{axis}
              \end{tikzpicture}
          \end{figure}

    \item Using \texttt{quiver} to plot the vector field in $ 2d $
          \begin{enumerate}
              \item The level curves are parabolas facing upwards with vertices on the
                    y axis.
                    \begin{figure}[H]
                        \centering
                        \begin{tikzpicture}
                            \def\U{x}
                            \def\V{x*x}
                            \def\LEN{sqrt(\U * \U + \V * \V)}
                            \begin{axis}[
                                    legend pos = outer north east,
                                    title = {$\vec{v} = x\ \vec{i} + x^2\ \vec{j}$},
                                    width = 8cm,
                                    height = 8cm,
                                    Ani,
                                    axis equal,
                                    view     = {0}{90}, % for a view 'from above'
                                    domain = -4:4,
                                    restrict y to domain = -4:4,
                                ]
                                \addplot3 [
                                    forget plot,
                                    color = gray!50,
                                    point meta = {\LEN},
                                    quiver={u={(\U) / \LEN},
                                            v={(\V) / \LEN},
                                            scale arrows = 0.5,},
                                    -stealth,
                                    samples=12,
                                ] (x, y, 0);
                                \addplot[GraphSmooth, y_h]{x^2/2 - 1};
                                \addplot[GraphSmooth, y_p]{x^2/2 + 3};
                            \end{axis}
                        \end{tikzpicture}
                    \end{figure}

              \item The level curves are straight lines through the origin.
                    \begin{figure}[H]
                        \centering
                        \begin{tikzpicture}
                            \def\U{1/y}
                            \def\V{1/x}
                            \def\LEN{sqrt(\U * \U + \V * \V)}
                            \begin{axis}[
                                    legend pos = outer north east,
                                    title = {$\vec{v} = y^{-1}\ \vec{i}
                                                + x^{-1}\ \vec{j}$},
                                    width = 8cm,
                                    height = 8cm,
                                    Ani,
                                    axis equal,
                                    view     = {0}{90}, % for a view 'from above'
                                    domain = -4:4,
                                    restrict y to domain = -4:4,
                                ]
                                \addplot3 [
                                    forget plot,
                                    color = gray!50,
                                    point meta = {\LEN},
                                    quiver={u={(\U) / \LEN},
                                            v={(\V) / \LEN},
                                            scale arrows = 0.5,},
                                    -stealth,
                                    samples=12,
                                ] (x, y, 0);
                                \addplot[GraphSmooth, y_h]{x/3};
                                \addplot[GraphSmooth, y_p]{-x/3};
                            \end{axis}
                        \end{tikzpicture}
                    \end{figure}

              \item The level curves are of the form $ y = \ln \abs{\sec(x)} $.
                    \begin{figure}[H]
                        \centering
                        \begin{tikzpicture}
                            \def\U{cos(x)}
                            \def\V{sin(x)}
                            \def\LEN{sqrt(\U * \U + \V * \V)}
                            \begin{axis}[
                                    PiStyleX, xtick distance = pi,
                                    legend pos = outer north east,
                                    title = {$\vec{v} = \cos(x)\ \vec{i}
                                                + \sin(x)\ \vec{j}$},
                                    width = 8cm,
                                    height = 8cm,
                                    Ani,
                                    axis equal,
                                    view     = {0}{90}, % for a view 'from above'
                                    domain = -4:4,
                                    restrict y to domain = -4:4,
                                ]
                                \addplot3 [
                                    forget plot,
                                    color = gray!50,
                                    point meta = {\LEN},
                                    quiver={u={(\U) / \LEN},
                                            v={(\V) / \LEN},
                                            scale arrows = 0.5,},
                                    -stealth,
                                    samples=12,
                                ] (x, y, 0);
                                \addplot[GraphSmooth, y_h]{-ln(abs(cos(x)))};
                                \addplot[GraphSmooth, y_p]{-ln(abs(cos(x))) - 2};
                            \end{axis}
                        \end{tikzpicture}
                    \end{figure}

              \item The level curves are hyperbolas.
                    \begin{figure}[H]
                        \centering
                        \begin{tikzpicture}
                            \def\U{e^(-x^2 - y^2) * x}
                            \def\V{e^(-x^2 - y^2) * (-y)}
                            \def\LEN{sqrt(\U * \U + \V * \V)}
                            \begin{axis}[
                                    legend pos = outer north east,
                                    title = {$\vec{v} = xe^{-(x^2 + y^2)}\ \vec{i}
                                                - ye^{-(x^2 + y^2)}\ \vec{j}$},
                                    width = 8cm,
                                    height = 8cm,
                                    Ani,
                                    axis equal,
                                    view     = {0}{90}, % for a view 'from above'
                                    domain = -4:4,
                                    restrict y to domain = -4:4,
                                ]
                                \addplot3 [
                                    forget plot,
                                    color = gray!50,
                                    point meta = {\LEN},
                                    quiver={u={(\U) / \LEN},
                                            v={(\V) / \LEN},
                                            scale arrows = 0.4,},
                                    -stealth,
                                    samples=12,
                                ] (x, y, 0);
                                \addplot[GraphSmooth, y_h]{2/x};
                                \addplot[GraphSmooth, y_p]{-2/x};
                            \end{axis}
                        \end{tikzpicture}
                    \end{figure}
          \end{enumerate}

    \item Differentiating each component,
          \begin{align}
              \vec{v}   & = \begin{bNiceMatrix}[r, margin]
                                3\cos(2t) \\ 3\sin(2t) \\ 4t
                            \end{bNiceMatrix} &
              \vec{v}'  & = \begin{bNiceMatrix}[r, margin]
                                -6\sin(2t) \\ 6\cos(2t) \\ 4
                            \end{bNiceMatrix}  \\
              \vec{v}'' & = \begin{bNiceMatrix}[r, margin]
                                -12\cos(2t) \\ -12\sin(2t) \\ 0
                            \end{bNiceMatrix}
          \end{align}

    \item Differentiating each component,
          \begin{align}
              \vec{v}   & = \begin{bNiceMatrix}[r, margin]
                                3\cos(2t) \\ 3\sin(2t) \\ 4t
                            \end{bNiceMatrix} &
              \vec{v}'  & = \begin{bNiceMatrix}[r, margin]
                                -6\sin(2t) \\ 6\cos(2t) \\ 4
                            \end{bNiceMatrix}  \\
              \vec{v}'' & = \begin{bNiceMatrix}[r, margin]
                                -12\cos(2t) \\ -12\sin(2t) \\ 0
                            \end{bNiceMatrix}
          \end{align}

    \item Using the definition of the derivative as a limit,
          \begin{align}
              (\vec{v} \dotp \vec{u})' & = \lim_{\Delta t \to 0}
              \frac{\vec{v}(t + \Delta t) \dotp \vec{u}(t + \Delta t) -
              \vec{v}(t) \dotp \vec{u}(t)}{\Delta t}             \\
                                       & = \lim_{\Delta t \to 0}
              \frac{\vec{v}(t + \Delta t) \dotp \vec{u}(t + \Delta t) -
              \vec{v}(t + \Delta t) \dotp \vec{u}(t)}{\Delta t}  \\
                                       & + \lim_{\Delta t \to 0}
              \frac{\vec{v}(t + \Delta t) \dotp \vec{u}(t) -
              \vec{v}(t) \dotp \vec{u}(t)}{\Delta t}             \\
                                       & = \lim_{\Delta t \to 0}
              \vec{v}(t + \Delta t) \dotp \vec{u}'(t) +
              \lim_{\Delta t \to 0} \vec{v}'(t) \dotp \vec{u}(t) \\
                                       & =
              \vec{v}(t) \dotp \vec{u}'(t) + \vec{v}'(t) \dotp \vec{u}(t)
          \end{align}
          Using the definition of the derivative as a limit,
          \begin{align}
              (\vec{v} \times \vec{u})' & = \lim_{\Delta t \to 0}
              \frac{\vec{v}(t + \Delta t) \times \vec{u}(t + \Delta t) -
              \vec{v}(t) \times \vec{u}(t)}{\Delta t}             \\
                                        & = \lim_{\Delta t \to 0}
              \frac{\vec{v}(t + \Delta t) \times \vec{u}(t + \Delta t) -
              \vec{v}(t + \Delta t) \times \vec{u}(t)}{\Delta t}  \\
                                        & + \lim_{\Delta t \to 0}
              \frac{\vec{v}(t + \Delta t) \times \vec{u}(t) -
              \vec{v}(t) \times \vec{u}(t)}{\Delta t}             \\
                                        & = \lim_{\Delta t \to 0}
              \vec{v}(t + \Delta t) \times \vec{u}'(t) +
              \lim_{\Delta t \to 0} \vec{v}'(t) \times \vec{u}(t) \\
                                        & =
              \vec{v}(t) \times \vec{u}'(t) + \vec{v}'(t) \times \vec{u}(t)
          \end{align}

          Using the above two results,
          \begin{align}
              [\vec{a} \dotp (\vec{b} \times \vec{c})]' & = \vec{a}' \dotp
              [\vec{b} \times \vec{c}] + \vec{a} \dotp [\vec{b} \times \vec{c}]' \\
                                                        & = \vec{a}' \dotp
              (\vec{b} \times \vec{c}) + \vec{a} \dotp (\vec{b}' \times \vec{c})
              + \vec{a} \dotp (\vec{b} \times \vec{c}')
          \end{align}

          Examples TBC.

    \item Finding the component-wise partial derivatives,
          \begin{align}
              \vec{v}            & = \bmatcol{e^x \cos y}{e^x \sin y}           &
              \diffp{\vec{v}}{x} & = \bmatcol{e^x \cos y}{e^x \sin y}             \\
              \diffp{\vec{v}}{y} & = \bmatcol{-e^x \sin y}{e^x \cos y}            \\
              \vec{u}            & = \bmatcol{\cos x \cosh y}{-\sin x \sinh y}  &
              \diffp{\vec{u}}{x} & = \bmatcol{-\sin x \cosh y}{-\cos x \sinh y}   \\
              \diffp{\vec{u}}{y} & = \bmatcol{\cos x \sinh y}{-\sin x \cosh y}
          \end{align}

    \item TBC. Refer notes.
\end{enumerate}