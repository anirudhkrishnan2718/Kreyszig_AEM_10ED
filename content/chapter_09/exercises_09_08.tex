\section{Divergence of a Vector Field}

\begin{enumerate}
    \item Finding the divergence,
          \begin{align}
              \vec{v}                          & = \begin{bNiceMatrix}[margin]
                                                       x^2 \\ 4y^2 \\ 9z^2
                                                   \end{bNiceMatrix} &
              P                                & : (-1, 0, 0.5)                  \\
              \nabla \dotp \vec{v}             & = 2x + 8y + 18z               &
              \Big[\nabla \dotp \vec{v}\Big]_P & = 7
          \end{align}

    \item Finding the divergence,
          \begin{align}
              \vec{v}                          & = \begin{bNiceMatrix}[margin]
                                                       0 \\ \cos(xyz) \\ \sin(xyz)
                                                   \end{bNiceMatrix}     &
              P                                & : (2, \pi/2, 0)                     \\
              \nabla \dotp \vec{v}             & = 0  -xz \sin(xyz) + xy \cos(xyz) &
              \Big[\nabla \dotp \vec{v}\Big]_P & = \pi
          \end{align}

    \item Finding the divergence,
          \begin{align}
              \vec{v}              & = \frac{1}{x^2 + y^2}
              \bmatcol{x}{y}                                                       \\
              \nabla \dotp \vec{v} & = \frac{x^2 - y^2}{(x^2 + y^2)^2}\ (-1 + 1) &
                                   & = 0
          \end{align}
          Field is solenoidal

    \item Finding the divergence,
          \begin{align}
              \vec{v}                             & =
              \begin{bNiceMatrix}[margin]
                  v_1(y, z) \\ v_2(z, x) \\ v_3(x, y)
              \end{bNiceMatrix} &
              P                                   & : (3, 1, -1)       \\
              \nabla \dotp \vec{v}                & = 0  + 0 + 0 = 0 &
              \Big[\nabla \dotp \vec{v}\Big]_P    & = 0
          \end{align}
          Field is solenoidal

    \item Finding the divergence,
          \begin{align}
              \vec{v}                          & = x^2y^2z^2
              \begin{bNiceMatrix}[margin]
                  x \\ y \\ z
              \end{bNiceMatrix}      &
              P                                & : (3, -1, 4)                  \\
              \nabla \dotp \vec{v}             & = y^2z^2(3x^2) + x^2z^2(3y^2)
              + x^2y^2(3z^2)                   &
                                               & = (3xyz)^2                    \\
              \Big[\nabla \dotp \vec{v}\Big]_P & = 1296
          \end{align}

    \item Finding the divergence,
          \begin{align}
              \vec{v}              & = (x^2 + y^2 + z^2)^{-3/2}
              \begin{bNiceMatrix}[margin]
                  x \\ y \\ z
              \end{bNiceMatrix}                        \\
              \nabla \dotp \vec{v} & = \frac{1}
              {(x^2 + y^2 + z^2)^{5/2}} \left[ (y^2 + z^2 - 2x^2) + (x^2 + z^2 - 2y^2)
              + (x^2 + y^2 - 2z^2) \right]                      \\
                                   & = 0
          \end{align}
          Field is solenoidal

    \item Finding $ v_3 $, given the field is solenoidal
          \begin{align}
              \vec{v}              & = \begin{bNiceMatrix}[margin]
                                           e^x \cos(y) \\ e^x \sin(y) \\ v_3
                                       \end{bNiceMatrix}                \\
              \nabla \dotp \vec{v} & = e^x \cos(y) + e^x \cos(y) + \diffp {v_3}{z} = 0 \\
              \diffp{v_3}{z}       & = -2e^x \cos(y)                                   \\
              z                    & = (-2e^x \cos(y))\ z + f(x, y)
          \end{align}

    \item Finding the condition on $ v_3 $,
          \begin{enumerate}
              \item divergence is always positive
                    \begin{align} \vec{v}                     & =
              \begin{bNiceMatrix}[margin]
                                      x \\ y \\ v_3
                                  \end{bNiceMatrix} &
              \nabla \dotp \vec{v}        & = 1 + 1 + \diffp{v_3}{z} > 0 \\
              v_3                         & > -2z + c + f(x, y)
                    \end{align}

              \item divergence depends on $ \abs{z} $
                    \begin{align} \vec{v}                     & =
              \begin{bNiceMatrix}[margin]
                                      x \\ y \\ v_3
                                  \end{bNiceMatrix} &
              \nabla \dotp \vec{v}        & = 1 + 1 + \diffp{v_3}{z}                  \\
              v_3                         & = -z \left( 1 + \frac{\abs{z}}{2} \right)
              + c + f(x, y)               &
              \diffp{v_3}{z}              & = -\abs{z} - 1                            \\
              \nabla \dotp \vec{v}        & = 1 - \abs{z} =
              \begin{dcases}
                                      > 0 & \quad \abs{z} < 1 \\
                                      < 0 & \quad \abs{z} > 1 \\
                                  \end{dcases}
                    \end{align}
          \end{enumerate}

    \item Proving the relations,
          \begin{enumerate}
              \item Scalar multiplication,
                    \begin{align}
                        \nabla \dotp (k\vec{v}) & = \diffp{(kv_1)}{x}
                        + \diffp{(kv_2)}{y}
                        + \diffp{(kv_3)}{z}     &
                                                & = k \left[ \diffp{v_1}{x}
                            + \diffp{v_2}{y}
                        + \diffp{v_3}{z} \right]                            \\
                                                & = k\ \nabla \dotp \vec{v}
                    \end{align}

              \item Scalar function $ f $
                    \begin{align}
                        \nabla \dotp (f\vec{v}) & = \diffp{(fv_1)}{x}
                        + \diffp{(fv_2)}{y}
                        + \diffp{(fv_3)}{z}                                 \\
                                                & = f \left[ \diffp{v_1}{x}
                            + \diffp{v_2}{y}
                            + \diffp{v_3}{z} \right] + v_1\ \diffp fx + v_2\ \diffp fy
                        + v_3\ \diffp fz                                    \\
                                                & = f\ \nabla \dotp \vec{v}
                        + \vec{v} \dotp (\nabla f)
                    \end{align}
                    Verifying this result,
                    \begin{align}
                        f                       & = \exp(xyz)                   \\
                        \vec{v}                 & = \begin{bNiceMatrix}[margin]
                                                        ax \\ by \\ cz
                                                    \end{bNiceMatrix}  \\
                        \nabla \dotp (f\vec{v}) & = e^{xyz}[a + axyz + b + byzx
                        + c + czxy]                                             \\
                                                & =
                        e^{xyz} [a + b + c] + ax(yz\ e^{xyz}) + by(xz\ e^{xyz})
                        + cz(xy\ e^{xyz})                                       \\
                                                & = f\ (\nabla \dotp \vec{v})
                        + \vec{v} \dotp (\nabla f)
                    \end{align}
                    Solving problem $ 6 $ using this result,
                    \begin{align}
                        f                        & = (x^2 + y^2 + z^2)^{-3/2}      &
                        \vec{v}                  & = \begin{bNiceMatrix}[margin]
                                                         x \\ y \\ z
                                                     \end{bNiceMatrix}      \\
                        f\ (\nabla \dotp v)      & = 3f                            &
                        \vec{v} \dotp (\nabla f) & = \frac{-3f\ (x^2 + y^2 + z^2)}
                        {(x^2 + y^2 + z^2)}                                          \\
                        \nabla \dotp (f\vec{v})  & = 0
                    \end{align}

              \item Scalar functions $ f, g $
                    \begin{align}
                        \nabla \dotp (f \nabla g) & =
                        \diffp*{\left( f\diffp gx \right)}{x}
                        + \diffp*{\left( f\diffp gy \right)}{y}
                        + \diffp*{\left( f\diffp gz \right)}{z}                \\
                                                  & = f \left[ \diffp[2]{g}{x}
                            + \diffp[2]{g}{y}
                            + \diffp[2]{g}{z} \right] + v_1\ \diffp fx + v_2\ \diffp fy
                        + v_3\ \diffp fz                                       \\
                                                  & = f\ (\nabla^2 g)
                        + \nabla f \dotp \nabla g
                    \end{align}

                    Verifying the result for the given functions,
                    \begin{align}
                        f                          & = x^2 - y^2                \\
                        g                          & = \exp(x + y)              \\
                        f\ \nabla^2 g              & = (x^2 - y^2)
                        \left[ 2\exp(x + y)  \right]                            \\
                        \nabla f \dotp \nabla g    & = \exp(x + y)(2x - 2y)     \\
                        f\ (\nabla g)              & = x^2 - y^2
                        \bmatcol{\exp(x+y)}{\exp(x+y)}                          \\
                        \nabla \dotp (f\ \nabla g) & = \exp(x+y)[x^2 - y^2 + 2x
                        + x^2 - y^2 - 2y]                                       \\
                                                   & = f\ (\nabla^2 g)
                        + \nabla f \dotp \nabla g
                    \end{align}

              \item Using the result from part $ c $,
                    \begin{align}
                        \nabla \dotp (f \nabla g) - \nabla \dotp (g \nabla f) & =
                        f\ (\nabla^2 g) + \nabla f \dotp \nabla g                 \\
                                                                              & -
                        g\ (\nabla^2 f) - \nabla g \dotp \nabla f                 \\
                                                                              & =
                        f\ (\nabla^2 g) - g\ (\nabla^2 f)
                    \end{align}

              \item Other examples TBC.
          \end{enumerate}

    \item Graphing the velocity fields,
          \begin{figure}[H]
              \centering
              \begin{subfigure}[b]{0.49\textwidth}
                  \begin{tikzpicture}
                      \def\U{1}
                      \def\V{0}
                      \begin{axis}[
                              legend pos = outer north east,
                              title = {$\vec{v} = \vec{\hat{i}}$},
                              width = 8cm,
                              height = 8cm,
                              Ani,
                              axis equal,
                              view     = {0}{90}, % for a view 'from above'
                              domain = -2:2,
                              restrict y to domain = -2:2,
                          ]
                          \addplot3 [
                              forget plot,
                              color = black,
                              quiver={u={(\U)},
                                      v={(\V)},
                                      scale arrows = 0.25,},
                              -stealth,
                              samples=11,
                          ] (x, y, 0);
                          \draw[thick, y_p]
                          (1, 1, 0) -- (1, -1, 0) -- (-1, -1, 0) -- (-1, 1, 0) -- cycle;
                      \end{axis}
                  \end{tikzpicture}
              \end{subfigure}
              \hfill
              \begin{subfigure}[b]{0.49\textwidth}
                  \begin{tikzpicture}
                      \def\U{x}
                      \def\V{0}
                      \begin{axis}[
                              legend pos = outer north east,
                              title = {$\vec{v} = x\ \vec{\hat{i}}$},
                              width = 8cm,
                              height = 8cm,
                              Ani,
                              axis equal,
                              view     = {0}{90}, % for a view 'from above'
                              domain = -2:2,
                              restrict y to domain = -2:2,
                          ]
                          \addplot3 [
                              forget plot,
                              color = black,
                              quiver={u={(\U)},
                                      v={(\V)},
                                      scale arrows = 0.2,},
                              -stealth,
                              samples=11,
                          ] (x, y, 0);
                          \draw[thick, y_p]
                          (1, 1, 0) -- (1, -1, 0) -- (-1, -1, 0) -- (-1, 1, 0) -- cycle;
                      \end{axis}
                  \end{tikzpicture}
              \end{subfigure}
          \end{figure}
          \begin{figure}[H]
              \centering
              \begin{subfigure}[b]{0.49\textwidth}
                  \begin{tikzpicture}
                      \def\U{x}
                      \def\V{-y}
                      \begin{axis}[
                              legend pos = outer north east,
                              title = {$\vec{v} = x\ \vec{i} - y\ \vec{\hat{j}}$},
                              width = 8cm,
                              height = 8cm,
                              Ani,
                              axis equal,
                              view     = {0}{90}, % for a view 'from above'
                              domain = -2:2,
                              restrict y to domain = -2:2,
                          ]
                          \addplot3 [
                              forget plot,
                              color = black,
                              quiver={u={(\U)},
                                      v={(\V)},
                                      scale arrows = 0.2,},
                              -stealth,
                              samples=11,
                          ] (x, y, 0);
                          \draw[thick, y_p]
                          (1, 1, 0) -- (1, -1, 0) -- (-1, -1, 0) -- (-1, 1, 0) -- cycle;
                      \end{axis}
                  \end{tikzpicture}
              \end{subfigure}
              \hfill
              \begin{subfigure}[b]{0.49\textwidth}
                  \begin{tikzpicture}
                      \def\U{x}
                      \def\V{y}
                      \begin{axis}[
                              legend pos = outer north east,
                              title = {$\vec{v} = x\ \vec{i} + y\ \vec{\hat{j}}$},
                              width = 8cm,
                              height = 8cm,
                              Ani,
                              axis equal,
                              view     = {0}{90}, % for a view 'from above'
                              domain = -2:2,
                              restrict y to domain = -2:2,
                          ]
                          \addplot3 [
                              forget plot,
                              color = black,
                              quiver={u={(\U)},
                                      v={(\V)},
                                      scale arrows = 0.15,},
                              -stealth,
                              samples=11,
                          ] (x, y, 0);
                          \draw[thick, y_p]
                          (1, 1, 0) -- (1, -1, 0) -- (-1, -1, 0) -- (-1, 1, 0) -- cycle;
                      \end{axis}
                  \end{tikzpicture}
              \end{subfigure}
          \end{figure}
          \begin{figure}[H]
              \centering
              \begin{subfigure}[b]{0.49\textwidth}
                  \begin{tikzpicture}
                      \def\U{-x}
                      \def\V{-y}
                      \begin{axis}[
                              legend pos = outer north east,
                              title = {$\vec{v} = -x\ \vec{i} - y\ \vec{\hat{j}}$},
                              width = 8cm,
                              height = 8cm,
                              Ani,
                              axis equal,
                              view     = {0}{90}, % for a view 'from above'
                              domain = -2:2,
                              restrict y to domain = -2:2,
                          ]
                          \addplot3 [
                              forget plot,
                              color = black,
                              quiver={u={(\U)},
                                      v={(\V)},
                                      scale arrows = 0.2,},
                              -stealth,
                              samples=11,
                          ] (x, y, 0);
                          \draw[thick, y_p]
                          (1, 1, 0) -- (1, -1, 0) -- (-1, -1, 0) -- (-1, 1, 0) -- cycle;
                      \end{axis}
                  \end{tikzpicture}
              \end{subfigure}
              \hfill
              \begin{subfigure}[b]{0.49\textwidth}
                  \begin{tikzpicture}
                      \def\U{-y / (x^2 + y^2)}
                      \def\V{x / (x^2 + y^2)}
                      \begin{axis}[
                              legend pos = outer north east,
                              title = {$\vec{v} = \frac{1}{x^2 + y^2}\ (-y\ \vec{i}
                                          + x\ \vec{\hat{j}})$},
                              width = 8cm,
                              height = 8cm,
                              Ani,
                              axis equal,
                              view     = {0}{90}, % for a view 'from above'
                              domain = -2:2,
                              restrict y to domain = -2:2,
                          ]
                          \addplot3 [
                              forget plot,
                              color = black,
                              quiver={u={(\U)},
                                      v={(\V)},
                                      scale arrows = 0.15,},
                              -stealth,
                              samples=11,
                          ] (x, y, 0);
                          \draw[thick, y_p]
                          (1, 1, 0) -- (1, -1, 0) -- (-1, -1, 0) -- (-1, 1, 0) -- cycle;
                      \end{axis}
                  \end{tikzpicture}
              \end{subfigure}
          \end{figure}

          Flux from looking at the vector fields can be estimated to be either positive,
          negative or zero in the figures.

    \item For incompressible flow,
          \begin{align}
              \vec{v} = \vec{r}'   & = \begin{bNiceMatrix}[margin]
                                           y \\ 0 \\ 0
                                       \end{bNiceMatrix} &
              \nabla \dotp \vec{v} & = 0                          \\
              \vec{r}(t)           & = \begin{bNiceMatrix}[margin]
                                           x(t) \\ y(t) \\ z(t)
                                       \end{bNiceMatrix}
              = \begin{bNiceMatrix}[margin]
                    x_0 + y_0 t \\ y_0 \\ z_0
                \end{bNiceMatrix}
          \end{align}
          Consider the cube as a stack of $ xz $ planes. The planes move in the $ x $
          direction at different speeds depending on their initial $ y $ coordinate. \par
          The points have no motion in the $ x $ or $ z $ directions. This means that
          each plane simply moves in the $ x $ direction by a different amount.

          \begin{figure}[H]
              \centering
              \begin{tikzpicture}[scale = 2]
                  \draw[black, thick] (0,0) -- (0, 1) -- (1, 1) -- (1, 0) -- cycle;
                  \draw[thick, dashed, y_p] (0,0) -- (1, 1) -- (2, 1) -- (1, 0) -- cycle;
              \end{tikzpicture}
          \end{figure}
          The area of a parallelogram made from deforming a square is the same as that
          square. This means that the volume of the fluid at $ t = 1 $ remains the same
          as the initial cube.

    \item For compressible flow,
          \begin{align}
              \vec{v} = \vec{r}' & = \begin{bNiceMatrix}[margin]
                                         x(t) \\ 0 \\ 0
                                     \end{bNiceMatrix} &
              \vec{r}(t)         & = \begin{bNiceMatrix}[margin]
                                         x(t) \\ y(t) \\ z(t)
                                     \end{bNiceMatrix}
              = \begin{bNiceMatrix}[margin]
                    c_1 \exp(t) \\ c_2 \\ c_3
                \end{bNiceMatrix}
          \end{align}
          Consider the cube as a stack of $ yz $ planes. The planes move in the $ x $
          direction at different speeds depending on their $ x $ coordinate. \par
          The points have no motion in the $ y $ or $ z $ directions. This means that
          each plane simply moves in the $ x $ direction by a different amount.

          For the slice initially at $ x = 1 $, $ c_1 = 1 $. This slice has moved to
          $ x = e $ at time $ t = 1 $. Since the slice at $ x = 0 $ never moves, the
          volume is a cuboid of dimensions $ e \times 1 \times 1 $. \par

    \item For rotational flow in a cylinder about the $ z $ axis, with constant angular
          velocity $ \boldsymbol{\omega} $
          \begin{align}
              \vec{r}              & = \begin{bNiceMatrix}[margin, r]
                                           x \\ y \\ z
                                       \end{bNiceMatrix}       &
              \vec{v}              & = \boldsymbol{\omega} \times \vec{r} =
              \begin{vNiceMatrix}[margin, r]
                  \vec{\hat{i}} & \vec{\hat{j}} & \vec{\hat{k}} \\
                  0             & 0             & \omega        \\
                  x             & y             & z
              \end{vNiceMatrix} =
              \begin{bNiceMatrix}[margin, r]
                  -\omega y \\ \omega x \\ 0
              \end{bNiceMatrix}                                 \\
              \nabla \dotp \vec{v} & = 0
          \end{align}
          This flow is solenoidal. This is an incompressible flow in a circle centered
          around the origin, which has zero flux through its perimeter.

    \item Considering only the first component,
          \begin{align}
              \nabla \dotp \vec{v} & = \diffp {v_1}{x} + \dots &
              \nabla \dotp \vec{u} & = \diffp {u_1}{x} + \dots   \\
              u_1 = v_1 + f(y, z) + c
          \end{align}
          There is very little constraint on $ \vec{u} $ to satisfy this condition.

    \item Using the definition of the Laplacian,
          \begin{align}
              f                        & = \cos^2(x) + \sin^2(y)                &
              \nabla f                 & = \bmatcol{-\sin(2x)}{\sin(2y)}          \\
              \nabla \dotp (\nabla f)  & = \color{y_h} -2\cos(2x) + 2\cos(2y)     \\
              \nabla^2 f               & = \diffp*{[-\sin(2x)]}{x}
              + \diffp*{[\sin(2y)]}{y} &
                                       & = \color{y_p} -2 \cos(2x) + 2 \cos(2y)
          \end{align}

    \item Using the definition of the Laplacian,
          \begin{align}
              f                       & = \exp(xyz)                              &
              \nabla f                & = e^{xyz} \begin{bNiceMatrix}[margin, r]
                                                      yz \\ xz \\ xy
                                                  \end{bNiceMatrix}    \\
              \nabla \dotp (\nabla f) & = \color{y_h} e^{xyz}
              \Big[(yz)^2 + (xz)^2 + (xy)^2\Big]                                   \\
              \nabla^2 f              & = \diffp*{\Big[yz\ e^{xyz}\Big]}{x}
              + \diffp*{\Big[xz\ e^{xyz}\Big]}{y}
              + \diffp*{\Big[xy\ e^{xyz}\Big]}{z}                                  \\
                                      & = \color{y_p} e^{xyz}
              \Big[(yz)^2 + (xz)^2 + (xy)^2\Big]
          \end{align}

    \item Using the definition of the Laplacian,
          \begin{align}
              f                       & = \ln(x^2 + y^2)                      \\
              \nabla f                & = \frac{1}{x^2 + y^2}\bmatcol{2x}{2y} \\
              \nabla \dotp (\nabla f) & = \frac{2x^2 + 2y^2 - 4x^2}
              {(x^2 + y^2)^2} + \frac{2x^2 + 2y^2 - 4y^2}
              {(x^2 + y^2)^2} = \color{y_h} 0                                 \\
              \nabla^2 f              & =
              \diffp*{\Bigg[ \frac{2x}{x^2 + y^2} \Bigg]}{x}
              + \diffp*{\Bigg[ \frac{2y}{x^2 + y^2} \Bigg]}{y}                \\
                                      & = \frac{2x^2 + 2y^2 - 4x^2}
              {(x^2 + y^2)^2} + \frac{2x^2 + 2y^2 - 4y^2}
              {(x^2 + y^2)^2} = \color{y_p} 0
          \end{align}

    \item Using the definition of the Laplacian,
          \begin{align}
              f                       & = z - \sqrt{x^2 + y^2} = z - r      \\
              \nabla f                & = \begin{bNiceMatrix}[margin, r]
                                              -x\ (x^2 + y^2)^{-1/2} \\
                                              -y\ (x^2 + y^2)^{-1/2} \\
                                              1
                                          \end{bNiceMatrix}     \\
              \nabla \dotp (\nabla f) & = \frac{-r^2 + x^2- r^2 + y^2}{r^3}
              = \color{y_h} \frac{-1}{\sqrt{x^2 + y^2}}                     \\
              \nabla^2 f              & =
              \diffp*{\Bigg[ \frac{-x}{r} \Bigg]}{x}
              + \diffp*{\Bigg[ \frac{-y}{r} \Bigg]}{y}                      \\
                                      & = \frac{-r^2 + x^2}{r^3}
              + \frac{-r^2 + y^2}{r^3} = \color{y_p} \frac{-1}{\sqrt{x^2 + y^2}}
          \end{align}

    \item Using the definition of the Laplacian,
          \begin{align}
              f                       & = \frac{1}{x^2 + y^2 + z^2} = \frac{1}{r^2}   \\
              \nabla f                & = \frac{-2}{r^4}\begin{bNiceMatrix}[margin, r]
                                                            x \\
                                                            y \\
                                                            z \\
                                                        \end{bNiceMatrix} \\
              \nabla \dotp (\nabla f) & = \frac{-2r^2 + 8x^2}{r^6}
              + \frac{-2r^2 + 8y^2}{r^6} + \frac{-2r^2 + 8z^2}{r^6}
              = \color{y_h} \frac{2}{(x^2 + y^2 + z^2)^2}                             \\
              \nabla^2 f              & =
              \diffp*{\Bigg[ \frac{-2x}{r^4} \Bigg]}{x}
              + \diffp*{\Bigg[ \frac{-2y}{r^4} \Bigg]}{y}
              + \diffp*{\Bigg[ \frac{-2z}{r^4} \Bigg]}{z}                             \\
                                      & = \frac{-2r^2 + 8x^2}{r^6}
              + \frac{-2r^2 + 8y^2}{r^6} + \frac{-2r^2 + 8z^2}{r^6}
              = \color{y_p} \frac{2}{(x^2 + y^2 + z^2)^2}
          \end{align}

    \item Using the definition of the Laplacian,
          \begin{align}
              f                       & = e^{2x} \cosh(2y)                             \\
              \nabla f                & = \bmatcol{2e^{2x}\cosh(2y)}{2e^{2x}\sinh(2y)} \\
              \nabla \dotp (\nabla f) & = 4e^{2x}\cosh(2y) + 4e^{2x}\cosh(2y)
              = \color{y_h} 8e^{2x}\cosh(2y)                                           \\
              \nabla^2 f              & = \color{y_p} 8e^{2x}\cosh(2y)
          \end{align}
\end{enumerate}