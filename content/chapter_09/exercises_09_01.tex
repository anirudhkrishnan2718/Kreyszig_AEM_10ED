\section{Vectors in 2-Space and 3-Space}

\begin{enumerate}
    \item Finding the vector using the given points,
          \begin{align}
              P                     & = (1, 1, 0)                              &
              Q                     & = (6, 2, 0)                                \\
              (v_x, v_y, v_z)       & = (5, 1, 0)                              &
              \abs{ \vec{v} } & = \sqrt{5^2 + 1^2} = \sqrt{26}             \\
              \vec{v}               & = \begin{bNiceMatrix}
                                            5 \\ 1 \\ 0
                                        \end{bNiceMatrix}                    &
              \vec{u}               & = \frac{1}{\sqrt{26}}\begin{bNiceMatrix}
                                                               5 \\ 1 \\ 0
                                                           \end{bNiceMatrix}
          \end{align}

    \item Finding the vector using the given points,
          \begin{align}
              P                     & = (1, 1, 1)                             &
              Q                     & = (2, 2, 0)                               \\
              (v_x, v_y, v_z)       & = (1, 1, -1)                            &
              \abs{ \vec{v} } & = \sqrt{1^2 + 1^2 + (-1)^2} = \sqrt{3}    \\
              \vec{v}               & = \begin{bNiceMatrix}
                                            1 \\ 1 \\ -1
                                        \end{bNiceMatrix}                   &
              \vec{u}               & = \frac{1}{\sqrt{3}}\begin{bNiceMatrix}
                                                              1 \\ 1 \\ -1
                                                          \end{bNiceMatrix}
          \end{align}

    \item Finding the vector using the given points,
          \begin{align}
              P                     & = (-3, 4, -0.5)                                &
              Q                     & = (5.5, 0, 1.2)                                  \\
              (v_x, v_y, v_z)       & = (8.5, -4, 1.7)                               &
              \abs{ \vec{v} } & = \sqrt{8.5^2 + (-4)^2 + 1.7^2} = \sqrt{91.14}   \\
              \vec{v}               & = \begin{bNiceMatrix}
                                            8.5 \\ -4 \\ 1.7
                                        \end{bNiceMatrix}                          &
              \vec{u}               & = \frac{1}{\sqrt{91.14}}\begin{bNiceMatrix}
                                                                  8.5 \\ -4 \\ 1.7
                                                              \end{bNiceMatrix}
          \end{align}

    \item Finding the vector using the given points,
          \begin{align}
              P                     & = (1, 4, 2)                              &
              Q                     & = (-1, -4, -2)                             \\
              (v_x, v_y, v_z)       & = (-2, -8, -4)                           &
              \abs{ \vec{v} } & = \sqrt{(-2)^2 + (-8)^2 + (-4)^2}
              = \sqrt{84}                                                        \\
              \vec{v}               & = \begin{bNiceMatrix}
                                            -2 \\ -8 \\ -4
                                        \end{bNiceMatrix}                    &
              \vec{u}               & = \frac{1}{\sqrt{84}}\begin{bNiceMatrix}
                                                               -2 \\ -8 \\ -4
                                                           \end{bNiceMatrix}
          \end{align}

    \item Finding the vector using the given points,
          \begin{align}
              P                     & = (0, 0, 0)                      &
              Q                     & = (2, 1, -2)                       \\
              (v_x, v_y, v_z)       & = (2, 1, -2)                     &
              \abs{ \vec{v} } & = \sqrt{2^2 + (-1)^2 + 2^2}
              = 3                                                        \\
              \vec{v}               & = \begin{bNiceMatrix}
                                            2 \\ 1 \\ -2
                                        \end{bNiceMatrix}            &
              \vec{u}               & = \frac{1}{3}\begin{bNiceMatrix}
                                                       2 \\ 1 \\ -2
                                                   \end{bNiceMatrix}
          \end{align}

    \item Finding the terminal point,
          \begin{align}
              P                     & = (0, 2, 13)                        &
              (v_x, v_y, v_z)       & = (4, 0, 0)                           \\
              \abs{ \vec{v} } & = \sqrt{4^2} =  4                   &
              Q                     & = (v_x + P_x, v_y + P_y, v_z + P_z)   \\
              Q                     & = (4, 2, 13)
          \end{align}

    \item Finding the terminal point,
          \begin{align}
              P                     & = \left(\frac{7}{2}, -3, \frac{3}{4}\right) &
              (v_x, v_y, v_z)       & = \left(\frac{1}{2}, 3, -\frac{1}{4}\right)   \\
              \abs{ \vec{v} } & = \sqrt{\frac{1}{4} + 3^2 + \frac{1}{16}}
              =  4                  &
              Q                     & = (v_x + P_x, v_y + P_y, v_z + P_z)           \\
              Q                     & = \left(4, 0, \frac{1}{2}\right)
          \end{align}

    \item Finding the terminal point,
          \begin{align}
              P                       & = (0, 0, 0)                         &
              (v_x, v_y, v_z)         & = (13.1, 0.8, -2)                     \\
              \abs{ \vec{v} }   & = \sqrt{13.1^2 + 0.8^2 + (-2)^2}
              =  \sqrt{\frac{705}{2}} &
              Q                       & = (v_x + P_x, v_y + P_y, v_z + P_z)   \\
              Q                       & = (13.1, 0.8, -2)
          \end{align}

    \item Finding the terminal point,
          \begin{align}
              P                     & = (-6, -1, -4)                      &
              (v_x, v_y, v_z)       & = (6, 1, -4)                          \\
              \abs{ \vec{v} } & = \sqrt{(-8)^2}
              =  8                  &
              Q                     & = (v_x + P_x, v_y + P_y, v_z + P_z)   \\
              Q                     & = (0, 0, -8)
          \end{align}

    \item Finding the terminal point,
          \begin{align}
              P                     & = (0, 3, -3)                        &
              (v_x, v_y, v_z)       & = (0, -3, 3)                          \\
              \abs{ \vec{v} } & = \sqrt{(-6)^2 + 6^2}
              =  \sqrt{72}          &
              Q                     & = (v_x + P_x, v_y + P_y, v_z + P_z)   \\
              Q                     & = (0, -6, 0)
          \end{align}

    \item Performing the given computations,
          \begin{align}
              2\vec{a}             & = (6, 4, 0)                        &
              \frac{1}{2}\ \vec{a} & = \left( \frac{3}{2}, 1, 0 \right)   \\
              -\vec{a}             & = (-3, -2, 0)                      &
          \end{align}

    \item Performing the given computations,
          \begin{align}
              (\vec{a} + \vec{b})           & = (-1, 8, 0) &
              (\vec{a} + \vec{b}) + \vec{c} & = (4, 7, 8)    \\
              (\vec{b} + \vec{c})           & = (1, 5, 8)  &
              \vec{a} + (\vec{b} + \vec{c}) & = (4, 7, 8)    \\
          \end{align}

    \item Performing the given computations,
          \begin{align}
              \vec{b} + \vec{c} & = (1, 5, 8) &
              \vec{c} + \vec{b} & = (1, 5, 8)
          \end{align}

    \item Performing the given computations,
          \begin{align}
              3\vec{c} - 6\vec{d}   & = (15, -3, 24) - (0, 0, 24) = (15, -3, 0) \\
              3(\vec{c} - 2\vec{d}) & = 3 \cdot (5, -1, 0) = (15, -3, 0)
          \end{align}

    \item Performing the given computations,
          \begin{align}
              7\vec{c} - 7\vec{b}   & = (15, -3, 24) - (0, 0, 24) = (15, -3, 0) \\
              3(\vec{c} - 2\vec{d}) & = 3 \cdot (5, -1, 0) = (15, -3, 0)
          \end{align}

    \item Performing the given computations,
          \begin{align}
              \frac{9}{2}\vec{a} - 3\vec{c}                                 &
              = \left( \frac{27}{2}, 9, 0 \right) - (15, -3, 24)
              = \left( -\frac{3}{2}, 12, -24 \right)                          \\
              9\ \left( \frac{1}{2}\ \vec{a} - \frac{1}{3}\ \vec{c} \right) &
              = 9 \cdot \left( \frac{3}{2}, 1, 0 \right)
              - \left( \frac{5}{3}, -\frac{1}{3}, \frac{8}{3} \right)         \\
                                                                            &
              = 9 \cdot \left( -\frac{1}{6}, \frac{4}{3}, -\frac{8}{3} \right)
              = \left( -\frac{3}{2} , 12, -24\right)
          \end{align}

    \item Performing the given computations,
          \begin{align}
              (7 - 3)\vec{a}      & = 4 \cdot (3, 2, 0) = (12, 8, 0)       \\
              7\vec{a} - 3\vec{a} & = (21, 14, 0) - (9, 6, 0) = (12, 8, 0)
          \end{align}

    \item Performing the given computations,
          \begin{align}
              4\vec{a} + 3\vec{b}  & = (12, 8, 0) +  (-12, 18, 0) = (0, 26, 0)   \\
              -4\vec{a} - 3\vec{b} & = (-12, -8, 0) + (12, -18, 0) = (0, -26, 0)
          \end{align}

    \item The laws are,
          \begin{align}
              12 & - \text{associative} & 13 & - \text{commutative} \\
              14 & - \text{additive}    & 15 & - \text{additive}    \\
              16 & - \text{additive}
          \end{align}

    \item Proving equation 4a, using the commutativity of real numbers
          \begin{align}
              \vec{a} + \vec{b} =
              \begin{bNiceMatrix}[margin]
                  a_1 \\ a_2 \\ \vdots \\ a_n
              \end{bNiceMatrix} + \begin{bNiceMatrix}[margin]
                                      b_1 \\ b_2 \\ \vdots \\ b_n
                                  \end{bNiceMatrix} =
              \begin{bNiceMatrix}[margin]
                  a_1 + b_1 \\ a_2 + b_2 \\ \vdots \\ a_n + b_n
              \end{bNiceMatrix} =
              \begin{bNiceMatrix}[margin]
                  b_1 + a_1 \\ b_2 + a_2 \\ \vdots \\ b_n + a_n
              \end{bNiceMatrix} =
              \vec{b} + \vec{a}
          \end{align}
          Proving equation 4b, using the associativity of real numbers,
          \begin{align}
              \vec{u} + (\vec{v} + \vec{w}) =
              \begin{bNiceMatrix}[margin]
                  u_1 \\ u_2 \\ \vdots \\ u_n
              \end{bNiceMatrix} +
              \begin{bNiceMatrix}[margin]
                  v_1 + w_1 \\ v_2 + w_2 \\ \vdots \\ v_n + w_n
              \end{bNiceMatrix} =
              \begin{bNiceMatrix}[margin]
                  u_1 + v_1 + w_1 \\ u_2 + v_2 + w_2 \\ \vdots \\ u_3 + v_3 + w_3
              \end{bNiceMatrix} \\
              (\vec{u} + \vec{v}) + \vec{w} =
              \begin{bNiceMatrix}[margin]
                  u_1 + v_1 \\ u_2 + v_2 \\ \vdots \\ u_n + v_n
              \end{bNiceMatrix} +
              \begin{bNiceMatrix}[margin]
                  w_1 \\ w_2 \\ \vdots \\ w_n
              \end{bNiceMatrix} =
              \begin{bNiceMatrix}[margin]
                  u_1 + v_1 + w_1 \\ u_2 + v_2 + w_2 \\ \vdots \\ u_3 + v_3 + w_3
              \end{bNiceMatrix}
          \end{align}
          Proving equation 4c, using 4a and the additive identity for real numbers
          \begin{align}
              \vec{a} + \vec{0} = \begin{bNiceMatrix}[margin]
                                      a_1 + 0 \\ a_2 + 0 \\ \vdots \\ a_n + 0
                                  \end{bNiceMatrix}
              = \begin{bNiceMatrix}[margin]
                    a_1 \\ a_2 \\ \vdots \\ a_n
                \end{bNiceMatrix} =
              \begin{bNiceMatrix}[margin]
                  0 + a_1 \\ 0 + a_2 \\ \vdots \\ 0 + a_n
              \end{bNiceMatrix}
              = \vec{0} + \vec{a}
          \end{align}
          Proving equation 4d, using the additive inverse for real numbers
          \begin{align}
              \vec{a} + \vec{-a} =
              \begin{bNiceMatrix}[margin]
                  a_1 - a_1 \\ a_2 - a_2 \\ \vdots \\ a_n - a_n
              \end{bNiceMatrix}
              = \begin{bNiceMatrix}[margin]
                    0 \\ 0 \\ \vdots \\ 0
                \end{bNiceMatrix}
              = \vec{0}
          \end{align}
          Proving equation 6a, using distributivity of multiplication in real numbers
          \begin{align}
              c\ (\vec{a} + \vec{b}) =
              c \cdot \begin{bNiceMatrix}[margin]
                          a_1 + b_1 \\ a_2 + b_2 \\ \vdots \\ a_n + b_n
                      \end{bNiceMatrix}
              = \begin{bNiceMatrix}[margin]
                    c \cdot (a_1 + b_1) \\
                    c \cdot (a_2 + b_2) \\
                    \vdots              \\
                    c \cdot (a_n + b_n)
                \end{bNiceMatrix}
              = \begin{bNiceMatrix}[margin]
                    c a_1 + c b_1 \\
                    c a_2 + c b_2 \\
                    \vdots        \\
                    c a_n + c b_n
                \end{bNiceMatrix}
              = c \vec{a} + c \vec{b}
          \end{align}
          Proving equation 6b, using distributivity of multiplication in real numbers
          \begin{align}
              (c + k)\vec{a} =
              c \cdot \begin{bNiceMatrix}[margin]
                          (c + k)a_1 \\ (c + k)a_2 \\ \vdots \\ (c + k)a_n
                      \end{bNiceMatrix}
              = \begin{bNiceMatrix}[margin]
                    ca_1 + ka_1 \\
                    ca_2 + ka_2 \\
                    \vdots      \\
                    ca_n + ka_n
                \end{bNiceMatrix}
              = c \vec{a} + k \vec{a}
          \end{align}
          Proving equation 6c, using associativity of multiplication in real numbers
          \begin{align}
              c\ (k\vec{a}) =
              c \cdot \begin{bNiceMatrix}[margin]
                          ka_1 \\ ka_2 \\ \vdots \\ ka_n
                      \end{bNiceMatrix}
              = \begin{bNiceMatrix}[margin]
                    cka_1  \\
                    cka_2  \\
                    \vdots \\
                    cka_n
                \end{bNiceMatrix}
              = \begin{bNiceMatrix}[margin]
                    (ck)\ a_1 \\
                    (ck)\ a_2 \\
                    \vdots    \\
                    (ck)\ a_n
                \end{bNiceMatrix}
              = (ck)\ \vec{a}
          \end{align}
          Proving equation 6d, using multiplicative identity in real numbers
          \begin{align}
              1 \cdot \vec{a} =
              \begin{bNiceMatrix}[margin]
                  1 \cdot a_1 \\ 1 \cdot a_2 \\ \vdots \\ 1 \cdot a_n
              \end{bNiceMatrix}
              = \begin{bNiceMatrix}[margin]
                    a_1    \\
                    a_2    \\
                    \vdots \\
                    a_n
                \end{bNiceMatrix}
              = \vec{a}
          \end{align}

    \item Finding the resultant force using vector addition,
          \begin{align}
              \vec{p}                           & = (2,3,0)                   &
              \vec{q}                           & = (0, 6, 1)                   \\
              \vec{u}                           & = (2, 0, -4)                &
              \vec{F}_{\text{res}}              & = (4, 9, -3)                  \\
              \abs{ \vec{F}_{\text{res}}} & = \sqrt{4^2 + 9^2 + (-3)^2} &
                                                & = \sqrt{106}
          \end{align}

    \item Finding the resultant force using vector addition,
          \begin{align}
              \vec{p}                           & = (1,-2,3)     &
              \vec{q}                           & = (3, 21, -16)   \\
              \vec{u}                           & = (-4, -9, 13) &
              \vec{F}_{\text{res}}              & = (0, 10, 0)     \\
              \abs{ \vec{F}_{\text{res}}} & = \sqrt{10^2}  &
                                                & = 10
          \end{align}

    \item Finding the resultant force using vector addition,
          \begin{align}
              \vec{p}                                       & = (8, -1, 0) &
              \vec{q}                                       & =
              \left( \frac{1}{2}, 0, \frac{4}{3} \right)                     \\
              \vec{u}                                       & =
              \left( -\frac{17}{2}, 1, \frac{11}{3} \right) &
              \vec{F}_{\text{res}}                          & =
              \left( 0, 0, 5 \right)                                         \\
              \abs{ \vec{F}_{\text{res}}}             & =
              \sqrt{5^2}                                    &
                                                            & = 5
          \end{align}

    \item Finding the resultant force using vector addition,
          \begin{align}
              \vec{p}                           & = (-1, 2, -3)      &
              \vec{q}                           & = (1, 1, 1)          \\
              \vec{u}                           & = (1, -2, 2)       &
              \vec{F}_{\text{res}}              & = (1, 1, 0)          \\
              \abs{ \vec{F}_{\text{res}}} & = \sqrt{1^2 + 1^2} &
                                                & = \sqrt{2}
          \end{align}

    \item Finding the resultant force using vector addition,
          \begin{align}
              \vec{p}                           & = (3, 1, -6)                 &
              \vec{q}                           & = (0, 2, 5)                    \\
              \vec{u}                           & = (3, -1, -13)               &
              \vec{F}_{\text{res}}              & = (6, 2, -14)                  \\
              \abs{ \vec{F}_{\text{res}}} & = \sqrt{6^2 + 2^2 + (-14)^2} &
                                                & = \sqrt{236}
          \end{align}

    \item Equilibrium requires the vector sum to be zero.
          \begin{align}
              \vec{p} + \vec{q} + \vec{u} + \vec{v} & = 0 &
              \vec{v}                               & =
              -1 \cdot (4, 9, -3) = (-4, -9, 3)
          \end{align}

    \item Equilibrium requires the vector sum to be zero.
          \begin{align}
              \vec{p} + \vec{q} + \vec{u} + \vec{v} & = 0 &
              \vec{v}                               & =
              -1 \cdot (0, 0, 5) = (0, 0, -5)
          \end{align}

    \item Normalizing the vector by making its norm equal to 1,
          \begin{align}
              \vec{v}       & = (1, 1, 0)                  &
              \vec{\hat{v}} & =
              \frac{\vec{v}}{\abs{ \vec{v} }}          \\
              \vec{\hat{v}} & = \left( \frac{1}{\sqrt{2}},
              \frac{1}{\sqrt{2}}, 0 \right)
          \end{align}

    \item Parallel to the $ xy $-plane requires a vector of the form
          \begin{align}
              \vec{v} + \vec{p} + \vec{q} + \vec{u} & = (F_1, F_2, 0) &
              (v_1 + 4, v_2 + 9, v_3 - 3)           & = (F_1, F_2, 0)   \\
              \vec{v}                               & = (v_1, v_2, 3)
          \end{align}

    \item Parallel to the $ z $-axis requires a vector of the form
          \begin{align}
              \vec{v} + \vec{p} + \vec{q} + \vec{u} & = (0, 0, F_3)   &
              (v_1 + 4, v_2 + 9, v_3 - 3)           & = (0, 0, F_3)     \\
              \vec{v}                               & = (-4, -9, F_3)
          \end{align}

    \item Parallel to the $ xy $-plane requires a vector of the form
          \begin{align}
              \vec{a} + \vec{b} + \vec{c} & = (F_1, F_2, 0) &
              \begin{bNiceMatrix}[margin]
                  2 + 1 + 0 \\ 0 + 2 + 3 \\ -7 - 3 + k
              \end{bNiceMatrix} = \begin{bNiceMatrix}[margin]
                                      F_1 \\ F_2 \\ 0
                                  \end{bNiceMatrix}  \\
              k                           & = 10
          \end{align}

    \item Using the triangle inequality,
          \begin{align}
              \abs{ \vec{p} + \vec{q} } & \leq \abs{ \vec{p} }
              + \abs{ \vec{q} }         &
              \abs{ \vec{p} + \vec{q} } & \leq 10
          \end{align}

    \item Using the triangle inequality,
          \begin{align}
              \abs{ \vec{p} + \vec{q} + \vec{u} } & \leq \abs{ \vec{p} }
              + \abs{ \vec{q} }
              + \abs{ \vec{u} }                   &
              \abs{ \vec{p} + \vec{q} + \vec{u} } & \leq 18
          \end{align}

    \item Finding the relative velocity,
          \begin{align}
              \vec{v}_A             & = 550\ \text{mph south-west} =
              550 \left( -\frac{\vec{\hat{i}}}{\sqrt{2}}
              - \frac{\vec{\hat{j}}}{\sqrt{2}} \right)               \\
              \vec{v}_B             & = 550\ \text{mph north-west} =
              450 \left( -\frac{\vec{\hat{i}}}{\sqrt{2}}
              + \frac{\vec{\hat{j}}}{\sqrt{2}} \right)               \\
              \vec{v}_B - \vec{v}_A & =
              \left( 100\ \frac{\vec{\hat{i}}}{\sqrt{2}}
              + 1000\ \frac{\vec{\hat{j}}}{\sqrt{2}} \right)
          \end{align}

    \item Finding the relative velocity,
          \begin{align}
              \vec{v}_A             & = 22\ \text{knots north-east} =
              22 \left( \frac{\vec{\hat{i}}}{\sqrt{2}}
              + \frac{\vec{\hat{j}}}{\sqrt{2}} \right)                \\
              \vec{v}_B             & = -19\ \text{knots west} =
              19 \vec{\hat{i}}                                        \\
              \vec{v}_B - \vec{v}_A & =
              (-19 - 11\sqrt{2})\ \vec{\hat{i}} - 11 \sqrt{2}\ \vec{\hat{j}}
          \end{align}

    \item The effect of reflecting a vector about a mirror is to reverse the
          component perpendicular to the mirror's surface and leave unaffected the
          component parallel to the mirror's surface. \par
          Let the mirrors be aligned along the positive $ x $ and $ y $ directions.
          \begin{align}
              \vec{v}_0 & = a \vec{\hat{i}} + b \vec{\hat{j}}  &
              \vec{v}_1 & = a \vec{\hat{i}} - b \vec{\hat{j}}    \\
              \vec{v}_2 & = -a \vec{\hat{i}} - b \vec{\hat{j}}
          \end{align}

    \item Finding the magnitudes of the forces,
          \begin{align}
              \vec{v} & = (l, l)  & \vec{p}                     & = (0, -1000) \\
              \vec{u} & = (-l, 0) & \vec{v} + \vec{p} + \vec{u} & = \vec{0}    \\
              l       & = 1000
          \end{align}

    \item Proving geometric identities,
          \begin{enumerate}
              \item Let $ P $ be the midpoint of the main diagonal. \par
                    and $ Q $ be the midpoint of the other diagonal.
                    \begin{align}
                        P & = \frac{\vec{a} + \vec{b}}{2}           &
                        Q & = \vec{b} + \frac{\vec{a} - \vec{b}}{2}
                    \end{align}
                    $ P $ and $ Q $ coincide, which means that each diagonal of a
                    parallellogram bisects the other.

                    \begin{figure}[H]
                        \centering
                        \begin{tikzpicture}[scale = 2]
                            \coordinate (O) at (0,0); \coordinate (a) at (3, 0);
                            \coordinate (b) at (1, 2); \coordinate (c) at (4, 2);
                            \draw[black] (O) -- (a); \draw[black] (O) -- (b);
                            \draw[black] (a) -- (c); \draw[black] (b) -- (c);
                            \draw[dashed,y_h] (a) -- (b);
                            \draw[dashed,y_p] (O) -- (c);
                            \node[below] at (O) {$ O $};
                            \node[left] at (b) {$ B $};
                            \node[right] at (a) {$ A $};
                            \node[right] at (c) {$ C $};
                            \node[circle, fill = white] at (2, 1) {$ P $};
                        \end{tikzpicture}
                    \end{figure}

              \item Let $ P, Q $ be the midpoints of $ OA, AC $ respectively. Further let
                    $ R $ be the midpoint of $ PQ $
                    \begin{align}
                        \vec{r} & = \vec{p} + \frac{\vec{q} - \vec{p}}{2} &
                                & = \frac{\vec{a}}{2}
                        + \frac{1}{2}\left( \vec{a} + \frac{\vec{b}}{2}
                        - \frac{\vec{a}}{2} \right)                         \\
                                & =\frac{3\vec{a} + \vec{b}}{4}
                    \end{align}
                    Let this point divide the off-diagonal in the ratio
                    $ \lambda : 1 - \lambda $.
                    \begin{align}
                        \vec{r}                        & =
                        \vec{b} + \lambda (\vec{a} - \vec{b})
                        = \frac{3\vec{a} + \vec{b}}{4} &
                        \lambda                        & = \frac{3}{4}
                    \end{align}


                    \begin{figure}[H]
                        \centering
                        \begin{tikzpicture}[scale = 2]
                            \coordinate (O) at (0,0); \coordinate (a) at (3, 0);
                            \coordinate (b) at (1, 2); \coordinate (c) at (4, 2);
                            \draw[black] (O) -- (a); \draw[black] (O) -- (b);
                            \draw[black] (a) -- (c); \draw[black] (b) -- (c);
                            \draw[dashed,y_h] (1.5, 0) -- (3.5, 1);
                            \draw[dashed,y_p] (b) -- (a);
                            \node[below] at (O) {$ O $};
                            \node[left] at (b) {$ B $};
                            \node[right] at (a) {$ A $};
                            \node[right] at (c) {$ C $};
                            \node[below] at (1.5, 0) {$ P $};
                            \node[right] at (3.5, 1) {$ Q $};
                            \node[circle, fill = white] at (2.5, 0.5) {$ R $};
                        \end{tikzpicture}
                    \end{figure}

              \item Consider the parallellogram with $ PA $ and $ QA $ as sides.
                    Call the fourth vertex $ X $.
                    \par Its main diagonal $ PQ $ is bisected by and bisects $ AX $
                    \begin{align}
                        \overline{BX} & = \frac{1}{2}\ \overline{BA} &
                        \overline{XR} & = \frac{1}{2}\ \overline{XA}
                        = \frac{1}{4}\ \overline{BA}                   \\
                        \overline{BR} & = \frac{3}{4}\ \overline{BA}
                    \end{align}

              \item Drawing the medians to all 3 sides of a triangle, \par
                    The point $ M $ which divides $ AP $ in the ratio $ 2:1 $ is given by
                    \begin{align}
                        \vec{m} & = \vec{a} + \frac{2}{3}\ (\vec{p} - \vec{a}) &
                        \vec{p} & = \frac{\vec{b} + \vec{c}}{2}                  \\
                        \vec{m} & = \frac{\vec{a}}{3}
                        + \left( \frac{\vec{b} + \vec{c}}{3} \right)
                    \end{align}
                    Clearly $ \vec{m} $ is symmetric in $ \vec{a}, \vec{b} $ and
                    $ \vec{c} $, which means the other 2 ways of arriving at $ M $ will
                    lead to the same position.
                    \begin{figure}[H]
                        \centering
                        \begin{tikzpicture}[scale = 2]
                            \coordinate (a) at (0,0); \coordinate (b) at (3, 0);
                            \coordinate (c) at (2, 2); \coordinate (R) at (1.5, 0);
                            \coordinate (Q) at (1, 1); \coordinate (P) at (2.5, 1);
                            \draw[black] (a) -- (b); \draw[black] (b) -- (c);
                            \draw[black] (a) -- (c);
                            \draw[y_h] (a) -- (P); \draw[y_h] (b) -- (Q);
                            \draw[y_h] (c) -- (R);
                            \node[below right] at (b) {$ B $};
                            \node[below left] at (a) {$ A $};
                            \node[above] at (c) {$ C $};
                            \node[above right] at (P) {$ P $};
                            \node[above left] at (Q) {$ Q $};
                            \node[below] at (R) {$ R $};
                            \node[circle, fill = white, inner sep = 2pt] at (5/3, 2/3)
                            {$ M $};
                            \node[circle, fill = white, inner sep = 0pt] at (5/6, 1/3)
                            {\color{y_p} \scriptsize $ 2 $};
                            \node[circle, fill = white, inner sep = 0pt] at (25/12, 5/6)
                            {\color{y_p} \scriptsize $ 1 $};
                        \end{tikzpicture}
                    \end{figure}

              \item Arbitraty quadrilateral has vertices $ A,B,C,D $ with
                    midpoints $ P,Q,R,S $. \par
                    Two vectors are parallel if one is a nonzero scalar multiple of the
                    other.
                    \begin{align}
                        \overline{PQ}                                             &
                        = \frac{\vec{b} + \vec{c}}{2} -
                        \frac{\vec{b} + \vec{a}}{2} = \frac{\vec{c} - \vec{a}}{2} &
                        \overline{SR}                                             &
                        = \frac{\vec{c} + \vec{d}}{2} -
                        \frac{\vec{a} + \vec{d}}{2} = \frac{\vec{c} - \vec{a}}{2}   \\
                        \overline{PS}                                             &
                        = \frac{\vec{a} + \vec{d}}{2} -
                        \frac{\vec{a} + \vec{b}}{2} = \frac{\vec{d} - \vec{b}}{2} &
                        \overline{QR}                                             &
                        = \frac{\vec{d} + \vec{c}}{2} -
                        \frac{\vec{b} + \vec{c}}{2} = \frac{\vec{d} - \vec{b}}{2}
                    \end{align}
                    The quadrilateral $ PQRS $ has identical pairs of opposite sides and
                    is therefore a parallellogram by definition.
                    \begin{figure}[H]
                        \centering
                        \begin{tikzpicture}[scale = 2]
                            \coordinate (a) at (0,0); \coordinate (b) at (4, 0);
                            \coordinate (c) at (3, 2); \coordinate (d) at (1, 2);
                            \coordinate (P) at (2,0); \coordinate (Q) at (3.5, 1);
                            \coordinate (R) at (2, 2); \coordinate (S) at (0.5, 1);
                            \draw[black] (a) -- (b); \draw[black] (b) -- (c);
                            \draw[black] (c) -- (d); \draw[black] (d) -- (a);
                            \draw[y_h] (P) -- (Q); \draw[y_h] (Q) -- (R);
                            \draw[y_h] (R) -- (S); \draw[y_h] (S) -- (P);
                            \node[below right] at (b) {$ B $};
                            \node[below left] at (a) {$ A $};
                            \node[above right] at (c) {$ C $};
                            \node[above left] at (d) {$ D $};
                            \node[below = 1] at (P) {\color{y_p} \small $ P $};
                            \node[right = 1] at (Q) {\color{y_p} \small $ Q $};
                            \node[above = 1] at (R) {\color{y_p} \small $ R $};
                            \node[left = 1] at (S) {\color{y_p} \small $ S $};
                        \end{tikzpicture}
                    \end{figure}

              \item A parallellopiped is the 3d analog of a parallellogram using
                    three main vectors $ \vec{a}, \vec{b}, \vec{c} $. Let the other 4
                    vertices be
                    \begin{align}
                        \vec{a} + \vec{b}           & = \vec{p} &
                        \vec{b} + \vec{c}           & = \vec{q}   \\
                        \vec{a} + \vec{c}           & = \vec{r} &
                        \vec{a} + \vec{b} + \vec{c} & = \vec{s}
                    \end{align}
                    The positions of the midpoints of all four body-diagonals are,
                    \begin{align}
                        \vec{m}_1 & = \frac{\vec{p} + \vec{c}}{2} &
                        \vec{m}_2 & = \frac{\vec{q} + \vec{a}}{2}   \\
                        \vec{m}_3 & = \frac{\vec{r} + \vec{b}}{2} &
                        \vec{m}_4 & = \frac{\vec{0} + \vec{s}}{2}
                    \end{align}
                    Clearly all positions of all four $ \vec{m}_j $ coincide, which
                    proves the relation.

              \item Consider an $ n $ sided regular polygon. Construct it by
                    a set of $ n $ vectors whose starting point is the origin and
                    terminal point is $ P_i $. \par
                    For a regular polygon, the angle between successive vectors has to be
                    constant and equal to $ 2\pi / n $.
                    \begin{align}
                        \vec{p}_1     & = \vec{\hat{i}} &
                        \vec{p}_{k+1} & =
                        \cos\left( \frac{2\pi k}{n} \right) \vec{\hat{i}}
                        + \sin\left( \frac{2\pi k}{n} \right) \vec{\hat{j}} \\
                    \end{align}
                    Calculating the vector sum,
                    \begin{align}
                        \sum_{k=0}^{n-1} \vec{p}_{kx} & = \sum_{k=0}^{n-1}
                        \cos\left( \frac{2\pi k}{n} \right)
                        = \sum_{k=0}^{n-1}
                        \text{Re}\left[ \exp\left( i\ \frac{2\pi k}{n} \right) \right] \\
                                                      & =
                        \exp\left( i\ \frac{2\pi k}{n} \right)^{-1}
                        \text{Re}[\exp(i\ 2k\pi) - 1]
                    \end{align}
                    Since $ z = \exp(i\ 2k \pi) - 1 \equiv 0 \ \forall\ k $, this
                    summation is identically zero. \par
                    A similar summation for the sine terms involves the imaginary part of
                    this complex number which is also identically zero.
          \end{enumerate}

\end{enumerate}