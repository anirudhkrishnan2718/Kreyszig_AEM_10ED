\tikzstyle{force}=[-Triangle,y_p,thick,line cap=round]

\chapter{Vector Differential Calculus: Grad, Div, Curl}

\section{Vectors in 2-Space and 3-Space}
\begin{description}
    \item[Scalar] A quantity determined solely by its magnitude.
    \item[Vector] A quantity determined by magnitude and direction. It is represented
        by a directed line segment (an arrow).
    \item[Norm] The length of a vector represented by $ \abs{ \vec{v} } $.
    \item[Translation] Displacement without rotation (as a linear transform or operation
        , usually acting on vectors)
    \item[Unit vector] A vector with length 1
    \item[Equality of vectors] Two vectors are equal if and only if their magnitudes and
        directions are both equal.
        \begin{align}
            \vec{a} = \vec{b} \iff \abs{ \vec{a} } = \abs{ \vec{b} }
            \quad \text{and} \quad \vec{\hat{a}} = \vec{\hat{b}}
        \end{align}

    \item[Components of a vector] If a vector has initial point $ P(x_1, y_1, z_1) $
        and terminal point $ Q(x_2, y_2, z_2) $, then
        \begin{align}
            \vec{v}                                   & = \begin{bNiceMatrix}[r, margin]
                                                              v_1 \\ v_2 \\ v_3
                                                          \end{bNiceMatrix}
            \equiv \begin{bNiceMatrix}[r, margin]
                       (x_2 - x_1) \\ (y_2 - y_1) \\ (z_2 - z_1)
                   \end{bNiceMatrix} &
            \abs{ \vec{a} }                           & = \sqrt{v_1^2 + v_2^2 + v_3^2}
        \end{align}
        This definition of a vector's components is specific to a Cartesian coordinate
        system in 3-d.

    \item[Position vector] A vector whose initial point is the origin and terminal point
        is a given point in 3-d space.

    \item[Algebraic vector] Given a fixed Cartesian coordinate system, every ordered
        triple of real numbers $ (a_1, a_2, a_3) $ is associated to a unique vector
        $ \vec{a} $ (and vice versa). Only the zero vector has no direction. \par
        A vector equation is the same as a scalar equation for each of its components.

    \item[Vector addition] Adding two vectors is performed by adding their
        corresponding components.
        \begin{align}
            \vec{a} + \vec{b} & \equiv \begin{bNiceMatrix}[r, margin]
                                           a_1 + b_1 \\ a_2 + b_2 \\ a_3 + b_3
                                       \end{bNiceMatrix}
        \end{align}

        Geometrically, the initial point of $ \vec{b} $ is placed at the terminal point
        of $ \vec{a} $ and the vector sum is drawn from the initial point of $ \vec{a} $
        to the terminal point of $ \vec{b} $.

        \begin{figure}[H]
            \centering
            \begin{tikzpicture}[scale = 3]
                \draw[force, y_h] (0,0.05) -- (0,0.95) node[midway,right]
                {$\vec{a}$};
                \draw[force, y_p] (0,1) -- (0.95,1) node[midway, above]
                {$\vec{b}$};
                \draw[force, black] (0.05,0.05) -- (0.95,0.95) node[midway, right = 2]
                {$\vec{a} + \vec{b}$};
            \end{tikzpicture}
            \hspace{2in}
            \begin{tikzpicture}[scale = 2]
                \def\ang{30}
                \def\F{1.4}  % force magnitude
                \coordinate (O) at (0,0);
                \coordinate (a) at (1, 1);
                \coordinate (b) at (-1, 1);
                \coordinate (c) at (0, 2);
                \draw[dotted,black] (a) -- (c);
                \draw[dotted,black] (b) -- (c);
                \draw[force, y_h] (O) -- (a) node[midway, right = 2] {$\vec{a}$};
                \draw[force, y_p] (O) -- (b) node[midway, left = 2] {$\vec{b}$};
                \draw[force,black] (O) -- (c) node[above] {$\vec{a} + \vec{b}$};
            \end{tikzpicture}
        \end{figure}

        The paralleogram rule is used in physics to calculate the resultant (vector sum)
        of many forces acting on the same point. \par

    \item[Properties of vector addition] Similar to scalar addition,
        \begin{align}
            \vec{a} + \vec{b}             & = \vec{b} + \vec{a}             &
                                          &                                 &
            \text{commutative}                                                \\
            (\vec{u} + \vec{v}) + \vec{w} & = \vec{u} + (\vec{v} + \vec{w}) &
                                          &                                 &
            \text{associative}                                                \\
            \vec{a} + \vec{0}             & = \vec{a}                       &
                                          &                                 &
            \text{additive null}                                              \\
            \vec{a} + (-\vec{a})          & = \vec{0}                       &
                                          &                                 &
            \text{additive inverse}
        \end{align}

    \item[Scalar multiplication] Multiplying a vector by a scalar multiplies it by
        every component of the vector.
        \begin{align}
            c\ \vec{v}         & \equiv \begin{bNiceMatrix}[r, margin]
                                            c\ v_1 \\ c\ v_2 \\ c\ v_3
                                        \end{bNiceMatrix} &
            \abs{ c\ \vec{v} } & = \abs{ c } \abs{ \vec{v} }
        \end{align}
        Geometrically, if $ c<0  $, then $ c \vec{v} $ faces the opposite direction.

    \item[Properties of scalar multiplication] Similar to scalars,
        \begin{align}
            c(\vec{a} + \vec{b}) & = c\vec{a} + c\vec{b} &
            (c + k)\vec{a}       & = c\vec{a} + k\vec{a}   \\
            c\ (k\vec{a})        & = (ck)\ \vec{a}       &
            1 \cdot \vec{a}      & = \vec{a}               \\
            (-1) \cdot \vec{a}   & = -\vec{a}            &
            0 \cdot \vec{a}      & = \vec{0}
        \end{align}

    \item[Standard basis] $ \mathcal{R}^3 $ represented geometrically by 3-d
        Cartesian space, has dimension 3, and needs a set of 3 basis vectors defined by
        \begin{align}
            \vec{\hat{i}} & = \begin{bNiceMatrix}[r, margin]
                                  1 \\ 0 \\ 0
                              \end{bNiceMatrix}          &
            \vec{\hat{j}} & = \begin{bNiceMatrix}[r, margin]
                                  0 \\ 1 \\ 0
                              \end{bNiceMatrix}          &
            \vec{\hat{k}} & = \begin{bNiceMatrix}[r, margin]
                                  0 \\ 0 \\ 1
                              \end{bNiceMatrix}           \\
            \vec{a}       & = a_1 \vec{\hat{i}} + a_2 \vec{\hat{j}} +
            a_3 \vec{\hat{k}}
        \end{align}
\end{description}

\section{Inner Product (Dot Product)}
\begin{description}
    \item[Inner product] The product of lengths of two vectors and the cosine of the angle
        between them. This is a scalar.
        \begin{align}
            \vec{a} \dotp \vec{b} &
            = \begin{dcases}
                  \abs{\vec{a}} \abs{\vec{b}} \cos(\theta) &
                  \abs{\vec{a}} \neq 0\ \text{and}\ \abs{\vec{b}} \neq 0 \\
                  0                                        &
                  \abs{\vec{a}} = 0\ \text{or}\ \abs{\vec{b}} = 0
              \end{dcases}
        \end{align}
        Here $ \theta $ is the smaller angle between the vectors measured after
        making their initial points coincide. \par
        In terms of the components,
        \begin{align}
            \vec{a} \dotp \vec{b} & \equiv \sum_{k=1}^{n} a_k b_k
        \end{align}

    \item[Orthogonal vectors] Two vectors are perpendicular if and only if their
        inner product is zero.

    \item[Length of a vector] The square root of the inner product of a vector with
        itself. This is geomatrically equal to its magnitude. \par
        Additionally, the angle between two nonzero vectors can be found using,
        \begin{align}
            \abs{\vec{v}} & \equiv \sqrt{\vec{v} \dotp \vec{v}} \\
            \cos \theta   & = \frac{\vec{a} \dotp \vec{b}}
            {\abs{\vec{a}} \abs{\vec{b}}}
        \end{align}

    \item[Properties of inner product] Some properties similar to multiplication,
        \begin{align}
            (p\vec{a} + q\vec{b}) \dotp \vec{c} & = p \vec{a} \dotp \vec{c}
            + q \vec{b} \dotp \vec{c}           &
                                                & \text{linear}                \\
            \vec{a} \dotp \vec{b}               & = \vec{b} \dotp \vec{a}    &
                                                & \text{symmetry}              \\
            \vec{a} \dotp \vec{a}               & \geq 0                       \\
            \vec{a} \dotp \vec{a}               & = 0 \iff \vec{a = \vec{0}} &
                                                & \text{positive-definite}     \\
            (\vec{a} + \vec{b}) \dotp \vec{c}   & = \vec{a} \dotp \vec{c}
            + \vec{b} \dotp \vec{c}             &
                                                & \text{distributive}
        \end{align}

    \item[Inequality relations] The fact that $ \abs{\cos\theta} \leq 1$ gives,
        \begin{align}
            \abs{\vec{a} + \vec{b}}                               &
            \leq \abs{\vec{a}} + \abs{\vec{b}}                    &
                                                                  &
            \text{Triangle inequality}                              \\
            \abs{\vec{a} \dotp \vec{b}}                           &
            \leq \abs{\vec{a}} \abs{\vec{b}}                      &
                                                                  &
            \text{Cauchy-Schwraz inequality}                        \\
            \abs{\vec{a} + \vec{b}}^2 + \abs{\vec{a} - \vec{b}}^2 &
            = 2 (\abs{\vec{a}}^2 + \abs{\vec{b}}^2)               &
                                                                  &
            \text{parallellogram equality}
        \end{align}

    \item[Orthogonal projection] The orthogonal projection (or component) of a vector
        along a line whose unit vector is $ \vec{\hat{b}} $ is defined as
        \begin{align}
            p & = \vec{a} \dotp \vec{\hat{b}} = \abs{\vec{a}} \cos \theta
        \end{align}
        Here $ \theta $ is the angle between the vector $ \vec{a} $ and the line it is
        projected upon. \par

        A plane can be defined in Cartesian 3d by requiring the projection of all points
        on it upon the normal vector to be constant.
        \begin{align}
            \vec{\hat{n}} \dotp \vec{r} & = p
        \end{align}
        Here, $ p $ is the distance of $ \vec{\hat{n}} $ from the origin. All planes
        parallel to this plane simply vary in their $ p $ value.
\end{description}

\section{Vector Product (Cross Product)}

\begin{description}
    \item[Cross product] Geometrically, a vector orthogonal to the two given vectors
        whose magnitude is equal to the area of the parallellogram formed by them.
        \begin{align}
            \vec{v} & \equiv \vec{a} \times \vec{b}
        \end{align}
        \begin{itemize}
            \item If $ \vec{a} = 0 $ or $ \vec{b} = 0 $, then by definition,
                  $ \vec{v}  = \vec{0} $.
            \item Else if $ \vec{a} $ and $ \vec{b} $ are parallel, such that the angle
                  between then is $ 0 $ or $ \pi $, then again $ \vec{v} = \vec{0} $
            \item Outside of these special cases, the magnitude of the cross product is
                  \begin{align}
                      \abs{\vec{v}} & \equiv \abs{\vec{a} \times \vec{b}}
                      = \abs{\vec{a}} \abs{\vec{b}} \sin \theta           \\
                      \vec{v}       &
                      = \begin{vNiceMatrix}[r, margin]
                            \vec{\hat{i}} & \vec{\hat{j}} & \vec{\hat{k}} \\
                            a_1           & a_2           & a_3           \\
                            b_1           & b_2           & b_3
                        \end{vNiceMatrix}
                  \end{align}
                  Its direction is such that $ \vec{v} $ is perpendicular to both
                  $ \vec{a} $ and $ \vec{b} $ and the three vectors $ \vec{a}, \vec{b},
                      \vec{v} $ form a right-handed triple.
        \end{itemize}

    \item[Right handed triple] If the fingers of the right hand are on the plane formed
        by $ \vec{a} $ and $ \vec{b} $, then the right thumb points in the direction of
        $ \vec{c} $. \par
        This convention picks one of the two possible choices for the direction of
        $ \vec{c} $.

    \item[Properties of cross product] Similar to scalar multiplication,
        \begin{align}
            (p\vec{a}) \times \vec{b}               & = p (\vec{a} \times \vec{b}) =
            \vec{a} \times (p\vec{b})                                                  \\
            \vec{a} \times (\vec{b} + \vec{c})      & = \vec{a} \times \vec{b}
            + \vec{a} \times \vec{c}                                                   \\
            (\vec{a} + \vec{b}) \times \vec{c}      & = \vec{a} \times \vec{c}
            + \vec{b} \times \vec{c}                &
                                                    & \text{distributive}              \\
            \vec{a} \times \vec{b}                  & = - (\vec{b} \times \vec{a})   &
                                                    & \text{anti-commutative}          \\
            \vec{a} \times (\vec{b} \times \vec{c}) & \neq
            (\vec{a} \times \vec{b}) \times \vec{c} &
                                                    & \text{not associative}
        \end{align}

    \item[Scalar Triple product] Also called the mixed product or box product. This
        is a scalar resulting from three input vectors. \par
        \begin{align}
            (\vec{a}\ \vec{b}\ \vec{c}) & \equiv \vec{a} \dotp
            (\vec{b} \times \vec{c})                                      \\
                                        & = \begin{vNiceMatrix}[r, margin]
                                                a_1 & a_2 & a_3 \\
                                                b_1 & b_2 & b_3 \\
                                                c_1 & c_2 & c_3
                                            \end{vNiceMatrix}
        \end{align}

    \item[Properties of box product] Some important properties,
        \begin{itemize}
            \item Geometrically, it is the volume of the parallellopiped with the three
                  vectors as the edge vectors.
            \item The dot and cross can be interchanged
                  \begin{align}
                      \vec{a} \dotp (\vec{b} \times \vec{c}) & =
                      (\vec{a} \times \vec{b}) \dotp \vec{c}
                  \end{align}
            \item Three vectors in Cartesian $ \mathcal{R}^3 $ are L.I. if and only if
                  their box product is nonzero.
        \end{itemize}
\end{description}

\section{Vector and Scalar Functions and Their Fields, Vector Calculus: Derivatives}

\begin{description}
    \item[Vector function] Functions of scalars that produce vectors as outputs. Such a
        function defines a vector field in the domain of definiton.
        \begin{align}
            \vec{v}(P)       & = \begin{bNiceMatrix}[margin]
                                     v_1(P) \\ v_2(P) \\ v_3(P)
                                 \end{bNiceMatrix}          &
            \vec{v}(x, y, z) & = \begin{bNiceMatrix}[margin]
                                     v_1(x,y,z) \\ v_2(x,y,z) \\ v_3(x,y,z)
                                 \end{bNiceMatrix}
        \end{align}
        A vector function may additionally depend on some other variable such as time.

    \item[Scalar function] Functions of scalars that produce scalars as outputs. Such a
        function defines a scalar field in the domain of definiton.
        \begin{align}
            f \equiv f(P) & f & \equiv f(x, y, z)
        \end{align}
        A scalar function may additionally depend on some other variable such as time.
        \par The value of the scalar function is independent of the choice of coordinate
        system.

    \item[Vector calculus] Many properties outlined here carry over from regular
        calculus.

    \item[Convergence] A sequence of vectors $ \{\vec{a}_k\} $ is said to converge if
        there exists some vector $ \vec{a} $ such that
        \begin{align}
            \lim_{n \rightarrow \infty} \abs{\vec{a}_n - \vec{a}} &
            = 0                                                     \\
            \lim_{n \rightarrow \infty} \vec{a}_n                 &
            = \vec{a} \qquad \text{(Limit vector)}
        \end{align}
        The corresponding definition of the convergence of vector functions of a scalar
        are identical to the standard definition of the convergence of scalar functions
        $ f(x) $ and omitted here.

    \item[Continuity] If a vector function $ \vec{v}(t) $ is defined in some
        neighbourhood of $ t = t_0 $ (including $ t_0 $ itself) and
        \begin{align}
            \lim_{t \rightarrow t_0} \vec{v}(t) & = \vec{v}(t_0)
        \end{align}
        then, the vector function is continuous at $ t = t_0 $.

    \item[Derivative of vector function] The derivative of $ \vec{v}(t) $ at $ t= t_0 $
        exists, is the limit
        \begin{align}
            \lim_{\Delta t \rightarrow 0} \frac{\vec{v}(t + \Delta t) - \vec{v}(t)}
            {\Delta t} \equiv \vec{v}'(t)
        \end{align}
        assuming this limit exists. \par
        When a coordinate system is defined, each component of the vector function is
        differentiated separately to yield the derivative of the vector.

    \item[Properties of vector derivatives] Similar to the properties of scalar
        derivatives,
        \begin{align}
            (c\vec{v})'                  & = c\ \vec{v}'                  \\
            (\vec{u} + \vec{v})'         & = \vec{u}' + \vec{v}'          \\
            (\vec{u} \dotp \vec{v})'     &
            = \vec{u}' \dotp \vec{v} + \vec{u} \dotp \vec{v}'             \\
            (\vec{u} \times \vec{v})'    &
            = \vec{u}' \times \vec{v} + \vec{u} \times \vec{v}'           \\
            (\vec{u}\ \vec{v}\ \vec{w})' & = (\vec{u}'\ \vec{v}\ \vec{w})
            + (\vec{u}\ \vec{v}'\ \vec{w}) + (\vec{u}\ \vec{v}\ \vec{w}')
        \end{align}

    \item[Partial derivatives of vector function] The operation to be applied to the
        vector function is simply applied to each component individually.
        \begin{align}
            \vec{v}(\{t_k\})          & \equiv \vec{v}(t_1, t_2, \dots, t_n)   \\
            \diffp{\vec{v}}{t_1}      & = \diffp{v_1}{t_1}\ \vec{\hat{i}}
            + \diffp{v_2}{t_1}\ \vec{\hat{j}}
            + \diffp{v_3}{t_1}\ \vec{\hat{k}}                                  \\
            \diffp{\vec{v}}{t_1, t_2} & = \diffp{v_1}{t_1, t_2}\ \vec{\hat{i}}
            + \diffp{v_2}{t_1, t_2}\ \vec{\hat{j}}
            + \diffp{v_3}{t_1, t_2}\ \vec{\hat{k}}
        \end{align}
\end{description}

\section{Curves, Arc Length, Curvature, Torsion}

\begin{description}
    \item[Parametric vector functions] Replacing the independent variable in each
        component of a vector by a common parameter ($ t $) yields,
        \begin{align}
            \vec{v}(t) & =
            \begin{bNiceMatrix}[r, margin]
                x(t) \\ y(t) \\ z(t)
            \end{bNiceMatrix}
            = x(t)\ \vec{\hat{i}} + y(t)\ \vec{\hat{j}} + z(t)\ \vec{\hat{k}}
        \end{align}
        This makes the curve oriented in the direction of increasing $ t $. (called the
        positive sense on the curce $ C $)

    \item[Twisted curve] A curve that does not lie in a plane in $ 3d $ space. Else, it
        is called a plane curve.

    \item[Simple curve] A curve which does not touch or intersect itself. There is a
        unique value of the parameter $ t $ for every point of the curve,

    \item[Tangent vector] The limiting position of a straight line through $ P $ on
        curve $ C $ and another close point $ Q $
        \begin{align}
            P           & : \vec{r}(t)                    &
            Q           & : \vec{r}(t + \Delta t)           \\
            \vec{r}'(t) & = \lim_{\Delta t \rightarrow 0}
            \frac{\vec{r}(t + \Delta t) - \vec{r}(t)}{\Delta t}
        \end{align}
        If $ \vec{r} \neq \vec{0} $, it is called the tangent vector to curve $ C $ at
        point $ P $.

    \item[Parametric straight line] The equation of a line passing through $ \vec{a} $
        in the direction of $ \vec{b} $ is given by
        \begin{align}
            \vec{v}    & = \vec{a} + t\vec{b}                           \\
            \vec{q}(w) & = \vec{r} + w\vec{r}' &  & \text{Tangent line}
        \end{align}
        Here, both $ \vec{r} $ and $ \vec{r}' $ are functions of the original parameter
        $ t $.

    \item[Length of curve] If a curve $ C $ is specified using a paramteric vector
        functionm $ \vec{r}(t) $, then the length of a curve corresponding to
        $ t \in [a,b] $ is given by
        \begin{align}
            l & = \int_{a}^{b} \sqrt{\vec{r}' \dotp \vec{r}'}\ \dl t
        \end{align}
        assuming $ \vec{r} $ is differentiable.

    \item[Arc length of a curve] The length of a curve from $ t = a $ to a variable end
        point $ t = w $. It is defined as
        \begin{align}
            s(w) & = \int_{a}^{w} \sqrt{\vec{r}' \dotp \vec{r}'}\ \dl t
        \end{align}

    \item[Linear element] From the definition of Pythagoras theorem, the linear element
        in $ 3d $ Cartesian space is
        \begin{align}
            \dl s^2 \equiv \vec{\dl r} \dotp  \vec{\dl r}
            = \dl x^2 + \dl y^2 + \dl z^2
        \end{align}

    \item[Trajectories] In mechanics, the parameter is usually time $ (t) $ and a curve
        represents the path taken by an object through $ 3d $ space. \par
        The first and second derivaties of the position $ \vec{r}(t) $ represent the
        velocity vector and acceleration vector respectively.
        \begin{align}
            \vec{v}(t) & \equiv \vec{r}'(t)  &
            \vec{a}(t) & \equiv \vec{r}''(t) &
        \end{align}
        The acceleration can be split into tangential and normal components. \par
        Defining the unit tangent vector in terms of the arc length $ s $,
        \begin{align}
            \vec{u}(s) & = \frac{\vec{r}'(t)}{\abs{\vec{r}'(t)}} = \vec{r}'(s) \\
            \vec{v}(t) & = \vec{u}(s)\ \diff st                                \\
            \vec{a}(t) & = \diff*{\vec{u(s)}}{s}\ \left( \diff st \right)^2 +
            \vec{u}(s)\ \diff[2] st                                            \\
        \end{align}
        Since the tangent vector $ \vec{u}(s) $ is always a unit vector, the first
        acceleration term has to be perpendicular to the velocity.

    \item[Normal acceleration] The acceleration vector can be split into two
        components, normal and tangential using the projection rule,
        \begin{align}
            \vec{a}(t)                 & = \vec{a}_{\text{norm}}
            + \vec{a}_{\text{tangent}} &
            \vec{a}_{\text{tangent}}   & = \vec{a} \dotp \vec{\hat{v}}
        \end{align}

    \item[Curvature] The rate of change of the unit tangent vector at a point $ P $ on
        the curve $ C $, give by
        \begin{align}
            \kappa (s) & \equiv \abs{\diff*{\vec{u}(s)}{s}}
            = \abs{\diff*[2]{\vec{r}(s)}{s}}
        \end{align}
        Here, the arc length $ (s) $ is used as the parameter of the curve instead of the
        usual $ t $.

    \item[Normal plane] The plane through $ P $ on curve $ C $ whose normal vector is
        the tangent vector $ \vec{u} $.

    \item[Osculating plane] The plane spanned by the tangent vector and its derivative
        (which happens to be the normal vector at that point). \par
        This plane contains the osculating (kissing) circle to the curve $ C $ at the
        point $ P $. Its normal vector is $ \vec{b} = \vec{u} \times \vec{n} $.

    \item[Rectifying plane] The third plane which is spanned by $ \vec{u} $ and
        $ \vec{b} $. Its normal vector is the unit normal vector $ \vec{u}' = \vec{p} $.

    \item[Torsion] The rate of change of the osculating plane defined as,
        \begin{align}
            \vec{p}       & \equiv \frac{1}{\kappa}\ \vec{u}'   &
                          & \text{unit principal normal vector}   \\
            \vec{b}       & \equiv \vec{u} \times \vec{p}       &
                          & \text{unit binormal vector}           \\
            \abs{\tau(s)} & = \abs{\vec{b}'(s)}                   \\
            \tau (s)      & = -\vec{p}(s) \dotp \vec{b}'(s)
        \end{align}
        Since the unit tangent vector $ \vec{u} $ has constant magnitude, its derivate is
        orthogonal to itself by definition. The right handed triple of vectors
        $ \vec{u}, \vec{p}, \vec{b} $ are defined at each point on the curve.
\end{description}

\section{Calculus Review: Functions of Several Variables}

\begin{description}
    \item[Chain rule] Consider a mapping from the domain $ B $ in the $ uv $ plane
        onto the domain $ D $ in $ xyz $ space using the mapping,
        \begin{align}
            x & \equiv x(u, v) & y & \equiv y(u, v) & z & \equiv z(u, v)
        \end{align}
        These functions are continuous and have continuous first partial derivatives in
        $ B $. \par
        Further, let $ f \equiv f(x, y, z) $ be continouous and have continouous first
        partial derivatices in $ D $. Then,
        \begin{align}
            w         & = f\Big( x(u, v),\ y(u, v),\ z(u, v) \Big)    \\
            \diffp wu & = \diffp wx\ \diffp xu + \diffp wy\ \diffp yu
            + \diffp wz\ \diffp zu                                    \\
            \diffp wv & = \diffp wx\ \diffp xv + \diffp wy\ \diffp yv
            + \diffp wz\ \diffp zv
        \end{align}

    \item[Mean Value theorem] With $ f(x, y, z) $ and $ D $ as defined above, and any
        two points $ P_0 $ and $ P $ connected by a straight line that lies entirely in
        $ D $,
        \begin{align}
            P_0 : (x_0, y_0, z_0)  \qquad
            P                                         & : (x_0 + h, y_0 + k, z_0 + l) \\
            f(x_0 + h, y_0 + k, z_0 + l) - f(x, y, z) & = h\ \diffp fx + k\ \diffp fy
            + l\ \diffp fz
        \end{align}
        This is a generalization of the much more familiar special case,
        \begin{align}
            f & \equiv f(x) & f(x_0 + h) - f(x_0) & = h\ \diffp fx
        \end{align}
\end{description}

\section{Gradient of a Scalar Field, Directional Derivative}

\begin{description}
    \item[Gradient] A vector function derived from a scalar function $ f $ in the $ 3d $
        Cartesian space, defined by,
        \begin{align}
            \nabla f & \equiv \diffp fx\ \vec{\hat{i}} + \diffp fy\ \vec{\hat{j}}
            + \diffp fz\ \vec{\hat{k}}
        \end{align}
        The differential operator for $ 3d $ Cartsian coordinates is defined as
        \begin{align}
            \nabla & \equiv \diffp {}{x}\ \vec{\hat{i}} + \diffp {}{y}\ \vec{\hat{j}}
            + \diffp {}{z}\ \vec{\hat{k}}
        \end{align}

    \item[Directional derivative] The directional derivative of a function $ f $ at a
        point of $ P $ in the direction of $ \vec{b} $,
        \begin{align}
            D_{\vec{b}}f & \equiv \diff fs =
            \lim_{s \rightarrow 0} \frac{f(Q) - f(P)}{s}
        \end{align}
        Here, $ Q $ is a point on the line passing through $ P $ in the direction of
        $ \vec{b} $, and $ s $ is the distance between these two points. \par
        Defining the line in terms of a unit vector $ \vec{b} $ and arc length $ s $ from
        an intitial point $ \vec{p_0} $,
        \begin{align}
            \vec{r}                  & = \vec{p_0} + s\ \vec{b}       &
            \diff{\vec{r}}{s}        & = \vec{b}                        \\
            D_{\vec{b}}f = \diffp fs & = \vec{\hat{b}} \dotp \nabla f
        \end{align}

    \item[Gradient is a vector] The magnitude and direction of the gradient are
        independent of the choice of Cartesian coordinates. \par
        Also, the gradient (if it is a nonzero vector) points in the direction of maximum
        increase of the function at that point.

    \item[Normal vector to surface] For a function $ f $ whose level surface is defined
        as $ S : f(x, y, z) = c $,
        \begin{align}
            0 & = \diffp fx\ \diffp xs + \diffp fy\ \diffp ys + \diffp fz\ \diffp zs &
            0 & = \nabla f \dotp \vec{r}'
        \end{align}
        Since the gradient is perpendicluar all possible tangent vectors at the point
        $ P $ on surface $ S $, it is the surface normal vector at $ P $.

    \item[Potentials] A scalar field whose gradient happens to be a vector field.
        \begin{align}
            \vec{v}(P) & = \nabla f(P)
        \end{align}
        The corresponding vector field is called conservative.

    \item[Gravitational field] THe force of gravitation between two particles is the best
        known exmaple of a conservative force. \par
        For two particles at $ P_0: (x_0, y_0, z_0) $ and $ P: (x, y, z) $ separated by
        distance $ r $.
        \begin{align}
            \vec{p}                                      & = \frac{-c}{r^3}\ \vec{r}
            = \frac{-c}{r^3} \begin{bNiceMatrix}[margin]
                                 (x - x_0) \\ (y - y_0) \\ (z - z_0)
                             \end{bNiceMatrix} &
            \phi(x, y, z)                                & = \frac{c}{r}             \\
            \vec{p}                                      & = \nabla \phi
        \end{align}

        The vector field of force produced by any mass distribution in a space is the
        gradient of a scalar field (potential) $ \phi $ which satisfies Laplace's
        equation in any region free of matter.

    \item[Laplace's equation] A partial differential equation defined as
        \begin{align}
            \nabla^2 f & \equiv \diffp[2] fx + \diffp[2] fy + \diffp[2] fz = 0
        \end{align}

    \item[Laplacian] The operator which condenses the Laplacian equation, defined as
        \begin{align}
            \nabla \dotp \nabla f & \equiv \nabla^2 f              \\
            \nabla^2              & \equiv \Delta \equiv \diffp[2]
            {}{x} + \diffp[2] {}{y} + \diffp[2] {}{z}
        \end{align}
\end{description}

\section{Divergence of a Vector Field}

\begin{description}
    \item[Divergence] This obtains a scalar field from a vector field. Consider a
        differentiable vector function $ \vec{v} $ with components in $ 3d $ Cartesian
        space, given by
        \begin{align}
            \vec{v}              & = \begin{bNiceMatrix}
                                         v_1(x, y, z) \\ v_2(x, y, z) \\ v_3(x, y, z)
                                     \end{bNiceMatrix} \\
            \nabla \dotp \vec{v} & =
            \begin{bNiceMatrix}
                \difsp {}{x} \\ \difsp {}{y} \\ \difsp {}{z}
            \end{bNiceMatrix} \dotp \begin{bNiceMatrix}
                                        v_1 \\ v_2 \\ v_3
                                    \end{bNiceMatrix} =
            \diffp{v_1}{x} + \diffp{v_2}{y} + \diffp{v_3}{z}
        \end{align}
        Note that the dot product notation is not the usual multiplication. It represents
        the partial derivative acting on the components of $ \vec{v} $.

    \item[Invariance of divergence] Since the divergence is a scalar field, it is
        independent of the choice of coordinate system. \par
        It only depends on the point $ P $ in space and the form of $ \vec{v} $ for a
        particular choice of coordinate system.

    \item[Fluid flow] An example of divergence used in physical systems is the flow of
        a fluid that has density $ \rho $ in a region $ R $ with no sources and sinks.
        \par
        Let $ \vec{v} $ be the velocity vector of the fluid.
        \begin{align}
            \diffp{\rho}{t} + \nabla \dotp (\rho\vec{v}) & = 0
        \end{align}
        This is the continuity equation of a compressible fluid flow

    \item[Solenoidal vector field] A special case of the above system is steady flow,
        which means that $ \rho $ is independent of time, and constant fluid density. This
        gives the incompressibility condition
        \begin{align}
            \nabla \dotp \vec{v} & = 0
        \end{align}
        Such a vector field is called solenoidal.
\end{description}

\section{Curl of a Vector Field}

\begin{description}
    \item[Curl] Let $ \vec{v} $ be a differentiable vector function in $ 3d $ defined in
        Cartesian coordinates,
        \begin{align}
            \nabla \times \vec{v} & =
            \begin{vNiceMatrix}[margin]
                \vec{\hat{i}} & \vec{\hat{j}} & \vec{\hat{k}} \\
                \difsp{}{x}   & \difsp{}{y}   & \difsp{}{z}   \\
                v_1           & v_2           & v_3
            \end{vNiceMatrix}
        \end{align}
        This assumes the coordinate system is right handed. \par
        The curl is a vector independent of the choice of coordinate system.

    \item[Curl of rotating body] The curl is in the direction of the axis of rotation and
        its magnitude is equal to twice the angular speed.

    \item[Relating curl, grad and div] A vector function that is the gradient of a
        continouous differentiable scalar function is irrotational.
        \begin{align}
            \nabla \times (\nabla f) & = 0 &  & \text{Irrotational}
        \end{align}

        For a vector function $ \vec{v} $ that is twice continuously differentiable,
        \begin{align}
            \nabla \dotp (\nabla \times \vec{v}) = 0
        \end{align}
        These relations are proved using the fact that second order partial derivaties
        commute.
\end{description}