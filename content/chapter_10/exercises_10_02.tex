\section{Path Independence of Line Integrals}

\begin{enumerate}
    \item Refer notes. TBC.

    \item The domain is still not simply connected, as the origin is still excluded from
          it. The situation does not change.

    \item The integrand is exact
          \begin{align}
              I            & = M \dl x + N \dl y                       &
              I            & = \difcp fx\ \dl x + \difcp fy\ \dl y       \\
              \int M \dl x & =\sin(0.5x) \cos(2y) + g(y)               &
              \difcp fy    & = \diff gy - 2\sin(0.5x)\sin(2y)            \\
              \difcp fy    & = N                                       &
              \diff gy     & = 0                                         \\
              f            & =\color{y_h} \sin(0.5x)\cos(2y)           &
              A            & : (\pi/2, \pi) \quad B: (\pi, 0)            \\
              I            & = f(B) - f(A) = \Bigg[ \sin(0.5x)\cos(2y)
              \Bigg]_A^B   &
              I            & = \color{y_p} 1 - \frac{1}{\sqrt{2}}
          \end{align}

    \item The integrand is exact
          \begin{align}
              I            & = M \dl x + N \dl y                   &
              I            & = \difcp fx\ \dl x + \difcp fy\ \dl y   \\
              \int M \dl x & = x^2 e^{4y} + g(y)                   &
              \difcp fy    & = \diff gy + 4 x^2 e^{4y}               \\
              \difcp fy    & = N                                   &
              \diff gy     & = 0                                     \\
              f            & =\color{y_h} x^2 e^{4y}               &
              A            & : (4, 0) \quad B: (6, 1)                \\
              I            & = f(B) - f(A) = \Bigg[ x^2 e^{4y}
              \Bigg]_A^B   &
              I            & = \color{y_p} 36e^4 - 16
          \end{align}

    \item The integrand is exact
          \begin{align}
              I                & = M \dl x + N \dl y + P \dl z =
              \difcp fx\ \dl x + \difcp fy\ \dl y + \difcp fz\ \dl z             \\
              \int M \dl x = f & = e^{xy}\sin(z) + g(y, z)                       \\
              \difcp fy = N    & = \diffp gy + xe^{xy}\sin(z)
              \quad \implies \quad  \diffp gy = 0                                \\
              g                & = h(z)                                          \\
              \difcp fz = P    & = \diff hz + e^{xy} \cos(z)
              \quad \implies \quad \diff hz = 0                                  \\
              f                & = {\color{y_h} e^{xy} \sin(z)}
              \qquad A : (0, 0, \pi) \qquad B: (2, 1/2, \pi/2)                   \\
              I                & = f(B) - f(A) = \Bigg[ e^{xy}\sin(z) \Bigg]_A^B
              = \color{y_p} e
          \end{align}

    \item The integrand is exact
          \begin{align}
              I                & = M \dl x + N \dl y + P \dl z =
              \difcp fx\ \dl x + \difcp fy\ \dl y
              + \difcp fz\ \dl z                                                 \\
              \int M \dl x = f & = 0.5 \cdot \exp(x^2 + y^2 + z^2) + g(y, z)     \\
              \difcp fy = N    & = \diffp gy + y\exp(x^2 + y^2 + z^2)
              \quad \implies \quad  \diffp gy = 0                                \\
              g                & = h(z)                                          \\
              \difcp fz = P    & = \diff hz + z \exp(x^2 + y^2 + z^2)
              \quad \implies \quad \diff hz = 0                                  \\
              f                & = {\color{y_h} 0.5 \cdot e^{x^2 + y^2 + z^2}}
              \qquad A : (0, 0, 0) \qquad B: (1, 1, 0)                           \\
              I                & = f(B) - f(A) = \Bigg[ e^{xy}\sin(z) \Bigg]_A^B
              = \color{y_p} \frac{e^2 - 1}{2}
          \end{align}

    \item The integrand is exact
          \begin{align}
              I                & = M \dl x + N \dl y + P \dl z =
              \difcp fx\ \dl x + \difcp fy\ \dl y
              + \difcp fz\ \dl z                                              \\
              \int M \dl x = f & = y\cosh(xz) + g(y, z)                       \\
              \difcp fy = N    & = \diffp gy + \cosh(xz)
              \quad \implies \quad  \diffp gy = 0                             \\
              g                & = h(z)                                       \\
              \difcp fz = P    & = \diff hz + xy \sinh(xz)
              \quad \implies \quad \diff hz = 0                               \\
              f                & = {\color{y_h} y\cosh(xz)}
              \qquad A : (0, 2, 3) \qquad B: (1, 1, 1)                        \\
              I                & = f(B) - f(A) = \Bigg[ y\cosh(xz) \Bigg]_A^B
              = \color{y_p} \cosh(1) - 2
          \end{align}

    \item The integrand is exact
          \begin{align}
              I                & = M \dl x + N \dl y + P \dl z =
              \difcp fx\ \dl x + \difcp fy\ \dl y
              + \difcp fz\ \dl z                                              \\
              \int M \dl x = f & = x \cos(yz) + g(y, z)                       \\
              \difcp fy = N    & = \diffp gy - xz \sin(yz)
              \quad \implies \quad  \diffp gy = 0                             \\
              g                & = h(z)                                       \\
              \difcp fz = P    & = \diff hz - xy \sin(yz)
              \quad \implies \quad \diff hz = 0                               \\
              f                & = {\color{y_h} x \cos(yz)}
              \qquad A : (5, 3, \pi) \qquad B: (3, \pi, 3)                    \\
              I                & = f(B) - f(A) = \Bigg[ x \cos(yz) \Bigg]_A^B
              = \color{y_p} 2
          \end{align}

    \item The integrand is exact
          \begin{align}
              I                & = M \dl x + N \dl y + P \dl z =
              \difcp fx\ \dl x + \difcp fy\ \dl y
              + \difcp fz\ \dl z                                                    \\
              \int M \dl x = f & = e^x \cosh(y) + g(y, z)                           \\
              \difcp fy = N    & = \diffp gy + e^x \sinh(y)
              \quad \implies \quad  \diffp gy = e^z \cosh(y)                        \\
              g                & = e^z\sinh(y) + h(z)                               \\
              \difcp fz = P    & = \diff hz + e^z \sinh(y)
              \quad \implies \quad \diff hz = 0                                     \\
              f                & = {\color{y_h} e^x \cosh(y) + e^z \sinh(y)}
              \qquad A : (0, 1, 0) \qquad B: (1, 0, 1)                              \\
              I                & = f(B) - f(A) = \Bigg[ e^x \cosh(y) + e^z \sinh(y)
                  \Bigg]_A^B = \color{y_p} 0
          \end{align}

    \item Path Dependence
          \begin{enumerate}
              \item Checking path dependence,
                    \begin{align}
                        \vec{F} \dotp \dl{\vec{r}} & = x^2 y\ \dl x + 2xy^2\ \dl y &
                        \vec{F}                    & =
                        \begin{bNiceMatrix}[margin]
                            x^2y \\ 2xy^2 \\ 0
                        \end{bNiceMatrix}                                   \\
                        \nabla \times \vec{F}      & = 2y^2 - x^2                  &
                        \nabla \times \vec{F}      & \neq 0
                    \end{align}
                    Since the vector field has nonzero curl, the line integral is path
                    dependent.

              \item Integrating along the first path,
                    \begin{align}
                        \vec{F}             & = \bmatcol{x^2y}{2xy^2}              &
                        \vec{r}             & = \bmatcol{t}{bt} \quad t \in [0, 1]   \\
                        \vec{r}'            & = \bmatcol{1}{b}                     &
                        \vec{F}(\vec{r}(t)) & = \bmatcol{bt^3}{2b^2t^3}              \\
                        W                   & = \int_{0}^{1}(b + 2b^3)t^3\ \dl t   &
                                            & = \Bigg[\frac{b + 2b^3}{4}\ t^4
                        \Bigg]_0^1                                                   \\
                                            & = \color{y_h} \frac{b + 2b^3}{4}
                    \end{align}
                    Integrating the second path
                    \begin{align}
                        \vec{F}             & = \bmatcol{x^2y}{2xy^2}             &
                        \vec{r}             & = \bmatcol{1}{t} \quad t \in [b, 1]   \\
                        \vec{r}'            & = \bmatcol{0}{1}                    &
                        \vec{F}(\vec{r}(t)) & = \bmatcol{t}{2t^2}                   \\
                        W                   & = \int_{b}^{1} 2t^2\ \dl t          &
                                            & = \Bigg[ \frac{2t^3}{3}
                        \Bigg]_b^1                                                  \\
                                            & = \color{y_h} \frac{2(1 - b^3)}{3}
                    \end{align}
                    Optimizing the line integral w.r.t. $ b $,
                    \begin{align}
                        I        & = \frac{2}{3} + \frac{b}{4} - \frac{b^3}{6} &
                        \diff Ib & = \frac{1}{4} - \frac{b^2}{2}                 \\
                        b^*      & = \frac{1}{\sqrt{2}}                        &
                        I^*      & = \frac{2}{3} + \frac{1}{6\sqrt{2}}
                    \end{align}
              \item Integrating along the third path,
                    \begin{align}
                        \vec{F}             & = \bmatcol{x^2y}{2xy^2}              &
                        \vec{r}             & = \bmatcol{ct}{t} \quad t \in [0, 1]   \\
                        \vec{r}'            & = \bmatcol{c}{1}                     &
                        \vec{F}(\vec{r}(t)) & = \bmatcol{c^2t^3}{2ct^3}              \\
                        W                   & = \int_{0}^{1}(c^3 + 2c)t^3\ \dl t   &
                                            & = \Bigg[\frac{c^3 + 2c}{4}\ t^4
                        \Bigg]_0^1                                                   \\
                                            & = \color{y_p} \frac{c^3 + 2c}{4}
                    \end{align}
                    Integrating the fourth path
                    \begin{align}
                        \vec{F}             & = \bmatcol{x^2y}{2xy^2}             &
                        \vec{r}             & = \bmatcol{t}{1} \quad t \in [c, 1]   \\
                        \vec{r}'            & = \bmatcol{1}{0}                    &
                        \vec{F}(\vec{r}(t)) & = \bmatcol{t^2}{2t}                   \\
                        W                   & = \int_{c}^{1} t^2\ \dl t           &
                                            & = \Bigg[ \frac{t^3}{3}
                        \Bigg]_c^1                                                  \\
                                            & = \color{y_p} \frac{(1 - c^3)}{3}
                    \end{align}
                    Optimizing the line integral w.r.t. $ c $,
                    \begin{align}
                        I        & = \frac{4 - c^3 + 6c}{12}                    &
                        \diff Ic & = \frac{-3c^2 + 6}{12}                         \\
                        c^*      & = \sqrt{2} \quad (\text{out of range [0,1]})
                    \end{align}
                    Since the optimal $ c $ is out of range, setting $ c = 1 $ gives
                    $ I^* = 3/4 $. \par
                    Comparing the values at $ b = 1 $ and $ c = 1 $
                    \begin{align}
                        b = 1 & \implies I_{12} = 3/4 \\
                        c = 1 & \implies I_{34} = 3/4
                    \end{align}
          \end{enumerate}

    \item Checking the differential form in Example $ 4 $,
          \begin{align}
              F_1             & = \frac{-y}{x^2 + y^2}            &
              F_2             & = \frac{x}{x^2 + y^2}               \\
              \diffp {F_2}{x} & = \frac{y^2 - x^2}{(x^2 + y^2)^2} &
              \diffp {F_1}{y} & = \frac{y^2 - x^2}{(x^2 + y^2)^2}   \\
              \diffp {F_2}{y} & = \diffp {F_1}{y}
          \end{align}
          The form is exact. Finding the underlying scalar function,
          \begin{align}
              \int F_1 \ \dl x & = -\arctan(x/y) + C + g(y)       &
              \diffp fy        & = \diff gy + \frac{x}{x^2 + y^2}   \\
              g(y)             & = 0                              &
              f                & = \arctan(y/x)
          \end{align}
          For $ x = y = 0 $, the function $ \arctan(y/x) $ is not defined. So any
          domain not including this point in $ \mathcal{R} $ is acceptable.

    \item The centres of the circles lie on the perpendicular bisector of $ A $ and
          \begin{align}
              l_1 & : y = x                                      &
              l_2 & : y = 1 - x                                    \\
              C   & : (\alpha, \beta) = (\alpha, 1 - \alpha)     &
              r^2 & = (\alpha - 0.5)^2 + (0.5 - \alpha)^2 + 0.25   \\
                  &                                              &
                  & = \alpha^2 - 2\alpha + 0.75
          \end{align}
          Integrating the vector function $ \vec{F} $ over the this circle,
          \begin{align}
              \vec{F}  & = \bmatcol{x^2y}{2xy^2}         &
              \vec{r}  & = \bmatcol{\alpha + r\cos t}
              {(1 - \alpha) + r \sin t} \quad
              t \in [0, 2\pi]                              \\
              \vec{r}' & = \bmatcol{-r \sin t}{r \cos t}
          \end{align}
          Calculating the line integral using a CAS,
          \begin{align}
              W                 & = \frac{\pi r^2}{4}\ (4\alpha^2 - 16\alpha + 8
              + r^2)                                                             \\
              W                 & = \frac{\pi}{4}\ (\alpha^2 - 2\alpha + 0.75)
              (5\alpha^2 - 18\alpha + 8.75)
          \end{align}
          This is a fourth order polynomial in $ \alpha $. It has one local maximum and
          no global maximum. 
          \begin{align}
            W^* &= 0.954 & (\alpha^*, \beta^*) &= (1.143, -0.143)
          \end{align}

    \item Checking path independence,
          \begin{align}
              \vec{F}               & =
              \begin{bNiceMatrix}[margin]
                  2xe^{x^2}\cos(2y)   \\
                  -2e^{x^{2}}\sin(2y) \\
                  0
              \end{bNiceMatrix}
              \\
              \nabla \times \vec{F} & =
              \begin{vNiceMatrix}[margin]
                  \vec{\hat{i}}     & \vec{\hat{j}}     & \vec{\hat{k}} \\
                  \difcp{}{x}       & \difcp{}{y}       & \difcp{}{z}   \\
                  2xe^{x^2}\cos(2y) & -2e^{x^2}\sin(2y) & 0
              \end{vNiceMatrix} \\ = \vec{0}
          \end{align}
          This line integral is path independent, and the value of the integral is
          \begin{align}
              I & = f(b) - f(a) = \Bigg[e^{x^2}\cos(2y)\Bigg]_A^B
              = e^{a^2} \cos(2b) - 1
          \end{align}

    \item Checking path independence,
          \begin{align}
              \vec{F}               & =
              \begin{bNiceMatrix}[margin]
                  z\sinh(xy) \\
                  0          \\
                  -x\sinh(xy)
              \end{bNiceMatrix}
              \\
              \nabla \times \vec{F} & =
              \begin{vNiceMatrix}[margin]
                  \vec{\hat{i}} & \vec{\hat{j}} & \vec{\hat{k}} \\
                  \difcp{}{x}   & \difcp{}{y}   & \difcp{}{z}   \\
                  z\sinh(xy)    & 0             & -x\sinh(xy)
              \end{vNiceMatrix} =
              \begin{bNiceMatrix}[margin]
                  -x^2 \cosh(xy)           \\
                  2\sinh(xy) + xy\cosh(xy) \\
                  xz\cosh(xy)
              \end{bNiceMatrix}
          \end{align}
          This line integral is path dependent.

    \item Checking path independence,
          \begin{align}
              \vec{F}               & =
              \begin{bNiceMatrix}[margin]
                  x^2y \\ -4xy^2 \\ 8z^2x
              \end{bNiceMatrix}
              \\
              \nabla \times \vec{F} & =
              \begin{vNiceMatrix}[margin]
                  \vec{\hat{i}} & \vec{\hat{j}} & \vec{\hat{k}} \\
                  \difcp{}{x}   & \difcp{}{y}   & \difcp{}{z}   \\
                  x^2y          & -4xy^2        & 8z^2x
              \end{vNiceMatrix} =
              \begin{bNiceMatrix}[margin]
                  0 \\ -8z^2 \\ -4y^2 + x^2
              \end{bNiceMatrix}
          \end{align}
          This line integral is path dependent.

    \item Checking path independence,
          \begin{align}
              \vec{F}               & =
              \begin{bNiceMatrix}[margin]
                  e^y \\ xe^y - e^z \\ -ye^z
              \end{bNiceMatrix}
              \\
              \nabla \times \vec{F} & =
              \begin{vNiceMatrix}[margin]
                  \vec{\hat{i}} & \vec{\hat{j}} & \vec{\hat{k}} \\
                  \difcp{}{x}   & \difcp{}{y}   & \difcp{}{z}   \\
                  e^y           & xe^y - e^z    & -ye^z
              \end{vNiceMatrix} = \begin{bNiceMatrix}[margin]
                                      0 \\ 0 \\ 0
                                  \end{bNiceMatrix}
          \end{align}
          This line integral is path independent, and the value of the integral is
          \begin{align}
              I & = f(b) - f(a) = \Bigg[ xe^y - ye^z \Bigg]_A^B
              = ae^b - be^c
          \end{align}

    \item Checking path independence,
          \begin{align}
              \vec{F}               & =
              \begin{bNiceMatrix}[margin]
                  4y \\ z \\ y - 2z
              \end{bNiceMatrix}
              \\
              \nabla \times \vec{F} & =
              \begin{vNiceMatrix}[margin]
                  \vec{\hat{i}} & \vec{\hat{j}} & \vec{\hat{k}} \\
                  \difcp{}{x}   & \difcp{}{y}   & \difcp{}{z}   \\
                  4y            & z             & y - 2z
              \end{vNiceMatrix} = \begin{bNiceMatrix}[margin]
                                      0 \\ 0 \\ -4
                                  \end{bNiceMatrix}
          \end{align}
          This line integral is path dependent.

    \item Checking path independence,
          \begin{align}
              \vec{F}               & =
              \begin{bNiceMatrix}[margin]
                  yz \cos(xy) \\ xz \cos(xy) \\ -2\sin(xy)
              \end{bNiceMatrix}
              \\
              \nabla \times \vec{F} & =
              \begin{vNiceMatrix}[margin]
                  \vec{\hat{i}} & \vec{\hat{j}} & \vec{\hat{k}} \\
                  \difcp{}{x}   & \difcp{}{y}   & \difcp{}{z}   \\
                  yz \cos(xy)   & xz \cos(xy)   & -2\sin(xy)
              \end{vNiceMatrix} =
              \begin{bNiceMatrix}[margin]
                  -x\cos(xy) \\
                  2y\cos(xy) \\
                  0
              \end{bNiceMatrix}
          \end{align}
          This line integral is path dependent.

    \item Checking path independence, using $ w = x^2 + 2y^2 + z^2 $
          \begin{align}
              \vec{F}               & = \cos(w)
              \begin{bNiceMatrix}[margin]
                  2x \\ 4y \\ 2z
              \end{bNiceMatrix}
              \\
              \nabla \times \vec{F} & = \cos(w)
              \begin{vNiceMatrix}[margin]
                  \vec{\hat{i}} & \vec{\hat{j}} & \vec{\hat{k}} \\
                  \difcp{}{x}   & \difcp{}{y}   & \difcp{}{z}   \\
                  2x            & 4y            & 2z
              \end{vNiceMatrix} = \begin{bNiceMatrix}[margin]
                                      0 \\ 0 \\ 0
                                  \end{bNiceMatrix}
          \end{align}
          This line integral is path independent, and the value of the integral is
          \begin{align}
              I & = f(b) - f(a) = \Bigg[ \sin(x^2 + 2y^2 + z^2) \Bigg]_A^B
              = \sin(a^2 + 2b^2 + c^2)
          \end{align}

    \item The three conditions are,
          \begin{align}
              \difcp{F_3}{y} & = \difcp{F_2}{z}              &
              \difcp{F_1}{z} & = \difcp{F_3}{x}                \\
              \difcp{F_2}{x} & = \difcp{F_1}{y}              &
              \vec{F}        & = \begin{bNiceMatrix}[margin]
                                     0 \\ x \\ 0
                                 \end{bNiceMatrix}
          \end{align}
          Other simple examples have a different permutation of zero components.

\end{enumerate}