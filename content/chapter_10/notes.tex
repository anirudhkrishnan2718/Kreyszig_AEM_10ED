\chapter{Vector Integral Calculus, Integral Theorems}

\section{Line Integrals}

\begin{description}
    \item[Curve integral] A generalization of the usual definite integral, which now
        integrates over a general one-dimensional curve in space, instead of the $ x $
        axis.
        \begin{align}
            I          & = \int_{a}^{b} f(x)\ \dl x              \\
            \vec{r}(t) & = \begin{bNiceMatrix}[margin]
                               x(t) \\ y(t) \\ z(t)
                           \end{bNiceMatrix} \qquad t \in [a, b]
        \end{align}
        The path of integration is parametrized using a single parameter $ (t) $.
        \par If the initial and terminal point of the curve coincide, it is called a
        closed path. \par

    \item[Smooth curve] A curve which has a unique tangent defined at each point that
        varies continuously when traversing $ C $. \par
        $ \vec{r}(t) $ is differentiable and $ \vec{r}'(t) $ is continuous and not the
        zero vector at any point along $ C $.

    \item[Line Integral] Curve integrals are commonly called line integrals, even if the
        path itself is not a straight line. They are defined for some vector function
        $ \vec{F}(\vec{r}) $ over a curve $ C $ by,
        \begin{align}
            \int_{C} \vec{F}(\vec{r}) \dotp \dl{\vec{r}} & =
            \int_{a}^{b} \vec{F}(\vec{r}(t)) \dotp \vec{r}'(t)\ \dl t \\
            \int_{C} (F_1 \dl x + F_2 \dl y + F_3 \dl z) & =
            \int_{a}^{b} (F_1 x' + F_2 y' + F_3 z')\ \dl t
        \end{align}
        The above definition requires right-handed $ 3d $ Cartesian coordinates. \par
        The differentiation is w.r.t. the parameter $ t $.

        The line integral can be thought of as the sum of many infinitesimal vectors each
        of which is the tangential component of $ \vec{F} $ at $ \vec{r} $. Thus,
        \begin{align}
            \dl I & = \vec{F} \dotp \frac{\vec{r}'}{\abs{\vec{r'}}}
        \end{align}
        The integrand is piecewise continuous if $ \vec{F} $ is continuous and the curve
        $ C $ is at least piecewise smooth.

    \item[Properties of line integral] Similar to the definite scalar integral,
        \begin{align}
            \int_{C} k\vec{F}  \dotp \dl{\vec{r}}           & =
            k\ \int_{C} \vec{F} \dotp \dl{\vec{r}}              \\
            \int_{C} (\vec{F} + \vec{G}) \dotp \dl{\vec{r}} & =
            \int_{C} \vec{F} \dotp \dl{\vec{r}} +
            \int_{C} \vec{G} \dotp \dl{\vec{r}}                 \\
            \int_{C} \vec{F} \dotp \dl{\vec{r}}             & =
            \int_{C_1} \vec{F} \dotp \dl{\vec{r}} +
            \int_{C_2} \vec{F} \dotp \dl{\vec{r}}
        \end{align}
        Splitting a path into pieces that have the same orientation does not change
        the line integral.

    \item[Invariance of line integral] Any representation of the path $ C $ that
        preserve the orientation of the curve, do not change the line integral.

    \item[Work done by a force] When the path is no longer a straight line, the total
        work done by a force is the sum of many small displacements along the path $ C $
        \begin{align}
            \dl W & = \vec{F}(\vec{r}) \dotp \dl{\vec{r}}          &
            W     & = \int_{C} \vec{F}(\vec{r}) \dotp \dl{\vec{r}}
        \end{align}

    \item[Vector line integral] Without taking the dot product, the line integral
        can be evaluated separately for each component to give,
        \begin{align}
            \int_{C} \vec{F}(\vec{r})\ \dl t &
            = \int_{a}^{b} \vec{F}(\vec{r}(t))\ \dl t =
            \begin{bNiceMatrix}[margin]
                \int_{a}^{b} F_1 (\vec{r}(t))\ \dl t \\
                \int_{a}^{b} F_2 (\vec{r}(t))\ \dl t \\
                \int_{a}^{b} F_3 (\vec{r}(t))\ \dl t \\
            \end{bNiceMatrix}
        \end{align}

    \item[Path Dependence] Generally, the path taken affects the outcome of the line
        integral even if the function $ \vec{F}(\vec{r}) $ and the endpoints remain the
        same.
\end{description}

\section{Path Independence of Line Integrals}

\begin{description}
    \item[Path Independene] A line integral is path independent in domain $ D $ if for
        all pairs of points $ (A, B) $ in this domain, the path integral has the same
        value regardless of the path taken from $ A $ to $ B $.

    \item[Gradient of scalar function] A line integral (with continouous vector
        components in the integrand) is path independent if and only if the vector
        happens to
        be the gradient of some scalar function in the same domain.
        \begin{align}
            \vec{F} = \nabla f \quad \iff \quad
            \int_{A}^{B} \vec{F} \dotp \dl{\vec{r}} = f(B) - f(A)
        \end{align}
        This scalar function $ f $ is called the potential of the vector field
        $ \vec{F} $.

    \item[Closed line integral] A line integral is path independent if and only if
        its value around any possible closed path is zero.

    \item[Conservative field] In physics, if the work done by a force $ \vec{F} $ is
        path independent, then the vector field is called conservative. Else, it is called
        dissipative.

    \item[Exact differential form] If the integrand $ \vec{F} \dotp \dl{\vec{r}} $ is
        equal to an infinitesimal change in some scalar function $ f $,
        \begin{align}
            \vec{F} \dotp \dl{\vec{r}} & = F_1 \dl x + F_2 \dl y + F_3 \dl z    \\
                                       & =\diffp fx\ \dl x + \diffp fy\ \dl y +
            \diffp fz\ \dl z = \dl f
        \end{align}
        Here, $ f(x, y, z) $ is some differentiable function in the domain $ D $. \par
        This reduces to the condition that the above equation is exact if and only if
        the vector function $ \vec{F} $ is the gradient of the differentiable scalar
        function $ f $ everywhere in the domain $ D $.

    \item[Simply connected domain] Any closed curve in the domain $ D $ can be
        continuously shrunk to any point in $ D $ without leaving $ D $.

    \item[Criterion for exactness] Let the components of a vector function be continuous
        and have continuous first partial derivatives  in a domain $ D $. Then,

        \begin{itemize}
            \item If the differential form is exact in $ D $ then the vector field is
                  irrotational.
                  \begin{align}
                      \vec{F} \dotp \dl{\vec{r}} = \dl f \quad \implies
                      \quad  \nabla \times F = 0
                  \end{align}

            \item Conversely, if the vector field $ \vec{F} $ is irrotational in a
                  simply connected domain $ D $, then the line integral is path
                  independent.
        \end{itemize}
\end{description}

\section{Calculus Review: Double Integrals}

\begin{description}
    \item[Area] Double integration happens over any bounded region in $ \mathcal{R}^2 $
        whose boundary curve has a unique tanget at all points. \par
        The only exception is a finite number of cusps. (Discontinuity in the tangent
        vector over the boundary). \par
        This is analogous to a regular integral having at most a finite number of jump
        discontinuities.

    \item[Definition of double integral] Assuming some function $ f(x, y) $ is
        continuous in $ \mathcal{R} $ and this region is bounded by finitely many smooth
        curves, the integral over the area is
        \begin{align}
            I & \equiv \iint_R f(x, y)\ \dl A = \iint_R f(x, y)\ \dl x\ \dl y
        \end{align}

    \item[Properties of double integral] Similar to the definite single integral,
        \begin{align}
            \iint_R\ kf\ \dl A    & = k\ \iint_R f\ \dl A                          \\
            \iint_R\ (f+g)\ \dl A & = \iint_R f\ \dl A +  \iint_R g\ \dl A         \\
            \iint_R\ (f+g)\ \dl A & = \iint_{R_1} f\ \dl A +  \iint_{R_2} f\ \dl A
        \end{align}
        The area to be integrated over can be split into parts just like the line segment
        for the case of the single integral.

    \item[Mean value theorem] If the area of integration $ R $ as defined above is also
        simply connected, then
        \begin{align}
            \iint_R f(x, y)\ \dl A & = f(x_0, y_0)\ A
        \end{align}
        for some point $ (x_0, y_0) $ in $ R $ and $ A $ being the total area of $ R $.

    \item[Evaluation] Sucessively integrating over $ x $ then $ y $ is the most
        common means of performing double integration.
        \begin{align}
            \iint_R f\ \dl x\ \dl y & = \int_{a}^{b} \left[ \int_{g(x)}^{h(x)}
                f(x, y)\ \dl y \right] \dl x
        \end{align}
        Here, $ x $ is considered a constant when performing the integration over $ y $.
        \par The order of integration can also be reversed if the computation happens to
        be simpler.
        \begin{figure}[H]
            \centering
            \begin{tikzpicture}
                \begin{axis}[
                        width = 8cm, height = 8cm,
                        xmin = 0, xmax = 3.5, ymin = 0, ymax = 3.5,
                        axis lines = middle,
                        xlabel = \normalsize $ x $, ylabel = \normalsize $ y $,
                        ymajorticks = false,
                        axis equal, xtick = {1, 3},
                        xticklabels = {\normalsize $ a $, \normalsize $ b $},
                        Ani]
                    \addplot[GraphSmooth, y_h, domain= pi : 2*pi, variable=\t]
                    ({2 + cos(\t)}, {2 + sin(\t)})
                    node[midway, below] {$ g(x) $};
                    \addplot[GraphSmooth, y_p, domain= 0 : pi, variable=\t]
                    ({2 + cos(\t)}, {2 + sin(\t)})
                    node[midway, above] {$ h(x) $};
                    \draw[dashed, black] (1, 0) -- (1, 2);
                    \draw[dashed, black] (3, 0) -- (3, 2);
                    \filldraw[draw = black!0, fill=gray!15](2, 2) circle (0.975);
                    \node[GraphNode] at (axis cs:1,2) {};
                    \node[GraphNode] at (axis cs:3,2) {};
                \end{axis}
            \end{tikzpicture}
        \end{figure}
        If the region $ R $ cannot be represented in the above form, it must at least be
        divisible into a finite number of areas that can.

    \item[Applications] The double integral of the unit function is simply
        the area enclosed by the boundary curves.
        \begin{align}
            \iint_R \dl A & = A_{\text{total}}
        \end{align}
        The volume beneath a surface $ z = f(x, y) > 0 $ and the $ xy $ plane is equal
        to its double integral over the area projected onto the $ xy $ plane.
        \begin{align}
            \iint_R f(x, y)\ \dl x\ \dl y = V
        \end{align}
        This is analogous to the area between the $ x $ axis and a one dimensional curve
        being its definite integral.

    \item[Change of variables] Starting with the single integral analog
        \begin{align}
            \int_{a}^{b}f(x) \dl x      & = \int_{\alpha}^{\beta}
            f\Big(x(u)\Big)\ \diff xu \dl u                       \\
            \iint_R f(x, y) \dl x \dl y & =
            \iint_{R^*} f\Big(x(u, v),\ y(u, v)\Big)\ \abs{J} \dl u \dl v
        \end{align}

    \item[Jacobian] A determinant that supplies the partial derivatives of each of the
        old variables w.r.t. each of the new variables.
        \begin{align}
            J & = \begin{vNiceMatrix}[r, margin]
                      \difcp xu & \difcp xv \\ \difcp yu & \difcp yv
                  \end{vNiceMatrix} =
            \diffp xu \ \diffp yv - \diffp xv \ \diffp yu
        \end{align}
        Assuming the functions $ x(u, v) $ and $ y(u, v) $ are continuous and have
        continuous first partial derivatives in some region $ R^* $ in the $ uv $ plane.

        \par There has to be a bijective mapping from $ (u, v) $ in $ R^* $ to $ (x, y) $
        in $ R $. The Jacobian has the same sign throughout $ R^* $

\end{description}

\section{Green's Theorem in the Plane}

\begin{description}
    \item[Utility] Converting a double integral over a region $ R $ into a line integral
        over its boundary in order to simplify computations.

    \item[Green' theorem] For a closed region $ R $ in 2 dimensions bounded by a finite
        number of smooth curves. Call this boundary $ C $. \par
        Let $ F_1(x, y) $ and $ F_2(x, y) $ be continuous functions in $ R $. Let the
        derivatives $ \difsp{F_1}{y} $ and $ \difsp{F_2}{x} $ also be continuous in a
        domain that is a superset of $ R $.

        \begin{align}
            \iint_R \left( \diffp{F_1}{y} - \diffp{F_2}{x} \right) \dl A & =
            \oint_C F_1 \dl x + F_2 \dl y                                    \\
            \iint (\nabla \times \vec{F}) \dotp \vec{k}\ \dl A           & =
            \oint_C \vec{F} \dotp \dl{\vec{r}}
        \end{align}
        The region $ R $ is to the left when moving along its boundary in order to
        preserve orientation. (in accordance with the direction of cross product)

    \item[Proof] General proof is complex and not covered here. For a specific kind of
        region $ R $ which can be represented by boundary curves in $ x $ or $ y $,
        \begin{align}
            \iint_R \diffp {F_1}{y}\ \dl A & = \int_{a}^{b} \Bigg[
            \int_{u(x)}^{v(x)} \diffp{F_1}{y}\ \dl y \Bigg]\ \dl x                   \\
                                           & = \int_{a}^{b} \Bigg[F_1
            \big( x, v(x) \big) - F_1 \big( x, u(x) \big) \Bigg] \dl x               \\
                                           & = -\int_{a}^{b} F_1 \big( x, u(x) \big)
            - \int_{b}^{a} F_1 \big( x, v(x) \big) = -\oint_C F_1 \big( x, y \big)
        \end{align}
        Since $ u(x) $ and $ v(x) $ both travel from $ x = a $ to $ x = b $, the above
        relation is a round trip starting and ending at $ x = a $, which reduces it
        to the closed line integral needed. \par
        A similar procedure for the closed region traversed from $ y = c $ to $ y = d $,
        via the two curves $ p(y) $ and $ q(y) $ gives the other half of the proof. \par

    \item[Area of region] Using Green's theorem to bypass the double integration,
        \begin{align}
            \iint_R \dl A & = \frac{1}{2} \oint_C \ (x \dl y) - (y \dl x) &
                          & \text{cartesian}                                \\
                          & = \frac{1}{2} \oint_C r^2 \dl \theta          &
                          & \text{polar}
        \end{align}

    \item[Normal derivative] Consider a scalar functionm $ w(x, y) $ that is continuous
        and has continuous first partial derivatives in the region $ R $.
        \begin{align}
            \vec{r}'               & = \diff {\vec{r}}{s} &
            \vec{r}' \dotp \vec{n} & = 0
        \end{align}
        Here, the parameter $ s $ is the arc length of the boundary curve. This ensures that
        the derivative $ \vec{r}' $ is the unit tangent vector. \par
        The outward facing vector perpendicular to $ \vec{r}' $ is the unit normal vector
        $ \vec{n} $. \par
        The component of the gradient along the unit normal vector is called the normal
        derivative.
        \begin{align}
            \nabla w \dotp \vec{n} = \diffp wn
        \end{align}

    \item[Relating Laplacian and normal derivative] Using the substitutions,
        \begin{align}
            F_1                       & = - \diffp wy               &
            F_2                       & = \diffp wx                   \\
            \iint_R \nabla^2 w\ \dl A & = \oint_C\ \diffp wn\ \dl s
        \end{align}
\end{description}

\section{Surfaces for Surface Integrals}

\begin{description}
    \item[Parametric representation] Similar to curves, surfaces can be parametrized
        using two parameters $ (u, v) $ to give
        \begin{align}
            \vec{r} & = \begin{bNiceMatrix}[margin]
                            x \\ y \\ z
                        \end{bNiceMatrix} \equiv
            \begin{bNiceMatrix}[margin]
                x(u, v) \\ y(y, v) \\ z(u, v)
            \end{bNiceMatrix}
        \end{align}
        This maps every point $ (u, v) $ in a region $ R $ of the $ uv $ plane onto
        the surface $ S $ in $ \mathcal{R}^3 $.

    \item[Tangent Plane] Consider one possible curve $ C $ on surface $ S $ that passes
        through a given point $ P $. \par
        Since the curve can be parametrized using a single parameter ($ t $), the chain
        rule gives,
        \begin{align}
            u,\ v                      & \equiv u(t),\ v(t)                  &
            \vec{\tilde{r}}(t)         & \equiv \vec{r}\Big(u(t),\ v(t)\Big)   \\
            \diff {\vec{\tilde{r}}}{t} & = \vec{\tilde{r}}'
            = \diffp {\vec{r}}{u}\ u'
            + \diffp {\vec{r}}{v}\ v'  &
            \vec{\tilde{r}}'           & = \vec{r}_u\ u' + \vec{r}_v\ v'
        \end{align}
        Here the functions $ u(t) $ and $ v(t) $ are continuous and have continuous
        derivatives w.r.t. $ t $.

    \item[Surface Normal vector] The partial derivatives are assumed L.I. and thus span
        the tangent plane at $ P $. Their cross product becomes the normal to the surface
        at $ P $
        \begin{align}
            \vec{N} & = \vec{r}_u \times \vec{r}_v \neq \vec{0}
        \end{align}
        Using the fact that the gradient of a level curve at a given point is the
        surface normal vector,
        \begin{align}
            g(x, y, z) = 0 \quad & \implies \quad \vec{N} = \nabla g
        \end{align}

    \item[Smooth surface] A surface whose normal vector depends continuously on the
        points of the surface.  \par
        At best, the surfaces encountered in practical applications can have a finite
        number of smooth portions.
\end{description}

\section{Surface Integrals}

\begin{description}
    \item[Parametrized surface integral] If the surface $ S $ is parametrized using
        the parameters $ u, v $which lie in region $ R $ of the $ uv $ plane, then,
        \begin{align}
            S :\ \vec{r}(u, v) & = \begin{bNiceMatrix}[margin]
                                       x(u, v) \\ y(u, v) \\ z(u, v)
                                   \end{bNiceMatrix}
        \end{align}
        This surface is piecewise smooth with a normal vector defined at every point.
        \begin{align}
            \vec{N}       & = \vec{r}_u \times \vec{r}_v                          &
            \vec{\hat{n}} & = \frac{\vec{N}}{\abs{\vec{N}}}                         \\
            \iint_S \vec{F} \dotp \vec{\hat{n}}
            \ \dl A       & = \iint_R \vec{F}(\vec{r}) \dotp \vec{N}\ \dl u \dl v
        \end{align}
        The area element of the actual surface $ \vec{\hat{n}}\ \dl A $ maps on to the
        area element $ \abs{\vec{N}} \dl u \dl v $ of the parameter plane $ R $.

    \item[Flux] In physics, the flux through a surface is the amount of some physical
        quantity (mass, heat) passing through the surface. \par
        It can mathematically be represented as the component of a vector field normal
        to the surface.
        \begin{align}
            \iint_S \vec{F} \dotp \vec{\hat{n}}\ \dl A &
            = \iint_S F_1\ \dl y \dl z + F_2\ \dl x \dl z + F_3\ \dl x \dl y
        \end{align}

    \item[Orientation of Surfaces] Changing the orientation of the surface (by choosing
        $ -\vec{n}$ instead of $ \vec{n} $), the value of the surface integral gets
        multiplied by $ -1 $.

    \item[Non-Orientable surface] Usually surfaces which have a positive direction of
        the normal vector at a given point $ P $ also have the same positive direction of
        the normal vector at all other points that can be reached by smoothly translating
        $ P $ along the surface. \par
        Surfaces like the Mobius strip are non-oreintable, because they do not satisfy
        this condition.

    \item[Surface Integral disregarding orientation] Consider a scalar function of
        position $ G(\vec{r}) $ to be integrated over a surface,
        \begin{align}
            \iint_S G(\vec{r})\ \dl A & = \iint_R G(\vec{r})\ \abs{\vec{N}}
            \ \dl u \dl v
        \end{align}
        Here, the normal vector and position vector are both parametrized using
        $ (u, v) $. \par
        One application is to find the mass of a surface, in which case the scalar
        function $ G(\vec{r}) $ is its area density. \par
        The total area of a surface $ A $ can be found using
        \begin{align}
            A & = \iint_S \dl A = \iint_R \abs{\vec{r}_u \times \vec{r}_v}\ \dl u \dl v
        \end{align}

    \item[Parametrization using coordinates] If the parameters chosen happen to be the
        $ x $ and $ y $ coordinates themselves, then,
        \begin{align}
            \vec{r}                   & = \begin{bNiceMatrix}[margin]
                                              x \\ y \\ f(x, y)
                                          \end{bNiceMatrix} \\
            \iint_S G(\vec{r})\ \dl A & = \iint_{R^*} G(x, y, f)\
            \Bigg[ \sqrt{1 + (\difcp fx)^2 + (\difcp fy)^2} \Bigg]\ \dl x \dl y
        \end{align}
        Here the parameter plane $ R^* $ is the $ xy $ plane and geometrically is the
        projection of the surface $ S $ onto the $ xy $ plane. \par
        By convention the normal vector points upwards away from the $ xy $ plane.

\end{description}

\section{Triple Integrals, Divergence Theorem of Gauss}

\begin{description}
    \item[Triple Integral] An integral over a closed, bounded $ 3d $ region in space.
        This volume is bounded by finitely many smooth surfaces. \par
        \begin{align}
            I & = \iiint_T\ f(x, y, z)\ \dl V
        \end{align}

    \item[Gauss' Divergence theorem] Triple integrals over a volume can be transformed
        into double integrals over the boundary surface using the divergence of the
        position vector,
        \begin{align}
            \vec{r}              & = \begin{bNiceMatrix}[margin]
                                         F_1 \\ F_2 \\ F_3
                                     \end{bNiceMatrix}                       &
            \nabla \dotp \vec{r} & = \diffp {F_1}{x} + \diffp{F_2}{y} + \diffp{F_3}{z}
        \end{align}
        For some continuous vector function $ \vec{F} $ in the region $ T $ which is
        continuous and has continuous first partial derivatives in some domain
        containing $ T $,
        \begin{align}
            \iiint_T\ \nabla \dotp F\ \dl V           &
            = \iint_S\ \vec{F} \dotp \vec{n}\ \dl S                        \\
            \iiint_T\ \Bigg[\diffp {F_1}{x} + \diffp{F_2}{y}
            + \diffp{F_3}{z}\Bigg]\ \dl x \dl y \dl z &
            = \iint_S\ F_1 \dl y \dl z + F_2 \dl x \dl z + F_3 \dl x \dl y \\
        \end{align}

    \item[Co-ordinate Invariance of divergence] Since the divergence is a scalar
        function of the position $ P $ in space, it is invariant under change of
        co-ordinate system.
        \begin{align}
            \nabla \dotp \vec{F}(P) & = \lim_{d(T) \to 0}
            \ \frac{1}{V(T)}\ \iint_{S(T)}\ \vec{F} \dotp \vec{n}\ \dl A
        \end{align}
        Here, $ V(T) $ is the volume of the region in space $ T $. \par
        $ S(T) $ is the boundary surface of the region $ T $.
        $ d(T) $ is the distance of the points in $ T $ from a specific point $ P $
        chosen in $ T $ to satisfy the mean value theorem for triple integrals.
\end{description}

\section{Further Applications of the Divergence Theorem}

\begin{description}
    \item[Fluid flow] Consider an incompressible fluid with density $ \rho = 1 $, with
        steady flow that does not varyin time. \par
        If the outward normal vector is $ \vec{n} $ and the fluid flow is characterized by
        the velocity field $ \vec{v} $, then
        \begin{align}
            \iint_S \vec{v} \dotp \vec{n}\ \dl A
        \end{align}
        represents the total mass of fluid flowing out of the region $ T $ bounded by the
        surface $ S $. \par
        There are no sources or sinks in a region $ T $ if and only if
        $ \nabla \dotp \vec{v} = 0$ everywhere in $ T $.

    \item[Heat equation] Since heat flows in the direction of decreasing temperature at a
        rate proportional to the gradient,
        \begin{align}
            \vec{v} & = -K\ \nabla U & \nabla \dotp \vec{v} & = -K\ \nabla^2 U
        \end{align}
        where $ U $ is the temperature, $ t $ is the time and $ K $ is the thermal
        conductivity of the body. \par
        Equating the rate of decrease of heat of the region $ T $ to the total heat
        flowing out of the bounding surface $ S $,
        \begin{align}
            \diffp Ut & = c^2\ \nabla^2 U
        \end{align}
        where $ c $ is the thermal diffusivity of the material. This equation is also
        called the diffusion equation.

    \item[Potential theory] Looking at solutions of Laplace's equation,
        \begin{align}
            \nabla^2 f & = \diff[2] fx + \diffp[2] fy + \diffp[2] fz = 0
        \end{align}
        Any solution of Laplace's equation $ f $ with continuous second-order partial
        derivatives is called a harmonic function.

    \item[Solutions of Laplace's equation] Using the definition of directional derivative
        to define the normal derivative in the direction of the outward normal,
        \begin{align}
            \diffp fn & \equiv \nabla f \dotp \vec{n}
        \end{align}
        If the underlying scalar function $ f $ is such that $ \vec{F} = \nabla f $,
        \begin{align}
            \iiint_T\ \nabla^2 f\ \dl V & = \iint_S\ \diffp fn\ \dl A
        \end{align}
        If $ f $ is a harmonic function then the integral of the normal derivative over
        the bounding surface $ S $ of some region in space $ T $ is zero.

    \item[Green's first formula] A special case of Green's theorem when the vector
        function $ \vec{F} $ is
        \begin{align}
            \vec{F}              & = f\ \nabla g                                 &
            \nabla \dotp \vec{F} & = f\ \nabla^2 g + (\nabla f) \dotp (\nabla g)
        \end{align}
        Substituting into Green's theorem gives,
        \begin{align}
            \iiint_T\ \Big[f\ \nabla^2 g + (\nabla f) \dotp (\nabla g)\Big]\ \dl V
             & = \iint_S \ f\ \diffp gn\ \dl A
        \end{align}

    \item[Green's second formula] An even more special case of Green's theorem, using
        the symmetry in Green's first formula upon interchange of $ f $ and $ g $.
        \begin{align}
            \iiint_T\ \Big[f\ \nabla^2 g - g\ \nabla^2 f \Big]\ \dl V
             & = \iint_S \ \left( f\ \diffp gn - g\ \diffp fn \right)\ \dl A
        \end{align}

    \item[Uniqueness of Harmonic functions] Let $ f $ be harmonic in some domain $ D $
        and equal to zero at every point of the bounding surface $ S $  of a region in
        space $ T $ as defined above. \par
        Then, $ f $ is identically zero in $ T $ \par
        This harmonic function $ f $ is uniquely determined in $ T $ by its values on the
        bounding surface $ S $
\end{description}

\section{Stokes' Theorem}

\begin{description}
    \item[Stokes' theorem] A generalization of Green's theorem in the plane to the full
        $ 3d $ space, using the curl. \par
        Consider a surface $ S $ whose bounding curve is parametrized using the arc length
        $ s $, with the outward normal unit vector being $ \vec{n} $
        \begin{align}
            \iint_R\ (\nabla \times \vec{F}) \dotp \vec{n}\ \dl A
             & = \oint_C\ \vec{F} \dotp \vec{r}'(s)\ \dl s
        \end{align}
        The unit tangent vector $ \vec{r}' $ is differentiated w.r.t. the arc length
        $ s $. \par
        The orientation of the curve is the same convention as for the cross product.
        \begin{align}
            \vec{n}\ \dl A  & = \vec{N}\ \dl u \dl v                      \\
            \vec{r}'\ \dl s & = \dl x \vec{\hat{i}} + \dl y \vec{\hat{j}}
            + \dl z \vec{\hat{k}}
        \end{align}
        In terms of the $ xyz $ coordinate system,
        \begin{align}
             & \iint_R \Bigg[\left( \diffp {F_3}{y} - \diffp{F_2}{z} \right) N_1
                + \left( \diffp {F_1}{z} - \diffp{F_3}{x} \right) N_2
                + \left( \diffp {F_2}{x} - \diffp{F_1}{y} \right) N_3\Bigg]
            \ \dl u \dl v                                                        \\
             & = \oint_C\ F_1\ \dl x + F_2\ \dl y + F_3\ \dl z
        \end{align}

    \item[Relation to Green's theorem] Green's theorem in the plane is a special case
        of Stokes' theorem,
        \begin{align}
            \vec{F}               & = \begin{bNiceMatrix}[margin]
                                          F_1 \\ F_2 \\ 0
                                      \end{bNiceMatrix}    &
            \nabla \times \vec{F} & =
            \begin{bNiceMatrix}[margin]
                -\difcp {F_2}{z} \\
                \difcp {F_1}{z}  \\
                \difcp {F_2}{x} - \difcp{F_1}{y}
            \end{bNiceMatrix}                          \\
            \vec{n}               & = \begin{bNiceMatrix}[margin]
                                          0 \\ 0 \\ 1
                                      \end{bNiceMatrix}    &
            \vec{r}'              & = \begin{bNiceMatrix}[margin]
                                          \dl x \\ \dl y \\ 0
                                      \end{bNiceMatrix}     \\
            \iint_S\ \Biggl( \diffp{F_2}{x} - \diffp{F_1}{y} \Biggr) \dl A
                                  & = \oint_C\ F_1 \dl x + F_2 \dl y
        \end{align}

    \item[Physical meaning of curl] Consider a disk of radius $ r_0 $ whose boundary
        is $ C_0 $ and the enlosed area is $ S_0 $,
        \begin{align}
            \oint_{C_0} \vec{v} \dotp \vec{r}'\ \dl s
             & = \iint_{S_0}\ (\nabla \times \vec{v}) \dotp \vec{n}\ \dl A \\
             & = \Big[(\nabla \times \vec{v}) \dotp \vec{n}\Big]_{Q}\ A_0
        \end{align}
        This is the $ 2d $ analog of the mean value theorem with the area of the disk
        $ A_0 $ and $ Q $ being some point in the area for which the equality holds.

        \begin{align}
            \Big[(\nabla \times \vec{v}) \dotp \vec{n}\Big]_Q
             & = \lim_{r_0 \to 0}
            \ \frac{1}{A_0}\ \oint_{C_0}\ \Biggl( \vec{v} \dotp \vec{r}'
            \Biggr)\ \dl s
        \end{align}
        If $ \vec{v} $ is the velocity vector, then the component of the curl in the
        outward normal direction can be visualized as a measure of the circulation of
        the fluid flow around the point $ Q $.

    \item[Path Independence] The proof for path independence of a line integral is
        straightforward using Stokes' theorem. \par
        Starting with the fact that the curl is zero, which means that the vector
        function is the divergence of a scalar field, Stokes' theorem makes the
        line integral over any closed path identically zero. \par
\end{description}