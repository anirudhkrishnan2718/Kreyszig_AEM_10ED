\section{Surfaces for Surface Integrals}

\begin{enumerate}
    \item Finding the parameter curves in $ 2d $,
          \begin{align}
              \vec{r} & = \bmatcol{u}{v}   \\
              u = c   & \implies x = c   &
              v = c   & \implies y = c
          \end{align}
          The parameter curves are straight lines parallel to one of the axes. \par
          The normal vector is,
          \begin{align}
              \vec{r}_u & = \bmatcol{1}{0}               &
              \vec{r}_v & = \bmatcol{0}{1}                 \\
              \vec{N}   & = \vec{r}_u \times \vec{r}_v =
              \begin{bNiceMatrix}[margin]
                  0 \\ 0 \\ 1
              \end{bNiceMatrix}
          \end{align}

    \item Finding the parameter curves in $ 2d $,
          \begin{align}
              \vec{r} & = \bmatcol{u \cos v}{u \sin v}   \\
              u = c   & \implies x^2 + y^2 = c^2       &
              v = c   & \implies \frac{y}{x} = \tan(c)
          \end{align}
          The parameter curves are circles centered on the origin and straight lines
          passing through the origin. \par
          The normal vector is,
          \begin{align}
              \vec{r}_u & = \bmatcol{\cos v}{\sin v}
                        &
              \vec{r}_v & = \bmatcol{- u\sin v}{u \cos v}
              \\
              \vec{N}   & = \vec{r}_u \times \vec{r}_v
                        &
                        & = \begin{vNiceMatrix}[margin]
                                \vec{\hat{i}} & \vec{\hat{j}} & \vec{\hat{k}} \\
                                \cos v        & \sin v        & 0             \\
                                -u \sin v     & u \cos v      & 0             \\
                            \end{vNiceMatrix} \\
                        & = \begin{bNiceMatrix}[margin]
                                0 \\ 0 \\ u
                            \end{bNiceMatrix}
          \end{align}

    \item Finding the parameter curves in $ 3d $,
          \begin{align}
              \vec{r} & = \begin{bNiceMatrix}[margin]
                              u\cos v \\ u \sin v \\ \lambda u
                          \end{bNiceMatrix}             \\
              u = c   & \implies x^2 + y^2 = c^2 \qquad z = \lambda c \\
              v = c   & \implies \frac{y}{x} = \tan(c)
          \end{align}
          The parameter curves for constant $ u $ are circles in the $ xy $ plane
          centered at the origin at $ z = \lambda c $. \par
          For constant $ v $, the parameter curves are straight lines through the
          origin. \par
          The normal vector is,
          \begin{align}
              \vec{r}_u & =   \begin{bNiceMatrix}[margin]
                                  \cos v \\ \sin v \\ \lambda
                              \end{bNiceMatrix}
                        &
              \vec{r}_v & = \begin{bNiceMatrix}[margin]
                                -u\sin v \\ u \cos v \\ 0
                            \end{bNiceMatrix}
              \\
              \vec{N}   & = \vec{r}_u \times \vec{r}_v
                        &
                        & = \begin{vNiceMatrix}[margin]
                                \vec{\hat{i}} & \vec{\hat{j}} & \vec{\hat{k}} \\
                                \cos v        & \sin v        & \lambda       \\
                                -u \sin v     & u \cos v      & 0             \\
                            \end{vNiceMatrix} \\
                        & = \begin{bNiceMatrix}[margin]
                                -\lambda u \cos(v) \\ -\lambda u \sin(v) \\ u
                            \end{bNiceMatrix}
          \end{align}

    \item Finding the parameter curves in $ 3d $,
          \begin{align}
              \vec{r} & = \begin{bNiceMatrix}[margin]
                              a\cos v \\ b\sin v \\ u
                          \end{bNiceMatrix}                                 \\
              u = c   & \implies \frac{x^2}{a^2} + \frac{y^2}{b^2} = 1  \qquad z = c \\
              v = c   & \implies x = a \cos c \qquad y = b\sin c \qquad z = u
          \end{align}
          The parameter curves for constant $ u $ are ellipses in the $ xy $ plane
          centered at the origin at $ z = c $. \par
          For constant $ v $, the parameter curves are straight lines parallel to the
          $ z $ axis intersecting the $ xy $ plane at $ (a\cos c, b\sin c) $. \par
          The normal vector is,
          \begin{align}
              \vec{r}_u & =   \begin{bNiceMatrix}[margin]
                                  0 \\ 0 \\ 1
                              \end{bNiceMatrix}
                        &
              \vec{r}_v & = \begin{bNiceMatrix}[margin]
                                -a\sin v \\ b \cos v \\ 0
                            \end{bNiceMatrix}
              \\
              \vec{N}   & = \vec{r}_u \times \vec{r}_v
                        &
                        & = \begin{vNiceMatrix}[margin]
                                \vec{\hat{i}} & \vec{\hat{j}} & \vec{\hat{k}} \\
                                0             & 0             & 1             \\
                                -a \sin v     & b \cos v      & 0             \\
                            \end{vNiceMatrix} \\
                        & = \begin{bNiceMatrix}[margin]
                                -b\cos v \\ -a\sin v \\ 0
                            \end{bNiceMatrix}
          \end{align}

    \item Finding the parameter curves in $ 3d $,
          \begin{align}
              \vec{r} & = \begin{bNiceMatrix}[margin]
                              u \cos v \\ u \sin v \\ u^2
                          \end{bNiceMatrix}                  \\
              u = c   & \implies x^2 + y^2 = c^2  \qquad z = c^2      \\
              v = c   & \implies \frac{y}{x} = \tan(c) \qquad z = u^2
          \end{align}
          The parameter curves for constant $ u $ are circles in the $ xy $ plane
          centered at the origin at $ z = c^2 $. \par
          For constant $ v $, the parameter curves are parabolas parallel to the
          $ z $ axis lying in the plane $ y = \tan(c)\ x $. \par
          The normal vector is,
          \begin{align}
              \vec{r}_u & =   \begin{bNiceMatrix}[margin]
                                  \cos v \\ \sin v \\ 2u
                              \end{bNiceMatrix}
                        &
              \vec{r}_v & = \begin{bNiceMatrix}[margin]
                                -u\sin v \\ u \cos v \\ 0
                            \end{bNiceMatrix}
              \\
              \vec{N}   & = \vec{r}_u \times \vec{r}_v
                        &
                        & = \begin{vNiceMatrix}[margin]
                                \vec{\hat{i}} & \vec{\hat{j}} & \vec{\hat{k}} \\
                                \cos v        & \sin v        & 2u            \\
                                -u \sin v     & u \cos v      & 0             \\
                            \end{vNiceMatrix} \\
                        & = \begin{bNiceMatrix}[margin]
                                -2u^2\ \cos v \\ -2u^2\ \sin v \\ u
                            \end{bNiceMatrix}
          \end{align}

    \item Finding the parameter curves in $ 3d $,
          \begin{align}
              \vec{r} & = \begin{bNiceMatrix}[margin]
                              u \cos v \\ u \sin v \\ v
                          \end{bNiceMatrix}                            \\
              u = c   & \implies x = c\cos v \qquad  y = c \sin v  \qquad z = v \\
              v = c   & \implies \frac{y}{x} = \tan(c) \qquad z = c
          \end{align}
          The parameter curves for constant $ u $ are helices in the $ xy $ plane
          centered at the origin, axis being the $ z $ axis and radius $ c $. \par
          For constant $ v $, the parameter curves are straight lines on the $ xy $
          plane at $ z = c $. \par
          The normal vector is,
          \begin{align}
              \vec{r}_u & =   \begin{bNiceMatrix}[margin]
                                  \cos v \\ \sin v \\ 0
                              \end{bNiceMatrix}
                        &
              \vec{r}_v & = \begin{bNiceMatrix}[margin]
                                -u\sin v \\ u \cos v \\ 1
                            \end{bNiceMatrix}
              \\
              \vec{N}   & = \vec{r}_u \times \vec{r}_v
                        &
                        & = \begin{vNiceMatrix}[margin]
                                \vec{\hat{i}} & \vec{\hat{j}} & \vec{\hat{k}} \\
                                \cos v        & \sin v        & 0             \\
                                -u \sin v     & u \cos v      & 1             \\
                            \end{vNiceMatrix} \\
                        & = \begin{bNiceMatrix}[margin]
                                \sin v \\ - \cos v \\ u
                            \end{bNiceMatrix}
          \end{align}

    \item Finding the parameter curves in $ 3d $,
          \begin{align}
              \vec{r}     & = \begin{bNiceMatrix}[margin]
                                  a \cos v \cos u \\ b \cos v \sin u \\ c \sin v
                              \end{bNiceMatrix} \\
              u = \lambda & \implies y = x\ \frac{b\ \tan \lambda}{a}
              \qquad \frac{z^2}{c^2} + \frac{x^2}{a^2 \cos^2 \lambda} = 1   \\
              v = c       & \implies \frac{y}{x} = \tan(c) \qquad z = c
          \end{align}
          The parameter curves for constant $ u $ are ellipses through the $ z $ axis
          lying in the plane $ y = (\tan c)\ bx/a $. \par
          For constant $ v $, the parameter curves are ellipses on the $ xy $
          plane centered at the z axis at $ z = c \sin \lambda $. \par
          The normal vector is,
          \begin{align}
              \vec{r}_u & =   \begin{bNiceMatrix}[margin]
                                  -a \cos v \sin u \\ b \cos v \cos u \\ 0
                              \end{bNiceMatrix}
                        &
              \vec{r}_v & = \begin{bNiceMatrix}[margin]
                                -a\sin v \cos u \\ -b \sin v \sin u \\ c \cos v
                            \end{bNiceMatrix}
              \\
              \vec{N}   & = \vec{r}_u \times \vec{r}_v
                        &
                        &
              = \cos v \begin{vNiceMatrix}[margin]
                           \vec{\hat{i}}    & \vec{\hat{j}}    & \vec{\hat{k}} \\
                           -a \sin u        & b \cos u         & 0             \\
                           -a \sin v \cos u & -b \sin v \sin u & c \cos v      \\
                       \end{vNiceMatrix} \\
                        & = \begin{bNiceMatrix}[margin]
                                bc \cos u \cos^2 v \\
                                ac \sin u \cos^2 v \\
                                ab \sin v \cos v
                            \end{bNiceMatrix}
          \end{align}

    \item Finding the parameter curves in $ 3d $,
          \begin{align}
              \vec{r}     & = \begin{bNiceMatrix}[margin]
                                  au \cosh v \\ bu \sinh v \\ u^2
                              \end{bNiceMatrix}                       \\
              u = \lambda & \implies \frac{x^2}{a^2} - \frac{y^2}{b^2} = \lambda^2
              \qquad z = \lambda^2                                                 \\
              v = c       & \implies y = x\ \frac{b\ \tanh \lambda}{a}
              \qquad z = \Bigg( \frac{x}{a \cosh \lambda} \Bigg)^2
          \end{align}
          The parameter curves for constant $ u $ are hyperbolae on the $ xy $ plane
          at $ z = \lambda^2 $. \par
          For constant $ v $, the parameter curves are parabolas along the z axis
          lying on the plane $ y = \tanh \lambda (bx/a)  $. \par
          The normal vector is,
          \begin{align}
              \vec{r}_u & =   \begin{bNiceMatrix}[margin]
                                  a \cosh v \\ b \sinh v \\ 2u
                              \end{bNiceMatrix}
                        &
              \vec{r}_v & = \begin{bNiceMatrix}[margin]
                                au \sinh v \\ bu \cosh v \\ 0
                            \end{bNiceMatrix}
              \\
              \vec{N}   & = \vec{r}_u \times \vec{r}_v
                        &
                        &
              = \begin{vNiceMatrix}[margin]
                    \vec{\hat{i}} & \vec{\hat{j}} & \vec{\hat{k}} \\
                    a \cosh v     & b \sinh v     & 2u            \\
                    au \sinh v    & bu \cosh v    & 0             \\
                \end{vNiceMatrix} \\
                        & = \begin{bNiceMatrix}[margin]
                                -2bu^2 \cosh v \\
                                2au^2 \sinh v  \\
                                abu
                            \end{bNiceMatrix}
          \end{align}

    \item Plotting the various parametric surfaces,
          \begin{figure}[H]
              \centering
              \begin{subfigure}{0.49 \textwidth}
                  \begin{tikzpicture}
                      \begin{axis}[
                              title = Circular cone,
                              width = 8cm, height = 8cm,
                              grid = both,
                              ticks = none,
                              view={30}{30},
                              xlabel=$x$,ylabel=$y$,zlabel=$z$,
                              enlargelimits=0.1, colormap/viridis,
                              Ani, axis equal]
                          \addplot3 [mesh, shader = interp, y domain = 0:2*pi,
                              domain = -1.5:1.5, samples = 50]
                          ({x * cos(y)},{x * sin(y)},{x});
                      \end{axis}
                  \end{tikzpicture}
              \end{subfigure}
              \hfill\begin{subfigure}{0.49 \textwidth}
                  \begin{tikzpicture}
                      \begin{axis}[
                              title = Elliptic cylinder,
                              width = 8cm, height = 8cm,
                              grid = both,
                              ticks = none,
                              view={30}{30},
                              xlabel=$x$,ylabel=$y$,zlabel=$z$,
                              enlargelimits=0.1, colormap/jet,
                              Ani, axis equal]
                          \addplot3 [mesh, shader = interp, y domain = 0:2*pi,
                              domain = -1.5:1.5, samples = 50]
                          ({2 * cos(y)},{1 * sin(y)},{x});
                      \end{axis}
                  \end{tikzpicture}
              \end{subfigure}
          \end{figure}
          \begin{figure}[H]
              \centering
              \begin{subfigure}{0.49 \textwidth}
                  \begin{tikzpicture}
                      \begin{axis}[
                              title = Praboloid of revolution,
                              width = 8cm, height = 8cm,
                              grid = both,
                              ticks = none,
                              view={30}{30},
                              xlabel=$x$,ylabel=$y$,zlabel=$z$,
                              enlargelimits=0.1, colormap/viridis,
                              Ani, axis equal]
                          \addplot3 [mesh, shader = interp, y domain = 0:2*pi,
                              domain = 0:2, samples = 50]
                          ({x * cos(y)},{x * sin(y)},{x^2});
                      \end{axis}
                  \end{tikzpicture}
              \end{subfigure}
              \hfill\begin{subfigure}{0.49 \textwidth}
                  \begin{tikzpicture}
                      \begin{axis}[
                              title = Helicoid,
                              width = 8cm, height = 8cm,
                              grid = both,
                              ticks = none,
                              view={30}{30},
                              xlabel=$x$,ylabel=$y$,zlabel=$z$,
                              enlargelimits=0.1, colormap/jet,
                              Ani]
                          \addplot3 [mesh, shader = interp, y domain = 0:2*pi,
                              domain = 0:1, samples = 30]
                          ({x * cos(y)},{x * sin(y)},{y});
                      \end{axis}
                  \end{tikzpicture}
              \end{subfigure}
          \end{figure}
          \begin{figure}[H]
              \centering
              \begin{subfigure}{0.49 \textwidth}
                  \begin{tikzpicture}
                      \begin{axis}[
                              title = Ellipsoid,
                              width = 8cm, height = 8cm,
                              grid = both,
                              ticks = none,
                              view={30}{30},
                              xlabel=$x$,ylabel=$y$,zlabel=$z$,
                              enlargelimits=0.1, colormap/viridis,
                              Ani, axis equal]
                          \addplot3 [mesh, shader = interp, y domain = 0:2*pi,
                              domain = -pi:pi, samples = 20]
                          ({1 * cos(x) * cos(y)},{1 * cos(x) * sin(y)},{2 * sin(x)});
                      \end{axis}
                  \end{tikzpicture}
              \end{subfigure}
              \hfill\begin{subfigure}{0.49 \textwidth}
                  \begin{tikzpicture}
                      \begin{axis}[
                              title = Hyperbolic Paraboloid,
                              width = 8cm, height = 8cm,
                              grid = both,
                              ticks = none,
                              view={45}{45},
                              xlabel=$x$,ylabel=$y$,zlabel=$z$,
                              enlargelimits=0.1, colormap/viridis,
                              Ani, axis equal]
                          \addplot3 [mesh, shader = interp, y domain = -4:4,
                              domain = -4:4, samples = 30]
                          ({2 * x},{3 * y},{x^2 - y^2});
                      \end{axis}
                  \end{tikzpicture}
              \end{subfigure}
          \end{figure}
          The effects of varying the parameters $ a, b, c $ are not shown here.

    \item Examples TBC. For the forward case, start with
          \begin{align}
              \vec{r}_u \dotp \vec{r}_v & = 0                                      \\
              u = c_1                   & \implies \vec{r}_1'(t) = \vec{r}_v\ v' &
              v = c_2                   & \implies \vec{r}_2'(t) = \vec{r}_u\ u'
          \end{align}
          If the vectors $ \vec{r}_u $ and $ \vec{r}_v $ are orthogonal, then the
          tangent vectors to the two curves $ \vec{r}_1' $ and $ \vec{r}_2' $ are also
          orthogonal. \par
          This means that the curves with $ u = c_1 $ and $ v = c_2 $ are orthogonal at
          $ P $ on surface $ S $. \par
          The backward proof is exactly these steps in reverse.

    \item The normal vector from Problem $ 5 $ is
          \begin{align}
              \vec{N} & = \begin{bNiceMatrix}[margin]
                              -2u^2\ \cos v \\ -2u^2\ \sin v \\ u
                          \end{bNiceMatrix} &
              \vec{N}(u = 0, v = 0) = \vec{0}
          \end{align}
          To redefine the normal vector and avoid this,
          \begin{align}
              \vec{\tilde{r}}(u, v) & = \begin{bNiceMatrix}[margin]
                                            u \\ v \\ u^2 + v^2
                                        \end{bNiceMatrix}  &
              \vec{\tilde{r}}_u     & = \begin{bNiceMatrix}[margin]
                                            1 \\ 0 \\ 2u
                                        \end{bNiceMatrix}     \\
              \vec{\tilde{r}}_v     & = \begin{bNiceMatrix}[margin]
                                            0 \\ 1 \\ 2v
                                        \end{bNiceMatrix}  &
              \vec{\tilde{N}}       & = \begin{bNiceMatrix}[margin]
                                            -2u \\ -2v \\ 1
                                        \end{bNiceMatrix}     \\
              \vec{\tilde{N}}(0, 0) & = \begin{bNiceMatrix}[margin]
                                            0 \\ 0 \\ 1
                                        \end{bNiceMatrix} \neq \vec{0}
          \end{align}

    \item Finding the points at which the normal vector is zero,
          \begin{enumerate}
              \item No such points exist for the given surface.
                    \begin{align}
                        \vec{N}_1 & = \begin{bNiceMatrix}[margin]
                                          0 \\ 0 \\ 1
                                      \end{bNiceMatrix} &
                        \vec{N}   & \neq \vec{0}
                    \end{align}
              \item This is caused by the \textcolor{y_p}{representation}.
                    \begin{align}
                        \vec{N}_2   & = \begin{bNiceMatrix}[margin]
                                            0 \\ 0 \\ u
                                        \end{bNiceMatrix} &
                        \vec{N} = 0 & \implies (0, 0, 0)
                    \end{align}
              \item This is caused by the \textcolor{y_h}{surface itself}.
                    \begin{align}
                        \vec{N}_3                                     & =
                        \begin{bNiceMatrix}[margin]
                            -\lambda u \cos(v) \\ -\lambda u \sin(v) \\ u
                        \end{bNiceMatrix} &
                        \vec{N} = 0                                   &
                        \implies (0, 0, 0)
                    \end{align}
              \item No such points exist for the given surface.
                    \begin{align}
                        \vec{N}_4                   & =
                        \begin{bNiceMatrix}[margin]
                            -b\cos v \\ -a\sin v \\ 0
                        \end{bNiceMatrix} &
                        \vec{N}                     & \neq 0
                    \end{align}
              \item This is caused by the \textcolor{y_p}{representation}.
                    \begin{align}
                        \vec{N}_5                           & =
                        \begin{bNiceMatrix}[margin]
                            -2u^2\ \cos v \\ -2u^2\ \sin v \\ u
                        \end{bNiceMatrix} &
                        \vec{N} = \vec{0}                   & \implies
                        (0, \alpha, \beta)
                    \end{align}
              \item No such points exist for the given surface.
                    \begin{align}
                        \vec{N}_6                   & =
                        \begin{bNiceMatrix}[margin]
                            \sin v \\ - \cos v \\ u
                        \end{bNiceMatrix} &
                        \vec{N}                     & \neq \vec{0}
                    \end{align}
              \item This is caused by the \textcolor{y_p}{representation}.
                    \begin{align}
                        \vec{N}_7                   & =
                        \begin{bNiceMatrix}[margin]
                            bc \cos u \cos^2 v \\
                            ac \sin u \cos^2 v \\
                            ab \sin v \cos v
                        \end{bNiceMatrix} &
                        \vec{N} = 0                 &
                        \implies v = n\pi + \frac{\pi}{2}
                    \end{align}
              \item This is caused by the \textcolor{y_p}{representation}.
                    \begin{align}
                        \vec{N}_8                   & =
                        \begin{bNiceMatrix}[margin]
                            -2b\ u^2 \cosh v \\
                            2a\ u^2 \sinh v  \\
                            ab\ u
                        \end{bNiceMatrix} &
                        \vec{N} = 0                 &
                        \implies u = 0
                    \end{align}
          \end{enumerate}

    \item Representing the surface as
          \begin{align}
              z        & = f(x, y)                     &
              \vec{r}  & = \begin{bNiceMatrix}[margin]
                               u \\ v \\ f(u, v)
                           \end{bNiceMatrix}    \\
              g        & \equiv z - f(u, v) = 0        &
              \nabla g & = \begin{bNiceMatrix}[margin]
                               -\difcp fu \\ -\difcp fv \\ 1
                           \end{bNiceMatrix}
          \end{align}

    \item Finding the parametric representation,
          \begin{align}
              4x + 3y + 2z & = 12                          &
              \vec{r}      & = \begin{bNiceMatrix}[margin]
                                   u \\ v \\ 6 - 2u - 1.5v
                               \end{bNiceMatrix}    \\
              \vec{r}_u    & = \begin{bNiceMatrix}[margin]
                                   1 \\ 0 \\ -2
                               \end{bNiceMatrix} &
              \vec{r}_v    & = \begin{bNiceMatrix}[margin]
                                   0 \\ 1 \\ -1.5
                               \end{bNiceMatrix}
          \end{align}
          The normal vector is,
          \begin{align}
              \vec{N} =
              \begin{vNiceMatrix}[margin]
                  \vec{\hat{i}} & \vec{\hat{j}} & \vec{\hat{k}} \\
                  1             & 0             & -2            \\
                  0             & 1             & -1.5          \\
              \end{vNiceMatrix} &
              = \begin{bNiceMatrix}[margin]
                    2 \\ 1.5 \\ 1
                \end{bNiceMatrix}
          \end{align}
          The parameter curves are families of parallel straight lines lying on this
          plane with either a fixed $ x $ or $ y $ coordinate.

    \item Finding the parametric representation,
          \begin{align}
              (x-2)^2 + (y + 1)^2 & = 25                              &
              \vec{r}             & = \begin{bNiceMatrix}[margin]
                                          2 + 5\cos u \\ - 1 + 5\sin u \\ v
                                      \end{bNiceMatrix}  \\
              \vec{r}_u           & = \begin{bNiceMatrix}[margin]
                                          -5\sin u \\ 5 \cos u \\ 0
                                      \end{bNiceMatrix}     &
              \vec{r}_v           & = \begin{bNiceMatrix}[margin]
                                          0 \\ 0 \\ 1
                                      \end{bNiceMatrix}
          \end{align}
          The normal vector is,
          \begin{align}
              \vec{N} =
              \begin{vNiceMatrix}[margin]
                  \vec{\hat{i}} & \vec{\hat{j}} & \vec{\hat{k}} \\
                  -5\sin u      & 5 \cos u      & 0             \\
                  0             & 0             & 1             \\
              \end{vNiceMatrix} &
              = \begin{bNiceMatrix}[margin]
                    5\cos u \\ 5\sin u \\ 0
                \end{bNiceMatrix}
          \end{align}
          One set of parameter curves are families straight lines parallel to the $ z $
          axis lying on the perimeter of the cross-section. \par
          The other parameter curves are circles parallel to the $ xy $ plane centered
          on $ (2, -1, z) $ and with radius $ 5 $.

    \item Finding the parametric representation,
          \begin{align}
              x^2 + y^2 + \frac{z^2}{9} & = 1                           &
              \vec{r}                   & = \begin{bNiceMatrix}[margin]
                                                \cos u \cos v \\
                                                \sin u \cos v \\
                                                3 \sin v
                                            \end{bNiceMatrix}    \\
              \vec{r}_u                 & = \begin{bNiceMatrix}[margin]
                                                -5\sin u \\ 5 \cos u \\ 0
                                            \end{bNiceMatrix} &
              \vec{r}_v                 & = \begin{bNiceMatrix}[margin]
                                                0 \\ 0 \\ 1
                                            \end{bNiceMatrix}
          \end{align}
          The normal vector is,
          \begin{align}
              \vec{N} =
              \begin{vNiceMatrix}[margin]
                  \vec{\hat{i}} & \vec{\hat{j}} & \vec{\hat{k}} \\
                  -5\sin u      & 5 \cos u      & 0             \\
                  0             & 0             & 1             \\
              \end{vNiceMatrix} &
              = \begin{bNiceMatrix}[margin]
                    5\cos u \\ 5\sin u \\ 0
                \end{bNiceMatrix}
          \end{align}
          One set of parameter curves are families straight lines parallel to the $ z $
          axis lying on the perimeter of the cross-section. \par
          The other parameter curves are circles parallel to the $ xy $ plane centered
          on $ (2, -1, z) $ and with radius $ 5 $.

    \item Finding the parametric representation,
          \begin{align}
              x^2 + (y + 2.8)^2 + (z - 3.2)^2 & = 1.5^2                       &
              \vec{r}                         & = \begin{bNiceMatrix}[margin]
                                                      1.5\cos u \cos v        \\
                                                      -2.8 + 1.5\sin u \cos v \\
                                                      3.2 +  1.5\sin v
                                                  \end{bNiceMatrix}    \\
              \vec{r}_u                       & = \begin{bNiceMatrix}[margin]
                                                      -1.5\sin u \cos v \\
                                                      1.5\cos u \cos v  \\
                                                      0
                                                  \end{bNiceMatrix} &
              \vec{r}_v                       & = \begin{bNiceMatrix}[margin]
                                                      -1.5\cos u \sin v  \\
                                                      -1.5 \sin u \sin v \\
                                                      1.5 \cos v
                                                  \end{bNiceMatrix}
          \end{align}
          The normal vector is,
          \begin{align}
              \vec{N} = 2.25\
              \begin{vNiceMatrix}[margin]
                  \vec{\hat{i}}  & \vec{\hat{j}}  & \vec{\hat{k}} \\
                  -\sin u \cos v & \cos u \cos v  & 0             \\
                  -\cos u \sin v & -\sin u \sin v & \cos v        \\
              \end{vNiceMatrix} &
              = 2.25\ \begin{bNiceMatrix}[margin]
                          \cos u \cos^2 v \\ \sin u \cos^2 v \\ \cos v \sin v
                      \end{bNiceMatrix}
          \end{align}
          One set of parameter curves are circles parallel to the $ xy $ plane
          (latitudes) at elevation $ z = 3.2 + 1.5 \sin v $. \par
          The other parameter curves are circles meridians similar to a globe.

    \item Finding the parametric representation,
          \begin{align}
              \sqrt{x^2 + 4y^2} & = z^2                         &
              \vec{r}           & = \begin{bNiceMatrix}[margin]
                                        u\ \cos v    \\
                                        0.5u\ \sin v \\
                                        u
                                    \end{bNiceMatrix}    \\
              \vec{r}_u         & = \begin{bNiceMatrix}[margin]
                                        \cos v     \\
                                        0.5 \sin v \\
                                        1
                                    \end{bNiceMatrix} &
              \vec{r}_v         & = \begin{bNiceMatrix}[margin]
                                        -u \sin v   \\
                                        0.5u \cos v \\
                                        0
                                    \end{bNiceMatrix}
          \end{align}
          The normal vector is,
          \begin{align}
              \vec{N} = 2.25\
              \begin{vNiceMatrix}[margin]
                  \vec{\hat{i}} & \vec{\hat{j}} & \vec{\hat{k}} \\
                  \cos v        & 0.5 \sin v    & 1             \\
                  -u \sin v     & 0.5 u \cos v  & 0             \\
              \end{vNiceMatrix} &
              = \begin{bNiceMatrix}[margin]
                    -0.5u\ \cos v \\ u\ \sin v \\ 0.5u
                \end{bNiceMatrix}
          \end{align}
          One set of parameter curves are ellipses parallel to the $ xy $ plane at
          elevation $ z = u $(latitudes). \par
          The other parameter curves are straight lines passing through the origin
          lying on the cone's surface.

    \item Finding the parametric representation,
          \begin{align}
              x^2 - y^2 & = 1                           &
              \vec{r}   & = \begin{bNiceMatrix}[margin]
                                \cosh v \\ \sinh v \\ u
                            \end{bNiceMatrix}    \\
              \vec{r}_u & = \begin{bNiceMatrix}[margin]
                                0 \\ 0 \\ 1
                            \end{bNiceMatrix} &
              \vec{r}_v & = \begin{bNiceMatrix}[margin]
                                \sinh v \\ \cosh v \\ 0
                            \end{bNiceMatrix}
          \end{align}
          The normal vector is,
          \begin{align}
              \vec{N} = \begin{vNiceMatrix}[margin]
                            \vec{\hat{i}} & \vec{\hat{j}} & \vec{\hat{k}} \\
                            0             & 0             & 1             \\
                            \sinh v       & \cosh v       & 0             \\
                        \end{vNiceMatrix} &
              = \begin{bNiceMatrix}[margin]
                    -\cosh v \\ \sinh v \\ 0
                \end{bNiceMatrix}
          \end{align}
          One set of parameter curves are hyperbolae parallel to the $ xy $ plane at
          elevation $ z = u $. \par
          The other parameter curves are straight lines parallel to the $ z $ axis
          passing through $ (\cosh v, \sinh v , 0) $.

    \item Tangent planes. Examples TBC.
          \begin{enumerate}
              \item Using the tangent vectors $ \vec{r}_u $ and $ \vec{r}_v $
                    \begin{align}
                        \vec{r}_u \times \vec{r}_v & = \vec{N}                     &
                        \vec{r}^* \dotp \vec{N}    & = \vec{r}(P) \dotp \vec{N}(P)
                    \end{align}
                    The tangent plane is defined using the surface normal at $ P $ which
                    also happnes to be normal to the plane.

              \item Using the fact that the gradient of a scalar function at a point
                    is the surface normal at that point and the result from part $ a $,
                    \begin{align}
                        g(x, y, z)                 & = 0                              &
                        \vec{N}                    & = \nabla g                         \\
                        \vec{r}^* \dotp (\nabla g) & = \vec{r}(P) \dotp (\nabla g)(P)
                    \end{align}

              \item Using the results from parts $ a $ and $ b $,
                    \begin{align}
                        S                       & : z = f(x, y)                 &
                        \vec{r}                 & = \begin{bNiceMatrix}[margin]
                                                        x \\ y \\ f(x, y)
                                                    \end{bNiceMatrix}    \\
                        \vec{r}_x               & =\begin{bNiceMatrix}[margin]
                                                       1 \\ 0 \\ \difcp fx
                                                   \end{bNiceMatrix}  &
                        \vec{r}_y               & =\begin{bNiceMatrix}[margin]
                                                       0 \\ 1 \\ \difcp fy
                                                   \end{bNiceMatrix}     \\
                        \vec{N}                 & = \vec{r}_x \times \vec{r}_y  &
                        \vec{N}                 & =\begin{bNiceMatrix}[margin]
                                                       -\difcp fx \\ -\difcp fy \\ 1
                                                   \end{bNiceMatrix}   \\
                        \vec{r}^* \dotp \vec{N} & = \vec{r}(P) \dotp \vec{N}(P) &
                    \end{align}
                    This can be rearranged into the required form.
          \end{enumerate}
\end{enumerate}