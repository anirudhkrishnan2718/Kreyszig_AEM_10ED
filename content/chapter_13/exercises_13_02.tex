\section{Polar Form of Complex Numbers, Powers and Roots}

\begin{enumerate}
    \item Plotting on the complex plane,
          \begin{figure}[H]
              \centering
              \begin{tikzpicture}
                  \begin{axis}[
                          grid = none,Ani,xlabel = $ \Re{} $, ylabel = $ \Im{} $,
                          xmin = -0.25, xmax = 1.25, ymin = -0.25, ymax = 1.25,
                          axis equal, width = 8cm, axis lines = center,
                          title = {$ z = 1 + \i $}]
                      \coordinate (O) at (axis cs:0,0);
                      \coordinate (Z) at (axis cs:1,1);
                      \coordinate (X) at (axis cs:1,0);
                      \draw[thick, black] (O) -- (Z)
                      node[color = y_p, midway, above left]{$ \sqrt{2} $};
                      \node[GraphNode, fill = white, draw = black] at (Z){};
                      \pic [draw, thick, color = y_h, -{Triangle}, "$\pi/4$",
                      angle eccentricity=1.5, angle radius = 1cm] {angle = X--O--Z};
                  \end{axis}
              \end{tikzpicture}
          \end{figure}
    \item Plotting on the complex plane,
          \begin{figure}[H]
              \centering
              \begin{tikzpicture}
                  \begin{axis}[
                          grid = none,Ani,xlabel = $ \Re{} $, ylabel = $ \Im{} $,
                          xmin = -5, xmax = 2, ymin = -1, ymax = 5,
                          axis equal, width = 8cm, axis lines = center,
                          title = {$ z = -4 + 4\i $}]
                      \coordinate (O) at (axis cs:0,0);
                      \coordinate (Z) at (axis cs:-4,4);
                      \coordinate (X) at (axis cs:1,0);
                      \draw[thick, black] (O) -- (Z)
                      node[color = y_p, midway, above right]{$ 4\sqrt{2} $};
                      \node[GraphNode, fill = white, draw = black] at (Z){};
                      \pic [draw, thick, color = y_h, -{Triangle}, "$3\pi/4$",
                      angle eccentricity=1.5, angle radius = 1cm] {angle = X--O--Z};
                  \end{axis}
              \end{tikzpicture}
          \end{figure}
    \item Plotting on the complex plane,
          \begin{figure}[H]
              \centering
              \begin{tikzpicture}
                  \begin{axis}[
                          grid = none,Ani,xlabel = $ \Re{} $, ylabel = $ \Im{} $,
                          xmin = -2, xmax = 2, ymin = -1, ymax = 3,
                          axis equal, width = 8cm, axis lines = center,
                          title = {$ z = 2\i $}]
                      \coordinate (O) at (axis cs:0,0);
                      \coordinate (Z) at (axis cs:0,2);
                      \coordinate (X) at (axis cs:1,0);
                      \draw[thick, black] (O) -- (Z)
                      node[color = y_p, midway, right]{$ 2 $};
                      \node[GraphNode, fill = white, draw = black] at (Z){};
                      \pic [draw, thick, color = y_h, -{Triangle}, "$\pi/2$",
                      angle eccentricity=1.5, angle radius = 1cm] {angle = X--O--Z};
                  \end{axis}
              \end{tikzpicture}
              \begin{tikzpicture}
                  \begin{axis}[
                          grid = none,Ani,xlabel = $ \Re{} $, ylabel = $ \Im{} $,
                          xmin = -2, xmax = 2, ymin = -3, ymax = 1,
                          axis equal, width = 8cm, axis lines = center,
                          title = {$ z = -2\i $}]
                      \coordinate (O) at (axis cs:0,0);
                      \coordinate (Z) at (axis cs:0,-2);
                      \coordinate (X) at (axis cs:1,0);
                      \draw[thick, black] (O) -- (Z)
                      node[color = y_p, midway, right]{$ 2 $};
                      \node[GraphNode, fill = white, draw = black] at (Z){};
                      \pic [draw, thick, color = y_h, {Triangle}-, "$-\pi/2$",
                          angle eccentricity=1.5, angle radius = 1cm] {angle = Z--O--X};
                  \end{axis}
              \end{tikzpicture}
          \end{figure}
    \item Plotting on the complex plane,
          \begin{figure}[H]
              \centering
              \begin{tikzpicture}
                  \begin{axis}[
                          grid = none,Ani,xlabel = $ \Re{} $, ylabel = $ \Im{} $,
                          xmin = -6, xmax = 3, ymin = -2, ymax = 2,
                          axis equal, width = 8cm, axis lines = center,
                          title = {$ z = -5 $}]
                      \coordinate (O) at (axis cs:0,0);
                      \coordinate (Z) at (axis cs:-5,0);
                      \coordinate (X) at (axis cs:1,0);
                      \draw[thick, black] (O) -- (Z)
                      node[color = y_p, midway, above]{$ 5 $};
                      \node[GraphNode, fill = white, draw = black] at (Z){};
                      \pic [draw, thick, color = y_h, -{Triangle}, "$\pi$",
                      angle eccentricity=1.5, angle radius = 1cm] {angle = X--O--Z};
                  \end{axis}
              \end{tikzpicture}
          \end{figure}

    \item Simplifying,
          \begin{align}
              z & = \frac{\sqrt{2} + \i/3}{-\sqrt{8} - 2\i/3}
              = \frac{-4 - 2/9}{8 + 4/9} = \frac{-38}{76} = \frac{-1}{2} + 0\ \i
          \end{align}
          \begin{figure}[H]
              \centering
              \begin{tikzpicture}
                  \begin{axis}[
                          grid = none,Ani,xlabel = $ \Re{} $, ylabel = $ \Im{} $,
                          xmin = -1, xmax = 1, ymin = -1, ymax = 1,
                          axis equal, width = 8cm, axis lines = center,
                          title = {$ z = -0.5 $}]
                      \coordinate (O) at (axis cs:0,0);
                      \coordinate (Z) at (axis cs:-0.5,0);
                      \coordinate (X) at (axis cs:1,0);
                      \draw[thick, black] (O) -- (Z)
                      node[color = y_p, above]{$ 0.5 $};
                      \node[GraphNode, fill = white, draw = black] at (Z){};
                      \pic [draw, thick, color = y_h, -{Triangle}, "$\pi$",
                      angle eccentricity=1.5, angle radius = 0.5cm] {angle = X--O--Z};
                  \end{axis}
              \end{tikzpicture}
          \end{figure}

    \item Simplifying,
          \begin{align}
              z & = \frac{\sqrt{3} - 10\i}{-0.5\sqrt{3} + 5\i}
              = \frac{-51.5}{103/4} = -2 + 0\ \i
          \end{align}
          \begin{figure}[H]
              \centering
              \begin{tikzpicture}
                  \begin{axis}[
                          grid = none,Ani,xlabel = $ \Re{} $, ylabel = $ \Im{} $,
                          xmin = -3, xmax = 3, ymin = -1, ymax = 1,
                          axis equal, width = 8cm, axis lines = center,
                          title = {$ z = -2 $}]
                      \coordinate (O) at (axis cs:0,0);
                      \coordinate (Z) at (axis cs:-2,0);
                      \coordinate (X) at (axis cs:1,0);
                      \draw[thick, black] (O) -- (Z)
                      node[color = y_p, midway, above]{$ 2 $};
                      \node[GraphNode, fill = white, draw = black] at (Z){};
                      \pic [draw, thick, color = y_h, -{Triangle}, "$\pi$",
                      angle eccentricity=1.5, angle radius = 0.5cm] {angle = X--O--Z};
                  \end{axis}
              \end{tikzpicture}
          \end{figure}

    \item Plotting on the complex plane, with $ \theta = \arctan(\pi/2) $ and
          $ \abs{z} = \sqrt{1 + \pi^2/4} $
          \begin{figure}[H]
              \centering
              \begin{tikzpicture}
                  \begin{axis}[
                          grid = none,Ani,xlabel = $ \Re{} $, ylabel = $ \Im{} $,
                          xmin = -1, xmax = 2, ymin = -1, ymax = 3,
                          axis equal, width = 8cm, axis lines = center,
                          title = {$ z = 1 + \frac{\pi}{2}\ \i $}]
                      \coordinate (O) at (axis cs:0,0);
                      \coordinate (Z) at (axis cs:1,pi/2);
                      \coordinate (X) at (axis cs:1,0);
                      \draw[thick, black] (O) -- (Z)
                      node[color = y_p, midway, right]{$ \abs{z} $};
                      \node[GraphNode, fill = white, draw = black] at (Z){};
                      \pic [draw, thick, color = y_h, -{Triangle}, "$\theta$",
                      angle eccentricity=1.25, angle radius = 0.8cm] {angle = X--O--Z};
                  \end{axis}
              \end{tikzpicture}
          \end{figure}

    \item Simplifying,
          \begin{align}
              z      & = \frac{-4 + 19\i}{2 + 5\i}
              = \frac{87 + 58 \i}{29} = 3 + 2 \i   \\
              \theta & = \arctan(2/3)
          \end{align}
          \begin{figure}[H]
              \centering
              \begin{tikzpicture}
                  \begin{axis}[
                          grid = none,Ani,xlabel = $ \Re{} $, ylabel = $ \Im{} $,
                          xmin = -0.5, xmax = 2.5, ymin = -0.5, ymax = 3.5,
                          axis equal, width = 8cm, axis lines = center,
                          title = {$ z = 2 + 3\ \i $}]
                      \coordinate (O) at (axis cs:0,0);
                      \coordinate (Z) at (axis cs:2,3);
                      \coordinate (X) at (axis cs:1,0);
                      \draw[thick, black] (O) -- (Z)
                      node[color = y_p, midway, right]{$ \sqrt{13} $};
                      \node[GraphNode, fill = white, draw = black] at (Z){};
                      \pic [draw, thick, color = y_h, -{Triangle}, "$\theta$",
                      angle eccentricity=1.25, angle radius = 0.8cm] {angle = X--O--Z};
                  \end{axis}
              \end{tikzpicture}
          \end{figure}

    \item Plotting on the complex plane,
          \begin{figure}[H]
              \centering
              \begin{tikzpicture}
                  \begin{axis}[
                          grid = none,Ani,xlabel = $ \Re{} $, ylabel = $ \Im{} $,
                          xmin = -1.5, xmax = 1.5, ymin = -0.5, ymax = 1.5,
                          axis equal, width = 8cm, axis lines = center,
                          title = {$ z = -1 + \i $}]
                      \coordinate (O) at (axis cs:0,0);
                      \coordinate (Z) at (axis cs:-1,1);
                      \coordinate (X) at (axis cs:1,0);
                      \draw[thick, black] (O) -- (Z)
                      node[color = y_p, midway, left=6]{$ \sqrt{2} $};
                      \node[GraphNode, fill = white, draw = black] at (Z){};
                      \pic [draw, thick, color = y_h, -{Triangle}, "$3\pi/4$",
                      angle eccentricity=1.5, angle radius = 0.8cm] {angle = X--O--Z};
                  \end{axis}
              \end{tikzpicture}
          \end{figure}

    \item Plotting on the complex plane, with
          \begin{align}
              \Arg(z_1) & = \pi                 &
              \Arg(z_2) & = -\pi + \arctan(1/5)   \\
              \Arg(z_3) & = \pi + \arctan(-1/5)
          \end{align}
          \begin{figure}[H]
              \centering
              \begin{tikzpicture}
                  \begin{axis}[
                          grid = none,Ani,xlabel = $ \Re{} $, ylabel = $ \Im{} $,
                          xmin = -6, xmax = 2, ymin = -1, ymax = 1,
                          axis equal, width = 8cm, axis lines = center,
                          title = {$ z = -5 $}]
                      \coordinate (O) at (axis cs:0,0);
                      \coordinate (Z) at (axis cs:-5,0);
                      \coordinate (X) at (axis cs:1,0);
                      \draw[thick, black] (O) -- (Z)
                      node[color = y_p, midway, above=6]{$ 5 $};
                      \node[GraphNode, fill = white, draw = black] at (Z){};
                      \pic [draw, thick, color = y_h, -{Triangle}, "$\pi$",
                      angle eccentricity=1.5, angle radius = 0.8cm] {angle = X--O--Z};
                  \end{axis}
              \end{tikzpicture}
          \end{figure}
          \begin{figure}[H]
              \centering
              \begin{tikzpicture}
                  \begin{axis}[
                          grid = none,Ani,xlabel = $ \Re{} $, ylabel = $ \Im{} $,
                          xmin = -6, xmax = 2, ymin = -2, ymax = 2,
                          axis equal, width = 8cm, axis lines = center,
                          title = {$ z = -5 - \i $}]
                      \coordinate (O) at (axis cs:0,0);
                      \coordinate (Z) at (axis cs:-5,-1);
                      \coordinate (X) at (axis cs:1,0);
                      \draw[thick, black] (O) -- (Z)
                      node[color = y_p, midway, below]{$ \sqrt{26} $};
                      \node[GraphNode, fill = white, draw = black] at (Z){};
                      \pic [draw, thick, color = y_h, {Triangle}-, "$ \theta_2 $",
                          angle eccentricity=1.5, angle radius = 0.8cm]
                      {angle = Z--O--X};
                  \end{axis}
              \end{tikzpicture}
              \begin{tikzpicture}
                  \begin{axis}[
                          grid = none,Ani,xlabel = $ \Re{} $, ylabel = $ \Im{} $,
                          xmin = -6, xmax = 2, ymin = -2, ymax = 2,
                          axis equal, width = 8cm, axis lines = center,
                          title = {$ z = -5 + \i $}]
                      \coordinate (O) at (axis cs:0,0);
                      \coordinate (Z) at (axis cs:-5,1);
                      \coordinate (X) at (axis cs:1,0);
                      \draw[thick, black] (O) -- (Z)
                      node[color = y_p, midway, above]{$ \sqrt{26} $};
                      \node[GraphNode, fill = white, draw = black] at (Z){};
                      \pic [draw, thick, color = y_h, -{Triangle}, "$ \theta_3 $",
                      angle eccentricity=1.5, angle radius = 0.8cm]
                      {angle = X--O--Z};
                  \end{axis}
              \end{tikzpicture}
          \end{figure}
    \item Plotting on the complex plane, with
          \begin{align}
              \Arg(z_1) & = \arctan(4/3)  &
              \Arg(z_2) & = \arctan(-4/3)
          \end{align}
          \begin{figure}[H]
              \centering
              \begin{tikzpicture}
                  \begin{axis}[
                          grid = none,Ani,xlabel = $ \Re{} $, ylabel = $ \Im{} $,
                          xmin = -1, xmax = 4, ymin = -0.5, ymax = 4.5,
                          axis equal, width = 8cm, axis lines = center,
                          title = {$ z = 3 + 4\i $}]
                      \coordinate (O) at (axis cs:0,0);
                      \coordinate (Z) at (axis cs:3,4);
                      \coordinate (X) at (axis cs:1,0);
                      \draw[thick, black] (O) -- (Z)
                      node[color = y_p, midway, right]{$ 5 $};
                      \node[GraphNode, fill = white, draw = black] at (Z){};
                      \pic [draw, thick, color = y_h, -{Triangle}, "$ \theta_1 $",
                      angle eccentricity=1.5, angle radius = 0.8cm]
                      {angle = X--O--Z};
                  \end{axis}
              \end{tikzpicture}
              \begin{tikzpicture}
                  \begin{axis}[
                          grid = none,Ani,xlabel = $ \Re{} $, ylabel = $ \Im{} $,
                          xmin = -1, xmax = 4, ymin = -4.5, ymax = 0.5,
                          axis equal, width = 8cm, axis lines = center,
                          title = {$ z = 3 - 4\i $}]
                      \coordinate (O) at (axis cs:0,0);
                      \coordinate (Z) at (axis cs:3,-4);
                      \coordinate (X) at (axis cs:1,0);
                      \draw[thick, black] (O) -- (Z)
                      node[color = y_p, midway, right]{$ 5 $};
                      \node[GraphNode, fill = white, draw = black] at (Z){};
                      \pic [draw, thick, color = y_h, {Triangle}-, "$ \theta_2 $",
                          angle eccentricity=1.5, angle radius = 0.8cm]
                      {angle = Z--O--X};
                  \end{axis}
              \end{tikzpicture}
          \end{figure}
    \item Plotting on the complex plane, with
          \begin{figure}[H]
              \centering
              \begin{tikzpicture}
                  \begin{axis}[
                          grid = none,Ani,xlabel = $ \Re{} $, ylabel = $ \Im{} $,
                          xmin = -4, xmax = 2, ymin = -4, ymax = 2,
                          axis equal, width = 8cm, axis lines = center,
                          title = {$ z = -\pi - \pi\i $}]
                      \coordinate (O) at (axis cs:0,0);
                      \coordinate (Z) at (axis cs:-pi,-pi);
                      \coordinate (X) at (axis cs:1,0);
                      \draw[thick, black] (O) -- (Z)
                      node[color = y_p, midway, below right]{$ \pi\sqrt{2} $};
                      \node[GraphNode, fill = white, draw = black] at (Z){};
                      \pic [draw, thick, color = y_h, {Triangle}-, "$ -3\pi/4 $",
                          angle eccentricity=1.5, angle radius = 1cm]
                      {angle = Z--O--X};
                  \end{axis}
              \end{tikzpicture}
          \end{figure}

    \item Simplifying,
          \begin{align}
              w & = (1 + i) & w^4 & = (1 + i)^4 = [(1 + i)^2]^2 = -4 \\
              z & = w^{20}  & z   & = (-4)^5 = -1024 + 0\i
          \end{align}
          \begin{figure}[H]
              \centering
              \begin{tikzpicture}
                  \begin{axis}[
                          grid = none,Ani,xlabel = $ \Re{} $, ylabel = $ \Im{} $,
                          xmin = -1100, xmax = 400, ymin = -650, ymax = 650,
                          axis equal, width = 8cm, axis lines = center,
                          xtick = {-512, -1024}, ytick = {-512, 512},
                          title = {$ z = -1024 $}]
                      \coordinate (O) at (axis cs:0,0);
                      \coordinate (Z) at (axis cs:-1024,0);
                      \coordinate (X) at (axis cs:1,0);
                      \draw[thick, black] (O) -- (Z)
                      node[color = y_p, midway, above=6]{$ 1024 $};
                      \node[GraphNode, fill = white, draw = black] at (Z){};
                      \pic [draw, thick, color = y_h, -{Triangle}, "$\pi$",
                      angle eccentricity=1.5, angle radius = 0.8cm] {angle = X--O--Z};
                  \end{axis}
              \end{tikzpicture}
          \end{figure}
    \item Plotting on the complex plane, with
          \begin{align}
              \Arg(z_1) & = \pi + \arctan(-1/10) &
              \Arg(z_2) & = -\pi + \arctan(1/10)
          \end{align}
          \begin{figure}[H]
              \centering
              \begin{tikzpicture}
                  \begin{axis}[
                          grid = none,Ani,xlabel = $ \Re{} $, ylabel = $ \Im{} $,
                          xmin = -1.1, xmax = 0.5, ymin = -1, ymax = 1,
                          axis equal, width = 8cm, axis lines = center,
                          title = {$ z = -1 + 0.1\i $}]
                      \coordinate (O) at (axis cs:0,0);
                      \coordinate (Z) at (axis cs:-1,0.1);
                      \coordinate (X) at (axis cs:1,0);
                      \draw[thick, black] (O) -- (Z)
                      node[color = y_p, midway, above]{$ \sqrt{1.01} $};
                      \node[GraphNode, fill = white, draw = black] at (Z){};
                      \pic [draw, thick, color = y_h, -{Triangle}, "$ \theta_1 $",
                      angle eccentricity=1.5, angle radius = 0.8cm]
                      {angle = X--O--Z};
                  \end{axis}
              \end{tikzpicture}
              \begin{tikzpicture}
                  \begin{axis}[
                          grid = none,Ani,xlabel = $ \Re{} $, ylabel = $ \Im{} $,
                          xmin = -1.1, xmax = 0.5, ymin = -1, ymax = 1,
                          axis equal, width = 8cm, axis lines = center,
                          title = {$ z = -1 - 0.1\i $}]
                      \coordinate (O) at (axis cs:0,0);
                      \coordinate (Z) at (axis cs:-1,-0.1);
                      \coordinate (X) at (axis cs:1,0);
                      \draw[thick, black] (O) -- (Z)
                      node[color = y_p, midway, below = 10]{$ \sqrt{1.01} $};
                      \node[GraphNode, fill = white, draw = black] at (Z){};
                      \pic [draw, thick, color = y_h, {Triangle}-, "$ \theta_2 $",
                          angle eccentricity=1.5, angle radius = 0.8cm]
                      {angle = Z--O--X};
                  \end{axis}
              \end{tikzpicture}
          \end{figure}

    \item Using the polar representation, with $ r = 3, \theta = \pi/2 $,
          \begin{figure}[H]
              \centering
              \begin{tikzpicture}
                  \begin{axis}[
                          grid = none,Ani,xlabel = $ \Re{} $, ylabel = $ \Im{} $,
                          xmin = -2.5, xmax = 2.5, ymin = -4, ymax = 1,
                          axis equal, width = 8cm, axis lines = center,
                          title = {$ z = 0 - 3\i $}]
                      \coordinate (O) at (axis cs:0,0);
                      \coordinate (Z) at (axis cs:0,-3);
                      \coordinate (X) at (axis cs:1,0);
                      \draw[thick, black] (O) -- (Z)
                      node[color = y_p, midway, right = 2]{$ 3 $};
                      \node[GraphNode, fill = white, draw = black] at (Z){};
                      \pic [draw, thick, color = y_h, {Triangle}-, "$ -\pi/2 $",
                          angle eccentricity=1.5, angle radius = 1cm]
                      {angle = Z--O--X};
                  \end{axis}
              \end{tikzpicture}
          \end{figure}

    \item Using the polar representation, with $ r = 6, \theta = \pi/3 $,
          \begin{figure}[H]
              \centering
              \begin{tikzpicture}
                  \begin{axis}[
                          grid = none,Ani,xlabel = $ \Re{} $, ylabel = $ \Im{} $,
                          xmin = -1, xmax = 4, ymin = -1, ymax = 6,
                          axis equal, width = 8cm, axis lines = center,
                          title = {$ z = 3 + 3\sqrt{3}\i $}]
                      \coordinate (O) at (axis cs:0,0);
                      \coordinate (Z) at (axis cs:3, 5.196);
                      \coordinate (X) at (axis cs:1,0);
                      \draw[thick, black] (O) -- (Z)
                      node[color = y_p, midway, right = 2]{$ 6 $};
                      \node[GraphNode, fill = white, draw = black] at (Z){};
                      \pic [draw, thick, color = y_h, -{Triangle}, "$ \pi/3 $",
                      angle eccentricity=1.5, angle radius = 1cm]
                      {angle = X--O--Z};
                  \end{axis}
              \end{tikzpicture}
          \end{figure}

    \item Using the polar representation, with $ r = \sqrt{8}, \theta = \pi/4 $,
          \begin{figure}[H]
              \centering
              \begin{tikzpicture}
                  \begin{axis}[
                          grid = none,Ani,xlabel = $ \Re{} $, ylabel = $ \Im{} $,
                          xmin = -0.5, xmax = 2.5, ymin = -0.5, ymax = 2.5,
                          axis equal, width = 8cm, axis lines = center,
                          title = {$ z = 2 + 2\i $}]
                      \coordinate (O) at (axis cs:0,0);
                      \coordinate (Z) at (axis cs:2, 2);
                      \coordinate (X) at (axis cs:1,0);
                      \draw[thick, black] (O) -- (Z)
                      node[color = y_p, midway, right = 5]{$ 2\sqrt{2} $};
                      \node[GraphNode, fill = white, draw = black] at (Z){};
                      \pic [draw, thick, color = y_h, -{Triangle}, "$ \pi/4 $",
                      angle eccentricity=1.5, angle radius = 1cm]
                      {angle = X--O--Z};
                  \end{axis}
              \end{tikzpicture}
          \end{figure}

    \item Using the polar representation, with $ r = \sqrt{50}, \theta = 3\pi/4 $,
          \begin{figure}[H]
              \centering
              \begin{tikzpicture}
                  \begin{axis}[
                          grid = none,Ani,xlabel = $ \Re{} $, ylabel = $ \Im{} $,
                          xmin = -30, xmax = 10, ymin = -10, ymax = 30,
                          axis equal, width = 8cm, axis lines = center,
                          title = {$ z = -25 + 25\i $}]
                      \coordinate (O) at (axis cs:0,0);
                      \coordinate (Z) at (axis cs:-25, 25);
                      \coordinate (X) at (axis cs:1,0);
                      \draw[thick, black] (O) -- (Z)
                      node[color = y_p, midway, above right]{$ 25\sqrt{2} $};
                      \node[GraphNode, fill = white, draw = black] at (Z){};
                      \pic [draw, thick, color = y_h, -{Triangle}, "$ 3\pi/4 $",
                      angle eccentricity=1.5, angle radius = 0.8cm]
                      {angle = X--O--Z};
                  \end{axis}
              \end{tikzpicture}
          \end{figure}

    \item Using \texttt{numpy}, to find the zeros of $ -1 $, for
          the illustrative case $ n = 10$ as illustrative examples. \par
          For a general complex number $ z \neq -1 $, such that
          \begin{align}
              w & = z^{1/n}
          \end{align}
          The set of all roots of $ z $ is now $ \{\omega_k \cdot w\} $. Since the roots
          have equal magnitude and are evenly oriented around the full $ 2\pi $ angle,
          this is readily visible in a plot.
          \begin{figure}[H]
              \centering
              \pgfplotstableread[col sep=comma]{./tables/table_13_01_19.csv}
              \anitableseven
              \begin{tikzpicture}
                  \begin{axis}[
                          title = {$ n=10 $}, axis equal,
                          height = 10cm, width = 10cm,
                          grid = both,Ani,]
                      \addplot[GraphSmooth, domain = 0:2*pi, dotted, black]
                      ({cos(x)}, {sin(x)});
                      \addplot[mark options={mark size = 2pt}, only marks,
                          color = y_p] table[x index=0,y index=1, col sep=comma, ]
                          {\anitableseven};
                      \coordinate (O) at (axis cs:0,0);
                      \coordinate (Z) at (axis cs:0.809, 0.5878);
                      \coordinate (X) at (axis cs:1,0);
                      \draw[thin, y_h] (O) -- (X);
                      \draw[thin, y_h] (O) -- (Z);
                      \node[GraphNode, fill = white, draw = black] at (Z){};
                      \pic [draw, thick, color = y_h, -{Triangle}, "$ \pi/5 $",
                      angle eccentricity=1.5, angle radius = 1cm]
                      {angle = X--O--Z};
                  \end{axis}
              \end{tikzpicture}
          \end{figure}

    \item Square roots of complex numbers,
          \begin{enumerate}
              \item To find the $ n $ roots of a complex number when $ n=2 $,
                    \begin{align}
                        w           & = \sqrt{z}                         &
                        z           & = r\ [\cos\theta + \i\ \sin\theta]   \\
                        \abs{w_k}   & = \abs{z}^{1/2}                    &
                                    & = \sqrt{r}                           \\
                        \arg{(w_k)} & = \frac{\theta + 2k\pi}{2}         &
                                    & = \frac{\theta}{2} + k\pi            \\
                        w_1         & = \sqrt{r}\ [\cos(\theta/2)
                        + \i\ \sin(\theta/2)]                              \\
                        w_2         & = \sqrt{r}\ [\cos(\pi + \theta/2)
                        + \i\ \sin(\pi + \theta/2)]
                    \end{align}
                    Using the properties of trigonometric functions, $ w_2 = -w_1 $

              \item Using the half angle formulas,
                    \begin{align}
                        \cos^2(\theta/2)   & = \frac{1 + \cos\theta}{2} &
                        \sin^2(\theta/2)   & = \frac{1 - \cos\theta}{2}   \\
                        r\cos(\theta/2)    & = \sqrt{r} \cdot
                        \sqrt{\frac{r
                        + r\cos\theta}{2}} &
                        r\sin(\theta/2)    & = \sqrt{r} \cdot
                        \sqrt{\frac{r
                        - r\cos\theta}{2}} &
                    \end{align}
                    Now, using the polar representation of $ \sqrt{z} $,
                    \begin{align}
                        w_1 = \sqrt{z} & = \sqrt{r}\ [\cos(\theta/2)
                        + \i\ \sin(\theta/2)]                           \\
                                       & = \sqrt{\frac{\abs{z} + x}{2}}
                        + \i\ (\sgn{y}) \ \sqrt{\frac{\abs{z} - x}{2}}  \\
                        w_2            & = -w_1
                    \end{align}
                    The signum comes from the fact that the sign of $ y $ determines
                    whether the positive or negative root of $ \sin^2\theta $ will be
                    taken in the second step.

              \item Using the new method,
                    \begin{align}
                        z_1        & = -14\i                               &
                        \sqrt{z_1} & = \sqrt{\frac{14 + 0}{2}}
                        + \i\ (-1) \cdot \sqrt{\frac{14 - 0}{2}}             \\
                        \sqrt{z_1} & = \color{y_h} \sqrt{7} - \sqrt{7}\ \i   \\
                        z_2        & = -9 - 40\i                           &
                        \sqrt{z_2} & = \sqrt{\frac{41 - 9}{2}}
                        + \i\ (-1) \cdot \sqrt{\frac{41 + 9}{2}}             \\
                        \sqrt{z_2} & = \color{y_h} 4 - 5\ \i                 \\
                        z_3        & = 1 + \sqrt{48}\i                     &
                        \sqrt{z_3} & = \sqrt{\frac{7 + 1}{2}}
                        + \i\ (1) \cdot \sqrt{\frac{7 - 1}{2}}               \\
                        \sqrt{z_3} & = \color{y_h} 2 + \sqrt{3}\ \i
                    \end{align}
                    Using the old method,
                    \begin{align}
                        z_1        & = -14\i                         &
                        \sqrt{z_1} & = \sqrt{14}\ \Big[ \cos(-\pi/4)
                        + \i\ \sin(-\pi/4) \Big]                       \\
                        \sqrt{z_1} & = \color{y_p} \sqrt{7}
                        - \sqrt{7}\ \i
                    \end{align}
                    For $ z_2 $ whose argument is not readily apparent,
                    \begin{align}
                        z_2                              & = -9 - 40\i              &
                        \tan\theta                       & = 40/9                     \\
                        \frac{40}{9}                     & = \frac{2\tan(\theta/2)}
                        {1 - \tan^2(\theta/2)}           &
                        \tan(\theta/2)                   & = \frac{-5}{4}             \\
                        \sqrt{z_2}                       & = \sqrt{41}
                        \ \Bigg[ \frac{4}{\sqrt{41}}
                        - \i\ \frac{5}{\sqrt{41}} \Bigg] &
                        \sqrt{z_2}                       & = \color{y_p}
                        4 - 5\ \i
                    \end{align}
                    For $ z_3 $ whose argument is not readily apparent,
                    \begin{align}
                        z_3                    & = 1 + \sqrt{48}\i        &
                        \tan\theta             & = \sqrt{48}                \\
                        \sqrt{48}              & = \frac{2\tan(\theta/2)}
                        {1 - \tan^2(\theta/2)} &
                        \tan(\theta/2)         & = \frac{\sqrt{3}}{2}       \\
                        \sqrt{z_3}             & = \sqrt{7}
                        \ \Bigg[ \frac{2}{\sqrt{7}}
                            + \i\ \frac{\sqrt{3}}{\sqrt{7}}
                        \Bigg]                 &
                        \sqrt{z_3}             & = \color{y_p}
                        2 + \sqrt{3}\ \i
                    \end{align}
                    The newer method is clearly better.

              \item TBC. Both methods are coded in \texttt{sympy}.
          \end{enumerate}

    \item Finding the roots of the complex number,
          \begin{align}
              w_1 & = (1 + i)^{1/3}                                \\
              w_1 & = 2^{1/6}\ [\cos(\pi/12) + \i\ \sin(\pi/12)]   \\
              w_2 & = \omega \cdot w_1 = 2^{1/6}\ [\cos(5\pi/12)
              + \i\ \sin(5\pi/12)]                                 \\
              w_3 & = \omega^2 \cdot w_1 = 2^{1/6}\ [\cos(9\pi/12)
              + \i\ \sin(9\pi/12)]
          \end{align}
          \begin{figure}[H]
              \centering
              \pgfplotstableread[col sep=comma]{./tables/table_13_01_19.csv}
              \anitableseven
              \begin{tikzpicture}
                  \begin{axis}[
                          title = {$ n=3 $}, axis equal,
                          height = 8cm, width = 8cm,
                          grid = both,Ani,]
                      \addplot[GraphSmooth, domain = 0:2*pi, dotted, black]
                      ({2^(1/6) * cos(x)}, {2^(1/6) * sin(x)});
                      \addplot[mark options={mark size = 2pt}, only marks,
                          color = y_p] coordinates
                          {(1.0842, 0.29051) (-0.29051, -1.0842) (-0.7937, 0.7937)};
                      \coordinate (O) at (axis cs:0,0);
                      \coordinate (Z) at (axis cs:1.0842, 0.29051);
                      \coordinate (X) at (axis cs:1.122,0);
                      \draw[thin, y_h] (O) -- (X);
                      \draw[thin, y_h] (O) -- (Z);
                      \node[GraphNode, fill = white, draw = black] at (Z){};
                      \pic [draw, thick, color = y_h, -{Triangle}, "$ \pi/12 $",
                      angle eccentricity=1.4, angle radius = 1.5cm]
                      {angle = X--O--Z};
                  \end{axis}
              \end{tikzpicture}
          \end{figure}

    \item Finding the roots of the complex number,
          \begin{align}
              w_1            & = (3 + 4i)^{1/3}                     &
              \tan(\theta)   & = \frac{4}{3}                          \\
              w_1            & = 5^{1/6}\ \Bigg[ \frac{2}{\sqrt{5}}
                  + \i\ \frac{1}{\sqrt{5}}
              \Bigg]         &
              \tan(\theta/2) & = \frac{1}{2}
          \end{align}
          \begin{figure}[H]
              \centering
              \pgfplotstableread[col sep=comma]{./tables/table_13_01_19.csv}
              \anitableseven
              \begin{tikzpicture}
                  \begin{axis}[
                          title = {$ n=3 $}, axis equal,
                          height = 8cm, width = 8cm,
                          grid = both,Ani,]
                      \addplot[GraphSmooth, domain = 0:2*pi, dotted, black]
                      ({5^(1/3) * cos(x)}, {5^(1/3) * sin(x)});
                      \addplot[mark options={mark size = 2pt}, only marks,
                          color = y_p] coordinates
                          {(1.6289, 0.52017) (-0.36398, -1.6708) (-1.265, 1.1506)};
                      \coordinate (O) at (axis cs:0,0);
                      \coordinate (Z) at (axis cs:1.6289, 0.52017);
                      \coordinate (X) at (axis cs:1.71, 0);
                      \draw[thin, y_h] (O) -- (X);
                      \draw[thin, y_h] (O) -- (Z);
                      \node[GraphNode, fill = white, draw = black] at (Z){};
                      \pic [draw, thick, color = y_h, -{Triangle}, "$ \theta/2 $",
                      angle eccentricity=1.4, angle radius = 1.5cm]
                      {angle = X--O--Z};
                  \end{axis}
              \end{tikzpicture}
          \end{figure}

    \item Finding the roots of the complex number,
          \begin{align}
              w_1            & = (216)^{1/3} &
              \tan(\theta)   & = 0             \\
              w_1            & = 6 + 0\ \i   &
              \tan(\theta/2) & = 0
          \end{align}
          \begin{figure}[H]
              \centering
              \pgfplotstableread[col sep=comma]{./tables/table_13_01_19.csv}
              \anitableseven
              \begin{tikzpicture}
                  \begin{axis}[
                          title = {$ n=3 $}, axis equal,
                          height = 8cm, width = 8cm,
                          grid = both,Ani,]
                      \addplot[GraphSmooth, domain = 0:2*pi, dotted, black]
                      ({6 * cos(x)}, {6 * sin(x)});
                      \addplot[mark options={mark size = 2pt}, only marks,
                          color = y_p] coordinates
                          {(6, 0) (-3, -5.1962) (-3, 5.1962)};
                      \coordinate (O) at (axis cs:0,0);
                      \coordinate (Z) at (axis cs:-3, +5.1962);
                      \coordinate (X) at (axis cs:6, 0);
                      \draw[thin, y_h] (O) -- (X);
                      \draw[thin, y_h] (O) -- (Z);
                      \node[GraphNode, fill = white, draw = black] at (Z){};
                      \pic [draw, thick, color = y_h, -{Triangle}, "$ 2\pi/3 $",
                      angle eccentricity=1.5, angle radius = 1cm]
                      {angle = X--O--Z};
                  \end{axis}
              \end{tikzpicture}
          \end{figure}

    \item Finding the roots of the complex number,
          \begin{align}
              w_1          & = (-4)^{1/4}                                &
              \tan(\theta) & = 0                                           \\
              w_1          & = \sqrt{2}\ [\cos(\pi/4) + \i\ \sin(\pi/4)] &
              w_1          & = 1 + \i
          \end{align}
          \begin{figure}[H]
              \centering
              \pgfplotstableread[col sep=comma]{./tables/table_13_01_19.csv}
              \anitableseven
              \begin{tikzpicture}
                  \begin{axis}[
                          title = {$ n=4 $}, axis equal,
                          height = 8cm, width = 8cm,
                          grid = both,Ani,]
                      \addplot[GraphSmooth, domain = 0:2*pi, dotted, black]
                      ({1.414 * cos(x)}, {1.414 * sin(x)});
                      \addplot[mark options={mark size = 2pt}, only marks,
                          color = y_p] coordinates
                          {(1, 1) (-1, 1) (-1, -1) (1, -1)};
                      \coordinate (O) at (axis cs:0,0);
                      \coordinate (Z) at (axis cs:1, 1);
                      \coordinate (X) at (axis cs:1.414, 0);
                      \draw[thin, y_h] (O) -- (X);
                      \draw[thin, y_h] (O) -- (Z);
                      \node[GraphNode, fill = white, draw = black] at (Z){};
                      \pic [draw, thick, color = y_h, -{Triangle}, "$ \pi/4 $",
                      angle eccentricity=1.5, angle radius = 1cm]
                      {angle = X--O--Z};
                  \end{axis}
              \end{tikzpicture}
          \end{figure}

    \item Finding the roots of the complex number,
          \begin{align}
              w_1          & = (\i)^{1/4}                    &
              \tan(\theta) & = \frac{\pi}{2}                   \\
              w_1          & = \cos(\pi/8) + \i\ \sin(\pi/8) &
              w_1          & = 1 + \i
          \end{align}
          \begin{figure}[H]
              \centering
              \pgfplotstableread[col sep=comma]{./tables/table_13_01_19.csv}
              \anitableseven
              \begin{tikzpicture}
                  \begin{axis}[
                          title = {$ n=4 $}, axis equal,
                          height = 8cm, width = 8cm,
                          grid = both,Ani,]
                      \addplot[GraphSmooth, domain = 0:2*pi, dotted, black]
                      ({cos(x)}, {sin(x)});
                      \addplot[mark options={mark size = 2pt}, only marks,
                          color = y_p] coordinates
                          {(-0.92388, -0.38268) (0.92388, 0.38268)
                              (0.38268, -0.92388) (-0.38268, 0.92388)};
                      \coordinate (O) at (axis cs:0,0);
                      \coordinate (Z) at (axis cs:0.92388, 0.38268);
                      \coordinate (X) at (axis cs:1, 0);
                      \draw[thin, y_h] (O) -- (X);
                      \draw[thin, y_h] (O) -- (Z);
                      \node[GraphNode, fill = white, draw = black] at (Z){};
                      \pic [draw, thick, color = y_h, -{Triangle}, "$ \pi/8 $",
                      angle eccentricity=1.5, angle radius = 1cm]
                      {angle = X--O--Z};
                  \end{axis}
              \end{tikzpicture}
          \end{figure}

    \item Finding the roots of the complex number,
          \begin{align}
              w_1          & = (1)^{1/8}                     &
              \tan(\theta) & = 0                               \\
              w_1          & = \cos(\pi/4) + \i\ \sin(\pi/4) &
              w_1          & = \frac{1}{\sqrt{2}}\ [1 + \i]
          \end{align}
          \begin{figure}[H]
              \centering
              \pgfplotstableread[col sep=comma]{./tables/table_13_01_19.csv}
              \anitableseven
              \begin{tikzpicture}
                  \begin{axis}[
                          title = {$ n=8 $}, axis equal,
                          height = 8cm, width = 8cm,
                          grid = both,Ani,]
                      \addplot[GraphSmooth, domain = 0:2*pi, dotted, black]
                      ({cos(x)}, {sin(x)});
                      \addplot[mark options={mark size = 2pt}, only marks,
                          color = y_p] coordinates
                          {(-1, 0) (1, 0) (0, 1) (0, -1)
                              (0.707, 0.707) (-0.707, 0.707)
                              (-0.707, -0.707) (0.707, -0.707)};
                      \coordinate (O) at (axis cs:0,0);
                      \coordinate (Z) at (axis cs:0.707, 0.707);
                      \coordinate (X) at (axis cs:1, 0);
                      \draw[thin, y_h] (O) -- (X);
                      \draw[thin, y_h] (O) -- (Z);
                      \node[GraphNode, fill = white, draw = black] at (Z){};
                      \pic [draw, thick, color = y_h, -{Triangle}, "$ \pi/4 $",
                      angle eccentricity=1.5, angle radius = 1cm]
                      {angle = X--O--Z};
                  \end{axis}
              \end{tikzpicture}
          \end{figure}

    \item Finding the roots of the complex number,
          \begin{align}
              w_1    & = (-1)^{1/5}                    &
              \theta & = \pi                             \\
              w_1    & = \cos(\pi/5) + \i\ \sin(\pi/5)
          \end{align}
          \begin{figure}[H]
              \centering
              \pgfplotstableread[col sep=comma]{./tables/table_13_01_19.csv}
              \anitableseven
              \begin{tikzpicture}
                  \begin{axis}[
                          title = {$ n=5 $}, axis equal,
                          height = 8cm, width = 8cm,
                          grid = both,Ani,]
                      \addplot[GraphSmooth, domain = 0:2*pi, dotted, black]
                      ({cos(x)}, {sin(x)});
                      \addplot[mark options={mark size = 2pt}, only marks,
                          color = y_p] coordinates
                          {(-0.30902, -0.95106) (-0.30902, 0.95106)
                              (0.80902, -0.58799) (0.80902, 0.58799)
                              (-1, 0)};
                      \coordinate (O) at (axis cs:0,0);
                      \coordinate (Z) at (axis cs:0.80902, 0.58799);
                      \coordinate (X) at (axis cs:1, 0);
                      \draw[thin, y_h] (O) -- (X);
                      \draw[thin, y_h] (O) -- (Z);
                      \node[GraphNode, fill = white, draw = black] at (Z){};
                      \pic [draw, thick, color = y_h, -{Triangle}, "$ \pi/5 $",
                      angle eccentricity=1.5, angle radius = 1cm]
                      {angle = X--O--Z};
                  \end{axis}
              \end{tikzpicture}
          \end{figure}

    \item Solving the equation,
          \begin{align}
              0        & = z^2 - (6 - 2\i)\ z + (17 - 6\i) &
              D        & = 32 - 24\i - 68 + 24\i = -36       \\
              \sqrt{D} & = \pm 6\i                         &
              z        & = \frac{6 - 2\i \pm 6\i}{2}         \\
              z        & =3 - 4\i,\ 3 + 2\i
          \end{align}

    \item Solving the equation,
          \begin{align}
              0        & = z^2 + z + (1 - \i)     &
              D        & = 1 - 4 + 4\i = -3 + 4\i   \\
              \sqrt{D} & = 1 + 2\ \i,\ -1 -2\ \i  &
              z        & = \i,\ -1 - \i
          \end{align}

    \item Solving the equation,
          \begin{align}
              0             & = z^4 + 324                           &
              D             & = -1296                                 \\
              \sqrt{D}      & = \pm 36\ \i                          &
              z^2           & = \pm 18\ \i                            \\
              (36\i)^{1/2}  & = 3\ [1 + \i],\ 3\ [-1-\i]            &
              (-36\i)^{1/2} & = 3\ [1 - \i],\ 3\ [-1+\i]              \\
              z^4 + 324     & = (z^2 - 6 + 18) \cdot (z^2 + 6 + 18)
          \end{align}
          The last step uses pairs of roots which are complex conjugates, which is a
          necessary condition for quadratic polynomials with real coefficients.

    \item Solving the equation, with $ w = z^2 $
          \begin{align}
              0          & = w^2 - 6\i\ w + 16                        &
              D          & = -36 - 64 = -100                            \\
              \sqrt{D}   & = \pm 10\ \i                               &
              w          & = 3\i \pm 5\i = -2\i,\ 8\i                   \\
              \sqrt{w_1} & = \sqrt{2}\ [1 - \i],\ \sqrt{2}\ [-1 + \i] &
              \sqrt{w_2} & = 2\ [1 + \i],\ 2\ [-1 - \i]
          \end{align}

    \item Verifying the triangle inequality,
          \begin{align}
              \abs{z_1}       & = \abs{3 + \i} = \sqrt{10}   &
              \abs{z_2}       & = \abs{-2 + 4\i} = \sqrt{20}   \\
              z_1 + z_2       & = 1 + 5\i                    &
              \abs{z_1 + z_2} & = \sqrt{26}                    \\
              \abs{z_1 + z_2} & \leq \abs{z_1} + \abs{z_2}
          \end{align}

    \item Proving the triangle inequality,
          \begin{align}
              \abs{z_1 + z_2}^2 & = (z_1 + z_2)(\bar{z_1} + \bar{z_2})
              = z_1\bar{z_1} + z_2\bar{z_2} + z_1\bar{z_2} + \bar{z_1}z_2 \\
                                & = \abs{z_1}^2 + \abs{z_2}^2 + 2\cdot
              \Re{(z_1\bar{z_2})}                                         \\
                                & \leq \abs{z_1}^2 + \abs{z_2}^2
              + 2 \abs{z_1\ \bar{z_2}}                                    \\
                                & \leq \abs{z_1}^2 + \abs{z_2}^2
              + 2 \abs{z_1}\ \abs{z_2}
          \end{align}
          Taking the positive square root of both sides proves the inequality.
          The real part of a complex number is not larger than its modulus.
          The modulus of a complex number is equal to its conjugate.

    \item Starting with the properties of sine and cosine functions,
          \begin{align}
              \Re{z}           & = x = \abs{z}\ \cos\theta &
              \Im{z}           & = y = \abs{z}\ \sin\theta   \\
              \cos\theta       & \leq 1                    &
              \implies \quad x & \leq \abs{z}                \\
              \sin\theta       & \leq 1                    &
              \implies \quad y & \leq \abs{z}
          \end{align}

    \item The name comes from the fact that the left side is each of the diagonals
          of a parallellogram and the right side is each of its adjacent sides. \par
          Using the fact that $ \abs{w}^2 = w\bar{w} $,
          \begin{align}
              \abs{z_1 + z_2}^2   & = z_1\bar{z_1} + z_2\bar{z_2} + z_1\bar{z_2}
              + \bar{z_1}z_2                                                           \\
                                  & = \abs{z_1}^2 + \abs{z_2}^2 + 2\Re{(z_1\bar{z_2})} \\
              \abs{z_1 - z_2}^2   & = z_1\bar{z_1} + z_2\bar{z_2} - z_1\bar{z_2}
              - \bar{z_1}z_2                                                           \\
                                  & = \abs{z_1}^2 + \abs{z_2}^2 - 2\Re{(z_1\bar{z_2})} \\
              \abs{z_1 + z_2}^2
              + \abs{z_1 - z_2}^2 & = 2\abs{z_1}^2 + 2\abs{z_2}^2
          \end{align}

\end{enumerate}