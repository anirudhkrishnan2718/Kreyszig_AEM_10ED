\section{Derivative, Analytic Function}

\begin{enumerate}
    \item Plotting the region in the complex plane, with center of circle $ z_0 =
              1 - 5\i$ and radius of circle $ \rho = 1.5 $
          \begin{figure}[H]
              \centering
              \begin{tikzpicture}
                  \begin{axis}[
                          title = {Disc}, axis equal,
                          height = 8cm, width = 8cm,
                          ymin = -7, ymax = -3, xmin = -1, xmax = 3,
                          grid = both,Ani,]
                      \path [draw=y_h,fill=y_h, fill opacity = 0.08]
                      (axis cs:1,-5) circle (1.5);
                      \draw[thin, black, dashed] (1, -5) -- (2.5, -5)
                      node[color = y_p, midway, above] {$ \rho$};
                  \end{axis}
              \end{tikzpicture}
          \end{figure}

    \item Plotting the region in the complex plane, with center of circle $ z_0 =
              0$ and radius of circle $ 1 $
          \begin{figure}[H]
              \centering
              \begin{tikzpicture}
                  \begin{axis}[
                          title = {Disc}, axis equal,
                          height = 8cm, width = 8cm,
                          ymin = -1.1, ymax = 1.1, xmin = -1.1, xmax = 1.1,
                          grid = both,Ani,]
                      \path [draw=none,fill=y_h, fill opacity = 0.08]
                      (axis cs:0,0) circle (1);
                      \draw[thin, black, dashed] (0, 0) -- (1, 0)
                      node[color = y_p, midway, above] {$ \rho$};
                      \node[GraphNode, fill = white, draw = black] at (axis cs:0, 0){};
                  \end{axis}
              \end{tikzpicture}
          \end{figure}

    \item Plotting the region in the complex plane, with center of circle $ z_0 =
              4 - 2\i $ and radius of circle in the range $ (\pi, 3\pi) $
          \begin{figure}[H]
              \centering
              \begin{tikzpicture}
                  \begin{axis}[
                          title = {Annulus}, axis equal,
                          height = 8cm, width = 8cm,
                          ymin = -13, ymax = 9, xmin = -7, xmax = 15,
                          grid = both,Ani,
                          xtick = {4-3*pi, 4-pi, 4, 4+pi, 4+3*pi},
                          ytick = {-2-3*pi, -2-pi, -2, -2+pi, -2+3*pi}]
                      \path [draw=none,fill=y_h, fill opacity = 0.08,even odd rule]
                      (axis cs:4,-2) circle (pi) (axis cs:4,-2) circle (3*pi);
                      \draw[thin, black, densely dashed] (4, -2) -- (4+pi, -2)
                      node[color = y_p, midway, above] {$ \rho_1 $};
                      \draw[thin, black, densely dashed] (4, -2) -- (4, -2+3*pi)
                      node[color = y_p, midway, left] {$ \rho_2 $};
                  \end{axis}
              \end{tikzpicture}
          \end{figure}

    \item Plotting the region in the complex plane,
          \begin{figure}[H]
              \centering
              \begin{tikzpicture}
                  \begin{axis}[
                          title = {Horizontal open band}, axis equal,
                          height = 8cm, width = 8cm,
                          ymin = -5, ymax = 5, xmin = -5, xmax = 5,
                          grid = both,Ani,
                          PiStyleY, ytick distance = pi]
                      \path[name path  = U] (axis cs:-5, pi) -- (axis cs:5, pi);
                      \path[name path  = D] (axis cs:-5, -pi) -- (axis cs:5, -pi);
                      \addplot[y_h, opacity = 0.08] fill between[
                              of = U and D];
                  \end{axis}
              \end{tikzpicture}
          \end{figure}

    \item Plotting the region in the complex plane,
          \begin{figure}[H]
              \centering
              \begin{tikzpicture}
                  \begin{axis}[
                          title = {Sector of disc}, axis equal,
                          height = 8cm, width = 8cm,
                          ymin = -5, ymax = 5, xmin = -5, xmax = 5,
                          grid = both,Ani,]
                      \path[name path  = U] (axis cs:0, 0) -- (axis cs:5, 5);
                      \path[name path  = D] (axis cs:0, 0) -- (axis cs:5, -5);
                      \addplot[y_h, opacity = 0.08] fill between[
                              of = U and D];
                  \end{axis}
              \end{tikzpicture}
          \end{figure}

    \item Plotting the region in the complex plane,
          \begin{align}
              \Re{\Big( \frac{1}{z} \Big)} & < 1 &
              \Re{\Big( z \Big)}           & > 1
          \end{align}
          \begin{figure}[H]
              \centering
              \begin{tikzpicture}
                  \begin{axis}[legend pos = outer north east,
                          title = {Vertical band of exclusion}, axis equal,
                          height = 8cm, width = 8cm,
                          ymin = -5, ymax = 5, xmin = -5, xmax = 5,
                          grid = both,Ani,]
                      \path[name path  = A] (axis cs:1, 5) -- (axis cs:5, 5);
                      \path[name path  = B] (axis cs:1, -5) -- (axis cs:5, -5);
                      \addplot[y_h, opacity = 0.08] fill between[
                              of = A and B];
                  \end{axis}
              \end{tikzpicture}
          \end{figure}

    \item Plotting the region in the complex plane,
          \begin{figure}[H]
              \centering
              \begin{tikzpicture}
                  \begin{axis}[legend pos = outer north east,
                          title = {Off center horizontal half-plane}, axis equal,
                          height = 8cm, width = 8cm,
                          ymin = -5, ymax = 5, xmin = -5, xmax = 5,
                          grid = both,Ani,]
                      \path[name path  = A] (axis cs:-1, 5) -- (axis cs:5, 5);
                      \path[name path  = B] (axis cs:-1, -5) -- (axis cs:5, -5);
                      \addplot[y_h, thick] coordinates {(-1, -5)
                              (-1, 5)};
                      \addplot[y_h, opacity = 0.08] fill between[
                              of = A and B];
                  \end{axis}
              \end{tikzpicture}
          \end{figure}

    \item The distance of $ z $ from $ A = -\i $ is not less than its distance
          from $ B = \i $. The perpendicular bisector of the line joining these two
          points happens to be the real axis. \par
          The region is simply $ \Im{(z)} \geq 0 $
          \begin{figure}[H]
              \centering
              \begin{tikzpicture}
                  \begin{axis}[legend pos = outer north east,
                          title = {Vertical half-plane}, axis equal,
                          height = 8cm, width = 8cm,
                          ymin = -2, ymax = 2, xmin = -2, xmax = 2,
                          grid = both,Ani,]
                      \path[name path  = A] (axis cs:-5, 5) -- (axis cs:5, 5);
                      \path[name path  = B] (axis cs:-5, 0) -- (axis cs:5, 0);
                      \addplot[y_h, thick] coordinates {(-5, 0)
                              (5, 0)};
                      \addplot[y_h, opacity = 0.08] fill between[
                              of = A and B];
                      \node[GraphNode, fill = white, draw = black]
                      at (axis cs:0, 1){\tiny \textbf{A}};
                      \node[GraphNode, fill = white, draw = black]
                      at (axis cs:0, -1){\tiny \textbf{B}};
                  \end{axis}
              \end{tikzpicture}
          \end{figure}

    \item Refer notes. Use Problems $ 1-8 $ as examples.

    \item Finding the real and imaginary parts of the function.
          \begin{align}
              f(z) & = u(x, y) + \i\ v(x, y)                         &
              f(z) & = 5z^2 - 12z + (3 + 2\i)                          \\
              u    & = \color{y_h} 5(x^2 - y^2) - 12x + 3            &
              v    & = \color{y_p} 10xy - 12y + 2                      \\
              P    & = 4 - 3\ \i                                     &
              f(P) & = \textcolor{y_h}{10} - \textcolor{y_p}{82}\ \i
          \end{align}

    \item Finding the real and imaginary parts of the function.
          \begin{align}
              f(z) & = u(x, y) + \i\ v(x, y)                       &
              f(z) & = \frac{1}{1-z}                                 \\
              f(z) & = \frac{1}{(1-x) - y\ \i}                     &
              f(z) & = \frac{(1-x) + y\ \i}{(1-x)^2 + y^2}           \\
              u    & = \color{y_h} \frac{1-x}{(1-x)^2 + y^2}       &
              v    & = \color{y_p} \frac{y}{(1-x)^2 + y^2}           \\
              P    & = 1 - \i                                      &
              f(P) & = \textcolor{y_h}{0} - \textcolor{y_p}{1}\ \i
          \end{align}

    \item Finding the real and imaginary parts of the function.
          \begin{align}
              f(z) & = u(x, y) + v(x, y)\ \i                            &
              f(z) & = \frac{z-2}{z+2}                                    \\
              f(z) & = \frac{(x-2) + y\ \i}{(x+2) + y\ \i}              &
              f(z) & = \frac{(x^2 + y^2 - 4) + (4y)\ \i}{(x+2)^2 + y^2}   \\
              u    & = \color{y_h} \frac{x^2 + y^2 - 4}{(x+2)^2 + y^2}  &
              v    & = \color{y_p} \frac{4y}{(x+2)^2 + y^2}               \\
              P    & = 8\i                                              &
              f(P) & = \textcolor{y_h}{15/17} + \textcolor{y_p}
              {8/17}\ \i
          \end{align}

    \item Plotting the graphs separately to avoid clutter.
          \begin{enumerate}
              \item Finding the real and imaginary parts of the function.
                    \begin{align}
                        f(z) & = u(x, y) + v(x, y)\ \i         &
                        f(z) & = z^2 = (x^2 - y^2) + (2xy)\ \i
                    \end{align}
                    \begin{figure}[H]
                        \centering
                        \begin{tikzpicture}
                            \begin{axis}[
                                    title = {$ \Re{z} $},
                                    width = 8cm, height = 8cm,
                                    grid = none,
                                    ticks = none,
                                    view={30}{30},domain = -2:2, y domain = -2:2,
                                    xlabel=$x$,ylabel=$y$,zlabel=$z$,
                                    colormap/jet,
                                    Ani]
                                \addplot3 [surf, opacity= 0.5, samples = 50,
                                    faceted color = black!50] {x^2 - y^2};
                            \end{axis}
                        \end{tikzpicture}
                        \begin{tikzpicture}
                            \begin{axis}[
                                    title = {$ \Im{z} $},
                                    width = 8cm, height = 8cm,
                                    grid = none,
                                    ticks = none,
                                    view={30}{30},domain = -2:2, y domain = -2:2,
                                    xlabel=$x$,ylabel=$y$,zlabel=$z$,
                                    colormap/jet,
                                    Ani]
                                \addplot3 [surf, opacity= 0.5, samples = 50,
                                    faceted color = black!50] {2*x*y};
                            \end{axis}
                        \end{tikzpicture}
                        \begin{tikzpicture}
                            \begin{axis}[
                                    title = {$ \abs{z} $},
                                    width = 8cm, height = 8cm,
                                    grid = none,
                                    ticks = none,
                                    view={30}{30},domain = -2:2, y domain = -2:2,
                                    xlabel=$x$,ylabel=$y$,zlabel=$z$,
                                    colormap/jet,
                                    Ani]
                                \addplot3 [surf, opacity= 0.5, samples = 50,
                                    faceted color = black!50] {(x^2 + y^2)};
                            \end{axis}
                        \end{tikzpicture}
                    \end{figure}
                    \begin{figure}[H]
                        \centering
                        \begin{tikzpicture}
                            \begin{axis}[
                                    enlargelimits = false, title = $ \Re{z} $,
                                    xlabel = $ x $, ylabel = $ y $,
                                    width = 8cm,
                                    Ani,
                                    axis equal,
                                    view     = {0}{90}, % for a view 'from above'
                                    domain = -5:5,
                                    restrict y to domain = -5:5,
                                    colormap/viridis, colorbar horizontal
                                ]
                                \addplot3 [
                                    contour gnuplot={
                                            % number = 10,
                                            levels={-9,-4,0,4,9},
                                            labels=false,
                                        },
                                    samples=100
                                ] {x^2 - y^2};
                            \end{axis}
                        \end{tikzpicture}
                        \begin{tikzpicture}
                            \begin{axis}[
                                    enlargelimits = false, title = $ \Im{z} $,
                                    xlabel = $ x $, ylabel = $ y $,
                                    width = 8cm,
                                    Ani,
                                    axis equal,
                                    view     = {0}{90}, % for a view 'from above'
                                    domain = -5:5,
                                    restrict y to domain = -5:5,
                                    colormap/viridis, colorbar horizontal
                                ]
                                \addplot3 [
                                    contour gnuplot={
                                            % number = 10,
                                            levels={-9,-4,-1, 1,4,9},
                                            labels=false,
                                        },
                                    samples=100
                                ] {2*x*y};
                            \end{axis}
                        \end{tikzpicture}
                        \begin{tikzpicture}
                            \begin{axis}[
                                    enlargelimits = false, title = $ \abs{z} $,
                                    xlabel = $ x $, ylabel = $ y $,
                                    width = 8cm,
                                    Ani,
                                    axis equal,
                                    view     = {0}{90}, % for a view 'from above'
                                    domain = -5:5,
                                    restrict y to domain = -5:5,
                                    colormap/viridis, colorbar horizontal
                                ]
                                \addplot3 [
                                    contour gnuplot={
                                            % number = 10,
                                            levels={1,4,9,16,25},
                                            labels=false,
                                        },
                                    samples=100
                                ] {x^2 + y^2};
                            \end{axis}
                        \end{tikzpicture}
                    \end{figure}

              \item Finding the real and imaginary parts of the function.
                    \begin{align}
                        f(z) & = u(x, y) + v(x, y)\ \i                     &
                        f(z) & = \frac{1}{z} = \frac{x - \i\ y}{x^2 + y^2}
                    \end{align}
                    \begin{figure}[H]
                        \centering
                        \begin{tikzpicture}
                            \begin{axis}[
                                    title = {$ \Re{z} $},
                                    width = 8cm, height = 8cm,
                                    grid = none,
                                    ticks = none,
                                    view={30}{30},domain = -1:1, y domain = -1:1,
                                    xlabel=$x$,ylabel=$y$,zlabel=$z$,
                                    colormap/jet,
                                    Ani]
                                \addplot3 [surf, opacity= 0.5, samples = 50,
                                    faceted color = black!50] {x/(x^2 + y^2)};
                            \end{axis}
                        \end{tikzpicture}
                        \begin{tikzpicture}
                            \begin{axis}[
                                    title = {$ \Im{z} $},
                                    width = 8cm, height = 8cm,
                                    grid = none,
                                    ticks = none,
                                    view={30}{30},domain = -1:1, y domain = -1:1,
                                    xlabel=$x$,ylabel=$y$,zlabel=$z$,
                                    colormap/jet,
                                    Ani]
                                \addplot3 [surf, opacity= 0.5, samples = 50,
                                    faceted color = black!50] {-y/(x^2 + y^2)};
                            \end{axis}
                        \end{tikzpicture}
                        \begin{tikzpicture}
                            \begin{axis}[
                                    title = {$ \abs{z} $},
                                    width = 8cm, height = 8cm,
                                    grid = none,
                                    ticks = none,
                                    view={30}{30},domain = -1:1, y domain = -1:1,
                                    xlabel=$x$,ylabel=$y$,zlabel=$z$,
                                    colormap/jet,
                                    Ani]
                                \addplot3 [surf, opacity= 0.5, samples = 50,
                                    faceted color = black!50] {(x^2 + y^2)^(-1/2)};
                            \end{axis}
                        \end{tikzpicture}
                    \end{figure}
                    \begin{figure}[H]
                        \centering
                        \begin{tikzpicture}
                            \begin{axis}[
                                    enlargelimits = false, title = $ \Re{z} $,
                                    xlabel = $ x $, ylabel = $ y $,
                                    width = 8cm,
                                    Ani,
                                    axis equal,
                                    view     = {0}{90}, % for a view 'from above'
                                    domain = -1:1,
                                    restrict y to domain = -1:1,
                                    colormap/viridis, colorbar horizontal
                                ]
                                \addplot3 [
                                    contour gnuplot={
                                            % number = 10,
                                            levels={-9,-4,-1,1,4,9},
                                            labels=false,
                                        },
                                    samples=100
                                ] {x/ (x^2 + y^2)};
                            \end{axis}
                        \end{tikzpicture}
                        \begin{tikzpicture}
                            \begin{axis}[
                                    enlargelimits = false, title = $ \Im{z} $,
                                    xlabel = $ x $, ylabel = $ y $,
                                    width = 8cm,
                                    Ani,
                                    axis equal,
                                    view     = {0}{90}, % for a view 'from above'
                                    domain = -1:1,
                                    restrict y to domain = -1:1,
                                    colormap/viridis, colorbar horizontal
                                ]
                                \addplot3 [
                                    contour gnuplot={
                                            % number = 10,
                                            levels={-9,-4,-1, 1,4,9},
                                            labels=false,
                                        },
                                    samples=100
                                ] {-y/(x^2 + y^2)};
                            \end{axis}
                        \end{tikzpicture}
                        \begin{tikzpicture}
                            \begin{axis}[
                                    enlargelimits = false, title = $ \abs{z} $,
                                    xlabel = $ x $, ylabel = $ y $,
                                    width = 8cm,
                                    Ani,
                                    axis equal,
                                    view     = {0}{90}, % for a view 'from above'
                                    domain = -5:5,
                                    restrict y to domain = -5:5,
                                    colormap/viridis, colorbar horizontal
                                ]
                                \addplot3 [
                                    contour gnuplot={
                                            % number = 10,
                                            levels={1,2,3,4,5},
                                            labels=false,
                                        },
                                    samples=200
                                ] {(x^2 + y^2)^(-1/2)};
                            \end{axis}
                        \end{tikzpicture}
                    \end{figure}

              \item Finding the real and imaginary parts of the function.
                    \begin{align}
                        f(z) & = u(x, y) + v(x, y)\ \i                              \\
                        f(z) & = z^4 = (x^4 + y^4 - 6x^2y^2) + (4xy)(x^2 - y^2)\ \i
                    \end{align}
                    \begin{figure}[H]
                        \centering
                        \begin{tikzpicture}
                            \begin{axis}[
                                    title = {$ \Re{z} $},
                                    width = 8cm, height = 8cm,
                                    grid = none,
                                    ticks = none,
                                    view={30}{30},domain = -2:2, y domain = -2:2,
                                    xlabel=$x$,ylabel=$y$,zlabel=$z$,
                                    colormap/jet,
                                    Ani]
                                \addplot3 [surf, opacity= 0.5, samples = 50,
                                    faceted color = black!50] {x^4 + y^4 - 6*x^2*y^2};
                            \end{axis}
                        \end{tikzpicture}
                        \begin{tikzpicture}
                            \begin{axis}[
                                    title = {$ \Im{z} $},
                                    width = 8cm, height = 8cm,
                                    grid = none,
                                    ticks = none,
                                    view={30}{30},domain = -2:2, y domain = -2:2,
                                    xlabel=$x$,ylabel=$y$,zlabel=$z$,
                                    colormap/jet,
                                    Ani]
                                \addplot3 [surf, opacity= 0.5, samples = 50,
                                    faceted color = black!50] {4*x*y*(x^2 - y^2)};
                            \end{axis}
                        \end{tikzpicture}
                        \begin{tikzpicture}
                            \begin{axis}[
                                    title = {$ \abs{z} $},
                                    width = 8cm, height = 8cm,
                                    grid = none,
                                    ticks = none,
                                    view={30}{30},domain = -2:2, y domain = -2:2,
                                    xlabel=$x$,ylabel=$y$,zlabel=$z$,
                                    colormap/jet,
                                    Ani]
                                \addplot3 [surf, opacity= 0.5, samples = 50,
                                    faceted color = black!50] {(x^2 + y^2)^2};
                            \end{axis}
                        \end{tikzpicture}
                    \end{figure}
                    \begin{figure}[H]
                        \centering
                        \begin{tikzpicture}
                            \begin{axis}[
                                    enlargelimits = false, title = $ \Re{z} $,
                                    xlabel = $ x $, ylabel = $ y $,
                                    width = 8cm,
                                    Ani,
                                    axis equal,
                                    view     = {0}{90}, % for a view 'from above'
                                    domain = -2:2,
                                    restrict y to domain = -2:2,
                                    colormap/viridis, colorbar horizontal
                                ]
                                \addplot3 [
                                    contour gnuplot={
                                            % number = 10,
                                            levels={-9,-4,0,4,9},
                                            labels=false,
                                        },
                                    samples=100
                                ] {x^4 + y^4 - 6*x^2*y^2};
                            \end{axis}
                        \end{tikzpicture}
                        \begin{tikzpicture}
                            \begin{axis}[
                                    enlargelimits = false, title = $ \Im{z} $,
                                    xlabel = $ x $, ylabel = $ y $,
                                    width = 8cm,
                                    Ani,
                                    axis equal,
                                    view     = {0}{90}, % for a view 'from above'
                                    domain = -2:2,
                                    restrict y to domain = -2:2,
                                    colormap/viridis, colorbar horizontal
                                ]
                                \addplot3 [
                                    contour gnuplot={
                                            % number = 10,
                                            levels={-9,-4,-1, 1,4,9},
                                            labels=false,
                                        },
                                    samples=100
                                ] {4*x*y*(x^2 - y^2)};
                            \end{axis}
                        \end{tikzpicture}
                        \begin{tikzpicture}
                            \begin{axis}[
                                    enlargelimits = false, title = $ \abs{z} $,
                                    xlabel = $ x $, ylabel = $ y $,
                                    width = 8cm,
                                    Ani,
                                    axis equal,
                                    view     = {0}{90}, % for a view 'from above'
                                    domain = -5:5,
                                    restrict y to domain = -5:5,
                                    colormap/viridis, colorbar horizontal
                                ]
                                \addplot3 [
                                    contour gnuplot={
                                            % number = 10,
                                            levels={1,4,9,16,25},
                                            labels=false,
                                        },
                                    samples=100
                                ] {(x^2 + y^2)^2};
                            \end{axis}
                        \end{tikzpicture}
                    \end{figure}
          \end{enumerate}

    \item Testing the continuity at the origin,
          \begin{align}
              f(z)                                 & = \frac{\Re{(z^2)}}{\abs{z}}
              = \frac{x^2 - y^2}{\sqrt{x^2 + y^2}} &
              y                                    & = mx                             \\
              f(z)                                 & = \frac{1 - m^2}{\sqrt{1 + m^2}}
              \ x                                  &
              \lim_{x \to 0}f(z)                   & = 0 \quad \forall \quad
              m\ \in \mathcal{R}
          \end{align}
          Thus, the function is continuous.

    \item Testing the continuity at the origin,
          \begin{align}
              f(z)                       & = \abs{z}^2 \cdot \Im{(1/z)} &
              f(z)                       & = -y                           \\
              \lim_{(x,y) \to (0,0)}f(z) & = 0 \quad \forall \quad
              m\ \in \mathcal{R}
          \end{align}
          Thus, the function is continuous.

    \item Testing the continuity at the origin,
          \begin{align}
              f(z)                       & = \frac{\Im{(z^2)}}{\abs{z}^2} &
              f(z)                       & = \frac{2xy}{x^2 + y^2}          \\
              y                          & = mx                           &
              \implies \quad f(z)        & = \frac{2m}{(1 + m^2)}           \\
              \lim_{(x,y) \to (0,0)}f(z) & = \frac{2m}{1+m^2}
          \end{align}
          Thus, the limit at the origin depends on the direction of approach, making
          $ f(z) $ discontinuous.

    \item Testing the continuity at the origin,
          \begin{align}
              f(z)                       & = \frac{\Re{(z)}}{1 - \abs{z}} &
              f(z)                       & = \frac{x}{1-\sqrt{x^2 + y^2}}   \\
              \lim_{(x,y) \to (0,0)}f(z) & = \frac{\to 0}
              {\to 1} = 0
          \end{align}
          Thus, the function is continuous.

    \item To find the derivative of the complex function,
          \begin{align}
              f(z)   & = \frac{z-\i}{z + \i}                   &
              f'(z)  & = \frac{(z + \i) - (z- \i)}{(z + \i)^2}   \\
              f'(z)  & = \frac{2\i}{(z + \i)^2}                &
              f'(\i) & = \frac{-\i}{2}
          \end{align}

    \item To find the derivative of the complex function,
          \begin{align}
              f(z)        & = (z - 4 \i)^8   &
              f'(z)       & = 8\ (z - 4\i)^7   \\
              f'(3 + 4\i) & = 17496
          \end{align}

    \item To find the derivative of the complex function,
          \begin{align}
              f(z)  & = \frac{1.5z + 2\i}{3\i\ z - 4}                          &
              f'(z) & = \frac{4.5\i\ z - 6 - 4.5\i\ z + 6}{(3\i\ z - 4)^2} = 0
          \end{align}
          This is because $ f(z) $ simplifies to $ (1/2\i) $ and is a constant.

    \item To find the derivative of the complex function,
          \textcolor{y_p}{Typo in question}
          \begin{align}
              f(z)  & = \i\ (1-z)^{-n}    &
              f'(z) & = n\i\ (1-z)^{-n-1}   \\
              P     & = 0                 &
              f'(P) & = n\i
          \end{align}

    \item To find the derivative of the complex function,
          \begin{align}
              f(z)  & = (\i z^3 + 3z^2)^3                             &
              f'(z) & = 3\ (\i z^3 + 3z^2)^2 \cdot (3\i\ z) (z - 2\i)   \\
              P     & = 2\i                                           &
              f'(P) & = 0
          \end{align}

    \item To find the derivative of the complex function,
          \begin{align}
              f(z)  & = \frac{z^3}{(z + \i)^3}      &
              f'(z) & = \frac{3z^2\ \i}{(z + \i)^4}   \\
              P     & = \i                          &
              f'(P) & = \frac{-3}{16}\ \i
          \end{align}

    \item Team Project
          \begin{enumerate}
              \item Using the linearity of the limit operator,
                    \begin{align}
                        f(z)                      & = \Re f(z)
                        + \i\ \Im f(z)            &
                        \lim_{z \to z_0} f(z)     & = l = \Re (l)
                        + \i\ \Im (l)                               \\
                        \lim_{z \to z_0} \Re f(z) & = \Re (l)     &
                        \lim_{z \to z_0} \Im f(z) & = \Im (l)       \\
                    \end{align}

              \item Suppose the limit is not unique. Let the two possible limits be
                    $ A, B $ at $ z = z_0 $
                    \begin{align}
                        \abs{z - z_0}           & < \delta_1   &
                        \implies \abs{f(z) - A} & < \epsilon/2   \\
                        \abs{z - z_0}           & < \delta_2   &
                        \implies \abs{f(z) - B} & < \epsilon/2
                    \end{align}
                    for some fixed value of $ \epsilon $. \par Let $ \delta $ be the
                    smaller among $ \delta_1 > 0,\ \delta_2 > 0 $.
                    \begin{align}
                        \abs{A - B} & = \abs{A - f(z) + f(z) - B}          \\
                                    & \leq \abs{A - f(z)} + \abs{f(z) - B} \\
                                    & \leq \epsilon
                    \end{align}
                    the distance between the two limits $ \abs{A - B} $ is less than
                    $ \epsilon $ for any choice of positive real epsilon, however small.
                    \par This means that $ \abs{A - B} = 0 $ and the limit is unique, if
                    it exists.

              \item Given a infinite series of complex numbers $ \{z_i\} $ such that
                    \begin{align}
                        \lim_{n \to \infty} z_n & = a    &
                        \lim_{z \to a} f(z)     & = f(a)
                    \end{align}
                    Consider some sufficiently large $ n $ such that,
                    \begin{align}
                        \abs{z_n - a}                      & < \delta            &
                                                           & \forall \quad n > N   \\
                        \implies \quad \abs{f(z_n) - f(a)} & < \epsilon          &
                        \implies \lim_{n \to
                        \infty}f(z_n)                      & = f(a)
                    \end{align}

              \item If $ f(z) $ is differentiable at $ z = z_0 $ then this limit
                    exists.
                    \begin{align}
                        f'(z_0)                              &
                        = \lim_{z \to z_0} \frac{f(z) - f(z_0)}{z - z_0} \\
                        \lim_{z \to z_0} [f(z) - f(z_0)]     &
                        = \lim_{z \to z_0} (z - z_0) \cdot f'(z_0) = 0   \\
                        \implies \quad \lim_{z \to z_0} f(z) & = f(z_0)
                    \end{align}
                    This uses the linearity of the limit operator. The last expression
                    implies that $ f(z) $ is continuous at $ z_0 $.

              \item Checking the differentiability,
                    \begin{align}
                        \lim_{z \to z_0} \frac{x - x_0}{z - z_0} & = L
                    \end{align}
                    Consider two approaches, one vertical $ z_1 = x_0 + \i\ y $, and
                    the other horizontal $ z_2 = x + \i\ y_0 $
                    \begin{align}
                        L_1 & = \lim_{z_1 \to z_0} \frac{x - x_0}{z - z_0}
                        = \lim_{y \to y_0} \frac{x_0 - x_0}{y - y_0} =
                        \color{y_h} 0                                      \\
                        L_2 & = \lim_{z_2 \to z_0} \frac{x - x_0}{z - z_0}
                        = \lim_{x \to x_0} \frac{x - x_0}{x - x_0} =
                        \color{y_p} 1
                    \end{align}
                    Since the two limits at the same point $ z_0 $ are different, the
                    function is not differentiable at $ z_0 $, regardless of where in the
                    complex plane $ z_0 $ is located. \par
                    Another such function is $ g(z) = \Im{(z)}$

              \item Checking the differentiability,
                    \begin{align}
                        \lim_{\Delta z \to 0} \frac{\abs{z_0 + \Delta z}^2
                        - \abs{z_0}^2}{\Delta z} & =
                        \lim_{\Delta z \to 0} \frac{\Delta x^2 + \Delta y^2
                            + 2x_0 \Delta x + 2y_0 \Delta y}{\Delta x + \i\ \Delta y}
                    \end{align}
                    Consider two approaches, one vertical $ z_1 = 0 + \i\ \Delta y $,
                    and the other horizontal $ z_2 = \Delta x + 0\ \i $
                    \begin{align}
                        L_1 & = \lim_{\Delta y \to 0} \frac{\Delta y
                            (\Delta y + 2y_0)}{\i\ \Delta y} =
                        \color{y_h} \frac{2y_0}{\i}                            \\
                        L_2 & = \lim_{\Delta x \to 0} \frac{\Delta x (\Delta x
                            + 2x_0)}{\Delta x} =
                        \color{y_p} 2x_0
                    \end{align}
                    Since the two limits at the same point $ z_0 $ are different, the
                    function is not differentiable at $ z_0 $, regardless of where in the
                    complex plane $ z_0 $ is located except the origin. \par
                    Even when $ x_0 = y_0 = 0 $, the function is not differentiable
                    everywhere in the neighbourhood of $ z_0 $. Hence, the function is
                    nowhere analytic.
          \end{enumerate}

    \item Refer notes. TBC.

\end{enumerate}