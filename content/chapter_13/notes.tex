\chapter{Complex Numbers and Functions, Complex Differentiation}

\section{Complex Numbers and Their Geometric Representation}

\begin{description}
    \item[Complex number] An ordered pair of real numbers geometrically represented
        as the two coordinates in a complex Cartesian plane.
        \begin{align}
            z & = x + y\i & \Re(z) = x \qquad \Im(z) = y
        \end{align}
        The real and imaginary parts of $ z $ are $ x $ and $ y $ respectively. \par
        Two complex numbers are equal if and only if both their real and imaginary
        parts are equal.

    \item[Imaginary unit] The unit imaginary part of a complex number is defined as
        the imaginary unit $ \i $. Algebraically, it is $ \sqrt{-1} $.
        \begin{align}
            \i & \equiv \sqrt{-1} &  & \text{Albegra}  \\
            \i & = (0, 1)         &  & \text{Geometry}
        \end{align}

    \item[Complex arithmetic] Complex arithmetic simply adds the extra rule that
        $ \i^2 = -1 $. Otherwise, it behaves the same arithmetic acting independently
        on the real and imaginary parts of $ z $.
        \begin{align}
            z_1 + z_2       & = (x_1 + x_2) + (y_1 + y_2) \i                       \\
            z_1 - z_2       & = (x_1 - x_2) + (y_1 - y_2) \i                       \\
            z_1 \cdot z_2   & = (x_1x_2 - y_1y_2) + (x_1y_2 + x_2y_1) \i           \\
            \frac{z_1}{z_2} & = \frac{(x_1 + y_1 \i)(x_2 - y_2 \i)}{x_2^2 + y_2^2}
        \end{align}

    \item[Complex plane] A Cartesian $ xy $ coordinate system with the $ x $ and $ y $
        axes being called the real and imaginary axes respectively. \par
        The real and imaginary parts of $ z $ now become the coordinates of a point in
        this $ 2d $ Cartesian plane.

    \item[Complex conjugate] The complex number mirrored on the real plane, defined as
        \begin{align}
            \bar{z}  & \equiv x - y\i            &
            z\bar{z} & = x^2 + y^2                 \\
            \Re(z)   & = \frac{z + \bar{z}}{2}   &
            \Im(z)   & = \frac{z - \bar{z}}{2\i}
        \end{align}
        The complex conjugate is distributive over all 4 basic arithmetic operations.
\end{description}

\section{Derivative, Analytic Function}

\begin{description}
    \item[Polar form]  Using the polar coordinates in $ 2d $,
        \begin{align}
            z & = r \cos\theta + \i\ r \sin\theta
        \end{align}
        Here, $ r $ is called the modulus and $ \theta $, the argument of the complex
        number. \par
        Geometrically, $ (r, \theta) $ is simply the polar representation of the point
        $ (x, y) $ in Ccartesian coordinates.
        \begin{align}
            r & \equiv \abs{z} = \sqrt{z\bar{z}}
        \end{align}

    \item[Argument] The slope of the line joining $ z $ to the origin. This is
        conventionally single valued by restricting it to the range
        \begin{align}
            \Arg(z) & \equiv \theta   & \arg(z)     & = \Arg(z) + 2n\pi \\
            \theta  & \in (-\pi, \pi] & \tan \theta & = \frac{y}{x}
        \end{align}
        $ \Arg(z) $ is the principal value of the argument. Other values are separated by
        multiples of $ 2\pi $. \par
        Positive and negative real numbers have $ \Arg(z) = 0, \pi $ respectively.

    \item[Inequalities] The triangle inequality applied to complex numbers by way of
        their moduli is,
        \begin{align}
            \abs{z_1 + z_2} \leq \abs{z_1} + \abs{z_2}
        \end{align}

    \item[Multiplication] Multiplying two complex numbers using sine and cosine addition
        formulas,
        \begin{align}
            z_1 \cdot z_2       & \equiv r_1r_2\ \Big[\cos(\theta_1 + \theta_2)
            + \i\ \sin(\theta_1 + \theta_2) \Big]                               \\
            \abs{z_1 \cdot z_2} & = \abs{z_1} \cdot \abs{z_2}                   \\
            \arg(z_1 \cdot z_2) & = \arg(z_1) + \arg(z_2)
        \end{align}

    \item[Division] Dividing two complex numbers using sine and cosine subtraction
        formulas,
        \begin{align}
            z_1 \cdot z_2                   & \equiv \frac{r_1}{r_2}
            \ \Big[\cos(\theta_1 - \theta_2) + \i\ \sin(\theta_1 - \theta_2) \Big] \\
            \abs{\frac{z_1}{z_2}}           & = \frac{\abs{z_1}}{\abs{z_2}}        \\
            \arg\Big( \frac{z_1}{z_2} \Big) & = \arg(z_1) - \arg(z_2)
        \end{align}

    \item[Integer powers] Using Euler's formula for complex numbers,
        \begin{align}
            z^n & = r^n\ \Big[\cos(n\theta) + \i\ \sin(n\theta)\Big]
        \end{align}
        Taking $ r = 1 $ enables the derivation of $ cos(n\theta) $ and $ \sin(n\theta) $
        formulas by comparing the real and imaginary parts of the equation.

    \item[De Moivre's formula] A formula restating Euler's formula for complex numbers,
        \begin{align}
            z                  & = \cos \theta + \i\ \sin\theta      &
            \implies \quad z^n & = \cos(n\theta) + \i\ \sin(n\theta)
        \end{align}

    \item[Roots] Raising a complex number to an integer power results in a unique answer.
        \begin{align}
            z & = w^n     &  & z\ \text{is unique}                                \\
            w & = z^{1/n} &  & w\ \text{has}\ n\ \text{different possible values}
        \end{align}
        To find the $ n^{\text{th}} $ root of $ z $, DeMoivre's formula gives,
        \begin{align}
            z                   & = r\ \Big[ \cos\theta + \i\ \sin\theta \Big] &
            z^{1/n} = w         & = R\ \Big[ \cos\phi + \i\ \sin\phi \Big]       \\
            \implies \quad R    & = r^{1/n}                                    &
            \implies \quad \phi & = \frac{\theta}{n} + \frac{2k\pi}{n}
        \end{align}
        There are $ n $ distinct values of $ \phi $ for $ k \in \{0, 1, 2, \dots, (n-1)
            \} $. These are all complex numbers with the same modulus evenly oriented
        about the full $ 2\pi $ range. \par
        The set $ z^{1/n} $ also constitute the vertices of a regular polygon of $ n $
        sides, (with one vertex on the positive real line). \par
    \item[Roots of -1] For the special case of $ z = -1 $,
        \begin{align}
            (-1)^{1/n} & = \cos \Big( \frac{2k\pi}{n} \Big)
            + \i\ \sin\Big( \frac{2k\pi}{n} \Big)                       \\
            (-1)^{1/n} & = \{1, \omega, \omega^2, \dots, \omega^{n-1}\}
        \end{align}
        Here, $ \omega $ is the first complex root of $ -1 $, and all the others in the
        set happen to be higher powers of $ \omega $ since $ r = 1 $. \par
        This set lies on the unit circle and operates on other complex numbres to
        increase their argument by $ 2k\pi/n $.
\end{description}

\section{Derivative, Analytic Function}

\begin{description}
    \item[Circle] A circle in the complex plane is defined as the set of all points whose
        distance from the center $ (z_0) $ is fixed.
        \begin{align}
            C & :\ \abs{z - z_0} = \rho &
            A :\ \rho_1 \leq \abs{z - z_0} \leq \rho_2
        \end{align}
        The interior and exterior of this circle involve the appropriate inequalities
        instead. \par
        An annulus (ring) involves a lower limit as well as an upper limit on the
        distance from the center.

    \item[Neighbourhood] Any open circular disk around $ z_0 $ is a neighbourhood of
        $ z_0 $. This neighbourhood includes $ z_0 $ by definition.

    \item[Half-plane] A restriction on the sign of either the real or complex part of
        $ z $.

    \item[Open set] Every point in $ S $ has a neighbourhood consisting entirely of
        points that belong to $ S $
    \item[Connected set] Any two points of $ S $ can be joined by finitely many line
        segments all of whose points belong to $ S $.
    \item[Domain] An open and connected set.
    \item[Complement of a set] The set of all points in the complex plane that do not
        belong to $ S $
    \item[Closed set] A set whose complement is open.
    \item[Boundary point] A point whose every neighbourhood consists of points that
        belong to $ S $ as well as points that dont. \par
        If a set is closed then all of its boundary points belong to that set. The
        converse is true if the set is open.

    \item[Complex function] A map from a set of complex numbers $ S $ to another set of
        complex numbers $ W $ such that every $ z $ in $ S $ has a unique member of
        $ W $ to map onto.
        \begin{align}
            f(z) & = w                                       \\
            z    & = x + \i\ y & w & = u(x, y) + \i\ v(x, y)
        \end{align}
        Here, $ z $ is called a complex variable, $ S $ is the domain of definition, and
        the set of all values of $ f $ is called its range.

    \item[Limit] If a complex function $ f $ is defined in a neighbourhood of $ z_0 $
        (except possibly at $ z_0 $) and if the values of $ f(z) $ are close to some finite
        complex number $ l $ for all $ z $ close to $ z_0 $, then,
        \begin{align}
            \abs{f(z) - l}                       & < \epsilon &
            \forall \quad \abs{z - z_0} < \delta                \\
            \implies \quad \lim_{z \to z_0} f(z) & \equiv l
        \end{align}
        Geometrically, for every $ z \neq z_0 $ in this disk of radius $ \delta $,
        $ f(z) $ lies in the disk of radius $ \epsilon $ centered on $ l $. \par
        Unlike a real function, $ z $ can approach $ z_0 $ from any direction.

    \item[Continuous function] A function $ f $ is continuous at $ z_0 $ if
        \begin{align}
            \lim_{z \to z_0} f(z) & = f(z_0)
        \end{align}
        This requires that $ f $ is defined in some neighbourhood of $ z_0 $. \par
        A function is continuous in a domain if it is continuous at each point in the
        domain.

    \item[Derivative] The function $ f $ is differentiable at $ z_0 $ if the limit
        \begin{align}
            f'(z_0) & \equiv \lim_{\Delta z \to 0}
            \frac{f(z + \Delta z - f(z))}
            {\Delta z}                                             \\
            f'(z_0) & \equiv \lim_{z \to z_0} \frac{f(z) - f(z_0)}
            {z - z_0}                                              \\
        \end{align}
        exists and is finite. \par
        Since this limit can be approached from any direction, all of those limits have
        to exist and be equal to the derivative. \par
        The rules of real differentiation are carried over as is for complex functions.

    \item[Analytic function] A function is analytic in some domain $ D $ if it is defined
        and differentiable at all points in $ D $. \par
        As a special case, the function is analytic at some point $ z_0 $ in the
        domain if it is analytic in a neighbourhood of $ z_0 $. \par
        Alternative terminology is \emph{holomorphic}
\end{description}

\section{Cauchy-Riemann Equations, Laplace's Equation}

\begin{description}
    \item[Cauchy-Riemann equations] A test to check whether a function is analytic.
        \begin{align}
            f(z)      & = u(x, y) + \i\ v(x, y) &
                      & \text{is analytic}        \\
            \diffp ux & = \diffp vy             &
            \diffp uy & = -\diffp vx
        \end{align}
        The fact that $ f(z) $ is defined and continuous in some neighbourhood of
        $ z $ and differentiable at $ z $ itself $ \iff $
        The first order partial derivatives of $ u(x, y) $ and $ v(x, y) $ exist and
        satisfy Cauchy-Riemann equations.

    \item[Corollary relation] If two real-valued continuous functions $ u(x, y) $
        and $ v(x, y) $ exist and have continuous first partial derivatives satisfying
        the Cauchy-Riemann equations, then \par
        the complex function $ f(z) = u + \i\ v $ is analytic in $ D $

    \item[Laplace's equation] Let $ f(z) = u(x, y) + \i\ v(x, y) $ be analytic in some
        domain $ D $, then
        \begin{align}
            \nabla^2 u & = u_{xx} + u_{yy} = 0 &
            \nabla^2 v & = v_{xx} + v_{yy} = 0
        \end{align}
        hold in $ D $ and these second partial derivatives are continuous. \par
        The real and imaginary parts of an analytic function are harmonic functions.
\end{description}

\section{Exponential Function}

\begin{description}
    \item[Definition] The complex exponential function is defined as,
        \begin{align}
            z & = x + \i\ y & e^z & = e^x\ [\cos y + \i \sin y]
        \end{align}

    \item[Entire function] A function that is analytic in the entire complex plane.
        $ e^z $ is an example.

    \item[Euler formula] Using the polar form of a complex number,
        \begin{align}
            z           & = re^{\i \theta} = r\cos\theta + \i\ r\sin\theta \\
            e^{\i 2\pi} & = 1
        \end{align}
        Comparing this to the definition of an exponential function,
        \begin{align}
            \abs{e^z}   & = e^x                        &
            \Arg{(e^z)} & = y                            \\
            e^x         & \neq 0 \quad \forall \quad z
        \end{align}
        This is an entire function that has no zeros in the complex plane.

    \item[Fundamental region] The horizontal strip $ y \in (-\pi, \pi] $
        corresponds to $ y $ being the principal argument. The range of $ e^z $ is
        constrained to be in the region $(e^{-\pi}, e^{\pi}]$.
\end{description}

\section{Trigonometric and Hyperbolic Functions, Euler's Formula}

\begin{description}
    \item[Complex trigonometric functions] Using Euler's formula, and eliminating
        \begin{align}
            \cos(z) & = \frac{e^{\i z} + e^{-\i z}}{2}   &
            \sin(z) & = \frac{e^{\i z} - e^{-\i z}}{2\i} &
        \end{align}
        This uses the definition of $ e^z $ using $ z_1 = x $ and $ z_2 = \i y $. Then,
        the two linear equations yield the complex trigonometric functions.

    \item[Properties of complex trig functions] All properties are similar to real
        trigonometric functions. \par
        The sine and cosine modulus are no longer bounded in complex.
        \begin{align}
            \abs{\cos(z)}^2 & = \cos^2 x + \sinh^2 y &
            \abs{\sin(z)}^2 & = \sin^2 x + \sinh^2 y
        \end{align}

    \item[Hyperbolic functions] These are defined just as in the real case,
        \begin{align}
            \cosh(z) & = \frac{e^z + e^{-z}}{2} &
            \sinh(z) & = \frac{e^z - e^{-z}}{2}
        \end{align}
        These two functions happen to be intimately related in the complex plane, since,
        \begin{align}
            \cosh(\i z) & = \cos z  &
            \sinh(\i z) & = \sin z    \\
            \cos(\i z)  & = \cosh z &
            \sin(\i z)  & = \sinh z
        \end{align}
\end{description}

\section{Logarithm, General Power, Principal Value}

\begin{description}
    \item[Complex logarithm] Just like the real case, this is defined as the inverse of
        the exponential function.
        \begin{align}
            \ln z   & = \ln r + \i\ \theta &
            \abs{z} & > 0
        \end{align}
        This is a multi-valued function since the argument is only determined upto a
        multiple of $ 2\pi $. \par
        Each of these functions is called a \emph{branch} of the logarithm.

    \item[Principal value of logarithm] The value of $ ln z $ corresponding to the
        principal value of the argument. \par
        This is also called the \emph{principal branch} of the logarithm.

    \item[Properties of logarithm] Since the logarithm is multivalued for complex
        numbers, the following identities,
        \begin{align}
            \ln(z_1 \cdot z_2) & = \ln z_1 + \ln z_2 &
            \ln(z_1 / z_2)     & = \ln z_1 - \ln z_2
        \end{align}
        are an equality of infinite sets, instead of an equality relating two quantities.

    \item[Analyticity] For all integers $ n $, the logarithm is analytic,
        \begin{align}
            f(z)  & = \ln z = \Ln z + 2n\pi\ \i \\
            f'(z) & = \frac{1}{z}
        \end{align}
        except at the origin and the negative real axis. This is called a
        \emph{branch cut} of the logarithm.

    \item[General powers] The power function is defined using the logarithm as,
        \begin{align}
            z^c & = \exp(c \cdot \ln z)                   &
            z   & \neq 0, \quad c \quad \text{is complex}
        \end{align}
        Once again, this function is multivalued with the principal value of $ z^c $
        corresponds to the principal branch $ \Ln z $. \par
        \begin{itemize}
            \item For $ c $ being a nonzero integer, the power function is single
                  valued. \par
            \item For $ c $ being a quotient of positive integers, the function is
                  finitely multivalued. \par
            \item Else, the function is infinitely multivalued.
        \end{itemize}
\end{description}