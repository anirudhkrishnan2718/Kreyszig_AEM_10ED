\section{Bessel Functions Yv(x), General Solution}
\begin{enumerate}
    \item Since $ \nu \in \mathcal{I} $,
          \begin{align}
              x^2y'' + xy' + (x^2 - 4^2)y & = 0                  \\
              \nu                         & = 4                  \\
              y_1                         & = \color{y_h} J_4(x) \\
              y_2                         & = \color{y_p} Y_4(x)
          \end{align}

    \item Substituting $ u = yx^2 $,
          \begin{align}
              0   & = xy'' + 5y' + xy                                            &
              y   & = \frac{u}{x^2}                                                \\
              y'  & = \frac{u'}{x^2} - \frac{2u}{x^3}                            &
              y'' & = \frac{u''}{x^2} - \frac{4u'}{x^3} + \frac{6u}{x^4}           \\
              0   & = u'' \left[ \frac{1}{x} \right] + u'\left[ \frac{-4+5}{x^2}
              \right] + u\left[ \frac{6-10 + x^2}{x^3} \right]                     \\
              0   & = x^2 u'' + xu' + u\ [x^2 - 2^2]                               \\
              u_1 & = J_2(x)                                                     &
              u_2 & = Y_2(x)                                                       \\
              y_1 & = \color{y_h} x^{-2}\ J_2(x)                                 &
              y_2 & = \color{y_p} x^{-2}\ Y_2(x)
          \end{align}
          Since $ \nu \in \mathcal{I} $, the Neumann function is necessary.

    \item Substituting $ z = x^2 $,
          \begin{align}
              0   & = 9x^2y'' + 9xy' + (36x^4 - 16)y                             &
              z   & = x^2                                                          \\
              y'  & = 2x \dot{y} = 2\sqrt{z}\dot{y}                              &
              y'' & = 4z\ddot{y} + 2\dot{y}                                        \\
              0   & = \ddot{y}\ [36z^2] + \dot{y}\ [18z + 18z] + y\ [36z^2 - 16]   \\
              0   & = \ddot{y}\ [z^2] + \dot{y}\ [z] + y\ \left[ z^2
              - \frac{2^2}{3^2} \right]                                            \\
              y_1 & = J_{2/3}(z)                                                 &
              y_2 & = J_{-2/3}(z)                                                  \\
              y_1 & = \color{y_h} J_{2/3}(x^2)                                   &
              y_2 & = \color{y_p} J_{-2/3}(x^2)
          \end{align}

    \item Substituting $ y = u\sqrt{x},\ z = (2/3)x^{3/2} $,
          \begin{align}
              0   & = y'' + xy                                                       \\
              y   & = u \sqrt{x} \qquad\qquad z = (2/3)x^{3/2}                       \\
              y'  & = (1.5z)^{1/3} \left[ (1.5z)^{1/3} \dot{u}
              + \frac{(1.5z)^{-2/3}}{2} u \right]                                    \\
                  & = (1.5z)^{2/3}\ \dot{u} + \frac{(1.5z)^{-1/3}}{2}\ u             \\
              y'' & =(1.5z)^{1/3} \left[ (1.5z)^{2/3}\ \ddot{u}
              + (1.5)(1.5z)^{-1/3}\ \dot{u} - \frac{(1.5z)^{-4/3}}{4}\ u\right]      \\
                  & = 1.5z\ \ddot{u} + 1.5\ \dot{u} - (0.25)(1.5z)^{-1}\ u           \\
              0   & = z\ddot{u} + \dot{u} + u \left[z - \frac{1}{9z} \right]         \\
                  & = z^2 \ddot{u} + z\dot{u} + u \left[ z^2 - \frac{1}{3^2} \right] \\
              u_1 & = J_{1/3}(z) \qquad\qquad u_2 = J_{-1/3}(z)                      \\
              y_1 & ={\color{y_h} \sqrt{x}\  J_{1/3}
              \left( \frac{2x^{3/2}}{3} \right)} \qquad\qquad
              y_2 = {\color{y_p} \sqrt{x}\ J_{-1/3}
              \left( \frac{2x^{3/2}}{3} \right)}
          \end{align}

    \item Substituting $ \sqrt{x} = z $,
          \begin{align}
              0           & = 4xy'' + 4y' + y                                     &
              z           & = \sqrt{x}                                              \\
              y'          & = \frac{\dot{y}}{2\sqrt{x}} = \frac{\dot{y}}{2z}      &
              y''         & = \frac{1}{2z} \left[ \frac{\ddot{y}}{2z}
              - \frac{\dot{y}}{2z^2} \right]                                        \\
              0           & = \ddot{y} + \dot{y} \left[ \frac{2}{z} - \frac{1}{z}
              \right] + y &
              0           & = z^2 \ddot{y} + z\dot{y} + z^2y                        \\
              y_1         & = J_0(z)                                              &
              y_2         & = Y_0(z)                                                \\
              y_1         & = \color{y_h}J_0(\sqrt{x})                            &
              y_2         & = \color{y_p}Y_0(\sqrt{x})
          \end{align}
          Since $ \nu \in \mathcal{I} $, the Neumann function is necessary.

    \item Substituting $ 12\sqrt{x} = z $,
          \begin{align}
              0   & = xy'' + y' + 36y                                 &
              z   & = 12\sqrt{x}                                        \\
              y'  & = \frac{6\dot{y}}{\sqrt{x}} = \frac{72\dot{y}}{z} &
              y'' & = \frac{72^2}{z} \left[ \frac{\ddot{y}}{z}
              - \frac{\dot{y}}{z^2} \right]                             \\
              0   & = 36 \ddot{y} + \frac{36\dot{y}}{z} + 36y         &
              0   & = z^2 \ddot{y} + z\dot{y} + (z^2 - 0)y              \\
              y_1 & = J_0(z)                                          &
              y_2 & = Y_0(z)                                            \\
              y_1 & = \color{y_h}J_0(12\sqrt{x})                      &
              y_2 & = \color{y_p}Y_0(12\sqrt{x})
          \end{align}
          Since $ \nu \in \mathcal{I} $, the Neumann function is necessary.

    \item Substituting $ y = u\sqrt{x},\ z = kx^2/2 $,
          \begin{align}
              0   & = y'' + k^2x^2 y                                        \\
              y   & = u \sqrt{x} \qquad\qquad z = \frac{kx^2}{2}            \\
              y'  & = \sqrt{2kz} \left[ (2z/k)^{1/4}\ \dot{u}
              + \frac{(2z/k)^{-3/4}}{2k}\ u \right]                         \\
                  & = (8kz^3)^{1/4}\ \dot{u} + (32z/k)^{-1/4}\ u            \\
              y'' & = \sqrt{2kz} \left[ (8kz^3)^{1/4}\ \ddot{u}
                  + \frac{3(8k)^{1/4}}{4z^{1/4}}\ \dot{u}
                  + \frac{(8k)^{1/4}}{4z^{1/4}}\ \dot{u}
              - \frac{k^{1/4}}{4z^{5/4}\ 32^{1/4}}\ u \right]               \\
                  & = (32k^3z^5)^{1/4}\ \ddot{u} + \dot{u} [(32k^3z)^{1/4}]
              - 0.25u\ (k/2z)^{3/4}                                         \\
              0   & = \ddot{u}\ [(32k^3z^5)^{1/4}]
              + \dot{u}\ [(32k^3z)^{1/4}]
              + u \left[(2kz)(2z/k)^{1/4} - 0.25(k/2z)^{3/4} \right]        \\
                  & = z^{5/4} \ddot{u} + z^{1/4}\dot{u}
              + u \left[ z^{5/4} - \frac{z^{-3/4}}{16} \right]              \\
                  & = z^{2} \ddot{u} + z\dot{u}
              + u \left[ z^{2} - \frac{1}{16} \right]                       \\
              u_1 & = J_{1/4}(z) \qquad\qquad u_2 = J_{-1/4}(z)             \\
              y_1 & ={\color{y_h} \sqrt{x}\  J_{1/4}
              \left( \frac{kx^{2}}{2} \right)} \qquad\qquad
              y_2 = {\color{y_p} \sqrt{x}\ J_{-1/4}
              \left( \frac{kx^{2}}{2} \right)}
          \end{align}

    \item Substituting $ y = u\sqrt{x},\ z = kx^3/3 $,
          \begin{align}
              0   & = y'' + k^2x^4 y                                 \\
              y   & = u \sqrt{x} \qquad\qquad z = \frac{kx^3}{3}     \\
              y'  & = u'x^{1/2} + \frac{ux^{-1/2}}{2}                \\
              y'' & = u'' x^{1/2} + u'x^{-1/2} - \frac{ux^{-3/2}}{4} \\
              0   & = u'' x^{1/2} + u'x^{-1/2}
              + u\left[ k^2x^{9/2} - 0.25x^{-3/2} \right]            \\
                  & = u''x + u' + \frac{u}{x} [k^2x^6 - 0.25]
          \end{align}
          Replacing $ x $ with $ z $,
          \begin{align}
              u'  & = \dot{u}\ kx^2 = (9kz^2)^{1/3}\ \dot{u}                        \\
              u'' & = (9kz^2)^{1/3} \left[ \ddot{u}\ (9kz^2)^{1/3}
              + \dot{u}\ (9k)^{1/3} \frac{2}{3z^{1/3}} \right]                      \\
                  & = (9kz^2)^{2/3}\ \ddot{u} + (2/3)\dot{u}\ (9k)^{2/3}z^{1/3}     \\
              0   & = \ddot{u} \left[ (9z^2) \right]
              + \dot{u} \left[ 6z + 3z \right] + u\ [9z^2 - 1/4]                    \\
              0   & =z^2 \ddot{u} + z\dot{u} + u \left[ z^2 - \frac{1}{6^2} \right] \\
              u_1 & = J_{1/6}(z) \qquad\qquad u_2 = J_{-1/6}(z)                     \\
              y_1 & ={\color{y_h} \sqrt{x}\  J_{1/6}
              \left( \frac{kx^{3}}{3} \right)} \qquad\qquad
              y_2 = {\color{y_p} \sqrt{x}\ J_{-1/6}
              \left( \frac{kx^{3}}{3} \right)}
          \end{align}

    \item Substituting $y = ux^3 $,
          \begin{align}
              0   & = xy'' - 5y' + xy                             &
              y   & = ux^3                                          \\
              y'  & = x^3\ u'+ 3x^2\ u                            &
              y'' & = x^3\ u'' + 6x^2\ u' + 6x\ u                   \\
              0   & = u''\ x^4 + u'\ x^3 + u\ [6x^2 -15x^2 + x^4]   \\
              0   & = x^2 u'' + xu' + u\ [x^2 - 3^2]                \\
              u_1 & = J_3(x)                                      &
              u_2 & = Y_3(x)                                        \\
              y_1 & = \color{y_h} x^{3 }\ J_3(x)                  &
              y_2 & = \color{y_p} x^{3 }\ Y_3(x)
          \end{align}
          Since $ \nu \in \mathcal{I} $, the Neumann function is necessary.

    \item
          \begin{enumerate}
              \item Graphing on common axes, and using the asymptotic approximation for
                    $ J_n(x) $,
                    \begin{figure}[H]
                        \centering
                        \begin{tikzpicture}
                            \begin{axis}[
                                    legend pos = north east,
                                    grid = both,
                                    width = 12cm,
                                    height = 12cm,
                                    Ani,
                                    colormap/viridis,
                                    cycle list = {[samples of colormap = 6]},
                                ]
                                \foreach [evaluate=\c as \n using (\c)*100/(5)]
                                \c in {0,...,5}
                                    {
                                        \edef\temp{%
                                            \noexpand
                                            \addplot+[thick,
                                                samples = 200,
                                                domain=1000:1010,
                                            ] gnuplot[id=besyn] {besyn(\c, x)};
                                            \noexpand \addlegendentry{$ n = \c$};
                                        }\temp
                                    }
                            \end{axis}
                        \end{tikzpicture}
                    \end{figure}
                    After the transients have decayed, the $ Y_n(x) $ practically
                    resembles $ k\sin(x) $ for even $ n $ and $ k\cos(x) $ for odd $ n $.
                    \par
                    Similar to the relation for $ J_n $, extrema of $ Y_1 $ correspond to
                    zero crossings of $ Y_0 $.

              \item Checking the difference function between the $ Y_n $ and its
                    approximation, the difference goes to zero for around
                    $ x_n = 200\pi $. A second plot is shown at large $ x $ to see the
                    small magnitude of error in the approximation.
                    \begin{figure}[H]
                        \centering
                        \begin{tikzpicture}
                            \begin{axis}[
                                    legend pos = outer north east,
                                    grid = both,
                                    width = 12cm,
                                    height = 8cm,
                                    ylabel = Error,
                                    Ani,
                                    restrict y to domain = -2:1,
                                    domain = 0:20*pi,
                                    PiStyleX,
                                    xtick distance = 4*pi,
                                    colormap/viridis,
                                    cycle list = {[samples of colormap = 11]},
                                ]
                                \foreach [evaluate=\c as \n using (\c)*100/(10)]
                                \c in {0,...,10}
                                    {
                                        \edef\temp{%
                                            \noexpand \addplot+[thick,
                                                samples = 200,
                                            ] gnuplot[id = besyn] {besyn(\c, x)-
                                                    ((sin(x - \c*0.5*pi
                                                    - 0.25*pi))*sqrt(2/(x*pi)))};
                                            \noexpand \addlegendentry{$ Y_{\c}(x)$};
                                        }\temp
                                    }
                            \end{axis}
                        \end{tikzpicture}
                    \end{figure}
                    \begin{figure}[H]
                        \centering
                        \begin{tikzpicture}
                            \begin{axis}[
                                    legend pos = outer north east,
                                    grid = both,
                                    width = 12cm,
                                    height = 8cm,
                                    ylabel = Error,
                                    Ani,
                                    % restrict y to domain = -2:1,
                                    domain = 190*pi:200*pi,
                                    PiStyleX,
                                    xtick distance = 4*pi,
                                    colormap/viridis,
                                    cycle list = {[samples of colormap = 11]},
                                ]
                                \foreach [evaluate=\c as \n using (\c)*100/(10)]
                                \c in {0,...,10}
                                    {
                                        \edef\temp{%
                                            \noexpand \addplot+[thick,
                                                samples = 200,
                                            ] gnuplot[id = besyn] {besyn(\c, x)-
                                                    ((sin(x - \c*0.5*pi
                                                    - 0.25*pi))*sqrt(2/(x*pi)))};
                                            \noexpand \addlegendentry{$ Y_{\c}(x)$};
                                        }\temp
                                    }
                            \end{axis}
                        \end{tikzpicture}
                    \end{figure}
                    $ x_n $ increases with increasing $ n $, since the decay is slower.

              \item Using the approximation formula, it gets better as $ x $ increases.
                    This is seen as the reduction in error for successively higher zeros
                    \begin{align}
                        Y_n(x) & = \sqrt{\frac{2}{\pi x}}\ \sin\left( x
                        - \frac{n\pi}{2} - \frac{\pi}{4} \right)
                    \end{align}
                    \begin{figure}[H]
                        \centering
                        \SetTblrInner{rowsep=0.5em}
                        \begin{tblr}{colspec={Q[r]|Q[r]|Q[r]}, colsep = 1em}
                            \SetCell[c=3]{c} $ Y_0 $ &          &        \\ \hline[dotted]
                            Approx.                  & Accurate & Error  \\ \hline[dotted]
                            0.7854                   & 0.8936   & 0.1082 \\
                            3.9270                   & 3.9577   & 0.0307 \\
                            7.0686                   & 7.0861   & 0.0175 \\
                            10.2102                  & 10.2223  & 0.0122 \\
                            13.3518                  & 13.3611  & 0.0093 \\
                            16.4934                  & 16.5009  & 0.0076 \\
                            19.6350                  & 19.6413  & 0.0064 \\
                            22.7765                  & 22.7820  & 0.0055 \\
                            25.9181                  & 25.9230  & 0.0048 \\
                            29.0597                  & 29.0640  & 0.0043 \\ \hline
                        \end{tblr}
                    \end{figure}

              \item Repeating the above procedure for $ Y_1 $ and $ Y_2 $, the
                    approximation still works better for later zeros, but as the order
                    increases, the approximation clearly gets worse for the same zero
                    crossings.
                    \begin{figure}[H]
                        \centering
                        \SetTblrInner{rowsep=0.5em}
                        \begin{tblr}{colspec={Q[r]|Q[r]|Q[r]}, colsep = 1em}
                            \SetCell[c=3]{c} $ Y_1 $ &          &         \\\hline[dotted]
                            Approx.                  & Accurate & Error   \\\hline[dotted]
                            2.3562                   & 2.1971   & -0.1591 \\
                            5.4978                   & 5.4297   & -0.0681 \\
                            8.6394                   & 8.5960   & -0.0434 \\
                            11.7810                  & 11.7492  & -0.0318 \\
                            14.9226                  & 14.8974  & -0.0251 \\
                            18.0642                  & 18.0434  & -0.0208 \\
                            21.2058                  & 21.1881  & -0.0177 \\
                            24.3473                  & 24.3319  & -0.0154 \\
                            27.4889                  & 27.4753  & -0.0136 \\
                            30.6305                  & 30.6183  & -0.0122 \\ \hline
                        \end{tblr}
                        \hspace{0.5in}
                        \begin{tblr}{colspec={Q[r]|Q[r]|Q[r]}, colsep = 1em}
                            \SetCell[c=3]{c} $ J_1 $ &          &         \\\hline[dotted]
                            Approx.                  & Accurate & Error   \\\hline[dotted]
                            3.9270                   & 3.3842   & -0.5427 \\
                            7.0686                   & 6.7938   & -0.2748 \\
                            10.2102                  & 10.0235  & -0.1867 \\
                            13.3518                  & 13.2100  & -0.1418 \\
                            16.4934                  & 16.3790  & -0.1144 \\
                            19.6350                  & 19.5390  & -0.0959 \\
                            22.7765                  & 22.6940  & -0.0826 \\
                            25.9181                  & 25.8456  & -0.0725 \\
                            29.0597                  & 28.9951  & -0.0647 \\
                            32.2013                  & 32.1430  & -0.0583 \\ \hline
                        \end{tblr}
                    \end{figure}
          \end{enumerate}

    \item Suppose the two Hankel functions are L.D., then there exists some $ k $
          (constant) for which,
          \begin{align}
              H_\nu^{(1)} & = k H_\nu^{(2)} \\
              J_\nu       & = k J_\nu       \\
              kY_\nu      & = -kY_\nu
          \end{align}
          No such $ k $ exists which means that the two solutions are L.I. \par
          Since $ J_\nu $ and $ Y_\nu $ are themselves solutions of the Bessel ODE,
          Hankel functions are also solutions to the Bessel ODE by being linear
          superpositions of Bessel and Neumann functions.

    \item Modified Bessel function,
          \begin{align}
              I_\nu(x) & = i^{-\nu}J_\nu (ix)                                 \\
              0        & = x^2y'' + xy' - (x^2 + \nu^2)y                      \\
              0        & = x^2[-i^{-\nu}J_\nu''(ix)] + x[i^{1-\nu}J_\nu'(ix)]
              - i^{-\nu}(x^2 + \nu^2)J_\nu(ix)                                \\
              z        & = ix                                                 \\
              0        & = z^2J_\nu''(z) + zJ_\nu'(z) + (z^2 - \nu^2)J_\nu(z)
          \end{align}
          The fact that the last equality holds by the definition of $ J_\nu $, means
          $ I_\nu(x) $ satisfies the given ODE.

    \item Using the power series definition fo $ J_\nu $,
          \begin{align}
              I_\nu(x) & = i^{-\nu}J_\nu(ix)                                  \\
                       & = i^{-\nu}(ix)^\nu \iser{0} \frac{(-1)^m\ (ix)^{2m}}
              {2^{2m + \nu}\ m!\ \Gamma(\nu + m + 1)}                         \\
                       & = x^{\nu} \iser{0} \frac{x^{2m}}
              {2^{2m + \nu}\ m!\ \Gamma(\nu + m + 1)}
          \end{align}

    \item From the power series definition of $ I_\nu $ in Problem 14, all terms are
          nonzero for $ x \neq 0 $, which means the function is nonzero and monotonically
          increasing in $ x $, for $ x \in \mathcal{R} $ and is real. \par
          Real $ \nu $ TBC. \par
          To prove the relation, for some integer $ n $,
          \begin{align}
              I_{-n}(x) & = i^{n} J_{-n}(ix)             \\
                        & = i^{n} (-1)^n J_n(ix)         \\
                        & = (-i)^n \frac{I_n(x)}{i^{-n}} \\
                        & = (-i^2)^n I_n(x) = I_n(x)
          \end{align}

    \item Modified Bessel functions of the third kind,
          \begin{align}
              K_\nu(x)    & = \frac{\pi}{2\sin(\nu \pi)} [I_{-\nu}(x) - I_\nu(x)] \\
              I_{-\nu}(x) & = i^{\nu} J_{-\nu}(ix)                                \\
          \end{align}
          Since $ I_\nu $ already satisfies the ODE from Problem 12, checking
          $ I_{-\nu} $,
          \begin{align}
              0 & = -x^2 [i^{\nu} J_{-\nu}''(ix)] + ix [i^{\nu}J_{-\nu}'(ix)] -
              (x^2 + \nu^2) i^{\nu} J_{-\nu}(ix)                                 \\
              z & = ix                                                           \\
              0 & = z^2 J_{-\nu}''(z) + z J_{\nu}'(z) + (z^2 - \nu^2)J_{-\nu}(z) \\
          \end{align}
          Since $ J_{-\nu} $ satisfies the original Bessel ODE, $I_{-\nu}(x)$
          satisfies the given ODE. Since the given expression is a linear combination of
          $ I_{\nu} $ and $ I_{-\nu} $, it also solves the ODE
\end{enumerate}