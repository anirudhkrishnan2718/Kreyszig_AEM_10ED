\section{Extended Power Series Method: Frobenius Method}
\begin{enumerate}
    \item TBC. Refer notes and chapter end exercises from C2 and C5.

    \item Finding indicial equation, using the coefficients of the lowest power, after
          $ x+2 \to x $,
          \begin{align}
              y'' + \frac{y'}{(x+2)} - \frac{y}{(x+2)^2} & = 0   \\
              x^2y'' + xy' - y                           & = 0   \\
              r(r - 1) + r - 1                           & = 0 &
              r_1 = -1,\ r_2                             & = 1   \\
          \end{align}
          Finding the first solution,
          \begin{align}
              y_1               & = \iser{1} a_{m-1}\ x^{m}        &
              xy_1'             & = \iser{1} a_{m-1}\ m x^{m}        \\
              x^2 y_1''         & = \iser{2} a_{m-1}\ m(m-1) x^{m}   \\
              a_{m-1} (m^2 - 1) & = 0                                \\
              y_1               & = \color{y_h} x
          \end{align}
          Finding the second solution using reduction of order,
          \begin{align}
              y_2    & = gy_1                               &
              y_2'   & = g + xg'                              \\
              y_2''  & = 2g' + xg''                           \\
              0      & = x^2 (2g' + xg'') + x(g + xg') - gx   \\
              0      & = g''(x^3) + g'(3x^2)                &
              h      & = g'                                   \\
              h'     & = -h \frac{3}{x}                     &
              \ln(h) & = -3\ln(x)                             \\
              g'     & = x^{-3}                             &
              g      & = \frac{1}{2x^2}                       \\
              y_2    & = gy_1 = \color{y_p} \frac{1}{2x}
          \end{align}
          After reversing the change of variables, the general solution is
          \begin{align}
              y & = c_1 y_1 + c_2 y_2             \\
                & = c_1 (x+2) + \frac{c_2}{(x+2)}
          \end{align}

    \item Finding indicial equation, using the coefficients of the lowest power,
          \begin{align}
              x^2 y'' + (2)x y' + (x^2) y & = 0    \\
              r(r - 1) + 2r + 0           & = 0  &
              r_1 = 0,\ r_2               & = -1
          \end{align}
          Finding the first solution,
          \begin{align}
              y_1            & = \iser{0} a_{m}\ x^{m}          &
              xy_1'          & = \iser{1} a_{m}\ m x^{m}          \\
              x^2 y_1''      & = \iser{2} a_{m}\ m(m-1) x^{m}   &
              a_m            & = -a_{m-2}\ \frac{1}{m(m+1)}       \\
              y_1            & = a_0\left[ 1 - \frac{x^2}{3!}
                  + \frac{x^4}{5!} - \frac{x^6}{7!}
              + \dots\right] &
              y_1            & = \color{y_h}  \frac{\sin(x)}{x}
          \end{align}
          Finding the second solution using reduction of order,
          \begin{align}
              y_2   & = gy_1                                                          \\
              y_2'  & = g\ \frac{x \cos(x) - \sin(x)}{x^2} + g'\ \frac{\sin(x)}{x}    \\
              y_2'' & = g''\ \frac{\sin(x)}{x} + 2g'\ \frac{x \cos(x) - \sin(x)}{x^2}
              + g\ \frac{(2 - x^2)\sin(x) - 2x\cos(x)}{x^3}                           \\
              0     & = g'\ [ 2x\cos(x) - 2\sin(x) + 2\sin(x)]
              + g''\ [x \sin(x)]                                                      \\
              0     & = g''[x\sin(x)] + g'[2x\cos(x)]
          \end{align}
          Solving the reduced order equation,
          \begin{align}
              h      & = g'                                   &
              h'     & = -h\ [2\cot(x)]                         \\
              \ln(h) & = - 2\ln(\sin x)                       &
              g'     & = \frac{1}{\sin^{2}x}                    \\
              g      & = -\cot(x)                             &
              y_2    & = gy_1 = \color{y_p} \frac{\cos(x)}{x}
          \end{align}

    \item Finding indicial equation, using the coefficients of the lowest power,
          \begin{align}
              x^2 y'' + (x) y   & = 0   \\
              r(r - 1) + 0r + 0 & = 0 &
              r_1 = 1,\ r_2     & = 0
          \end{align}
          Finding the first solution,
          \begin{align}
              y_1            & = \iser{1} a_{m-1}\ x^{m}      &
              xy_1''         & = \iser{1} a_{m}\ m(m+1) x^{m}   \\
              a_{m}          & = \frac{-a_{m-1}}{(m)(m+1)}      \\
              y_1            & = x\left[ 1 - \frac{x}{2!}
                  + \frac{x^2}{2 \cdot 3!} - \frac{x^3}{6 \cdot 4!}
              + \dots\right] &
              y_1            & = \color{y_h}  \iser{1}
              \frac{(-1)^{m-1}}{(m-1)!\ m!}\ x^{m}
          \end{align}
          Finding the second solution using standard forumla,
          \begin{align}
              y_2     & = ky_1\ln(x) + \iser{0}A_m x^m                            \\
              y_2''   & = k\left[y_1'' \ln(x) - \frac{y_1}{x^2}
              + \frac{2y_1'}{x}\right] + \iser{2} A_m\ m(m-1) x^{m-2}             \\
              0       & = kx \ln(x) y_1'' - \frac{ky_1}{x} + 2ky_1' + k\ln(x) y_1 \\
                      & + 1 + \iser{1}[A_m + A_{m+1}\ m(m+1)] x^m                 \\
              -(k+1)  & = k \left[\iser{1}\frac{(-1)^m (2m + 1)}
                  {m!\ (m+1)!}\ x^m\right]
              + \iser{1}[A_m + A_{m+1}\ m(m+1)] x^m                               \\
              k       & = -1                                                      \\
              A_{m+1} & = \left[ \frac{(-1)^{m} (2m+1)}{m!\ (m+1)!} - A_m \right]
              \ \frac{1}{m(m+1)}                                                  \\
              y_2(x)  & = {\color{y_p}-y_1(x)\ln(x) + 1
              + \iser{1} A_m x^m}
          \end{align}

    \item Finding indicial equation, using the coefficients of the lowest power,
          \begin{align}
              x y'' + (2x+1)y' + (x+1)y         & = 0   \\
              x^2 y'' + (2x+1)\ xy' + x(x+1)\ y & = 0   \\
              r(r - 1) + r + 0                  & = 0 &
              r_1 = 0,\ r_2                     & = 0
          \end{align}
          Finding the first solution, by matching coefficients
          \begin{align}
              y_1   & = x^0 \iser{0} a_m x^m
              \qquad\qquad\qquad x^2 y_1'' = \iser{2} a_m\ m(m-1)x^m                   \\
              xy_1' & = \iser{1} a_m\ m x^{m}
              \qquad\qquad\qquad x^2y_1' = \iser{2} a_{m-1}\ (m-1) x^m                 \\
              a_0   & = 1  \qquad a_1 = -a_0 = -1                                      \\
              0     & = m^2\ a_m + (2m-1)\ a_{m-1} + a_{m-2}                           \\
              a_m   & = -\frac{(2m-1)\ a_{m-1} + a_{m-2}}{m^2}                         \\
              y_1   & = 1 - x + \frac{x^2}{2} - \frac{x^3}{6} + \frac{x^4}{24} - \dots \\
                    & = \iser{0} \frac{(-x)^m}{m!} = \color{y_h} \exp(-x)
          \end{align}
          Finding second root by reduction of order,
          \begin{align}
              y_2       & = gy_1                              &
              y_2'      & = g'y_1 + gy_1'                       \\
              y_2''     & = g''y_1 + 2g'y_1' + gy_1''           \\
              0         & = x(g''y_1 + 2g'y_1' + gy_1'')        \\
                        & + (2x+1)(g'y_1 + gy_1') + (x+1)gy_1   \\
              g''[xy_1] & = -g' [2xy_1' + (2x+1)y_1]          &
              g''x      & = -g'                                 \\
              \ln(g')   & = -\ln(x)                           &
              g         & = \ln(x)                              \\
              y_2       & = \color{y_p} \exp(-x)\ \ln(x)
          \end{align}

    \item Finding indicial equation, using the coefficients of the lowest power,
          \begin{align}
              x y'' + 2x^3 y' + (x^2 - 2)y          & = 0   \\
              x^2 y'' + (2x^3)\ xy' + (x^3 - 2x)\ y & = 0   \\
              r(r - 1) + 0r + 0                     & = 0 &
              r_1 = 1,\ r_2                         & = 0
          \end{align}
          Finding the first solution, by matching coefficients
          \begin{align}
              y_1      & = \iser{1} a_{m-1} x^{m}
              \qquad\qquad\qquad x y_1'' = \iser{1} a_m\ m(m+1)x^{m}              \\
              x^3 y_1' & = \iser{3} a_{m-3}\ (m-2) x^{m}
              \qquad\qquad\qquad x^2y_1 = \iser{3} a_{m-3}\ x^{m}                 \\
              2a_1     & = 2a_0  \qquad a_1 = a_0 = 1                             \\
              6a_2     & = 2a_1  \qquad a_2 = a_1/3 = 1/3                         \\
              0        & = m(m+1)\ a_m + (2m-3)\ a_{m-3} - 2a_{m-1}               \\
              a_m      & = \frac{2 a_{m-1} - (2m-3) a_{m-3}}{m(m+1)}              \\
              y_1      & = {\color{y_h} x + x^2 + \frac{x^3}{3} - \frac{7x^4}{36}
              - \frac{97x^5}{360} - \dots}
          \end{align}
          Finding the second solution using standard forumla,
          \begin{align}
              y_2   & = ky_1\ln(x) + \iser{0}A_m x^m                             \\
              y_2'  & = \frac{ky_1}{x} + k\ln(x)y_1' + \iser{0}A_{m+1}\ (m+1)x^m \\
              y_2'' & = k\left[y_1'' \ln(x) - \frac{y_1}{x^2}
              + \frac{2y_1'}{x}\right] + \iser{2}A_m\ (m)(m-1)x^{m-2}            \\
              0     & = \frac{-ky_1}{x} + 2ky_1' + 2kx^2y_1                      \\
                    & + \iser{1}A_{m+1}\ m(m+1) x^m
              + \iser{3}2 A_{m-2}\ (m-2) x^{m}                                   \\
                    & - \iser{0}2 A_{m}\ x^{m} + \iser{2}A_{m-2} x^{m}
          \end{align}
          Equating coefficients of $ x^0 $ and $ x^1 $,
          \begin{align}
              0       & = (k - 2A_0) + (2A_2 - 2A_1 + 3k) x
              + (-k/3 + 2k + 6A_3 - 2A_2 + A_0) x^2                               \\
              k       & = 2 \qquad A_0 = 1 \qquad A_2 = A_1 - 3 \qquad
              A_3 = \frac{A_2 - 13}{3}                                            \\
                      & + \iser{1}\frac{(-1)^m (2m + 1)}
              {m!\ (m+1)!}\ x^m
              + \iser{1}[A_m + A_{m+1}\ m(m+1)] x^m                               \\
              k       & = -1                                                      \\
              A_{m+1} & = \left[ \frac{(-1)^{m} (2m+1)}{m!\ (m+1)!} - A_m \right]
              \ \frac{1}{m(m+1)}                                                  \\
              y_2(x)  & = {\color{y_p} 2y_1(x)\ln(x) + 1
              + A_1\ x + (A_1 - 3)\ x^2 + \frac{A_1 - 16}{3}\ x^3 + \dots}
          \end{align}
          No elegant closed form for the higher powers. Need to manually calculate the
          rest of the $ \{A_m\} $ for $ m \geq 2 $.

    \item Finding indicial equation, using the coefficients of the lowest power,
          \begin{align}
              y'' + (x-1)y             & = 0   \\
              x^2 y'' + (x^3 - x^2)\ y & = 0   \\
              r(r - 1) + 0r + 0        & = 0 &
              r_1 = 1,\ r_2            & = 0
          \end{align}
          Finding the first solution, by matching coefficients
          \begin{align}
              y_1     & = \iser{1} a_{m-1} x^{m}            &
              y_1''   & = \iser{0} a_{m+1}\ (m+2)(m+1)x^{m}   \\
              x y_1   & = \iser{2} a_{m-2}\ x^{m}             \\
              2a_1    & = 0                                 &
              a_1     & = 0                                   \\
              6a_2    & = a_0                               &
              a_2     & = 1/6                                 \\
              a_{m-1} & = a_{m-2} + (m+2)(m+1)\ a_{m+1}     &
              a_{m}   & = \frac{a_{m-2} - a_{m-3}}{(m+1)m}    \\
              y_1     & = {\color{y_h} x + \frac{x^3}{6}
              - \frac{x^4}{12} + \frac{x^5}{120} - \frac{x^6}{120} + \dots}
          \end{align}
          Finding the second solution using standard forumla,
          \begin{align}
              y_2   & = ky_1\ln(x) + \iser{0}A_m x^m                             \\
              y_2'  & = \frac{ky_1}{x} + k\ln(x)y_1' + \iser{0}A_{m+1}\ (m+1)x^m \\
              y_2'' & = k\left[y_1'' \ln(x) - \frac{y_1}{x^2}
              + \frac{2y_1'}{x}\right] + \iser{2}A_m\ (m)(m-1)x^{m-2}            \\
              0     & = \frac{-ky_1}{x^2} + \frac{2ky_1'}{x}                     \\
                    & + \iser{0}A_{m+2}\ (m+1)(m+2) x^m
              + \iser{1}2 A_{m-1}\ x^{m} - \iser{0} A_m\ x^m
          \end{align}
          Finding coefficients of $ x^0 $ and $ x^1 $ terms,
          \begin{align}
              0   & = -\frac{k}{x} - \frac{kx}{6} + \frac{kx^2}{12} + \frac{2k}{x}
              + kx - \frac{2kx^2}{3}                                                \\
                  & + (2A_2 - A_0) + (6A_3 + 2A_0 - A_1)x + (12A_4 + 2A_1 - A_2)x^2
              + \dots                                                               \\
              k   & = 0 \qquad A_0 = 1 \qquad A_1 = 0\ \text{(arbitrary)}           \\
              y_2 & = {\color{y_p} 1 + \frac{x^2}{2} - \frac{x^3}{3}
              + \frac{x^4}{24} + \dots  }
          \end{align}
          There is no elegant closed form for the rest of the coefficients $ \{A_i\} $ and
          successive comparison of higher powers of $ x $ needs to be used to find them.

    \item Finding indicial equation, using the coefficients of the lowest power,
          \begin{align}
              xy'' + y' - xy         & = 0   \\
              x^2 y'' + xy' - (x^2)y & = 0   \\
              r(r - 1) + r + 0       & = 0 &
              r_1 = 0,\ r_2          & = 0
          \end{align}
          Finding the first solution, by matching coefficients
          \begin{align}
              y_1         & = \iser{0} a_{m} x^{m}                                 &
              xy_1''      & = \iser{1} a_{m+1}\ (m+1)(m)x^{m}                        \\
              x y_1       & = \iser{1} a_{m-1}\ x^{m}                              &
              y'          & = \iser{0} a_{m+1}\ (m+1) x^{m}                          \\
              a_1         & = 0                                                      \\
              2a_2 + 2a_2 & = a_0                                                  &
              a_2         & = 1/4                                                    \\
              a_{m-1}     & = (m+1)^2\ a_{m+1}                                     &
              a_{m}       & = \frac{a_{m-2}}{m^2}                                    \\
              y_1         & = 1 + \frac{x^2}{2^2} + \frac{x^4}{2^2\ 4^2}
              + \frac{x^6}{2^2\ 4^2\ 6^2} + \dots                                    \\
              y_1         & = {\color{y_h} \iser{0} \frac{x^{2m}}{2^{2m}\ (m!)^2}}
          \end{align}
          Finding second solution by using standard result,
          \begin{align}
              y_2   & = y_1 \ln(x) + \iser{1}A_m\ x^m                              \\
              y_2'  & = y_1' \ln(x) + \frac{y_1}{x} + \iser{0} A_{m+1}\ (m+1)x^{m} \\
              y_2'' & = y_1'' \ln(x) + \frac{2y_1'}{x} - \frac{y_1}{x^2}
              + \iser{0}A_{m+2}\ (m+1)(m+2)x^{m}                                   \\
              0     & = 2y_1' + \iser{1}A_{m+1}\ m(m+1)x^{m}
              + \iser{0}A_{m+1}\ (m+1)x^m - \iser{2}A_{m-1}\ x^m
          \end{align}
          Equating coefficients of $ x^0, x^1 $,
          \begin{align}
              0   & = 0 + x/2 + 2A_2 x + A_1 + 2A_2 x              &
              A_2 & = -1/8 \quad A_1 = 0                                           \\
              A_2 & = 16A_4 + 1/8                                  & A_4 & = -1/64 \\
              A_4 & = 36A_6 + 1/192                                &
              A_6 & = -1/1728                                                      \\
              A_1 & = 9A_3                                         & A_3 & = 0     \\
              A_3 & = 25A_5                                        &
              A_5 & = 0                                                            \\
              y_2 & = \color{y_p} y_1\ln(x) - \left[ \frac{x^2}{8}
                  + \frac{x^4}{64} + \frac{x^6}{1728} + \dots \right]
          \end{align}

    \item Finding indicial equation, using the limits as $ x \to 0 $
          of $ b(x) $ and $ c(x) $ after rewriting the ODE in standard form,
          \begin{align}
              2x(x-1)y'' - (x+1)y' + y                                 & = 0 \\
              x^2y'' - \frac{(x+1)}{2(x-1)}\ xy' + \frac{x}{2(x-1)}\ y & = 0 \\
              r(r-1) + \frac{r}{2} + 0                                 & = 0 \\
              r_1 = 1/2 \qquad\qquad r_2                               & = 0
          \end{align}
          Finding the first solution using $ r_1 = 1/2 $, and noting that the series
          never truncates upon differentiation of fractional powers,
          \begin{align}
              y_1       & = \iser{0} a_m x^{m + 0.5}                          \\
              y_1'      & = \iser{-1}a_{m+1}\ (m + 1.5)x^{m+0.5}            &
              xy_1'     & = \iser{0}a_{m}\ (m + 0.5)x^{m+0.5}                 \\
              y_1''     & = \iser{0}a_{m}\ (m + 0.5)(m - 0.5)x^{m - 1.5}    &
              x^2 y_1'' & = \iser{0}a_{m}\ (m + 0.5)(m - 0.5)x^{m + 0.5}      \\
              x y_1''   & = \iser{-1}a_{m+1}\ (m + 1.5)(m + 0.5)x^{m + 0.5}
          \end{align}
          Finding the recursive relation for higher coefficients,
          \begin{align}
              0       & = 2a_m(m+0.5)(m-0.5) - 2a_{m+1}(m+1.5)(m+0.5) \\
                      & - a_m(m+0.5) - a_{m+1}(m+1.5) + a_m           \\
              a_{m+1} & = a_m\ \frac{m(2m-1)}{(m+1.5)(2m+1.5)}        \\
              y_1     & = \color{y_h} x^{1/2}
          \end{align}
          Finding the second solution using $ r_2 = 0 $,
          \begin{align}
              y_1       & = \iser{0} a_m x^m                   \\
              y_1'      & = \iser{0}a_{m+1}\ (m+1)x^{m}      &
              xy_1'     & = \iser{1}a_{m}\ (m)x^{m}            \\
              y_1''     & = \iser{0}a_{m+2}\ (m+2)(m+1)x^{m} &
              x^2 y_1'' & = \iser{2}a_{m}\ (m)(m - 1)x^{m}     \\
              x y_1''   & = \iser{1}a_{m+1}\ (m+1)(m)x^{m}
          \end{align}
          Finding the recursive relation for higher coefficients,
          \begin{align}
              0       & = 2m(m-1)\ a_m - 2m(m+1)\ a_{m+1} - m\ a_m
              - (m+1)\ a_{m+1} + a_m                                 \\
              a_{m+1} & = \frac{(2m - 1)(m - 1)}{(2m+1)(m+1)}\ a_m   \\
              a_{1}   & = a_0 = 1 \qquad\qquad a_2 = a_3 = \dots = 0 \\
              y_2     & = \color{y_p} 1 + x
          \end{align}

    \item Finding indicial equation, using the limits as $ x \to 0 $
          of $ b(x) $ and $ c(x) $ after rewriting the ODE in standard form,
          \begin{align}
              xy'' + 2y' + 4xy              & = 0  \\
              x^2y'' + (2)\ xy' + (4x^2)\ y & = 0  \\
              r(r-1) + 2r + 0               & = 0  \\
              r_1 = 0 \qquad\qquad r_2      & = -1
          \end{align}
          Finding the first solution using $ r_1 = 0 $,
          \begin{align}
              x^2y'' & = \iser{2}a_m\ m(m-1)x^m &
              xy'    & = \iser{1}a_m\ m x^m       \\
              x^2y   & = \iser{2}a_{m-2}\ x^{m}   \\
              a_1    & = 0 = a_3 = a_5 = \dots
          \end{align}
          Finding the recursive relation for higher coefficients,
          \begin{align}
              0   & = m(m-1)\ a_m + 2m\ a_m + 4a_{m-2}                \\
              a_m & = \frac{-4}{m(m+1)}a_{m-2}                        \\
              y_1 & =  1 - \frac{2^2x^2}{3!} + \frac{2^4 x^4}{5!}
              - \frac{2^6 x^6}{7!} + \dots                            \\
              y_1 & = \frac{1}{2x}\left[ 2x - \frac{(2x)^3}{3!}
              + \frac{(2x)^5}{5!} - \frac{(2x)^7}{7!} + \dots \right] \\
              y_1 & = \color{y_h} \frac{\sin(2x)}{2x}
          \end{align}
          Finding second root by reduction of order,
          \begin{align}
              y_2       & = gy_1                            &
              y_2'      & = g'y_1 + gy_1'                     \\
              y_2''     & = g''y_1 + 2g'y_1' + gy_1''         \\
              0         & = x(g''y_1 + 2g'y_1' + gy_1'')      \\
                        & + 2(g'y_1 + gy_1') + (4x)gy_1       \\
              g''[xy_1] & = -g' [2xy_1' + 2y_1]               \\
              g''       & = -g' [4\cot(2x)]                   \\
              \ln(g')   & = -2\ln(|\sin(2x)|)               &
              g'        & = \csc^{2}(2x)                      \\
              g         & = -\frac{\cot(2x)}{2}             &
              y_2       & = \color{y_p} \frac{\cos(2x)}{2x}
          \end{align}

    \item Finding indicial equation, using the limits as $ x \to 0 $
          of $ b(x) $ and $ c(x) $ after rewriting the ODE in standard form,
          \begin{align}
              xy'' + (2-2x)y' + (x-2)y               & = 0  \\
              x^2y'' + (2 - 2x)\ xy' + (x^2 - 2x)\ y & = 0  \\
              r(r-1) + 2r + 0                        & = 0  \\
              r_1 = 0 \qquad\qquad r_2               & = -1
          \end{align}
          Finding the first solution using $ r_1 = 0 $,
          \begin{align}
              xy''        & = \iser{1}a_{m+1}\ (m+1)(m)x^{m} & \\
              2y'         & = 2\iser{0}a_{m+1}\ (m+1) x^{m}  &
              -2xy'       & = -2\iser{1}a_m\ m x^{m}           \\
              -2y         & = -2\iser{0}a_{m}\ x^{m}         &
              xy          & = \iser{1}a_{m-1}\ x^{m}           \\
              2a_1 - 2a_0 & = 0                              &
              a_1         & = a_0 = 1                          \\
              6a_2        & = 4a_1 - a_0                     &
              a_2         & = 1/2
          \end{align}
          Finding the recursive relation for higher coefficients,
          \begin{align}
              0       & = (m^2+ 3m + 2)\ a_{m+1} - (2m + 2)\ a_m + a_{m-1}         \\
              a_{m+1} & = \frac{2(m+1)\ a_m - a_{m-1}}{(m+1)(m+2)}                 \\
              y_1     & =  1  + x + \frac{x^2}{2} + \frac{x^3}{6} + \frac{x^4}{24}
              + \frac{x^5}{120} + \dots                                            \\
              y_1     & = \color{y_h} \exp(x)
          \end{align}
          Finding second root by reduction of order,
          \begin{align}
              y_2       & = gy_1                              &
              y_2'      & = g'y_1 + gy_1'                       \\
              y_2''     & = g''y_1 + 2g'y_1' + gy_1''           \\
              0         & = x(g''y_1 + 2g'y_1' + gy_1'')        \\
                        & + (2-2x)(g'y_1 + gy_1') + (x-2)gy_1   \\
              g''[xy_1] & = -g' [2xy_1' + 2y_1 - 2xy_1]         \\
              g''[x]    & = -g' [2]                             \\
              \ln(g')   & = -2\ln(x)                          &
              g'        & = x^{-2}                              \\
              g         & = \frac{-1}{x}                      &
              y_2       & = \color{y_p} \frac{\exp(x)}{x}
          \end{align}

    \item Finding indicial equation, using the limits as $ x \to 0 $
          of $ b(x) $ and $ c(x) $ after rewriting the ODE in standard form,
          \begin{align}
              x^2y'' + (6)\ xy' + (4x^2 + 6)\ y & = 0  \\
              r^2 + 5r + 6                      & = 0  \\
              r_1 = -3 \qquad\qquad r_2         & = -2
          \end{align}
          Finding the first solution using $ r_1 = -3 $,
          \begin{align}
              y                    & = \iser{-3}a_{m+3}\ x^{m}          &
              x^2 y                & = \iser{-1}a_{m+1}\ x^{m}            \\
              y'                   & = \iser{-4}a_{m+4}\ (m+1)x^{m}     &
              xy'                  & = \iser{-3}a_{m+3}\ (m)x^{m}         \\
              x^2 y''              & =  \iser{-3}a_{m+3}\ (m)(m-1)x^{m} & \\
              12a_0 - 18a_0 + 6a_0 & = 0                                &
              a_0                  & = 0                                  \\
              6a_1 - 12 a_1 + 6a_1 & = 0                                &
              a_1                  & = 1\ (\text{free})
          \end{align}
          Finding the recursive relation for higher coefficients,
          \begin{align}
              0       & = (m^2 + 5m + 6)\ a_{m+3} + 4a_{m+1}          \\
              a_{m+3} & = \frac{-4a_{m+1}}{(m+2)(m+3)}                \\
              y_1     & =  \frac{1}{x^3} \left[ x - \frac{2^2x^3}{3!}
              + \frac{2^4x^5}{5!} - \frac{2^6x^7}{7!} + \dots \right] \\
              y_1     & = \color{y_h} \frac{\sin(2x)}{2x^3}
          \end{align}
          Finding second root by reduction of order,
          \begin{align}
              y_2      & = gy_1                                      &
              y_2'     & = g'y_1 + gy_1'                               \\
              y_2''    & = g''y_1 + 2g'y_1' + gy_1''                   \\
              0        & = x^2(g''y_1 + 2g'y_1' + gy_1'')              \\
                       & + (6x)(g'y_1 + gy_1') + (4x^2 + 6)gy_1        \\
              g''[y_1] & = -g' \left[ 2y_1' + 6\frac{y_1}{x} \right]   \\
              g''      & = -g' [4\cot(2x)]                             \\
              \ln(g')  & = -2\ln(|\sin(2x)|)                         &
              g'       & = \csc^2(2x)                                  \\
              g        & = -\frac{\cot(2x)}{2}                       &
              y_2      & = \color{y_p} \frac{\cos(2x)}{4x^3}
          \end{align}

    \item Finding indicial equation, using the coefficients of the lowest power,
          \begin{align}
              x y'' + (-2x+1)y' + (x-1)y         & = 0   \\
              x^2 y'' + (-2x+1)\ xy' + x(x-1)\ y & = 0   \\
              r(r - 1) + r + 0                   & = 0 &
              r_1 = 0,\ r_2                      & = 0
          \end{align}
          Finding the first solution, by matching coefficients
          \begin{align}
              y_1         & = \iser{0} a_m x^m                      &
              xy_1        & = \iser{1} a_{m-1} x^{m}                  \\
              xy_1''      & = \iser{1} a_{m+1}\ (m+1)(m)x^{m}         \\
              y_1'        & = \iser{0} a_{m+1}\ (m+1) x^{m}         &
              xy_1'       & = \iser{1} a_{m}\ (m) x^m                 \\
              a_1 - a_0   & = 0                                     &
              a_1         & = a_0 = 1                                 \\
              (2m+1)\ a_m & = (m^2+ 2m + 1)\ a_{m+1} + a_{m-1}        \\
              a_{m+1}     & = \frac{(2m+1)\ a_m - a_{m-1}}{(m+1)^2}   \\
              y_1         & = 1 + x + \frac{x^2}{2} + \frac{x^3}{6}
              + \frac{x^4}{24} + \dots                                \\
                          & = \color{y_h} \exp(x)
          \end{align}
          Finding second root by reduction of order,
          \begin{align}
              y_2       & = gy_1                              &
              y_2'      & = g'y_1 + gy_1'                       \\
              y_2''     & = g''y_1 + 2g'y_1' + gy_1''           \\
              0         & = x(g''y_1 + 2g'y_1' + gy_1'')        \\
                        & + (1-2x)(g'y_1 + gy_1') + (x-1)gy_1   \\
              g''[xy_1] & = -g' [2xy_1' + (1 - 2x)y_1]        &
              g''x      & = -g'                                 \\
              \ln(g')   & = -\ln(x)                           &
              g         & = \ln(x)                              \\
              y_2       & = \color{y_p} \exp(x)\ \ln(x)
          \end{align}

    \item Hypergeometric ODE,
          \begin{enumerate}
              \item Indicial equation,
                    \begin{align}
                        x(1-x)\ y'' + [c - (a+b+1)x]\ y' - ab\ y       & = 0     \\
                        (1-x)\ x^2y'' + [c - (a+b+1)x]\ xy' - (abx)\ y & = 0     \\
                        r(r-1) + cr + 0                                & = 0     \\
                        r_1 = 0 \qquad r_2                             & = 1 - c
                    \end{align}
                    Applying Frobenius method with $ r_1  = 0$,
                    \begin{align}
                        y       & = \iser{0} j_m\ x^m                  \\
                        y'      & = \iser{0} j_{m+1}\ (m+1)x^{m}     &
                        xy'     & = \iser{1} j_m\ mx^{m}               \\
                        y''     & = \iser{2} j_m\ m(m-1)x^{m-2}      &
                        x^2 y'' & = \iser{2} j_m\ m(m-1)x^{m}          \\
                        xy''    & = \iser{1}j_{m+1}\ m(m+1)x^{m}       \\
                        c\ j_1  & = ab\ j_0                          &
                        j_1     & = \frac{ab}{c}\ j_0 = \frac{ab}{c}
                    \end{align}
                    Finding recursive relation for higher coefficients, using $ j_0=1 $
                    \begin{align}
                        j_{m+1} & = \frac{(m+a)(m+b)} {(m+c)(m+1)}\ j_m \\
                        y       & = 1 + \frac{ab}{1!\ c}\ x
                        + \frac{a(a+1)\ b(b+1)}{2!\ c(c+1)}\ x^2        \\
                                & + \frac{a(a+1)(a+2)\ b(b+1)(b+2)}
                        {3!\ c(c+1)(c+2)}\ x^3 + \dots
                    \end{align}
                    To arrive at the geometric series sum,
                    \begin{align}
                        F(1,1,1\ ;x) & = 1 + x + x^2 + x^3 + \dots = \frac{1}{1-x}
                    \end{align}
                    This is also true for $ a = 1, b = c $ or $ b = 1, a = c $ since
                    those two expresisons will cancel.

              \item For the infinite series to reduce to a polynomial, it must truncate.
                    This requires $ b $ or $ a $ to be non-positive integers.
                    Performing the ratio test,
                    \begin{align}
                        \frac{t_{m+1}}{t_m}                     &
                        = \frac{(a+m-1)(b+m-1)}{m\ (c+m-1)}\ x        \\
                        \lim_{m \to \infty} \frac{t_{m+1}}{t_m} & = x
                    \end{align}
                    For the ratio test to guarantee convergence, $ |x| < 1 $.

              \item Showing elementray functions to be special cases of the solution
                    to the hypergeometric ODE,
                    \begin{align}
                        F(-n,b,b\ ;-x) & = 1 + nx + \frac{n(n-1)}{2!}\ x^2 +
                        \frac{n(n-1)(n-2)}{3!}\ x^3 + \dots                          \\
                                       & = \sum_{r=0}^{n} \frac{n!}{(n-r)!\ r!}\ x^r
                        = \color{y_h} (1+x)^n
                    \end{align}
                    Starting with $ F(1-n,1,2\ ; x) $,
                    \begin{align}
                        F(1-n,1,2\ ;x)        & = 1 - \frac{(n-1)\ 1!}{1!\ 2!}\ x
                        + \frac{(n-1)(n-2)\ 2!}{2!\ 3!}\ x^2 + \dots                \\
                                              & + \frac{(n-1)(n-2)\dots(1)\ (n-1)!}
                        {(n-1)!\ n!}\ (-1)^n x^{n-1}                                \\
                                              & = \frac{-1}{nx} \left[ -nx
                            + \frac{n(n-1)}{2!}\ x^2
                        - \frac{n(n-1)(n-2)}{3!}\ x^3 + \dots \right]               \\
                                              & = \frac{-1}{nx} [ (1-x)^n - 1 ]     \\
                        {\color{y_p} (1-x)^n} & = 1 - nx\ F(1-n,1,2\ ;x)
                    \end{align}
                    Starting with $ a = 1/2, b = 1, c = 3/2 $,
                    \begin{align}
                        F(1/2,1,3/2\ ;-x^2)    & = 1 - \frac{1}{1!\ 3}\ x^2
                        + \frac{(3/4)\ 2!}{2!\ (15/4)}\ x^4
                        - \frac{(15/8)\ 3!}{3!\ (105/8)}\ x^6 + \dots                  \\
                                               & =\frac{1}{x} \left[ x - \frac{x^3}{3}
                        + \frac{x^5}{5} - \frac{x^7}{7} + \dots \right]                \\
                        x\ F(1/2,1,3/2\ ;-x^2) & = \color{y_h} \arctan(x)
                    \end{align}
                    Starting with $ a = 1/2, b = 1/2, c = 3/2 $,
                    \begin{align}
                        F\left( \frac{1}{2}, \frac{1}{2},
                        \frac{3}{2}\ ; x^2 \right) & = 1 + \frac{(1/4)}
                        {1!\ (6/4)}\ x^2 + \frac{(9/16)}{2!\ (15/4)}\ x^4
                        + \frac{(15/8)(15/8)}{3!\ (105/8)}\ x^6 + \dots       \\
                                                   & =\frac{1}{x} \left[ x
                            + \frac{x^3}{6} + \frac{3x^5}{40}
                        + \frac{6!\ x^7}{4^3\ 3!\ 3!\ 7} + \dots \right]      \\
                        x\ F(1/2,1,3/2\ ;-x^2)     & = \color{y_p} \arcsin(x)
                    \end{align}
                    Starting with $ a = 1 = b, c = 2 $,
                    \begin{align}
                        F(1,1,2\ ; -x)    & = 1 - \frac{1!\ 1!}{1!\ 2!}\ x
                        + \frac{2!\ 2!}{2!\ 3!}\ x^2 - \frac{3!\ 3!}{3!\ 4!}\ x^3
                        + \dots                                                   \\
                                          & = \frac{1}{x}\left[ x - \frac{x^2}{2}
                            + \frac{x^3}{3}
                        - \frac{x^4}{4} + \dots \right] = \frac{\ln(x)}{x}        \\
                        x\ F(1,1,2\ ; -x) & = \color{y_h} \frac{\ln(x)}{x}
                    \end{align}
                    Starting with $ a = 1/2, b = 1, c = 3/2 $,
                    \begin{align}
                        F(1/2,1,3/2\ ;x^2)      & = 1 + \frac{1}{1!\ 3}\ x^2
                        + \frac{(3/4)\ 2!}{2!\ (15/4)}\ x^4
                        + \frac{(15/8)\ 3!}{3!\ (105/8)}\ x^6 + \dots                   \\
                                                & =\frac{1}{x} \left[ x + \frac{x^3}{3}
                        + \frac{x^5}{5} + \frac{x^7}{7} + \dots \right]                 \\
                                                & = \frac{1}{2x} [\ln(1+x) - \ln(1-x)]  \\
                        2x\ F(1/2,1,3/2\ ;-x^2) & = \color{y_p}
                        \frac{\ln(1+x)}{\ln(1-x)}
                    \end{align}
                    More relations TBC.

              \item Frobenius method with $ r_2 = 1-c $,
                    \begin{align}
                        y       & = \iser{0} j_m\ x^{m-c+1}                     \\
                        y'      & = \iser{-1} j_{m+1}\ (m-c+2)x^{m-c+1}         \\
                        xy'     & = \iser{0} j_m\ (m-c+1)x^{m-c+1}              \\
                        y''     & = \iser{-2} j_{m+2}\ (m-c+3)(m-c+2)x^{m-c+1}  \\
                        x^2 y'' & = \iser{0} j_m\ (m-c+1)(m-c)x^{m-c+1}         \\
                        xy''    & = \iser{-1} j_{m+1}\ (m-c+2)(m-c+1)x^{m-c+1}  \\
                        0       & = j_{m+1}\ (m-c+2)(m-c+1) - j_m\ (m-c+1)(m-c) \\
                                & + j_{m+1}\ c(m-c+2) - j_{m}\ (a+b+1)(m-c+1)
                        - j_m\ (ab)                                             \\
                        j_1     & = j_0\ \frac{(a-c+1)(b-c+1)}{1\cdot(2-c)}     \\
                        j_2     & = j_1\ \frac{(a-c+2)(b-c+2)}{2\cdot(3-c)}
                    \end{align}
                    Finding coefficients of $ x^{1-c} $ and $ x^{2-c} $,
                    \begin{align}
                        y_2 & = x^{1-c} \Bigg[ 1 + \frac{(a-c+1)(b-c+1)}{1!\ (2-c)}\ x \\
                            & + \frac{(a-c+1)(a-c+2)(b-c+1)(b-c+2)}{2!\ (2-c)(3-c)}
                        \ x^2 + \dots \Bigg]                                           \\
                            & = x^{(1-c)}\ F(a-c+1, b-c+1, 2-c\ ; x)
                    \end{align}
                    The correspondence to the hypergeometric fucntion is readily seen
                    from the form of the solution $ y_2(x) $.

              \item General hypergeometric function,
                    \begin{align}
                        x           & = \frac{t - t_1}{t_2 - t_1}                &
                        x(1-x)      & = \frac{(t_2 - t)(t - t_1)}{(t_2 - t_1)^2}   \\
                        (t_2 - t_1) & = A^2 - 4B                                 &
                        t_1 t_2     & = B                                          \\
                        x(1-x)      & = \frac{t^2 + At + B}{t_2 - t_1}             \\
                        \diff yt    & = \diff yx\ \frac{1}{(t_2 - t_1)}          &
                        \diff[2] yt & = \diff[2]yx\ \frac{1}{(t_2 - t_1)^2}        \\
                        (Ct + D)    & = C(t_2 - t_1)\ x + [C t_1 + D]            &
                    \end{align}
                    Consolidating all terms,
                    \begin{align}
                        (t^2 + At + B)\ \ddot{y} & = \frac{x(x-1)}{(t_2 - t_1)}\ y'' \\
                        (Ct + D)\ \dot{y}        & = [(a+b+1)x - c] y'               \\
                        K                        & = ab                              \\
                        Ct_1 + D                 & = -c(t_2 - t_1)                   \\
                        C                        & = a + b + 1
                    \end{align}
                    This reduces the general hypergeometric equation to the special form
                    involving $ a,b,c $.
          \end{enumerate}

    \item Solving using hypergeometric ODE,
          \begin{align}
              0     & = x(1-x)y'' - (0.5 + 3x)y' - y                       \\
              c     & = -0.5                                             &
              (a+b) & = 2 \qquad ab = 1                                    \\
              c     & = \frac{-1}{2}                                     &
              a     & = 1 \qquad b = 1                                     \\
              y_1   & = \color{y_h} F\left( 1,1,\frac{-1}{2}\ ;x \right) &
              y_2   & = \color{y_p} x^{3/2}F \left(
              \frac{5}{2}, \frac{5}{2}, \frac{5}{2}\ ;x \right)
          \end{align}

    \item Solving using hypergeometric ODE,
          \begin{align}
              0     & = x(1-x)y'' + (0.5 + 2x)y' - 2y                       \\
              c     & = 0.5                                               &
              (a+b) & = -3 \qquad ab = 2                                    \\
              c     & = \frac{1}{2}                                       &
              a     & = -2 \qquad b = -1                                    \\
              y_1   & = \color{y_h} F\left( -2,-1,\frac{1}{2}\ ;x \right) &
              y_2   & = \color{y_p} x^{-1/2}F \left(
              \frac{-3}{2}, \frac{-1}{2}, \frac{3}{2}\ ;x \right)
          \end{align}

    \item Solving using hypergeometric ODE,
          \begin{align}
              0     & = x(1-x)y'' + (0.25)y' + 2y                \\
              c     & = 0.25                                 &
              (a+b) & = -1 \qquad ab = -2                        \\
              c     & = \frac{1}{4}                          &
              a     & = -2 \qquad b = 1                          \\
              y_1   & = F\left( -2,1,\frac{1}{4}\ ;x \right) &
              y_2   & = \color{y_p} x^{3/4}F \left(
              \frac{-5}{4}, \frac{7}{4}, \frac{7}{4}\ ;x \right) \\
              y_1   & = \color{y_h} 1 - 8x + \frac{32x^2}{5}
          \end{align}

    \item Solving using general hypergeometric form,
          \begin{align}
              (t^2 - 3t + 2)\ddot{y} - (0.5) \dot{y} + (0.25)y = 0                   \\
              A = -2 \qquad B = 2 \qquad C = 0 \qquad D = -0.5 \qquad K  = 0.25      \\
              t_1 = 1 \qquad t_2 = 2 \qquad a+b = -1 \qquad c = 0.5 \qquad ab = 0.25 \\
              a = \frac{-1}{2} \qquad b = \frac{-1}{2} \qquad c = \frac{1}{4}
              \qquad x = t-1                                                         \\
              y_1 = \color{y_h} F\left( \frac{-1}{2}, \frac{-1}{2}, \frac{1}{4}
              \ ;(t-1) \right)                                                       \\
              y_1 = \color{y_p} (t-1)^{3/4} F\left( \frac{1}{4}, \frac{1}{4}, \frac{7}{4}
              \ ;(t-1) \right)
          \end{align}

    \item Solving using general hypergeometric form,
          \begin{align}
              (t^2 - 5t + 6)\ddot{y} + (t - 1.5) \dot{y} - (4)y = 0                \\
              A = -5 \qquad B = 6 \qquad C = 1 \qquad D = -1.5 \qquad K  = -4      \\
              t_1 = 2 \qquad t_2 = 3 \qquad a+b = 0 \qquad c = -0.5 \qquad ab = -4 \\
              a = 2 \qquad b = -2 \qquad c = \frac{-1}{2} \qquad x = t-2           \\
              y_1 = \color{y_h} F\left( 2, -2, \frac{-1}{2}
              \ ;t-2 \right)                                                       \\
              y_1 = \color{y_p} (t-2)^{1/2} F\left( \frac{7}{2}, \frac{-1}{2},
              \frac{5}{2}\ ;t-2 \right)
          \end{align}

    \item Solving using general hypergeometric form,
          \begin{align}
              (t^2 + t)\ddot{y} + (t/3) \dot{y} - (1/3)y = 0                            \\
              A = 1 \qquad B = 0 \qquad C = 1/3 \qquad D = 0 \qquad K  = -1/3           \\
              t_1 = -1 \qquad t_2 = 0 \qquad a+b = -2/3 \qquad c = 1/3 \qquad ab = -1/3 \\
              a = -1 \qquad b = \frac{1}{3} \qquad c = \frac{1}{3} \qquad x = t+1       \\
              y_1 = \color{y_h} F\left( -1, \frac{1}{3}, \frac{1}{3}
              \ ;t+1 \right)                                                            \\
              y_1 = \color{y_p} (t+1)^{2/3} F\left( \frac{-1}{3}, 1,
              \frac{5}{3}\ ;t+1 \right)
          \end{align}
\end{enumerate}