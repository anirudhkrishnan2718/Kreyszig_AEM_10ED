\section{Linear ODEs, Bernoulli equation, Population Dynamics}

\begin{enumerate}
    \item To prove the two relations,

          \begin{align}
              \exp(-\ln x)      & = x^{-\ln e} = x^{-1} = \frac{1}{x}            \\
              \exp(-\ln \sec x) & = \sec x ^{-\ln e} = \sec x ^{-1}   & = \cos x
          \end{align}


    \item Suppose $ c \neq 0 $,

          \begin{align}
              h & = \int\ p(x)\ \dl x + c                             \\
              y & = e^{-c}\ e^{-h}\ \int\ e^{c}\ e^{h}r(x)\ \dl x + b \\
              y & = e^{-h}\ \int\ e^{h}r(x)\ \dl x + b
          \end{align}

          It is evident that the choice of $ c $ did not affect the final expression.
          So, the easy way out is to set $ c = 0 $

    \item Solving ODE and graphing,

          \begin{align}
              y'  - y & = 5.2                                                  \\
              p(x)    & = -1                                                 &
              r(x)    & = 5.2                                                  \\
              h       & = \int\ p(x)\ \dl x                                  &
                      & = \int -1\ \dl x                                       \\
                      & = -x                                                   \\
              y       & = e^{-h}\ \int\ e^{h}r(x)\ \dl x + c                   \\
                      & = e^{x} \left[ \int\ e^{-x} (5.2)\ \dl x + c \right]   \\
              y       & = -5.2 + ce^{x}                                      &
              c       & = 5.2
          \end{align}

          \begin{figure}[H]
              \centering
              \begin{tikzpicture}
                  \begin{axis}[
                          legend pos = outer north east,
                          grid = both,
                          width = 8cm,
                          height = 8cm,
                          Ani,
                          view     = {0}{90}, % for a view 'from above'
                      ]
                      \addplot[GraphSmooth, y_h, domain = -5:10,
                          restrict y to domain = -100:100]
                      {-5.2 + 5.2 * e^(x)};
                      \addplot[GraphSmooth, y_p, domain = -5:10,
                          restrict y to domain = -100:100, dotted]
                      {e^(x)};
                      \node[GraphNode, label={0:{(0, 0)}}] at (axis cs:0, 0) {};
                      \addlegendentry{$ y = 5.2 (e^{x} - 1)$};
                      \addlegendentry{$y_{\text{transient}} = 5.2\ e^{x}$};
                  \end{axis}
              \end{tikzpicture}
          \end{figure}

    \item Solving ODE and graphing,

          \begin{align}
              y'  - 2y                   & = -4x                                   \\
              p(x)                       & = -2                                  &
              r(x)                       & = -4x                                   \\
              h                          & = \int\ p(x)\ \dl x                   &
                                         & = \int -2\ \dl x                        \\
                                         & = -2x                                   \\
              y                          & = e^{-h}\ \int\ e^{h}r(x)\ \dl x + c    \\
                                         & = e^{2x} \left[ \int\ e^{-2x} (-4x)
              \ \dl x + c \right]                                                  \\
              \int\ e^{-2x} (-4x)\ \dl x & = 2x e^{-2x} - \int\ 2 e^{-2x}\ \dl x   \\
                                         & = e^{-2x} (2x+1)                        \\
              y                          & = (2x+1) +  ce^{2x}                   &
              c                          & = 1
          \end{align}

          \begin{figure}[H]
              \centering
              \begin{tikzpicture}
                  \begin{axis}[
                          legend pos = outer north east,
                          grid = both,
                          width = 8cm,
                          height = 8cm,
                          Ani,
                          view     = {0}{90}, % for a view 'from above'
                      ]
                      \addplot[GraphSmooth, y_h, domain = -2:10,
                          restrict y to domain = -2:10]
                      {2*x + 1 + e^(2*x)};
                      \addplot[GraphSmooth, y_p, domain = -2:10,
                          restrict y to domain = -2:10, dotted]
                      {e^(2*x)};
                      \node[GraphNode, label={135:{(0, 2)}}] at (axis cs:0, 2) {};
                      \addlegendentry{$ y = 2x + 1 + e^{2x}$};
                      \addlegendentry{$y_{\text{transient}} = e^{2x}$};
                  \end{axis}
              \end{tikzpicture}
          \end{figure}

    \item Solving ODE and graphing,

          \begin{align}
              y'  + ky                    & = e^{-kx}                               \\
              p(x)                        & = k                                   &
              r(x)                        & = e^{-kx}                               \\
              h                           & = \int\ p(x)\ \dl x                   &
                                          & = \int k\ \dl x                         \\
                                          & = kx                                    \\
              y                           & = e^{-h}\ \int\ e^{h}r(x)\ \dl x + c    \\
                                          & = e^{-kx} \left[ \int\ e^{kx} e^{-kx}
              \ \dl x + c \right]                                                   \\
              \int\ e^{kx} e^{-kx}\ \dl x & = x                                     \\
                                          & = e^{-kx} (x+c)                         \\
              y                           & = xe^{-kx} +  ce^{-kx}                &
              c  = k                      & = 1
          \end{align}

          \begin{figure}[H]
              \centering
              \begin{tikzpicture}
                  \begin{axis}[
                          legend pos = outer north east,
                          grid = both,
                          width = 8cm,
                          height = 8cm,
                          Ani,
                          view     = {0}{90}, % for a view 'from above'
                      ]
                      \addplot[GraphSmooth, y_h, domain = -4:10,
                          restrict y to domain = -2:6]
                      {e^(-x) * (x + 1)};
                      \addplot[GraphSmooth, y_p, domain = -4:10,
                          restrict y to domain = -2:6, dotted]
                      {e^(-x)};
                      \node[GraphNode, label={45:{(0, 1)}}] at (axis cs:0, 1) {};
                      \addlegendentry{$ y = xe^{-x} + e^{-x}$};
                      \addlegendentry{$y_{\text{transient}} = e^{-x}$};
                  \end{axis}
              \end{tikzpicture}
          \end{figure}

    \item Solving ODE and graphing, using the IC $ y(\pi/4) = 3 $

          \begin{align}
              y'  + 2y                     & = 4 \cos 2x                            \\
              p(x)                         & = 2                                  &
              r(x)                         & = 4 \cos 2x                            \\
              h                            & = \int\ p(x)\ \dl x                  &
                                           & = \int 2\ \dl x                        \\
                                           & = 2x                                   \\
              y                            & = e^{-h}\ \int\ e^{h}r(x)\ \dl x + c   \\
                                           & = e^{-2x} \left[ \int\ e^{2x}
              \ 4 \cos 2x\ \dl x + c \right]                                        \\
              4\int\ e^{2x} \cos 2x\ \dl x & = \frac{4e^{2x}}{2^{2} + 2^{2}}
              \ (2\cos 2x + 2 \sin 2x)                                              \\
                                           & = e^{2x}(\cos 2x + \sin 2x)            \\                                                                 \\
              y                            & = \cos 2x + \sin 2x +  ce^{-2x}      &
              c                            & = \frac{2}{e^{-\pi/2}}
          \end{align}

          \begin{figure}[H]
              \centering
              \begin{tikzpicture}
                  \begin{axis}[
                          PiStyleX,
                          xtick distance = 0.5*\PI,
                          legend pos = north east,
                          grid = both,
                          width = 12cm,
                          height = 8cm,
                          Ani,
                          view     = {0}{90}, % for a view 'from above'
                      ]
                      \addplot[GraphSmooth, y_h, domain = -0.5:12,
                          restrict y to domain = -10:20]
                      {cos(2*x) + sin(2*x) + (2*e^(pi/2)) * e^(-2*x)};
                      \addplot[GraphSmooth, y_p, domain = -0.5:12,
                          restrict y to domain = -10:20, dotted]
                      {(2*e^(pi/2)) * e^(-2*x)};
                      \node[GraphNode, label={0:{($ \pi/4 $, 3)}}] at (axis cs:pi/4, 3)
                      {};
                      \addlegendentry{$ y = \cos 2x + \sin 2x +  2e^{\pi/2}\ e^{-2x}$};
                      \addlegendentry{$y_{\text{transient}} = 2e^{\pi/2}\ e^{-2x}$};
                  \end{axis}
              \end{tikzpicture}
          \end{figure}

    \item Solving ODE and graphing, using the IC $ y(\pi/4) = 3 $

          \begin{align}
              xy'  - 2y & = x^{3}e^{x}                                \\
              p(x)      & = \frac{-2}{x}                            &
              r(x)      & = x^{2}e^{x}                                \\
              h         & = \int\ p(x)\ \dl x                       &
                        & = \int \frac{-2}{x}\ \dl x                  \\
                        & = -2\ln (x)                                 \\
              y         & = e^{-h}\ \int\ e^{h}r(x)\ \dl x + c        \\
                        & = x^{2} \left[ \int\ x^{-2}\ (x^{2}e^{x})
              \ \dl x + c \right]                                     \\
              y         & = x^{2}e^{x} +  cx^{2}                    &
              c         & = 1
          \end{align}

          \begin{figure}[H]
              \centering
              \begin{tikzpicture}
                  \begin{axis}[
                          legend pos = outer north east,
                          grid = both,
                          width = 8cm,
                          height = 8cm,
                          Ani,
                          view     = {0}{90}, % for a view 'from above'
                      ]
                      \addplot[GraphSmooth, y_h, domain = -2:2,
                          %   restrict y to domain = -10:100
                      ]
                      {x^2 * (1  + e^(x))};
                      \addplot[GraphSmooth, y_p, domain = -2:2,
                          %   restrict y to domain = -10:100,
                          dotted, red]
                      {x^2};
                      \node[GraphNode, label={90:{(0, 0)}}] at (axis cs:0, 0) {};
                      \addlegendentry{$ y = x^{2}e^{x} + x^{2} $};
                      \addlegendentry{$y_{\text{transient}} = x^{2} $};
                  \end{axis}
              \end{tikzpicture}
          \end{figure}

    \item Solving ODE and graphing, using the IC $ y(0) = 0 $

          \begin{align}
              y'  + y\tan x & = e^{-0.01x}\cos x                               \\
              p(x)          & = \tan x                                       &
              r(x)          & = e^{-0.01x}\cos x                               \\
              h             & = \int\ p(x)\ \dl x                            &
                            & = \int\ \tan x\ \dl x                            \\
                            & = \ln |\sec x|                                   \\
              y             & = e^{-h}\ \int\ e^{h}r(x)\ \dl x + c             \\
                            & = \cos x \left[ \int\ \sec x\ \left(
              e^{-0.01x}\cos x \right)\ \dl x + c \right]                      \\
              y             & = -100\ \cos x\ e^{-0.01x} + c\ \cos x         &
              c             & = 100                                            \\
              y             & = -100\ \cos x\ \left( e^{-0.01x}  - 1 \right)
          \end{align}

          \begin{figure}[H]
              \centering
              \begin{tikzpicture}
                  \begin{axis}[
                          %   legend pos = outer north east,
                          enlargelimits = false,
                          grid = both,
                          width = 16cm,
                          height = 8cm,
                          Ani,
                          view     = {0}{90}, % for a view 'from above'
                      ]
                      \addplot[GraphSmooth, y_h, samples = 2000,
                          domain = -10:400,
                          %   restrict y to domain = -10:100
                      ]
                      {-100*cos(x)*(e^(-0.01*x) - 1)};
                      \addplot[GraphSmooth, samples = 2000,
                          domain = -10:400,
                          %   restrict y to domain = -10:100,
                          dotted, y_p]
                      {100*cos(x)};
                      \node[GraphNode, label={90:{(0, 0)}}] at (axis cs:0, 0) {};
                  \end{axis}
              \end{tikzpicture}
          \end{figure}

    \item Solving ODE and graphing, using the IC $ y(0) = -2.5 $

          \begin{align}
              y'  + y\sin x & = e^{\cos x}                            \\
              p(x)          & = \sin x                              &
              r(x)          & = e^{\cos x}                            \\
              h             & = \int\ p(x)\ \dl x                   &
                            & = \int \sin x\ \dl x                    \\
                            & = -\cos x                               \\
              y             & = e^{-h}\ \int\ e^{h}r(x)\ \dl x + c    \\
                            & = e^{\cos x} \left[ \int\ e^{-\cos x}
              \ (e^{\cos x})\ \dl x + c \right]                       \\
              y             & = e^{\cos x} (x + c)                  &
              c             & = \frac{-2.5}{e}
          \end{align}

          \begin{figure}[H]
              \centering
              \begin{tikzpicture}
                  \begin{axis}[
                          legend pos = outer north east,
                          grid = both,
                          width = 8cm,
                          height = 8cm,
                          Ani,
                          view     = {0}{90}, % for a view 'from above'
                      ]
                      \addplot[GraphSmooth, y_h, domain = -2*pi:8*pi,
                      ]
                      {(x - 2.5/e) * e^(cos(x))};
                      \addplot[GraphSmooth, y_p, domain = -2*pi:8*pi,
                          dotted, red]
                      {(-2.5/e) * e^(cos(x))};
                      \node[GraphNode, label={270:{(0, -2.5)}}] at (axis cs:0, -2.5)
                      {};
                      \addlegendentry{$ y = e^{\cos x}
                              \left( x - \frac{2.5}{e} \right) $};
                      \addlegendentry{$y_{\text{transient}}
                              = - \frac{2.5}{e}\ e^{\cos x} $};
                  \end{axis}
              \end{tikzpicture}
          \end{figure}

    \item Solving ODE and graphing, using the IC $ y(\pi/4) = 4/3 $

          \begin{align}
              y'\cos x            & =  (1 - 3y)\sec x                       \\
              y' + 3\sec^{2} x\ y & = \sec^{2}x                             \\
              p(x)                & = 3\sec^{2} x                           \\
              r(x)                & = \sec^{2} x                            \\
              h                   & = \int\ p(x)\ \dl x                     \\
                                  & = 3\int \sec^{2} x\ \dl x
              = 3 \tan x                                                    \\
              y                   & = e^{-h}\ \int\ e^{h}r(x)\ \dl x + c    \\
                                  & = e^{-3\tan x} \left[ \int\ e^{3\tan x}
              \ (\sec ^{2} x)\ \dl x + c \right]                            \\
              \int\ e^{3\tan x}\ (\sec ^{2} x)\ \dl x
                                  & = \int e^{3u}\ \dl u                    \\
                                  & =\frac{e^{3u}}{3}
              = \frac{e^{3\tan x}}{3}                                       \\
              y                   & = \frac{1}{3} + ce^{-3\tan x}           \\
              c                   & = e^{3}
          \end{align}

          \begin{figure}[H]
              \centering
              \begin{tikzpicture}
                  \begin{axis}[
                          PiStyleX,
                          legend pos = north east,
                          grid = both,
                          width = 14cm,
                          height = 8cm,
                          Ani,
                          view     = {0}{90}, % for a view 'from above'
                      ]
                      \addplot[JumpPlot, domain = -2*pi:2*pi,
                          restrict y to domain = -100:100
                      ]
                      {1/3 + e^(3 * (1 - tan(x)))};
                      \addplot[JumpPlot, domain = -2*pi:2*pi,
                          restrict y to domain = -100:100,
                          dotted, red]
                      {e^(3 * (1 - tan(x)))};
                      \node[GraphNode, label={45:{($ \pi/4 $, 4/3)}}]
                      at (axis cs:pi/4, 4/3) {};
                      \addlegendimage{blue,mark size=2}
                      \addlegendimage{red,mark size=2}
                      \addlegendentry{$ y = \frac{1}{3} + e^{3 -3\tan x} $};
                      \addlegendentry{$y_{\text{transient}} = e^{3 -3\tan x} $};
                  \end{axis}
              \end{tikzpicture}
          \end{figure}

    \item Solving ODE and graphing, using the IC $ y(0) = -2.5 $

          \begin{align}
              y'             & = (y - 2)\cot x                            \\
              y' - \cot x\ y & = -2\cot x                                 \\
              p(x)           & = -\cot x                                &
              r(x)           & = -2\cot x                                 \\
              h              & = \int\ p(x)\ \dl x                      &
                             & = -\int \cot x\ \dl x                      \\
                             & = -\ln |\sin x|                            \\
              y              & = e^{-h}\ \int\ e^{h}r(x)\ \dl x + c       \\
                             & = \sin x \left[ \int\ \csc x\ (-2\cot x)
              \ \dl x + c \right]                                         \\
              y              & = 2 + c\sin x                            &
              c              & = 1
          \end{align}

          \begin{figure}[H]
              \centering
              \begin{tikzpicture}
                  \begin{axis}[
                          PiStyleX,
                          legend pos = outer north east,
                          grid = both,
                          width = 8cm,
                          height = 8cm,
                          Ani,
                          view     = {0}{90}, % for a view 'from above'
                      ]
                      \addplot[GraphSmooth, y_h, domain = 0:4*pi,
                      ]
                      {2 + sin(x)};
                      \addplot[GraphSmooth, y_p, domain = 0:4*pi,
                          dotted]
                      {sin(x)};
                      \node[GraphNode, label={0:{($ \pi/2 $, 3)}}]
                      at (axis cs:pi/2, 3) {};
                      \addlegendentry{$ y = 2 + \sin x $};
                      \addlegendentry{$y_{\text{transient}} = \sin x $};
                  \end{axis}
              \end{tikzpicture}
          \end{figure}

    \item Solving ODE and graphing, using the IC $ y(1) = 2 $

          \begin{align}
              xy'  + 4y & = 8x^{4}                                \\
              p(x)      & = 4/x                                 &
              r(x)      & = 8x^{3}                                \\
              h         & = \int\ p(x)\ \dl x                   &
                        & = \int \frac{4}{x}\ \dl x               \\
                        & = 4\ln x                                \\
              y         & = e^{-h}\ \int\ e^{h}r(x)\ \dl x + c    \\
                        & = x^{-4} \left[ \int\ x^{4}\ (8x^{3})
              \ \dl x + c \right]                                 \\
              y         & = cx^{-4} + x^{4}                     &
              c         & = 1
          \end{align}

          \begin{figure}[H]
              \centering
              \begin{tikzpicture}
                  \begin{axis}[
                          legend pos = north east,
                          grid = both,
                          width = 12cm,
                          height = 8cm,
                          Ani,
                          view     = {0}{90}, % for a view 'from above'
                      ]
                      \addplot[GraphSmooth, y_h, domain = 0:4,
                          restrict y to domain = -1000:1000
                      ]
                      {x^4 + x^(-4)};
                      \addplot[GraphSmooth, y_p, domain = 0:4,
                          restrict y to domain = -1000:1000,
                          dotted]
                      {x^(-4)};
                      \node[GraphNode, label={90:{(1, 2)}}] at (axis cs:1, 2) {};
                      \addlegendentry{$ y = x^{-4} + x^{4} $};
                      \addlegendentry{$y_{\text{transient}} = x^{0-4} $};
                  \end{axis}
              \end{tikzpicture}
          \end{figure}

    \item Solving ODE by separation of variables and graphing

          \begin{align}
              y'                             & = (6y - 15)\tanh (1.5x)          \\
              \int\ \frac{1}{6y - 15}\ \dl y & =  \int\ \tanh(1.5x)\ \dl x      \\
              \frac{1}{6}\ \ln \left( \frac{1}{6y - 15} \right)
                                             & = \frac{2}{3}\ \ln[\cosh (1.5x)] \\
              6y - 15                        & = b[\cosh (1.5x)]^{-4}           \\
              y                              & = 2.5 + c\ [\cosh(1.5x)]^{4}
          \end{align}

          \begin{figure}[H]
              \centering
              \begin{tikzpicture}
                  \begin{axis}[
                          % ymode = log,
                          legend pos = north west,
                          grid = both,
                          width = 12cm,
                          height = 8cm,
                          Ani,
                          view     = {0}{90}, % for a view 'from above'
                      ]
                      \addplot[GraphSmooth, y_h, domain = 0:2,]
                      {2.5 + (cosh(1.5*x))^(4)};
                      \node[GraphNode, label={90:{(0, 3.5)}}] at (axis cs:0, 3.5) {};
                      \addlegendentry{$ y = 2.5 + [\cosh(1.5x)]^{4} $};
                  \end{axis}
              \end{tikzpicture}
          \end{figure}

    \item Solving,
          \begin{enumerate}
              \item Solving ODE

                    \begin{align}
                        y' - \frac{y}{x} & = \frac{-1}{x}\ \cos(1/x)              \\
                        p(x)             & = -1/x                               &
                        r(x)             & = \frac{-1}{x}\ \cos(1/x)              \\
                        h                & = \int\ p(x)\ \dl x                  &
                                         & = \int \frac{-1}{x}\ \dl x             \\
                                         & = -\ln x                               \\
                        y                & = e^{-h}\ \int\ e^{h}r(x)\ \dl x + c   \\
                                         & = x \left[ \int\ \frac{-1}{x^{2}}
                        \ \cos(1/x)\ \dl x + c \right]                            \\
                        I                & =\int\ \frac{-1}{x^{2}}\ \cos(1/x)
                        \ \dl x                                                   \\
                                         & =\int\ \cos(u)\ du                   &
                        u                & = \frac{1}{x}\quad du
                        = \frac{-1}{x^{2}}dx                                      \\
                                         & = \sin u = \sin(1/x)                   \\
                        y                & = cx + x\sin(1/x)
                    \end{align}

                    For $ c = 0 $, an IC is $ (2/\pi, 2/\pi) $
                    \begin{figure}[H]
                        \centering
                        \begin{tikzpicture}
                            \begin{axis}[
                                    legend pos = north west,
                                    grid = both,
                                    width = 12cm,
                                    height = 8cm,
                                    Ani,
                                    view     = {0}{90}, % for a view 'from above'
                                ]
                                \addplot[GraphSmooth, y_h, domain = -3:3, samples = 200,
                                    %   restrict y to domain = -10:100
                                ]
                                {x*sin(1/x)};
                                \node[GraphNode, label={-45:{($ 2/\pi $, $ 2/\pi $)}}]
                                at (axis cs:2/pi, 2/pi) {};
                                \addlegendentry{$ y = x\ \sin(1/x) $};
                            \end{axis}
                            ymode = log,
                        \end{tikzpicture}
                    \end{figure}
                    \begin{figure}[H]
                        \centering
                        \begin{tikzpicture}
                            \begin{axis}[
                                    title = zoomed in,
                                    legend pos = north west,
                                    grid = both,
                                    width = 12cm,
                                    height = 8cm,
                                    Ani,
                                    view     = {0}{90}, % for a view 'from above'
                                ]
                                \addplot[GraphSmooth, y_h, domain = 0.001:0.05,
                                    samples = 1000]
                                {x*sin(1/x)};
                                \addplot[GraphSmooth, y_h, domain = -0.05:-0.001,
                                    samples = 1000]
                                {x*sin(1/x)};
                                \addlegendentry{$ y = x\ \sin(1/x) $};
                            \end{axis}
                        \end{tikzpicture}
                    \end{figure}

              \item Solving ODE for general $ n $

                    \begin{align}
                        y' - \frac{ny}{x} & = -x^{n-2}\ \cos(1/x)                  \\
                        p(x)              & = -n/x                               &
                        r(x)              & = -x^{n-2}\ \cos(1/x)                  \\
                        h                 & = \int\ p(x)\ \dl x                  &
                                          & = \int \frac{-n}{x}\ \dl x             \\
                                          & = -n\ln x                              \\
                        y                 & = e^{-h}\ \int\ e^{h}r(x)\ \dl x + c   \\
                                          & = x^{n} \left[ \int\ -x^{-2}
                        \ \cos(1/x)\ \dl x + c \right]                             \\
                        I                 & =\int\ \frac{-1}{x^{2}}\ \cos(1/x)
                        \ \dl x                                                    \\
                                          & =\int\ \cos(u)\ du                   &
                        u                 & = \frac{1}{x}\quad du
                        = \frac{-1}{x^{2}}dx                                       \\
                                          & = \sin u = \sin(1/x)                   \\
                        y                 & = cx^{n} + x^{n}\sin(1/x)
                    \end{align}

                    The graph is untractable close to $ x = 0 $. For large positive
                    powers, the expression is dominated by the polynomial term. TBC.
          \end{enumerate}

    \item $ y_{1} $ and $ y_{2} $ are solutions of the homogenous ODE, then

          \begin{align}
              \text{Let} \quad y_{1}' + py_{1}   & = 0            &
              y_{2}' + py_{2}                    & = 0              \\
              y_{1}' + y_{2}' + (y_{1} + y_{2})p & = 0            &
                                                 & \text{summing}   \\
              y_{3}' + py_{3}                    & = 0            &
              y_{1} + y_{2}                      & = y_{3}
          \end{align}

          $ y_{1} + y_{2} $ is also a solution to the homogenous ODE. This does not
          hold true for the nonhomogenous ODE as the RHS is $ 2r $ instead of $ r $.

          \begin{align}
              \text{Let} \quad y_{1}' + py_{1}   & = r            &
              y_{2}' + py_{2}                    & = r              \\
              y_{1}' + y_{2}' + (y_{1} + y_{2})p & = 2r           &
                                                 & \text{summing}   \\
              y_{3}' + py_{3}                    & \neq r         &
              y_{1} + y_{2}                      & = y_{3}
          \end{align}


          For a scalar multiple $ a $, $ y_{4} $ is also a solution to the homogenous
          ODE. This does not hold for the inhomogenous ODE.

          \begin{align}
              \text{Let} \quad y_{1}' + py_{1} & = 0      &
              ay_{1}' + apy_{1}                & = 0        \\
              y_{4}' + py_{4}                  & = 0      &
              y_{4}                            & = ay_{1}   \\
              \text{Let} \quad y_{1}' + py_{1} & = r      &
              ay_{1}' + apy_{1}                & = ar       \\
              y_{4}' + py_{4}                  & \neq r   &
              y_{4}                            & = ay_{1}
          \end{align}


    \item The trivial solution $ y_{t}(x) \equiv 0\ \forall\ x$ is a solution of every homogenous
          ODE, but not every nonhomogenous ODE.

          \begin{align}
              \text{Let} \quad y_{1}' + py_{1} & = 0    \\
              y_{t}' + py_{t}                  & = 0    \\
              0                                & = 0    \\
              \text{Let} \quad y_{1}' + py_{1} & = r    \\
              y_{t}' + py_{t}                  & = r    \\
              0                                & \neq r
          \end{align}


    \item Let $ y_{n} $ and $ y_{h} $ solve the nonhomogenous and homogenous ODE
          respectively.

          \begin{align}
              \text{Let} \quad y_{h}' + py_{h}   & = 0     &
              y_{n}' + py_{n}                    & = r       \\
              y_{h}' + y_{n}' + p(y_{h} + y_{n}) & = 0 + r   \\
              y_{c}' + py_{c}                    & = r     &
              y_{h} + y_{n}                      & = y_{c}
          \end{align}

          Their sum is also a solution to the nonhomogenous ODE.

    \item Let $ y_{1} $ and $ y_{2} $ solve the nonhomogenous ODE,

          \begin{align}
              \text{Let} \quad y_{1}' + py_{1}   & = r               &
              y_{2}' + py_{2}                    & = r                 \\
              y_{1}' - y_{2}' + (y_{1} - y_{2})p & = 0               &
                                                 & \text{difference}   \\
              y_{3}' + py_{3}                    & = 0               &
              y_{1} - y_{2}                      & = y_{3}
          \end{align}

          Their difference solves the homogenous ODE.

    \item Let $ y_{1} $ solve the nonhomogenous ODE and $ y_{2} = cy_{1} $,

          \begin{align}
              \text{Let} \quad y_{1}' + py_{1} & = r  &
              cy_{1}' + cpy_{1}                & = cr   \\
              y_{2}' + py_{2}                  & = cr
          \end{align}

          $ cy_{1} $ solves a different nonhomogenous ODE with RHS being $ cr $.

    \item Both ODEs have the same $ p $ and differ in their RHS.

          \begin{align}
              \text{Let} \quad y_{1}' + py_{1}   & = r_{1}         &
              y_{2}' + py_{2}                    & = r_{2}           \\
              y_{1}' + y_{2}' + (y_{1} + y_{2})p & = r_{1} + r_{2} &
                                                 & \text{summing}    \\
              y_{3}' + py_{3}                    & = r_{3}         &
              r_{1} + r_{2}                      & = r_{3}
          \end{align}

          $ y_{1} + y_{2} $ solves the ODE with the same $ p $ and RHS $ r_{1} + r_{2} $.

    \item Method of variation,

          \begin{align}
              y               & = c\ \exp\left( -\int\ p(x)\ \dl x \right)= cy_{*} \\
              y_{*} \quad \text{solves the ODE} \quad y' + py
                              & = 0                                                \\
              y_{*}' + py_{*} & = 0
          \end{align}

          $ uy_{*} $ is a solution of the nonhomogenous ODE $ y' +py = r $,

          \begin{align}
              y  & = u(x)\ \exp\left( -\int\ p(x)\ \dl x \right)        &
                 & = u(x)\ e^{-h}                                         \\
              r  & = u'y_{*} + uy_{*}' + p(uy_{*})                      &
              r  & = u'y_{*} + u (y_{*}' + py_{*})                        \\
              u' & = \frac{r}{y_{*}}                                    &
              h  & = \int\ p(x)\ \dl x                                    \\
              u' & = re^{h}                                               \\
              u  & = \int\ e^{h}\ r(x)\ \dl x + c                         \\
              y  & = u(x)\ e^{-h}                                       &
                 & = e^{-h} \left[ \int\ e^{h}\ r(x)\ \dl x + c \right]
          \end{align}

          which matches the earlier result.

    \item Solving ODE by separation of variables and graphing, given the IC $ y(0) = -1/3 $

          \begin{align}
              y' + y                          & = y^{2}                          \\
              \int\ \frac{1}{y(y - 1)}\ \dl y & =  \int\ \dl x                   \\
              \ln y - \ln (y-1)               & = x + b                          \\
              \frac{y}{y-1}                   & = ce^{x}                         \\
              y                               & = \frac{1}{1 - ce^{-x}} & c & =4
          \end{align}

          \begin{figure}[H]
              \centering
              \begin{tikzpicture}
                  \begin{axis}[
                          legend pos = north east,
                          grid = both,
                          width = 14cm,
                          height = 8cm,
                          Ani,
                          view     = {0}{90}, % for a view 'from above'
                      ]
                      \addplot[GraphSmooth, y_h, domain = -4:6,
                          restrict y to domain = -100:100
                      ]
                      {1/(1 - 4*e^(-x))};
                      \node[GraphNode, label={90:{(0, -1/3)}}] at (axis cs:0, -1/3) {};
                      \addlegendentry{$ y = \frac{1}{1 - 4e^{-x}} $};
                  \end{axis}
              \end{tikzpicture}
          \end{figure}

    \item Solving ODE by separation of variables and graphing, given the IC $ y(0) = 3 $

          \begin{align}
              y' + xy         & = xy^{-1}                           \\
              \int\ \frac{y}{(1 - y^{2})}\ \dl y
                              & =  \int\ x\ \dl x                   \\
              \ln (1 - y^{2}) & = -x^{2} + b                        \\
              (1 - y^{2})     & = ce^{-x^{2}}                       \\
              y               & = \sqrt{1 - ce^{-x^{2}}} & c & = -8
          \end{align}

          \begin{figure}[H]
              \centering
              \begin{tikzpicture}
                  \begin{axis}[
                          legend pos = north east,
                          grid = both,
                          width = 14cm,
                          height = 8cm,
                          Ani,
                          view     = {0}{90}, % for a view 'from above'
                      ]
                      \addplot[GraphSmooth, y_h, domain = -5:5,
                          restrict y to domain = -100:100
                      ]
                      {(1 + 8*e^(-x*x))^(1/2)};
                      \node[GraphNode, label={0:{(0, 3)}}] at (axis cs:0, 3) {};
                      \addlegendentry{$ y = \sqrt{1 +8e^{-x^{2}}} $};
                  \end{axis}
              \end{tikzpicture}
          \end{figure}

    \item Solving ODE and graphing,

          \begin{align}
              y' + y       & = -xy^{-1}                                  \\
              y' +py       & = gy^{a}                                  &
              a            & = -1                                        \\
              p(x)         & = 1                                       &
              g(x)         & = -x                                        \\
              u' + (1-a)pu & = (1-a)g                                  &
              u            & = y^{(1-a)}                                 \\
              u' + 2pu     & = 2g                                        \\
              h            & = \int\ 2p(x)\ \dl x                      &
                           & = \int 2\ \dl x                             \\
                           & = 2x                                        \\
              y            & = e^{-h}\ \int\ e^{h}\{2g(x)\}\ \dl x + c   \\
                           & = e^{-2x} \left[ \int\ e^{2x}\ \{-2x\}
              \ \dl x + c \right]                                        \\
              I            & =\int\ e^{2x}\ \{2x\}\ \dl x                \\
                           & = xe^{2x} - \int\ e^{2x}\ \dl x             \\
                           & = \left( x - \frac{1}{2} \right)e^{2x}      \\
                           & = \sin u = \sin(1/x)                        \\
              y            & = \frac{1}{2} - x + ce^{-2x}              &
              c            & = 1
          \end{align}

          \begin{figure}[H]
              \centering
              \begin{tikzpicture}
                  \begin{axis}[
                          legend pos = north east,
                          grid = both,
                          width = 14cm,
                          height = 8cm,
                          Ani,
                          view     = {0}{90}, % for a view 'from above'
                      ]
                      \addplot[GraphSmooth, y_h, domain = -1:5]
                      {0.5 - x + e^(-2*x)};
                      \addplot[GraphSmooth, y_p, domain = -1:5, dotted]
                      {e^(-2*x)};
                      \node[GraphNode, label={45:{(0, 1.5)}}] at (axis cs:0, 1.5) {};
                      \addlegendentry{$ y = 1/2 - x + e^{-2x} $};
                      \addlegendentry{$ y_{\text{transient}} = e^{-2x} $};
                  \end{axis}
              \end{tikzpicture}
          \end{figure}

    \item Solving ODE by separation and graphing,

          \begin{align}
              y'                  & = 3.2y - 10y^{2}                        \\
              \int\ \frac{1}{y(3.2 - 10y)}\ \dl y
                                  & =  \int\ \dl x                          \\
              3.2A - 10Ay + By    & = 1                                   &
              A                   & = \frac{1}{3.2},\ B = \frac{10}{3.2}    \\
              \ln \left( \frac{y}{3.2 - 10y} \right)
                                  & = 3.2x + b                              \\
              \frac{y}{3.2 - 10y} & = ce^{3.2x}                             \\
              y                   & = \frac{3.2e^{3.2x}}{1 + 10ce^{3.2x}} &
              c                   & = 0.32
          \end{align}

          \begin{figure}[H]
              \centering
              \begin{tikzpicture}
                  \begin{axis}[
                          legend pos = north west,
                          grid = both,
                          width = 14cm,
                          height = 8cm,
                          Ani,
                          view     = {0}{90}, % for a view 'from above'
                      ]
                      \addplot[GraphSmooth, y_h, domain = -3:3,
                          %   restrict y to domain = -100:100
                      ]
                      {1/(1 + (5/16)*e^(-3.2*x))};
                      \node[GraphNode, label={-45:{(0, 32/42)}}]
                      at (axis cs:0, 32/42) {};
                      \addlegendentry{$ y = \left[ 1 + (5/16)e^{-3.2x} \right]^{-1} $};
                  \end{axis}
              \end{tikzpicture}
          \end{figure}

    \item Solving ODE by separation and graphing,

          \begin{align}
              y'                  & = \frac{\tan y}{x-1}                     \\
              \int\ \cot y\ \dl y & =  \int\ \frac{1}{x-1}\ \dl x            \\
              \ln |\sin y|        & = \ln (x-1) + b                          \\
              y                   & = \arcsin [c \cdot (x-1)]     & c & = -1
          \end{align}

          \begin{figure}[H]
              \centering
              \begin{tikzpicture}
                  \begin{axis}[
                          PiStyleY,
                          axis equal,
                          legend pos = outer north east,
                          grid = both,
                          width = 8cm,
                          height = 8cm,
                          Ani,
                          view     = {0}{90}, % for a view 'from above'
                      ]
                      \addplot[GraphSmooth, y_h, domain = 0:2,
                          %   restrict y to domain = -100:100
                      ]
                      {asin(1-x)};
                      \node[GraphNode, label={-0:{(0, $ \pi/2 $)}}]
                      at (axis cs:0, pi/2) {};
                      \addlegendentry{$ y = \arcsin(1-x) $};
                  \end{axis}
              \end{tikzpicture}
          \end{figure}

    \item Solving ODE by separation and graphing, setting $ c = 1 $,

          \begin{align}
              y'            & = \frac{1}{6e^{y} - 2x}                     \\
              \frac{dx}{dy} & = 6e^{y} - 2x                               \\
              y' + 2y       & = 6e^{x}                                  &
              x             & \rightleftharpoons y                        \\
              p(x)          & = 2                                       &
              r(x)          & = 6e^{x}                                    \\
              h             & = \int\ p(x)\ \dl x                       &
                            & = \int 2\ \dl x                             \\
                            & = 2x                                        \\
              y             & = e^{-h}\ \int\ e^{h}\ r(x)\ \dl x + c      \\
                            & = e^{-2x} \left[ \int\ e^{2x}\ \{6e^{x}\}
              \ \dl x + c \right]                                         \\
              y             & =  2e^{x} + ce^{-2x}                        \\
              x             & = 2e^{y} + ce^{-2y}                       &
              y             & \rightleftharpoons x
          \end{align}

          \begin{figure}[H]
              \centering
              \begin{tikzpicture}
                  \begin{axis}[
                          xmode = log,
                          legend pos = north east,
                          grid = both,
                          width = 12cm,
                          height = 8cm,
                          Ani,
                          view     = {0}{90}, % for a view 'from above'
                      ]
                      \addplot[GraphSmooth, y_h, domain = -5:5, variable = \t,
                          %   restrict y to domain = -100:100
                      ]
                      ({2*e^(t) + e^(-2*t)}, {t});
                      \node[GraphNode, label={0:{(3, 0)}}] at (axis cs:3,0) {};
                      \addlegendentry{$ x = 2e^{y} + e^{-2y} $};
                  \end{axis}
              \end{tikzpicture}
          \end{figure}
          \begin{figure}[H]
              \centering
              \begin{tikzpicture}
                  \begin{axis}[
                          legend pos = north east,
                          grid = both,
                          width = 12cm,
                          height = 8cm,
                          Ani,
                          view     = {0}{90}, % for a view 'from above'
                      ]
                      \addplot[GraphSmooth, y_h, domain = -5:5, variable = \t,
                          %   restrict y to domain = -100:100
                      ]
                      ({2*e^(t) + e^(-2*t)}, {t});
                      \node[GraphNode, label={0:{(3, 0)}}] at (axis cs:3,0) {};
                      \addlegendentry{$ x = 2e^{y} + e^{-2y} $};
                  \end{axis}
              \end{tikzpicture}
          \end{figure}

    \item Solving ODE by separation and graphing,

          \begin{align}
              2xyy' + (x-1)y^{2}         & = x^{2}e^{x}                             \\
              y^{2}                      & = z                                    &
              2yy'                       & = z'                                     \\
              z' + \frac{x-1}{x}\ z      & = xe^{x}                                 \\
              p(x)                       & = 1 - 1/x                              &
              r(x)                       & = xe^{x}                                 \\
              h                          & = \int\ p(x)\ \dl x                    &
                                         & = \int (1 - 1/x)\ \dl x                  \\
                                         & = x - \ln x                              \\
              z                          & = e^{-h}\ \int\ e^{h}\ r(x)\ \dl x + c   \\
                                         & = xe^{-x} \left[ \int\ \frac{e^{x}}{x}
              \ \{xe^{x}\}\ \dl x + c \right]                                       \\
              z                          & =  cxe^{-x} + \frac{xe^{x}}{2}           \\
              y                          & = \sqrt{\frac{xe^{x}}{2}
              \left( 1 + e^{-2x}\right)} &
              c                          & =1
          \end{align}

          \begin{figure}[H]
              \centering
              \begin{tikzpicture}
                  \begin{axis}[
                          %   ymode = log,
                          %   axis equal,
                          legend pos = north west,
                          grid = both,
                          width = 12cm,
                          height = 8cm,
                          Ani,
                          view     = {0}{90}, % for a view 'from above'
                      ]
                      \addplot[GraphSmooth, y_h, domain = 0:3,
                          %   restrict y to domain = -100:100
                      ]
                      {(x*e^(x) * 0.5*(1 + e^(-2*x)))^(1/2)};
                      \node[GraphNode, label={0:{(0, 0)}}] at (axis cs:0, 0) {};
                      \addlegendentry{$ y = [(1/2)xe^{x} \left( 1 + e^{-2x}\right)]^{1/2} $};
                  \end{axis}
              \end{tikzpicture}
          \end{figure}

    \item Examples are in the exercises above.
          \begin{enumerate}
              \item Separable ODEs are of the form,

                    \begin{align}
                        \frac{dy}{dx}     & = \frac{g(x)}{f(y)} \\
                        \int\ f(y)\ \dl y & = \int\ g(x)\ \dl x
                    \end{align}

              \item Exact ODEs are of the form,

                    \begin{align}
                        M(x, y)\ \dl x + N(x, y)\ \dl y & = 0            \\
                        \diffp{M}{y}                    & = \diffp{N}{x}
                    \end{align}

                    They are useful when separation of variables is not possible.
              \item Linear ODEs are of the form,

                    \begin{align}
                        y' + p(x)y & = r(x)                                 \\
                        y          & = e^{-h}\ \int\ e^{h}\ r(x)\ \dl x + c \\
                        h          & = \int\ p(x)\ \dl x
                    \end{align}

                    This form helps separate the transient response (term with $ c $)
                    from the steady state response (term without $ c $).
          \end{enumerate}

    \item Solving,
          \begin{enumerate}
              \item Riccati's equation, (with $ p, g, h $ being functions of $ x $)

                    \begin{align}
                        y' + py           & = gy^{2} + h                          \\
                        Y' +pY            & = gY^{2} + h                          \\
                        Y                 & \ \text{solves the ODE}               \\
                        \left( Y + \frac{1}{u} \right)' + p\left( Y
                        + \frac{1}{u} \right)
                                          & = g\left( Y + \frac{1}{u} \right)^{2}
                        + h                                                       \\
                        y                 & = Y + \frac{1}{u}                     \\
                        Y' + pY -\frac{u'}{u^{2}} + \frac{p}{u}
                                          & = gY^{2} + h + \frac{g}{u^{2}}
                        + \frac{2Yg}{u}                                           \\
                        u' - pu           & = -g - 2Ygu                           \\
                        u' + (2Yg - p)\ u & = -g
                    \end{align}


              \item $ Y = x $ is a solution of the ODE,

                    \begin{align}
                        y' - (2x^{3} + 1)y & = -x^{2}y^{2}-x^{4} -x +1   \\
                        y' + py            & = gy^{2} + h                \\
                        p(x)               & = -(2x^{3} + 1)           &
                        g(x)               & = -x^{2}                    \\
                        h(x)               & = 1 - x - x^{4}
                    \end{align}

                    Checking if the given $ Y $ is a solution,

                    \begin{align}
                        1 - (2x^{3} + 1)x + (x^{4} + x -1) + x^{4} & = 0
                    \end{align}

                    Solving the Riccati equation,

                    \begin{align}
                        u' + (2Yg - p)\ u & = -g                                  \\
                        u' + (-2x^{3} + 2x^{3} + 1)\ u
                                          & = x^{2}                               \\
                        u' + u            & = x^{2}                               \\
                        u                 & = e^{-x}\ \left[ \int\ e^{x}\ x^{2}
                        \ \dl x + c \right]                                       \\
                        I                 & = \int\ e^{x}\ x^{2}\ \dl x           \\
                                          & = x^{2}e^{x} - \int\ 2x e^{x}\ \dl x  \\
                                          & = x^{2}e^{x} - 2xe^{x} + \int\ 2e^{x}
                        \ \dl x                                                   \\
                                          & = (x^{2} - 2x + 2)e^{x}               \\
                        u                 & = (x^{2} - 2x + 2) + ce^{-x}          \\
                                          & = \frac{1}{y - x}                     \\
                        y                 & = x + \frac{1}{(x^{2} - 2x + 2)
                        + ce^{-x}}        &
                        c                 & = 1
                    \end{align}


                    This graph passes through $ (0, 1/3) $.
                    \begin{figure}[H]
                        \centering
                        \begin{tikzpicture}
                            \begin{axis}[
                                    %   ymode = log,
                                    %   axis equal,
                                    legend pos = north west,
                                    grid = both,
                                    width = 12cm,
                                    height = 8cm,
                                    Ani,
                                    view     = {0}{90}, % for a view 'from above'
                                ]
                                \addplot[GraphSmooth, y_h, domain = -1:4,
                                    %   restrict y to domain = -100:100
                                ]
                                {x + (e^(-x) + (x^2 - 2*x + 2))^(-1)};
                                \node[GraphNode, label={0:{(0, 1/3)}}]
                                at (axis cs:0, 1/3) {};
                                \addlegendentry{$ y = x + [(x^{2} - 2x + 2)
                                    + e^{-x}]^{-1} $};
                            \end{axis}
                        \end{tikzpicture}
                    \end{figure}

              \item Clairaut equation,

                    \begin{align}
                        y'^{2}- xy' + y       & = 0                      \\
                        2y'y'' -xy'' - y' +y' & = 0                    &
                                              & \text{differentiating}   \\
                        y''\ (2y' - x)        & = 0                      \\
                        y''                   & = 0                    &
                        y                     & = ax + b                 \\
                        y'                    & = x/2                  &
                        y                     & = x^{2}/4 + c            \\
                        Y                     & = ax + b                 \\
                        Y'^{2}- xY' + Y       & =a^{2} - ax + ax + b   &
                                              & = 0                      \\
                        -a^{2}                & = b                    &
                        c                     & = 0
                    \end{align}

                    General solution is $ Y = ax - a^{2} $. A singular solution
                    $ y = x^{2}/4 $ which
                    cannot be obtained from the general solution also exists.
                    \begin{figure}[H]
                        \centering
                        \begin{tikzpicture}
                            \begin{axis}[
                                    %   ymode = log,
                                    %   axis equal,
                                    colormap/jet,
                                    cycle list = {[of colormap]},
                                    restrict y to domain = -20:40,
                                    legend pos = south west,
                                    grid = both,
                                    width = 12cm,
                                    height = 8cm,
                                    Ani,
                                    view     = {0}{90}, % for a view 'from above'
                                ]
                                \addplot[GraphSmooth, black, domain = -15:15]
                                {0.25*x^2};
                                \addlegendentry{Envelope $ y = x^{2}/4 $};
                                \foreach [evaluate=\i as \n using (\i + 4)*100/(8)]
                                \i in {-4,-3.75,...,4} {%
                                        \edef\temp{%
                                            \noexpand
                                            \addplot[forget plot,
                                                domain=-15:15,
                                                color=red!\n!blue,
                                            ]
                                            {\i * (x - \i)};
                                        }\temp
                                    }
                            \end{axis}
                        \end{tikzpicture}
                    \end{figure}

              \item The general Clairaut equation is,

                    \begin{align}
                        y           & = xy' + g(y')     \\
                        Y           & = cx +g(c)      &
                        Y'          & = c               \\
                        xY' + g(Y') & = xc + g(c) = Y
                    \end{align}

                    This straight line family $ Y = xc + g(c) $ is thus a solution to the ODE.

                    \begin{align}
                        g'(y') & = -x                   & g(y') & = \frac{-x^{2}}{2} \\
                        y'     & = xy'' + y' +g'(y')y'' & 0     & = xy'' + g'(y')y'' \\
                        -x     & = g'(y')
                    \end{align}

                    $g'(y') = -x$ is thus a solution to the ODE.
          \end{enumerate}

    \item Newton's law of cooling, with temperature $ y $ and time $ t $,

          \begin{align}
              y'              & = k(y - y_{a})                     \\
              \ln (y - y_{a}) & = kt + b                         &
              y               & = y_{a} + ce^{kt}                  \\
              y(0)            & = 300                            &
              c               & = 240                              \\
              y(10)           & = 200                            &
              k               & = \frac{1}{10}\ \ln (7/12)         \\
              y(t^{*})        & = 61                             &
              t^{*}           & = 10\ \frac{\ln(240)}{\ln(12/7)}   \\
              t^{*}           & = \SI{101.68}{\min}
          \end{align}

          \begin{figure}[H]
              \centering
              \begin{tikzpicture}
                  \begin{axis}[
                          xlabel = Time ($ t $),
                          ylabel = Temperature ($ y $),
                          legend pos = north east,
                          grid = both,
                          width = 12cm,
                          height = 8cm,
                          Ani,
                          view     = {0}{90}, % for a view 'from above'
                      ]
                      \addplot[GraphSmooth, y_h, domain = 0:150,
                          %   restrict y to domain = -100:100
                      ]
                      {60 + 240*e^(-0.054*x)};
                      \node[GraphNode, label={45:{(101, 61)}}] at (axis cs:101, 61) {};
                      \addlegendentry{$ y = y_{a} + ce^{-kt} $};
                  \end{axis}
              \end{tikzpicture}
          \end{figure}

    \item Newton's law of cooling, with temperature $ y $ and time $ x $,

          \begin{align}
              y'   & = k_{1}(y - y_{a}) + k_{2}(y - y_{\omega}) + P             \\
              y' - y\ (k_{1} + k_{2})
                   & = P - k_{1}y_{a} - k_{2}y_{\omega}                         \\
                   & = [ P - k_{2}y_{\omega} -k_{1}A ] + k_{1}C \cos(\lambda x) \\
              p(x) & = -m                                                       \\
              r(x) & = a + b\cos(\lambda x)                                     \\
              h    & = \int\ p(x)\ \dl x                                        \\
                   & = \int -m\ \dl x                                           \\
                   & = -mx                                                      \\
              y    & = e^{-h}\ \int\ e^{h}\ r(x)\ \dl x + c                     \\
                   & = e^{mx} \left[ \int\ e^{-mx}
              \ \left\{ a + b\cos (\lambda x)  \right\}\ \dl x + c \right]      \\
              y    & =  ce^{mx} -\frac{a}{m} +
              \left[ \frac{b}{m^{2}+\lambda^{2}} \right]
              \left( -m \cos (\lambda x) + \lambda \sin (\lambda x)\right)      \\
                   & = ce^{(k_{1} + k_{2})x}
              - \frac{P - k_{1}y_{a} - k_{2}y_{\omega}}{k_{1} + k_{2}}          \\
                   & + \frac{k_{1}C}{(k_{1}
                  + k_{2})^{2} + \lambda^{2}}\ \left[ -(k_{1} + k_{2})
              \cos (\lambda x) + \lambda \sin (\lambda x) \right]               \\
              y    & = ce^{-\mu x} + \nu + \alpha \cos (\lambda x)
              + \beta \sin(\lambda x)
          \end{align}

          The solution contains a transient term (dependent on the IC), a constant
          term to vertically shift the solution,
          and the sinusoidal terms with their own respective coefficients.
          The two sinusoidal terms, share the same time period.
          \begin{figure}[H]
              \centering
              \begin{tikzpicture}
                  \begin{axis}[
                          xlabel = Time ($ x $),
                          ylabel = Temperature ($ y $),
                          legend pos = north east,
                          grid = both,
                          width = 12cm,
                          height = 8cm,
                          Ani,
                          view     = {0}{90}, % for a view 'from above'
                      ]
                      \addplot[GraphSmooth, y_h, domain = -3:40, ]
                      {1 + e^(-x) - cos(x) + sin(x)};
                      \addplot[GraphSmooth, y_p, domain = -3:40, dotted, ]
                      {e^(-x)};
                      \addlegendentry{$ y = e^{-x} + 1 - \cos (\lambda x)
                              + \sin(\lambda x) $};
                      \addlegendentry{$ y_{\text{transient}} = e^{-x}$};
                  \end{axis}
              \end{tikzpicture}
          \end{figure}

    \item Drug injection model, with drug quantity $ y $ and time $ x $,

          \begin{align}
              y'           & = A - ky                                     \\
              \ln (A - ky) & = -kx + b                                    \\
              y            & = \frac{A}{k} - ce^{-kx}                     \\
              y(0)         & = 0                      & c & = \frac{A}{k} \\
              A            & = 3                      & k & = 1
          \end{align}

          \begin{figure}[H]
              \centering
              \begin{tikzpicture}
                  \begin{axis}[
                          xlabel = Time ($ t $),
                          ylabel = Drug Quantity ($ y $),
                          legend pos = north east,
                          grid = both,
                          width = 12cm,
                          height = 8cm,
                          Ani,
                          view     = {0}{90}, % for a view 'from above'
                      ]
                      \addplot[GraphSmooth, y_h, domain = 0:7,]
                      {3 + 2*e^(-x)} node[pos = 0.75,above]{$ y_{0} = 5 $};
                      \addplot[GraphSmooth, y_p, domain = 0:7]
                      {3 - 3*e^(-x)} node[pos = 0.25, above left]{$ y_{0} = 0 $};
                      \addlegendentry{$ y = 3 + 2e^{-x}$};
                      \addlegendentry{$ y = 3 - 3e^{-x}$};
                  \end{axis}
              \end{tikzpicture}
          \end{figure}

    \item Epidemic model. Let the fraction diseased be $ y $, and the healthy fraction be
          $ 1-y $. Every diseased individual has contact with every healthy individual.

          \begin{align}
              y'                            & = Ky(1-y)               \\
              \int\ \frac{1}{y(1-y)}\ \dl y & = K \int\ \dl x         \\
              \ln y - \ln (y-1)             & = Kx + b                \\
              \frac{y}{y-1}                 & =     ce^{Kx}           \\
              y                             & = \frac{1}{1 - c^{-Kx}}
          \end{align}

          \begin{figure}[H]
              \centering
              \begin{tikzpicture}
                  \begin{axis}[
                          xlabel = Time ($ x $),
                          ylabel = Infected fraction ($ y $),
                          legend pos = south east,
                          grid = both,
                          width = 12cm,
                          height = 8cm,
                          Ani,
                          view     = {0}{90}, % for a view 'from above'
                      ]
                      \addplot[GraphSmooth, y_h, domain = 0:15]
                      {(1   + (2/3)*e^(-x))^(-1)} node[pos = 0,below]{$ y_{0} = 0.6 $};
                      \addplot[GraphSmooth, y_p, domain = 0:15]
                      {(1  + 9*e^(-x))^(-1)} node[pos = 0.25,right]{$ y_{0} = 0.1 $};
                      \addplot[GraphSmooth, y_t, domain = 0:15]
                      {(1  + 999*e^(-x))^(-1)} node[pos = 0.5, right]{$ y_{0} = 0.001 $};
                      \addlegendentry{$ y = 1 + 0.67e^{-x}$};
                      \addlegendentry{$ y = 1 + 9e^{-x}$};
                      \addlegendentry{$ y = 1  + 999e^{-x}$};
                  \end{axis}
              \end{tikzpicture}
          \end{figure}

    \item Water inflow and outflow rate is $ \lambda $, and total volume of
          the lake is $ V $, with pollutant concentration $ y $ and time $ x $,

          \begin{align}
              y'       & = \frac{\lambda}{V} \left( \frac{p}{4} - y \right)   \\
              y        & = \frac{p}{4} + ce^{-\lambda x/V}                    \\
              y(0)     & = \frac{p}{4} + c = p                              &
              c        & = \frac{3p}{4}                                       \\
              y        & = \frac{p}{4} \left( 1 + 3e^{-\lambda x/V} \right)   \\
              y(t^{*}) & = \frac{p}{2}                                      &
              t^{*}    & = \frac{\ln 3}{\lambda / V}
          \end{align}

          Time needed is $ t^{*} = 2.825 $ years.

    \item Schaefer model, based on logistic equation, with quantity of fish $ y $,
          time $ x $,

          \begin{align}
              y'                               & = (A-H)y - By^{2}                &
              H                                & < A                                \\
              \int \frac{\lambda}{y} + \frac{\mu}{A-H -By}\ \dl y
                                               & = \int\ \dl x                      \\
              \lambda(A-H) + y(\mu - B\lambda) & = 1                                \\
              \lambda                          & = \frac{1}{A-H}                  &
              \mu                              & = \frac{B}{A-H}                    \\
              \ln y - \ln (A-H - By)           & = (A-H)x + D                       \\
              y                                & = \frac{(A-H)}{B + ce^{-(A-H)x}}
          \end{align}

          To find the equilibrium solutions,

          \begin{align}
              y'        & = 0                                  \\
              y_{1}     & = 0                                &
              y_{2}     & = \frac{A-H}{B} > 0                  \\
              y'(y_{2}) & = Ay_{2} - By_{2}^{2} - Hy_{2} = 0
          \end{align}

          At $ y = y_{2} $, the harvesting term $ -Hy_{2} $ is equal to the natural
          change term $ Ay_{2} - By_{2}^{2} $. This results in a steady state of the
          fish quantity in spite of a non-zero harvesting of fish.

    \item $ A = B = 1 $, $ H = 0.2 $, $ y(0) = 2 $,

          \begin{align}
              y'   & = 0.8y - y^{2}                            \\
              y    & = \frac{0.8}{1 + ce^{-0.8x}}              \\
              y(0) & = 2                          & c & = -0.6
          \end{align}

          From the direction field, $ y_{1} $ is an unstable equilibrium and
          $ y_{2} $ is a stable equilibrium, and is the asymptotic value of $ y $.
          \begin{figure}[H]
              \centering
              \begin{tikzpicture}
                  \def\U{1}
                  \def\V{(0.8*y - y^2)}
                  \def\LEN{sqrt(\U * \U + \V * \V)}
                  \begin{axis}[
                          legend pos = outer north east,
                          width = 8cm,
                          height = 8cm,
                          Ani,
                          axis equal,
                          view     = {0}{90}, % for a view 'from above'
                      ]
                      \addplot3 [
                          forget plot,
                          domain = -1:5,
                          color = gray!50,
                          point meta = {\LEN},
                          quiver={u={(\U) / \LEN},
                                  v={(\V) / \LEN},
                                  scale arrows = 0.2,},
                          -stealth,
                          samples=18,
                      ] (x, y, 0);
                      \addplot[GraphSmooth, y_h, domain = -1:5,
                          restrict y to domain = -1:4]
                      {0.8/(1 - 0.6*e^(-0.8*x))};
                      \addplot[GraphSmooth, y_p, domain = -1:5,
                          restrict y to domain = -1:4]
                      {1/(1 - 0.5*e^(-x))};
                      \node[GraphNode, label={0:{(0, 2)}}] at (axis cs:0, 2) {};
                      \addlegendentry{fishing $ H = 0.2$};
                      \addlegendentry{no fishing $ H = 0$};
                  \end{axis}
              \end{tikzpicture}
          \end{figure}
          In the absence of fishing the asymptotic value is $ A = 1 $.

    \item Same equation as last problem with intermittient fishing every 3 years.

          \begin{align}
              y    & = \frac{0.8}{1 - 0.6e^{-0.8x}}   & y(0)  & = 2      \\
              y(3) & = 1.0835                                            \\
              z    & = \frac{1}{1 - c_{2}e^{-x}}      & c_{2} & = -3.655 \\
              z(6) & = 0.991                                             \\
              w    & = \frac{0.8}{1 - c_{3}e^{-0.8x}} & c_{3} & = 23.419
          \end{align}

          \begin{figure}[H]
              \centering
              \begin{tikzpicture}
                  \begin{axis}[
                          xlabel = Time ($ x $),
                          ylabel = Fish quantity ($ y $),
                          legend pos = north east,
                          grid = both,
                          width = 12cm,
                          height = 8cm,
                          Ani,
                          view     = {0}{90}, % for a view 'from above'
                      ]
                      \addplot[GraphSmooth, y_h, domain = 0:3]
                      {0.8/(1 - 0.6*e^(-0.8*x))} node[pos = 0.5,right]{Phase 1};
                      \addplot[GraphSmooth, y_p, domain = 3:6]
                      {1/(1 + 3.655*e^(-x))} node[pos = 0.5,above]{Phase 2};
                      \addplot[GraphSmooth, y_t, domain = 6:9]
                      {0.8/(1 - 23.419*e^(-0.8*x))} node[pos = 0.5,above]{Phase 3};
                      \node[GraphNode, label={0:{(0, 2)}}] at (axis cs:0, 2) {};
                      \addlegendentry{$ y = 0.8/(1 - 0.6e^{-0.8x})$};
                      \addlegendentry{$ y = 1/(1 + 3.655e^{-x})$};
                      \addlegendentry{$ y = 0.8/(1 - 23.419e^{-0.8x})$};
                  \end{axis}
              \end{tikzpicture}
          \end{figure}

    \item Death rate $ B $, Birth rate $ A $, with population $ y $ and time $ x $,

          \begin{align}
              y'                & = Ay^{2} - By               &       & = Ay(y - B/A) \\
              y_{1}             & = 0                         & y_{2} & = B/A         \\
              \int \frac{1}{y} - \frac{1}{y - B/A}\ \dl x
                                & = -B\int\ \dl x                                     \\
              \frac{y}{y - B/A} & = ce^{-Bx}                                          \\
              y                 & = \frac{(B/A)}{1 - ce^{Bx}}
          \end{align}

          \begin{figure}[H]
              \centering
              \begin{tikzpicture}
                  \begin{axis}[
                          xlabel = Time ($ x $),
                          ylabel = Population ($ y $),
                          legend pos = north east,
                          grid = both,
                          width = 12cm,
                          height = 8cm,
                          Ani,
                          view     = {0}{90}, % for a view 'from above'
                          restrict y to domain = 0:3,
                      ]
                      \addplot[GraphSmooth, y_h, domain = 0:15]
                      {(1   - (1/11)*e^(x))^(-1)} node[pos = 0.75,right]
                      {\footnotesize$ y_{0} = 1.1$};
                      \addplot[GraphSmooth, y_p, domain = 0:15]
                      {(1   - (1/101)*e^(x))^(-1)} node[pos = 0.75,right]
                      {\footnotesize$ y_{0} = 1.01$};
                      \addplot[GraphSmooth, y_s, domain = 0:15]
                      {(1  + (1/99)*e^(x))^(-1)} node[pos = 0.25,above right]
                      {\footnotesize$ y_{0} = 0.99 $};
                      \addplot[GraphSmooth, y_t, domain = 0:15]
                      {(1  + 3*e^(x))^(-1)} node[pos = 0.25, above]
                      {\footnotesize$ y_{0} = 0.25 $};
                      \addlegendentry{$ y = [1 - (1/11)e^{-x}]^{-1}$};
                      \addlegendentry{$ y = [1 - (1/101)e^{-x}]^{-1}$};
                      \addlegendentry{$ y = [1 + (1/99e^{-x}]^{-1}$};
                      \addlegendentry{$ y = [1 + (1/3)e^{-x}]^{-1}$};
                  \end{axis}
              \end{tikzpicture}
          \end{figure}
          If $ y_{0} < B/A $, then $ y'_{0} < 0 $ and extinction happens.\\
          If $ y_{0} > B/A $, then $ y'_{0} > 0 $ and exponential growth happens.

    \item Air inflow and outflow rate is $ \lambda $, and total volume of
          the room is $ V $, with fresh air $ y $ and time $ x $,

          \begin{align}
              y'              & = \lambda \left( 1 - \frac{y}{V} \right)   \\
              \ln(1 - y/V)    & = \frac{-\lambda}{V}x + b                  \\
              1 - \frac{y}{V} & = ce^{-\lambda x/V}                        \\
              \frac{y}{V}     & =  1 - ce^{-\lambda x/V}                 &
              c               & = 1                                        \\
              y(t^{*})        & = 0.9V                                   &
              t^{*}           & = \frac{\ln 0.1}{-\lambda / V}
          \end{align}

          Time needed is $ t^{*} = \SI{76.753}{\minute} $.
\end{enumerate}