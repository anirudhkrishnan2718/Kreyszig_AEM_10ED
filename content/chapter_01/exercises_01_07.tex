\section{Existence and Uniqueness of Solutions for Initial Value Problems}

\begin{enumerate}
    \item $ p $ and $ r $ are continuous
          \begin{align}
              p,q          & \text{ are continuous } &
              \forall\ x   & \in |x - x_{0}| \leq a    \\
              y' = f(x, y) & = r(x) - yp(x)            \\
              \diffp fy    & = -p(x)
          \end{align}

          The fact than $ r,p $ are continuous means that they are bounded in
          that interval.
          Thus, $ f $ and $ f_{y} $ are bounded in the interval.

          This means a unique solution exists for an IVP over this interval.

    \item Checking for existence of solution,
          \begin{align}
              (x-2)y' & = y             & y(2) & = 1      \\
              y'      & = \frac{y}{x-2} & y    & = c(x-2)
          \end{align}
          $ y' $ is not bounded in any region including $ x = 2 $. So the IVP above has
          no solution.By explicitly solving, the solution cannot pass through
          $ (2, 1) $ for any finite $ c $.

    \item Solution exists in the smaller region among
          \begin{align}
              |x - x_{0}| & \leq a & |x - x_{0}| & \leq b/K
          \end{align}
          For very large $ b $, $ a < b/K $ and thus the region in which a solution
          exists is $ |x - x_{0}| \leq a $.

    \item $ k = 0 $ gives infinitely many solutions. \\
          $ k \neq 0 $ gives no solution for finite $ k $, as evident from the explicit
          solution to the ODE shown above $ y = c(x-2) $.

    \item Using the existence theorem,
          \begin{align}
              y'      & = 2y^{2} & y(1)  & = 1  \\
              f(x, y) & = 2y^{2} & f_{y} & = 4y
          \end{align}
          In the rectangle of length $ a $ and height $ b $ centered on $ (1, 1) $,
          \begin{align}
              |f|                                     & \leq K & |f_{y}| & \leq M \\
              \left| 2\left( 1 + \frac{b}{2} \right)^{2}
              \right|                                 & \leq K                    \\
              \left| 2 + 2b + \frac{b^{2}}{2} \right| & \leq K
          \end{align}
          $ \alpha = b/K$ is the subinterval of the rectangle in which the solution
          exists. Maximizing $ \alpha $,
          \begin{align}
              \diff{\alpha}{b}                        & = 0 &
              \diff*{\frac{2b}{(b+2)^{2}}}{b}         & = 0   \\
              \frac{(b+2) \cdot (2b+4-4b)}{(b+2)^{4}} & = 0 &
              \frac{b-2}{(b+2)^{3}}                   & = 0
          \end{align}
          The optimal value of $ b = 2 $ and $ \alpha = 1/4 $, which is the largest
          possible rectangle in which a solution is guaranteed to exist. \par
          An explicit solution to the ODE is
          \begin{align}
              y = \frac{1}{3 - 2x}
          \end{align}

    \item \begin{enumerate}
              \item Picard Iteration,
                    \begin{align}
                        \int_{y}^{y_{0}}\  \dl y & = \int_{x}^{x_{0}}\ f(x, y)
                        \ \dl x                                                   \\
                        y - y_{0}                & = \int_{x_{0}}^{x}\ f(t, y(t))
                        \ \dl t                                                   \\
                        y_{n}(x)                 & = y_{0} + \int_{x_{0}}^{x}
                        \ f(t, y_{n-1}(t))\ \dl t
                    \end{align}

              \item By the iterative method,
                    \begin{align}
                        y'    & = x+y                                              &
                        y(0)  & = 0                                                  \\
                        y_{1} & = y_{0} + \int_{x_{0}}^{x}\ (t + y_{0}(t))\ \dl t  &
                        y_{1} & = \frac{t^{2}}{2}\Bigg|_{0}^{x}                      \\
                              & = \frac{x^{2}}{2}                                    \\
                        y_{2} & = y_{0} + \int_{x_{0}}^{x}\ (t + y_{1}(t))\ \dl t  &
                        y_{2} & = \frac{t^{2}}{2} + \frac{t^{3}}{3!}\Bigg|_{0}^{x}   \\
                        y_{n} & = \frac{x^{2}}{2!} + \frac{x^{3}}{3!}
                        + \dots + \frac{x^{n+1}}{(n+1)!}
                    \end{align}
                    Solving exactly,
                    \begin{align}
                        y'                       & = x+y                           &
                        y' - y                   & = x                               \\
                        h                        & = \int\ -1\ \dl x               &
                                                 & = -x                              \\
                        y                        & = e^{-h}\left[ \int\ e^{h}
                        \{r(x)\}\ \dl x + c \right]                                  \\
                        \int\ xe^{-x}\ \dl x = I & = -xe^{-x} + \int e^{-x}\ \dl x   \\
                        y                        & = -(x + 1) + ce^{x}             &
                        c                        & = 1
                    \end{align}
                    \begin{figure}[H]
                        \centering
                        \begin{tikzpicture}
                            \begin{axis}[
                                    legend pos = north west,
                                    grid = both,
                                    width = 12cm,
                                    height = 8cm,
                                    Ani,
                                    domain = -3:3,
                                    xmax = 4.1,
                                    % restrict y to domain = 0:3,
                                ]
                                \addplot[GraphSmooth, thin, color = red!100!blue]
                                {e^(x) - x - 1} node[pos = 1,right]
                                {\footnotesize$ y_{\text{analytical}}$};
                                \addplot[GraphSmooth, thin, color = red!0!blue]
                                {x^(2)/2} node[pos = 1,right]
                                {\footnotesize$ n = 1$};
                                \addplot[GraphSmooth, thin, color = red!50!blue]
                                {x^(2)/2 + x^(3)/6 + x^(4)/24 + x^(5)/120}
                                node[pos = 0.9,right]
                                {\footnotesize$ n = 4$};
                                \node[GraphNode, label={90:{\footnotesize (0, 0)}}]
                                at (axis cs:0, 0) {};
                                \addlegendentry{$ y = e^{x} - x - 1$};
                            \end{axis}
                        \end{tikzpicture}
                    \end{figure}

              \item By the iterative method,
                    \begin{align}
                        y'    & = 2y^{2}                                         &
                        y(0)  & = 1                                                \\
                        y_{1} & = y_{0} + \int_{x_{0}}^{x}\ (2y_{0}^{2})\ \dl t  &
                        y_{1} & = 1 + 2t\Big|_{0}^{x}                              \\
                              & = 1 + 2x                                           \\
                        y_{2} & = y_{0} + \int_{x_{0}}^{x}\ 2y_{1}^{2}(t)\ \dl t   \\
                        y_{2} & = 1 + 2x + 4x^{2} + \frac{8x^{3}}{3}
                    \end{align}
                    Solving exactly,
                    \begin{align}
                        y'           & = 2y^{2}                      \\
                        \frac{-1}{y} & = 2x+c                        \\
                        y            & = \frac{-1}{2x+c}  & c & = -1 \\
                        y            & = \frac{1}{1 - 2x}
                    \end{align}
                    \begin{figure}[H]
                        \centering
                        \begin{tikzpicture}
                            \begin{axis}[
                                    legend pos = north west,
                                    grid = both,
                                    width = 12cm,
                                    height = 8cm,
                                    Ani,
                                    domain = -0.2:0.25,
                                    xmax = 0.4,
                                    % restrict y to domain = 0:3,
                                ]
                                \addplot[GraphSmooth, thin, color = red!100!blue]
                                {1/(1-2*x)} node[pos = 1,right]
                                {\footnotesize$ y_{\text{analytical}}$};
                                \addplot[GraphSmooth, thin, color = red!0!blue]
                                {1+2*x} node[pos = 1,right]
                                {\footnotesize$ n = 1$};
                                \addplot[GraphSmooth, thin, color = red!50!blue]
                                {1 + 2*x + 4*x^(2) + (8/3)*x^(3)} node[pos = 1,right]
                                {\footnotesize$ n = 2$};
                                \node[GraphNode, label={90:{\footnotesize (0, 1)}}]
                                at (axis cs:0, 1) {};
                                \addlegendentry{$ y = (1-2x)^{-1}$};
                            \end{axis}
                        \end{tikzpicture}
                    \end{figure}

              \item By the iterative method,
                    \begin{align}
                        y'    & = 2y^{1/2}                                        &
                        y(1)  & = 0                                                 \\
                        y_{1} & = y_{0} + \int_{x_{0}}^{x}\ (2y_{0}^{1/2})\ \dl t &
                        y_{1} & = 0\Big|_{1}^{x}                                    \\
                              & = 0                                                 \\
                        y_{2} & = y_{3} = \dots = y_{n} = 0
                    \end{align}
                    Solving exactly,
                    \begin{align}
                        y'      & = 2y^{1/2}               \\
                        y^{1/2} & = x+c       & c   & = -1 \\
                        y       & = (x-1)^{2} & x-1 & >0
                    \end{align}
                    \begin{figure}[H]
                        \centering
                        \begin{tikzpicture}
                            \begin{axis}[
                                    legend pos = north west,
                                    grid = both,
                                    width = 12cm,
                                    height = 8cm,
                                    Ani,
                                    domain = 1:2,
                                    xmax = 2.3,
                                    % restrict y to domain = 0:3,
                                ]
                                \addplot[GraphSmooth, thin, color = y_h]
                                {(x-1)^(2)} node[pos = 1,right]
                                {\footnotesize$ y_{\text{analytical}}$};
                                \addplot[GraphSmooth, thin, color = y_p]
                                {0} node[pos = 1,right]
                                {\footnotesize$ n = 1,2,\dots$};
                                \node[GraphNode, label={90:{\footnotesize (1, 0)}}]
                                at (axis cs:1, 0) {};
                                \addlegendentry{$ y^{1/2} = (x-1)$};
                            \end{axis}
                        \end{tikzpicture}
                    \end{figure}
                    Picard's iteration approximates the trivial solution, not the
                    parabolic one.

              \item TBC. Proof requires more real analysis than I know.
          \end{enumerate}

    \item From example, rectangle $ \mathcal{R} $ has length $ 2a $ and height $ 2b $.
          \begin{align}
              y'      & = 1 + y^{2} & y(0)                     & = 0    \\
              f(x, y) & = 1 + y^{2} & f_{y}                    & = 2y   \\
              |f|     & \leq K      & \implies |1 + (0+b)^{2}| & \leq K \\
          \end{align}
          $ \alpha = b/K$ is the subinterval of the rectangle in which the solution
          exists. Maximizing $ \alpha $,
          \begin{align}
              \diff{\alpha}{b}                          & = 0 &
              \diff*{\frac{b}{1+b^{2}}}{b}              & = 0   \\
              \frac{(1+b^{2}) - b(2b)}{(1 + b^{2})^{2}} & = 0 &
              \frac{1-b^{2}}{(1+b^{2})^{2}}             & = 0
          \end{align}
          The optimal value of $ b = 1 $ and $ \alpha = 1/2 $, which is the largest
          possible rectangle in which a solution is guaranteed to exist. \par
          An explicit solution to the ODE is
          \begin{align}
              y = \tan x
          \end{align}

    \item Showing that a Lipschitz condition holds for a linear ODE, with $ p, r $
          continuous in $ |x - x_{0}| \leq a $. From the mean value theorem of
          differential calculus,
          \begin{align}
              y' = f(x, y)                 & = r(x) - yp(x)        \\
              f(x, y_{2}) - f(x, y_{1})    & = (y_{2} - y_{1})
              \ \diffp fy[y = \tilde{y}]                           \\
                                           & = -(y_{2} - y_{1})
              \  p(x) \Big|_{y = \tilde{y}}                        \\
              \text{if} \quad |p(x)|       & \leq M \quad          \\
              \text{then} \quad |f(x, y_{2}) - f(x, y_{1})|
                                           & \leq M|y_{2} - y_{1}| \\
              |p(x)| \cdot |y_{2} - y_{1}| & \leq M|y_{2} - y_{1}|
          \end{align}
          $ p(x) $is continuous in the rectangle is thus bounded in the rectangle.
          Let this bound be $ M $This proves the existence of a Lipschitz condition
          for linear first order ODEs. \par
          Additionally, the continuity of $ f(x, y) $ automatically guarantees
          the uniqueness of a solution to these IVPs. (using the integrating
          factor method)

    \item If the existence and uniqueness theorems are satisfied in the rectangle
          $ \mathcal{R} $, then the point of intersection of two distinct solutions to
          the ODE would satisfy the IVP for both of the IVPs. \par
          This directly violates the uniqueness theorem.

    \item All three possible kinds of IVP needed.
          \begin{align}
              (x^{2} - x)y' & = (2x - 1)y         \\
              \ln y         & = \ln (x^{2}-x) + b \\
              y             & = cx(x-1)
          \end{align}
          \begin{enumerate}
              \item for infinitely many solutions, $ y(0) = 0 $ and $ y(1) = 0 $
              \item for no solution, $ y(0) = k $ and $ y(1) = k $ with $ k \neq 0 $
              \item for a unique solution, $ y(\alpha) = \beta $ with
                    $ \alpha \in \mathbb{R} - \{0, 1\} $ and $ \beta \in \mathbb{R} $
          \end{enumerate}
\end{enumerate}