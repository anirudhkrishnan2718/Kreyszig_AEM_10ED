\section{Basic Concepts: Modeling}

\begin{enumerate}
    \item Using the substitution $v = 2 \pi x$

          \begin{align}
              y' & = -2 \sin(2 \pi x)                     \\
              dy & = -2 \int \sin(2 \pi x) \quad dx       \\
              dy & = \frac{-1}{\pi} \int \sin(v) \quad dv \\
              y  & = \frac{\cos(2\pi x)}{\pi} + c
          \end{align}


    \item Using the substitution $v = -x^{2} / 2$

          \begin{align}
              y' & = -x \exp(-x^{2} / 2)                \\
              dy & = - \int x \exp(-x^{2} / 2) \quad dx \\
              dy & = \int \exp(v) \quad dv              \\
              y  & = \exp(-x^{2} / 2) + c
          \end{align}


    \item Using the integration over $y$

          \begin{align}
              y' & = y                         \\
              dx & = \int \frac{1}{y} \quad dy \\
              x  & = \ln(y) + c                \\
              y  & = C\exp(x)
          \end{align}


    \item Using the integration over $y$

          \begin{align}
              y' & = -1.5 y                      \\
              dx & = \int \frac{-2}{3y} \quad dy \\
              x  & = \frac{-2 \ln(y)}{3} + c     \\
              y  & = C\exp(-1.5x)
          \end{align}


    \item Using the substitution $v = 2 \pi x$

          \begin{align}
              y'            & = 4 e^{-x} \cos(x)                                                      \\
              \int \quad dy & = 4 \int e^{-x} \cos(x) \quad dx                                        \\
              y             & = 4 e^{-x} \sin(x)  + 4 \int e^{-x} \sin(x) \quad dx                    \\
              y             & = 4 e^{-x} \sin(x)  - 4 e^{-x} \cos(x) - 4 \int e^{-x} \cos(x) \quad dx \\
              y             & = 2 e^{-x} (\sin(x) - \cos(x))                                          \\
              y             & = -2 \sqrt{2} e^{-x} \cos\left(x + \frac{\pi}{4}\right) + c
          \end{align}


    \item Using the standard result for trigonometric ODEs, (second order ODE will result in two arbitrary constants)

          \begin{align}
              y'' & = -y                            \\
              y   & = c_{1} \cos(x) + c_{2} \sin(x)
          \end{align}


    \item Using the substitution $a  = 5.13$

          \begin{align}
              y'            & = \cosh(5.13 x)                             \\
              \int \quad dy & = \int \cosh(a x)  \quad dx                 \\
              y             & = \int \frac{e^{ax} + e^{-ax}}{2}  \quad dx \\
              y             & = \frac{\sinh(ax)}{a} + c
          \end{align}


    \item Using the substitution $a  = -0.2$ (Third order ODE results in 3 arbitrary constants)

          \begin{align}
              y''' & = \exp(-0.2x)                              \\
              y    & = \frac{\exp(ax)}{a^{3}} + bx^{2} + cx + d
          \end{align}


    \item IC is $y(0) = 2$

          \begin{align}
              4y      & = 4c\exp(-4x) + 1.4 \\
              y'      & =-4c \exp(-4x)      \\
              y' + 4y & = 1.4               \\
              y(0)    & = c + 0.35 = 2      \\
              c       & = 1.65
          \end{align}

          \begin{figure}[H]
              \centering
              \begin{tikzpicture}
                  \begin{axis}[Ani, grid=both]
                      \addplot [GraphSmooth, domain=-0.5:0.5]
                      {0.35 + 1.65*e^(-4*x)};
                      \node[GraphNode, label={45:{(0,2)}}]
                      at (axis cs:0,2) {};
                      \addlegendentry{$0.35 + 1.65 e^{-4x}$}
                  \end{axis}
              \end{tikzpicture}
          \end{figure}

    \item IC is $y(0) = \pi$

          \begin{align}
              5xy      & = (5x) \ ce^{-2.5 x^{2}} \\
              y'       & = (-5x)\ ce^{-2.5 x^{2}} \\
              y' + 5xy & = 0                      \\
              y(0)     & = c  = \pi
          \end{align}

          \begin{figure}[H]
              \centering
              \begin{tikzpicture}
                  \begin{axis}[Ani, grid=both]
                      \addplot [GraphSmooth, domain=-2:2]
                      {pi * e^(-2.5 * x^(2))};
                      \node[GraphNode, label={0:{(0,$\pi$)}}]
                      at (axis cs:0,pi) {};
                      \addlegendentry{$\pi e^{-2.5x^{2}}$}
                  \end{axis}
              \end{tikzpicture}
          \end{figure}

    \item IC is $y(0) = 1/2$

          \begin{align}
              y'     & = (1 + x + c)\ e^{x} \\
              y' - y & = e^{x}              \\
              y(0)   & = c  = 1/2
          \end{align}

          \begin{figure}[H]
              \centering
              \begin{tikzpicture}
                  \begin{axis}[Ani, grid=both, legend pos = north west]
                      \addplot [GraphSmooth, domain=-2:2]
                      {(x + 0.5) * e^(x))};
                      \node[GraphNode, label={135:{(0,1/2)}}]
                      at (axis cs:0,1/2) {};
                      \addlegendentry{$(x + 1/2)e^{x}$}
                  \end{axis}
              \end{tikzpicture}
          \end{figure}

    \item IC is $y(1) = 4$

          \begin{align}
              2yy' & = 8x          & y & > 0 \\
              yy'  & = 4x          & y & > 0 \\
              y(1) & = c + 4  = 16           \\
              c    & = 12
          \end{align}

          \begin{figure}[H]
              \centering
              \begin{tikzpicture}
                  \begin{axis}[Ani, grid=both, legend pos = south west]
                      \addplot [GraphSmooth, domain=-3:3]
                      {(4*x^(2) + 12)^(1/2)};
                      \node[GraphNode, label={135:{(1, 4)}}]
                      at (axis cs:1, 4) {};
                      \addlegendentry{$\sqrt{4x^{2} + 12}$}
                  \end{axis}
              \end{tikzpicture}
          \end{figure}

    \item IC is $y(0) = 1/4$

          \begin{align}
              y'        & = \frac{ce^{-x}}{(1 + ce^{-x})^{2}}         \\
              y - y^{2} & = \frac{1 + ce^{-x} - 1}{(1 + ce^{-x})^{2}} \\
              y'        & = y - y^{2}                                 \\
              y(0)      & = \frac{1}{(1+c)}  = 1/4                    \\
              c         & = 3
          \end{align}

          \begin{figure}[H]
              \centering
              \begin{tikzpicture}
                  \begin{axis}[Ani, grid=both, legend pos = north west]
                      \addplot [GraphSmooth, domain=-5:5]
                      {1 / (1 + 3 * e^(-x))};
                      \node[GraphNode, label={135:{(0, 1/4)}}]
                      at (axis cs:0,1/4) {};
                      \addlegendentry{$(1 + 3e^{-x})^{-1}$}
                  \end{axis}
              \end{tikzpicture}
          \end{figure}

    \item IC is $y(\pi / 2) = 0$

          \begin{align}
              2y - 8     & = 2c\ \sin^{2} x              \\
              y' \tan x  & = 2c\ \sin x \ \cos x\ \tan x \\
              y' \tan x  & = 2y -  8                     \\
              y(\pi / 2) & = c + 4  = 0                  \\
              c          & = -4
          \end{align}

          \begin{figure}[H]
              \centering
              \begin{tikzpicture}
                  \begin{axis}[Ani, grid=both, legend pos = north east, PiStyleX]
                      \addplot [GraphSmooth, domain=-pi:pi]
                      {4 - 4*(sin(deg(x)))^(2)};
                      \node[GraphNode, label={0:{($\pi / 2$, 0)}}]
                      at (axis cs:pi/2, 0) {};
                      \addlegendentry{$-4 \sin^{2} x + 4$}
                  \end{axis}
              \end{tikzpicture}
          \end{figure}

    \item By Inspection , the ODE in Problem 13 has the constant solutions

          \begin{align}
              y & = 1 \\
              y & = 0
          \end{align}


    \item Verifying the general solution by substitution

          \begin{align}
              xy'    & = cx           \\
              y'^{2} & = c^{2}        \\
              y      & = cx - c^{2}   \\
              y      & = xy' - y'^{2}
          \end{align}

          Verifying the singular solution by substitution

          \begin{align}
              xy'    & = \frac{x^{2}}{2} \\
              y'^{2} & = \frac{x^{2}}{4} \\
              y      & = \frac{x^{2}}{4} \\
              y      & = xy' - y'^{2}
          \end{align}

          The parabola happens to have the same derivative for all $x$ as the member of the family of straight lines tangent to it. This makes the parabola a singular solution to the ODE.

    \item Given a starting mass of $1$ gm, find the time taken to reach a mass of $0.5$ gm. This is equal to half-life.

          \begin{align}
              \frac{dy}{dt} & = -ky                                                                           \\
              \ln y         & = -kt + c                                                                       \\
              y             & = c \ e^{-kt}                                                                   \\
              y_{T}         & = \frac{y_{0}}{2}                                                               \\
              e^{-kT}       & = 1/2                                                                           \\
              T             & = \frac{\ln 2}{k} = \frac{ln 2}{1.4 \times 10^{-11}} = 1568.89\  \mathrm{years}
          \end{align}


    \item Given the half life $ T = 3.6 $ days, in 1 day,

          \begin{align}
              y        & = c\ e^{-kt}                                               \\
              y(t = 1) & = c\ e^{-k} = y_{0} e ^{-k}                                \\
              y(t = 1) & = 1\ \mathrm{g} \times \exp \left(\frac{- \ln 2}{T}\right) \\
              y(t = 1) & = 0.825\ \mathrm{g}
          \end{align}
          and for 1 year,
          \begin{align}
              y(t = 365) & = c\ e^{-k} = y_{0} e ^{-365\ k}                                      \\
              y(t = 365) & = 1\ \mathrm{g} \times \exp \left(\frac{- \ln 2 \times 365}{T}\right) \\
              y(t = 365) & = 0\ \mathrm{g}
          \end{align}


    \item Given IC is $y(0) = 0$ and $y'(0) = 0$,

          \begin{align}
              \frac{d^{2}y}{dt^{2}} & = g                                            \\
              y                     & = a + bt + \frac{gt^{2}}{2}                    \\
              y(0)                  & = 0                         & \implies a & = 0 \\
              y'(0)                 & = 0                         & \implies b & = 0 \\
              y                     & = \frac{gt^{2}}{2}
          \end{align}


    \item Given IC is $y(18,000) = 1/2 \times y(0)$ and height $t$

          \begin{align}
              \frac{dy}{dt} & = -ky                                                     \\
              \ln y         & = -kt + c                                                 \\
              y             & = c \ e^{-kt}                                             \\
              y_{T}         & = \frac{y_{0}}{2}                                         \\
              e^{-kT}       & = 1/2                                                     \\
              k             & = \frac{\ln 2}{T} = \frac{ln 2}{18000}\  \mathrm{ft}^{-1} \\
              y(35,000)     & = y(0) \times 2^{-35/18}                                  \\
              y(35,000)     & = y(0) \times 0.259
          \end{align}

\end{enumerate}

