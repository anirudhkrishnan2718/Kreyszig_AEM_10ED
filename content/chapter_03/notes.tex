\chapter{Higher Order Linear ODEs}
\section{Homogeneous Linear ODEs}

\begin{description}
    \item[Linear ODE of nth order] The standard form of an ODE on $ n $-th order is,
        \begin{align}
            y^{(n)} + p_{n-1}(x)y^{(n-1)} + \dots + p_{1}(x)y' + p_{0}(x)y & = r(x)
        \end{align}
        Here, $ \{p_{i}(x)\} $ are any continuous functions of $ x $, and the solution
        $ h(x) $, is defined and $ n $-times differentiable on the interval $ \mathcal{I} $
        in which the set $ \{p_{i}(x)\} $ are defined and continuous.
    \item[Superposition principle] If $ y_{1},\ y_{2} $ are solutions to a linear h-ODE
        of order $ n $, then
        \begin{align}
            c_{1}y_{1} + c_{2}y_{2} = y_{3}
        \end{align}
        is also a solution for some constants $ c_{1}, c_{2} $.
    \item[Linearly Independent functions] A set of functions $ y_{1}(x), \dots, y_{n}(x) $
        are linearly independent (L.I.) on some interval $ \mathcal{I} $ if,
        \begin{align}
            k_{1}y_{1}(x) + \dots + k_{n}y_{n}(x) = 0 \qquad \implies \qquad
            k_{1} = \dots = k_{n} = 0
        \end{align}
        Conversely, if some solution to the above equation exists for which not all $ k_{i} $
        are zero, then the functions are linearly dependent (L.D.)
    \item[Basis of solutions] A set of L.I. solutions to the linear h-ODE of order $ n $.
    \item[General solution] Given a basis of solutions $ \{y_{i}\} $ of the h-ODE,
        \begin{align}
            y & = c_{1}y_{1} + \dots + c_{n}y_{n}
        \end{align}
        is a general solution of the ODE. Particular solutions can be obtained by assigning
        values to the coefficients $ \{c_{i}\} $. \par
        No singular solutions exist which cannot be obtained from the general solution.
    \item[Initial Value Problem] Given the set of $ n $ initial conditions,
        \begin{align}
            y(x_{0}) = K_{0} \quad \ y'(x_{0}) = K_{1}\quad \dots \quad y^{(n)}(X_{0}) = K_{n}
        \end{align}
        There exists a unique solution for a linear h-ODE with coefficients $ \{p_{i}(x)\} $
        continuous on $ \mathcal{I} $ given that $ x_{0} \in \mathcal{I} $.
    \item[Wronskian] Using the determinant of order $ n $,
        \begin{align}
            \renewcommand*{\arraystretch}{1.5}
            W(y_{1},\dots,y_{n}) = \begin{vNiceMatrix}[r, margin]
                                       y_{1}         & y_{2}         & \dots  & y_{n}         \\
                                       y_{1}'        & y_{2}'        & \dots  & y_{n}'        \\
                                       \vdots        & \vdots        & \ddots & \vdots        \\
                                       y_{1}^{(n-1)} & y_{2}^{(n-1)} & \dots  & y_{n}^{(n-1)} \\
                                   \end{vNiceMatrix}
        \end{align}
        If $ W(x) \neq 0 $ for some $ x \in \mathcal{I} $, where the coefficients of the
        h-ODE $ \{p_{i}(x)\} $ are continuous on $ \mathcal{I} $, then the solutions
        $ \{y_{i}(x)\} $ are L.I. \par

        Conversely, if $ W(x) = 0 $ for some $ x = x_{0} \in \mathcal{I} $, then $W \equiv 0$
        identically for all $ x_{0} \in \mathcal{I} $ and the functions are L.D.
    \item[Existence and Uniqueness] If the coefficients $ \{p_{i}(x)\} $ are continuous on
        some interval $ \mathcal{I} $, then the h-ODE has a general solution on $ \mathcal{I} $.
        \par
        Using the fact that the Wronskian is merely the coefficient matrix of the system of linear
        equations in the unknowns $ \{c_{i}\} $ in the general solution, \par
        The Wronskisan of a solution composed of an L.I. basis of solutions is guaranteed to
        cover all possible solutions. Thus the general solution is unique. \par
\end{description}

\section{Homogeneous Linear ODEs with Constant Coefficients}

\begin{description}
    \item[Standard form] In standard form, the h-ODE with constant coefficients of order $ n $,
        alongside its characteristic equation is,
        \begin{align}
            y^{(n)} + a_{n-1}y^{(n-1)} + \dots + a_{1}y' + a_{0} & = 0             \\
            y_{h}                                                & = e^{\lambda x} \\
            \lambda^{n} + a_{n-1}\lambda^{n-1} + \dots a_{1}\lambda + a_{0} = 0
        \end{align}
    \item[Distinct real roots] Each distinct root $ \lambda_{k} $ corresponds to a solution
        to the ODE $ e^{k\lambda} $
        \begin{align}
            W & = \begin{vNiceMatrix}[r, margin]
                      e^{\lambda_{1}x}                  & e^{\lambda_{2}x}                  & \dots  & e^{\lambda_{n}x}                  \\
                      \lambda_{1}e^{\lambda_{1}x}       & \lambda_{2}e^{\lambda_{2}x}       & \dots  & \lambda_{n}e^{\lambda_{n}x}       \\
                      \lambda_{1}^{2}e^{\lambda_{1}x}   & \lambda_{2}^{2}e^{\lambda_{2}x}   & \dots  & \lambda_{n}^{2}e^{\lambda_{n}x}   \\
                      \vdots                            & \vdots                            & \ddots & \vdots                            \\
                      \lambda_{1}^{n-1}e^{\lambda_{1}x} & \lambda_{2}^{n-1}e^{\lambda_{2}x} & \dots  & \lambda_{n}^{n-1}e^{\lambda_{n}x}
                  \end{vNiceMatrix}
        \end{align}
    \item[Vandermode determinant] A simplified version of the above determinant given by,
        \begin{align}
            W & = \exp\left( x\sum_{k=1}^{n} \lambda_{k} \right)\ \begin{vNiceMatrix}[r, margin]
                                                                      1                 & 1                 & \dots  & 1                 \\
                                                                      \lambda_{1}       & \lambda_{2}       & \dots  & \lambda_{n}       \\
                                                                      \lambda_{1}^{2}   & \lambda_{2}^{2}   & \dots  & \lambda_{n}^{2}   \\
                                                                      \vdots            & \vdots            & \ddots & \vdots            \\
                                                                      \lambda_{1}^{n-1} & \lambda_{2}^{n-1} & \dots  & \lambda_{n}^{n-1}
                                                                  \end{vNiceMatrix}                       \\
              & = \exp\left( x\sum_{k=1}^{n} \lambda_{k} \right)\ (-1)^{n(n-1)/2} \cdot \prod_{j = 1}^{n}\ (\lambda_{j} - \lambda_{k}) \qquad n \geq j > k
        \end{align}
        The above expression is simply the difference between all possible combinations of roots.
        Since the first term is a product of exponentials, it is never zero. \par
        So, the Wronksian provides L.I only if all the roots are distinct, and thus none
        of the terms $ (\lambda_{j} - \lambda_{k}) $ are zero.
    \item[Repeated roots] For repeated roots of multiplicity $ m $, whether real or complex,
        \begin{align}
            y_{1}  & = e^{\lambda x}      & z_{1}, z_{2}     & = e^{\alpha x}\sin \beta x, e^{\alpha x} \cos \beta x           \\
            y_{2}  & = xe^{\lambda x}     & z_{3}, z_{4}     & = xe^{\alpha x}\sin \beta x, xe^{\alpha x} \cos \beta x         \\
            y_{3}  & = x^{2}e^{\lambda x} & z_{5}, z_{6}     & = x^{2}e^{\alpha x}\sin \beta x, x^{2}e^{\alpha x} \cos \beta x \\
            \vdots &                      & \vdots           & \nonumber                                                       \\
            y_{m}  & = x^{m}e^{\lambda x} & z_{2m-1}, z_{2m} & = x^{m}e^{\alpha x}\sin \beta x, x^{m}e^{\alpha x} \cos \beta x
        \end{align}
        Other solutions are obtained by multiplying higher powers of $ x $ to existing solutions.
        \par
        To derive the above multiple roots factor, consider the root $ \lambda_{1} $ with
        multiplicity $ m $,
        \begin{align}
            \mathcal{L}[y]             & = y^{(n)} + a_{n-1}y^{(n-1)} + \dots + a_{1}y' + a_{0}y                              \\
            \mathcal{L}[e^{\lambda x}] & = (\lambda^{n} + a_{n-1}\lambda^{n-1} + \dots + a_{1}\lambda + a_{0})\ e^{\lambda x} \\
                                       & = (\lambda - \lambda_{1})^{m} \cdot h(\lambda) \cdot e^{\lambda x}
        \end{align}
        Here $ h(\lambda) $ is the leftover polynomial after factoring out all the $ \lambda_{1} $.
        \begin{align}
            \diffp**{\lambda}{\mathcal{L}[e^{\lambda x}]}
             & = \mathcal{L}\left[ \diffp**{\lambda}{e^{\lambda x}} \right]
            = \mathcal{L}[xe^{\lambda x}]                                                       \\
             & = m(\lambda - \lambda_{1})^{m-1} \cdot h(\lambda)e^{\lambda x} \nonumber         \\
             & + (\lambda - \lambda_{1})^{m} \cdot \diffp**{\lambda}{[h(\lambda)e^{\lambda x}]} \\
        \end{align}
        Since $ m > 2 $, the RHS is zero at $ \Lambda = \lambda_{1} $. This means that
        $\mathcal{L}[xe^{\lambda x}] = 0$, and thus $ xe^{\lambda x} $ is a solution.
        \par Further differentiation w.r.t. $ \lambda $ can be used to arrive at further
        solutions corresponding to $ \lambda_{1} $.
\end{description}

\section{Nonhomogeneous Linear ODEs}

\begin{description}
    \item[Standard form] Ensuring the coefficient of $ y^{(n)} $ is $ 1 $,
        \begin{align}
            y^{(n)} + a_{n-1}y^{(n-1)} + \dots + a_{1}y' + a_{0} & = r(x)                \\
            y                                                    & = y_{h}(x) + y_{p}(x)
        \end{align}
        The Initial value problem is the same as for h-ODEs.
    \item[Undetermined coefficients] Similar to second order nh-ODEs, higher order ODEs
        can be solved using the pre-factor $ x^{m} $ to deal with roots of multiplicity $ m $.
    \item[Variation of parameters] Generalizing to order $ n $,
        \begin{align}
            y_{p} & = \sum_{k = 1}^{n}\ y_{k} \int \frac{W_{k}}{W}\ r \dl x
        \end{align}
        $ W $ is the Wronskian of the h-ODE with a basis of solutions $ \{y_{1}, y_{2},
            \dots,y_{n}\} $. $ W_{k} $ is the reduced Wronskian produced by replacing the
        $ k $-th column of $ W $ with the column vector $ [0\ 0\ \dots\ 0\ 1]^{T} $
\end{description}