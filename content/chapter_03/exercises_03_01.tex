\section{Homogeneous Linear ODEs}
\begin{enumerate}
      \item The given functions are solutions, trivially. \par
            Checking if the given solutions are L.I.
            \begin{align}
                  y^{iv} & = 0                             \\
                  y      & = \{1, x, x^{2}, x^{3}\}        \\
                  W      & = \begin{vNiceMatrix}[r, margin]
                                   1 & x & x^{2} & x^{3}  \\
                                   0 & 1 & 2x    & 3x^{2} \\
                                   0 & 0 & 2     & 6x     \\
                                   0 & 0 & 0     & 6      \\
                             \end{vNiceMatrix} = 12
            \end{align}
            Since $ W \neq 0 $, the solutions form a basis.

      \item Checking if the given functions are solutions, true
            \begin{align}
                  y''' - 2y'' - y' + 2y & = 0                           \\
                  y                     & = \{e^{x}, e^{-x}, e^{2x}\}   \\
                  0                     & = 1 - 2 - 1 + 2             &
                  \cdots\cdots          & [e^{x}]                       \\
                  0                     & = -1 - 2 + 1 + 2            &
                  \cdots\cdots          & [e^{-x}]                      \\
                  0                     & = 8 - 8 - 2 + 2             &
                  \cdots\cdots          & [e^{2x}]
            \end{align}
            Checking if the given solutions are L.I.
            \begin{align}
                  y & = \{e^{x}, e^{-x}, e^{2x}\}     \\
                  W & = \begin{vNiceMatrix}[r, margin]
                              e^{x} & e^{-x}  & e^{2x}  \\
                              e^{x} & -e^{-x} & 2e^{2x} \\
                              e^{x} & e^{-x}  & 4e^{2x} \\
                        \end{vNiceMatrix} = - 6e^{2x}
            \end{align}
            Since $ W \neq 0 $, the solutions form a basis.

      \item Checking if the given functions are solutions, true
            \begin{align}
                  y^{iv} + 2y'' + y & = 0                                        \\
                  y                 & = \{\cos x, \sin x, x\cos x, x\sin x\}     \\
                  0                 & = x\cos x + 4\sin x - 2x\cos x - 4\sin x
                  + x \cos x        &
                  \cdots\cdots      & [x\cos x]                                  \\
                  0                 & = x\sin x - 4\cos x -2x\sin x + 4\cos x
                  + x\sin x         &
                  \cdots\cdots      & [x\sin x]                                  \\
                  0                 & = 1 - 2 + 1                              &
                  \cdots\cdots      & [\cos x]                                   \\
                  0                 & = 1 - 2 + 1                              &
                  \cdots\cdots      & [\sin x]
            \end{align}
            Checking if the given solutions are L.I.
            \begin{align}
                  y & = \{\cos x, \sin x, x\cos x, x\sin x\}     \\
                  W & = \begin{vNiceMatrix}[r, margin]
                              \cos x  & \sin x  & x\cos x            &
                              x\sin x                                  \\
                              -\sin x & \cos x  & -x\sin x + \cos x  &
                              x\cos x + \sin x                         \\
                              -\cos x & -\sin x & -x\cos x - 2\sin x &
                              -x\sin x + 2\cos x                       \\
                              \sin x  & -\cos x & x\sin x - 3\cos x  &
                              -x\cos x - 3\sin x
                        \end{vNiceMatrix} = 4
            \end{align}
            Since $ W \neq 0 $, the solutions form a basis.

      \item Checking if the given functions are solutions, true
            \begin{align}
                  0            & = y''' + 12y'' + 48y' + 64y             \\
                  y            & = \{e^{-4x}, xe^{-4x}, x^{2}e^{-4x}\}   \\
                  0            & = -64 + 192 - 192 + 64                &
                  \cdots\cdots & [e^{-4x}]                               \\
                  0            & = e^{-4x}[-192x + 48 + 192x - 96 + 48
                  - 64x + 64x] &
                  \cdots\cdots & [xe^{-4x}]                              \\
                  0            & = e^{-4x}[(-64 + 192 - 192 + 64)x^{2}
                              + (96 - 192 + 96)x - (24 + 24)]
                               &
                  \cdots\cdots & [xe^{-4x}]
            \end{align}
            Checking if the given solutions are L.I.
            \begin{align}
                  y & = \{e^{-4x}, xe^{-4x}, x^{2}e^{-4x}\}                  \\
                  W & = \begin{vNiceMatrix}[r, margin]
                              e^{-4x}   & xe^{-4x}          & x^{2}e^{-4x}         \\
                              -4e^{-4x} & e^{-4x}(1 - 4x)   & e^{-4x}(2x - 4x^{2}) \\
                              16e^{-4x} & e^{-4x}(-8 + 16x) & e^{-4x}(2 - 16x
                              + 16x^{2})
                        \end{vNiceMatrix} = 2 e^{-12x}
            \end{align}
            Since $ W \neq 0 $, the solutions form a basis.

      \item Checking if the given functions are solutions, true
            \begin{align}
                  0             & = y''' + 2y'' + 5y'                        \\
                  y             & = \{1, e^{-x}\cos(2x), e^{-x}\sin(2x)\}    \\
                  0             & = 0 + 0 + 0                              &
                  \cdots\cdots  & [1]                                        \\
                  0             & = e^{-x}[\cos(2x)(11 - 6 - 5) + \sin(2x)
                  (2 + 8 - 10)] &
                  \cdots\cdots  & [e^{-x}\cos(2x)]                           \\
                  0             & = e^{-x}[\cos(2x)(-2 -8 + 10) + \sin(2x)
                  (11 -6 - 5)]  &
                  \cdots\cdots  & [e^{-x}\sin(2x)]
            \end{align}
            Checking if the given solutions are L.I.
            \begin{align}
                  y & = \{1, e^{-x}\cos(2x), e^{-x}\sin(2x)\}                                                          \\
                  W & = \begin{vNiceMatrix}[r, margin]
                              1 & e^{-x}\cos(2x)                & e^{-x}\sin(2x)    \\
                              0 & e^{-x}[-2\sin(2x) - \cos(2x)] & e^{-x}[-\sin(2x)
                              + 2\cos(2x)]                                          \\
                              0 & e^{-x}[4\sin(2x) - 3\cos(2x)] & e^{-x}[-3\sin(2x)
                                          - 4\cos(2x)]
                        \end{vNiceMatrix}  = 10 e^{-2x}
            \end{align}
            Since $ W \neq 0 $, the solutions form a basis.

      \item Checking if the given functions are solutions, true
            \begin{align}
                  0 & = x^{3}y''' - 3x^{2}y'' + 3xy'                          \\
                  y & = \{1, x^{2}, x^{4}\}                                   \\
                  0 & = 0 + 0 + 0                    & \cdots\cdots & [1]     \\
                  0 & = 0 - 6x^{2} + 6x^{2}          & \cdots\cdots & [x^{2}] \\
                  0 & = 24x^{4} - 36x^{4} + 12x^{4}  & \cdots\cdots & [x^{4}]
            \end{align}
            Checking if the given solutions are L.I.
            \begin{align}
                  y & = \{1, x^{2}, x^{4}\}           \\
                  W & = \begin{vNiceMatrix}[r, margin]
                              1 & x^{2} & x^{4}   \\
                              0 & 2x    & 4x^{3}  \\
                              0 & 2     & 12x^{2}
                        \end{vNiceMatrix} = 16 x^{3}
            \end{align}
            Since $ W \neq 0 $, the solutions form a basis.

      \item General properties of solutions of $ n $-th order ODEs with constant
            coefficients.
            \begin{enumerate}
                  \item Relation between solutions of characteristic equation and
                        coefficients,
                        \begin{align}
                              y^{(n)} + a_{n-1}y^{(n-1)} + \dots + a_{1}y' + a_{0}y
                                                           & = 0                  \\
                              \text{Try} \quad y           & = e^{\lambda x}      \\
                              \difoverride{k} \diff[k] yx  & = \lambda^{k}
                              e^{\lambda x}                                       \\
                              \lambda^{n} + a_{n-1}\lambda^{n-1} +
                              \dots + a_{1}\lambda + a_{0} & = 0 = P_{n}(\lambda)
                        \end{align}
                        The relation between the $ n $ roots of the polynomial
                        $ P_{n}(\lambda) $, and the coefficients of the characteristic
                        equation is a standard result. Let $ S_{k} $ be the sum of
                        products of roots taken $ k $ at a time. Then,
                        \begin{align}
                              a_{k} & = (-1)^{k} S_{k}
                        \end{align}

                  \item Reduction of order by substituting
                        \begin{align}
                              z       & = y^{(k)}   \\
                              z^{(m)} & = y^{(k+m)}
                        \end{align}
                        extends to higher order ODEs since differentiation of higher
                        order works the same as double differentiation.

                  \item Let $ \lambda $ be a root of the characteristic equation with
                        multiplicity
                        $ m $,
                        \begin{align}
                              \diff* {e^{\lambda x}u}{x}
                               & = \lambda e^{\lambda x}u + e^{\lambda x}\diff ux \\
                              (D - \lambda) e^{\lambda x} u
                               & = e^{\lambda x} \diff ux                         \\
                              (D - \lambda)^{m} e^{\lambda x} u
                               & = e^{\lambda x}D^{m}u
                        \end{align}
                        The LHS is zero by virtue of $ \lambda $ being a repeated root.
                        This means that the RHS is a polynomial of degree $ (m-1) $.
                        In the interest of L.I. basis of solutions, each solution
                        corresponding to the repeated root is multiplied by increasing
                        powers of $ x $ until $ x^{m-1} $.

                  \item For twice repeated root, the derivation is the same as for an
                        ODE of order 2.
                        For higher multiplicity, with $ \mu = \lambda + \delta $
                        \begin{align}
                              y_{2} & = \lim_{\lambda \to \mu} \frac{x^{m}e^{\mu x}
                              - x^{m}e^{\lambda x}}{\mu - \lambda}                   \\
                              y_{2} & = \lim_{\delta \to 0} \frac{x^{m}e^{\lambda x}
                              e^{\delta x} - x^{m}e^{\lambda x}}{\delta}             \\
                                    & = \lim_{\delta \to 0} \diff* {(x^{m}
                              e^{\lambda x}
                              e^{\delta x} - x^{m}e^{\lambda x})}{\delta}            \\
                                    & = x^{m+1}e^{\lambda x}
                        \end{align}
                        This result also generalizes from second order to higher order
                        linear h-ODEs with constant coefficients.
            \end{enumerate}

      \item The given functions are {\color{y_p} not L.I.} on the interval $ x \geq 0 $
            , since the zero function is
            a member of the set and $ W \equiv 0 $.

      \item The given functions are {\color{y_h} L.I.} on the interval $ x \geq 0 $,
            \begin{align}
                  W & = \begin{vNiceMatrix}[r, margin]
                              \tan x           & \cot x           & 1 \\
                              \sec^{2}x        & -\csc^{2} x      & 0 \\
                              2\sec^{2}x\tan x & 2\csc^{2}x\cot x & 0 \\
                        \end{vNiceMatrix} = \frac{2}{\sin^{3}x \cos^{3}x}
            \end{align}

      \item The given functions are {\color{y_h} L.I.} on the interval $ x \geq 0 $,
            \begin{align}
                  W & = \begin{vNiceMatrix}[r, margin]
                              e^{2x}  & xe^{2x}        & x^{2}e^{2x}             \\
                              2e^{2x} & [1 + 2x]e^{2x} & [2x + 2x^{2}]e^{2x}     \\
                              4e^{2x} & [4 + 4x]e^{2x} & [2 + 6x + 4x^{2}]e^{2x}
                        \end{vNiceMatrix} = 2(1 - x)e^{6 x}
            \end{align}

      \item The given functions are {\color{y_h} L.I.} on the interval $ x \geq 0 $,
            \begin{align}
                  W & = \begin{vNiceMatrix}[r, margin]
                              e^{x}\cos x            & e^{x}\sin x            & e^{x} \\
                              e^{x}[\cos x - \sin x] & e^{x}[\cos x + \sin x] & e^{x} \\
                              e^{x}[-2\sin x]        & e^{x}[2\cos x]         & e^{x}
                        \end{vNiceMatrix} = e^{3x}
            \end{align}

      \item The given functions are {\color{y_p} not L.I.} on the interval
            $ x \geq 0 $, \par either using the expansion, $ \cos(2x) = \cos^{2}x -
                  \sin^{2}x $ or,
            \begin{align}
                  W & = \begin{vNiceMatrix}[r, margin]
                              \sin^{2} x & \cos^{2} x & \cos(2x)   \\
                              \sin(2x)   & -\sin(2x)  & -2\sin(2x) \\
                              2\cos(2x)  & -2\cos(2x) & -4\cos(2x)
                        \end{vNiceMatrix} = 0
            \end{align}

      \item The given functions are {\color{y_h} L.I.} on the interval $ x \geq 0 $,
            \begin{align}
                  W & = \begin{vNiceMatrix}[r, margin]
                              \sin x  & \cos x  & \sin(2x)   \\
                              \cos x  & -\sin x & 2\cos(2x)  \\
                              -\sin x & -\cos x & -4\sin(2x)
                        \end{vNiceMatrix} = 3 \sin{(2x)}
            \end{align}

      \item The given functions are {\color{y_p} not L.I.} on the interval $ x \geq 0 $,
            \par
            either using the superposition, $ 2\pi\cos^{2}x + 2\pi\sin^{2}x = 2\pi $ or,
            \begin{align}
                  W & = \begin{vNiceMatrix}[r, margin]
                              \sin^{2} x & \cos^{2} x & 2\pi \\
                              \sin(2x)   & -\sin(2x)  & 0    \\
                              2\cos(2x)  & -2\cos(2x) & 0
                        \end{vNiceMatrix} = 0
            \end{align}

      \item The given functions are {\color{y_p} not L.I.} on the interval $ x \geq 0 $,
            \par
            either using the superposition, $ \cosh 2x + \sinh 2x = e^{2x} $ or,
            \begin{align}
                  W & = \begin{vNiceMatrix}[r, margin]
                              \cosh 2x  & \sinh 2x  & e^{2x}  \\
                              2\sinh 2x & 2\cosh 2x & 2e^{2x} \\
                              4\cosh 2x & 4\sinh 2x & 4e^{2x}
                        \end{vNiceMatrix} = 0
            \end{align}

      \item \begin{enumerate}
                  \item Checking each condition,
                        \begin{enumerate}[itemsep = 2em]
                              \item If $ S $ contains the zero function, can $ S $
                                    be linearly independent? \par
                                    {\color{y_p} No}, if $ 0 \in S $, then S cannot be
                                    an L.I. set
                              \item If $ S $ is linearly independent on a subinterval
                                    $\mathcal{J}$ of $\mathcal{I}$, is it linearly
                                    independent on $ \mathcal{I} $? \par
                                    {\color{y_h} Yes}, since $ W \equiv 0 $ has already
                                    been violated when the set $ S $ is L.I. on
                                    $ \mathcal{J} $
                              \item If $S$ is linearly dependent on a subinterval
                                    $\mathcal{J}$ of $\mathcal{I}$, is it linearly
                                    dependent on $\mathcal{I}$?
                                    \par {\color{y_p} No}, since $ W \neq 0$ may be true
                                    for some $ x \in \mathcal{I} - \mathcal{J} $
                              \item If $S$ is linearly independent on $\mathcal{I}$, is
                                    it linearly independent on a subinterval
                                    $\mathcal{J}$?
                                    \par {\color{y_p} No}, since the set $ S $ might
                                    have $ W \equiv 0 \ \forall\ \mathcal{J}$
                                    but $ W \neq 0 $ for some point in $ \mathcal{I}
                                          - \mathcal{J} $
                              \item If $S$ is linearly dependent on $\mathcal{I}$, is it
                                    linearly independent on a subinterval $\mathcal{J}$?
                                    \par {\color{y_p} No}, since $ W \equiv 0\ \forall
                                          \ x \in \mathcal{I} $, this has to hold for all
                                    subintervals $ \mathcal{J} $ as well
                              \item If $S$ is linearly dependent on $\mathcal{I}$, and
                                    if $T$ contains $S$, is $T$ linearly dependent on
                                    $\mathcal{I}$? \par
                                    {\color{y_p} No}, since one member of $ S $ can be
                                    expressed as a linear superposition of other members
                                    of $ S $.
                                    This means that the superset $ T $ is also L.D.
                        \end{enumerate}
                  \item To use the Wronskian on a set of functions $ S $ to test their
                        linear independence, they have to be $ n $ times differentiable
                        on the interval $ \mathcal{I} $ of testing, but need not be
                        continuous.
                        \par Other methods might include simple ratios of member
                        functions in $ S $ to check for linear independence, or the use
                        of pre-established relations between member functions to come
                        up with linear superpositions which disprove L.I.
            \end{enumerate}


\end{enumerate}