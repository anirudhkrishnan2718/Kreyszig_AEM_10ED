\section{Matrix Multiplication}
\begin{enumerate}
    \item Matrix multiplication requires the two matrices to have the same inner
          dimension because the linear transform requires every output variable to be
          defined as a linear function of every input variable.
          \begin{align}
              y_j        & = f(x_1, x_2,\dots,x_n) \\
              \forall\ j & \in \{1,\dots,m\}
          \end{align}

    \item A matrix is skew-symmetric and symmetric. This means that
          \begin{align}
              \vec{A}          & = \vec{A}^T & \vec{A} & = -\vec{A}^T \\
              \implies \vec{A} & = \vec{0}
          \end{align}

    \item Outer product of a column and row vector, is
          \begin{align}
              \begin{bmatrix*}[r]
                  a_1 \\ a_2 \\ a_3
              \end{bmatrix*} \begin{bmatrix*}[r]
                                 b_1 & b_2 & b_3
                             \end{bmatrix*} = \begin{bmatrix*}[r]
                                                  a_1b_1 & a_1b_2 & a_1b_3 \\
                                                  a_2b_1 & a_2b_2 & a_2b_3 \\
                                                  a_3b_1 & a_3b_2 & a_3b_3
                                              \end{bmatrix*}
          \end{align}
          The ratio of any two columns is fixed as is the ratio of any two rows. This
          means that not every square matrix can be represented using the above form.

    \item A skew symmetric matrix of order $ n $ has to have all diagonal entries
          be zero. All lower traingular entries can be distinct. So the number of
          different entries is 6 (for order 4) and $ 0.5(n^2 - n) $ for order $ n $.

    \item For symmetric matrices, the diagonal elements are unconstrained. The
          lower triangular and upper triangular entries match. So the number of distinct
          entries is $ 0.5(n^2 + n) $. For order 4, this is 10 entries.

    \item Verifying algebra of triangular matrices,
          \begin{align}
              \vec{U}_1 + \vec{U}_2 & \qquad  \text{is upper triangular} \\
              \vec{U}_1\vec{U}_2    & \qquad  \text{is upper triangular} \\
              \vec{U}_1^2           & \qquad  \text{is upper triangular} \\
              \vec{U}_1 + \vec{L}_1 & \qquad  \text{is not triangular}   \\
              \vec{U}_1 \vec{L}_1   & \qquad  \text{is not triangular}   \\
              \vec{L}_1 \vec{L}_2   & \qquad  \text{is lower triangular}
          \end{align}

    \item Idempotent matrix means,
          \begin{align}
              \vec{A}^2 = \vec{A}
          \end{align}
          Consider the general $ 2 \times 2 $ case,
          \begin{align}
              \begin{bmatrix*}[r]
                  a^2 + bc & ab + bd \\
                  ac + dc & bc + d^2
              \end{bmatrix*} & = \bmattt{a}{b}{c}{d}                                  \\
              b                     & = 0                   & \text{or}\  & (a+d) = 1 \\
              c                     & = 0                   & \text{or}\  & (a+d) = 1 \\
              a(1 - a)              & = bc                                            \\
              d(1 - d)              & = bc                                            \\
          \end{align}
          If only one among $ b $ or $ c $ is zero, then $ (a + d) = 1 $. \par
          If both $ b $ and $ c $ are zero, then the matrix is diagonal with $ a $ and
          $ d $ limited to the set $ \{0, 1\} $.

    \item Nilpotent matrix defined for some positive integer $ m $,
          \begin{align}
              \vec{B}^m = \vec{0} \\
          \end{align}
          Examples are any traingular matrix with zero along the diagonal.
          \begin{align}
              \bmattt{0}{a}{0}{0} \qquad \bmattt{0}{0}{b}{0} \qquad
              \bmattt{c}{c}{-c}{-c}
          \end{align}

    \item Transpose is an involution,
          \begin{align}
              \vec{B}  & = \vec{A}^T                         &
              \vec{C}  & = \vec{B}^T                           \\
              [b_{jk}] & = [a_{kj}]                          &
              [c_{jk}] & = [b_{kj}] = [a_{jk}]                 \\
              \vec{C}  & = \Big( \vec{A}^T \Big)^T = \vec{A}
          \end{align}
          Transpose is associative under addtion,
          \begin{align}
              \vec{C}   & = \vec{A} + \vec{B}     &
              [c_{jk}]  & = [a_{jk} + b_{jk}]       \\
              \vec{D}   & = \vec{A}^T + \vec{B}^T &
              [d_{jk}]  & = [a_{kj} + b_{kj}]       \\
              \vec{C}^T & = \vec{D}
          \end{align}
          Transpose is associative under scalar multiplication,
          \begin{align}
              (c\vec{A})^T & = [c\ a_{kj}] = c\ [a_{kj}] = c\ \vec{A}^T
          \end{align}

    \item To show the transpose of the product of two matrices,
          \begin{align}
              \vec{C}              & = \vec{AB}                      &
              [c_{jk}]
                                   & = \sum_{l=1}^{n} a_{jl}\ b_{lk}   \\
              \vec{C}^T            & = \vec{D}                       &
              [d_{jk}] = [c_{kj}]
                                   & = \sum_{l=1}^{n} a_{kl}\ b_{lj}   \\
              \vec{B}^T\ \vec{A}^T & = \vec{E}                       &
              [e_{jk}]             & = \sum_{m=1}^{n} b_{mj}\ a_{km}   \\
              \vec{D}              & = \vec{E}                       &
          \end{align}
          This proves the relation since the two expressions are identical but for the
          choice of dummy variable $ l,m $ inside the summation.

    \item Performing the given matrix multiplications,
          \begin{align}
              \vec{AB} & = \begin{bNiceMatrix}[r, margin]
                               4  & -2 & 3 \\
                               -2 & 1  & 6 \\
                               1  & 2  & 2
                           \end{bNiceMatrix} \begin{bNiceMatrix}[r, margin]
                                                 1  & -3 & 0  \\
                                                 -3 & 1  & 0  \\
                                                 0  & 0  & -2
                                             \end{bNiceMatrix} =
              \begin{bNiceMatrix}[r, margin]
                  10 & -14 & -6  \\
                  -5 & 7   & -12 \\
                  -5 & -1  & -4
              \end{bNiceMatrix}                                \\
          \end{align}
          Not showing the intermediate steps since they are all mental arithmetic.
          \begin{align}
              \vec{A}\vec{B}^T   & = \vec{AB}                                   &
              \vec{BA}           & = \begin{bNiceMatrix}[r, margin]
                                         10 & -5 & -15 \\ -14 & 7 & -3 \\-2 & -4 & -4
                                     \end{bNiceMatrix}  \\
              \vec{B}^T\ \vec{A} & = \vec{BA}
          \end{align}

    \item Performing the given matrix multiplication,
          \begin{align}
              \vec{AA}^T & = \begin{bNiceMatrix}[r, margin]
                                 29 & 8  & 6  \\
                                 8  & 41 & 12 \\
                                 6  & 12 & 9
                             \end{bNiceMatrix} &
              \vec{A}^2  & = \begin{bNiceMatrix}[r, margin]
                                 23 & -4 & 6  \\
                                 -4 & 17 & 12 \\
                                 2  & 4  & 19
                             \end{bNiceMatrix} \\
              \vec{BB}^T & = \begin{bNiceMatrix}[r, margin]
                                 10 & -6 & 0 \\
                                 -6 & 10 & 0 \\
                                 0  & 0  & 4
                             \end{bNiceMatrix} &
              \vec{B}^2  & = \vec{BB}^T
          \end{align}

    \item Performing the given matrix multiplication,
          \begin{align}
              \vec{CC}^T         & = \begin{bNiceMatrix}[r, margin]
                                         1 & 2  & 0  \\
                                         2 & 13 & -6 \\
                                         0 & -6 & 4
                                     \end{bNiceMatrix} &
              \vec{BC}           & = \begin{bNiceMatrix}[r, margin]
                                         -9 & -5 \\
                                         3  & -1 \\
                                         4  & 0
                                     \end{bNiceMatrix}    \\
              \vec{CB}           & = \text{not defined}             &
              \vec{C}^T\ \vec{B} & = \begin{bNiceMatrix}[r, margin]
                                         -9 & 3  & 4 \\
                                         -5 & -1 & 0
                                     \end{bNiceMatrix}
          \end{align}

    \item Performing the given matrix multiplication,
          \begin{align}
              3\vec{A} - 2\vec{B}                & = \begin{bNiceMatrix}[r, margin]
                                                         10 & 0 & 9  \\
                                                         0  & 1 & 18 \\
                                                         3  & 6 & 10
                                                     \end{bNiceMatrix} \\
              (3\vec{A} - 2\vec{B})^T            & = \begin{bNiceMatrix}[r, margin]
                                                         10 & 0  & 3  \\
                                                         0  & 1  & 6  \\
                                                         9  & 18 & 10
                                                     \end{bNiceMatrix} \\
              3\vec{A}^T - 2\vec{B}^T            & = \begin{bNiceMatrix}[r, margin]
                                                         12 & -6 & 3 \\
                                                         -6 & 3  & 6 \\
                                                         9  & 18 & 6
                                                     \end{bNiceMatrix}
              - \begin{bNiceMatrix}[r, margin]
                    2  & -6 & 0  \\
                    -6 & 2  & 0  \\
                    0  & 0  & -4
                \end{bNiceMatrix}                                      \\
              (3\vec{A} - 2\vec{B})^T\ \vec{a}^T & = \begin{bNiceMatrix}[r, margin]
                                                         10 & 0  & 3  \\
                                                         0  & 1  & 6  \\
                                                         9  & 18 & 10
                                                     \end{bNiceMatrix}
              \begin{bNiceMatrix}[r, margin]
                  1 \\ -2 \\ 0
              \end{bNiceMatrix} = \begin{bNiceMatrix}[r, margin]
                                      10 \\ -2 \\ -27
                                  \end{bNiceMatrix}
          \end{align}

    \item Performing the given matrix multiplication,
          \begin{align}
              \vec{Aa}             & = \text{not defined}             &
              \vec{A}\ \vec{a}^T   & = \begin{bNiceMatrix}[r, margin]
                                           10 \\ -2 \\ -3
                                       \end{bNiceMatrix}    \\
              (\vec{Ab})^T         & = \begin{bNiceMatrix}[r, margin]
                                           7 & -11 & 3
                                       \end{bNiceMatrix} &
              \vec{b}^T\ \vec{A}^T & = \vec{Ab}^T
          \end{align}

    \item Performing the given matrix multiplication,
          \begin{align}
              \vec{BC}           & = \begin{bNiceMatrix}[r, margin]
                                         -9 & -5 \\ 3 & -1 \\ 4 & 0
                                     \end{bNiceMatrix} &
              \vec{B}\ \vec{C}^T & = \text{not defined}            \\
              (\vec{Bb})         & = \begin{bNiceMatrix}[r, margin]
                                         0 & -8 & 2
                                     \end{bNiceMatrix} &
              \vec{b}^T\ \vec{B} & = \begin{bNiceMatrix}[r, margin]
                                         0 \\ -8 \\ 2
                                     \end{bNiceMatrix}
          \end{align}

    \item Performing the given matrix multiplication,
          \begin{align}
              \vec{ABC}                        & = \vec{A} \cdot
              \begin{bNiceMatrix}[r, margin]
                  -9 & -5 \\ 3 & -1 \\ 4 & 0
              \end{bNiceMatrix}
              = \begin{bNiceMatrix}[r, margin]
                    -30 & -18 \\45 & 9 \\ 5 & -7
                \end{bNiceMatrix} &
              \vec{ABa}                        & = \text{not defined} \\
              \vec{ABb}                        & = \vec{A} \cdot
              \begin{bNiceMatrix}[r, margin]
                  0 \\ -8 \\ 2
              \end{bNiceMatrix} =
              \begin{bNiceMatrix}[r, margin]
                  22 \\ 4 \\ -12
              \end{bNiceMatrix}   &
              \vec{C}\ \vec{a}^T               & = \text{not defined}
          \end{align}

    \item Performing the given matrix multiplication,
          \begin{align}
              \vec{ab} & =\begin{bNiceMatrix}[r, margin]
                              1
                          \end{bNiceMatrix}
                       &
              \vec{ba} & = \begin{bNiceMatrix}[r, margin]
                               3  & -6 & 0 \\
                               1  & -2 & 0 \\
                               -1 & 2  & 0
                           \end{bNiceMatrix} \\
              \vec{aA} & = \begin{bNiceMatrix}[r, margin]
                               8 & -4 & -9
                           \end{bNiceMatrix} &
              \vec{Bb} & = \begin{bNiceMatrix}[r, margin]
                               0 \\ -8 \\ 2
                           \end{bNiceMatrix}
          \end{align}

    \item Performing the given matrix multiplication,
          \begin{align}
              1.5\vec{a} + 3\vec{b}            & =\text{not defined}              &
              1.5\vec{a}^T + 3\vec{b}          & = \begin{bNiceMatrix}[r, margin]
                                                       1.5 \\ 0 \\ -3
                                                   \end{bNiceMatrix}    \\
              (\vec{A} - \vec{B})\vec{b}       & = \begin{bNiceMatrix}[r, margin]
                                                       3 & 1 & 3 \\
                                                       1 & 0 & 6 \\
                                                       1 & 2 & 4
                                                   \end{bNiceMatrix} \cdot \vec{b}
              = \begin{bNiceMatrix}[r, margin]
                    7 \\ -3 \\ 1
                \end{bNiceMatrix} &
              \vec{Ab} - \vec{Bb}              & = (\vec{A} - \vec{B})\vec{b}
          \end{align}

    \item Performing the given matrix multiplication,
          \begin{align}
              \vec{b}^T\ \vec{Ab} & =\vec{b}^T \cdot \begin{bNiceMatrix}[r, margin]
                                                         7 \\ -11 \\ 3
                                                     \end{bNiceMatrix}
              = \begin{bNiceMatrix}[r, margin]
                    7
                \end{bNiceMatrix}                                      \\
              \vec{aB}\ \vec{a}^T & = \begin{bNiceMatrix}[r, margin]
                                          7 & -5 & 0
                                      \end{bNiceMatrix} \cdot \vec{a}^T
              = \begin{bNiceMatrix}[r, margin]
                    17
                \end{bNiceMatrix}                                      \\
              \vec{aC}\ \vec{C}^T & = \vec{a} \cdot \begin{bNiceMatrix}[r, margin]
                                                        1 & 2  & 0  \\
                                                        2 & 13 & -6 \\
                                                        0 & -6 & 4
                                                    \end{bNiceMatrix}
              = \begin{bNiceMatrix}[r, margin]
                    3 & -4 & 12
                \end{bNiceMatrix}                                      \\
              \vec{C}^T\ \vec{ba} & = \begin{bNiceMatrix}[r, margin]
                                          5 \\ 5
                                      \end{bNiceMatrix} \cdot \vec{a}
              = \begin{bNiceMatrix}[r, margin]
                    5 & -10 & 0 \\ 5 & -10 & 0
                \end{bNiceMatrix}
          \end{align}

    \item Proving the relations for $ 2 \times 2 $ matrices,
          \begin{align}
              (k\vec{A})\ \vec{B} & = \bmattt{ka_{11}}{ka_{12}}{ka_{21}}{ka_{22}}
              \cdot \bmattt{b_{11}}{b_{12}}{b_{21}}{b_{22}} =
              \bmattt{ka_{11}b_{11} + ka_{12}b_{21}}
              {ka_{11}b_{12} + ka_{12}b_{22}}
              {ka_{21}b_{11} + ka_{22}b_{21}}
              {ka_{21}b_{12} + ka_{22}b_{22}}                                      \\
                                  & = k \cdot \bmattt{a_{11}b_{11} + a_{12}b_{21}}
              {a_{11}b_{12} + a_{12}b_{22}}
              {a_{21}b_{11} + a_{22}b_{21}}
              {a_{21}b_{12} + a_{22}b_{22}} = k\ (\vec{AB})
          \end{align}

          Proving $ 2b $,
          \begin{align}
              (\vec{AB})\ \vec{C} & = \bmattt{a_{11}b_{11} + a_{12}b_{21}}
              {a_{11}b_{12} + a_{12}b_{22}}
              {a_{21}b_{11} + a_{22}b_{21}}
              {a_{21}b_{12} + a_{22}b_{22}} \cdot
              \bmattt{c_{11}}{c_{12}}{c_{21}}{c_{22}}                         \\
              \vec{A}\ (\vec{BC}) & = \bmattt{a_{11}}{a_{12}}{a_{21}}{a_{22}}
              \cdot \bmattt{b_{11}c_{11} + b_{12}c_{21}}
              {b_{11}c_{12} + b_{12}c_{22}}
              {b_{21}c_{11} + b_{22}c_{21}}
              {b_{21}c_{12} + b_{22}c_{22}}
          \end{align}
          The elementwise comparison of the two lines yields the equality.

          Proving $ 2c $, using only the first element,
          \begin{align}
              \Big[ (\vec{A} + \vec{B})\ \vec{C} \Big]_{11} &
              = (a_{11} + b_{11})c_{11} + (a_{12} + b_{12})c_{21} \\
              \Big[ (\vec{AC} + \vec{BC}) \Big]_{11}        &
              = (a_{11}c_{11} + b_{11}c_{11}) + (a_{12}c_{21} + b_{12}c_{21})
          \end{align}
          The equality follows from the distributive law of scalar multiplication.

          Proving $ 2d $, using the first element,
          \begin{align}
              \Big[\vec{C}\ (\vec{A} + \vec{B}) \Big]_{11} &
              = c_{11}(a_{11} + b_{11}) + c_{21}(a_{12} + b_{12}) \\
              \Big[ (\vec{CA} + \vec{CB}) \Big]_{11}       &
              = (c_{11}a_{11} + c_{11}b_{11}) + (c_{12}a_{12} + c_{12}b_{12})
          \end{align}
          The equality follows from the distributive law of scalar multiplication.

    \item Expressing matrix multiplication in terms of row and column vectors, using
          superscipt and subscript to indicate a row and column of the matrix.
          \begin{align}
              \vec{AB} & = \begin{bNiceMatrix}[r, margin]
                               \vec{A}\vec{b}_1 & \vec{A}\vec{b}_2 & \vec{A}\vec{b}_3
                           \end{bNiceMatrix}
              = \begin{bNiceMatrix}[r, margin]
                    \vec{a}^1 \vec{b}_1 & \vec{a}^2 \vec{b}_1 & \vec{a}^3 \vec{b}_1 \\
                    \vec{a}^1 \vec{b}_2 & \vec{a}^2 \vec{b}_2 & \vec{a}^3 \vec{b}_2 \\
                    \vec{a}^1 \vec{b}_3 & \vec{a}^2 \vec{b}_3 & \vec{a}^3 \vec{b}_3
                \end{bNiceMatrix}
          \end{align}
          This convention is used for this problem and not present in the text.

    \item Calculating the product columnwise,
          \begin{align}
              \vec{AB} & = \begin{bNiceMatrix}[r, margin]
                               \vec{A}\vec{b}_1 & \vec{A}\vec{b}_2 & \vec{A}\vec{b}_3
                           \end{bNiceMatrix}
              = \begin{bNiceMatrix}[r, margin]
                    10 & -14 & -6 \\ -5 & 7 & -12 \\ -5 & -1 & -4
                \end{bNiceMatrix}
          \end{align}

    \item Given the constraint on $ \vec{B} $,
          \begin{align}
              b_{jk}                                      & = j + k             &
              \vec{AB}                                    & = \vec{BA}            \\
              \begin{bNiceMatrix}[r, margin]
                  a_{11} & a_{12} \\ a_{21} & a_{22}
              \end{bNiceMatrix} \cdot \bmattt{2}{3}{3}{4} & =
              \bmattt{2}{3}{3}{4} \cdot \begin{bNiceMatrix}[r, margin]
                                            a_{11} & a_{12} \\ a_{21} & a_{22}
                                        \end{bNiceMatrix}         \\
              2a_{11} + 3a_{12}                           & = 2a_{11} + 3a_{21} &
              3a_{11} + 4a_{12}                           & = 2a_{12} + 3a_{22}   \\
              2a_{21} + 3a_{22}                           & = 3a_{11} + 4a_{21} &
              3a_{21} + 4a_{22}                           & = 3a_{12} + 4a_{22}
          \end{align}
          This is a system of 4 equations 4 variables. Solving,
          \begin{align}
              a_{12}            & = a_{21}                               &
              3a_{11} + 2a_{12} & = 3a_{11} + 2a_{21}                      \\
              a_{22}            & = a_{11} + \frac{2}{3}\ a_{12}         &
              a_{11}            & = \text{free}                            \\
              \vec{A}           & = \bmattt{x}{y}{y}{(x + \frac{2y}{3})}
          \end{align}

    \item Symmetric matrices
          \begin{enumerate}
              \item Verifying,
                    \begin{align}
                        \vec{A} & = \vec{A}^T  & \implies a_{jk} & = a_{kj}  \\
                        \vec{A} & = -\vec{A}^T & \implies a_{jk} & = -a_{kj}
                    \end{align}
                    Examples TBC.
              \item Writing the matrix $ C $ in the given form,
                    \begin{align}
                        (\vec{C} + \vec{C}^T) & = \vec{D}                           &
                        \vec{D}^T             & = \vec{C}^T + \Big(\vec{C}^T\Big)^T   \\
                        \vec{D}^T             & = \vec{D}                             \\
                        (\vec{C} - \vec{C}^T) & = \vec{E}                           &
                        \vec{E}^T             & = \vec{C}^T - \Big(\vec{C}^T\Big)^T   \\
                        \vec{E}^T             & = -\vec{E}
                    \end{align}
                    Using the previous result as a starting point,
                    \begin{align}
                        \vec{A} & = \frac{(\vec{A} + \vec{A}^T)
                        + (\vec{A} - \vec{A}^T)}{2}               \\
                        \vec{S} & = 0.5(\vec{A} + \vec{A}^T)
                                & \vec{T}                       &
                        = 0.5(\vec{A} - \vec{A}^T)
                    \end{align}
                    Representing $ A $ and $ B $ in this form,
                    \begin{align}
                        \vec{A} & = \begin{bNiceMatrix}[r, margin]
                                        4 & -2 & 2 \\ -2 & 1 & 4 \\ 2 & 4 & 2
                                    \end{bNiceMatrix} +
                        \begin{bNiceMatrix}[r, margin]
                            0 & 0 & 1 \\ 0 & 0 & 2 \\ -1 & -2 & 0
                        \end{bNiceMatrix}              \\
                        \vec{B} & = \begin{bNiceMatrix}[r, margin]
                                        1 & -3 & 0 \\ -3 & 1 & 0 \\ 0 & 0 & -2
                                    \end{bNiceMatrix} +
                        \begin{bNiceMatrix}[r, margin]
                            0 & 0 & 0 \\ 0 & 0 & 0 \\ 0 & 0 & 0
                        \end{bNiceMatrix}
                    \end{align}
              \item Let the matrices be $ \{\vec{A}^i\} $ and the scalar coefficients
                    $ \{\lambda_i\} $,
                    \begin{align}
                        \vec{S} & = \sum_{i=1}^{n} \lambda_i \vec{A}^i               &
                        S_{jk}  & = \sum_{i=1}^{n} \lambda_i \vec{A}^i_{jk}            \\
                        S_{jk}  & = \sum_{i=1}^{n} \lambda_i \vec{A}^i_{kj} = S_{kj} &
                        \vec{S} & = \vec{S}^T
                    \end{align}
                    Repeating the procedure for skew symmetric matrices
                    $ \{\vec{B}^i\} $,
                    \begin{align}
                        \vec{P}   & = \sum_{i=1}^{n} \lambda_i \vec{B}^i       &
                        P_{jk}    & = \sum_{i=1}^{n} \lambda_i \vec{B}^i_{jk}    \\
                        P_{jk}    & = -\sum_{i=1}^{n} \lambda_i \vec{B}^i_{kj}
                        = -P_{kj} &
                        \vec{P}   & = -\vec{P}^T
                    \end{align}

              \item Given $ \vec{A} $ and $ \vec{B} $ are symmetric,
                    \begin{align}
                        (\vec{AB})^T      & = \vec{B}^T\ \vec{A}^T = \vec{BA}   \\
                        (\vec{AB})^T      & = \vec{AB}                        &
                        \implies \vec{BA} & = \vec{AB}
                    \end{align}
              \item Given $ \vec{A} $ and $ \vec{B} $ are skew-symmetric,
                    \begin{align}
                        (\vec{AB})^T      & = \vec{B}^T\ \vec{A}^T
                        = \vec{-B} \cdot \vec{-A}                    \\
                        (\vec{AB})^T      & = -\vec{AB}            &
                        \implies \vec{BA} & = -\vec{AB}
                    \end{align}
          \end{enumerate}

    \item Building the transition matrix, where the column and row indices are the
          start and end states respectively.
          \begin{align}
              \vec{T}   & = \bmattt{0.8}{0.5}{0.2}{0.5}         &
              \vec{A}_0 & = \bmatcol{1}{0}                        \\
              \vec{A}_1 & = \vec{TA}_0 = \bmatcol{0.8}{0.2}     &
              \vec{A}_2 & = \vec{TA}_1 = \bmatcol{0.74}{0.26}     \\
              \vec{A}_3 & = \vec{TA}_2 = \bmatcol{0.722}{0.278} &
          \end{align}

    \item Calculating the first few steps of the Markov matrix in Example 13,
          \begin{align}
              \vec{T}   & = \begin{bNiceMatrix}[r, margin]
                                0.7 & 0.1 & 0 \\ 0.2 & 0.9 & 0.2 \\ 0.1 & 0 & 0.8
                            \end{bNiceMatrix}           \\
              \vec{S}^i & = \begin{bNiceMatrix}[r, margin]25\\20\\55\end{bNiceMatrix},
              \begin{bNiceMatrix}[r, margin]19.5 \\
                  34.0            \\
                  46.5\end{bNiceMatrix},
              \begin{bNiceMatrix}[r, margin]17.05 \\
                  43.8             \\3
                  9.15\end{bNiceMatrix},
              \begin{bNiceMatrix}[r, margin]16.31 \\
                  50.66            \\
                  33.02\end{bNiceMatrix},
              \begin{bNiceMatrix}[r, margin]16.48 \\
                  55.46            \\
                  28.05\end{bNiceMatrix},
              \begin{bNiceMatrix}[r, margin]17.08 \\
                  58.82            \\
                  24.08\end{bNiceMatrix}                                      \\
                        &
              \begin{bNiceMatrix}[r, margin]17.84 \\
                  61.17            \\
                  20.98\end{bNiceMatrix},
              \begin{bNiceMatrix}[r, margin]18.60 \\
                  62.82            \\
                  18.56\end{bNiceMatrix},
              \begin{bNiceMatrix}[r, margin]19.30 \\
                  63.97            \\
                  16.71\end{bNiceMatrix},
              \begin{bNiceMatrix}[r, margin]19.91 \\
                  64.78            \\
                  15.30\end{bNiceMatrix},
              \begin{bNiceMatrix}[r, margin]20.41 \\
                  65.34            \\
                  14.23\end{bNiceMatrix},
          \end{align}

          Since other transition matrices and other initial conditions involve the
          same procedure, they are omitted. TBC

    \item The transition matrix and I.C. are,
          \begin{align}
              \vec{T}   & = \begin{bNiceMatrix}[r, margin]
                                0.9 & 0.002 \\ 0.1 & 0.998
                            \end{bNiceMatrix}         &
              \vec{S}^0 & = \begin{bNiceMatrix}[r, margin]
                                1200 \\ 98800
                            \end{bNiceMatrix}       \\
              \vec{S}^1 & = \begin{bNiceMatrix}[r, margin]1278 \\
                                98722\end{bNiceMatrix}  &
              \vec{S}^2 & = \begin{bNiceMatrix}[r, margin]1347 \\
                                98653\end{bNiceMatrix}  \\
              \vec{S}^3 & = \begin{bNiceMatrix}[r, margin]1409. \\
                                98591.\end{bNiceMatrix},
          \end{align}

    \item Defining the profit matrix, (solutions are wrong because they have a typo
          in the first entry of $ \vec{p} $)
          \begin{align}
              \vec{A} & = \begin{bNiceMatrix}[r, margin]
                              400 & 60 & 240 \\ 100 & 120 & 500
                          \end{bNiceMatrix}           &
              \vec{p} & = \begin{bNiceMatrix}[r, margin]
                              35 \\ 62 \\ 30
                          \end{bNiceMatrix}            &
              \vec{v} & = \vec{Ap} = \begin{bNiceMatrix}[r, margin]
                                         24920 \\ 25940
                                     \end{bNiceMatrix}
          \end{align}

    \item Rotation matrices,
          \begin{enumerate}
              \item Single rotation counterclockwise by $ \theta $. Let $ R = 1 $ for
                    convenience,
                    \begin{align}
                        (x_1, x_2)         & = \{\cos(\alpha), \sin(\alpha)\}       \\
                        (y_1, y_2)         & = \{\cos(\alpha + \theta),
                        \sin(\alpha+ \theta)\}                                      \\
                        y_1                & = x_1 \cos(\theta) - x_2 \sin(\theta)  \\
                        y_2                & = x_2 \cos(\theta) + x_1 \sin(\theta)  \\
                        \bmatcol{y_1}{y_2} & = \bmattt{\cos(\theta)}{-\sin(\theta)}
                        {\sin(\theta)}{\cos(\theta)} \cdot \bmatcol{x_2}{x_2}
                    \end{align}

              \item Repeated rotation by angle $ \theta $,
                    \begin{align}
                        \vec{A}(n\theta)\ \vec{A}(\theta) & = \bmattt
                        {\cos(n\theta)}{-\sin(n\theta)}
                        {\sin(n\theta)}{\cos(n\theta)} \cdot
                        \bmattt{\cos(\theta)}{-\sin(\theta)}
                        {\sin(\theta)}{\cos(\theta)}                  \\
                                                          & = \bmattt
                        {\cos(n\theta +\theta)}{-\sin(n\theta +\theta)}
                        {\sin(n\theta +\theta)}{\cos(n\theta +\theta)}
                        = \vec{A}\{(n+1)\theta\}
                    \end{align}
                    By induction, this holds for all $ n $ since it can be proved by
                    brute force that it holds for $ n = 1 $. \par
                    \begin{align}
                        \vec{A}(n\theta) & = [\vec{A}(\theta)]^n
                    \end{align}

              \item Using the repeated rotation by $ \alpha $ and $ \beta $, and
                    matri multiplication representation,
                    \begin{align}
                        \cos(\alpha + \beta) & = \cos \alpha \cos \beta
                        - \sin \alpha \sin \beta                        \\
                        \cos(\alpha - \beta) & = \cos \alpha \cos \beta
                        + \sin \alpha \sin \beta                        \\
                        \sin(\alpha + \beta) & = \sin \alpha \cos \beta
                        + \cos \alpha \sin \beta                        \\
                        \sin(\alpha - \beta) & = \sin \alpha \cos \beta
                        - \cos \alpha \sin \beta
                    \end{align}
                    The subtraction formulas depend on the fact that $ \sin(-x) =
                        -\sin(x) $ whereas $ \cos(-x) = \cos(x) $

              \item A scalar matrix would scale all 3 coordinates equally by the scalar
                    $ k $.
                    \begin{align}
                        \begin{bNiceMatrix}[r, margin]
                            3 & 0 & 0 \\ 0 & 1 & 0 \\ 0 & 0 & 1/2
                        \end{bNiceMatrix} \cdot \begin{bNiceMatrix}[r, margin]
                                                    x_1 \\ x_2 \\ x_3
                                                \end{bNiceMatrix} & =
                        \begin{bNiceMatrix}[r, margin]
                            3x_1 \\ x_2 \\ 0.5x_3
                        \end{bNiceMatrix}                            \\
                        \vec{Dx}                                               &
                        = \vec{y}
                    \end{align}

              \item The row index contatining the ones and zeros represents the
                    coordinate that will be unchanged. The other two coordinates undergo
                    rotation by $ \phi $ in the counterclockwise direction. \par
                    The three matrices represent rotations in $ yz, xz $ and $ xy $ planes
                    respectively.
          \end{enumerate}
\end{enumerate}