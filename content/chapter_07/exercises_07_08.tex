\section{Inverse of a Matrix, Gauss-Jordan Elimination}
\begin{enumerate}
    \item Finding inverse using direct formula
          \begin{align}
              \vec{A}       & = \begin{bNiceMatrix}[r, margin]
                                    1.8 & -2.32 \\ -0.25 & 0.6
                                \end{bNiceMatrix}         &
              \det(\vec{A}) & = 1.8(0.6) - 0.25(2.32) = \frac{1}{2}   \\
              \vec{A}^{-1}  & = 2 \cdot \begin{bNiceMatrix}[r, margin]
                                            0.6 & 2.32 \\ 0.25 & 1.8
                                        \end{bNiceMatrix} &
              \vec{A}^{-1}  & = \begin{bNiceMatrix}[r, margin]
                                    1.2 & 4.64 \\ 0.5 & 3.6
                                \end{bNiceMatrix}
          \end{align}
          Verifying,
          \begin{align}
              \vec{AA}^{-1}       &
              = \begin{bNiceMatrix}[r, margin]
                    1.8(1.2) - 2.32(0.5)  & 1.8(4.64) - 2.32(3.6)  \\
                    -0.25(1.2) + 0.6(0.5) & -0.25(4.64) + 0.6(3.6)
                \end{bNiceMatrix}
              = \begin{bNiceMatrix}[r, margin]
                    1 & 0 \\
                    0 & 1
                \end{bNiceMatrix} \\
              \vec{A}^{-1}\vec{A} &
              = \begin{bNiceMatrix}[r, margin]
                    1.2(1.8) - 4.64(0.25)  & 0.5(1.8) - 3.6(0.25)  \\
                    1.2(-2.32) + 4.64(0.6) & 0.5(-2.32) + 3.6(0.6)
                \end{bNiceMatrix}
              = \begin{bNiceMatrix}[r, margin]
                    1 & 0 \\
                    0 & 1
                \end{bNiceMatrix}
          \end{align}

    \item Finding inverse using direct formula
          \begin{align}
              \vec{A}       & = \begin{bNiceMatrix}[r, margin]
                                    \cos(2\theta)  & \sin(2\theta) \\
                                    -\sin(2\theta) & \cos(2\theta)
                                \end{bNiceMatrix} &
              \det(\vec{A}) & = 1                              \\
              \vec{A}^{-1}  & = \begin{bNiceMatrix}[r, margin]
                                    \cos(2\theta) & -\sin(2\theta) \\
                                    \sin(2\theta) & \cos(2\theta)
                                \end{bNiceMatrix}
          \end{align}
          Verified using a CAS

    \item Finding inverse using Gauss-Jordan elimination
          \begin{align}
              \Big[\vec{A} | \vec{I} \Big]                         & =
              \begin{bNiceArray}{rrr|rrr}[margin]
                  \CodeBefore
                  \columncolor{y_h!10}{1-3}
                  \Body
                  0.3 & -0.1 & 0.5 & 1 & 0 & 0 \\
                  2   & 6    & 4   & 0 & 1 & 0 \\
                  5   & 0    & 9   & 0 & 0 & 1
              \end{bNiceArray}                  &   &
              = \begin{bNiceArray}{rrr|rrr}[margin]
                    0.3           & -0.1         & 0.5         &
                    1             & 0            & 0             \\
                    0             & \frac{20}{3} & \frac{2}{3} &
                    -\frac{20}{3} & 1            & 0             \\
                    0             & \frac{5}{3}  & \frac{2}{3} &
                    -\frac{50}{3} & 0            & 1
                \end{bNiceArray}           \\
                                                                   &
              = \begin{bNiceArray}{rrr|rrr}[margin]
                    0.3            & -0.1          & 0.5         &
                    1              & 0             & 0             \\
                    0              & \frac{20}{3}  & \frac{2}{3} &
                    - \frac{20}{3} & 1             & 0             \\
                    0              & 0             & \frac{1}{2} &
                    - \frac{45}{3} & - \frac{1}{4} & 1
                \end{bNiceArray}    &   &
              = \begin{bNiceArray}{rrr|rrr}[margin]
                    0.3            & 0             & \frac{51}{100} &
                    \frac{9}{10}   & \frac{3}{200} & 0                \\
                    0              & \frac{20}{3}  & \frac{2}{3}    &
                    - \frac{20}{3} & 1             & 0                \\
                    0              & 0             & \frac{1}{2}    &
                    - \frac{45}{3} & - \frac{1}{4} & 1
                \end{bNiceArray}      \\
                                                                   &
              = \begin{bNiceArray}{rrr|rrr}[margin]
                    0.3            & 0              & 0             &
                    \frac{81}{5}   & \frac{27}{100} & -1.02           \\
                    0              & \frac{20}{3}   & 0             &
                    \frac{40}{3}   & \frac{4}{3}    & - \frac{4}{3}   \\
                    0              & 0              & \frac{1}{2}   &
                    - \frac{45}{3} & - \frac{1}{4}  & 1
                \end{bNiceArray} &   &
              = \begin{bNiceArray}{rrr|rrr}[margin]
                    \CodeBefore
                    \columncolor{y_p!10}{4-6}
                    \Body
                    1 & 0 & 0 & 54  & 0.9  & -3.4 \\
                    0 & 1 & 0 & 2   & 0.2  & -0.2 \\
                    0 & 0 & 1 & -30 & -0.5 & 2
                \end{bNiceArray}
          \end{align}
          Verified using a CAS

    \item Finding inverse using Gauss-Jordan elimination
          \begin{align}
              \Big[\vec{A} | \vec{I} \Big]
               & = \begin{bNiceArray}{rrr|rrr}[margin]
                       \CodeBefore
                       \columncolor{y_h!10}{1-3}
                       \Body
                       0   & 0    & 0.1 & 1 & 0 & 0 \\
                       0   & -0.4 & 0   & 0 & 1 & 0 \\
                       2.5 & 0    & 0   & 0 & 0 & 1 \\
                   \end{bNiceArray} &
               & = \begin{bNiceArray}{rrr|rrr}[margin]
                       2.5 & 0    & 0   & 0 & 0 & 1 \\
                       0   & -0.4 & 0   & 0 & 1 & 0 \\
                       0   & 0    & 0.1 & 1 & 0 & 0 \\
                   \end{bNiceArray} \\
               & = \begin{bNiceArray}{rrr|rrr}[margin]
                       \CodeBefore
                       \columncolor{y_p!10}{4-6}
                       \Body
                       1 & 0 & 0 & 0  & 0    & 0.4 \\
                       0 & 1 & 0 & 0  & -2.5 & 0   \\
                       0 & 0 & 1 & 10 & 0    & 0   \\
                   \end{bNiceArray}
          \end{align}
          Verified using a CAS

    \item Finding inverse using Gauss-Jordan elimination
          \begin{align}
              \Big[\vec{A} | \vec{I} \Big]
               & = \begin{bNiceArray}{rrr|rrr}[margin]
                       \CodeBefore
                       \columncolor{y_h!10}{1-3}
                       \Body
                       1 & 0 & 0 & 1 & 0 & 0 \\
                       2 & 1 & 0 & 0 & 1 & 0 \\
                       5 & 4 & 1 & 0 & 0 & 1 \\
                   \end{bNiceArray}  &
               & =  \begin{bNiceArray}{rrr|rrr}[margin]
                        1 & 0 & 0 & 1  & 0 & 0 \\
                        0 & 1 & 0 & -2 & 1 & 0 \\
                        0 & 4 & 1 & -5 & 0 & 1 \\
                    \end{bNiceArray} \\
               & \begin{bNiceArray}{rrr|rrr}[margin]
                     \CodeBefore
                     \columncolor{y_p!10}{4-6}
                     \Body
                     1 & 0 & 0 & 1  & 0  & 0 \\
                     0 & 1 & 0 & -2 & 1  & 0 \\
                     0 & 0 & 1 & 3  & -4 & 1 \\
                 \end{bNiceArray}
          \end{align}
          Verified using a CAS

    \item Finding inverse using Gauss-Jordan elimination
          \begin{align}
              \Big[\vec{A} | \vec{I} \Big]
               & = \begin{bNiceArray}{rrr|rrr}[margin]
                       \CodeBefore
                       \columncolor{y_h!10}{1-3}
                       \Body
                       -4 & 0 & 0  & 1 & 0 & 0 \\
                       0  & 8 & 13 & 0 & 1 & 0 \\
                       0  & 3 & 5  & 0 & 0 & 1 \\
                   \end{bNiceArray}              &
               & =   \begin{bNiceArray}{rrr|rrr}[margin]
                         -4 & 0 & 0           & 1 & 0             & 0 \\
                         0  & 8 & 13          & 0 & 1             & 0 \\
                         0  & 0 & \frac{1}{8} & 0 & - \frac{3}{8} & 1 \\
                     \end{bNiceArray} \\
               & \begin{bNiceArray}{rrr|rrr}[margin]
                     -4 & 0 & 0           & 1 & 0             & 0    \\
                     0  & 8 & 0           & 0 & 40            & -104 \\
                     0  & 0 & \frac{1}{8} & 0 & - \frac{3}{8} & 1    \\
                 \end{bNiceArray} &
               & = \begin{bNiceArray}{rrr|rrr}[margin]
                       \CodeBefore
                       \columncolor{y_p!10}{4-6}
                       \Body
                       1 & 0 & 0 & -0.25 & 0  & 0   \\
                       0 & 1 & 0 & 0     & 5  & -13 \\
                       0 & 0 & 1 & 0     & -3 & 8   \\
                   \end{bNiceArray}
          \end{align}

    \item Finding inverse using Gauss-Jordan elimination
          \begin{align}
              \Big[\vec{A} | \vec{I} \Big]
               & = \begin{bNiceArray}{rrr|rrr}[margin]
                       \CodeBefore
                       \columncolor{y_h!10}{1-3}
                       \Body
                       0 & 1 & 0 & 1 & 0 & 0 \\
                       1 & 0 & 0 & 0 & 1 & 0 \\
                       0 & 0 & 1 & 0 & 0 & 1 \\
                   \end{bNiceArray}   &
               & =   \begin{bNiceArray}{rrr|rrr}[margin]
                         \CodeBefore
                         \columncolor{y_p!10}{4-6}
                         \Body
                         1 & 0 & 0 & 0 & 1 & 0 \\
                         0 & 1 & 0 & 1 & 0 & 0 \\
                         0 & 0 & 1 & 0 & 0 & 1 \\
                     \end{bNiceArray}
          \end{align}
          Verified using a CAS

    \item Finding inverse using Gauss-Jordan elimination
          \begin{align}
              \Big[\vec{A} | \vec{I} \Big]
               & = \begin{bNiceArray}{rrr|rrr}[margin]
                       \CodeBefore
                       \columncolor{y_h!10}{1-3}
                       \Body
                       1 & 2 & 3 & 1 & 0 & 0 \\
                       4 & 5 & 6 & 0 & 1 & 0 \\
                       7 & 8 & 9 & 0 & 0 & 1 \\
                   \end{bNiceArray}   &
               & =   \begin{bNiceArray}{rrr|rrr}[margin]
                         1 & 2  & 3   & 1  & 0 & 0 \\
                         0 & -3 & -6  & -4 & 1 & 0 \\
                         0 & -6 & -12 & -7 & 0 & 1 \\
                     \end{bNiceArray} \\
               & \begin{bNiceArray}{rrr|rrr}[margin]
                     1 & 2  & 3  & 1  & 0  & 0 \\
                     0 & -3 & -6 & -4 & 1  & 0 \\
                     0 & 0  & 0  & 1  & -2 & 1 \\
                 \end{bNiceArray}     &
               & = \text{singular, non-invertible}
          \end{align}

    \item Finding inverse using Gauss-Jordan elimination
          \begin{align}
              \Big[\vec{A} | \vec{I} \Big]
               & = \begin{bNiceArray}{rrr|rrr}[margin]
                       \CodeBefore
                       \columncolor{y_h!10}{1-3}
                       \Body
                       0 & 8 & 0 & 1 & 0 & 0 \\
                       0 & 0 & 4 & 0 & 1 & 0 \\
                       2 & 0 & 0 & 0 & 0 & 1 \\
                   \end{bNiceArray}        &
               & = \begin{bNiceArray}{rrr|rrr}[margin]
                       \CodeBefore
                       \columncolor{y_h!10}{1-3}
                       \Body
                       2 & 0 & 0 & 0 & 0 & 1 \\
                       0 & 8 & 0 & 1 & 0 & 0 \\
                       0 & 0 & 4 & 0 & 1 & 0 \\
                   \end{bNiceArray}          \\
               & =   \begin{bNiceArray}{rrr|rrr}[margin]
                         \CodeBefore
                         \columncolor{y_p!10}{4-6}
                         \Body
                         1 & 0 & 0 & 0           & 0 & \frac{1}{2} \\
                         0 & 1 & 0 & \frac{1}{8} & 0 & 0           \\
                         0 & 0 & 1 & 0           & 0 & \frac{1}{4} \\
                     \end{bNiceArray}
          \end{align}
          Verified using a CAS

    \item Finding inverse using Gauss-Jordan elimination
          \begin{align}
              \Big[\vec{A} | \vec{I} \Big]
               & = \frac{1}{3} \cdot \begin{bNiceArray}{rrr|rrr}[margin]
                                         \CodeBefore
                                         \columncolor{y_h!10}{1-3}
                                         \Body
                                         2  & 1 & 2  & 3 & 0 & 0 \\
                                         -2 & 2 & 1  & 0 & 3 & 0 \\
                                         1  & 2 & -2 & 0 & 0 & 3 \\
                                     \end{bNiceArray}   &
               & = \frac{1}{3} \cdot \begin{bNiceArray}{rrr|rrr}[margin]
                                         2 & 1   & 2  & 3    & 0 & 0 \\
                                         0 & 3   & 3  & 3    & 3 & 0 \\
                                         0 & 1.5 & -3 & -1.5 & 0 & 3 \\
                                     \end{bNiceArray}          \\
               & = \frac{1}{3} \cdot \begin{bNiceArray}{rrr|rrr}[margin]
                                         2 & 1 & 2    & 3  & 0    & 0 \\
                                         0 & 3 & 3    & 3  & 3    & 0 \\
                                         0 & 0 & -4.5 & -3 & -1.5 & 3 \\
                                     \end{bNiceArray}   &
               & = \frac{1}{3} \cdot \begin{bNiceArray}{rrr|rrr}[margin]
                                         2 & 0 & 1    & 2  & -1   & 0 \\
                                         0 & 3 & 3    & 3  & 3    & 0 \\
                                         0 & 0 & -4.5 & -3 & -1.5 & 3 \\
                                     \end{bNiceArray}          \\
               & = \frac{1}{3} \cdot \begin{bNiceArray}{rrr|rrr}[margin]
                                         2           & 0             & 0           &
                                         \frac{4}{3} & - \frac{4}{3} & \frac{2}{3}   \\
                                         0           & 3             & 0           &
                                         1           & 2             & 2             \\
                                         0           & 0             & -4.5        &
                                         -3          & -1.5          & 3             \\
                                     \end{bNiceArray}   &
               & =  \frac{1}{3} \cdot  \begin{bNiceArray}{rrr|rrr}[margin]
                                           \CodeBefore
                                           \columncolor{y_p!10}{4-6}
                                           \Body
                                           3 & 0 & 0 & 2 & -2 & 1  \\
                                           0 & 3 & 0 & 1 & 2  & 2  \\
                                           0 & 0 & 3 & 2 & 1  & -2 \\
                                       \end{bNiceArray}
          \end{align}
          Verified using a CAS

    \item Verifying the relation,
          \begin{align}
              \vec{A}                  & = \begin{bNiceMatrix}[r, margin]
                                               1.8 & -2.32 \\ -0.25 & 0.6
                                           \end{bNiceMatrix}             &
              \vec{A}^2                & = \begin{bNiceMatrix}[r, margin]
                                               3.82 & -5.568 \\ -0.6 & 0.94
                                           \end{bNiceMatrix}             \\
              \big(\vec{A}^2\big)^{-1} & = \color{y_h} \begin{bNiceMatrix}[r, margin]
                                                           3.76 & 22.272 \\ 2.4 & 15.28
                                                       \end{bNiceMatrix} \\
              \vec{A}^{-1}             & = \begin{bNiceMatrix}[r, margin]
                                               1.2 & 4.64 \\ 0.5 & 3.6
                                           \end{bNiceMatrix}             &
              \big(\vec{A}^{-1}\big)^2 & = \color{y_p} \begin{bNiceMatrix}[r, margin]
                                                           3.76 & 22.272 \\ 2.4 & 15.28
                                                       \end{bNiceMatrix}
          \end{align}

    \item Proving the relation,
          \begin{align}
              \big(\vec{A}^2\big)^{-1} & = \big(\vec{AA}\big)^{-1}
              = \vec{A}^{-1} \vec{A}^{-1}
              = \big(\vec{A}^{-1}\big)^2
          \end{align}

    \item Verifying the relation,
          \begin{align}
              \vec{A}                  & = \begin{bNiceMatrix}[r, margin]
                                               1.8 & -2.32 \\ -0.25 & 0.6
                                           \end{bNiceMatrix}             &
              \vec{A}^T                & = \begin{bNiceMatrix}[r, margin]
                                               1.8 & -0.25 \\ -2.32 & 0.6
                                           \end{bNiceMatrix}             \\
              \big(\vec{A}^T\big)^{-1} & = \color{y_h} \begin{bNiceMatrix}[r, margin]
                                                           1.2 & 0.5 \\ 4.64 & 3.6
                                                       \end{bNiceMatrix} \\
              \vec{A}^{-1}             & = \begin{bNiceMatrix}[r, margin]
                                               1.2 & 4.64 \\ 0.5 & 3.6
                                           \end{bNiceMatrix}             &
              \big(\vec{A}^{-1}\big)^T & = \color{y_p} \begin{bNiceMatrix}[r, margin]
                                                           1.2 & 0.5 \\ 4.64 & 3.6
                                                       \end{bNiceMatrix}
          \end{align}

    \item Proving the relation,
          \begin{align}
              \big(\vec{AA}^{-1}\big)^{T} & = \big( \vec{A}^{-1} \big)^T \vec{A}^T
              = \vec{I}                                                            \\
              \big( \vec{A}^{-1} \big)^T \vec{A}^T \big( \vec{A}^T \big)^{-1}
                                          & = \big( \vec{A}^T \big)^{-1}           \\
              \big( \vec{A}^{-1} \big)^T  & = \big( \vec{A}^T \big)^{-1}
          \end{align}

    \item Proving the relation,
          \begin{align}
              \big(\vec{AA}^{-1}\big)^{-1} & = \big(\vec{A}^{-1}\big)^{-1} \vec{A}^{-1}
              = \vec{I}                                                                 \\
              \big(\vec{A}^{-1}\big)^{-1}
              \vec{A}^{-1} \vec{A}         & = \vec{I} \vec{A}                          \\
              \big(\vec{A}^{-1}\big)^{-1}  & = \vec{A}
          \end{align}

    \item The matrix and its inverse are rotation by $ 2\theta $ and $ -2\theta $
          respectively. This shows that multiplying by the inverse of a matrix
          corresponds to performing the inverse linear transoform.

    \item Consider the Gauss jordan elimination process
          \begin{align}
              \Big[\vec{U} \big| \vec{I} \Big] &
              \implies \Big[\vec{I} \big| \vec{H} \Big]
          \end{align}
          The row operations that convert $ \vec{U} $ to $ \vec{I} $, automatically leave
          $ H $ triangular.

    \item Prob 7 shows a matrix that interchanges Row 1 and Row 2. Its inverse is a
          matrix that interchanges Row 2 and Row 1. \par
          This happens to be the same matrix since the linear transform is also its own
          inverse.

    \item Finding the inverse by the direct cofactor formula, with the cofactor matrix
          being $ \vec{B} $,
          \begin{align}
              \vec{A}       & = \begin{bNiceArray}{rrr}[margin]
                                    0.3 & -0.1 & 0.5 \\
                                    2   & 6    & 4   \\
                                    5   & 0    & 9
                                \end{bNiceArray}      &
              \det(\vec{A}) & = 0.3(54) - 2(-0.9) + 5(-3.4) = 1      \\
              \vec{B}       & = \begin{bNiceArray}{rrr}[margin]
                                    54   & 2    & -30  \\
                                    0.9  & 0.2  & -0.5 \\
                                    -3.4 & -0.2 & 2
                                \end{bNiceArray}      &
              \vec{A}^{-1}  & = \frac{1}{\det(\vec{A})}\ \vec{B}^{T} \\
              \vec{A}^{-1}  & = \begin{bNiceArray}{rrr}[margin]
                                    54  & 0.9  & -3.4 \\
                                    2   & 0.2  & -0.2 \\
                                    -30 & -0.5 & 2
                                \end{bNiceArray}
          \end{align}
          which matches the Gauss-Jordan procedure from Problem 3

    \item Finding the inverse by the direct cofactor formula, with the cofactor matrix
          being $ \vec{B} $,
          \begin{align}
              \vec{A}       & = \begin{bNiceArray}{rrr}[margin]
                                    -4 & 0 & 0  \\
                                    0  & 8 & 13 \\
                                    0  & 3 & 5  \\
                                \end{bNiceArray}      &
              \det(\vec{A}) & = -4(1) = -4                           \\
              \vec{B}       & = \begin{bNiceArray}{rrr}[margin]
                                    1 & 0   & 0   \\
                                    0 & -20 & 12  \\
                                    0 & 52  & -32
                                \end{bNiceArray}      &
              \vec{A}^{-1}  & = \frac{1}{\det(\vec{A})}\ \vec{B}^{T} \\
              \vec{A}^{-1}  & = \begin{bNiceArray}{rrr}[margin]
                                    -\frac{1}{4} & 0  & 0   \\
                                    0            & 5  & -13 \\
                                    0            & -3 & 8   \\
                                \end{bNiceArray}
          \end{align}
          which matches the Gauss-Jordan procedure from Problem 6

\end{enumerate}