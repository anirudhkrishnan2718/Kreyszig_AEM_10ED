\section{Linear Systems of Equations, Gauss Elimination}
\begin{enumerate}
\item The augmented matrix is,
\begin{align}
    \vec{\tilde{A}} & = \begin{bNiceArray}{rr|r}
                            4 & -6 & -11    \\
                            -3 & 8 & 10    \\
                        \end{bNiceArray}           &
    \vec{R}         & = \begin{bNiceArray}{rr|r}
                            4 & -6 & -11    \\
                            0 & 3.5 & 1.75    \\
                        \end{bNiceArray}              \\
    x_2             & = \frac{1.75}{3.5} = \color{y_p} 0.5 &
    x_1             & = \frac{-11 + 3}{4} = \color{y_h} -2
\end{align}

\item The augmented matrix is,
\begin{align}
    \vec{\tilde{A}} & = \begin{bNiceArray}{rr|r}
                            3 & -0.5 & 0.6        \\
                            1.5 & 4.5 & 6    \\
                        \end{bNiceArray}              &
    \vec{R}         & = \begin{bNiceArray}{rr|r}
                            3 & -0.5 & 0.6    \\
                            0 & 4.75 & 5.7    \\
                        \end{bNiceArray}                 \\
    x_2             & = \frac{5.7}{4.75} = \color{y_p} 1.2    &
    x_1             & = \frac{0.6 + 0.6}{3} = \color{y_h} 0.4
\end{align}

\item The augmented matrix is,
\begin{align}
    \vec{\tilde{A}} & = \begin{bNiceArray}{rrr|r}
                            1 & 1 & -1 & 9        \\
                            0 & 8 & 6 & -6    \\
                            -2 & 4 & -6 & 40    \\
                        \end{bNiceArray}              &
                    & \implies \begin{bNiceArray}{rrr|r}
                                   1 & 1 & -1 & 9        \\
                                   0 & 8 & 6 & -6    \\
                                   0 & 6 & -8 & 58    \\
                               \end{bNiceArray}          \\
    \vec{R}         & = \begin{bNiceArray}{rrr|r}
                            1 & 1 & -1 & 9        \\
                            0 & 8 & 6 & -6    \\
                            0 & 0 & -12.5 & 62.5    \\
                        \end{bNiceArray}                 \\
    x_3             & = \frac{62.5}{-12.5} = \color{brown6} -5 &
    x_2             & = \frac{-6 + 30}{8} = \color{y_p} 3        \\
    x_1             & = \frac{9 - 5  - 3}{1} = \color{y_h} 1
\end{align}

\item The augmented matrix is,
\begin{align}
    \vec{\tilde{A}} & = \begin{bNiceArray}{rrr|r}
                            4 & 1 & 0 & 4        \\
                            5 & -3 & 1 & 2    \\
                            -9 & 2 & -1 & 5    \\
                        \end{bNiceArray}        &
                    & \implies \begin{bNiceArray}{rrr|r}
                                   4 & 1 & 0 & 4        \\
                                   0 & -4.25 & 1 & -3    \\
                                   0 & 4.25 & -1 & 14    \\
                               \end{bNiceArray} \\
    \vec{R}         & = \begin{bNiceArray}{rrr|r}
                            4 & 1 & 0 & 4        \\
                            0 & -4.25 & 1 & -3    \\
                            0 & 0 & 0 & 8    \\
                        \end{bNiceArray}        \\
                    & \text{no solution}
\end{align}

\item The augmented matrix is,
\begin{align}
    \vec{\tilde{A}} & = \begin{bNiceArray}{rr|r}
                            13 & 12 & -6        \\
                            -4 & 7 & -73    \\
                            11 & -13 & 157
                        \end{bNiceArray}                  &
                    & = \begin{bNiceArray}{rr|r}
                            13 & 12 & -6        \\
                            0 & \frac{139}{13} & - \frac{973}{13}    \\
                            0 & - \frac{301}{13} & \frac{2107}{13}
                        \end{bNiceArray}    \\
    \vec{R}         & = \begin{bNiceArray}{rr|r}
                            13 & 12 & -6        \\
                            0 & \frac{139}{13} & - \frac{973}{13}    \\
                            0 & 0 & 0
                        \end{bNiceArray}    \\
    x_2             & = \frac{-973}{139} = \color{y_p} -7         &
    x_1             & = \frac{-6 + 84}{13} = \color{y_h} 6
\end{align}

\item The augmented matrix is,
\begin{align}
    \vec{\tilde{A}} & = \begin{bNiceArray}{rrr|r}
                            4 & -8 & 3 & 16        \\
                            -1 & 2 & -5 & -21    \\
                            3 & -6 & 1 & 7    \\
                        \end{bNiceArray}                       &
                    & \implies \begin{bNiceArray}{rrr|r}
                                   4 & -8 & 3 & 16        \\
                                   0 & 0 & - \frac{17}{4} & -17    \\
                                   0 & 0 & - \frac{5}{4} & -5    \\
                               \end{bNiceArray}            \\
    \vec{R}         & = \begin{bNiceArray}{rrr|r}
                            4 & -8 & 3 & 16        \\
                            0 & 0 & - \frac{17}{4} & -17    \\
                            0 & 0 & 0 & 0    \\
                        \end{bNiceArray}                   \\
    x_3             & = \frac{-17}{-17/4} = \color{brown6} 4            &
    x_2             & = \color{y_p} t_2 \ (\text{free})                   \\
    x_1             & = \frac{16 - 12 + 8t_2}{4} = \color{y_h} 1 + 2t_2
\end{align}

\item The augmented matrix is,
\begin{align}
    \vec{\tilde{A}}    & = \begin{bNiceArray}{rrr|r}
                               2 & 4 & 1 & 0        \\
                               -1 & 1 & -2 & 0    \\
                               4 & 0 & 6 & 0    \\
                           \end{bNiceArray}              &
                       & \implies \begin{bNiceArray}{rrr|r}
                                      2 & 4 & 1 & 0        \\
                                      0 & 3 & -1.5 & 0    \\
                                      0 & -8 & 4 & 0    \\
                                  \end{bNiceArray}        \\
    \vec{R}            & = \begin{bNiceArray}{rrr|r}
                               2 & 4 & 1 & 0        \\
                               0 & 3 & -\frac{3}{2} & 0    \\
                               0 & 0 & 0 & 0    \\
                           \end{bNiceArray}            \\
    x_3                & = \frac{-17}{-17/4} = \color{brown6}
    t_3\ (\text{free}) &
    x_2                & = \color{y_p} \frac{1}{2}\ t_3           \\
    x_1                & = \frac{16 - 12 + 8t_2}{4} = \color{y_h}
    -\frac{3}{2}\ t_3
\end{align}

\item The augmented matrix is,
\begin{align}
    \vec{\tilde{A}} & = \begin{bNiceArray}{rrr|r}
                            0 & 4 & 3 & 8        \\
                            2 & 0 & -1 & 2    \\
                            3 & 2 & 0 & 5    \\
                        \end{bNiceArray}               &
                    & \implies \begin{bNiceArray}{rrr|r}
                                   3 & 2 & 0 & 5    \\
                                   0 & 4 & 3 & 8        \\
                                   2 & 0 & -1 & 2    \\
                               \end{bNiceArray}        \\
                    & = \begin{bNiceArray}{rrr|r}
                            3 & 2 & 0 & 5    \\
                            0 & 4 & 3 & 8        \\
                            0 & - \frac{4}{3} & -1 & - \frac{4}{3} \\
                        \end{bNiceArray} &
    \vec{R}         & = \begin{bNiceArray}{rrr|r}
                            3 & 2 & 0 & 5    \\
                            0 & 4 & 3 & 8        \\
                            0 & 0 & 0 & \frac{4}{3} \\
                        \end{bNiceArray}               \\
                    & \text{no solution}
\end{align}

\item The augmented matrix is,
\begin{align}
    \vec{\tilde{A}}    & = \begin{bNiceArray}{rrr|r}
                               0 & -2 & -2 & -8        \\
                               3 & 4 & -5 & 13    \\
                           \end{bNiceArray}                     &
    \vec{R}            & \implies \begin{bNiceArray}{rrr|r}
                                      3 & 4 & -5 & 13    \\
                                      0 & -2 & -2 & -8 \\
                                  \end{bNiceArray}               \\
    x_3                & = \frac{-1.5}{-4.5} = \color{brown6}
    t_3\ (\text{free}) &
    x_2                & = \color{y_p} 4 - t_3                           \\
    x_1                & = \frac{13 + 5t_3 - 16 + 4t_3}{3} = \color{y_h}
    -1 +  3t_3
\end{align}

\item The augmented matrix is,
\begin{align}
    \vec{\tilde{A}} & = \begin{bNiceArray}{rrr|r}
                            5 & -7 & 3 & 17        \\
                            -15 & 21 & -9 & 50    \\
                        \end{bNiceArray}        &
    \vec{R}         & \implies \begin{bNiceArray}{rrr|r}
                                   5 & -7 & 3 & 17        \\
                                   0 & 0 & 0 & 101    \\
                               \end{bNiceArray} \\
                    & \text{no solution}
\end{align}

\item The augmented matrix is,
\begin{align}
    \vec{\tilde{A}} & = \begin{bNiceArray}{rrrr|r}
                            0 & 5 & 5 & -10 & 0        \\
                            2 & -3 & -3 & 6 & 2   \\
                            4 & 1 & 1 & -2 & 4   \\
                        \end{bNiceArray}        &
                    & \implies \begin{bNiceArray}{rrrr|r}
                                   2 & -3 & -3 & 6 & 2   \\
                                   0 & 5 & 5 & -10 & 0        \\
                                   4 & 1 & 1 & -2 & 4   \\
                               \end{bNiceArray}    \\
                    & \implies \begin{bNiceArray}{rrrr|r}
                                   2 & -3 & -3 & 6 & 2   \\
                                   0 & 5 & 5 & -10 & 0        \\
                                   0 & 7 & 7 & -14 & 0   \\
                               \end{bNiceArray}  &
    \vec{R}         & = \begin{bNiceArray}{rrrr|r}
                            2 & -3 & -3 & 6 & 2   \\
                            0 & 5 & 5 & -10 & 0        \\
                            0 & 0 & 0 & 0 & 0   \\
                        \end{bNiceArray}           \\
    x_4             & = \color{purple5} t_4\ (\text{free}) &
    x_3             & = \color{brown6} t_3\ (\text{free})    \\
    x_2             & = \color{y_p} 2t_4 - t_3             &
    x_1             & = \color{y_h} 1
\end{align}

\item The augmented matrix is,
\begin{align}
    \vec{\tilde{A}} & = \begin{bNiceArray}{rrrr|r}
                            2 & -2 & 4 & 0 & 0        \\
                            -3 & 3 & -6 & 5 & 15   \\
                            1 & -1 & 2 & 0 & 0   \\
                        \end{bNiceArray}         &
                    & \implies \begin{bNiceArray}{rrrr|r}
                                   2 & -2 & 4 & 0 & 0        \\
                                   -3 & 3 & -6 & 5 & 15   \\
                                   1 & -1 & 2 & 0 & 0   \\
                               \end{bNiceArray}     \\
                    & \implies \begin{bNiceArray}{rrrr|r}
                                   2 & -3 & -3 & 6 & 2   \\
                                   0 & 5 & 5 & -10 & 0        \\
                                   0 & 7 & 7 & -14 & 0   \\
                               \end{bNiceArray}  &
    \vec{R}         & = \begin{bNiceArray}{rrrr|r}
                            2 & -3 & -3 & 6 & 2   \\
                            0 & 5 & 5 & -10 & 0        \\
                            0 & 0 & 0 & 0 & 0   \\
                        \end{bNiceArray}           \\
    x_4             & = \color{purple5} t_4\ (\text{free}) &
    x_3             & = \color{brown6} t_3\ (\text{free})    \\
    x_2             & = \color{y_p} 2t_4 - t_3             &
    x_1             & = \color{y_h} 1
\end{align}

\item The augmented matrix is,
\begin{align}
    \vec{\tilde{A}} & = \begin{bNiceArray}{rrrr|r}
                            0 & 10 & 4 & -2 & -4        \\
                            -3 & -17 & 1 & 2 & 2   \\
                            1 & 1 & 1 & 0 & 6   \\
                            8 & -34 & 16 & -10 & 4
                        \end{bNiceArray}           &
                    & \implies \begin{bNiceArray}{rrrr|r}
                                   -3 & -17 & 1 & 2 & 2   \\
                                   1 & 1 & 1 & 0 & 6   \\
                                   8 & -34 & 16 & -10 & 4 \\
                                   0 & 10 & 4 & -2 & -4        \\
                               \end{bNiceArray}           \\
                    & \implies\begin{bNiceArray}{rrrr|r}
                                  -3 & -17 & 1 & 2 & 2   \\
                                  0 & - \frac{14}{3} & \frac{4}{3}
                                  & \frac{2}{3} & \frac{20}{3}   \\
                                  0 & - \frac{238}{3} & \frac{56}{3}
                                  & - \frac{14}{3} & \frac{28}{3} \\
                                  0 & 10 & 4 & -2 & -4        \\
                              \end{bNiceArray}       &
                    & \implies \begin{bNiceArray}{rrrr|r}
                                   -3 & -17 & 1 & 2 & 2   \\
                                   0 & - \frac{14}{3} & \frac{4}{3}
                                   & \frac{2}{3} & \frac{20}{3}   \\
                                   0 & 0 & -4 & -16 & -104 \\
                                   0 & 0 & \frac{48}{7}
                                   & - \frac{4}{7} & \frac{72}{7}        \\
                               \end{bNiceArray} \\
    \vec{R}         & = \begin{bNiceArray}{rrrr|r}
                            -3 & -17 & 1 & 2 & 2   \\
                            0 & - \frac{14}{3} & \frac{4}{3}
                            & \frac{2}{3} & \frac{20}{3}   \\
                            0 & 0 & -4 & -16 & -104 \\
                            0 & 0 & 0 & -28 & -168        \\
                        \end{bNiceArray}
\end{align}
Back substituion gives,
\begin{align}
    x_4 & = \frac{168}{28} = \color{purple5} 6        &
    x_3 & = \frac{-104 + 96}{-4} = \color{brown6} 2     \\
    x_2 & = \frac{20 - 12 - 8}{3} = \color{y_p} 0     &
    x_1 & = \frac{2 - 12 - 2 + 0}{-3} = \color{y_h} 4
\end{align}

\item The augmented matrix is,
\begin{align}
    \vec{\tilde{A}}
            & = \begin{bNiceArray}{rrrr|r}
                    2 & 3 & 1 & -11 & 1        \\
                    5 & -2 & 5 & -4 & 5   \\
                    1 & -1 & 3 & -3 & 3   \\
                    3 & 4 & -7 & 2 & -7
                \end{bNiceArray}          &
            & \implies \begin{bNiceArray}{rrrr|r}
                           2 & 3 & 1 & -11 & 1        \\
                           0 & - \frac{19}{2} & \frac{5}{2}
                           & \frac{47}{2} & \frac{5}{2}   \\
                           0 & - \frac{5}{2} & \frac{5}{2}
                           & \frac{5}{2} & \frac{5}{2}   \\
                           0 & - \frac{1}{2} & - \frac{17}{2}
                           & \frac{37}{2} & - \frac{17}{2}
                       \end{bNiceArray}   \\
            & \implies\begin{bNiceArray}{rrrr|r}
                          2 & 3 & 1 & -11 & 1        \\
                          0 & - \frac{19}{2} & \frac{5}{2}
                          & \frac{47}{2} & \frac{5}{2}   \\
                          0 & 0 & \frac{35}{19}
                          & - \frac{70}{19} & \frac{35}{19}   \\
                          0 & 0 & - \frac{164}{19}
                          & \frac{328}{19} & - \frac{164}{19}
                      \end{bNiceArray} &
            & \implies \begin{bNiceArray}{rrrr|r}
                           2 & 3 & 1 & -11 & 1        \\
                           0 & - \frac{19}{2} & \frac{5}{2}
                           & \frac{47}{2} & \frac{5}{2}   \\
                           0 & 0 & \frac{35}{19}
                           & - \frac{70}{19} & \frac{35}{19}   \\
                           0 & 0 & 0 & 0 & 0
                       \end{bNiceArray} \\
    \vec{R} & = \begin{bNiceArray}{rrrr|r}
                    2 & 3 & 1 & -11 & 1        \\
                    0 & -19 & 5 & 47 & 5   \\
                    0 & 0 & 1 & -2 & 1   \\
                    0 & 0 & 0 & 0 & 0
                \end{bNiceArray}
\end{align}
Back substituion gives,
\begin{align}
    x_4 & = \color{purple5}t_4\ (\text{free})                     &
    x_3 & = \color{brown6} 1 + 2t_4                                 \\
    x_2 & = \frac{5 - 47t_4 - 5 - 10t_4}{-19} = \color{y_p} 3t_4  &
    x_1 & = \frac{1 + 11t_4 - 1 - 2t_4 - 9t_4}{2} = \color{y_h} 0
\end{align}

\item Using the facts that,
\begin{itemize}
    \item Sequence of row operations is a row operation.
    \item Every row operation has an inverse operation.
\end{itemize}
The proof for all three properties of the row equivalence of matrices follows
trivially from the above two statements.

\item TBC. Coded in python. Cross verified with Sympy

\item Using KVL and KCL, the system is
\begin{align}
    j_1 + j_2         & = j_3                      &
    2j_1 + j_3 + 2j_1 & = 16                         \\
    4j_2 + j_3        & = 32                         \\
    \vec{A}           & = \begin{bNiceArray}{rr|r}
                              5 & 1 & 16 \\
                              1 & 5 & 32
                          \end{bNiceArray}
                      &                            &
    \implies \begin{bNiceArray}{rr|r}
                 5 & 1 & 16 \\
                 0 & \frac{24}{5} & \frac{144}{5}
             \end{bNiceArray}
\end{align}
Back-substituting gives,
\begin{align}
    j_2 & = \color{y_p} 6    &
    j_1 & = \color{y_h} 2      \\
    j_3 & = \color{brown6} 8
\end{align}

\item Using KVL and KCL, the system is
\begin{align}
    j_2 + j_3     & = j_1                      &
    4j_1 + 12j_2  & = 36                         \\
    -8j_3 + 12j_2 & = 24                         \\
    \vec{A}       & = \begin{bNiceArray}{rr|r}
                          4 & 12 & 36 \\
                          -8 & 20 & 24
                      \end{bNiceArray}
                  &                            &
    \implies \begin{bNiceArray}{rr|r}
                 4 & 12 & 36 \\
                 0 & 44 & 96
             \end{bNiceArray}
\end{align}
Back-substituting gives,
\begin{align}
    j_2 & = \color{y_p} \frac{24}{11}   &
    j_1 & = \color{y_h} \frac{27}{11}     \\
    j_3 & = \color{brown6} \frac{3}{11}
\end{align}

\item Using KVL and KCL, the system is
\begin{align}
    -R_1 j_2          & = E_0                       &
    R_2 j_3 + R_1 j_2 & = 0                           \\
    j_1 + j_2         & = j_3                         \\
    \vec{A}           & = \begin{bNiceArray}{rrr|r}
                              1 & 1 & -1 & 0 \\
                              0 & -R_1 & 0 & E_0 \\
                              0 & R_1 & R_2 & 0 \\
                          \end{bNiceArray}
                      & \vec{R}                     &
    \implies \begin{bNiceArray}{rrr|r}
                 1 & 1 & -1 & 0 \\
                 0 & -R_1 & 0 & E_0 \\
                 0 & 0 & R_2 & E_0 \\
             \end{bNiceArray}
\end{align}
Back-substituting gives,
\begin{align}
    j_3 & = \color{brown6} \frac{E_0}{R_2}               &
    j_2 & = \color{y_p} -\frac{E_0}{R_1}                   \\
    j_1 & = \color{y_h} \frac{E_0\ (R_1 + R_2)}{R_1 R_2}
\end{align}

\item Wheatstone bridge, with voltage arbitrarily set to 0 and $ E $ behind and in
front of the voltage source. Now, by KVL,
\begin{align}
    E - R_1 I_1 & = R_2 I_2 &
    E - R_x I_x & = R_3 I_3
\end{align}
For there to be zero current throught the resistor $ R_0 $, both its ends have
to be at the same voltage.
\begin{align}
    E - R_1 I_1 & = E - R_x I_x                                         &
    R_3 I_3     & = R_2 I_2                                               \\
    R_x         & = R_1\ \frac{I_1}{I_x}                                &
    R_x         & = R_1 \frac{I_1\ R_3}{R_2\ I_1} = \frac{R_1 R_3}{R_2}
\end{align}

\item By KVL,
\begin{align}
    x_1 + x_4 & = 1000 & x_1 + x_2 & = 1600 \\
    x_3 + x_4 & = 1600 & x_2 + x_3 & = 2200
\end{align}
Solving by Gauss elimination,
\begin{align}
    \vec{\tilde{A}} & = \begin{bNiceArray}{rrrr|r}
                            1 & 0 & 0 & 1 & 1000 \\
                            1 & 1 & 0 & 0 & 1600 \\
                            0 & 1 & 1 & 0 & 2200 \\
                            0 & 0 & 1 & 1 & 1600 \\
                        \end{bNiceArray} &
                    & = \begin{bNiceArray}{rrrr|r}
                            1 & 0 & 0 & 1 & 1000 \\
                            0 & 1 & 0 & -1 & 600 \\
                            0 & 1 & 1 & 0 & 2200 \\
                            0 & 0 & 1 & 1 & 1600 \\
                        \end{bNiceArray} \\
                    & = \begin{bNiceArray}{rrrr|r}
                            1 & 0 & 0 & 1 & 1000 \\
                            0 & 1 & 0 & -1 & 600 \\
                            0 & 0 & 1 & 1 & 1600 \\
                            0 & 0 & 1 & 1 & 1600 \\
                        \end{bNiceArray} &
    \vec{R}         & = \begin{bNiceArray}{rrrr|r}
                            1 & 0 & 0 & 1 & 1000 \\
                            0 & 1 & 0 & -1 & 600 \\
                            0 & 0 & 1 & 1 & 1600 \\
                            0 & 0 & 0 & 0 & 0 \\
                        \end{bNiceArray}
\end{align}
Back substitution gives,
\begin{align}
    x_4 & = \color{purple5} t_4\ (\text{free}) &
    x_3 & = \color{brown6} 1600 - t_4            \\
    x_2 & = \color{y_p} 600 + t_4              &
    x_1 & = \color{y_h} 1000 - t_4
\end{align}
The solution is not unique, as one variable is free.

\item Using Gauss elimination, with supply equaling demand,
\begin{align}
    \vec{\tilde{A}} & = \begin{bNiceArray}{rrrr|r}
                            2 & 1 & 1 & 0 & 40 \\
                            4 & -1 & -1 & 0 & -4 \\
                            5 & -2 & 0 & -1 & -16 \\
                            0 & 3 & 0 & -1 & 4 \\
                        \end{bNiceArray}                     &
                    & = \begin{bNiceArray}{rrrr|r}
                            2 & 1 & 1 & 0 & 40 \\
                            0 & -3 & -3 & 0 & -84 \\
                            0 & - \frac{9}{2} & - \frac{5}{2} & -1 & -116 \\
                            0 & 3 & 0 & -1 & 4 \\
                        \end{bNiceArray} \\
                    & = \begin{bNiceArray}{rrrr|r}
                            2 & 1 & 1 & 0 & 40 \\
                            0 & -3 & -3 & 0 & -84 \\
                            0 & 0 & -4 & 2 & -20 \\
                            0 & 0 & -3 & -1 & -80 \\
                        \end{bNiceArray}                     &
    \vec{R}         & = \begin{bNiceArray}{rrrr|r}
                            2 & 1 & 1 & 0 & 40 \\
                            0 & 1 & 1 & 0 & 28 \\
                            0 & 0 & -2 & 1 & -10 \\
                            0 & 0 & 0 & 1 & 26 \\
                        \end{bNiceArray}
\end{align}
Back substitution gives,
\begin{align}
    S_2 & = \color{purple5} 26 &
    S_1 & = \color{brown6} 18    \\
    P_2 & = \color{y_p} 10     &
    P_1 & = \color{y_h} 6
\end{align}

\item Using Gauss elimination, with the number of atoms of each element unchanged
during the reaction,
\begin{align}
    \vec{\tilde{A}} & = \begin{bNiceArray}{rrrr|r}
                            3 & 0 & -1 & 0 & 0 \\
                            0 & 2 & -2 & -1 & 0 \\
                            8 & 0 & 0 & -2 & 0 \\
                        \end{bNiceArray}    &
    \vec{R}         & = \begin{bNiceArray}{rrrr|r}
                            3 & 0 & -1 & 0 & 0 \\
                            0 & 2 & -2 & -1 & 0 \\
                            0 & 0 & \frac{8}{3} & -2 & 0 \\
                        \end{bNiceArray}
\end{align}
Back substitution gives,
\begin{align}
    x_4 & = \color{purple5} t_4\ (\text{free}) &
    x_3 & = \color{brown6} \frac{3}{4}\ t_4      \\
    x_2 & = \color{y_p} \frac{5}{4}\ t_4       &
    x_1 & = \color{y_h} \frac{1}{4}\ t_4
\end{align}
The set of smallest positive integers is thus, $ \{1, 5, 3, 4\} $

\item Elementary matrices.
\begin{enumerate}
\item Showing the effect of the matrices the $ n- $th column of
$ \vec{A} $,
\begin{align}
\vec{E}_1 \vec{A}_n                             & =
\begin{bNiceMatrix}[r, margin]
a_{1n} \\ a_{3n} \\ a_{2n} \\ a_{4n}
\end{bNiceMatrix}            &
\vec{E}_2 \vec{A}_n                             & =
\begin{bNiceMatrix}[r, margin]
a_{1n} \\ a_{2n} \\ -5a_{1n} + a_{3n} \\ a_{4n}
\end{bNiceMatrix} &
\vec{E}_3 \vec{A}_n                             & =
\begin{bNiceMatrix}[r, margin]
a_{1n} \\ a_{2n} \\ a_{3n} \\ 8a_{4n}
\end{bNiceMatrix}
\end{align}
Applying the three elementary operations to the general
$ 4 \times 2 $ matrix $ \vec{A} $,
\begin{align}
\vec{B} & = \vec{E}_3\vec{E}_2\vec{E}_1(\vec{A}) &
= \vec{E}_3\vec{E}_2\vec{E}_1 \cdot
\begin{bNiceMatrix}[r, margin]
a_{11} & a_{12} \\ a_{21} & a_{22} \\
a_{31} & a_{32} \\ a_{41} & a_{42}
\end{bNiceMatrix}                 \\
\vec{B} & =  \begin{bNiceMatrix}[r, margin]
a_{11}           & a_{12}           \\
a_{31}           & a_{32}           \\
a_{21} - 5a_{11} & a_{22} - 5a_{12} \\
8a_{41}          & 8a_{42}
\end{bNiceMatrix}
\end{align}
Checking if the row operations applied in the reverse order produce
the same result,
\begin{align}
\vec{C} & = \vec{E}_1\vec{E}_2\vec{E}_3(\vec{A}) &
& = \vec{E}_1\vec{E}_2\vec{E}_3 \cdot
\begin{bNiceMatrix}[r, margin]
a_{11} & a_{12} \\ a_{21} & a_{22} \\
a_{31} & a_{32} \\ a_{41} & a_{42}
\end{bNiceMatrix}                 \\
\vec{C} & =  \begin{bNiceMatrix}[r, margin]
a_{11}           & a_{12}           \\
a_{31} - 5a_{11} & a_{32} - 5a_{12} \\
a_{21}           & a_{22}           \\
8a_{41}          & 8a_{42}
\end{bNiceMatrix} &
\vec{B} & \neq \vec{C}
\end{align}

Numerical examples of applying elementary operations TBC.

\item Self-evident for the specific case of $ \vec{E}_1 $ etc.
For the general proof, simply use brute force for each kind of
row operation one at a time.
\end{enumerate}


\end{enumerate}