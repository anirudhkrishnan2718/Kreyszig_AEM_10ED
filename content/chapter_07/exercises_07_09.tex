\section{Vector Spaces, Inner Product Spaces, Linear Transformations}
\begin{enumerate}
    \item By observation, three possible bases for $ \mathcal{R}^2 $ are
          \begin{align}
              \begin{bNiceMatrix}[r, margin]
                  \vec{u_1} & \vec{u_2}
              \end{bNiceMatrix} & = \begin{bNiceMatrix}[r, margin]
                                        1 & 0 \\ 0 & 1
                                    \end{bNiceMatrix} &
              \begin{bNiceMatrix}[r, margin]
                  \vec{v_1} & \vec{v_2}
              \end{bNiceMatrix} & = \begin{bNiceMatrix}[r, margin]
                                        1 & 1 \\ -1 & 1
                                    \end{bNiceMatrix} \\
              \begin{bNiceMatrix}[r, margin]
                  \vec{w_1} & \vec{w_2}
              \end{bNiceMatrix} & = \begin{bNiceMatrix}[r, margin]
                                        0 & -1 \\ -1 & 0
                                    \end{bNiceMatrix}
          \end{align}

    \item Consider two possible representations of $ \vec{v} $,
          \begin{align}
              \vec{v} & = \sum_{i=1}^{n} c_i\ \vec{a}_{(i)}   &
                      & = \sum_{i=1}^{n} b_i\ \vec{a}_{(i)}     \\
              \vec{0} & = \sum_{i=1}^{n} (c_i - b_i)\ a_{(i)}
          \end{align}
          Since the set $ \{a_{(i)}\} $ form a basis for the vector space $ V $, the
          only solution to the last equation has to be $\{c_i - b_i\} = 0$. \par
          Thus, the representation of $ \vec{v} $ in a given basis is unique.

    \item Checking if the given set is a vector space, \textcolor{y_h}{yes}
          \begin{align}
              -v_1 + 2v_2 + 3v_3         & = 0               &
              -4v_1 + v_2 + v_3          & = 0                 \\
              \vec{v}                    & \in \mathcal{R}^3   \\
              \vec{a},\ \vec{b}          & \in V             &
              \implies \vec{a} + \vec{b} & \in V               \\
              \vec{a}                    & \in V             &
              \implies k\vec{a}          & \in V
          \end{align}
          Finding the dimension and basis,
          \begin{align}
              v_2                            & = -\frac{11}{7}\ v_3             &
              v_1                            & = -\frac{1}{7}\ v_3                \\
              \dim(V)                        & = 1                              &
              \begin{bNiceMatrix}[r, margin]
                  \vec{u}_1
              \end{bNiceMatrix} & = \begin{bNiceMatrix}[r, margin]
                                        -1 \\ -11 \\ 7
                                    \end{bNiceMatrix}
          \end{align}

    \item Checking if the given set is a vector space, \textcolor{y_h}{yes}
          \begin{align}
              \vec{M}                    & = \begin{bNiceMatrix}[r, margin]
                                                 0    & v_1  & v_2 \\
                                                 -v_1 & 0    & v_3 \\
                                                 -v_2 & -v_3 & 0
                                             \end{bNiceMatrix}    \\
              \vec{M},\ \vec{N}          & \in V                            &
              \implies \vec{M} + \vec{N} & \in V                              \\
              \vec{M}                    & \in V                            &
              \implies k\vec{M}          & \in V
          \end{align}
          Finding the dimension and basis,
          \begin{align}
              \dim(V)   & = 3                              &
              \vec{U}_1 & = \begin{bNiceMatrix}[r, margin]
                                0  & 1 & 0 \\
                                -1 & 0 & 0 \\
                                0  & 0 & 0
                            \end{bNiceMatrix}    \\
              \vec{U}_2 & = \begin{bNiceMatrix}[r, margin]
                                0  & 0 & 1 \\
                                0  & 0 & 0 \\
                                -1 & 0 & 0
                            \end{bNiceMatrix} &
              \vec{U}_3 & = \begin{bNiceMatrix}[r, margin]
                                0 & 0  & 0 \\
                                0 & 0  & 1 \\
                                0 & -1 & 0
                            \end{bNiceMatrix}
          \end{align}

    \item Checking if the given set is a vector space, \textcolor{y_p}{no}
          \begin{align}
              P(x)                 & = v_1 + v_2x + v_3x^2 + v_4x^3 + v_5x^4 &
              \{v_i\}              & \geq 0                                    \\
              P(x),\ Q(x)          & \in V                                   &
              \implies P(x) + Q(x) & \in V                                     \\
              P(x)                 & \in V                                   &
              \implies kP(x)       & \not\in V
          \end{align}
          Finding the dimension and basis,
          \begin{align}
              \dim(V)   & = 3                              &
              \vec{U}_1 & = \begin{bNiceMatrix}[r, margin]
                                0  & 1 & 0 \\
                                -1 & 0 & 0 \\
                                0  & 0 & 0
                            \end{bNiceMatrix}    \\
              \vec{U}_2 & = \begin{bNiceMatrix}[r, margin]
                                0  & 0 & 1 \\
                                0  & 0 & 0 \\
                                -1 & 0 & 0
                            \end{bNiceMatrix} &
              \vec{U}_3 & = \begin{bNiceMatrix}[r, margin]
                                0 & 0  & 0 \\
                                0 & 0  & 1 \\
                                0 & -1 & 0
                            \end{bNiceMatrix}
          \end{align}

    \item Checking if the given set is a vector space, \textcolor{y_h}{yes}
          \begin{align}
              y(x)                 & = v_1\cos(2x) + v_2\sin(2x) &
              \{v_i\}              & \in \mathcal{R}               \\
              y(x),\ z(x)          & \in V                       &
              \implies y(x) + z(x) & \in V                         \\
              y(x)                 & \in V                       &
              \implies ky(x)       & \not\in V
          \end{align}
          Finding the dimension and basis,
          \begin{align}
              \dim(V) & = 2          \\
              u_1(x)  & = \cos(2x) &
              u_2(x)  & = \sin(2x)
          \end{align}

    \item Checking if the given set is a vector space, \textcolor{y_h}{yes}
          \begin{align}
              y(x)                 & = (v_1 x + v_2)\ e^{-x} &
              \{v_i\}              & \in \mathcal{R}           \\
              y(x),\ z(x)          & \in V                   &
              \implies y(x) + z(x) & \in V                     \\
              y(x)                 & \in V                   &
              \implies ky(x)       & \not\in V
          \end{align}
          Finding the dimension and basis,
          \begin{align}
              \dim(V) & = 2         \\
              u_1(x)  & = e^{-x}  &
              u_2(x)  & = xe^{-x}
          \end{align}

    \item Checking if the given set is a vector space, \textcolor{y_h}{yes}
          \begin{align}
              \vec{A}                    & \rightarrow n \times n &
              \det(\vec{A})              & = 0                      \\
              \vec{M},\ \vec{N}          & \in V                  &
              \implies \vec{M} + \vec{N} & \not\in V                \\
          \end{align}
          An example is the addition of two matrices, each of which are the zero matrix
          and some columns of the identity matrix.
          \begin{align}
              \vec{A} + \vec{B}       & = \vec{I}             \\
              \det(\vec{A})           & = \det(\vec{B}) = 0 &
              \det(\vec{A} + \vec{B}) & = 1
          \end{align}

    \item Checking if the given set is a vector space, \textcolor{y_h}{yes}
          \begin{align}
              \vec{A}                    & = \begin{bNiceMatrix}[r, margin]
                                                 v_1 & v_2 \\ v_3 & -v_1
                                             \end{bNiceMatrix} &
              \{v_i\}                    & \in \mathcal{R}                    \\
              \vec{M},\ \vec{N}          & \in V                            &
              \implies \vec{M} + \vec{N} & \in V                              \\
              \vec{M}                    & \in V                            &
              \implies k\vec{M}          & \not\in V
          \end{align}
          Finding the dimension and basis,
          \begin{align}
              \dim(V)   & = 3                              &
              \vec{U}_1 & = \begin{bNiceMatrix}[r, margin]
                                1 & 0 \\ 0 & -1
                            \end{bNiceMatrix}    \\
              \vec{U}_2 & = \begin{bNiceMatrix}[r, margin]
                                0 & 1 \\ 0 & 0
                            \end{bNiceMatrix} &
              \vec{U}_3 & = \begin{bNiceMatrix}[r, margin]
                                0 & 0 \\ 1 & 0
                            \end{bNiceMatrix}
          \end{align}

    \item Checking if the given set is a vector space, \textcolor{y_h}{yes}
          \begin{align}
              \vec{A}                    & = \begin{bNiceMatrix}[r, margin]
                                                 3v_1  & v_2 \\
                                                 0     & v_3 \\
                                                 -5v_1 & v_4
                                             \end{bNiceMatrix} &
              \{v_i\}                    & \in \mathcal{R}                    \\
              \vec{M},\ \vec{N}          & \in V                            &
              \implies \vec{M} + \vec{N} & \in V                              \\
              \vec{M}                    & \in V                            &
              \implies k\vec{M}          & \not\in V
          \end{align}
          Finding the dimension and basis,
          \begin{align}
              \dim(V)   & = 4                             \\
              \vec{U}_1 & = \begin{bNiceMatrix}[r, margin]
                                3  & 0 \\
                                0  & 0 \\
                                -5 & 0
                            \end{bNiceMatrix} &
              \vec{U}_2 & = \begin{bNiceMatrix}[r, margin]
                                0 & 1 \\
                                0 & 0 \\
                                0 & 0
                            \end{bNiceMatrix} \\
              \vec{U}_3 & = \begin{bNiceMatrix}[r, margin]
                                0 & 0 \\
                                0 & 1 \\
                                0 & 0
                            \end{bNiceMatrix} &
              \vec{U}_4 & = \begin{bNiceMatrix}[r, margin]
                                0 & 0 \\
                                0 & 0 \\
                                0 & 1
                            \end{bNiceMatrix}
          \end{align}

    \item Finding the inverse of the matrix $ \vec{A} $ that represents the transform,
          \begin{align}
              \vec{y}                        & = \begin{bNiceMatrix}[r, margin]
                                                     0.5 & -0.5 \\ 1.5 & -2.5
                                                 \end{bNiceMatrix} \vec{x} &
              \vec{x}                        & = \vec{A}^{-1}\vec{y}           \\
              \vec{A}^{-1}                   & = \frac{1}{\det(\vec{A})} \cdot
              \begin{bNiceMatrix}[r, margin]
                  -2.5 & 0.5 \\ -1.5 & 0.5
              \end{bNiceMatrix} &
                                             & = \color{y_p}
              \begin{bNiceMatrix}[r, margin]
                  5 & -1 \\ 3 & -1
              \end{bNiceMatrix}
          \end{align}

    \item Finding the inverse of the matrix $ \vec{A} $ that represents the transform,
          \begin{align}
              \vec{y}                        & = \begin{bNiceMatrix}[r, margin]
                                                     3 & 2 \\ 4 & 1
                                                 \end{bNiceMatrix} \vec{x} &
              \vec{x}                        & = \vec{A}^{-1}\vec{y}           \\
              \vec{A}^{-1}                   & = \frac{1}{\det(\vec{A})} \cdot
              \begin{bNiceMatrix}[r, margin]
                  1 & -2 \\ -4 & 3
              \end{bNiceMatrix} &
                                             & = \color{y_p}
              \begin{bNiceMatrix}[r, margin]
                  -0.2 & 0.4 \\ 0.8 & -0.6
              \end{bNiceMatrix}
          \end{align}

    \item Finding the inverse of the matrix $ \vec{A} $ that represents the transform,
          \begin{align}
              \vec{y}                        & = \begin{bNiceMatrix}[r, margin]
                                                     5 & 3  & -3 \\
                                                     3 & 2  & -2 \\
                                                     2 & -1 & 2
                                                 \end{bNiceMatrix} \vec{x} &
              \vec{x}                        & = \vec{A}^{-1}\vec{y}           \\
              \vec{A}^{-1}                   & = \frac{1}{\det(\vec{A})} \cdot
              \begin{bNiceMatrix}[r, margin]
                  2  & -10 & -7 \\
                  -3 & 16  & 11 \\
                  0  & 1   & 1
              \end{bNiceMatrix}^T &
                                             & = \color{y_p}
              \begin{bNiceMatrix}[r, margin]
                  2   & -3 & 0 \\
                  -10 & 16 & 1 \\
                  -7  & 11 & 1
              \end{bNiceMatrix}
          \end{align}

    \item Finding the inverse of the matrix $ \vec{A} $ that represents the transform,
          \begin{align}
              \vec{y}                        & = \begin{bNiceMatrix}[r, margin]
                                                     0.2 & -0.1 & 0   \\
                                                     0   & -0.2 & 0.1 \\
                                                     0.1 & 0    & 0.1
                                                 \end{bNiceMatrix} \vec{x} &
              \vec{x}                        & = \vec{A}^{-1}\vec{y}           \\
              \vec{A}^{-1}                   & = \frac{1}{\det(\vec{A})} \cdot
              \begin{bNiceMatrix}[r, margin]
                  -0.02 & 0.01  & 0.02  \\
                  0.01  & 0.02  & -0.01 \\
                  -0.01 & -0.02 & -0.04
              \end{bNiceMatrix}^T &
                                             &
              = -2 \cdot \begin{bNiceMatrix}[r, margin]
                             -2 & 1  & 2  \\
                             1  & 2  & -1 \\
                             -1 & -2 & -4
                         \end{bNiceMatrix}^T                         \\
                                             &
              = \color{y_p}
              \begin{bNiceMatrix}[r, margin]
                  4  & -2 & 2 \\
                  -2 & -4 & 4 \\
                  -4 & 2  & 8
              \end{bNiceMatrix}
          \end{align}

    \item The Euclidean norm is,
          \begin{align}
              \vec{u}               & = \begin{bNiceMatrix}[r, margin]
                                            3 \\ 1 \\ -4
                                        \end{bNiceMatrix}        &
              \lVert \vec{u} \rVert & = \sqrt{3^2 + 1^2 + (-4)^2} = \sqrt{26}
          \end{align}

    \item The Euclidean norm is,
          \begin{align}
              \vec{u}               & = \begin{bNiceMatrix}[r, margin]
                                            \frac{1}{2}  \\ \frac{1}{3} \\
                                            -\frac{1}{2} \\ -\frac{1}{3}
                                        \end{bNiceMatrix}  &
              \lVert \vec{u} \rVert & = \sqrt{\frac{1}{4} + \frac{1}{9}
                  + \frac{1}{4} + \frac{1}{9}} = \frac{\sqrt{26}}{6}
          \end{align}

    \item The Euclidean norm is,
          \begin{align}
              \vec{u}               & = \begin{bNiceMatrix}[r, margin]
                                            1 & 0 & 0 & 1 & -1 & 0 & -1 & 1
                                        \end{bNiceMatrix}^T \\
              \lVert \vec{u} \rVert & = \sqrt{5}
          \end{align}

    \item The Euclidean norm is,
          \begin{align}
              \vec{u}               & = \begin{bNiceMatrix}[r, margin]
                                            -4 \\ 8 \\ -1
                                        \end{bNiceMatrix}   &
              \lVert \vec{u} \rVert & = \sqrt{(-4)^2 + 8^2 + (-1)^2} = 9
          \end{align}

    \item The Euclidean norm is,
          \begin{align}
              \vec{u}               & = \begin{bNiceMatrix}[r, margin]
                                            \frac{2}{3} \\ \frac{2}{3} \\
                                            \frac{1}{3} \\ 0
                                        \end{bNiceMatrix} &
              \lVert \vec{u} \rVert & = \frac{\sqrt{4 + 4 + 1}}{3} = 1
          \end{align}

    \item The Euclidean norm is,
          \begin{align}
              \vec{u}               & = \begin{bNiceMatrix}[r, margin]
                                            \frac{1}{2}  \\ -\frac{1}{2} \\
                                            -\frac{1}{2} \\ \frac{1}{2}
                                        \end{bNiceMatrix} &
              \lVert \vec{u} \rVert & = \frac{\sqrt{4}}{2} = 1
          \end{align}

    \item To ensure orthogonality,
          \begin{align}
              \vec{u} \dotp \vec{v}    & = 0   &
              10 + \frac{k}{2} + 0 + 0 & = 0     \\
              k                        & = -20
          \end{align}

    \item To ensure orthogonality,
          \begin{align}
              \vec{u} \dotp \vec{v}      & = 0   &
              2v_1 + v_3                 & = 0     \\
              \vec{M},\ \vec{N}          & \in V &
              \implies \vec{M} + \vec{N} & \in V   \\
              \vec{M}                    & \in V &
              \implies k\vec{M}          & \in V
          \end{align}
          So, $ V $ does form a vector space. In 3d Euclidean space, this is a plane
          orthogonal to the given normal vector.

    \item Verifying Triangle inequality,
          \begin{align}
              \vec{a}                         & = \begin{bNiceMatrix}[r, margin]
                                                      3 \\ 1 \\ -4
                                                  \end{bNiceMatrix} &
              \lVert \vec{a} \rVert           & = \sqrt{26}                      \\
              \vec{b}                         & = \begin{bNiceMatrix}[r, margin]
                                                      -4 \\ 8 \\ -1
                                                  \end{bNiceMatrix} &
              \lVert \vec{b} \rVert           & = 9                              \\
              \vec{a} + \vec{b}               & = \begin{bNiceMatrix}[r, margin]
                                                      -1 \\ 9 \\ -5
                                                  \end{bNiceMatrix} &
              \lVert \vec{a} + \vec{b} \rVert & = \sqrt{(-1)^2 + 9^2 + (-5)^2} =
              \sqrt{107}                                                         \\
              \lVert \vec{a} + \vec{b} \rVert & \leq \lVert \vec{a} \rVert
              + \lVert \vec{b} \rVert
          \end{align}

    \item Verifying Cauchy-Schwarz inequality,
          \begin{align}
              \vec{a}                            & = \begin{bNiceMatrix}[r, margin]
                                                         \frac{1}{2}  \\ \frac{1}{3} \\
                                                         -\frac{1}{2} \\ -\frac{1}{3}
                                                     \end{bNiceMatrix}  &
              \lVert \vec{a} \rVert              & = \sqrt{\frac{1}{4} + \frac{1}{9}
              + \frac{1}{4} + \frac{1}{9}} = \frac{\sqrt{26}}{6}                     \\
              \vec{b}                            & = \begin{bNiceMatrix}[r, margin]
                                                         \frac{2}{3} \\ \frac{2}{3} \\
                                                         \frac{1}{3} \\ 0
                                                     \end{bNiceMatrix}  &
              \lVert \vec{b} \rVert              & = \frac{\sqrt{4 + 4 + 1}}{3} = 1  \\
              \vec{a} \dotp \vec{b}              & = \frac{1}{3} +
              \frac{2}{9} -\frac{1}{6} = \frac{7}{18}                                \\
              \abs{ \vec{a} \dotp \vec{b} \rvert & \leq \lVert \vec{a} \rVert
              \cdot \lVert \vec{b} \rVert
          \end{align}

    \item Verifying the Parallellogram equality,
          \begin{align}
              \vec{a} + \vec{b}                   & = \begin{bNiceMatrix}[r, margin]
                                                          8 \\ 5 \\ 1
                                                      \end{bNiceMatrix} &
              \lVert \vec{a} + \vec{b} \rVert^2   & = 90                               \\
              \vec{a} - \vec{b}                   & = \begin{bNiceMatrix}[r, margin]
                                                          2 \\ 1 \\ 3
                                                      \end{bNiceMatrix} & \
              \lVert \vec{a} - \vec{b} \rVert^2   & = 14                               \\
              \lVert \vec{a} \rVert^2             & = 38                             &
              \lVert \vec{b} \rVert^2             & = 14                               \\
              \lVert \vec{a} + \vec{b} \rVert^2
              + \lVert \vec{a} - \vec{b} \rVert^2 &
              = 2\Big(\lVert \vec{a} \rVert^2 + \lVert \vec{b} \rVert^2 \Big)
          \end{align}
\end{enumerate}