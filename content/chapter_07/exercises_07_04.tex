\section{Linear Independence, Rank of a Matrix, Vector Space}
\begin{enumerate}
    \item Finding the rank,
          \begin{align}
              \vec{A}   & = \begin{bNiceMatrix}[r, margin]
                                4 & -2 & 6 \\ -2 & 1 & -3
                            \end{bNiceMatrix} &
              \vec{A}^T & = \begin{bNiceMatrix}[r, margin]
                                4 & -2 \\ -2 & 1 \\ 6 & -3
                            \end{bNiceMatrix} \\
              \vec{R}   & = \begin{bNiceMatrix}[r, margin]
                                4 & -2 & 6 \\ 0 & 0 & 0
                            \end{bNiceMatrix} &
              \vec{Q}   & = \begin{bNiceMatrix}[r, margin]
                                4 & -2 \\ 0 & 0 \\ 0 & 0
                            \end{bNiceMatrix}
          \end{align}
          Using the L.I. rows of $ \vec{R} $ and L.I. columns of $ \vec{Q}^T $,
          rank is 1
          \begin{align}
              \begin{bNiceMatrix}[r, margin]
                  \vec{v}_1 \\ \vec{v}_2 \\ \vec{v}_3
              \end{bNiceMatrix} & = \begin{bNiceMatrix}[r, margin]
                                        2 & -1 & 3
                                    \end{bNiceMatrix} &
              \begin{bNiceMatrix}[r, margin]
                  \vec{u}_1 & \vec{u}_2
              \end{bNiceMatrix}      & = \begin{bNiceMatrix}[r, margin]
                                             2 \\ -1
                                         \end{bNiceMatrix}
          \end{align}

    \item Finding the rank,
          \begin{align}
              \vec{A}   & = \begin{bNiceMatrix}[r, margin]
                                a & b \\ b & a
                            \end{bNiceMatrix} &
              \vec{A}^T & = \begin{bNiceMatrix}[r, margin]
                                a & b \\ b & a
                            \end{bNiceMatrix}   \\
              \vec{R}   & = \begin{bNiceMatrix}[r, margin]
                                a & b \\ 0 & \frac{a^2 - b^2}{a}
                            \end{bNiceMatrix} &
              \vec{Q}   & = \vec{R}
          \end{align}
          Using the L.I. rows of $ \vec{R} $ and L.I. columns of $ \vec{Q}^T $,
          \begin{align}
              \begin{bNiceMatrix}[r, margin]
                  \vec{v}_1 \\ \vec{v}_2
              \end{bNiceMatrix} & = \begin{bNiceMatrix}[r, margin]
                                        a & b \\ 0 & a^2 - b^2
                                    \end{bNiceMatrix} &
              \begin{bNiceMatrix}[r, margin]
                  \vec{u}_1 & \vec{u}_2
              \end{bNiceMatrix} & = \begin{bNiceMatrix}[r, margin]
                                        a & 0 \\ b & a^2 - b^2
                                    \end{bNiceMatrix}
          \end{align}
          \begin{itemize}
              \item If $ a^2 = b^2 $ and $ a \neq 0 $, then rank is 1
              \item Else if $ a^2 \neq b^2 $ and $ a \neq 0 $, rank is 2
              \item Else if $ a = b = 0 $, rank is 0
              \item Else if $ a = 0 $ and $ b \neq 0 $, rank is 2
          \end{itemize}

    \item Finding the rank,
          \begin{align}
              \vec{A}   & = \begin{bNiceMatrix}[r, margin]
                                0 & 3 & 5 \\ 3 & 5 & 0 \\ 5 & 0 & 10
                            \end{bNiceMatrix}             &
              \vec{A}^T & = \begin{bNiceMatrix}[r, margin]
                                0 & 3 & 5 \\ 3 & 5 & 0 \\ 5 & 0 & 10
                            \end{bNiceMatrix}             \\
              \vec{R}   & = \begin{bNiceMatrix}[r, margin]
                                3 & 5 & 0 \\ 0 & 3 & 5 \\ 0 & -\frac{25}{3} & 10
                            \end{bNiceMatrix} &
              \vec{Q}   & = \begin{bNiceMatrix}[r, margin]
                                3 & 5 & 0 \\ 0 & 3 & 5 \\ 0 & -\frac{25}{3} & 10
                            \end{bNiceMatrix} \\
              \vec{R}   & = \begin{bNiceMatrix}[r, margin]
                                3 & 5 & 0 \\ 0 & 3 & 5 \\ 0 & 0 & \frac{215}{9}
                            \end{bNiceMatrix}  &
              \vec{Q}   & = \begin{bNiceMatrix}[r, margin]
                                3 & 5 & 0 \\ 0 & 3 & 5 \\ 0 & 0 & \frac{215}{9}
                            \end{bNiceMatrix}
          \end{align}
          Using the L.I. rows of $ \vec{R} $ and L.I. columns of $ \vec{Q}^T $,
          rank is 3.
          \begin{align}
              \begin{bNiceMatrix}[r, margin]
                  \vec{v}_1 \\ \vec{v}_2 \\ \vec{v}_3
              \end{bNiceMatrix} & = \begin{bNiceMatrix}[r, margin]
                                        3 & 5 & 0 \\ 0 & 3 & 5 \\ 0 & 0 & 1
                                    \end{bNiceMatrix} &
              \begin{bNiceMatrix}[r, margin]
                  \vec{u}_1 & \vec{u}_2 & \vec{u}_3
              \end{bNiceMatrix}  & = \begin{bNiceMatrix}[r, margin]
                                         3 & 0 & 0 \\ 5 & 3 & 0 \\ 0 & 5 & 1
                                     \end{bNiceMatrix}
          \end{align}

    \item Finding the rank,
          \begin{align}
              \vec{A}   & = \begin{bNiceMatrix}[r, margin]
                                6 & -4 & 0 \\ -4 & 0 & 2 \\ 0 & 2 & 6
                            \end{bNiceMatrix}           &
              \vec{A}^T & = \begin{bNiceMatrix}[r, margin]
                                6 & -4 & 0 \\ -4 & 0 & 2 \\ 0 & 2 & 6
                            \end{bNiceMatrix}           \\
              \vec{R}   & = \begin{bNiceMatrix}[r, margin]
                                6 & -4 & 0 \\ 0 & 2 & 6 \\ 0 & -\frac{8}{3} & 2
                            \end{bNiceMatrix} &
              \vec{Q}   & = \begin{bNiceMatrix}[r, margin]
                                6 & -4 & 0 \\ 0 & 2 & 6 \\ 0 & -\frac{8}{3} & 2
                            \end{bNiceMatrix} \\
              \vec{R}   & = \begin{bNiceMatrix}[r, margin]
                                6 & -4 & 0 \\ 0 & 2 & 6 \\ 0 & 0 & 10
                            \end{bNiceMatrix}           &
              \vec{Q}   & = \begin{bNiceMatrix}[r, margin]
                                6 & -4 & 0 \\ 0 & 2 & 6 \\ 0 & 0 & 10
                            \end{bNiceMatrix}
          \end{align}
          Using the L.I. rows of $ \vec{R} $ and L.I. columns of $ \vec{Q}^T $,
          rank is 3.
          \begin{align}
              \begin{bNiceMatrix}[r, margin]
                  \vec{v}_1 \\ \vec{v}_2 \\ \vec{v}_3
              \end{bNiceMatrix} & = \begin{bNiceMatrix}[r, margin]
                                        3 & -2 & 0 \\ 0 & 1 & 3 \\ 0 & 0 & 1
                                    \end{bNiceMatrix} &
              \begin{bNiceMatrix}[r, margin]
                  \vec{u}_1 & \vec{u}_2 & \vec{u}_3
              \end{bNiceMatrix}  & = \begin{bNiceMatrix}[r, margin]
                                         3 & 0 & 0 \\ -2 & 1 & 0 \\ 0 & 3 & 1
                                     \end{bNiceMatrix}
          \end{align}

    \item Finding the rank,
          \begin{align}
              \vec{A}   & = \begin{bNiceMatrix}[r, margin]
                                0.2 & -0.1 & 0.4  \\
                                0   & 1.1  & -0.3 \\
                                0.1 & 0    & -2.1
                            \end{bNiceMatrix}  &
              \vec{A}^T & = \begin{bNiceMatrix}[r, margin]
                                0.2  & 0    & 0.1  \\
                                -0.1 & 1.1  & 0    \\
                                0.4  & -0.3 & -2.1
                            \end{bNiceMatrix}  \\
              \vec{R}   & = \begin{bNiceMatrix}[r, margin]
                                0.2 & -0.1 & 0.4  \\
                                0   & 1.1  & -0.3 \\
                                0   & 0.05 & -2.3
                            \end{bNiceMatrix}  &
              \vec{Q}   & = \begin{bNiceMatrix}[r, margin]
                                0.2 & 0    & 0.1  \\
                                0   & 1.1  & 0.05 \\
                                0   & -0.3 & -2.3
                            \end{bNiceMatrix}  \\
              \vec{R}   & = \begin{bNiceMatrix}[r, margin]
                                0.2 & -0.1 & 0.4               \\
                                0   & 1.1  & -0.3              \\
                                0   & 0    & - \frac{503}{220}
                            \end{bNiceMatrix} &
              \vec{Q}   & = \begin{bNiceMatrix}[r, margin]
                                0.2 & 0   & 0.1               \\
                                0   & 1.1 & 0.05              \\
                                0   & 0   & - \frac{503}{220}
                            \end{bNiceMatrix}
          \end{align}
          Using the L.I. rows of $ \vec{R} $ and L.I. columns of $ \vec{Q}^T $,
          rank is 3.
          \begin{align}
              \begin{bNiceMatrix}[r, margin]
                  \vec{v}_1 \\ \vec{v}_2 \\ \vec{v}_3
              \end{bNiceMatrix} & = \begin{bNiceMatrix}[r, margin]
                                        2 & -1 & 4  \\
                                        0 & 11 & -3 \\
                                        0 & 0  & 1
                                    \end{bNiceMatrix} &
              \begin{bNiceMatrix}[r, margin]
                  \vec{u}_1 & \vec{u}_2 & \vec{u}_3
              \end{bNiceMatrix}  & = \begin{bNiceMatrix}[r, margin]
                                         2 & 0  & 0 \\
                                         0 & 22 & 0 \\
                                         1 & 1  & 1
                                     \end{bNiceMatrix}
          \end{align}

    \item Finding the rank,
          \begin{align}
              \vec{A}   & = \begin{bNiceMatrix}[r, margin]
                                0  & 1 & 0  \\
                                -1 & 0 & -4 \\
                                0  & 4 & 0
                            \end{bNiceMatrix}  &
              \vec{A}^T & = \begin{bNiceMatrix}[r, margin]
                                0 & -1 & 0 \\
                                1 & 0  & 4 \\
                                0 & -4 & 0
                            \end{bNiceMatrix}  \\
              \vec{R}   & =  \begin{bNiceMatrix}[r, margin]
                                 -1 & 0 & -4 \\
                                 0  & 1 & 0  \\
                                 0  & 4 & 0
                             \end{bNiceMatrix} &
              \vec{Q}   & = \begin{bNiceMatrix}[r, margin]
                                1 & 0  & 4 \\
                                0 & -1 & 0 \\
                                0 & -4 & 0
                            \end{bNiceMatrix}  \\
              \vec{R}   & =  \begin{bNiceMatrix}[r, margin]
                                 -1 & 0 & -4 \\
                                 0  & 1 & 0  \\
                                 0  & 0 & 0
                             \end{bNiceMatrix} &
              \vec{Q}   & = \begin{bNiceMatrix}[r, margin]
                                1 & 0  & 4 \\
                                0 & -1 & 0 \\
                                0 & 0  & 0
                            \end{bNiceMatrix}
          \end{align}
          Using the L.I. rows of $ \vec{R} $ and L.I. columns of $ \vec{Q}^T $,
          rank is 2.
          \begin{align}
              \begin{bNiceMatrix}[r, margin]
                  \vec{v}_1 \\ \vec{v}_2
              \end{bNiceMatrix} & = \begin{bNiceMatrix}[r, margin]
                                        1 & 0 & 4 \\
                                        0 & 1 & 0
                                    \end{bNiceMatrix} &
              \begin{bNiceMatrix}[r, margin]
                  \vec{u}_1 & \vec{u}_2
              \end{bNiceMatrix} & = \begin{bNiceMatrix}[r, margin]
                                        1 & 0 \\ 0 & 1 \\ 4 & 0
                                    \end{bNiceMatrix}
          \end{align}

    \item Finding the rank,
          \begin{align}
              \vec{A}   & = \begin{bNiceMatrix}[r, margin]
                                8 & 0 & 4 & 0 \\
                                0 & 2 & 0 & 4 \\
                                4 & 0 & 2 & 0 \\
                            \end{bNiceMatrix}  &
              \vec{A}^T & = \begin{bNiceMatrix}[r, margin]
                                8 & 0 & 4 \\
                                0 & 2 & 0 \\
                                4 & 0 & 2 \\
                                0 & 4 & 0
                            \end{bNiceMatrix}  \\
              \vec{R}   & =  \begin{bNiceMatrix}[r, margin]
                                 8 & 0 & 4 & 0 \\
                                 0 & 2 & 0 & 4 \\
                                 0 & 0 & 0 & 0 \\
                             \end{bNiceMatrix} &
              \vec{Q}   & = \begin{bNiceMatrix}[r, margin]
                                8 & 0 & 4 \\
                                0 & 2 & 0 \\
                                0 & 0 & 0 \\
                                0 & 4 & 0
                            \end{bNiceMatrix}  \\
              \vec{R}   & =  \begin{bNiceMatrix}[r, margin]
                                 8 & 0 & 4 & 0 \\
                                 0 & 2 & 0 & 4 \\
                                 0 & 0 & 0 & 0 \\
                             \end{bNiceMatrix} &
              \vec{Q}   & = \begin{bNiceMatrix}[r, margin]
                                8 & 0 & 4 \\
                                0 & 2 & 0 \\
                                0 & 0 & 0 \\
                                0 & 0 & 0
                            \end{bNiceMatrix}
          \end{align}
          Using the L.I. rows of $ \vec{R} $ and L.I. columns of $ \vec{Q}^T $,
          rank is 2.
          \begin{align}
              \begin{bNiceMatrix}[r, margin]
                  \vec{v}_1 \\ \vec{v}_2
              \end{bNiceMatrix} & = \begin{bNiceMatrix}[r, margin]
                                        2 & 0 & 1 & 0 \\
                                        0 & 1 & 0 & 2
                                    \end{bNiceMatrix} &
              \begin{bNiceMatrix}[r, margin]
                  \vec{u}_1 & \vec{u}_2
              \end{bNiceMatrix} & = \begin{bNiceMatrix}[r, margin]
                                        2 & 0 \\
                                        0 & 1 \\
                                        1 & 0
                                    \end{bNiceMatrix}
          \end{align}

    \item Finding the rank,
          \begin{align}
              \vec{A}   & = \begin{bNiceMatrix}[r, margin]
                                2  & 4  & 8  & 16 \\
                                16 & 8  & 4  & 2  \\
                                4  & 8  & 16 & 2  \\
                                2  & 16 & 8  & 4
                            \end{bNiceMatrix}  &
              \vec{A}^T & = \begin{bNiceMatrix}[r, margin]
                                2  & 16 & 4  & 2  \\
                                4  & 8  & 8  & 16 \\
                                8  & 4  & 16 & 8  \\
                                16 & 2  & 2  & 4
                            \end{bNiceMatrix}  \\
              \vec{R}   & =  \begin{bNiceMatrix}[r, margin]
                                 2 & 4   & 8   & 16   \\
                                 0 & -24 & -60 & -126 \\
                                 0 & 0   & 0   & -30  \\
                                 0 & 12  & 0   & -12
                             \end{bNiceMatrix} &
              \vec{Q}   & = \begin{bNiceMatrix}[r, margin]
                                2 & 16   & 4   & 2   \\
                                0 & -24  & 0   & 12  \\
                                0 & -60  & 0   & 0   \\
                                0 & -126 & -30 & -12
                            \end{bNiceMatrix}  \\
              \vec{R}   & =  \begin{bNiceMatrix}[r, margin]
                                 2 & 4   & 8   & 16   \\
                                 0 & -24 & -60 & -126 \\
                                 0 & 0   & -30 & -75  \\
                                 0 & 0   & 0   & -30  \\
                             \end{bNiceMatrix} &
              \vec{Q}   & = \begin{bNiceMatrix}[r, margin]
                                2 & 16  & 4   & 2   \\
                                0 & -24 & 0   & 12  \\
                                0 & 0   & -30 & -75 \\
                                0 & 0   & 0   & -30 \\
                            \end{bNiceMatrix}
          \end{align}
          Using the L.I. rows of $ \vec{R} $ and L.I. columns of $ \vec{Q}^T $,
          rank is 2.
          \begin{align}
              \begin{bNiceMatrix}[r, margin]
                  \vec{v}_1 \\ \vec{v}_2 \\
                  \vec{v}_3 \\ \vec{v}_4
              \end{bNiceMatrix}                 & = \begin{bNiceMatrix}[r, margin]
                                                        1 & 2 & 4  & 8  \\
                                                        0 & 4 & 10 & 21 \\
                                                        0 & 0 & 2  & 5  \\
                                                        0 & 0 & 0  & 1
                                                    \end{bNiceMatrix} &
              \begin{bNiceMatrix}[r, margin]
                  \vec{u}_1 & \vec{u}_2 & \vec{u}_3 & \vec{u}_4
              \end{bNiceMatrix} & = \begin{bNiceMatrix}[r, margin]
                                        1 & 0  & 0 & 0 \\
                                        8 & -2 & 0 & 0 \\
                                        2 & 0  & 2 & 0 \\
                                        1 & 1  & 5 & 1
                                    \end{bNiceMatrix}
          \end{align}

    \item Finding the rank,
          \begin{align}
              \vec{A}   & = \begin{bNiceMatrix}[r, margin]
                                9 & 0 & 1 & 0 \\
                                0 & 0 & 1 & 0 \\
                                1 & 1 & 1 & 1 \\
                                0 & 0 & 1 & 0
                            \end{bNiceMatrix}  &
              \vec{A}^T & = \begin{bNiceMatrix}[r, margin]
                                9 & 0 & 1 & 0 \\
                                0 & 0 & 1 & 0 \\
                                1 & 1 & 1 & 1 \\
                                0 & 0 & 1 & 0
                            \end{bNiceMatrix}  \\
              \vec{R}   & =  \begin{bNiceMatrix}[r, margin]
                                 9 & 0 & 1           & 0 \\
                                 0 & 1 & \frac{8}{9} & 1 \\
                                 0 & 0 & 1           & 0 \\
                                 0 & 0 & 1           & 0
                             \end{bNiceMatrix} &
              \vec{Q}   & =\begin{bNiceMatrix}[r, margin]
                               9 & 0 & 1           & 0 \\
                               0 & 1 & \frac{8}{9} & 1 \\
                               0 & 0 & 1           & 0 \\
                               0 & 0 & 1           & 0
                           \end{bNiceMatrix}   \\
              \vec{R}   & =  \begin{bNiceMatrix}[r, margin]
                                 9 & 0 & 1           & 0 \\
                                 0 & 1 & \frac{8}{9} & 1 \\
                                 0 & 0 & 1           & 0 \\
                                 0 & 0 & 0           & 0
                             \end{bNiceMatrix} &
              \vec{Q}   & =\begin{bNiceMatrix}[r, margin]
                               9 & 0 & 1           & 0 \\
                               0 & 1 & \frac{8}{9} & 1 \\
                               0 & 0 & 1           & 0 \\
                               0 & 0 & 0           & 0
                           \end{bNiceMatrix}
          \end{align}
          Using the L.I. rows of $ \vec{R} $ and L.I. columns of $ \vec{Q}^T $,
          rank is 3.
          \begin{align}
              \begin{bNiceMatrix}[r, margin]
                  \vec{v}_1 \\ \vec{v}_2 \\
                  \vec{v}_3 \\
              \end{bNiceMatrix}     & = \begin{bNiceMatrix}[r, margin]
                                            9 & 0 & 1 & 0 \\
                                            0 & 9 & 8 & 9 \\
                                            0 & 0 & 1 & 0
                                        \end{bNiceMatrix} &
              \begin{bNiceMatrix}[r, margin]
                  \vec{u}_1 & \vec{u}_2 & \vec{u}_3
              \end{bNiceMatrix} & = \begin{bNiceMatrix}[r, margin]
                                        9 & 0 & 0 \\
                                        0 & 9 & 0 \\
                                        1 & 8 & 1 \\
                                        0 & 9 & 0
                                    \end{bNiceMatrix}
          \end{align}

    \item Finding the rank,
          \begin{align}
              \vec{A}   & = \begin{bNiceMatrix}[r, margin]
                                9 & 0 & 1 & 0 \\
                                0 & 0 & 1 & 0 \\
                                1 & 1 & 1 & 1 \\
                                0 & 0 & 1 & 0
                            \end{bNiceMatrix}  &
              \vec{A}^T & = \begin{bNiceMatrix}[r, margin]
                                9 & 0 & 1 & 0 \\
                                0 & 0 & 1 & 0 \\
                                1 & 1 & 1 & 1 \\
                                0 & 0 & 1 & 0
                            \end{bNiceMatrix}  \\
              \vec{R}   & =  \begin{bNiceMatrix}[r, margin]
                                 9 & 0 & 1           & 0 \\
                                 0 & 1 & \frac{8}{9} & 1 \\
                                 0 & 0 & 1           & 0 \\
                                 0 & 0 & 1           & 0
                             \end{bNiceMatrix} &
              \vec{Q}   & =\begin{bNiceMatrix}[r, margin]
                               9 & 0 & 1           & 0 \\
                               0 & 1 & \frac{8}{9} & 1 \\
                               0 & 0 & 1           & 0 \\
                               0 & 0 & 1           & 0
                           \end{bNiceMatrix}   \\
              \vec{R}   & =  \begin{bNiceMatrix}[r, margin]
                                 9 & 0 & 1           & 0 \\
                                 0 & 1 & \frac{8}{9} & 1 \\
                                 0 & 0 & 1           & 0 \\
                                 0 & 0 & 0           & 0
                             \end{bNiceMatrix} &
              \vec{Q}   & =\begin{bNiceMatrix}[r, margin]
                               9 & 0 & 1           & 0 \\
                               0 & 1 & \frac{8}{9} & 1 \\
                               0 & 0 & 1           & 0 \\
                               0 & 0 & 0           & 0
                           \end{bNiceMatrix}
          \end{align}
          Using the L.I. rows of $ \vec{R} $ and L.I. columns of $ \vec{Q}^T $,
          rank is 3.
          \begin{align}
              \begin{bNiceMatrix}[r, margin]
                  \vec{v}_1 \\ \vec{v}_2 \\
                  \vec{v}_3 \\
              \end{bNiceMatrix}     & = \begin{bNiceMatrix}[r, margin]
                                            9 & 0 & 1 & 0 \\
                                            0 & 9 & 8 & 9 \\
                                            0 & 0 & 1 & 0
                                        \end{bNiceMatrix} &
              \begin{bNiceMatrix}[r, margin]
                  \vec{u}_1 & \vec{u}_2 & \vec{u}_3
              \end{bNiceMatrix} & = \begin{bNiceMatrix}[r, margin]
                                        9 & 0 & 0 \\
                                        0 & 9 & 0 \\
                                        1 & 8 & 1 \\
                                        0 & 9 & 0
                                    \end{bNiceMatrix}
          \end{align}

    \item Proving the general case,
          \begin{enumerate}
              \item All intermediate points in between $ A $ and $ B $ can be written
                    as
                    \begin{align}
                        P & = \lambda A + (1 - \lambda) B & \lambda & \in \mathcal{R}
                    \end{align}
                    For $ a_{jk} = j + k - 1 $, for some integer $ n > 1 $, consider the
                    elements of the first column,
                    \begin{align}
                        a_{1k}  & = k                                   &
                        a_{nk}  & = n + k - 1                             \\
                        a_{jk}  & = (1 - \lambda)(k) + (\lambda)(n+k-1) &
                        a_{jk}  & = j+k-1                                 \\
                        \lambda & = \frac{j - 1}{n-1}
                    \end{align}
                    The fact that $ \lambda $ is independent of $ k $, means that every
                    intermediate row has a separate $ \lambda $ that can be used to
                    express it as a linear combination of the first and last rows. \par
                    All but the first and last row can thus be zeroed out and the rank
                    is 2. \par

              \item Replacing $ 1 $ with $ c $ in the above proof only changes
                    the expression for $ \lambda $ to
                    \begin{align}
                        j + k + c & = [1 + k + c](1 - \lambda) + \lambda[n+k+c] &
                        \lambda   & = \frac{j - 1}{n-1}
                    \end{align}
                    Since $ \lambda $ is still independent of $ k $, the result still
                    holds for some general $ c $.

              \item Every row is a scalar multiple of the first row, as can be seen by
                    \begin{align}
                        a_{j,k}    & = 2^{j+k-2} & a_{j+1, k} & = 2^{j+k-1} \\
                        a_{j+1, k} & = 2a_{j,k}
                    \end{align}
                    Thus, rank is 1. \par
                    Other exmaples TBC
          \end{enumerate}

    \item To prove the relation,
          \begin{align}
              \vec{C}        & = (\vec{AB})^T = \vec{B}^T\vec{A}^T &
              \vec{C}^T      & = \vec{AB}                            \\
              \rank(\vec{C}) & = \rank(\vec{C}^T)
          \end{align}

    \item Consider a nilpotent matrix $ \vec{A} $ and an identity matrix of equal
          rank $ \vec{B} $. This counterexample is now,
          \begin{align}
              \rank(\vec{A}^2) & = 0              &
              \rank(\vec{B}^2) & = \rank(\vec{A})
          \end{align}

    \item $\vec{A}$ is not a square matrix. Let it have $ m $ rows and $ n $ columns.
          \begin{align}
              m                & > n    & \rank(\vec{A}) & = \rank(\vec{A}^T) \\
              \rank(\vec{A}^T) & \leq n & \rank(\vec{A}) & < m
          \end{align}
          Since $ \rank(\vec{A}) $ is less than the number of rows, some of the rows are
          L.D. and can be row reduced to zero. \par
          A similar proof for the case $ m < n $ can be used to prove the column vectors
          of such a matrix are L.D.

    \item $\vec{A}$ is a square matrix of order $ n $ with L.I. rows.
          \begin{align}
              \rank(\vec{A})   & = n & \rank(\vec{A}^T) & = \rank(\vec{A}) \\
              \rank(\vec{A}^T) & = n
          \end{align}
          Since $ \vec{A}^T $ has rank $ n $, $ \vec{A} $ has L.I. columns. \par
          The exact same procedure can be used for the backwards proof, using the fact
          that the L.I. rows of $ \vec{A}^T $ is the same as the L.I. columns of
          $ \vec{A} $.

    \item Consider the column representation of $ \vec{AB} $,
          \begin{align}
              \vec{AB} & = \begin{bNiceMatrix}[r, margin]
                               \vec{Ab}_1 & \vec{Ab}_2 & \dots & \vec{Ab}_n
                           \end{bNiceMatrix}
          \end{align}
          Every column vector $ \vec{Ab}_k $ is a vector in the column space of
          $ \vec{A} $. So, the columns of $ \vec{AB} $ belong to the column space of
          $ \vec{A} $. \par
          The dimension of the column space of a vector is equal to its rank.
          \begin{align}
              \color{y_h} \rank(\vec{AB}) \leq \rank(\vec{A})
          \end{align}
          Using the fact that $ \rank(\vec{A}) = \rank(\vec{A}^T) $
          \begin{align}
              \rank\Big(\vec{(AB)}^T\Big) & = \rank(\vec{AB})               &
              \rank(\vec{B}^T \vec{A}^T)  & \leq \rank(\vec{B}^T)             \\
              \color{y_p} \rank(\vec{AB}) & \color{y_p} \leq \rank(\vec{B})
          \end{align}
          This proves the result.

    \item Checking if the set of vectors are L.I, \textcolor{y_p}{no}
          \begin{align}
              \vec{A} & = \begin{bNiceMatrix}[r, margin]
                              \vec{v}_1 \\ \vec{v}_2 \\ \vec{v}_3
                          \end{bNiceMatrix} &
                      & = \begin{bNiceMatrix}[r, margin]
                              3 & 4  & 0   & 2   \\
                              2 & -1 & 3   & 7   \\
                              1 & 16 & -12 & -22
                          \end{bNiceMatrix}      \\
                      & = \begin{bNiceMatrix}[r, margin]
                              1 & 16  & -12 & -22 \\
                              0 & -33 & 27  & 51  \\
                              0 & -44 & 36  & 68
                          \end{bNiceMatrix}    &
                      & =  \begin{bNiceMatrix}[r, margin]
                               1 & 16  & -12 & -22 \\
                               0 & -33 & 27  & 51  \\
                               0 & 0   & 0   & 0
                           \end{bNiceMatrix}
          \end{align}

    \item Checking if the set of vectors are L.I, \textcolor{y_h}{yes}
          \begin{align}
              \vec{A}                                                      & =
              \begin{bNiceMatrix}[r, margin]
                  \vec{v}_1 \\ \vec{v}_2 \\ \vec{v}_3 \\ \vec{v}_4
              \end{bNiceMatrix}             &
                                                                           & =
              \begin{bNiceMatrix}[r, margin]
                  \frac{1}{1} & \frac{1}{2} & \frac{1}{3} & \frac{1}{4} \\
                  \frac{1}{2} & \frac{1}{3} & \frac{1}{4} & \frac{1}{5} \\
                  \frac{1}{3} & \frac{1}{4} & \frac{1}{5} & \frac{1}{6} \\
                  \frac{1}{4} & \frac{1}{5} & \frac{1}{6} & \frac{1}{7} \\
              \end{bNiceMatrix}            \\
                                                                           & =
              \begin{bNiceMatrix}[r, margin]
                  \frac{1}{1} & \frac{1}{2}  & \frac{1}{3}  & \frac{1}{4}   \\
                  0           & \frac{1}{12} & \frac{1}{12} & \frac{3}{40}  \\
                  0           & \frac{1}{12} & \frac{4}{45} & \frac{1}{12}  \\
                  0           & \frac{3}{40} & \frac{1}{12} & \frac{9}{112} \\
              \end{bNiceMatrix} &
                                                                           & =
              \begin{bNiceMatrix}[r, margin]
                  \frac{1}{1} & \frac{1}{2}  & \frac{1}{3}   & \frac{1}{4}   \\
                  0           & \frac{1}{12} & \frac{1}{12}  & \frac{3}{40}  \\
                  0           & 0            & \frac{1}{180} & \frac{1}{120} \\
                  0           & 0            & \frac{1}{120} & \frac{9}{700} \\
              \end{bNiceMatrix}       \\
                                                                           & =
              \begin{bNiceMatrix}[r, margin]
                  \frac{1}{1} & \frac{1}{2}  & \frac{1}{3}   & \frac{1}{4}    \\
                  0           & \frac{1}{12} & \frac{1}{12}  & \frac{3}{40}   \\
                  0           & 0            & \frac{1}{180} & \frac{1}{120}  \\
                  0           & 0            & 0             & \frac{1}{2800} \\
              \end{bNiceMatrix}
          \end{align}

    \item Checking if the set of vectors are L.I, \textcolor{y_h}{yes}
          \begin{align}
              \vec{A} & = \begin{bNiceMatrix}[r, margin]
                              \vec{v}_1 \\ \vec{v}_2 \\ \vec{v}_3
                          \end{bNiceMatrix} &
                      & = \begin{bNiceMatrix}[r, margin]
                              0 & 1 & 1 \\
                              1 & 1 & 1 \\
                              0 & 0 & 1
                          \end{bNiceMatrix}      \\
                      & = \begin{bNiceMatrix}[r, margin]
                              1 & 1 & 1 \\
                              0 & 1 & 1 \\
                              0 & 0 & 1
                          \end{bNiceMatrix}
          \end{align}

    \item Checking if the set of vectors are L.I, \textcolor{y_p}{no}
          \begin{align}
              \vec{A} & = \begin{bNiceMatrix}[r, margin]
                              \vec{v}_1 \\ \vec{v}_2 \\ \vec{v}_3 \\ \vec{v}_4
                          \end{bNiceMatrix} &
                      & = \begin{bNiceMatrix}[r, margin]
                              1 & 2 & 3 & 4 \\
                              2 & 3 & 4 & 5 \\
                              3 & 4 & 5 & 6 \\
                              4 & 5 & 6 & 7
                          \end{bNiceMatrix}                   \\
                      & = \begin{bNiceMatrix}[r, margin]
                              1 & 2  & 3  & 4  \\
                              0 & -1 & -2 & -3 \\
                              0 & -2 & -4 & -6 \\
                              0 & -3 & -6 & -9
                          \end{bNiceMatrix}                 &
                      & =  \begin{bNiceMatrix}[r, margin]
                               1 & 2  & 3  & 4  \\
                               0 & -1 & -2 & -3 \\
                               0 & 0  & 0  & 0  \\
                               0 & 0  & 0  & 0
                           \end{bNiceMatrix}
          \end{align}

    \item Checking if the set of vectors are L.I, \textcolor{y_p}{no}
          \begin{align}
              \vec{A} & = \begin{bNiceMatrix}[r, margin]
                              \vec{v}_1 \\ \vec{v}_2 \\ \vec{v}_3 \\ \vec{v}_4
                          \end{bNiceMatrix} &
                      & = \begin{bNiceMatrix}[r, margin]
                              2 & 0 & 0 & 7 \\
                              2 & 0 & 0 & 8 \\
                              2 & 0 & 0 & 9 \\
                              2 & 0 & 1 & 0
                          \end{bNiceMatrix}                   \\
                      & = \begin{bNiceMatrix}[r, margin]
                              2 & 0 & 0 & 7  \\
                              0 & 0 & 0 & 1  \\
                              0 & 0 & 0 & 2  \\
                              0 & 0 & 1 & -7
                          \end{bNiceMatrix}                 &
                      & =  \begin{bNiceMatrix}[r, margin]
                               2 & 0 & 0 & 7  \\
                               0 & 0 & 1 & -7 \\
                               0 & 0 & 0 & 1  \\
                               0 & 0 & 0 & 0
                           \end{bNiceMatrix}
          \end{align}

    \item Checking if the set of vectors are L.I, \textcolor{y_p}{no}
          \begin{align}
              \vec{A} & = \begin{bNiceMatrix}[r, margin]
                              \vec{v}_1 \\ \vec{v}_2 \\ \vec{v}_3
                          \end{bNiceMatrix} &
                      & = \begin{bNiceMatrix}[r, margin]
                              0.4 & -0.2 & 0.2 \\
                              0   & 0    & 0   \\
                              3   & -0.6 & 1.5
                          \end{bNiceMatrix}      \\
                      & = \begin{bNiceMatrix}[r, margin]
                              0.4 & -0.2         & 0.2 \\
                              0   & \frac{9}{10} & 0   \\
                              0   & 0            & 0   \\
                          \end{bNiceMatrix}
          \end{align}

    \item Checking if the set of vectors are L.I, \textcolor{y_h}{yes}
          \begin{align}
              \vec{A} & = \begin{bNiceMatrix}[r, margin]
                              \vec{v}_1 \\ \vec{v}_2
                          \end{bNiceMatrix} &
                      & = \begin{bNiceMatrix}[r, margin]
                              9 & 8 & 7 & 6 & 5 \\
                              9 & 7 & 5 & 3 & 1 \\
                          \end{bNiceMatrix} \\
                      & = \begin{bNiceMatrix}[r, margin]
                              9 & 8 & 7 & 6 & 5 \\
                              0 & 1 & 2 & 3 & 4 \\
                          \end{bNiceMatrix}
          \end{align}

    \item Checking if the set of vectors are L.I, \textcolor{y_p}{no}
          \par Since the number of vectors is greater than the number of components.

    \item Checking if the set of vectors are L.I, \textcolor{y_h}{yes}
          \begin{align}
              \vec{A} & = \begin{bNiceMatrix}[r, margin]
                              \vec{v}_1 \\ \vec{v}_2 \\ \vec{v}_3
                          \end{bNiceMatrix} &
                      & = \begin{bNiceMatrix}[r, margin]
                              6  & 0  & -1 & 3  \\
                              2  & 2  & 5  & 0  \\
                              -4 & -4 & -4 & -4 \\
                          \end{bNiceMatrix}      \\
                      & = \begin{bNiceMatrix}[r, margin]
                              6 & 0  & -1             & 3  \\
                              0 & 2  & \frac{16}{3}   & -1 \\
                              0 & -4 & - \frac{14}{3} & -2 \\
                          \end{bNiceMatrix}    &
                      & =  \begin{bNiceMatrix}[r, margin]
                               6 & 0 & -1           & 3  \\
                               0 & 2 & \frac{16}{3} & -1 \\
                               0 & 0 & 6            & -4 \\
                           \end{bNiceMatrix}
          \end{align}

    \item Finding the row echelon form,
          \begin{align}
              \vec{A} & = \begin{bNiceMatrix}[r, margin]
                              \vec{v}_1 \\ \vec{v}_2 \\ \vec{v}_3 \\
                              \vec{v}_4 \\ \vec{v}_5
                          \end{bNiceMatrix} &
                      & = \begin{bNiceMatrix}[r, margin]
                              3  & 0 & 1 & 2 \\
                              6  & 1 & 0 & 0 \\
                              12 & 1 & 2 & 4 \\
                              6  & 0 & 2 & 4 \\
                              9  & 0 & 1 & 2 \\
                          \end{bNiceMatrix}       \\
                      & = \begin{bNiceMatrix}[r, margin]
                              3 & 0 & 1  & 2  \\
                              0 & 1 & -2 & -4 \\
                              0 & 1 & -2 & -4 \\
                              0 & 0 & 0  & 0  \\
                              0 & 0 & -2 & -4 \\
                          \end{bNiceMatrix}       &
                      & =  \begin{bNiceMatrix}[r, margin]
                               3 & 0 & 1  & 2  \\
                               0 & 1 & -2 & -4 \\
                               0 & 0 & 0  & 0  \\
                               0 & 0 & 0  & 0  \\
                               0 & 0 & -2 & -4 \\
                           \end{bNiceMatrix}      \\
          \end{align}
          The linearly independent set of vectors is $ \vec{v}_1, \vec{v}_2, \vec{v}_5 $

    \item Are the given set of vectors a vector space? \textcolor{y_h}{yes}
          \begin{align}
              v_1 - v_2 + 2v_3           & = 0               &
                                         & \in \mathcal{R}^3   \\
              \vec{A},\ \vec{B}          & \in V             &
              \implies \vec{A} + \vec{B} & \in V               \\
              \vec{A}                    & \in V             &
              \implies k\vec{A}          & \in V
          \end{align}
          The dimension is 2, since the relation between the components has 2 d.o.f.
          \par
          A basis is any set of 2 L.I. members of $ V $, for example,
          \begin{align}
              \vec{A} & = \begin{bNiceMatrix}[r, margin]
                              0 \\ 2 \\ 1
                          \end{bNiceMatrix} &
              \vec{B} & = \begin{bNiceMatrix}[r, margin]
                              -2 \\ 0 \\ 1
                          \end{bNiceMatrix}
          \end{align}

    \item Are the given set of vectors a vector space? \textcolor{y_p}{no}
          \begin{align}
              3v_2 + v_3                         & = k               &
                                                 & \in \mathcal{R}^3   \\
              \vec{A},\ \vec{B}                  & \in V             &
              \not\!\!\implies \vec{A} + \vec{B} & \in V               \\
              \vec{A}                            & \in V             &
              \not\!\!\implies k\vec{A}          & \in V
          \end{align}

    \item Are the given set of vectors a vector space? \textcolor{y_p}{no}
          \begin{align}
              v_1                        & \geq v_2          &
                                         & \in \mathcal{R}^2   \\
              \vec{A},\ \vec{B}          & \in V             &
              \implies \vec{A} + \vec{B} & \in V               \\
              \vec{A}                    & \in V             &
              \not\!\!\implies k\vec{A}  & \in V
          \end{align}

    \item Are the given set of vectors a vector space? \textcolor{y_h}{yes}
          \begin{align}
              \{v_1,v_2,\dots,v_{n-2}\}  & = 0               &
                                         & \in \mathcal{R}^n   \\
              \vec{A},\ \vec{B}          & \in V             &
              \implies \vec{A} + \vec{B} & \in V               \\
              \vec{A}                    & \in V             &
              \implies k\vec{A}          & \in V
          \end{align}

          The dimension is 2, since the relation between the components has 2 d.o.f.
          \par
          A basis is any set of 2 L.I. members of $ V $, for example,
          \begin{align}
              \vec{A} & = \begin{bNiceMatrix}[r, margin]
                              0 \\ \vdots \\ 0\\ 1 \\ 0
                          \end{bNiceMatrix} &
              \vec{B} & = \begin{bNiceMatrix}[r, margin]
                              0 \\ \vdots \\ 0\\ 0 \\ 1
                          \end{bNiceMatrix}
          \end{align}

    \item Are the given set of vectors a vector space? \textcolor{y_p}{no}
          \begin{align}
              v_i                        & > 0\ \forall\ i   &
                                         & \in \mathcal{R}^5   \\
              \vec{A},\ \vec{B}          & \in V             &
              \implies \vec{A} + \vec{B} & \in V               \\
              \vec{A}                    & \in V             &
              \not\!\!\implies k\vec{A}  & \in V
          \end{align}

    \item Are the given set of vectors a vector space? \textcolor{y_h}{yes}
          \begin{align}
              3v_1 - 2v_2 + v_3          & = 0                 \\
              4v_1 + 5v_2                & = 0               &
                                         & \in \mathcal{R}^3   \\
              \vec{A},\ \vec{B}          & \in V             &
              \implies \vec{A} + \vec{B} & \in V               \\
              \vec{A}                    & \in V             &
              \implies k\vec{A}          & \in V
          \end{align}

          Since two components can be expressed in terms of the first, there is only
          1 d.o.f. and a basis is
          \begin{align}
              \vec{A} & = \begin{bNiceMatrix}[r, margin]
                              5 \\ -4 \\ -23
                          \end{bNiceMatrix}
          \end{align}

    \item Are the given set of vectors a vector space? \textcolor{y_h}{yes}
          \begin{align}
              3v_1 - v_3                 & = 0                 \\
              2v_1 + 3v_2 - 4v_3         & = 0               &
                                         & \in \mathcal{R}^3   \\
              \vec{A},\ \vec{B}          & \in V             &
              \implies \vec{A} + \vec{B} & \in V               \\
              \vec{A}                    & \in V             &
              \implies k\vec{A}          & \in V
          \end{align}

          Since two components can be expressed in terms of the first, there is only
          1 d.o.f. and a basis is
          \begin{align}
              \vec{A} & = \begin{bNiceMatrix}[r, margin]
                              3 \\ 10 \\ 9
                          \end{bNiceMatrix}
          \end{align}

    \item Are the given set of vectors a vector space? \textcolor{y_p}{no}
          \begin{align}
              \abs{v_j}                          & = 1\ \forall\ j \in
              \{1,\dots,n\}                      &
                                                 & \in \mathcal{R}^n     \\
              \vec{A},\ \vec{B}                  & \in V               &
              \not\!\!\implies \vec{A} + \vec{B} & \in V                 \\
              \vec{A}                            & \in V               &
              \not\!\!\implies k\vec{A}          & \in V
          \end{align}

    \item Are the given set of vectors a vector space? \textcolor{y_h}{yes}
          \begin{align}
              v_1 = 2v_2 = 3v_3          & = 4v_4            &
                                         & \in \mathcal{R}^4   \\
              \vec{A},\ \vec{B}          & \in V             &
              \implies \vec{A} + \vec{B} & \in V               \\
              \vec{A}                    & \in V             &
              \implies k\vec{A}          & \in V
          \end{align}

          Since three components can be expressed in terms of the first, there is only
          1 d.o.f. and a basis is
          \begin{align}
              \vec{A} & = \begin{bNiceMatrix}[r, margin]
                              12 \\ 6 \\ 4 \\ 3
                          \end{bNiceMatrix}
          \end{align}

\end{enumerate}